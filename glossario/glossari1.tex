\newacronym{MCD}{MCD}{Massimo comune divisore}
\newacronym{mcm}{mcm}{Minimo comune multiplo}
\newglossaryentry{ag}{type=symbols,name={A},description={il numero dieci in base 16}}
\newglossaryentry{bg}{type=symbols,name={B},description={il numero undici in base 16}}
\newglossaryentry{binomiog}{name={Binomio},description={espressione algebrica ottenuta dalla somma o dalla differenza di due monomi non simili}}
\newglossaryentry{cnug}{type=symbols,name={C},description={insieme dei numeri complessi},see={numcompg}}
\newglossaryentry{crg}{type=symbols,name={C},description={il numero cento nel sistema di numerazione romano}}
\newglossaryentry{denominatoreg}{name={Denominatore},description={termine divisore in una frazione}}
\newglossaryentry{dg}{type=symbols,name={D},description={numero tredici in base 16}}
\newglossaryentry{disparig}{name={Dispari},description={numero non divisibile per due}}
\newglossaryentry{dividendog}{name={Dividendo},description={primo termine della divisione}}
\newglossaryentry{divisioneg}{name={Divisione},description={operazione binaria inversa della moltiplicazione}}
\newglossaryentry{divisoreg}{name={Divisore},description={secondo termine della divisione}}
\newglossaryentry{drg}{type=symbols,name={D},description={numero 500 in nella numerazione romana}}
\newglossaryentry{eg}{type=symbols,name={E},description={il numero quattordici in base 16}}
\newglossaryentry{elevamentoapotenzag}{name={Elevamento a potenza},description={operazione che associa ad un numero detto base ad un numero naturale detto esponente, il risultato detto potenza è uguale al prodotto della base per tante volte quanto è l'esponente},see={potenzag}}
\newglossaryentry{fattoreg}{name={Fattore},description={termine della moltiplicazione}}
\newglossaryentry{fattoreng}{name={Fattore},description={intero che divide esattamente un intero dato}}
\newglossaryentry{fattorialeg}{name={Fattoriale},description={il prodotto dei numeri interi positivi minori o uguali a tale numero}}
\newglossaryentry{fg}{type=symbols,name={F},description={il numero quindici in base 16}}
\newglossaryentry{frazineappag}{name={Frazione apparente},description={nella frazione apparente il numeratore è multiplo del denominatore}}
\newglossaryentry{frazinpropg}{name={Frazione impropria},description={nella frazione impropria il numeratore è maggiore del denominatore}}
\newglossaryentry{frazionegeng}{name={Frazione generatrice},description={la frazione corrispondente ad un numero decimale dato}}
\newglossaryentry{frazionesempg}{name={Semplificare una frazione},description={semplificare una frazione significa dividere il numeratore e il denominatore per il loro Massimo Comune Divisore ($\mcd$). Per la proprietà invariantiva la frazione ottenuta è equivalente a quella data}}
\newglossaryentry{frazioniequivg}{name={Frazioni equivalenti},description={due frazioni sono equivalenti quando rappresentano lo stesso quoziente.}}
\newglossaryentry{frazpropg}{name={Frazione propria},description={nella frazione propria il numeratore è minore del denominatore}}
\newglossaryentry{immaginariopurog}{name={Immaginario puro},description={numero complesso con solo la parte immaginaria},see={numcompg}}
\newglossaryentry{interog}{name={Intero},description={numero che può essere espresso come somma o differenza di due numeri naturali},see={numerorelativog}}
\newglossaryentry{irg}{type=symbols,name={I},description={il numero uno nel sistema di numerazione romano}}
\newglossaryentry{jg}{type=symbols,name={j},description={unità immaginaria}}
\newglossaryentry{lrg}{type=symbols,name={L},description={il numero cinquanta nel sistema di numerazione romano}}
\newglossaryentry{massimocomundivisoreg}{name={Massimo comune divisore},description={intero che divide senza resto due interi dati}}
\newglossaryentry{mediaritmeticag}{name={Media aritmetica},description={in un intervallo di valori valore ottenuto dalla loro somma diviso per il numero degli elementi}}
\newglossaryentry{mediog}{name={Medio},description={secondo e terzo termine di una proporzione}}
\newglossaryentry{minimocomunemultiplog}{name={Minimo comune multiplo},description={il più il piccolo numero che è divisibile per tutti i numeri di uninsieme dati}}
\newglossaryentry{moltiplicazioneg}{name={Moltiplicazione},description={operazione binaria}}
\newglossaryentry{mrg}{type=symbols,name={M},description={il numero mille nel sistema di numerazione romano}}
\newglossaryentry{multiplog}{name={Multiplo},description={un numero $a$ è multiplo di un altro numero $b$ se esiste un numero $c$ tale che $a=b\cdot c$}}
\newglossaryentry{negativog}{name={Negativo},description={quantità o insieme di valori minori di zero}}
\newglossaryentry{nnug}{name={N},description={insieme dei numeri naturali}}
\newglossaryentry{numcompg}{name={Numero complesso},description=un numero complesso ${z=a+\uimm b}$}
\newglossaryentry{numeratoreg}{name={Numeratore},description={termine che viene diviso in una frazione}}
\newglossaryentry{numerireciprocig}{name={Numeri reciproci},description={due numeri sono reciproci se il loro prodotto è uno}}
\newglossaryentry{numerodecimaleg}{name={Numero decimale},description={numero formato da due parti separate dalla virgola chiamate parte intera e parte decimale}}
\newglossaryentry{numerodecimalfinitog}{name={Numero decimale finito},description={numero decimale con la parte decimale composta da un numero finito di cifre},see={numerodecimaleg}}
\newglossaryentry{numerodecimalinfinitog}{name={Numero decimale infinito},description={numero decimale con la parte decimale composta da un numero infinito di cifre},see={numerodecimaleg}}
\newglossaryentry{numerodecimalinfinitoperg}{name={Numero decimale periodico},description={numero decimale con la parte decimale composta da un numero finito di cifre, dette periodo, che si ripetono all'infinito},see={numerodecimaleg}}
\newglossaryentry{numerodecimalinfinitopermistg}{name={Numero decimale periodico misto},description={numero decimale con la parte decimale divisa in una parte finita detta antiperiodo e da un numero finito di cifre, dette periodo, che si ripetono all'infinito},see={numerodecimaleg}}
\newglossaryentry{numerodivisibileg}{name={Numero divisibile},description={numero che può essere diviso esattamente da un altro numero}}
\newglossaryentry{numeroirrazionaleg}{name={Numero irrazionale},description={numero non esprimibile come rapporto di due numeri interi}}
\newglossaryentry{numeroppostog}{name={Numero opposto},description={Due numeri sono opposti se hanno lo stesso valore assoluto ma segno diverso}}
\newglossaryentry{numeroprimog}{name={Numero primo},description={numero divisibile solo per se stesso e per l'unità}}
\newglossaryentry{numerorazionaleg}{name={Numero razionale},description={numero esprimibile come rapporto di due numeri interi}}
\newglossaryentry{numerorelativog}{name={Numero relativo},description={numero che può essere positivo, negativo o zero}}
\newglossaryentry{numng}{type=symbols,name={N},description={insieme numerico $\Ni=\Set{0,1,2,3,\dots,}$}}
\newglossaryentry{numnprimifralorog}{name={Numeri primi fra loro},description={due numeri sono primi fra loro se l'unico numero che li divide entrambi è uno},see={numericoprimig}}
\newglossaryentry{numprimicuginig}{name={Numeri primi cugini},description={due numeri primi sono cugini se differiscono di quattro}}
\newglossaryentry{numprimigemellig}{name={Numeri primi gemelli},description={due numeri primi sono gemelli se differiscono di due}}
\newglossaryentry{numprimisexyg}{name={Numeri primi sexy},description={due numeri primi sono sexy se differiscono di sei}}
\newglossaryentry{potenzag}{name={Potenza},description={risultato dell'elevamento a potenza},see={elevamentoapotenzag}}
\newglossaryentry{primog}{name={Primo},description={numero divisibile solo per se stesso e per l'unità}}
\newglossaryentry{prodottog}{name={Prodotto},description={risultato moltiplicazione}}
\newglossaryentry{puntiallineatig}{name={Punti allineati},description={punti che appartengono alla stessa retta}}
\newglossaryentry{qnug}{type=symbols,name={Q},description={insieme dei numeri razionali}}
\newglossaryentry{rnug}{type=symbols,name={R},description={insieme dei numeri reali}}
\newglossaryentry{scompfatprimig}{name={Scomposizione in fattori primi},description={scrivere un numero come prodotto di numeri primi}}
\newglossaryentry{sommag}{name={Somma},description={risultato dell'addizione}}
\newglossaryentry{sottraendog}{name={Sottraendo},description={secondo termine sottrazione}}
\newglossaryentry{sottrazioneg}{name={Sottrazione},description={operazione binaria}}
\newglossaryentry{ternaPitag}{name={Terna pitagorica},description={tre numeri legati dalla relazione di Pitagora}}
\newglossaryentry{uniimgg}{name={Unità immaginaria},description={simbolo che ha la proprietà che se è elevato al quadrato vale meno uno}}
\newglossaryentry{znug}{ type=symbols,name={Z},description={insieme dei numeri interi}}
\newglossaryentry{numeriamicig}{name={Numeri amici},description={coppia di interi ciascuno dei quali è somma dei fattori propri distinti dell'altro}}
\newglossaryentry{numerifibonaccig}{name={Numeri di Fibonacci},description={successione di interi nel quale ogni termine è uguale alla somma dei due precedenti, i primi due termini sono zero e uno}}
\newglossaryentry{numerilucasg}{name={Numeri di Lucas},description={successione di interi nel quale ogni termine è uguale alla somma dei due precedenti, i primi due termini sono due e  uno}}
\newglossaryentry{numerifiguratoig}{name={Numero figurato},description={numero che può essere rappresntato da uno schema regolare di punti}}
\newglossaryentry{vrg}{type=symbols,name={V},description={il numero cinque nel sistema di numerazione romano}}
\newglossaryentry{xrg}{type=symbols,name={X},description={il numero dieci nel sistema di numerazione romano}}
\newglossaryentry{numericoprimig}{name={Numeri coprimi},description={due numeri sono coprimi se il loro MCD è uno}}