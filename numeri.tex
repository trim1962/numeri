 % !TeX root = numeri.tex
% !BIB TS-program = biber
% !TeX encoding = UTF-8
% !TeX spellcheck = it_IT

\documentclass[a4paper,oneside]{book}%
\usepackage{cmap}
\usepackage[big]{layaureo}
\usepackage{copyright}
\frenchspacing%
\usepackage{amsmath}

\usepackage{amssymb}
\usepackage[italian]{babel}
\usepackage[thmmarks,hyperref]{ntheorem}
\usepackage{miamatematica}

\usepackage{lmodern} % load vector font
\usepackage[T1]{fontenc} % font encoding
\usepackage[utf8]{inputenc} % input encoding
%\usepackage{noto}
\usepackage[babel=true]{microtype}
%\usepackage{geometry}
\usepackage{textcomp}

%\geometry{top=1.5cm,bottom=1.5cm}
\usepackage{grafica}

%Teorema
\theoremstyle{marginbreak}
\theoremheaderfont{\normalfont\bfseries}
\theorembodyfont{\slshape}
\theoremsymbol{\ensuremath{\diamondsuit}}
\theoremseparator{:} %
\newtheorem{thm}{Teorema}[section]
%Proprietà
\theoremstyle{marginbreak}
\theoremheaderfont{\normalfont\bfseries}
\theorembodyfont{\slshape}
\theoremsymbol{\ensuremath{\diamondsuit}}
\theoremseparator{:}
\newtheorem{prop}{Proprietà}[section]
%lemma
\theoremstyle{changebreak}
\theoremsymbol{\ensuremath{\heartsuit}}
\theoremindent0.5cm
\theoremnumbering{greek}
\newtheorem{lem}[thm]{Lemma}
%corollario
\theoremindent0cm
\theoremsymbol{\ensuremath{\spadesuit}}
\theoremnumbering{arabic}
\newtheorem{cor}[thm]{Corollario}
%esempio
\theoremstyle{change}
\theorembodyfont{\upshape}
\theoremsymbol{\ensuremath{\ast}}
\theoremseparator{}
\newtheorem{exmp}{Esempio}[section]
%controesempio
\theoremstyle{change}
\theorembodyfont{\upshape}
\theoremsymbol{\ensuremath{\odot}}
\theoremseparator{}
\newtheorem{cexmp}{Contro esempio}[section]
%definizione
\theoremstyle{plain}
\theoremsymbol{\ensuremath{\clubsuit}}
\theoremseparator{.}
\theoremstyle{marginbreak}
%\theoremprework{\hrule\bigskip}
%\theorempostwork{\hrule\bigskip}
\newtheorem{defn}{Definizione}[section]
%commento
\theoremstyle{plain}
\theorembodyfont{\upshape}
\theoremsymbol{\ensuremath{\blacklozenge}}
\theoremseparator{:}
\newtheorem{commento}{Commento}
%dimostrazione

\theoremstyle{plain}
\theoremheaderfont{\sc}
\theorembodyfont{\bfseries}
\theoremstyle{nonumberplain}
%^{}\theoremseparator{.}

\theoremsymbol{\ensuremath{\blacksquare}}
\theoremheaderfont{\bfseries}
%\theoremstyle{nonumberplain}
%\theoremstyle{marginbreak}
\theorembodyfont{\normalfont}
\newtheorem{proof}{Dimostrazione}
%\input{../Mod_base/tabelle}

\usepackage{tabelle}

%\usepackage{adjustbox}
%\input{../Mod_base/stand_class}
\usepackage{pagina}

\setlength{\headheight}{13pt}
\usepackage{indice}
\usepackage{date}
\usepackage{unita_misura}

\usepackage{imakeidx}
\makeindex[options=-s ../Mod_base/oldclaudio.sti]
\newcolumntype{C}{>{\sffamily $}c<{$}}
%\newcolumntype{F}{>{$\displaystyle }c<{$}}
%\newcolumntype{L}{>{\sffamily $}l<{$}}
%\newcolumntype{M}[1]{>{\centering}p{#1}}
\newcolumntype{A}{r@{\hspace*{1mm}}r@{\hspace*{1mm}}r}
\newcolumntype{Q}{r@{\hspace*{1mm}}r}
%\newcolumntype{O}{>{\centering\arraybackslash}p{1.5em}} 
\newcolumntype{R}{>{\sffamily $}r<{$}}
%\newcolumntype{T}{>{\centering\arraybackslash}p{1em}}
%\newcolumntype{W}{>{\sffamily\Large $}c<{$}}
%\newcommand{\Co}{\numberset{C}}
\newcommand{\HRule}{\rule{\linewidth}{0.5mm}}
%\newlength{\gnat}
%\newlength{\gnam}
%\newcommand{\pilH}{\rule{0pt}{2.5ex}}
%\newcommand{\pilD}{\rule[-1ex]{0pt}{0pt}}
% 10/09/2017 :: 10:18:49 :: \usepackage{draftwatermark}
% 10/09/2017 :: 10:19:00 :: \SetWatermarkText{BOZZA}
\usepackage{altrebasitabelle}

\makeatletter
\renewcommand\frontmatter{%
	\cleardoublepage
	\@mainmatterfalse
	%\pagenumbering{roman}
}
\renewcommand\mainmatter{%
	\cleardoublepage
	\@mainmattertrue
	%\pagenumbering{arabic}
}
\makeatother
\usepackage{listings}
\lstdefinestyle{pascalstyle}{
	language=Pascal,
	commentstyle=\color{red},
	sensitive=false,
	%morecomment=[l][\color{red}]{//},
	morecomment=[l]{//},
}
%\usepackage{tkz-berge}
\usepackage[toc,page]{appendix}
\usepackage{stand_class}
\renewcommand{\appendixtocname}{Appendici}
\renewcommand{\appendixpagename}{Appendici}
\usepackage[autostyle,italian=guillemets]{csquotes}
\usepackage[%
style=philosophy-modern,
% 9/01/2018 :: 9:24:04 :: annotation=true,
hyperref,
backend=biber,
backref]{biblatex}
\addbibresource{numeri.bib}
\newcommand{\citaoeis}[1]{La successione è la successione \citetitle{#1}~\citeurl{#1} del OEIS}
\usepackage[grumpy,mark,markifdirty,raisemark=0.95\paperheight]{gitinfo2}
\usepackage{hyperxmp}
\usepackage{hyperref}
\title{Numeri}
\author{Claudio Duchi}
\date{\datetime}
\hypersetup{%
pdfencoding=auto,
urlcolor={blue},
pdftitle={Tavole},
pdfsubject={Tavole numeriche},
pdfstartview={FitH},
pdfpagemode={UseOutlines},
pdflicenseurl={http://creativecommons.org/licenses/by-nc-nd/3.0/},
pdflang={it},
pdfmetalang={it},
pdfkeywords={Numeri},
pdfcopyright={Copyright (C) 2021, Claudio Duchi},
pdfcontacturl={http://breviariomatematico.altervista.org},
pdfcontactpostcode={},
pdfcontactphone={},
pdfcontactemail={claduc},
pdfcontactcountry={Italy},
pdfcontactcity={Perugia},
pdfcontactaddress={},
pdfcaptionwriter={Claudio Duchi},
pdfauthortitle={},%
pdfauthor={Claudio Duchi},
linkcolor={blue},
colorlinks=true,
citecolor={red},
breaklinks,
bookmarksopen,
verbose,
baseurl={http://breviariomatematico.altervista.org}
}
\usepackage[italian]{varioref}
\usepackage[italian]{cleveref}
\includeonly{%
%tabprimi,
primi,
tabprimiamici,	
tabella,
elencofattori,
tabPitagoriche,
ternepitagoriche2,
coprimi,
fibonacci,
fattoriali,
tartaglia,
numerifigurati,
quadratimagici,
CriteriDivisibilita,
formule,Collatz,numerodivisori
}%
\usepackage{CDloghi}
\listfiles
 \begin{document}
		\frontmatter
		\hypersetup{pageanchor=false}
		\begin{titlepage}
			\begin{center}
			\Lgrandedue\\[1cm]
				\textsc{\LARGE Claudio Duchi}\\[1.5cm]
				\HRule \\[0.4cm]
				{ \huge \bfseries Numeri}\\[0.4cm]
				\HRule \\[1.5cm]
	\includestandalone[scale=0.9]{logonumeri}
		\end{center}
		\end{titlepage}
	\hypersetup{pageanchor=true}
	L'immagine di copertina è di \cite{PatrickT2014}\par
	\CDcopyright
		\tableofcontents
		\listoftables
		\mainmatter


\include{tabprimi}	
% 14/01/2018 :: 14:29:47 :: \include{tabprimiamici}	
\include{tabella}
\include{elencofattori}
\include{tabPitagoriche}
\include{fattoriali}
\include{tartaglia}
\include{CriteriDivisibilita}
\include{fibonacci}
\include{ternepitagoriche2}
\include{coprimi}
\include{numerifigurati}
\include{quadratimagici}
\chapter{Congettura di Collatz}
\citaoeis{A006577}
\begin{longtable}{*{10}{l}}\toprule
\caption{Cicli}\\
\midrule
\endfirsthead
\multicolumn{10}{c} {\tablename\ \thetable\ -- \textit{Continua dalla pagina precedente}} \\
\toprule
\endhead
\bottomrule
\multicolumn{10}{r} {\textit{Continua nella pagina successiva}} \\
\endfoot
\endlastfoot

1&&&&&&&&&\\
4& 2& 1& \\

2&&&&&&&&&\\
1& \\

3&&&&&&&&&\\
10& 5& 16& 8& 4& 2& 1& \\

4&&&&&&&&&\\
2& 1& \\

5&&&&&&&&&\\
16& 8& 4& 2& 1& \\

6&&&&&&&&&\\
3& 10& 5& 16& 8& 4& 2& 1& \\

7&&&&&&&&&\\
22& 11& 34& 17& 52& 26& 13& 40& 20& 10\\
5& 16& 8& 4& 2& 1& \\

8&&&&&&&&&\\
4& 2& 1& \\

9&&&&&&&&&\\
28& 14& 7& 22& 11& 34& 17& 52& 26& 13\\
40& 20& 10& 5& 16& 8& 4& 2& 1& \\

10&&&&&&&&&\\
5& 16& 8& 4& 2& 1& \\

11&&&&&&&&&\\
34& 17& 52& 26& 13& 40& 20& 10& 5& 16\\
8& 4& 2& 1& \\

12&&&&&&&&&\\
6& 3& 10& 5& 16& 8& 4& 2& 1& \\

13&&&&&&&&&\\
40& 20& 10& 5& 16& 8& 4& 2& 1& \\

14&&&&&&&&&\\
7& 22& 11& 34& 17& 52& 26& 13& 40& 20\\
10& 5& 16& 8& 4& 2& 1& \\

15&&&&&&&&&\\
46& 23& 70& 35& 106& 53& 160& 80& 40& 20\\
10& 5& 16& 8& 4& 2& 1& \\

16&&&&&&&&&\\
8& 4& 2& 1& \\

17&&&&&&&&&\\
52& 26& 13& 40& 20& 10& 5& 16& 8& 4\\
2& 1& \\

18&&&&&&&&&\\
9& 28& 14& 7& 22& 11& 34& 17& 52& 26\\
13& 40& 20& 10& 5& 16& 8& 4& 2& 1\\

19&&&&&&&&&\\
58& 29& 88& 44& 22& 11& 34& 17& 52& 26\\
13& 40& 20& 10& 5& 16& 8& 4& 2& 1\\

20&&&&&&&&&\\
10& 5& 16& 8& 4& 2& 1& \\

21&&&&&&&&&\\
64& 32& 16& 8& 4& 2& 1& \\

22&&&&&&&&&\\
11& 34& 17& 52& 26& 13& 40& 20& 10& 5\\
16& 8& 4& 2& 1& \\

23&&&&&&&&&\\
70& 35& 106& 53& 160& 80& 40& 20& 10& 5\\
16& 8& 4& 2& 1& \\

24&&&&&&&&&\\
12& 6& 3& 10& 5& 16& 8& 4& 2& 1\\

25&&&&&&&&&\\
76& 38& 19& 58& 29& 88& 44& 22& 11& 34\\
17& 52& 26& 13& 40& 20& 10& 5& 16& 8\\
4& 2& 1& \\

26&&&&&&&&&\\
13& 40& 20& 10& 5& 16& 8& 4& 2& 1\\

27&&&&&&&&&\\
82& 41& 124& 62& 31& 94& 47& 142& 71& 214\\
107& 322& 161& 484& 242& 121& 364& 182& 91& 274\\
137& 412& 206& 103& 310& 155& 466& 233& 700& 350\\
175& 526& 263& 790& 395& 1186& 593& 1780& 890& 445\\
1336& 668& 334& 167& 502& 251& 754& 377& 1132& 566\\
283& 850& 425& 1276& 638& 319& 958& 479& 1438& 719\\
2158& 1079& 3238& 1619& 4858& 2429& 7288& 3644& 1822& 911\\
2734& 1367& 4102& 2051& 6154& 3077& 9232& 4616& 2308& 1154\\
577& 1732& 866& 433& 1300& 650& 325& 976& 488& 244\\
122& 61& 184& 92& 46& 23& 70& 35& 106& 53\\
160& 80& 40& 20& 10& 5& 16& 8& 4& 2\\
1& \\

28&&&&&&&&&\\
14& 7& 22& 11& 34& 17& 52& 26& 13& 40\\
20& 10& 5& 16& 8& 4& 2& 1& \\

29&&&&&&&&&\\
88& 44& 22& 11& 34& 17& 52& 26& 13& 40\\
20& 10& 5& 16& 8& 4& 2& 1& \\

30&&&&&&&&&\\
15& 46& 23& 70& 35& 106& 53& 160& 80& 40\\
20& 10& 5& 16& 8& 4& 2& 1& \\

31&&&&&&&&&\\
94& 47& 142& 71& 214& 107& 322& 161& 484& 242\\
121& 364& 182& 91& 274& 137& 412& 206& 103& 310\\
155& 466& 233& 700& 350& 175& 526& 263& 790& 395\\
1186& 593& 1780& 890& 445& 1336& 668& 334& 167& 502\\
251& 754& 377& 1132& 566& 283& 850& 425& 1276& 638\\
319& 958& 479& 1438& 719& 2158& 1079& 3238& 1619& 4858\\
2429& 7288& 3644& 1822& 911& 2734& 1367& 4102& 2051& 6154\\
3077& 9232& 4616& 2308& 1154& 577& 1732& 866& 433& 1300\\
650& 325& 976& 488& 244& 122& 61& 184& 92& 46\\
23& 70& 35& 106& 53& 160& 80& 40& 20& 10\\
5& 16& 8& 4& 2& 1& \\

32&&&&&&&&&\\
16& 8& 4& 2& 1& \\

33&&&&&&&&&\\
100& 50& 25& 76& 38& 19& 58& 29& 88& 44\\
22& 11& 34& 17& 52& 26& 13& 40& 20& 10\\
5& 16& 8& 4& 2& 1& \\

34&&&&&&&&&\\
17& 52& 26& 13& 40& 20& 10& 5& 16& 8\\
4& 2& 1& \\

35&&&&&&&&&\\
106& 53& 160& 80& 40& 20& 10& 5& 16& 8\\
4& 2& 1& \\

36&&&&&&&&&\\
18& 9& 28& 14& 7& 22& 11& 34& 17& 52\\
26& 13& 40& 20& 10& 5& 16& 8& 4& 2\\
1& \\

37&&&&&&&&&\\
112& 56& 28& 14& 7& 22& 11& 34& 17& 52\\
26& 13& 40& 20& 10& 5& 16& 8& 4& 2\\
1& \\

38&&&&&&&&&\\
19& 58& 29& 88& 44& 22& 11& 34& 17& 52\\
26& 13& 40& 20& 10& 5& 16& 8& 4& 2\\
1& \\

39&&&&&&&&&\\
118& 59& 178& 89& 268& 134& 67& 202& 101& 304\\
152& 76& 38& 19& 58& 29& 88& 44& 22& 11\\
34& 17& 52& 26& 13& 40& 20& 10& 5& 16\\
8& 4& 2& 1& \\

40&&&&&&&&&\\
20& 10& 5& 16& 8& 4& 2& 1& \\

41&&&&&&&&&\\
124& 62& 31& 94& 47& 142& 71& 214& 107& 322\\
161& 484& 242& 121& 364& 182& 91& 274& 137& 412\\
206& 103& 310& 155& 466& 233& 700& 350& 175& 526\\
263& 790& 395& 1186& 593& 1780& 890& 445& 1336& 668\\
334& 167& 502& 251& 754& 377& 1132& 566& 283& 850\\
425& 1276& 638& 319& 958& 479& 1438& 719& 2158& 1079\\
3238& 1619& 4858& 2429& 7288& 3644& 1822& 911& 2734& 1367\\
4102& 2051& 6154& 3077& 9232& 4616& 2308& 1154& 577& 1732\\
866& 433& 1300& 650& 325& 976& 488& 244& 122& 61\\
184& 92& 46& 23& 70& 35& 106& 53& 160& 80\\
40& 20& 10& 5& 16& 8& 4& 2& 1& \\

42&&&&&&&&&\\
21& 64& 32& 16& 8& 4& 2& 1& \\

43&&&&&&&&&\\
130& 65& 196& 98& 49& 148& 74& 37& 112& 56\\
28& 14& 7& 22& 11& 34& 17& 52& 26& 13\\
40& 20& 10& 5& 16& 8& 4& 2& 1& \\

44&&&&&&&&&\\
22& 11& 34& 17& 52& 26& 13& 40& 20& 10\\
5& 16& 8& 4& 2& 1& \\

45&&&&&&&&&\\
136& 68& 34& 17& 52& 26& 13& 40& 20& 10\\
5& 16& 8& 4& 2& 1& \\

46&&&&&&&&&\\
23& 70& 35& 106& 53& 160& 80& 40& 20& 10\\
5& 16& 8& 4& 2& 1& \\

47&&&&&&&&&\\
142& 71& 214& 107& 322& 161& 484& 242& 121& 364\\
182& 91& 274& 137& 412& 206& 103& 310& 155& 466\\
233& 700& 350& 175& 526& 263& 790& 395& 1186& 593\\
1780& 890& 445& 1336& 668& 334& 167& 502& 251& 754\\
377& 1132& 566& 283& 850& 425& 1276& 638& 319& 958\\
479& 1438& 719& 2158& 1079& 3238& 1619& 4858& 2429& 7288\\
3644& 1822& 911& 2734& 1367& 4102& 2051& 6154& 3077& 9232\\
4616& 2308& 1154& 577& 1732& 866& 433& 1300& 650& 325\\
976& 488& 244& 122& 61& 184& 92& 46& 23& 70\\
35& 106& 53& 160& 80& 40& 20& 10& 5& 16\\
8& 4& 2& 1& \\

48&&&&&&&&&\\
24& 12& 6& 3& 10& 5& 16& 8& 4& 2\\
1& \\

49&&&&&&&&&\\
148& 74& 37& 112& 56& 28& 14& 7& 22& 11\\
34& 17& 52& 26& 13& 40& 20& 10& 5& 16\\
8& 4& 2& 1& \\

50&&&&&&&&&\\
25& 76& 38& 19& 58& 29& 88& 44& 22& 11\\
34& 17& 52& 26& 13& 40& 20& 10& 5& 16\\
8& 4& 2& 1& \\

51&&&&&&&&&\\
154& 77& 232& 116& 58& 29& 88& 44& 22& 11\\
34& 17& 52& 26& 13& 40& 20& 10& 5& 16\\
8& 4& 2& 1& \\

52&&&&&&&&&\\
26& 13& 40& 20& 10& 5& 16& 8& 4& 2\\
1& \\

53&&&&&&&&&\\
160& 80& 40& 20& 10& 5& 16& 8& 4& 2\\
1& \\

54&&&&&&&&&\\
27& 82& 41& 124& 62& 31& 94& 47& 142& 71\\
214& 107& 322& 161& 484& 242& 121& 364& 182& 91\\
274& 137& 412& 206& 103& 310& 155& 466& 233& 700\\
350& 175& 526& 263& 790& 395& 1186& 593& 1780& 890\\
445& 1336& 668& 334& 167& 502& 251& 754& 377& 1132\\
566& 283& 850& 425& 1276& 638& 319& 958& 479& 1438\\
719& 2158& 1079& 3238& 1619& 4858& 2429& 7288& 3644& 1822\\
911& 2734& 1367& 4102& 2051& 6154& 3077& 9232& 4616& 2308\\
1154& 577& 1732& 866& 433& 1300& 650& 325& 976& 488\\
244& 122& 61& 184& 92& 46& 23& 70& 35& 106\\
53& 160& 80& 40& 20& 10& 5& 16& 8& 4\\
2& 1& \\

55&&&&&&&&&\\
166& 83& 250& 125& 376& 188& 94& 47& 142& 71\\
214& 107& 322& 161& 484& 242& 121& 364& 182& 91\\
274& 137& 412& 206& 103& 310& 155& 466& 233& 700\\
350& 175& 526& 263& 790& 395& 1186& 593& 1780& 890\\
445& 1336& 668& 334& 167& 502& 251& 754& 377& 1132\\
566& 283& 850& 425& 1276& 638& 319& 958& 479& 1438\\
719& 2158& 1079& 3238& 1619& 4858& 2429& 7288& 3644& 1822\\
911& 2734& 1367& 4102& 2051& 6154& 3077& 9232& 4616& 2308\\
1154& 577& 1732& 866& 433& 1300& 650& 325& 976& 488\\
244& 122& 61& 184& 92& 46& 23& 70& 35& 106\\
53& 160& 80& 40& 20& 10& 5& 16& 8& 4\\
2& 1& \\

56&&&&&&&&&\\
28& 14& 7& 22& 11& 34& 17& 52& 26& 13\\
40& 20& 10& 5& 16& 8& 4& 2& 1& \\

57&&&&&&&&&\\
172& 86& 43& 130& 65& 196& 98& 49& 148& 74\\
37& 112& 56& 28& 14& 7& 22& 11& 34& 17\\
52& 26& 13& 40& 20& 10& 5& 16& 8& 4\\
2& 1& \\

58&&&&&&&&&\\
29& 88& 44& 22& 11& 34& 17& 52& 26& 13\\
40& 20& 10& 5& 16& 8& 4& 2& 1& \\

59&&&&&&&&&\\
178& 89& 268& 134& 67& 202& 101& 304& 152& 76\\
38& 19& 58& 29& 88& 44& 22& 11& 34& 17\\
52& 26& 13& 40& 20& 10& 5& 16& 8& 4\\
2& 1& \\

60&&&&&&&&&\\
30& 15& 46& 23& 70& 35& 106& 53& 160& 80\\
40& 20& 10& 5& 16& 8& 4& 2& 1& \\

61&&&&&&&&&\\
184& 92& 46& 23& 70& 35& 106& 53& 160& 80\\
40& 20& 10& 5& 16& 8& 4& 2& 1& \\

62&&&&&&&&&\\
31& 94& 47& 142& 71& 214& 107& 322& 161& 484\\
242& 121& 364& 182& 91& 274& 137& 412& 206& 103\\
310& 155& 466& 233& 700& 350& 175& 526& 263& 790\\
395& 1186& 593& 1780& 890& 445& 1336& 668& 334& 167\\
502& 251& 754& 377& 1132& 566& 283& 850& 425& 1276\\
638& 319& 958& 479& 1438& 719& 2158& 1079& 3238& 1619\\
4858& 2429& 7288& 3644& 1822& 911& 2734& 1367& 4102& 2051\\
6154& 3077& 9232& 4616& 2308& 1154& 577& 1732& 866& 433\\
1300& 650& 325& 976& 488& 244& 122& 61& 184& 92\\
46& 23& 70& 35& 106& 53& 160& 80& 40& 20\\
10& 5& 16& 8& 4& 2& 1& \\

63&&&&&&&&&\\
190& 95& 286& 143& 430& 215& 646& 323& 970& 485\\
1456& 728& 364& 182& 91& 274& 137& 412& 206& 103\\
310& 155& 466& 233& 700& 350& 175& 526& 263& 790\\
395& 1186& 593& 1780& 890& 445& 1336& 668& 334& 167\\
502& 251& 754& 377& 1132& 566& 283& 850& 425& 1276\\
638& 319& 958& 479& 1438& 719& 2158& 1079& 3238& 1619\\
4858& 2429& 7288& 3644& 1822& 911& 2734& 1367& 4102& 2051\\
6154& 3077& 9232& 4616& 2308& 1154& 577& 1732& 866& 433\\
1300& 650& 325& 976& 488& 244& 122& 61& 184& 92\\
46& 23& 70& 35& 106& 53& 160& 80& 40& 20\\
10& 5& 16& 8& 4& 2& 1& \\

64&&&&&&&&&\\
32& 16& 8& 4& 2& 1& \\

65&&&&&&&&&\\
196& 98& 49& 148& 74& 37& 112& 56& 28& 14\\
7& 22& 11& 34& 17& 52& 26& 13& 40& 20\\
10& 5& 16& 8& 4& 2& 1& \\

66&&&&&&&&&\\
33& 100& 50& 25& 76& 38& 19& 58& 29& 88\\
44& 22& 11& 34& 17& 52& 26& 13& 40& 20\\
10& 5& 16& 8& 4& 2& 1& \\

67&&&&&&&&&\\
202& 101& 304& 152& 76& 38& 19& 58& 29& 88\\
44& 22& 11& 34& 17& 52& 26& 13& 40& 20\\
10& 5& 16& 8& 4& 2& 1& \\

68&&&&&&&&&\\
34& 17& 52& 26& 13& 40& 20& 10& 5& 16\\
8& 4& 2& 1& \\

69&&&&&&&&&\\
208& 104& 52& 26& 13& 40& 20& 10& 5& 16\\
8& 4& 2& 1& \\

70&&&&&&&&&\\
35& 106& 53& 160& 80& 40& 20& 10& 5& 16\\
8& 4& 2& 1& \\

71&&&&&&&&&\\
214& 107& 322& 161& 484& 242& 121& 364& 182& 91\\
274& 137& 412& 206& 103& 310& 155& 466& 233& 700\\
350& 175& 526& 263& 790& 395& 1186& 593& 1780& 890\\
445& 1336& 668& 334& 167& 502& 251& 754& 377& 1132\\
566& 283& 850& 425& 1276& 638& 319& 958& 479& 1438\\
719& 2158& 1079& 3238& 1619& 4858& 2429& 7288& 3644& 1822\\
911& 2734& 1367& 4102& 2051& 6154& 3077& 9232& 4616& 2308\\
1154& 577& 1732& 866& 433& 1300& 650& 325& 976& 488\\
244& 122& 61& 184& 92& 46& 23& 70& 35& 106\\
53& 160& 80& 40& 20& 10& 5& 16& 8& 4\\
2& 1& \\

72&&&&&&&&&\\
36& 18& 9& 28& 14& 7& 22& 11& 34& 17\\
52& 26& 13& 40& 20& 10& 5& 16& 8& 4\\
2& 1& \\

73&&&&&&&&&\\
220& 110& 55& 166& 83& 250& 125& 376& 188& 94\\
47& 142& 71& 214& 107& 322& 161& 484& 242& 121\\
364& 182& 91& 274& 137& 412& 206& 103& 310& 155\\
466& 233& 700& 350& 175& 526& 263& 790& 395& 1186\\
593& 1780& 890& 445& 1336& 668& 334& 167& 502& 251\\
754& 377& 1132& 566& 283& 850& 425& 1276& 638& 319\\
958& 479& 1438& 719& 2158& 1079& 3238& 1619& 4858& 2429\\
7288& 3644& 1822& 911& 2734& 1367& 4102& 2051& 6154& 3077\\
9232& 4616& 2308& 1154& 577& 1732& 866& 433& 1300& 650\\
325& 976& 488& 244& 122& 61& 184& 92& 46& 23\\
70& 35& 106& 53& 160& 80& 40& 20& 10& 5\\
16& 8& 4& 2& 1& \\

74&&&&&&&&&\\
37& 112& 56& 28& 14& 7& 22& 11& 34& 17\\
52& 26& 13& 40& 20& 10& 5& 16& 8& 4\\
2& 1& \\

75&&&&&&&&&\\
226& 113& 340& 170& 85& 256& 128& 64& 32& 16\\
8& 4& 2& 1& \\

76&&&&&&&&&\\
38& 19& 58& 29& 88& 44& 22& 11& 34& 17\\
52& 26& 13& 40& 20& 10& 5& 16& 8& 4\\
2& 1& \\

77&&&&&&&&&\\
232& 116& 58& 29& 88& 44& 22& 11& 34& 17\\
52& 26& 13& 40& 20& 10& 5& 16& 8& 4\\
2& 1& \\

78&&&&&&&&&\\
39& 118& 59& 178& 89& 268& 134& 67& 202& 101\\
304& 152& 76& 38& 19& 58& 29& 88& 44& 22\\
11& 34& 17& 52& 26& 13& 40& 20& 10& 5\\
16& 8& 4& 2& 1& \\

79&&&&&&&&&\\
238& 119& 358& 179& 538& 269& 808& 404& 202& 101\\
304& 152& 76& 38& 19& 58& 29& 88& 44& 22\\
11& 34& 17& 52& 26& 13& 40& 20& 10& 5\\
16& 8& 4& 2& 1& \\

80&&&&&&&&&\\
40& 20& 10& 5& 16& 8& 4& 2& 1& \\

81&&&&&&&&&\\
244& 122& 61& 184& 92& 46& 23& 70& 35& 106\\
53& 160& 80& 40& 20& 10& 5& 16& 8& 4\\
2& 1& \\

82&&&&&&&&&\\
41& 124& 62& 31& 94& 47& 142& 71& 214& 107\\
322& 161& 484& 242& 121& 364& 182& 91& 274& 137\\
412& 206& 103& 310& 155& 466& 233& 700& 350& 175\\
526& 263& 790& 395& 1186& 593& 1780& 890& 445& 1336\\
668& 334& 167& 502& 251& 754& 377& 1132& 566& 283\\
850& 425& 1276& 638& 319& 958& 479& 1438& 719& 2158\\
1079& 3238& 1619& 4858& 2429& 7288& 3644& 1822& 911& 2734\\
1367& 4102& 2051& 6154& 3077& 9232& 4616& 2308& 1154& 577\\
1732& 866& 433& 1300& 650& 325& 976& 488& 244& 122\\
61& 184& 92& 46& 23& 70& 35& 106& 53& 160\\
80& 40& 20& 10& 5& 16& 8& 4& 2& 1\\

83&&&&&&&&&\\
250& 125& 376& 188& 94& 47& 142& 71& 214& 107\\
322& 161& 484& 242& 121& 364& 182& 91& 274& 137\\
412& 206& 103& 310& 155& 466& 233& 700& 350& 175\\
526& 263& 790& 395& 1186& 593& 1780& 890& 445& 1336\\
668& 334& 167& 502& 251& 754& 377& 1132& 566& 283\\
850& 425& 1276& 638& 319& 958& 479& 1438& 719& 2158\\
1079& 3238& 1619& 4858& 2429& 7288& 3644& 1822& 911& 2734\\
1367& 4102& 2051& 6154& 3077& 9232& 4616& 2308& 1154& 577\\
1732& 866& 433& 1300& 650& 325& 976& 488& 244& 122\\
61& 184& 92& 46& 23& 70& 35& 106& 53& 160\\
80& 40& 20& 10& 5& 16& 8& 4& 2& 1\\

84&&&&&&&&&\\
42& 21& 64& 32& 16& 8& 4& 2& 1& \\

85&&&&&&&&&\\
256& 128& 64& 32& 16& 8& 4& 2& 1& \\

86&&&&&&&&&\\
43& 130& 65& 196& 98& 49& 148& 74& 37& 112\\
56& 28& 14& 7& 22& 11& 34& 17& 52& 26\\
13& 40& 20& 10& 5& 16& 8& 4& 2& 1\\

87&&&&&&&&&\\
262& 131& 394& 197& 592& 296& 148& 74& 37& 112\\
56& 28& 14& 7& 22& 11& 34& 17& 52& 26\\
13& 40& 20& 10& 5& 16& 8& 4& 2& 1\\

88&&&&&&&&&\\
44& 22& 11& 34& 17& 52& 26& 13& 40& 20\\
10& 5& 16& 8& 4& 2& 1& \\

89&&&&&&&&&\\
268& 134& 67& 202& 101& 304& 152& 76& 38& 19\\
58& 29& 88& 44& 22& 11& 34& 17& 52& 26\\
13& 40& 20& 10& 5& 16& 8& 4& 2& 1\\

90&&&&&&&&&\\
45& 136& 68& 34& 17& 52& 26& 13& 40& 20\\
10& 5& 16& 8& 4& 2& 1& \\

91&&&&&&&&&\\
274& 137& 412& 206& 103& 310& 155& 466& 233& 700\\
350& 175& 526& 263& 790& 395& 1186& 593& 1780& 890\\
445& 1336& 668& 334& 167& 502& 251& 754& 377& 1132\\
566& 283& 850& 425& 1276& 638& 319& 958& 479& 1438\\
719& 2158& 1079& 3238& 1619& 4858& 2429& 7288& 3644& 1822\\
911& 2734& 1367& 4102& 2051& 6154& 3077& 9232& 4616& 2308\\
1154& 577& 1732& 866& 433& 1300& 650& 325& 976& 488\\
244& 122& 61& 184& 92& 46& 23& 70& 35& 106\\
53& 160& 80& 40& 20& 10& 5& 16& 8& 4\\
2& 1& \\

92&&&&&&&&&\\
46& 23& 70& 35& 106& 53& 160& 80& 40& 20\\
10& 5& 16& 8& 4& 2& 1& \\

93&&&&&&&&&\\
280& 140& 70& 35& 106& 53& 160& 80& 40& 20\\
10& 5& 16& 8& 4& 2& 1& \\

94&&&&&&&&&\\
47& 142& 71& 214& 107& 322& 161& 484& 242& 121\\
364& 182& 91& 274& 137& 412& 206& 103& 310& 155\\
466& 233& 700& 350& 175& 526& 263& 790& 395& 1186\\
593& 1780& 890& 445& 1336& 668& 334& 167& 502& 251\\
754& 377& 1132& 566& 283& 850& 425& 1276& 638& 319\\
958& 479& 1438& 719& 2158& 1079& 3238& 1619& 4858& 2429\\
7288& 3644& 1822& 911& 2734& 1367& 4102& 2051& 6154& 3077\\
9232& 4616& 2308& 1154& 577& 1732& 866& 433& 1300& 650\\
325& 976& 488& 244& 122& 61& 184& 92& 46& 23\\
70& 35& 106& 53& 160& 80& 40& 20& 10& 5\\
16& 8& 4& 2& 1& \\

95&&&&&&&&&\\
286& 143& 430& 215& 646& 323& 970& 485& 1456& 728\\
364& 182& 91& 274& 137& 412& 206& 103& 310& 155\\
466& 233& 700& 350& 175& 526& 263& 790& 395& 1186\\
593& 1780& 890& 445& 1336& 668& 334& 167& 502& 251\\
754& 377& 1132& 566& 283& 850& 425& 1276& 638& 319\\
958& 479& 1438& 719& 2158& 1079& 3238& 1619& 4858& 2429\\
7288& 3644& 1822& 911& 2734& 1367& 4102& 2051& 6154& 3077\\
9232& 4616& 2308& 1154& 577& 1732& 866& 433& 1300& 650\\
325& 976& 488& 244& 122& 61& 184& 92& 46& 23\\
70& 35& 106& 53& 160& 80& 40& 20& 10& 5\\
16& 8& 4& 2& 1& \\

96&&&&&&&&&\\
48& 24& 12& 6& 3& 10& 5& 16& 8& 4\\
2& 1& \\

97&&&&&&&&&\\
292& 146& 73& 220& 110& 55& 166& 83& 250& 125\\
376& 188& 94& 47& 142& 71& 214& 107& 322& 161\\
484& 242& 121& 364& 182& 91& 274& 137& 412& 206\\
103& 310& 155& 466& 233& 700& 350& 175& 526& 263\\
790& 395& 1186& 593& 1780& 890& 445& 1336& 668& 334\\
167& 502& 251& 754& 377& 1132& 566& 283& 850& 425\\
1276& 638& 319& 958& 479& 1438& 719& 2158& 1079& 3238\\
1619& 4858& 2429& 7288& 3644& 1822& 911& 2734& 1367& 4102\\
2051& 6154& 3077& 9232& 4616& 2308& 1154& 577& 1732& 866\\
433& 1300& 650& 325& 976& 488& 244& 122& 61& 184\\
92& 46& 23& 70& 35& 106& 53& 160& 80& 40\\
20& 10& 5& 16& 8& 4& 2& 1& \\

98&&&&&&&&&\\
49& 148& 74& 37& 112& 56& 28& 14& 7& 22\\
11& 34& 17& 52& 26& 13& 40& 20& 10& 5\\
16& 8& 4& 2& 1& \\

99&&&&&&&&&\\
298& 149& 448& 224& 112& 56& 28& 14& 7& 22\\
11& 34& 17& 52& 26& 13& 40& 20& 10& 5\\
16& 8& 4& 2& 1& \\

100&&&&&&&&&\\
50& 25& 76& 38& 19& 58& 29& 88& 44& 22\\
11& 34& 17& 52& 26& 13& 40& 20& 10& 5\\
16& 8& 4& 2& 1& \\

101&&&&&&&&&\\
304& 152& 76& 38& 19& 58& 29& 88& 44& 22\\
11& 34& 17& 52& 26& 13& 40& 20& 10& 5\\
16& 8& 4& 2& 1& \\

102&&&&&&&&&\\
51& 154& 77& 232& 116& 58& 29& 88& 44& 22\\
11& 34& 17& 52& 26& 13& 40& 20& 10& 5\\
16& 8& 4& 2& 1& \\

103&&&&&&&&&\\
310& 155& 466& 233& 700& 350& 175& 526& 263& 790\\
395& 1186& 593& 1780& 890& 445& 1336& 668& 334& 167\\
502& 251& 754& 377& 1132& 566& 283& 850& 425& 1276\\
638& 319& 958& 479& 1438& 719& 2158& 1079& 3238& 1619\\
4858& 2429& 7288& 3644& 1822& 911& 2734& 1367& 4102& 2051\\
6154& 3077& 9232& 4616& 2308& 1154& 577& 1732& 866& 433\\
1300& 650& 325& 976& 488& 244& 122& 61& 184& 92\\
46& 23& 70& 35& 106& 53& 160& 80& 40& 20\\
10& 5& 16& 8& 4& 2& 1& \\

104&&&&&&&&&\\
52& 26& 13& 40& 20& 10& 5& 16& 8& 4\\
2& 1& \\

105&&&&&&&&&\\
316& 158& 79& 238& 119& 358& 179& 538& 269& 808\\
404& 202& 101& 304& 152& 76& 38& 19& 58& 29\\
88& 44& 22& 11& 34& 17& 52& 26& 13& 40\\
20& 10& 5& 16& 8& 4& 2& 1& \\

106&&&&&&&&&\\
53& 160& 80& 40& 20& 10& 5& 16& 8& 4\\
2& 1& \\

107&&&&&&&&&\\
322& 161& 484& 242& 121& 364& 182& 91& 274& 137\\
412& 206& 103& 310& 155& 466& 233& 700& 350& 175\\
526& 263& 790& 395& 1186& 593& 1780& 890& 445& 1336\\
668& 334& 167& 502& 251& 754& 377& 1132& 566& 283\\
850& 425& 1276& 638& 319& 958& 479& 1438& 719& 2158\\
1079& 3238& 1619& 4858& 2429& 7288& 3644& 1822& 911& 2734\\
1367& 4102& 2051& 6154& 3077& 9232& 4616& 2308& 1154& 577\\
1732& 866& 433& 1300& 650& 325& 976& 488& 244& 122\\
61& 184& 92& 46& 23& 70& 35& 106& 53& 160\\
80& 40& 20& 10& 5& 16& 8& 4& 2& 1\\

108&&&&&&&&&\\
54& 27& 82& 41& 124& 62& 31& 94& 47& 142\\
71& 214& 107& 322& 161& 484& 242& 121& 364& 182\\
91& 274& 137& 412& 206& 103& 310& 155& 466& 233\\
700& 350& 175& 526& 263& 790& 395& 1186& 593& 1780\\
890& 445& 1336& 668& 334& 167& 502& 251& 754& 377\\
1132& 566& 283& 850& 425& 1276& 638& 319& 958& 479\\
1438& 719& 2158& 1079& 3238& 1619& 4858& 2429& 7288& 3644\\
1822& 911& 2734& 1367& 4102& 2051& 6154& 3077& 9232& 4616\\
2308& 1154& 577& 1732& 866& 433& 1300& 650& 325& 976\\
488& 244& 122& 61& 184& 92& 46& 23& 70& 35\\
106& 53& 160& 80& 40& 20& 10& 5& 16& 8\\
4& 2& 1& \\

109&&&&&&&&&\\
328& 164& 82& 41& 124& 62& 31& 94& 47& 142\\
71& 214& 107& 322& 161& 484& 242& 121& 364& 182\\
91& 274& 137& 412& 206& 103& 310& 155& 466& 233\\
700& 350& 175& 526& 263& 790& 395& 1186& 593& 1780\\
890& 445& 1336& 668& 334& 167& 502& 251& 754& 377\\
1132& 566& 283& 850& 425& 1276& 638& 319& 958& 479\\
1438& 719& 2158& 1079& 3238& 1619& 4858& 2429& 7288& 3644\\
1822& 911& 2734& 1367& 4102& 2051& 6154& 3077& 9232& 4616\\
2308& 1154& 577& 1732& 866& 433& 1300& 650& 325& 976\\
488& 244& 122& 61& 184& 92& 46& 23& 70& 35\\
106& 53& 160& 80& 40& 20& 10& 5& 16& 8\\
4& 2& 1& \\

110&&&&&&&&&\\
55& 166& 83& 250& 125& 376& 188& 94& 47& 142\\
71& 214& 107& 322& 161& 484& 242& 121& 364& 182\\
91& 274& 137& 412& 206& 103& 310& 155& 466& 233\\
700& 350& 175& 526& 263& 790& 395& 1186& 593& 1780\\
890& 445& 1336& 668& 334& 167& 502& 251& 754& 377\\
1132& 566& 283& 850& 425& 1276& 638& 319& 958& 479\\
1438& 719& 2158& 1079& 3238& 1619& 4858& 2429& 7288& 3644\\
1822& 911& 2734& 1367& 4102& 2051& 6154& 3077& 9232& 4616\\
2308& 1154& 577& 1732& 866& 433& 1300& 650& 325& 976\\
488& 244& 122& 61& 184& 92& 46& 23& 70& 35\\
106& 53& 160& 80& 40& 20& 10& 5& 16& 8\\
4& 2& 1& \\

111&&&&&&&&&\\
334& 167& 502& 251& 754& 377& 1132& 566& 283& 850\\
425& 1276& 638& 319& 958& 479& 1438& 719& 2158& 1079\\
3238& 1619& 4858& 2429& 7288& 3644& 1822& 911& 2734& 1367\\
4102& 2051& 6154& 3077& 9232& 4616& 2308& 1154& 577& 1732\\
866& 433& 1300& 650& 325& 976& 488& 244& 122& 61\\
184& 92& 46& 23& 70& 35& 106& 53& 160& 80\\
40& 20& 10& 5& 16& 8& 4& 2& 1& \\

112&&&&&&&&&\\
56& 28& 14& 7& 22& 11& 34& 17& 52& 26\\
13& 40& 20& 10& 5& 16& 8& 4& 2& 1\\

113&&&&&&&&&\\
340& 170& 85& 256& 128& 64& 32& 16& 8& 4\\
2& 1& \\

114&&&&&&&&&\\
57& 172& 86& 43& 130& 65& 196& 98& 49& 148\\
74& 37& 112& 56& 28& 14& 7& 22& 11& 34\\
17& 52& 26& 13& 40& 20& 10& 5& 16& 8\\
4& 2& 1& \\

115&&&&&&&&&\\
346& 173& 520& 260& 130& 65& 196& 98& 49& 148\\
74& 37& 112& 56& 28& 14& 7& 22& 11& 34\\
17& 52& 26& 13& 40& 20& 10& 5& 16& 8\\
4& 2& 1& \\

116&&&&&&&&&\\
58& 29& 88& 44& 22& 11& 34& 17& 52& 26\\
13& 40& 20& 10& 5& 16& 8& 4& 2& 1\\

117&&&&&&&&&\\
352& 176& 88& 44& 22& 11& 34& 17& 52& 26\\
13& 40& 20& 10& 5& 16& 8& 4& 2& 1\\

118&&&&&&&&&\\
59& 178& 89& 268& 134& 67& 202& 101& 304& 152\\
76& 38& 19& 58& 29& 88& 44& 22& 11& 34\\
17& 52& 26& 13& 40& 20& 10& 5& 16& 8\\
4& 2& 1& \\

119&&&&&&&&&\\
358& 179& 538& 269& 808& 404& 202& 101& 304& 152\\
76& 38& 19& 58& 29& 88& 44& 22& 11& 34\\
17& 52& 26& 13& 40& 20& 10& 5& 16& 8\\
4& 2& 1& \\

120&&&&&&&&&\\
60& 30& 15& 46& 23& 70& 35& 106& 53& 160\\
80& 40& 20& 10& 5& 16& 8& 4& 2& 1\\

121&&&&&&&&&\\
364& 182& 91& 274& 137& 412& 206& 103& 310& 155\\
466& 233& 700& 350& 175& 526& 263& 790& 395& 1186\\
593& 1780& 890& 445& 1336& 668& 334& 167& 502& 251\\
754& 377& 1132& 566& 283& 850& 425& 1276& 638& 319\\
958& 479& 1438& 719& 2158& 1079& 3238& 1619& 4858& 2429\\
7288& 3644& 1822& 911& 2734& 1367& 4102& 2051& 6154& 3077\\
9232& 4616& 2308& 1154& 577& 1732& 866& 433& 1300& 650\\
325& 976& 488& 244& 122& 61& 184& 92& 46& 23\\
70& 35& 106& 53& 160& 80& 40& 20& 10& 5\\
16& 8& 4& 2& 1& \\

122&&&&&&&&&\\
61& 184& 92& 46& 23& 70& 35& 106& 53& 160\\
80& 40& 20& 10& 5& 16& 8& 4& 2& 1\\

123&&&&&&&&&\\
370& 185& 556& 278& 139& 418& 209& 628& 314& 157\\
472& 236& 118& 59& 178& 89& 268& 134& 67& 202\\
101& 304& 152& 76& 38& 19& 58& 29& 88& 44\\
22& 11& 34& 17& 52& 26& 13& 40& 20& 10\\
5& 16& 8& 4& 2& 1& \\

124&&&&&&&&&\\
62& 31& 94& 47& 142& 71& 214& 107& 322& 161\\
484& 242& 121& 364& 182& 91& 274& 137& 412& 206\\
103& 310& 155& 466& 233& 700& 350& 175& 526& 263\\
790& 395& 1186& 593& 1780& 890& 445& 1336& 668& 334\\
167& 502& 251& 754& 377& 1132& 566& 283& 850& 425\\
1276& 638& 319& 958& 479& 1438& 719& 2158& 1079& 3238\\
1619& 4858& 2429& 7288& 3644& 1822& 911& 2734& 1367& 4102\\
2051& 6154& 3077& 9232& 4616& 2308& 1154& 577& 1732& 866\\
433& 1300& 650& 325& 976& 488& 244& 122& 61& 184\\
92& 46& 23& 70& 35& 106& 53& 160& 80& 40\\
20& 10& 5& 16& 8& 4& 2& 1& \\

125&&&&&&&&&\\
376& 188& 94& 47& 142& 71& 214& 107& 322& 161\\
484& 242& 121& 364& 182& 91& 274& 137& 412& 206\\
103& 310& 155& 466& 233& 700& 350& 175& 526& 263\\
790& 395& 1186& 593& 1780& 890& 445& 1336& 668& 334\\
167& 502& 251& 754& 377& 1132& 566& 283& 850& 425\\
1276& 638& 319& 958& 479& 1438& 719& 2158& 1079& 3238\\
1619& 4858& 2429& 7288& 3644& 1822& 911& 2734& 1367& 4102\\
2051& 6154& 3077& 9232& 4616& 2308& 1154& 577& 1732& 866\\
433& 1300& 650& 325& 976& 488& 244& 122& 61& 184\\
92& 46& 23& 70& 35& 106& 53& 160& 80& 40\\
20& 10& 5& 16& 8& 4& 2& 1& \\

126&&&&&&&&&\\
63& 190& 95& 286& 143& 430& 215& 646& 323& 970\\
485& 1456& 728& 364& 182& 91& 274& 137& 412& 206\\
103& 310& 155& 466& 233& 700& 350& 175& 526& 263\\
790& 395& 1186& 593& 1780& 890& 445& 1336& 668& 334\\
167& 502& 251& 754& 377& 1132& 566& 283& 850& 425\\
1276& 638& 319& 958& 479& 1438& 719& 2158& 1079& 3238\\
1619& 4858& 2429& 7288& 3644& 1822& 911& 2734& 1367& 4102\\
2051& 6154& 3077& 9232& 4616& 2308& 1154& 577& 1732& 866\\
433& 1300& 650& 325& 976& 488& 244& 122& 61& 184\\
92& 46& 23& 70& 35& 106& 53& 160& 80& 40\\
20& 10& 5& 16& 8& 4& 2& 1& \\

127&&&&&&&&&\\
382& 191& 574& 287& 862& 431& 1294& 647& 1942& 971\\
2914& 1457& 4372& 2186& 1093& 3280& 1640& 820& 410& 205\\
616& 308& 154& 77& 232& 116& 58& 29& 88& 44\\
22& 11& 34& 17& 52& 26& 13& 40& 20& 10\\
5& 16& 8& 4& 2& 1& \\

128&&&&&&&&&\\
64& 32& 16& 8& 4& 2& 1& \\

129&&&&&&&&&\\
388& 194& 97& 292& 146& 73& 220& 110& 55& 166\\
83& 250& 125& 376& 188& 94& 47& 142& 71& 214\\
107& 322& 161& 484& 242& 121& 364& 182& 91& 274\\
137& 412& 206& 103& 310& 155& 466& 233& 700& 350\\
175& 526& 263& 790& 395& 1186& 593& 1780& 890& 445\\
1336& 668& 334& 167& 502& 251& 754& 377& 1132& 566\\
283& 850& 425& 1276& 638& 319& 958& 479& 1438& 719\\
2158& 1079& 3238& 1619& 4858& 2429& 7288& 3644& 1822& 911\\
2734& 1367& 4102& 2051& 6154& 3077& 9232& 4616& 2308& 1154\\
577& 1732& 866& 433& 1300& 650& 325& 976& 488& 244\\
122& 61& 184& 92& 46& 23& 70& 35& 106& 53\\
160& 80& 40& 20& 10& 5& 16& 8& 4& 2\\
1& \\

130&&&&&&&&&\\
65& 196& 98& 49& 148& 74& 37& 112& 56& 28\\
14& 7& 22& 11& 34& 17& 52& 26& 13& 40\\
20& 10& 5& 16& 8& 4& 2& 1& \\

131&&&&&&&&&\\
394& 197& 592& 296& 148& 74& 37& 112& 56& 28\\
14& 7& 22& 11& 34& 17& 52& 26& 13& 40\\
20& 10& 5& 16& 8& 4& 2& 1& \\

132&&&&&&&&&\\
66& 33& 100& 50& 25& 76& 38& 19& 58& 29\\
88& 44& 22& 11& 34& 17& 52& 26& 13& 40\\
20& 10& 5& 16& 8& 4& 2& 1& \\

133&&&&&&&&&\\
400& 200& 100& 50& 25& 76& 38& 19& 58& 29\\
88& 44& 22& 11& 34& 17& 52& 26& 13& 40\\
20& 10& 5& 16& 8& 4& 2& 1& \\

134&&&&&&&&&\\
67& 202& 101& 304& 152& 76& 38& 19& 58& 29\\
88& 44& 22& 11& 34& 17& 52& 26& 13& 40\\
20& 10& 5& 16& 8& 4& 2& 1& \\

135&&&&&&&&&\\
406& 203& 610& 305& 916& 458& 229& 688& 344& 172\\
86& 43& 130& 65& 196& 98& 49& 148& 74& 37\\
112& 56& 28& 14& 7& 22& 11& 34& 17& 52\\
26& 13& 40& 20& 10& 5& 16& 8& 4& 2\\
1& \\

136&&&&&&&&&\\
68& 34& 17& 52& 26& 13& 40& 20& 10& 5\\
16& 8& 4& 2& 1& \\

137&&&&&&&&&\\
412& 206& 103& 310& 155& 466& 233& 700& 350& 175\\
526& 263& 790& 395& 1186& 593& 1780& 890& 445& 1336\\
668& 334& 167& 502& 251& 754& 377& 1132& 566& 283\\
850& 425& 1276& 638& 319& 958& 479& 1438& 719& 2158\\
1079& 3238& 1619& 4858& 2429& 7288& 3644& 1822& 911& 2734\\
1367& 4102& 2051& 6154& 3077& 9232& 4616& 2308& 1154& 577\\
1732& 866& 433& 1300& 650& 325& 976& 488& 244& 122\\
61& 184& 92& 46& 23& 70& 35& 106& 53& 160\\
80& 40& 20& 10& 5& 16& 8& 4& 2& 1\\

138&&&&&&&&&\\
69& 208& 104& 52& 26& 13& 40& 20& 10& 5\\
16& 8& 4& 2& 1& \\

139&&&&&&&&&\\
418& 209& 628& 314& 157& 472& 236& 118& 59& 178\\
89& 268& 134& 67& 202& 101& 304& 152& 76& 38\\
19& 58& 29& 88& 44& 22& 11& 34& 17& 52\\
26& 13& 40& 20& 10& 5& 16& 8& 4& 2\\
1& \\

140&&&&&&&&&\\
70& 35& 106& 53& 160& 80& 40& 20& 10& 5\\
16& 8& 4& 2& 1& \\

141&&&&&&&&&\\
424& 212& 106& 53& 160& 80& 40& 20& 10& 5\\
16& 8& 4& 2& 1& \\

142&&&&&&&&&\\
71& 214& 107& 322& 161& 484& 242& 121& 364& 182\\
91& 274& 137& 412& 206& 103& 310& 155& 466& 233\\
700& 350& 175& 526& 263& 790& 395& 1186& 593& 1780\\
890& 445& 1336& 668& 334& 167& 502& 251& 754& 377\\
1132& 566& 283& 850& 425& 1276& 638& 319& 958& 479\\
1438& 719& 2158& 1079& 3238& 1619& 4858& 2429& 7288& 3644\\
1822& 911& 2734& 1367& 4102& 2051& 6154& 3077& 9232& 4616\\
2308& 1154& 577& 1732& 866& 433& 1300& 650& 325& 976\\
488& 244& 122& 61& 184& 92& 46& 23& 70& 35\\
106& 53& 160& 80& 40& 20& 10& 5& 16& 8\\
4& 2& 1& \\

143&&&&&&&&&\\
430& 215& 646& 323& 970& 485& 1456& 728& 364& 182\\
91& 274& 137& 412& 206& 103& 310& 155& 466& 233\\
700& 350& 175& 526& 263& 790& 395& 1186& 593& 1780\\
890& 445& 1336& 668& 334& 167& 502& 251& 754& 377\\
1132& 566& 283& 850& 425& 1276& 638& 319& 958& 479\\
1438& 719& 2158& 1079& 3238& 1619& 4858& 2429& 7288& 3644\\
1822& 911& 2734& 1367& 4102& 2051& 6154& 3077& 9232& 4616\\
2308& 1154& 577& 1732& 866& 433& 1300& 650& 325& 976\\
488& 244& 122& 61& 184& 92& 46& 23& 70& 35\\
106& 53& 160& 80& 40& 20& 10& 5& 16& 8\\
4& 2& 1& \\

144&&&&&&&&&\\
72& 36& 18& 9& 28& 14& 7& 22& 11& 34\\
17& 52& 26& 13& 40& 20& 10& 5& 16& 8\\
4& 2& 1& \\

145&&&&&&&&&\\
436& 218& 109& 328& 164& 82& 41& 124& 62& 31\\
94& 47& 142& 71& 214& 107& 322& 161& 484& 242\\
121& 364& 182& 91& 274& 137& 412& 206& 103& 310\\
155& 466& 233& 700& 350& 175& 526& 263& 790& 395\\
1186& 593& 1780& 890& 445& 1336& 668& 334& 167& 502\\
251& 754& 377& 1132& 566& 283& 850& 425& 1276& 638\\
319& 958& 479& 1438& 719& 2158& 1079& 3238& 1619& 4858\\
2429& 7288& 3644& 1822& 911& 2734& 1367& 4102& 2051& 6154\\
3077& 9232& 4616& 2308& 1154& 577& 1732& 866& 433& 1300\\
650& 325& 976& 488& 244& 122& 61& 184& 92& 46\\
23& 70& 35& 106& 53& 160& 80& 40& 20& 10\\
5& 16& 8& 4& 2& 1& \\

146&&&&&&&&&\\
73& 220& 110& 55& 166& 83& 250& 125& 376& 188\\
94& 47& 142& 71& 214& 107& 322& 161& 484& 242\\
121& 364& 182& 91& 274& 137& 412& 206& 103& 310\\
155& 466& 233& 700& 350& 175& 526& 263& 790& 395\\
1186& 593& 1780& 890& 445& 1336& 668& 334& 167& 502\\
251& 754& 377& 1132& 566& 283& 850& 425& 1276& 638\\
319& 958& 479& 1438& 719& 2158& 1079& 3238& 1619& 4858\\
2429& 7288& 3644& 1822& 911& 2734& 1367& 4102& 2051& 6154\\
3077& 9232& 4616& 2308& 1154& 577& 1732& 866& 433& 1300\\
650& 325& 976& 488& 244& 122& 61& 184& 92& 46\\
23& 70& 35& 106& 53& 160& 80& 40& 20& 10\\
5& 16& 8& 4& 2& 1& \\

147&&&&&&&&&\\
442& 221& 664& 332& 166& 83& 250& 125& 376& 188\\
94& 47& 142& 71& 214& 107& 322& 161& 484& 242\\
121& 364& 182& 91& 274& 137& 412& 206& 103& 310\\
155& 466& 233& 700& 350& 175& 526& 263& 790& 395\\
1186& 593& 1780& 890& 445& 1336& 668& 334& 167& 502\\
251& 754& 377& 1132& 566& 283& 850& 425& 1276& 638\\
319& 958& 479& 1438& 719& 2158& 1079& 3238& 1619& 4858\\
2429& 7288& 3644& 1822& 911& 2734& 1367& 4102& 2051& 6154\\
3077& 9232& 4616& 2308& 1154& 577& 1732& 866& 433& 1300\\
650& 325& 976& 488& 244& 122& 61& 184& 92& 46\\
23& 70& 35& 106& 53& 160& 80& 40& 20& 10\\
5& 16& 8& 4& 2& 1& \\

148&&&&&&&&&\\
74& 37& 112& 56& 28& 14& 7& 22& 11& 34\\
17& 52& 26& 13& 40& 20& 10& 5& 16& 8\\
4& 2& 1& \\

149&&&&&&&&&\\
448& 224& 112& 56& 28& 14& 7& 22& 11& 34\\
17& 52& 26& 13& 40& 20& 10& 5& 16& 8\\
4& 2& 1& \\

150&&&&&&&&&\\
75& 226& 113& 340& 170& 85& 256& 128& 64& 32\\
16& 8& 4& 2& 1& \\

151&&&&&&&&&\\
454& 227& 682& 341& 1024& 512& 256& 128& 64& 32\\
16& 8& 4& 2& 1& \\

152&&&&&&&&&\\
76& 38& 19& 58& 29& 88& 44& 22& 11& 34\\
17& 52& 26& 13& 40& 20& 10& 5& 16& 8\\
4& 2& 1& \\

153&&&&&&&&&\\
460& 230& 115& 346& 173& 520& 260& 130& 65& 196\\
98& 49& 148& 74& 37& 112& 56& 28& 14& 7\\
22& 11& 34& 17& 52& 26& 13& 40& 20& 10\\
5& 16& 8& 4& 2& 1& \\

154&&&&&&&&&\\
77& 232& 116& 58& 29& 88& 44& 22& 11& 34\\
17& 52& 26& 13& 40& 20& 10& 5& 16& 8\\
4& 2& 1& \\

155&&&&&&&&&\\
466& 233& 700& 350& 175& 526& 263& 790& 395& 1186\\
593& 1780& 890& 445& 1336& 668& 334& 167& 502& 251\\
754& 377& 1132& 566& 283& 850& 425& 1276& 638& 319\\
958& 479& 1438& 719& 2158& 1079& 3238& 1619& 4858& 2429\\
7288& 3644& 1822& 911& 2734& 1367& 4102& 2051& 6154& 3077\\
9232& 4616& 2308& 1154& 577& 1732& 866& 433& 1300& 650\\
325& 976& 488& 244& 122& 61& 184& 92& 46& 23\\
70& 35& 106& 53& 160& 80& 40& 20& 10& 5\\
16& 8& 4& 2& 1& \\

156&&&&&&&&&\\
78& 39& 118& 59& 178& 89& 268& 134& 67& 202\\
101& 304& 152& 76& 38& 19& 58& 29& 88& 44\\
22& 11& 34& 17& 52& 26& 13& 40& 20& 10\\
5& 16& 8& 4& 2& 1& \\

157&&&&&&&&&\\
472& 236& 118& 59& 178& 89& 268& 134& 67& 202\\
101& 304& 152& 76& 38& 19& 58& 29& 88& 44\\
22& 11& 34& 17& 52& 26& 13& 40& 20& 10\\
5& 16& 8& 4& 2& 1& \\

158&&&&&&&&&\\
79& 238& 119& 358& 179& 538& 269& 808& 404& 202\\
101& 304& 152& 76& 38& 19& 58& 29& 88& 44\\
22& 11& 34& 17& 52& 26& 13& 40& 20& 10\\
5& 16& 8& 4& 2& 1& \\

159&&&&&&&&&\\
478& 239& 718& 359& 1078& 539& 1618& 809& 2428& 1214\\
607& 1822& 911& 2734& 1367& 4102& 2051& 6154& 3077& 9232\\
4616& 2308& 1154& 577& 1732& 866& 433& 1300& 650& 325\\
976& 488& 244& 122& 61& 184& 92& 46& 23& 70\\
35& 106& 53& 160& 80& 40& 20& 10& 5& 16\\
8& 4& 2& 1& \\

160&&&&&&&&&\\
80& 40& 20& 10& 5& 16& 8& 4& 2& 1\\

161&&&&&&&&&\\
484& 242& 121& 364& 182& 91& 274& 137& 412& 206\\
103& 310& 155& 466& 233& 700& 350& 175& 526& 263\\
790& 395& 1186& 593& 1780& 890& 445& 1336& 668& 334\\
167& 502& 251& 754& 377& 1132& 566& 283& 850& 425\\
1276& 638& 319& 958& 479& 1438& 719& 2158& 1079& 3238\\
1619& 4858& 2429& 7288& 3644& 1822& 911& 2734& 1367& 4102\\
2051& 6154& 3077& 9232& 4616& 2308& 1154& 577& 1732& 866\\
433& 1300& 650& 325& 976& 488& 244& 122& 61& 184\\
92& 46& 23& 70& 35& 106& 53& 160& 80& 40\\
20& 10& 5& 16& 8& 4& 2& 1& \\

162&&&&&&&&&\\
81& 244& 122& 61& 184& 92& 46& 23& 70& 35\\
106& 53& 160& 80& 40& 20& 10& 5& 16& 8\\
4& 2& 1& \\

163&&&&&&&&&\\
490& 245& 736& 368& 184& 92& 46& 23& 70& 35\\
106& 53& 160& 80& 40& 20& 10& 5& 16& 8\\
4& 2& 1& \\

164&&&&&&&&&\\
82& 41& 124& 62& 31& 94& 47& 142& 71& 214\\
107& 322& 161& 484& 242& 121& 364& 182& 91& 274\\
137& 412& 206& 103& 310& 155& 466& 233& 700& 350\\
175& 526& 263& 790& 395& 1186& 593& 1780& 890& 445\\
1336& 668& 334& 167& 502& 251& 754& 377& 1132& 566\\
283& 850& 425& 1276& 638& 319& 958& 479& 1438& 719\\
2158& 1079& 3238& 1619& 4858& 2429& 7288& 3644& 1822& 911\\
2734& 1367& 4102& 2051& 6154& 3077& 9232& 4616& 2308& 1154\\
577& 1732& 866& 433& 1300& 650& 325& 976& 488& 244\\
122& 61& 184& 92& 46& 23& 70& 35& 106& 53\\
160& 80& 40& 20& 10& 5& 16& 8& 4& 2\\
1& \\

165&&&&&&&&&\\
496& 248& 124& 62& 31& 94& 47& 142& 71& 214\\
107& 322& 161& 484& 242& 121& 364& 182& 91& 274\\
137& 412& 206& 103& 310& 155& 466& 233& 700& 350\\
175& 526& 263& 790& 395& 1186& 593& 1780& 890& 445\\
1336& 668& 334& 167& 502& 251& 754& 377& 1132& 566\\
283& 850& 425& 1276& 638& 319& 958& 479& 1438& 719\\
2158& 1079& 3238& 1619& 4858& 2429& 7288& 3644& 1822& 911\\
2734& 1367& 4102& 2051& 6154& 3077& 9232& 4616& 2308& 1154\\
577& 1732& 866& 433& 1300& 650& 325& 976& 488& 244\\
122& 61& 184& 92& 46& 23& 70& 35& 106& 53\\
160& 80& 40& 20& 10& 5& 16& 8& 4& 2\\
1& \\

166&&&&&&&&&\\
83& 250& 125& 376& 188& 94& 47& 142& 71& 214\\
107& 322& 161& 484& 242& 121& 364& 182& 91& 274\\
137& 412& 206& 103& 310& 155& 466& 233& 700& 350\\
175& 526& 263& 790& 395& 1186& 593& 1780& 890& 445\\
1336& 668& 334& 167& 502& 251& 754& 377& 1132& 566\\
283& 850& 425& 1276& 638& 319& 958& 479& 1438& 719\\
2158& 1079& 3238& 1619& 4858& 2429& 7288& 3644& 1822& 911\\
2734& 1367& 4102& 2051& 6154& 3077& 9232& 4616& 2308& 1154\\
577& 1732& 866& 433& 1300& 650& 325& 976& 488& 244\\
122& 61& 184& 92& 46& 23& 70& 35& 106& 53\\
160& 80& 40& 20& 10& 5& 16& 8& 4& 2\\
1& \\

167&&&&&&&&&\\
502& 251& 754& 377& 1132& 566& 283& 850& 425& 1276\\
638& 319& 958& 479& 1438& 719& 2158& 1079& 3238& 1619\\
4858& 2429& 7288& 3644& 1822& 911& 2734& 1367& 4102& 2051\\
6154& 3077& 9232& 4616& 2308& 1154& 577& 1732& 866& 433\\
1300& 650& 325& 976& 488& 244& 122& 61& 184& 92\\
46& 23& 70& 35& 106& 53& 160& 80& 40& 20\\
10& 5& 16& 8& 4& 2& 1& \\

168&&&&&&&&&\\
84& 42& 21& 64& 32& 16& 8& 4& 2& 1\\

169&&&&&&&&&\\
508& 254& 127& 382& 191& 574& 287& 862& 431& 1294\\
647& 1942& 971& 2914& 1457& 4372& 2186& 1093& 3280& 1640\\
820& 410& 205& 616& 308& 154& 77& 232& 116& 58\\
29& 88& 44& 22& 11& 34& 17& 52& 26& 13\\
40& 20& 10& 5& 16& 8& 4& 2& 1& \\

170&&&&&&&&&\\
85& 256& 128& 64& 32& 16& 8& 4& 2& 1\\

171&&&&&&&&&\\
514& 257& 772& 386& 193& 580& 290& 145& 436& 218\\
109& 328& 164& 82& 41& 124& 62& 31& 94& 47\\
142& 71& 214& 107& 322& 161& 484& 242& 121& 364\\
182& 91& 274& 137& 412& 206& 103& 310& 155& 466\\
233& 700& 350& 175& 526& 263& 790& 395& 1186& 593\\
1780& 890& 445& 1336& 668& 334& 167& 502& 251& 754\\
377& 1132& 566& 283& 850& 425& 1276& 638& 319& 958\\
479& 1438& 719& 2158& 1079& 3238& 1619& 4858& 2429& 7288\\
3644& 1822& 911& 2734& 1367& 4102& 2051& 6154& 3077& 9232\\
4616& 2308& 1154& 577& 1732& 866& 433& 1300& 650& 325\\
976& 488& 244& 122& 61& 184& 92& 46& 23& 70\\
35& 106& 53& 160& 80& 40& 20& 10& 5& 16\\
8& 4& 2& 1& \\

172&&&&&&&&&\\
86& 43& 130& 65& 196& 98& 49& 148& 74& 37\\
112& 56& 28& 14& 7& 22& 11& 34& 17& 52\\
26& 13& 40& 20& 10& 5& 16& 8& 4& 2\\
1& \\

173&&&&&&&&&\\
520& 260& 130& 65& 196& 98& 49& 148& 74& 37\\
112& 56& 28& 14& 7& 22& 11& 34& 17& 52\\
26& 13& 40& 20& 10& 5& 16& 8& 4& 2\\
1& \\

174&&&&&&&&&\\
87& 262& 131& 394& 197& 592& 296& 148& 74& 37\\
112& 56& 28& 14& 7& 22& 11& 34& 17& 52\\
26& 13& 40& 20& 10& 5& 16& 8& 4& 2\\
1& \\

175&&&&&&&&&\\
526& 263& 790& 395& 1186& 593& 1780& 890& 445& 1336\\
668& 334& 167& 502& 251& 754& 377& 1132& 566& 283\\
850& 425& 1276& 638& 319& 958& 479& 1438& 719& 2158\\
1079& 3238& 1619& 4858& 2429& 7288& 3644& 1822& 911& 2734\\
1367& 4102& 2051& 6154& 3077& 9232& 4616& 2308& 1154& 577\\
1732& 866& 433& 1300& 650& 325& 976& 488& 244& 122\\
61& 184& 92& 46& 23& 70& 35& 106& 53& 160\\
80& 40& 20& 10& 5& 16& 8& 4& 2& 1\\

176&&&&&&&&&\\
88& 44& 22& 11& 34& 17& 52& 26& 13& 40\\
20& 10& 5& 16& 8& 4& 2& 1& \\

177&&&&&&&&&\\
532& 266& 133& 400& 200& 100& 50& 25& 76& 38\\
19& 58& 29& 88& 44& 22& 11& 34& 17& 52\\
26& 13& 40& 20& 10& 5& 16& 8& 4& 2\\
1& \\

178&&&&&&&&&\\
89& 268& 134& 67& 202& 101& 304& 152& 76& 38\\
19& 58& 29& 88& 44& 22& 11& 34& 17& 52\\
26& 13& 40& 20& 10& 5& 16& 8& 4& 2\\
1& \\

179&&&&&&&&&\\
538& 269& 808& 404& 202& 101& 304& 152& 76& 38\\
19& 58& 29& 88& 44& 22& 11& 34& 17& 52\\
26& 13& 40& 20& 10& 5& 16& 8& 4& 2\\
1& \\

180&&&&&&&&&\\
90& 45& 136& 68& 34& 17& 52& 26& 13& 40\\
20& 10& 5& 16& 8& 4& 2& 1& \\

181&&&&&&&&&\\
544& 272& 136& 68& 34& 17& 52& 26& 13& 40\\
20& 10& 5& 16& 8& 4& 2& 1& \\

182&&&&&&&&&\\
91& 274& 137& 412& 206& 103& 310& 155& 466& 233\\
700& 350& 175& 526& 263& 790& 395& 1186& 593& 1780\\
890& 445& 1336& 668& 334& 167& 502& 251& 754& 377\\
1132& 566& 283& 850& 425& 1276& 638& 319& 958& 479\\
1438& 719& 2158& 1079& 3238& 1619& 4858& 2429& 7288& 3644\\
1822& 911& 2734& 1367& 4102& 2051& 6154& 3077& 9232& 4616\\
2308& 1154& 577& 1732& 866& 433& 1300& 650& 325& 976\\
488& 244& 122& 61& 184& 92& 46& 23& 70& 35\\
106& 53& 160& 80& 40& 20& 10& 5& 16& 8\\
4& 2& 1& \\

183&&&&&&&&&\\
550& 275& 826& 413& 1240& 620& 310& 155& 466& 233\\
700& 350& 175& 526& 263& 790& 395& 1186& 593& 1780\\
890& 445& 1336& 668& 334& 167& 502& 251& 754& 377\\
1132& 566& 283& 850& 425& 1276& 638& 319& 958& 479\\
1438& 719& 2158& 1079& 3238& 1619& 4858& 2429& 7288& 3644\\
1822& 911& 2734& 1367& 4102& 2051& 6154& 3077& 9232& 4616\\
2308& 1154& 577& 1732& 866& 433& 1300& 650& 325& 976\\
488& 244& 122& 61& 184& 92& 46& 23& 70& 35\\
106& 53& 160& 80& 40& 20& 10& 5& 16& 8\\
4& 2& 1& \\

184&&&&&&&&&\\
92& 46& 23& 70& 35& 106& 53& 160& 80& 40\\
20& 10& 5& 16& 8& 4& 2& 1& \\

185&&&&&&&&&\\
556& 278& 139& 418& 209& 628& 314& 157& 472& 236\\
118& 59& 178& 89& 268& 134& 67& 202& 101& 304\\
152& 76& 38& 19& 58& 29& 88& 44& 22& 11\\
34& 17& 52& 26& 13& 40& 20& 10& 5& 16\\
8& 4& 2& 1& \\

186&&&&&&&&&\\
93& 280& 140& 70& 35& 106& 53& 160& 80& 40\\
20& 10& 5& 16& 8& 4& 2& 1& \\

187&&&&&&&&&\\
562& 281& 844& 422& 211& 634& 317& 952& 476& 238\\
119& 358& 179& 538& 269& 808& 404& 202& 101& 304\\
152& 76& 38& 19& 58& 29& 88& 44& 22& 11\\
34& 17& 52& 26& 13& 40& 20& 10& 5& 16\\
8& 4& 2& 1& \\

188&&&&&&&&&\\
94& 47& 142& 71& 214& 107& 322& 161& 484& 242\\
121& 364& 182& 91& 274& 137& 412& 206& 103& 310\\
155& 466& 233& 700& 350& 175& 526& 263& 790& 395\\
1186& 593& 1780& 890& 445& 1336& 668& 334& 167& 502\\
251& 754& 377& 1132& 566& 283& 850& 425& 1276& 638\\
319& 958& 479& 1438& 719& 2158& 1079& 3238& 1619& 4858\\
2429& 7288& 3644& 1822& 911& 2734& 1367& 4102& 2051& 6154\\
3077& 9232& 4616& 2308& 1154& 577& 1732& 866& 433& 1300\\
650& 325& 976& 488& 244& 122& 61& 184& 92& 46\\
23& 70& 35& 106& 53& 160& 80& 40& 20& 10\\
5& 16& 8& 4& 2& 1& \\

189&&&&&&&&&\\
568& 284& 142& 71& 214& 107& 322& 161& 484& 242\\
121& 364& 182& 91& 274& 137& 412& 206& 103& 310\\
155& 466& 233& 700& 350& 175& 526& 263& 790& 395\\
1186& 593& 1780& 890& 445& 1336& 668& 334& 167& 502\\
251& 754& 377& 1132& 566& 283& 850& 425& 1276& 638\\
319& 958& 479& 1438& 719& 2158& 1079& 3238& 1619& 4858\\
2429& 7288& 3644& 1822& 911& 2734& 1367& 4102& 2051& 6154\\
3077& 9232& 4616& 2308& 1154& 577& 1732& 866& 433& 1300\\
650& 325& 976& 488& 244& 122& 61& 184& 92& 46\\
23& 70& 35& 106& 53& 160& 80& 40& 20& 10\\
5& 16& 8& 4& 2& 1& \\

190&&&&&&&&&\\
95& 286& 143& 430& 215& 646& 323& 970& 485& 1456\\
728& 364& 182& 91& 274& 137& 412& 206& 103& 310\\
155& 466& 233& 700& 350& 175& 526& 263& 790& 395\\
1186& 593& 1780& 890& 445& 1336& 668& 334& 167& 502\\
251& 754& 377& 1132& 566& 283& 850& 425& 1276& 638\\
319& 958& 479& 1438& 719& 2158& 1079& 3238& 1619& 4858\\
2429& 7288& 3644& 1822& 911& 2734& 1367& 4102& 2051& 6154\\
3077& 9232& 4616& 2308& 1154& 577& 1732& 866& 433& 1300\\
650& 325& 976& 488& 244& 122& 61& 184& 92& 46\\
23& 70& 35& 106& 53& 160& 80& 40& 20& 10\\
5& 16& 8& 4& 2& 1& \\

191&&&&&&&&&\\
574& 287& 862& 431& 1294& 647& 1942& 971& 2914& 1457\\
4372& 2186& 1093& 3280& 1640& 820& 410& 205& 616& 308\\
154& 77& 232& 116& 58& 29& 88& 44& 22& 11\\
34& 17& 52& 26& 13& 40& 20& 10& 5& 16\\
8& 4& 2& 1& \\

192&&&&&&&&&\\
96& 48& 24& 12& 6& 3& 10& 5& 16& 8\\
4& 2& 1& \\

193&&&&&&&&&\\
580& 290& 145& 436& 218& 109& 328& 164& 82& 41\\
124& 62& 31& 94& 47& 142& 71& 214& 107& 322\\
161& 484& 242& 121& 364& 182& 91& 274& 137& 412\\
206& 103& 310& 155& 466& 233& 700& 350& 175& 526\\
263& 790& 395& 1186& 593& 1780& 890& 445& 1336& 668\\
334& 167& 502& 251& 754& 377& 1132& 566& 283& 850\\
425& 1276& 638& 319& 958& 479& 1438& 719& 2158& 1079\\
3238& 1619& 4858& 2429& 7288& 3644& 1822& 911& 2734& 1367\\
4102& 2051& 6154& 3077& 9232& 4616& 2308& 1154& 577& 1732\\
866& 433& 1300& 650& 325& 976& 488& 244& 122& 61\\
184& 92& 46& 23& 70& 35& 106& 53& 160& 80\\
40& 20& 10& 5& 16& 8& 4& 2& 1& \\

194&&&&&&&&&\\
97& 292& 146& 73& 220& 110& 55& 166& 83& 250\\
125& 376& 188& 94& 47& 142& 71& 214& 107& 322\\
161& 484& 242& 121& 364& 182& 91& 274& 137& 412\\
206& 103& 310& 155& 466& 233& 700& 350& 175& 526\\
263& 790& 395& 1186& 593& 1780& 890& 445& 1336& 668\\
334& 167& 502& 251& 754& 377& 1132& 566& 283& 850\\
425& 1276& 638& 319& 958& 479& 1438& 719& 2158& 1079\\
3238& 1619& 4858& 2429& 7288& 3644& 1822& 911& 2734& 1367\\
4102& 2051& 6154& 3077& 9232& 4616& 2308& 1154& 577& 1732\\
866& 433& 1300& 650& 325& 976& 488& 244& 122& 61\\
184& 92& 46& 23& 70& 35& 106& 53& 160& 80\\
40& 20& 10& 5& 16& 8& 4& 2& 1& \\

195&&&&&&&&&\\
586& 293& 880& 440& 220& 110& 55& 166& 83& 250\\
125& 376& 188& 94& 47& 142& 71& 214& 107& 322\\
161& 484& 242& 121& 364& 182& 91& 274& 137& 412\\
206& 103& 310& 155& 466& 233& 700& 350& 175& 526\\
263& 790& 395& 1186& 593& 1780& 890& 445& 1336& 668\\
334& 167& 502& 251& 754& 377& 1132& 566& 283& 850\\
425& 1276& 638& 319& 958& 479& 1438& 719& 2158& 1079\\
3238& 1619& 4858& 2429& 7288& 3644& 1822& 911& 2734& 1367\\
4102& 2051& 6154& 3077& 9232& 4616& 2308& 1154& 577& 1732\\
866& 433& 1300& 650& 325& 976& 488& 244& 122& 61\\
184& 92& 46& 23& 70& 35& 106& 53& 160& 80\\
40& 20& 10& 5& 16& 8& 4& 2& 1& \\

196&&&&&&&&&\\
98& 49& 148& 74& 37& 112& 56& 28& 14& 7\\
22& 11& 34& 17& 52& 26& 13& 40& 20& 10\\
5& 16& 8& 4& 2& 1& \\

197&&&&&&&&&\\
592& 296& 148& 74& 37& 112& 56& 28& 14& 7\\
22& 11& 34& 17& 52& 26& 13& 40& 20& 10\\
5& 16& 8& 4& 2& 1& \\

198&&&&&&&&&\\
99& 298& 149& 448& 224& 112& 56& 28& 14& 7\\
22& 11& 34& 17& 52& 26& 13& 40& 20& 10\\
5& 16& 8& 4& 2& 1& \\

199&&&&&&&&&\\
598& 299& 898& 449& 1348& 674& 337& 1012& 506& 253\\
760& 380& 190& 95& 286& 143& 430& 215& 646& 323\\
970& 485& 1456& 728& 364& 182& 91& 274& 137& 412\\
206& 103& 310& 155& 466& 233& 700& 350& 175& 526\\
263& 790& 395& 1186& 593& 1780& 890& 445& 1336& 668\\
334& 167& 502& 251& 754& 377& 1132& 566& 283& 850\\
425& 1276& 638& 319& 958& 479& 1438& 719& 2158& 1079\\
3238& 1619& 4858& 2429& 7288& 3644& 1822& 911& 2734& 1367\\
4102& 2051& 6154& 3077& 9232& 4616& 2308& 1154& 577& 1732\\
866& 433& 1300& 650& 325& 976& 488& 244& 122& 61\\
184& 92& 46& 23& 70& 35& 106& 53& 160& 80\\
40& 20& 10& 5& 16& 8& 4& 2& 1& \\

200&&&&&&&&&\\
100& 50& 25& 76& 38& 19& 58& 29& 88& 44\\
22& 11& 34& 17& 52& 26& 13& 40& 20& 10\\
5& 16& 8& 4& 2& 1& \\

201&&&&&&&&&\\
604& 302& 151& 454& 227& 682& 341& 1024& 512& 256\\
128& 64& 32& 16& 8& 4& 2& 1& \\

202&&&&&&&&&\\
101& 304& 152& 76& 38& 19& 58& 29& 88& 44\\
22& 11& 34& 17& 52& 26& 13& 40& 20& 10\\
5& 16& 8& 4& 2& 1& \\

203&&&&&&&&&\\
610& 305& 916& 458& 229& 688& 344& 172& 86& 43\\
130& 65& 196& 98& 49& 148& 74& 37& 112& 56\\
28& 14& 7& 22& 11& 34& 17& 52& 26& 13\\
40& 20& 10& 5& 16& 8& 4& 2& 1& \\

204&&&&&&&&&\\
102& 51& 154& 77& 232& 116& 58& 29& 88& 44\\
22& 11& 34& 17& 52& 26& 13& 40& 20& 10\\
5& 16& 8& 4& 2& 1& \\

205&&&&&&&&&\\
616& 308& 154& 77& 232& 116& 58& 29& 88& 44\\
22& 11& 34& 17& 52& 26& 13& 40& 20& 10\\
5& 16& 8& 4& 2& 1& \\

206&&&&&&&&&\\
103& 310& 155& 466& 233& 700& 350& 175& 526& 263\\
790& 395& 1186& 593& 1780& 890& 445& 1336& 668& 334\\
167& 502& 251& 754& 377& 1132& 566& 283& 850& 425\\
1276& 638& 319& 958& 479& 1438& 719& 2158& 1079& 3238\\
1619& 4858& 2429& 7288& 3644& 1822& 911& 2734& 1367& 4102\\
2051& 6154& 3077& 9232& 4616& 2308& 1154& 577& 1732& 866\\
433& 1300& 650& 325& 976& 488& 244& 122& 61& 184\\
92& 46& 23& 70& 35& 106& 53& 160& 80& 40\\
20& 10& 5& 16& 8& 4& 2& 1& \\

207&&&&&&&&&\\
622& 311& 934& 467& 1402& 701& 2104& 1052& 526& 263\\
790& 395& 1186& 593& 1780& 890& 445& 1336& 668& 334\\
167& 502& 251& 754& 377& 1132& 566& 283& 850& 425\\
1276& 638& 319& 958& 479& 1438& 719& 2158& 1079& 3238\\
1619& 4858& 2429& 7288& 3644& 1822& 911& 2734& 1367& 4102\\
2051& 6154& 3077& 9232& 4616& 2308& 1154& 577& 1732& 866\\
433& 1300& 650& 325& 976& 488& 244& 122& 61& 184\\
92& 46& 23& 70& 35& 106& 53& 160& 80& 40\\
20& 10& 5& 16& 8& 4& 2& 1& \\

208&&&&&&&&&\\
104& 52& 26& 13& 40& 20& 10& 5& 16& 8\\
4& 2& 1& \\

209&&&&&&&&&\\
628& 314& 157& 472& 236& 118& 59& 178& 89& 268\\
134& 67& 202& 101& 304& 152& 76& 38& 19& 58\\
29& 88& 44& 22& 11& 34& 17& 52& 26& 13\\
40& 20& 10& 5& 16& 8& 4& 2& 1& \\

210&&&&&&&&&\\
105& 316& 158& 79& 238& 119& 358& 179& 538& 269\\
808& 404& 202& 101& 304& 152& 76& 38& 19& 58\\
29& 88& 44& 22& 11& 34& 17& 52& 26& 13\\
40& 20& 10& 5& 16& 8& 4& 2& 1& \\

211&&&&&&&&&\\
634& 317& 952& 476& 238& 119& 358& 179& 538& 269\\
808& 404& 202& 101& 304& 152& 76& 38& 19& 58\\
29& 88& 44& 22& 11& 34& 17& 52& 26& 13\\
40& 20& 10& 5& 16& 8& 4& 2& 1& \\

212&&&&&&&&&\\
106& 53& 160& 80& 40& 20& 10& 5& 16& 8\\
4& 2& 1& \\

213&&&&&&&&&\\
640& 320& 160& 80& 40& 20& 10& 5& 16& 8\\
4& 2& 1& \\

214&&&&&&&&&\\
107& 322& 161& 484& 242& 121& 364& 182& 91& 274\\
137& 412& 206& 103& 310& 155& 466& 233& 700& 350\\
175& 526& 263& 790& 395& 1186& 593& 1780& 890& 445\\
1336& 668& 334& 167& 502& 251& 754& 377& 1132& 566\\
283& 850& 425& 1276& 638& 319& 958& 479& 1438& 719\\
2158& 1079& 3238& 1619& 4858& 2429& 7288& 3644& 1822& 911\\
2734& 1367& 4102& 2051& 6154& 3077& 9232& 4616& 2308& 1154\\
577& 1732& 866& 433& 1300& 650& 325& 976& 488& 244\\
122& 61& 184& 92& 46& 23& 70& 35& 106& 53\\
160& 80& 40& 20& 10& 5& 16& 8& 4& 2\\
1& \\

215&&&&&&&&&\\
646& 323& 970& 485& 1456& 728& 364& 182& 91& 274\\
137& 412& 206& 103& 310& 155& 466& 233& 700& 350\\
175& 526& 263& 790& 395& 1186& 593& 1780& 890& 445\\
1336& 668& 334& 167& 502& 251& 754& 377& 1132& 566\\
283& 850& 425& 1276& 638& 319& 958& 479& 1438& 719\\
2158& 1079& 3238& 1619& 4858& 2429& 7288& 3644& 1822& 911\\
2734& 1367& 4102& 2051& 6154& 3077& 9232& 4616& 2308& 1154\\
577& 1732& 866& 433& 1300& 650& 325& 976& 488& 244\\
122& 61& 184& 92& 46& 23& 70& 35& 106& 53\\
160& 80& 40& 20& 10& 5& 16& 8& 4& 2\\
1& \\

216&&&&&&&&&\\
108& 54& 27& 82& 41& 124& 62& 31& 94& 47\\
142& 71& 214& 107& 322& 161& 484& 242& 121& 364\\
182& 91& 274& 137& 412& 206& 103& 310& 155& 466\\
233& 700& 350& 175& 526& 263& 790& 395& 1186& 593\\
1780& 890& 445& 1336& 668& 334& 167& 502& 251& 754\\
377& 1132& 566& 283& 850& 425& 1276& 638& 319& 958\\
479& 1438& 719& 2158& 1079& 3238& 1619& 4858& 2429& 7288\\
3644& 1822& 911& 2734& 1367& 4102& 2051& 6154& 3077& 9232\\
4616& 2308& 1154& 577& 1732& 866& 433& 1300& 650& 325\\
976& 488& 244& 122& 61& 184& 92& 46& 23& 70\\
35& 106& 53& 160& 80& 40& 20& 10& 5& 16\\
8& 4& 2& 1& \\

217&&&&&&&&&\\
652& 326& 163& 490& 245& 736& 368& 184& 92& 46\\
23& 70& 35& 106& 53& 160& 80& 40& 20& 10\\
5& 16& 8& 4& 2& 1& \\

218&&&&&&&&&\\
109& 328& 164& 82& 41& 124& 62& 31& 94& 47\\
142& 71& 214& 107& 322& 161& 484& 242& 121& 364\\
182& 91& 274& 137& 412& 206& 103& 310& 155& 466\\
233& 700& 350& 175& 526& 263& 790& 395& 1186& 593\\
1780& 890& 445& 1336& 668& 334& 167& 502& 251& 754\\
377& 1132& 566& 283& 850& 425& 1276& 638& 319& 958\\
479& 1438& 719& 2158& 1079& 3238& 1619& 4858& 2429& 7288\\
3644& 1822& 911& 2734& 1367& 4102& 2051& 6154& 3077& 9232\\
4616& 2308& 1154& 577& 1732& 866& 433& 1300& 650& 325\\
976& 488& 244& 122& 61& 184& 92& 46& 23& 70\\
35& 106& 53& 160& 80& 40& 20& 10& 5& 16\\
8& 4& 2& 1& \\

219&&&&&&&&&\\
658& 329& 988& 494& 247& 742& 371& 1114& 557& 1672\\
836& 418& 209& 628& 314& 157& 472& 236& 118& 59\\
178& 89& 268& 134& 67& 202& 101& 304& 152& 76\\
38& 19& 58& 29& 88& 44& 22& 11& 34& 17\\
52& 26& 13& 40& 20& 10& 5& 16& 8& 4\\
2& 1& \\

220&&&&&&&&&\\
110& 55& 166& 83& 250& 125& 376& 188& 94& 47\\
142& 71& 214& 107& 322& 161& 484& 242& 121& 364\\
182& 91& 274& 137& 412& 206& 103& 310& 155& 466\\
233& 700& 350& 175& 526& 263& 790& 395& 1186& 593\\
1780& 890& 445& 1336& 668& 334& 167& 502& 251& 754\\
377& 1132& 566& 283& 850& 425& 1276& 638& 319& 958\\
479& 1438& 719& 2158& 1079& 3238& 1619& 4858& 2429& 7288\\
3644& 1822& 911& 2734& 1367& 4102& 2051& 6154& 3077& 9232\\
4616& 2308& 1154& 577& 1732& 866& 433& 1300& 650& 325\\
976& 488& 244& 122& 61& 184& 92& 46& 23& 70\\
35& 106& 53& 160& 80& 40& 20& 10& 5& 16\\
8& 4& 2& 1& \\

221&&&&&&&&&\\
664& 332& 166& 83& 250& 125& 376& 188& 94& 47\\
142& 71& 214& 107& 322& 161& 484& 242& 121& 364\\
182& 91& 274& 137& 412& 206& 103& 310& 155& 466\\
233& 700& 350& 175& 526& 263& 790& 395& 1186& 593\\
1780& 890& 445& 1336& 668& 334& 167& 502& 251& 754\\
377& 1132& 566& 283& 850& 425& 1276& 638& 319& 958\\
479& 1438& 719& 2158& 1079& 3238& 1619& 4858& 2429& 7288\\
3644& 1822& 911& 2734& 1367& 4102& 2051& 6154& 3077& 9232\\
4616& 2308& 1154& 577& 1732& 866& 433& 1300& 650& 325\\
976& 488& 244& 122& 61& 184& 92& 46& 23& 70\\
35& 106& 53& 160& 80& 40& 20& 10& 5& 16\\
8& 4& 2& 1& \\

222&&&&&&&&&\\
111& 334& 167& 502& 251& 754& 377& 1132& 566& 283\\
850& 425& 1276& 638& 319& 958& 479& 1438& 719& 2158\\
1079& 3238& 1619& 4858& 2429& 7288& 3644& 1822& 911& 2734\\
1367& 4102& 2051& 6154& 3077& 9232& 4616& 2308& 1154& 577\\
1732& 866& 433& 1300& 650& 325& 976& 488& 244& 122\\
61& 184& 92& 46& 23& 70& 35& 106& 53& 160\\
80& 40& 20& 10& 5& 16& 8& 4& 2& 1\\

223&&&&&&&&&\\
670& 335& 1006& 503& 1510& 755& 2266& 1133& 3400& 1700\\
850& 425& 1276& 638& 319& 958& 479& 1438& 719& 2158\\
1079& 3238& 1619& 4858& 2429& 7288& 3644& 1822& 911& 2734\\
1367& 4102& 2051& 6154& 3077& 9232& 4616& 2308& 1154& 577\\
1732& 866& 433& 1300& 650& 325& 976& 488& 244& 122\\
61& 184& 92& 46& 23& 70& 35& 106& 53& 160\\
80& 40& 20& 10& 5& 16& 8& 4& 2& 1\\

224&&&&&&&&&\\
112& 56& 28& 14& 7& 22& 11& 34& 17& 52\\
26& 13& 40& 20& 10& 5& 16& 8& 4& 2\\
1& \\

225&&&&&&&&&\\
676& 338& 169& 508& 254& 127& 382& 191& 574& 287\\
862& 431& 1294& 647& 1942& 971& 2914& 1457& 4372& 2186\\
1093& 3280& 1640& 820& 410& 205& 616& 308& 154& 77\\
232& 116& 58& 29& 88& 44& 22& 11& 34& 17\\
52& 26& 13& 40& 20& 10& 5& 16& 8& 4\\
2& 1& \\

226&&&&&&&&&\\
113& 340& 170& 85& 256& 128& 64& 32& 16& 8\\
4& 2& 1& \\

227&&&&&&&&&\\
682& 341& 1024& 512& 256& 128& 64& 32& 16& 8\\
4& 2& 1& \\

228&&&&&&&&&\\
114& 57& 172& 86& 43& 130& 65& 196& 98& 49\\
148& 74& 37& 112& 56& 28& 14& 7& 22& 11\\
34& 17& 52& 26& 13& 40& 20& 10& 5& 16\\
8& 4& 2& 1& \\

229&&&&&&&&&\\
688& 344& 172& 86& 43& 130& 65& 196& 98& 49\\
148& 74& 37& 112& 56& 28& 14& 7& 22& 11\\
34& 17& 52& 26& 13& 40& 20& 10& 5& 16\\
8& 4& 2& 1& \\

230&&&&&&&&&\\
115& 346& 173& 520& 260& 130& 65& 196& 98& 49\\
148& 74& 37& 112& 56& 28& 14& 7& 22& 11\\
34& 17& 52& 26& 13& 40& 20& 10& 5& 16\\
8& 4& 2& 1& \\

231&&&&&&&&&\\
694& 347& 1042& 521& 1564& 782& 391& 1174& 587& 1762\\
881& 2644& 1322& 661& 1984& 992& 496& 248& 124& 62\\
31& 94& 47& 142& 71& 214& 107& 322& 161& 484\\
242& 121& 364& 182& 91& 274& 137& 412& 206& 103\\
310& 155& 466& 233& 700& 350& 175& 526& 263& 790\\
395& 1186& 593& 1780& 890& 445& 1336& 668& 334& 167\\
502& 251& 754& 377& 1132& 566& 283& 850& 425& 1276\\
638& 319& 958& 479& 1438& 719& 2158& 1079& 3238& 1619\\
4858& 2429& 7288& 3644& 1822& 911& 2734& 1367& 4102& 2051\\
6154& 3077& 9232& 4616& 2308& 1154& 577& 1732& 866& 433\\
1300& 650& 325& 976& 488& 244& 122& 61& 184& 92\\
46& 23& 70& 35& 106& 53& 160& 80& 40& 20\\
10& 5& 16& 8& 4& 2& 1& \\

232&&&&&&&&&\\
116& 58& 29& 88& 44& 22& 11& 34& 17& 52\\
26& 13& 40& 20& 10& 5& 16& 8& 4& 2\\
1& \\

233&&&&&&&&&\\
700& 350& 175& 526& 263& 790& 395& 1186& 593& 1780\\
890& 445& 1336& 668& 334& 167& 502& 251& 754& 377\\
1132& 566& 283& 850& 425& 1276& 638& 319& 958& 479\\
1438& 719& 2158& 1079& 3238& 1619& 4858& 2429& 7288& 3644\\
1822& 911& 2734& 1367& 4102& 2051& 6154& 3077& 9232& 4616\\
2308& 1154& 577& 1732& 866& 433& 1300& 650& 325& 976\\
488& 244& 122& 61& 184& 92& 46& 23& 70& 35\\
106& 53& 160& 80& 40& 20& 10& 5& 16& 8\\
4& 2& 1& \\

234&&&&&&&&&\\
117& 352& 176& 88& 44& 22& 11& 34& 17& 52\\
26& 13& 40& 20& 10& 5& 16& 8& 4& 2\\
1& \\

235&&&&&&&&&\\
706& 353& 1060& 530& 265& 796& 398& 199& 598& 299\\
898& 449& 1348& 674& 337& 1012& 506& 253& 760& 380\\
190& 95& 286& 143& 430& 215& 646& 323& 970& 485\\
1456& 728& 364& 182& 91& 274& 137& 412& 206& 103\\
310& 155& 466& 233& 700& 350& 175& 526& 263& 790\\
395& 1186& 593& 1780& 890& 445& 1336& 668& 334& 167\\
502& 251& 754& 377& 1132& 566& 283& 850& 425& 1276\\
638& 319& 958& 479& 1438& 719& 2158& 1079& 3238& 1619\\
4858& 2429& 7288& 3644& 1822& 911& 2734& 1367& 4102& 2051\\
6154& 3077& 9232& 4616& 2308& 1154& 577& 1732& 866& 433\\
1300& 650& 325& 976& 488& 244& 122& 61& 184& 92\\
46& 23& 70& 35& 106& 53& 160& 80& 40& 20\\
10& 5& 16& 8& 4& 2& 1& \\

236&&&&&&&&&\\
118& 59& 178& 89& 268& 134& 67& 202& 101& 304\\
152& 76& 38& 19& 58& 29& 88& 44& 22& 11\\
34& 17& 52& 26& 13& 40& 20& 10& 5& 16\\
8& 4& 2& 1& \\

237&&&&&&&&&\\
712& 356& 178& 89& 268& 134& 67& 202& 101& 304\\
152& 76& 38& 19& 58& 29& 88& 44& 22& 11\\
34& 17& 52& 26& 13& 40& 20& 10& 5& 16\\
8& 4& 2& 1& \\

238&&&&&&&&&\\
119& 358& 179& 538& 269& 808& 404& 202& 101& 304\\
152& 76& 38& 19& 58& 29& 88& 44& 22& 11\\
34& 17& 52& 26& 13& 40& 20& 10& 5& 16\\
8& 4& 2& 1& \\

239&&&&&&&&&\\
718& 359& 1078& 539& 1618& 809& 2428& 1214& 607& 1822\\
911& 2734& 1367& 4102& 2051& 6154& 3077& 9232& 4616& 2308\\
1154& 577& 1732& 866& 433& 1300& 650& 325& 976& 488\\
244& 122& 61& 184& 92& 46& 23& 70& 35& 106\\
53& 160& 80& 40& 20& 10& 5& 16& 8& 4\\
2& 1& \\

240&&&&&&&&&\\
120& 60& 30& 15& 46& 23& 70& 35& 106& 53\\
160& 80& 40& 20& 10& 5& 16& 8& 4& 2\\
1& \\

241&&&&&&&&&\\
724& 362& 181& 544& 272& 136& 68& 34& 17& 52\\
26& 13& 40& 20& 10& 5& 16& 8& 4& 2\\
1& \\

242&&&&&&&&&\\
121& 364& 182& 91& 274& 137& 412& 206& 103& 310\\
155& 466& 233& 700& 350& 175& 526& 263& 790& 395\\
1186& 593& 1780& 890& 445& 1336& 668& 334& 167& 502\\
251& 754& 377& 1132& 566& 283& 850& 425& 1276& 638\\
319& 958& 479& 1438& 719& 2158& 1079& 3238& 1619& 4858\\
2429& 7288& 3644& 1822& 911& 2734& 1367& 4102& 2051& 6154\\
3077& 9232& 4616& 2308& 1154& 577& 1732& 866& 433& 1300\\
650& 325& 976& 488& 244& 122& 61& 184& 92& 46\\
23& 70& 35& 106& 53& 160& 80& 40& 20& 10\\
5& 16& 8& 4& 2& 1& \\

243&&&&&&&&&\\
730& 365& 1096& 548& 274& 137& 412& 206& 103& 310\\
155& 466& 233& 700& 350& 175& 526& 263& 790& 395\\
1186& 593& 1780& 890& 445& 1336& 668& 334& 167& 502\\
251& 754& 377& 1132& 566& 283& 850& 425& 1276& 638\\
319& 958& 479& 1438& 719& 2158& 1079& 3238& 1619& 4858\\
2429& 7288& 3644& 1822& 911& 2734& 1367& 4102& 2051& 6154\\
3077& 9232& 4616& 2308& 1154& 577& 1732& 866& 433& 1300\\
650& 325& 976& 488& 244& 122& 61& 184& 92& 46\\
23& 70& 35& 106& 53& 160& 80& 40& 20& 10\\
5& 16& 8& 4& 2& 1& \\

244&&&&&&&&&\\
122& 61& 184& 92& 46& 23& 70& 35& 106& 53\\
160& 80& 40& 20& 10& 5& 16& 8& 4& 2\\
1& \\

245&&&&&&&&&\\
736& 368& 184& 92& 46& 23& 70& 35& 106& 53\\
160& 80& 40& 20& 10& 5& 16& 8& 4& 2\\
1& \\

246&&&&&&&&&\\
123& 370& 185& 556& 278& 139& 418& 209& 628& 314\\
157& 472& 236& 118& 59& 178& 89& 268& 134& 67\\
202& 101& 304& 152& 76& 38& 19& 58& 29& 88\\
44& 22& 11& 34& 17& 52& 26& 13& 40& 20\\
10& 5& 16& 8& 4& 2& 1& \\

247&&&&&&&&&\\
742& 371& 1114& 557& 1672& 836& 418& 209& 628& 314\\
157& 472& 236& 118& 59& 178& 89& 268& 134& 67\\
202& 101& 304& 152& 76& 38& 19& 58& 29& 88\\
44& 22& 11& 34& 17& 52& 26& 13& 40& 20\\
10& 5& 16& 8& 4& 2& 1& \\

248&&&&&&&&&\\
124& 62& 31& 94& 47& 142& 71& 214& 107& 322\\
161& 484& 242& 121& 364& 182& 91& 274& 137& 412\\
206& 103& 310& 155& 466& 233& 700& 350& 175& 526\\
263& 790& 395& 1186& 593& 1780& 890& 445& 1336& 668\\
334& 167& 502& 251& 754& 377& 1132& 566& 283& 850\\
425& 1276& 638& 319& 958& 479& 1438& 719& 2158& 1079\\
3238& 1619& 4858& 2429& 7288& 3644& 1822& 911& 2734& 1367\\
4102& 2051& 6154& 3077& 9232& 4616& 2308& 1154& 577& 1732\\
866& 433& 1300& 650& 325& 976& 488& 244& 122& 61\\
184& 92& 46& 23& 70& 35& 106& 53& 160& 80\\
40& 20& 10& 5& 16& 8& 4& 2& 1& \\

249&&&&&&&&&\\
748& 374& 187& 562& 281& 844& 422& 211& 634& 317\\
952& 476& 238& 119& 358& 179& 538& 269& 808& 404\\
202& 101& 304& 152& 76& 38& 19& 58& 29& 88\\
44& 22& 11& 34& 17& 52& 26& 13& 40& 20\\
10& 5& 16& 8& 4& 2& 1& \\

250&&&&&&&&&\\
125& 376& 188& 94& 47& 142& 71& 214& 107& 322\\
161& 484& 242& 121& 364& 182& 91& 274& 137& 412\\
206& 103& 310& 155& 466& 233& 700& 350& 175& 526\\
263& 790& 395& 1186& 593& 1780& 890& 445& 1336& 668\\
334& 167& 502& 251& 754& 377& 1132& 566& 283& 850\\
425& 1276& 638& 319& 958& 479& 1438& 719& 2158& 1079\\
3238& 1619& 4858& 2429& 7288& 3644& 1822& 911& 2734& 1367\\
4102& 2051& 6154& 3077& 9232& 4616& 2308& 1154& 577& 1732\\
866& 433& 1300& 650& 325& 976& 488& 244& 122& 61\\
184& 92& 46& 23& 70& 35& 106& 53& 160& 80\\
40& 20& 10& 5& 16& 8& 4& 2& 1& \\

251&&&&&&&&&\\
754& 377& 1132& 566& 283& 850& 425& 1276& 638& 319\\
958& 479& 1438& 719& 2158& 1079& 3238& 1619& 4858& 2429\\
7288& 3644& 1822& 911& 2734& 1367& 4102& 2051& 6154& 3077\\
9232& 4616& 2308& 1154& 577& 1732& 866& 433& 1300& 650\\
325& 976& 488& 244& 122& 61& 184& 92& 46& 23\\
70& 35& 106& 53& 160& 80& 40& 20& 10& 5\\
16& 8& 4& 2& 1& \\

252&&&&&&&&&\\
126& 63& 190& 95& 286& 143& 430& 215& 646& 323\\
970& 485& 1456& 728& 364& 182& 91& 274& 137& 412\\
206& 103& 310& 155& 466& 233& 700& 350& 175& 526\\
263& 790& 395& 1186& 593& 1780& 890& 445& 1336& 668\\
334& 167& 502& 251& 754& 377& 1132& 566& 283& 850\\
425& 1276& 638& 319& 958& 479& 1438& 719& 2158& 1079\\
3238& 1619& 4858& 2429& 7288& 3644& 1822& 911& 2734& 1367\\
4102& 2051& 6154& 3077& 9232& 4616& 2308& 1154& 577& 1732\\
866& 433& 1300& 650& 325& 976& 488& 244& 122& 61\\
184& 92& 46& 23& 70& 35& 106& 53& 160& 80\\
40& 20& 10& 5& 16& 8& 4& 2& 1& \\

253&&&&&&&&&\\
760& 380& 190& 95& 286& 143& 430& 215& 646& 323\\
970& 485& 1456& 728& 364& 182& 91& 274& 137& 412\\
206& 103& 310& 155& 466& 233& 700& 350& 175& 526\\
263& 790& 395& 1186& 593& 1780& 890& 445& 1336& 668\\
334& 167& 502& 251& 754& 377& 1132& 566& 283& 850\\
425& 1276& 638& 319& 958& 479& 1438& 719& 2158& 1079\\
3238& 1619& 4858& 2429& 7288& 3644& 1822& 911& 2734& 1367\\
4102& 2051& 6154& 3077& 9232& 4616& 2308& 1154& 577& 1732\\
866& 433& 1300& 650& 325& 976& 488& 244& 122& 61\\
184& 92& 46& 23& 70& 35& 106& 53& 160& 80\\
40& 20& 10& 5& 16& 8& 4& 2& 1& \\

254&&&&&&&&&\\
127& 382& 191& 574& 287& 862& 431& 1294& 647& 1942\\
971& 2914& 1457& 4372& 2186& 1093& 3280& 1640& 820& 410\\
205& 616& 308& 154& 77& 232& 116& 58& 29& 88\\
44& 22& 11& 34& 17& 52& 26& 13& 40& 20\\
10& 5& 16& 8& 4& 2& 1& \\

255&&&&&&&&&\\
766& 383& 1150& 575& 1726& 863& 2590& 1295& 3886& 1943\\
5830& 2915& 8746& 4373& 13120& 6560& 3280& 1640& 820& 410\\
205& 616& 308& 154& 77& 232& 116& 58& 29& 88\\
44& 22& 11& 34& 17& 52& 26& 13& 40& 20\\
10& 5& 16& 8& 4& 2& 1& \\

256&&&&&&&&&\\
128& 64& 32& 16& 8& 4& 2& 1& \\

257&&&&&&&&&\\
772& 386& 193& 580& 290& 145& 436& 218& 109& 328\\
164& 82& 41& 124& 62& 31& 94& 47& 142& 71\\
214& 107& 322& 161& 484& 242& 121& 364& 182& 91\\
274& 137& 412& 206& 103& 310& 155& 466& 233& 700\\
350& 175& 526& 263& 790& 395& 1186& 593& 1780& 890\\
445& 1336& 668& 334& 167& 502& 251& 754& 377& 1132\\
566& 283& 850& 425& 1276& 638& 319& 958& 479& 1438\\
719& 2158& 1079& 3238& 1619& 4858& 2429& 7288& 3644& 1822\\
911& 2734& 1367& 4102& 2051& 6154& 3077& 9232& 4616& 2308\\
1154& 577& 1732& 866& 433& 1300& 650& 325& 976& 488\\
244& 122& 61& 184& 92& 46& 23& 70& 35& 106\\
53& 160& 80& 40& 20& 10& 5& 16& 8& 4\\
2& 1& \\

258&&&&&&&&&\\
129& 388& 194& 97& 292& 146& 73& 220& 110& 55\\
166& 83& 250& 125& 376& 188& 94& 47& 142& 71\\
214& 107& 322& 161& 484& 242& 121& 364& 182& 91\\
274& 137& 412& 206& 103& 310& 155& 466& 233& 700\\
350& 175& 526& 263& 790& 395& 1186& 593& 1780& 890\\
445& 1336& 668& 334& 167& 502& 251& 754& 377& 1132\\
566& 283& 850& 425& 1276& 638& 319& 958& 479& 1438\\
719& 2158& 1079& 3238& 1619& 4858& 2429& 7288& 3644& 1822\\
911& 2734& 1367& 4102& 2051& 6154& 3077& 9232& 4616& 2308\\
1154& 577& 1732& 866& 433& 1300& 650& 325& 976& 488\\
244& 122& 61& 184& 92& 46& 23& 70& 35& 106\\
53& 160& 80& 40& 20& 10& 5& 16& 8& 4\\
2& 1& \\

259&&&&&&&&&\\
778& 389& 1168& 584& 292& 146& 73& 220& 110& 55\\
166& 83& 250& 125& 376& 188& 94& 47& 142& 71\\
214& 107& 322& 161& 484& 242& 121& 364& 182& 91\\
274& 137& 412& 206& 103& 310& 155& 466& 233& 700\\
350& 175& 526& 263& 790& 395& 1186& 593& 1780& 890\\
445& 1336& 668& 334& 167& 502& 251& 754& 377& 1132\\
566& 283& 850& 425& 1276& 638& 319& 958& 479& 1438\\
719& 2158& 1079& 3238& 1619& 4858& 2429& 7288& 3644& 1822\\
911& 2734& 1367& 4102& 2051& 6154& 3077& 9232& 4616& 2308\\
1154& 577& 1732& 866& 433& 1300& 650& 325& 976& 488\\
244& 122& 61& 184& 92& 46& 23& 70& 35& 106\\
53& 160& 80& 40& 20& 10& 5& 16& 8& 4\\
2& 1& \\

260&&&&&&&&&\\
130& 65& 196& 98& 49& 148& 74& 37& 112& 56\\
28& 14& 7& 22& 11& 34& 17& 52& 26& 13\\
40& 20& 10& 5& 16& 8& 4& 2& 1& \\

261&&&&&&&&&\\
784& 392& 196& 98& 49& 148& 74& 37& 112& 56\\
28& 14& 7& 22& 11& 34& 17& 52& 26& 13\\
40& 20& 10& 5& 16& 8& 4& 2& 1& \\

262&&&&&&&&&\\
131& 394& 197& 592& 296& 148& 74& 37& 112& 56\\
28& 14& 7& 22& 11& 34& 17& 52& 26& 13\\
40& 20& 10& 5& 16& 8& 4& 2& 1& \\

263&&&&&&&&&\\
790& 395& 1186& 593& 1780& 890& 445& 1336& 668& 334\\
167& 502& 251& 754& 377& 1132& 566& 283& 850& 425\\
1276& 638& 319& 958& 479& 1438& 719& 2158& 1079& 3238\\
1619& 4858& 2429& 7288& 3644& 1822& 911& 2734& 1367& 4102\\
2051& 6154& 3077& 9232& 4616& 2308& 1154& 577& 1732& 866\\
433& 1300& 650& 325& 976& 488& 244& 122& 61& 184\\
92& 46& 23& 70& 35& 106& 53& 160& 80& 40\\
20& 10& 5& 16& 8& 4& 2& 1& \\

264&&&&&&&&&\\
132& 66& 33& 100& 50& 25& 76& 38& 19& 58\\
29& 88& 44& 22& 11& 34& 17& 52& 26& 13\\
40& 20& 10& 5& 16& 8& 4& 2& 1& \\

265&&&&&&&&&\\
796& 398& 199& 598& 299& 898& 449& 1348& 674& 337\\
1012& 506& 253& 760& 380& 190& 95& 286& 143& 430\\
215& 646& 323& 970& 485& 1456& 728& 364& 182& 91\\
274& 137& 412& 206& 103& 310& 155& 466& 233& 700\\
350& 175& 526& 263& 790& 395& 1186& 593& 1780& 890\\
445& 1336& 668& 334& 167& 502& 251& 754& 377& 1132\\
566& 283& 850& 425& 1276& 638& 319& 958& 479& 1438\\
719& 2158& 1079& 3238& 1619& 4858& 2429& 7288& 3644& 1822\\
911& 2734& 1367& 4102& 2051& 6154& 3077& 9232& 4616& 2308\\
1154& 577& 1732& 866& 433& 1300& 650& 325& 976& 488\\
244& 122& 61& 184& 92& 46& 23& 70& 35& 106\\
53& 160& 80& 40& 20& 10& 5& 16& 8& 4\\
2& 1& \\

266&&&&&&&&&\\
133& 400& 200& 100& 50& 25& 76& 38& 19& 58\\
29& 88& 44& 22& 11& 34& 17& 52& 26& 13\\
40& 20& 10& 5& 16& 8& 4& 2& 1& \\

267&&&&&&&&&\\
802& 401& 1204& 602& 301& 904& 452& 226& 113& 340\\
170& 85& 256& 128& 64& 32& 16& 8& 4& 2\\
1& \\

268&&&&&&&&&\\
134& 67& 202& 101& 304& 152& 76& 38& 19& 58\\
29& 88& 44& 22& 11& 34& 17& 52& 26& 13\\
40& 20& 10& 5& 16& 8& 4& 2& 1& \\

269&&&&&&&&&\\
808& 404& 202& 101& 304& 152& 76& 38& 19& 58\\
29& 88& 44& 22& 11& 34& 17& 52& 26& 13\\
40& 20& 10& 5& 16& 8& 4& 2& 1& \\

270&&&&&&&&&\\
135& 406& 203& 610& 305& 916& 458& 229& 688& 344\\
172& 86& 43& 130& 65& 196& 98& 49& 148& 74\\
37& 112& 56& 28& 14& 7& 22& 11& 34& 17\\
52& 26& 13& 40& 20& 10& 5& 16& 8& 4\\
2& 1& \\

271&&&&&&&&&\\
814& 407& 1222& 611& 1834& 917& 2752& 1376& 688& 344\\
172& 86& 43& 130& 65& 196& 98& 49& 148& 74\\
37& 112& 56& 28& 14& 7& 22& 11& 34& 17\\
52& 26& 13& 40& 20& 10& 5& 16& 8& 4\\
2& 1& \\

272&&&&&&&&&\\
136& 68& 34& 17& 52& 26& 13& 40& 20& 10\\
5& 16& 8& 4& 2& 1& \\

273&&&&&&&&&\\
820& 410& 205& 616& 308& 154& 77& 232& 116& 58\\
29& 88& 44& 22& 11& 34& 17& 52& 26& 13\\
40& 20& 10& 5& 16& 8& 4& 2& 1& \\

274&&&&&&&&&\\
137& 412& 206& 103& 310& 155& 466& 233& 700& 350\\
175& 526& 263& 790& 395& 1186& 593& 1780& 890& 445\\
1336& 668& 334& 167& 502& 251& 754& 377& 1132& 566\\
283& 850& 425& 1276& 638& 319& 958& 479& 1438& 719\\
2158& 1079& 3238& 1619& 4858& 2429& 7288& 3644& 1822& 911\\
2734& 1367& 4102& 2051& 6154& 3077& 9232& 4616& 2308& 1154\\
577& 1732& 866& 433& 1300& 650& 325& 976& 488& 244\\
122& 61& 184& 92& 46& 23& 70& 35& 106& 53\\
160& 80& 40& 20& 10& 5& 16& 8& 4& 2\\
1& \\

275&&&&&&&&&\\
826& 413& 1240& 620& 310& 155& 466& 233& 700& 350\\
175& 526& 263& 790& 395& 1186& 593& 1780& 890& 445\\
1336& 668& 334& 167& 502& 251& 754& 377& 1132& 566\\
283& 850& 425& 1276& 638& 319& 958& 479& 1438& 719\\
2158& 1079& 3238& 1619& 4858& 2429& 7288& 3644& 1822& 911\\
2734& 1367& 4102& 2051& 6154& 3077& 9232& 4616& 2308& 1154\\
577& 1732& 866& 433& 1300& 650& 325& 976& 488& 244\\
122& 61& 184& 92& 46& 23& 70& 35& 106& 53\\
160& 80& 40& 20& 10& 5& 16& 8& 4& 2\\
1& \\

276&&&&&&&&&\\
138& 69& 208& 104& 52& 26& 13& 40& 20& 10\\
5& 16& 8& 4& 2& 1& \\

277&&&&&&&&&\\
832& 416& 208& 104& 52& 26& 13& 40& 20& 10\\
5& 16& 8& 4& 2& 1& \\

278&&&&&&&&&\\
139& 418& 209& 628& 314& 157& 472& 236& 118& 59\\
178& 89& 268& 134& 67& 202& 101& 304& 152& 76\\
38& 19& 58& 29& 88& 44& 22& 11& 34& 17\\
52& 26& 13& 40& 20& 10& 5& 16& 8& 4\\
2& 1& \\

279&&&&&&&&&\\
838& 419& 1258& 629& 1888& 944& 472& 236& 118& 59\\
178& 89& 268& 134& 67& 202& 101& 304& 152& 76\\
38& 19& 58& 29& 88& 44& 22& 11& 34& 17\\
52& 26& 13& 40& 20& 10& 5& 16& 8& 4\\
2& 1& \\

280&&&&&&&&&\\
140& 70& 35& 106& 53& 160& 80& 40& 20& 10\\
5& 16& 8& 4& 2& 1& \\

281&&&&&&&&&\\
844& 422& 211& 634& 317& 952& 476& 238& 119& 358\\
179& 538& 269& 808& 404& 202& 101& 304& 152& 76\\
38& 19& 58& 29& 88& 44& 22& 11& 34& 17\\
52& 26& 13& 40& 20& 10& 5& 16& 8& 4\\
2& 1& \\

282&&&&&&&&&\\
141& 424& 212& 106& 53& 160& 80& 40& 20& 10\\
5& 16& 8& 4& 2& 1& \\

283&&&&&&&&&\\
850& 425& 1276& 638& 319& 958& 479& 1438& 719& 2158\\
1079& 3238& 1619& 4858& 2429& 7288& 3644& 1822& 911& 2734\\
1367& 4102& 2051& 6154& 3077& 9232& 4616& 2308& 1154& 577\\
1732& 866& 433& 1300& 650& 325& 976& 488& 244& 122\\
61& 184& 92& 46& 23& 70& 35& 106& 53& 160\\
80& 40& 20& 10& 5& 16& 8& 4& 2& 1\\

284&&&&&&&&&\\
142& 71& 214& 107& 322& 161& 484& 242& 121& 364\\
182& 91& 274& 137& 412& 206& 103& 310& 155& 466\\
233& 700& 350& 175& 526& 263& 790& 395& 1186& 593\\
1780& 890& 445& 1336& 668& 334& 167& 502& 251& 754\\
377& 1132& 566& 283& 850& 425& 1276& 638& 319& 958\\
479& 1438& 719& 2158& 1079& 3238& 1619& 4858& 2429& 7288\\
3644& 1822& 911& 2734& 1367& 4102& 2051& 6154& 3077& 9232\\
4616& 2308& 1154& 577& 1732& 866& 433& 1300& 650& 325\\
976& 488& 244& 122& 61& 184& 92& 46& 23& 70\\
35& 106& 53& 160& 80& 40& 20& 10& 5& 16\\
8& 4& 2& 1& \\

285&&&&&&&&&\\
856& 428& 214& 107& 322& 161& 484& 242& 121& 364\\
182& 91& 274& 137& 412& 206& 103& 310& 155& 466\\
233& 700& 350& 175& 526& 263& 790& 395& 1186& 593\\
1780& 890& 445& 1336& 668& 334& 167& 502& 251& 754\\
377& 1132& 566& 283& 850& 425& 1276& 638& 319& 958\\
479& 1438& 719& 2158& 1079& 3238& 1619& 4858& 2429& 7288\\
3644& 1822& 911& 2734& 1367& 4102& 2051& 6154& 3077& 9232\\
4616& 2308& 1154& 577& 1732& 866& 433& 1300& 650& 325\\
976& 488& 244& 122& 61& 184& 92& 46& 23& 70\\
35& 106& 53& 160& 80& 40& 20& 10& 5& 16\\
8& 4& 2& 1& \\

286&&&&&&&&&\\
143& 430& 215& 646& 323& 970& 485& 1456& 728& 364\\
182& 91& 274& 137& 412& 206& 103& 310& 155& 466\\
233& 700& 350& 175& 526& 263& 790& 395& 1186& 593\\
1780& 890& 445& 1336& 668& 334& 167& 502& 251& 754\\
377& 1132& 566& 283& 850& 425& 1276& 638& 319& 958\\
479& 1438& 719& 2158& 1079& 3238& 1619& 4858& 2429& 7288\\
3644& 1822& 911& 2734& 1367& 4102& 2051& 6154& 3077& 9232\\
4616& 2308& 1154& 577& 1732& 866& 433& 1300& 650& 325\\
976& 488& 244& 122& 61& 184& 92& 46& 23& 70\\
35& 106& 53& 160& 80& 40& 20& 10& 5& 16\\
8& 4& 2& 1& \\

287&&&&&&&&&\\
862& 431& 1294& 647& 1942& 971& 2914& 1457& 4372& 2186\\
1093& 3280& 1640& 820& 410& 205& 616& 308& 154& 77\\
232& 116& 58& 29& 88& 44& 22& 11& 34& 17\\
52& 26& 13& 40& 20& 10& 5& 16& 8& 4\\
2& 1& \\

288&&&&&&&&&\\
144& 72& 36& 18& 9& 28& 14& 7& 22& 11\\
34& 17& 52& 26& 13& 40& 20& 10& 5& 16\\
8& 4& 2& 1& \\

289&&&&&&&&&\\
868& 434& 217& 652& 326& 163& 490& 245& 736& 368\\
184& 92& 46& 23& 70& 35& 106& 53& 160& 80\\
40& 20& 10& 5& 16& 8& 4& 2& 1& \\

290&&&&&&&&&\\
145& 436& 218& 109& 328& 164& 82& 41& 124& 62\\
31& 94& 47& 142& 71& 214& 107& 322& 161& 484\\
242& 121& 364& 182& 91& 274& 137& 412& 206& 103\\
310& 155& 466& 233& 700& 350& 175& 526& 263& 790\\
395& 1186& 593& 1780& 890& 445& 1336& 668& 334& 167\\
502& 251& 754& 377& 1132& 566& 283& 850& 425& 1276\\
638& 319& 958& 479& 1438& 719& 2158& 1079& 3238& 1619\\
4858& 2429& 7288& 3644& 1822& 911& 2734& 1367& 4102& 2051\\
6154& 3077& 9232& 4616& 2308& 1154& 577& 1732& 866& 433\\
1300& 650& 325& 976& 488& 244& 122& 61& 184& 92\\
46& 23& 70& 35& 106& 53& 160& 80& 40& 20\\
10& 5& 16& 8& 4& 2& 1& \\

291&&&&&&&&&\\
874& 437& 1312& 656& 328& 164& 82& 41& 124& 62\\
31& 94& 47& 142& 71& 214& 107& 322& 161& 484\\
242& 121& 364& 182& 91& 274& 137& 412& 206& 103\\
310& 155& 466& 233& 700& 350& 175& 526& 263& 790\\
395& 1186& 593& 1780& 890& 445& 1336& 668& 334& 167\\
502& 251& 754& 377& 1132& 566& 283& 850& 425& 1276\\
638& 319& 958& 479& 1438& 719& 2158& 1079& 3238& 1619\\
4858& 2429& 7288& 3644& 1822& 911& 2734& 1367& 4102& 2051\\
6154& 3077& 9232& 4616& 2308& 1154& 577& 1732& 866& 433\\
1300& 650& 325& 976& 488& 244& 122& 61& 184& 92\\
46& 23& 70& 35& 106& 53& 160& 80& 40& 20\\
10& 5& 16& 8& 4& 2& 1& \\

292&&&&&&&&&\\
146& 73& 220& 110& 55& 166& 83& 250& 125& 376\\
188& 94& 47& 142& 71& 214& 107& 322& 161& 484\\
242& 121& 364& 182& 91& 274& 137& 412& 206& 103\\
310& 155& 466& 233& 700& 350& 175& 526& 263& 790\\
395& 1186& 593& 1780& 890& 445& 1336& 668& 334& 167\\
502& 251& 754& 377& 1132& 566& 283& 850& 425& 1276\\
638& 319& 958& 479& 1438& 719& 2158& 1079& 3238& 1619\\
4858& 2429& 7288& 3644& 1822& 911& 2734& 1367& 4102& 2051\\
6154& 3077& 9232& 4616& 2308& 1154& 577& 1732& 866& 433\\
1300& 650& 325& 976& 488& 244& 122& 61& 184& 92\\
46& 23& 70& 35& 106& 53& 160& 80& 40& 20\\
10& 5& 16& 8& 4& 2& 1& \\

293&&&&&&&&&\\
880& 440& 220& 110& 55& 166& 83& 250& 125& 376\\
188& 94& 47& 142& 71& 214& 107& 322& 161& 484\\
242& 121& 364& 182& 91& 274& 137& 412& 206& 103\\
310& 155& 466& 233& 700& 350& 175& 526& 263& 790\\
395& 1186& 593& 1780& 890& 445& 1336& 668& 334& 167\\
502& 251& 754& 377& 1132& 566& 283& 850& 425& 1276\\
638& 319& 958& 479& 1438& 719& 2158& 1079& 3238& 1619\\
4858& 2429& 7288& 3644& 1822& 911& 2734& 1367& 4102& 2051\\
6154& 3077& 9232& 4616& 2308& 1154& 577& 1732& 866& 433\\
1300& 650& 325& 976& 488& 244& 122& 61& 184& 92\\
46& 23& 70& 35& 106& 53& 160& 80& 40& 20\\
10& 5& 16& 8& 4& 2& 1& \\

294&&&&&&&&&\\
147& 442& 221& 664& 332& 166& 83& 250& 125& 376\\
188& 94& 47& 142& 71& 214& 107& 322& 161& 484\\
242& 121& 364& 182& 91& 274& 137& 412& 206& 103\\
310& 155& 466& 233& 700& 350& 175& 526& 263& 790\\
395& 1186& 593& 1780& 890& 445& 1336& 668& 334& 167\\
502& 251& 754& 377& 1132& 566& 283& 850& 425& 1276\\
638& 319& 958& 479& 1438& 719& 2158& 1079& 3238& 1619\\
4858& 2429& 7288& 3644& 1822& 911& 2734& 1367& 4102& 2051\\
6154& 3077& 9232& 4616& 2308& 1154& 577& 1732& 866& 433\\
1300& 650& 325& 976& 488& 244& 122& 61& 184& 92\\
46& 23& 70& 35& 106& 53& 160& 80& 40& 20\\
10& 5& 16& 8& 4& 2& 1& \\

295&&&&&&&&&\\
886& 443& 1330& 665& 1996& 998& 499& 1498& 749& 2248\\
1124& 562& 281& 844& 422& 211& 634& 317& 952& 476\\
238& 119& 358& 179& 538& 269& 808& 404& 202& 101\\
304& 152& 76& 38& 19& 58& 29& 88& 44& 22\\
11& 34& 17& 52& 26& 13& 40& 20& 10& 5\\
16& 8& 4& 2& 1& \\

296&&&&&&&&&\\
148& 74& 37& 112& 56& 28& 14& 7& 22& 11\\
34& 17& 52& 26& 13& 40& 20& 10& 5& 16\\
8& 4& 2& 1& \\

297&&&&&&&&&\\
892& 446& 223& 670& 335& 1006& 503& 1510& 755& 2266\\
1133& 3400& 1700& 850& 425& 1276& 638& 319& 958& 479\\
1438& 719& 2158& 1079& 3238& 1619& 4858& 2429& 7288& 3644\\
1822& 911& 2734& 1367& 4102& 2051& 6154& 3077& 9232& 4616\\
2308& 1154& 577& 1732& 866& 433& 1300& 650& 325& 976\\
488& 244& 122& 61& 184& 92& 46& 23& 70& 35\\
106& 53& 160& 80& 40& 20& 10& 5& 16& 8\\
4& 2& 1& \\

298&&&&&&&&&\\
149& 448& 224& 112& 56& 28& 14& 7& 22& 11\\
34& 17& 52& 26& 13& 40& 20& 10& 5& 16\\
8& 4& 2& 1& \\

299&&&&&&&&&\\
898& 449& 1348& 674& 337& 1012& 506& 253& 760& 380\\
190& 95& 286& 143& 430& 215& 646& 323& 970& 485\\
1456& 728& 364& 182& 91& 274& 137& 412& 206& 103\\
310& 155& 466& 233& 700& 350& 175& 526& 263& 790\\
395& 1186& 593& 1780& 890& 445& 1336& 668& 334& 167\\
502& 251& 754& 377& 1132& 566& 283& 850& 425& 1276\\
638& 319& 958& 479& 1438& 719& 2158& 1079& 3238& 1619\\
4858& 2429& 7288& 3644& 1822& 911& 2734& 1367& 4102& 2051\\
6154& 3077& 9232& 4616& 2308& 1154& 577& 1732& 866& 433\\
1300& 650& 325& 976& 488& 244& 122& 61& 184& 92\\
46& 23& 70& 35& 106& 53& 160& 80& 40& 20\\
10& 5& 16& 8& 4& 2& 1& \\

300&&&&&&&&&\\
150& 75& 226& 113& 340& 170& 85& 256& 128& 64\\
32& 16& 8& 4& 2& 1& \\

301&&&&&&&&&\\
904& 452& 226& 113& 340& 170& 85& 256& 128& 64\\
32& 16& 8& 4& 2& 1& \\

302&&&&&&&&&\\
151& 454& 227& 682& 341& 1024& 512& 256& 128& 64\\
32& 16& 8& 4& 2& 1& \\

303&&&&&&&&&\\
910& 455& 1366& 683& 2050& 1025& 3076& 1538& 769& 2308\\
1154& 577& 1732& 866& 433& 1300& 650& 325& 976& 488\\
244& 122& 61& 184& 92& 46& 23& 70& 35& 106\\
53& 160& 80& 40& 20& 10& 5& 16& 8& 4\\
2& 1& \\

304&&&&&&&&&\\
152& 76& 38& 19& 58& 29& 88& 44& 22& 11\\
34& 17& 52& 26& 13& 40& 20& 10& 5& 16\\
8& 4& 2& 1& \\

305&&&&&&&&&\\
916& 458& 229& 688& 344& 172& 86& 43& 130& 65\\
196& 98& 49& 148& 74& 37& 112& 56& 28& 14\\
7& 22& 11& 34& 17& 52& 26& 13& 40& 20\\
10& 5& 16& 8& 4& 2& 1& \\

306&&&&&&&&&\\
153& 460& 230& 115& 346& 173& 520& 260& 130& 65\\
196& 98& 49& 148& 74& 37& 112& 56& 28& 14\\
7& 22& 11& 34& 17& 52& 26& 13& 40& 20\\
10& 5& 16& 8& 4& 2& 1& \\

307&&&&&&&&&\\
922& 461& 1384& 692& 346& 173& 520& 260& 130& 65\\
196& 98& 49& 148& 74& 37& 112& 56& 28& 14\\
7& 22& 11& 34& 17& 52& 26& 13& 40& 20\\
10& 5& 16& 8& 4& 2& 1& \\

308&&&&&&&&&\\
154& 77& 232& 116& 58& 29& 88& 44& 22& 11\\
34& 17& 52& 26& 13& 40& 20& 10& 5& 16\\
8& 4& 2& 1& \\

309&&&&&&&&&\\
928& 464& 232& 116& 58& 29& 88& 44& 22& 11\\
34& 17& 52& 26& 13& 40& 20& 10& 5& 16\\
8& 4& 2& 1& \\

310&&&&&&&&&\\
155& 466& 233& 700& 350& 175& 526& 263& 790& 395\\
1186& 593& 1780& 890& 445& 1336& 668& 334& 167& 502\\
251& 754& 377& 1132& 566& 283& 850& 425& 1276& 638\\
319& 958& 479& 1438& 719& 2158& 1079& 3238& 1619& 4858\\
2429& 7288& 3644& 1822& 911& 2734& 1367& 4102& 2051& 6154\\
3077& 9232& 4616& 2308& 1154& 577& 1732& 866& 433& 1300\\
650& 325& 976& 488& 244& 122& 61& 184& 92& 46\\
23& 70& 35& 106& 53& 160& 80& 40& 20& 10\\
5& 16& 8& 4& 2& 1& \\

311&&&&&&&&&\\
934& 467& 1402& 701& 2104& 1052& 526& 263& 790& 395\\
1186& 593& 1780& 890& 445& 1336& 668& 334& 167& 502\\
251& 754& 377& 1132& 566& 283& 850& 425& 1276& 638\\
319& 958& 479& 1438& 719& 2158& 1079& 3238& 1619& 4858\\
2429& 7288& 3644& 1822& 911& 2734& 1367& 4102& 2051& 6154\\
3077& 9232& 4616& 2308& 1154& 577& 1732& 866& 433& 1300\\
650& 325& 976& 488& 244& 122& 61& 184& 92& 46\\
23& 70& 35& 106& 53& 160& 80& 40& 20& 10\\
5& 16& 8& 4& 2& 1& \\

312&&&&&&&&&\\
156& 78& 39& 118& 59& 178& 89& 268& 134& 67\\
202& 101& 304& 152& 76& 38& 19& 58& 29& 88\\
44& 22& 11& 34& 17& 52& 26& 13& 40& 20\\
10& 5& 16& 8& 4& 2& 1& \\

313&&&&&&&&&\\
940& 470& 235& 706& 353& 1060& 530& 265& 796& 398\\
199& 598& 299& 898& 449& 1348& 674& 337& 1012& 506\\
253& 760& 380& 190& 95& 286& 143& 430& 215& 646\\
323& 970& 485& 1456& 728& 364& 182& 91& 274& 137\\
412& 206& 103& 310& 155& 466& 233& 700& 350& 175\\
526& 263& 790& 395& 1186& 593& 1780& 890& 445& 1336\\
668& 334& 167& 502& 251& 754& 377& 1132& 566& 283\\
850& 425& 1276& 638& 319& 958& 479& 1438& 719& 2158\\
1079& 3238& 1619& 4858& 2429& 7288& 3644& 1822& 911& 2734\\
1367& 4102& 2051& 6154& 3077& 9232& 4616& 2308& 1154& 577\\
1732& 866& 433& 1300& 650& 325& 976& 488& 244& 122\\
61& 184& 92& 46& 23& 70& 35& 106& 53& 160\\
80& 40& 20& 10& 5& 16& 8& 4& 2& 1\\

314&&&&&&&&&\\
157& 472& 236& 118& 59& 178& 89& 268& 134& 67\\
202& 101& 304& 152& 76& 38& 19& 58& 29& 88\\
44& 22& 11& 34& 17& 52& 26& 13& 40& 20\\
10& 5& 16& 8& 4& 2& 1& \\

315&&&&&&&&&\\
946& 473& 1420& 710& 355& 1066& 533& 1600& 800& 400\\
200& 100& 50& 25& 76& 38& 19& 58& 29& 88\\
44& 22& 11& 34& 17& 52& 26& 13& 40& 20\\
10& 5& 16& 8& 4& 2& 1& \\

316&&&&&&&&&\\
158& 79& 238& 119& 358& 179& 538& 269& 808& 404\\
202& 101& 304& 152& 76& 38& 19& 58& 29& 88\\
44& 22& 11& 34& 17& 52& 26& 13& 40& 20\\
10& 5& 16& 8& 4& 2& 1& \\

317&&&&&&&&&\\
952& 476& 238& 119& 358& 179& 538& 269& 808& 404\\
202& 101& 304& 152& 76& 38& 19& 58& 29& 88\\
44& 22& 11& 34& 17& 52& 26& 13& 40& 20\\
10& 5& 16& 8& 4& 2& 1& \\

318&&&&&&&&&\\
159& 478& 239& 718& 359& 1078& 539& 1618& 809& 2428\\
1214& 607& 1822& 911& 2734& 1367& 4102& 2051& 6154& 3077\\
9232& 4616& 2308& 1154& 577& 1732& 866& 433& 1300& 650\\
325& 976& 488& 244& 122& 61& 184& 92& 46& 23\\
70& 35& 106& 53& 160& 80& 40& 20& 10& 5\\
16& 8& 4& 2& 1& \\

319&&&&&&&&&\\
958& 479& 1438& 719& 2158& 1079& 3238& 1619& 4858& 2429\\
7288& 3644& 1822& 911& 2734& 1367& 4102& 2051& 6154& 3077\\
9232& 4616& 2308& 1154& 577& 1732& 866& 433& 1300& 650\\
325& 976& 488& 244& 122& 61& 184& 92& 46& 23\\
70& 35& 106& 53& 160& 80& 40& 20& 10& 5\\
16& 8& 4& 2& 1& \\

320&&&&&&&&&\\
160& 80& 40& 20& 10& 5& 16& 8& 4& 2\\
1& \\

321&&&&&&&&&\\
964& 482& 241& 724& 362& 181& 544& 272& 136& 68\\
34& 17& 52& 26& 13& 40& 20& 10& 5& 16\\
8& 4& 2& 1& \\

322&&&&&&&&&\\
161& 484& 242& 121& 364& 182& 91& 274& 137& 412\\
206& 103& 310& 155& 466& 233& 700& 350& 175& 526\\
263& 790& 395& 1186& 593& 1780& 890& 445& 1336& 668\\
334& 167& 502& 251& 754& 377& 1132& 566& 283& 850\\
425& 1276& 638& 319& 958& 479& 1438& 719& 2158& 1079\\
3238& 1619& 4858& 2429& 7288& 3644& 1822& 911& 2734& 1367\\
4102& 2051& 6154& 3077& 9232& 4616& 2308& 1154& 577& 1732\\
866& 433& 1300& 650& 325& 976& 488& 244& 122& 61\\
184& 92& 46& 23& 70& 35& 106& 53& 160& 80\\
40& 20& 10& 5& 16& 8& 4& 2& 1& \\

323&&&&&&&&&\\
970& 485& 1456& 728& 364& 182& 91& 274& 137& 412\\
206& 103& 310& 155& 466& 233& 700& 350& 175& 526\\
263& 790& 395& 1186& 593& 1780& 890& 445& 1336& 668\\
334& 167& 502& 251& 754& 377& 1132& 566& 283& 850\\
425& 1276& 638& 319& 958& 479& 1438& 719& 2158& 1079\\
3238& 1619& 4858& 2429& 7288& 3644& 1822& 911& 2734& 1367\\
4102& 2051& 6154& 3077& 9232& 4616& 2308& 1154& 577& 1732\\
866& 433& 1300& 650& 325& 976& 488& 244& 122& 61\\
184& 92& 46& 23& 70& 35& 106& 53& 160& 80\\
40& 20& 10& 5& 16& 8& 4& 2& 1& \\

324&&&&&&&&&\\
162& 81& 244& 122& 61& 184& 92& 46& 23& 70\\
35& 106& 53& 160& 80& 40& 20& 10& 5& 16\\
8& 4& 2& 1& \\

325&&&&&&&&&\\
976& 488& 244& 122& 61& 184& 92& 46& 23& 70\\
35& 106& 53& 160& 80& 40& 20& 10& 5& 16\\
8& 4& 2& 1& \\

326&&&&&&&&&\\
163& 490& 245& 736& 368& 184& 92& 46& 23& 70\\
35& 106& 53& 160& 80& 40& 20& 10& 5& 16\\
8& 4& 2& 1& \\

327&&&&&&&&&\\
982& 491& 1474& 737& 2212& 1106& 553& 1660& 830& 415\\
1246& 623& 1870& 935& 2806& 1403& 4210& 2105& 6316& 3158\\
1579& 4738& 2369& 7108& 3554& 1777& 5332& 2666& 1333& 4000\\
2000& 1000& 500& 250& 125& 376& 188& 94& 47& 142\\
71& 214& 107& 322& 161& 484& 242& 121& 364& 182\\
91& 274& 137& 412& 206& 103& 310& 155& 466& 233\\
700& 350& 175& 526& 263& 790& 395& 1186& 593& 1780\\
890& 445& 1336& 668& 334& 167& 502& 251& 754& 377\\
1132& 566& 283& 850& 425& 1276& 638& 319& 958& 479\\
1438& 719& 2158& 1079& 3238& 1619& 4858& 2429& 7288& 3644\\
1822& 911& 2734& 1367& 4102& 2051& 6154& 3077& 9232& 4616\\
2308& 1154& 577& 1732& 866& 433& 1300& 650& 325& 976\\
488& 244& 122& 61& 184& 92& 46& 23& 70& 35\\
106& 53& 160& 80& 40& 20& 10& 5& 16& 8\\
4& 2& 1& \\

328&&&&&&&&&\\
164& 82& 41& 124& 62& 31& 94& 47& 142& 71\\
214& 107& 322& 161& 484& 242& 121& 364& 182& 91\\
274& 137& 412& 206& 103& 310& 155& 466& 233& 700\\
350& 175& 526& 263& 790& 395& 1186& 593& 1780& 890\\
445& 1336& 668& 334& 167& 502& 251& 754& 377& 1132\\
566& 283& 850& 425& 1276& 638& 319& 958& 479& 1438\\
719& 2158& 1079& 3238& 1619& 4858& 2429& 7288& 3644& 1822\\
911& 2734& 1367& 4102& 2051& 6154& 3077& 9232& 4616& 2308\\
1154& 577& 1732& 866& 433& 1300& 650& 325& 976& 488\\
244& 122& 61& 184& 92& 46& 23& 70& 35& 106\\
53& 160& 80& 40& 20& 10& 5& 16& 8& 4\\
2& 1& \\

329&&&&&&&&&\\
988& 494& 247& 742& 371& 1114& 557& 1672& 836& 418\\
209& 628& 314& 157& 472& 236& 118& 59& 178& 89\\
268& 134& 67& 202& 101& 304& 152& 76& 38& 19\\
58& 29& 88& 44& 22& 11& 34& 17& 52& 26\\
13& 40& 20& 10& 5& 16& 8& 4& 2& 1\\

330&&&&&&&&&\\
165& 496& 248& 124& 62& 31& 94& 47& 142& 71\\
214& 107& 322& 161& 484& 242& 121& 364& 182& 91\\
274& 137& 412& 206& 103& 310& 155& 466& 233& 700\\
350& 175& 526& 263& 790& 395& 1186& 593& 1780& 890\\
445& 1336& 668& 334& 167& 502& 251& 754& 377& 1132\\
566& 283& 850& 425& 1276& 638& 319& 958& 479& 1438\\
719& 2158& 1079& 3238& 1619& 4858& 2429& 7288& 3644& 1822\\
911& 2734& 1367& 4102& 2051& 6154& 3077& 9232& 4616& 2308\\
1154& 577& 1732& 866& 433& 1300& 650& 325& 976& 488\\
244& 122& 61& 184& 92& 46& 23& 70& 35& 106\\
53& 160& 80& 40& 20& 10& 5& 16& 8& 4\\
2& 1& \\

331&&&&&&&&&\\
994& 497& 1492& 746& 373& 1120& 560& 280& 140& 70\\
35& 106& 53& 160& 80& 40& 20& 10& 5& 16\\
8& 4& 2& 1& \\

332&&&&&&&&&\\
166& 83& 250& 125& 376& 188& 94& 47& 142& 71\\
214& 107& 322& 161& 484& 242& 121& 364& 182& 91\\
274& 137& 412& 206& 103& 310& 155& 466& 233& 700\\
350& 175& 526& 263& 790& 395& 1186& 593& 1780& 890\\
445& 1336& 668& 334& 167& 502& 251& 754& 377& 1132\\
566& 283& 850& 425& 1276& 638& 319& 958& 479& 1438\\
719& 2158& 1079& 3238& 1619& 4858& 2429& 7288& 3644& 1822\\
911& 2734& 1367& 4102& 2051& 6154& 3077& 9232& 4616& 2308\\
1154& 577& 1732& 866& 433& 1300& 650& 325& 976& 488\\
244& 122& 61& 184& 92& 46& 23& 70& 35& 106\\
53& 160& 80& 40& 20& 10& 5& 16& 8& 4\\
2& 1& \\

333&&&&&&&&&\\
1000& 500& 250& 125& 376& 188& 94& 47& 142& 71\\
214& 107& 322& 161& 484& 242& 121& 364& 182& 91\\
274& 137& 412& 206& 103& 310& 155& 466& 233& 700\\
350& 175& 526& 263& 790& 395& 1186& 593& 1780& 890\\
445& 1336& 668& 334& 167& 502& 251& 754& 377& 1132\\
566& 283& 850& 425& 1276& 638& 319& 958& 479& 1438\\
719& 2158& 1079& 3238& 1619& 4858& 2429& 7288& 3644& 1822\\
911& 2734& 1367& 4102& 2051& 6154& 3077& 9232& 4616& 2308\\
1154& 577& 1732& 866& 433& 1300& 650& 325& 976& 488\\
244& 122& 61& 184& 92& 46& 23& 70& 35& 106\\
53& 160& 80& 40& 20& 10& 5& 16& 8& 4\\
2& 1& \\

334&&&&&&&&&\\
167& 502& 251& 754& 377& 1132& 566& 283& 850& 425\\
1276& 638& 319& 958& 479& 1438& 719& 2158& 1079& 3238\\
1619& 4858& 2429& 7288& 3644& 1822& 911& 2734& 1367& 4102\\
2051& 6154& 3077& 9232& 4616& 2308& 1154& 577& 1732& 866\\
433& 1300& 650& 325& 976& 488& 244& 122& 61& 184\\
92& 46& 23& 70& 35& 106& 53& 160& 80& 40\\
20& 10& 5& 16& 8& 4& 2& 1& \\

335&&&&&&&&&\\
1006& 503& 1510& 755& 2266& 1133& 3400& 1700& 850& 425\\
1276& 638& 319& 958& 479& 1438& 719& 2158& 1079& 3238\\
1619& 4858& 2429& 7288& 3644& 1822& 911& 2734& 1367& 4102\\
2051& 6154& 3077& 9232& 4616& 2308& 1154& 577& 1732& 866\\
433& 1300& 650& 325& 976& 488& 244& 122& 61& 184\\
92& 46& 23& 70& 35& 106& 53& 160& 80& 40\\
20& 10& 5& 16& 8& 4& 2& 1& \\

336&&&&&&&&&\\
168& 84& 42& 21& 64& 32& 16& 8& 4& 2\\
1& \\

337&&&&&&&&&\\
1012& 506& 253& 760& 380& 190& 95& 286& 143& 430\\
215& 646& 323& 970& 485& 1456& 728& 364& 182& 91\\
274& 137& 412& 206& 103& 310& 155& 466& 233& 700\\
350& 175& 526& 263& 790& 395& 1186& 593& 1780& 890\\
445& 1336& 668& 334& 167& 502& 251& 754& 377& 1132\\
566& 283& 850& 425& 1276& 638& 319& 958& 479& 1438\\
719& 2158& 1079& 3238& 1619& 4858& 2429& 7288& 3644& 1822\\
911& 2734& 1367& 4102& 2051& 6154& 3077& 9232& 4616& 2308\\
1154& 577& 1732& 866& 433& 1300& 650& 325& 976& 488\\
244& 122& 61& 184& 92& 46& 23& 70& 35& 106\\
53& 160& 80& 40& 20& 10& 5& 16& 8& 4\\
2& 1& \\

338&&&&&&&&&\\
169& 508& 254& 127& 382& 191& 574& 287& 862& 431\\
1294& 647& 1942& 971& 2914& 1457& 4372& 2186& 1093& 3280\\
1640& 820& 410& 205& 616& 308& 154& 77& 232& 116\\
58& 29& 88& 44& 22& 11& 34& 17& 52& 26\\
13& 40& 20& 10& 5& 16& 8& 4& 2& 1\\

339&&&&&&&&&\\
1018& 509& 1528& 764& 382& 191& 574& 287& 862& 431\\
1294& 647& 1942& 971& 2914& 1457& 4372& 2186& 1093& 3280\\
1640& 820& 410& 205& 616& 308& 154& 77& 232& 116\\
58& 29& 88& 44& 22& 11& 34& 17& 52& 26\\
13& 40& 20& 10& 5& 16& 8& 4& 2& 1\\

340&&&&&&&&&\\
170& 85& 256& 128& 64& 32& 16& 8& 4& 2\\
1& \\

341&&&&&&&&&\\
1024& 512& 256& 128& 64& 32& 16& 8& 4& 2\\
1& \\

342&&&&&&&&&\\
171& 514& 257& 772& 386& 193& 580& 290& 145& 436\\
218& 109& 328& 164& 82& 41& 124& 62& 31& 94\\
47& 142& 71& 214& 107& 322& 161& 484& 242& 121\\
364& 182& 91& 274& 137& 412& 206& 103& 310& 155\\
466& 233& 700& 350& 175& 526& 263& 790& 395& 1186\\
593& 1780& 890& 445& 1336& 668& 334& 167& 502& 251\\
754& 377& 1132& 566& 283& 850& 425& 1276& 638& 319\\
958& 479& 1438& 719& 2158& 1079& 3238& 1619& 4858& 2429\\
7288& 3644& 1822& 911& 2734& 1367& 4102& 2051& 6154& 3077\\
9232& 4616& 2308& 1154& 577& 1732& 866& 433& 1300& 650\\
325& 976& 488& 244& 122& 61& 184& 92& 46& 23\\
70& 35& 106& 53& 160& 80& 40& 20& 10& 5\\
16& 8& 4& 2& 1& \\

343&&&&&&&&&\\
1030& 515& 1546& 773& 2320& 1160& 580& 290& 145& 436\\
218& 109& 328& 164& 82& 41& 124& 62& 31& 94\\
47& 142& 71& 214& 107& 322& 161& 484& 242& 121\\
364& 182& 91& 274& 137& 412& 206& 103& 310& 155\\
466& 233& 700& 350& 175& 526& 263& 790& 395& 1186\\
593& 1780& 890& 445& 1336& 668& 334& 167& 502& 251\\
754& 377& 1132& 566& 283& 850& 425& 1276& 638& 319\\
958& 479& 1438& 719& 2158& 1079& 3238& 1619& 4858& 2429\\
7288& 3644& 1822& 911& 2734& 1367& 4102& 2051& 6154& 3077\\
9232& 4616& 2308& 1154& 577& 1732& 866& 433& 1300& 650\\
325& 976& 488& 244& 122& 61& 184& 92& 46& 23\\
70& 35& 106& 53& 160& 80& 40& 20& 10& 5\\
16& 8& 4& 2& 1& \\

344&&&&&&&&&\\
172& 86& 43& 130& 65& 196& 98& 49& 148& 74\\
37& 112& 56& 28& 14& 7& 22& 11& 34& 17\\
52& 26& 13& 40& 20& 10& 5& 16& 8& 4\\
2& 1& \\

345&&&&&&&&&\\
1036& 518& 259& 778& 389& 1168& 584& 292& 146& 73\\
220& 110& 55& 166& 83& 250& 125& 376& 188& 94\\
47& 142& 71& 214& 107& 322& 161& 484& 242& 121\\
364& 182& 91& 274& 137& 412& 206& 103& 310& 155\\
466& 233& 700& 350& 175& 526& 263& 790& 395& 1186\\
593& 1780& 890& 445& 1336& 668& 334& 167& 502& 251\\
754& 377& 1132& 566& 283& 850& 425& 1276& 638& 319\\
958& 479& 1438& 719& 2158& 1079& 3238& 1619& 4858& 2429\\
7288& 3644& 1822& 911& 2734& 1367& 4102& 2051& 6154& 3077\\
9232& 4616& 2308& 1154& 577& 1732& 866& 433& 1300& 650\\
325& 976& 488& 244& 122& 61& 184& 92& 46& 23\\
70& 35& 106& 53& 160& 80& 40& 20& 10& 5\\
16& 8& 4& 2& 1& \\

346&&&&&&&&&\\
173& 520& 260& 130& 65& 196& 98& 49& 148& 74\\
37& 112& 56& 28& 14& 7& 22& 11& 34& 17\\
52& 26& 13& 40& 20& 10& 5& 16& 8& 4\\
2& 1& \\

347&&&&&&&&&\\
1042& 521& 1564& 782& 391& 1174& 587& 1762& 881& 2644\\
1322& 661& 1984& 992& 496& 248& 124& 62& 31& 94\\
47& 142& 71& 214& 107& 322& 161& 484& 242& 121\\
364& 182& 91& 274& 137& 412& 206& 103& 310& 155\\
466& 233& 700& 350& 175& 526& 263& 790& 395& 1186\\
593& 1780& 890& 445& 1336& 668& 334& 167& 502& 251\\
754& 377& 1132& 566& 283& 850& 425& 1276& 638& 319\\
958& 479& 1438& 719& 2158& 1079& 3238& 1619& 4858& 2429\\
7288& 3644& 1822& 911& 2734& 1367& 4102& 2051& 6154& 3077\\
9232& 4616& 2308& 1154& 577& 1732& 866& 433& 1300& 650\\
325& 976& 488& 244& 122& 61& 184& 92& 46& 23\\
70& 35& 106& 53& 160& 80& 40& 20& 10& 5\\
16& 8& 4& 2& 1& \\

348&&&&&&&&&\\
174& 87& 262& 131& 394& 197& 592& 296& 148& 74\\
37& 112& 56& 28& 14& 7& 22& 11& 34& 17\\
52& 26& 13& 40& 20& 10& 5& 16& 8& 4\\
2& 1& \\

349&&&&&&&&&\\
1048& 524& 262& 131& 394& 197& 592& 296& 148& 74\\
37& 112& 56& 28& 14& 7& 22& 11& 34& 17\\
52& 26& 13& 40& 20& 10& 5& 16& 8& 4\\
2& 1& \\

350&&&&&&&&&\\
175& 526& 263& 790& 395& 1186& 593& 1780& 890& 445\\
1336& 668& 334& 167& 502& 251& 754& 377& 1132& 566\\
283& 850& 425& 1276& 638& 319& 958& 479& 1438& 719\\
2158& 1079& 3238& 1619& 4858& 2429& 7288& 3644& 1822& 911\\
2734& 1367& 4102& 2051& 6154& 3077& 9232& 4616& 2308& 1154\\
577& 1732& 866& 433& 1300& 650& 325& 976& 488& 244\\
122& 61& 184& 92& 46& 23& 70& 35& 106& 53\\
160& 80& 40& 20& 10& 5& 16& 8& 4& 2\\
1& \\

351&&&&&&&&&\\
1054& 527& 1582& 791& 2374& 1187& 3562& 1781& 5344& 2672\\
1336& 668& 334& 167& 502& 251& 754& 377& 1132& 566\\
283& 850& 425& 1276& 638& 319& 958& 479& 1438& 719\\
2158& 1079& 3238& 1619& 4858& 2429& 7288& 3644& 1822& 911\\
2734& 1367& 4102& 2051& 6154& 3077& 9232& 4616& 2308& 1154\\
577& 1732& 866& 433& 1300& 650& 325& 976& 488& 244\\
122& 61& 184& 92& 46& 23& 70& 35& 106& 53\\
160& 80& 40& 20& 10& 5& 16& 8& 4& 2\\
1& \\

352&&&&&&&&&\\
176& 88& 44& 22& 11& 34& 17& 52& 26& 13\\
40& 20& 10& 5& 16& 8& 4& 2& 1& \\

353&&&&&&&&&\\
1060& 530& 265& 796& 398& 199& 598& 299& 898& 449\\
1348& 674& 337& 1012& 506& 253& 760& 380& 190& 95\\
286& 143& 430& 215& 646& 323& 970& 485& 1456& 728\\
364& 182& 91& 274& 137& 412& 206& 103& 310& 155\\
466& 233& 700& 350& 175& 526& 263& 790& 395& 1186\\
593& 1780& 890& 445& 1336& 668& 334& 167& 502& 251\\
754& 377& 1132& 566& 283& 850& 425& 1276& 638& 319\\
958& 479& 1438& 719& 2158& 1079& 3238& 1619& 4858& 2429\\
7288& 3644& 1822& 911& 2734& 1367& 4102& 2051& 6154& 3077\\
9232& 4616& 2308& 1154& 577& 1732& 866& 433& 1300& 650\\
325& 976& 488& 244& 122& 61& 184& 92& 46& 23\\
70& 35& 106& 53& 160& 80& 40& 20& 10& 5\\
16& 8& 4& 2& 1& \\

354&&&&&&&&&\\
177& 532& 266& 133& 400& 200& 100& 50& 25& 76\\
38& 19& 58& 29& 88& 44& 22& 11& 34& 17\\
52& 26& 13& 40& 20& 10& 5& 16& 8& 4\\
2& 1& \\

355&&&&&&&&&\\
1066& 533& 1600& 800& 400& 200& 100& 50& 25& 76\\
38& 19& 58& 29& 88& 44& 22& 11& 34& 17\\
52& 26& 13& 40& 20& 10& 5& 16& 8& 4\\
2& 1& \\

356&&&&&&&&&\\
178& 89& 268& 134& 67& 202& 101& 304& 152& 76\\
38& 19& 58& 29& 88& 44& 22& 11& 34& 17\\
52& 26& 13& 40& 20& 10& 5& 16& 8& 4\\
2& 1& \\

357&&&&&&&&&\\
1072& 536& 268& 134& 67& 202& 101& 304& 152& 76\\
38& 19& 58& 29& 88& 44& 22& 11& 34& 17\\
52& 26& 13& 40& 20& 10& 5& 16& 8& 4\\
2& 1& \\

358&&&&&&&&&\\
179& 538& 269& 808& 404& 202& 101& 304& 152& 76\\
38& 19& 58& 29& 88& 44& 22& 11& 34& 17\\
52& 26& 13& 40& 20& 10& 5& 16& 8& 4\\
2& 1& \\

359&&&&&&&&&\\
1078& 539& 1618& 809& 2428& 1214& 607& 1822& 911& 2734\\
1367& 4102& 2051& 6154& 3077& 9232& 4616& 2308& 1154& 577\\
1732& 866& 433& 1300& 650& 325& 976& 488& 244& 122\\
61& 184& 92& 46& 23& 70& 35& 106& 53& 160\\
80& 40& 20& 10& 5& 16& 8& 4& 2& 1\\

360&&&&&&&&&\\
180& 90& 45& 136& 68& 34& 17& 52& 26& 13\\
40& 20& 10& 5& 16& 8& 4& 2& 1& \\

361&&&&&&&&&\\
1084& 542& 271& 814& 407& 1222& 611& 1834& 917& 2752\\
1376& 688& 344& 172& 86& 43& 130& 65& 196& 98\\
49& 148& 74& 37& 112& 56& 28& 14& 7& 22\\
11& 34& 17& 52& 26& 13& 40& 20& 10& 5\\
16& 8& 4& 2& 1& \\

362&&&&&&&&&\\
181& 544& 272& 136& 68& 34& 17& 52& 26& 13\\
40& 20& 10& 5& 16& 8& 4& 2& 1& \\

363&&&&&&&&&\\
1090& 545& 1636& 818& 409& 1228& 614& 307& 922& 461\\
1384& 692& 346& 173& 520& 260& 130& 65& 196& 98\\
49& 148& 74& 37& 112& 56& 28& 14& 7& 22\\
11& 34& 17& 52& 26& 13& 40& 20& 10& 5\\
16& 8& 4& 2& 1& \\

364&&&&&&&&&\\
182& 91& 274& 137& 412& 206& 103& 310& 155& 466\\
233& 700& 350& 175& 526& 263& 790& 395& 1186& 593\\
1780& 890& 445& 1336& 668& 334& 167& 502& 251& 754\\
377& 1132& 566& 283& 850& 425& 1276& 638& 319& 958\\
479& 1438& 719& 2158& 1079& 3238& 1619& 4858& 2429& 7288\\
3644& 1822& 911& 2734& 1367& 4102& 2051& 6154& 3077& 9232\\
4616& 2308& 1154& 577& 1732& 866& 433& 1300& 650& 325\\
976& 488& 244& 122& 61& 184& 92& 46& 23& 70\\
35& 106& 53& 160& 80& 40& 20& 10& 5& 16\\
8& 4& 2& 1& \\

365&&&&&&&&&\\
1096& 548& 274& 137& 412& 206& 103& 310& 155& 466\\
233& 700& 350& 175& 526& 263& 790& 395& 1186& 593\\
1780& 890& 445& 1336& 668& 334& 167& 502& 251& 754\\
377& 1132& 566& 283& 850& 425& 1276& 638& 319& 958\\
479& 1438& 719& 2158& 1079& 3238& 1619& 4858& 2429& 7288\\
3644& 1822& 911& 2734& 1367& 4102& 2051& 6154& 3077& 9232\\
4616& 2308& 1154& 577& 1732& 866& 433& 1300& 650& 325\\
976& 488& 244& 122& 61& 184& 92& 46& 23& 70\\
35& 106& 53& 160& 80& 40& 20& 10& 5& 16\\
8& 4& 2& 1& \\

366&&&&&&&&&\\
183& 550& 275& 826& 413& 1240& 620& 310& 155& 466\\
233& 700& 350& 175& 526& 263& 790& 395& 1186& 593\\
1780& 890& 445& 1336& 668& 334& 167& 502& 251& 754\\
377& 1132& 566& 283& 850& 425& 1276& 638& 319& 958\\
479& 1438& 719& 2158& 1079& 3238& 1619& 4858& 2429& 7288\\
3644& 1822& 911& 2734& 1367& 4102& 2051& 6154& 3077& 9232\\
4616& 2308& 1154& 577& 1732& 866& 433& 1300& 650& 325\\
976& 488& 244& 122& 61& 184& 92& 46& 23& 70\\
35& 106& 53& 160& 80& 40& 20& 10& 5& 16\\
8& 4& 2& 1& \\

367&&&&&&&&&\\
1102& 551& 1654& 827& 2482& 1241& 3724& 1862& 931& 2794\\
1397& 4192& 2096& 1048& 524& 262& 131& 394& 197& 592\\
296& 148& 74& 37& 112& 56& 28& 14& 7& 22\\
11& 34& 17& 52& 26& 13& 40& 20& 10& 5\\
16& 8& 4& 2& 1& \\

368&&&&&&&&&\\
184& 92& 46& 23& 70& 35& 106& 53& 160& 80\\
40& 20& 10& 5& 16& 8& 4& 2& 1& \\

369&&&&&&&&&\\
1108& 554& 277& 832& 416& 208& 104& 52& 26& 13\\
40& 20& 10& 5& 16& 8& 4& 2& 1& \\

370&&&&&&&&&\\
185& 556& 278& 139& 418& 209& 628& 314& 157& 472\\
236& 118& 59& 178& 89& 268& 134& 67& 202& 101\\
304& 152& 76& 38& 19& 58& 29& 88& 44& 22\\
11& 34& 17& 52& 26& 13& 40& 20& 10& 5\\
16& 8& 4& 2& 1& \\

371&&&&&&&&&\\
1114& 557& 1672& 836& 418& 209& 628& 314& 157& 472\\
236& 118& 59& 178& 89& 268& 134& 67& 202& 101\\
304& 152& 76& 38& 19& 58& 29& 88& 44& 22\\
11& 34& 17& 52& 26& 13& 40& 20& 10& 5\\
16& 8& 4& 2& 1& \\

372&&&&&&&&&\\
186& 93& 280& 140& 70& 35& 106& 53& 160& 80\\
40& 20& 10& 5& 16& 8& 4& 2& 1& \\

373&&&&&&&&&\\
1120& 560& 280& 140& 70& 35& 106& 53& 160& 80\\
40& 20& 10& 5& 16& 8& 4& 2& 1& \\

374&&&&&&&&&\\
187& 562& 281& 844& 422& 211& 634& 317& 952& 476\\
238& 119& 358& 179& 538& 269& 808& 404& 202& 101\\
304& 152& 76& 38& 19& 58& 29& 88& 44& 22\\
11& 34& 17& 52& 26& 13& 40& 20& 10& 5\\
16& 8& 4& 2& 1& \\

375&&&&&&&&&\\
1126& 563& 1690& 845& 2536& 1268& 634& 317& 952& 476\\
238& 119& 358& 179& 538& 269& 808& 404& 202& 101\\
304& 152& 76& 38& 19& 58& 29& 88& 44& 22\\
11& 34& 17& 52& 26& 13& 40& 20& 10& 5\\
16& 8& 4& 2& 1& \\

376&&&&&&&&&\\
188& 94& 47& 142& 71& 214& 107& 322& 161& 484\\
242& 121& 364& 182& 91& 274& 137& 412& 206& 103\\
310& 155& 466& 233& 700& 350& 175& 526& 263& 790\\
395& 1186& 593& 1780& 890& 445& 1336& 668& 334& 167\\
502& 251& 754& 377& 1132& 566& 283& 850& 425& 1276\\
638& 319& 958& 479& 1438& 719& 2158& 1079& 3238& 1619\\
4858& 2429& 7288& 3644& 1822& 911& 2734& 1367& 4102& 2051\\
6154& 3077& 9232& 4616& 2308& 1154& 577& 1732& 866& 433\\
1300& 650& 325& 976& 488& 244& 122& 61& 184& 92\\
46& 23& 70& 35& 106& 53& 160& 80& 40& 20\\
10& 5& 16& 8& 4& 2& 1& \\

377&&&&&&&&&\\
1132& 566& 283& 850& 425& 1276& 638& 319& 958& 479\\
1438& 719& 2158& 1079& 3238& 1619& 4858& 2429& 7288& 3644\\
1822& 911& 2734& 1367& 4102& 2051& 6154& 3077& 9232& 4616\\
2308& 1154& 577& 1732& 866& 433& 1300& 650& 325& 976\\
488& 244& 122& 61& 184& 92& 46& 23& 70& 35\\
106& 53& 160& 80& 40& 20& 10& 5& 16& 8\\
4& 2& 1& \\

378&&&&&&&&&\\
189& 568& 284& 142& 71& 214& 107& 322& 161& 484\\
242& 121& 364& 182& 91& 274& 137& 412& 206& 103\\
310& 155& 466& 233& 700& 350& 175& 526& 263& 790\\
395& 1186& 593& 1780& 890& 445& 1336& 668& 334& 167\\
502& 251& 754& 377& 1132& 566& 283& 850& 425& 1276\\
638& 319& 958& 479& 1438& 719& 2158& 1079& 3238& 1619\\
4858& 2429& 7288& 3644& 1822& 911& 2734& 1367& 4102& 2051\\
6154& 3077& 9232& 4616& 2308& 1154& 577& 1732& 866& 433\\
1300& 650& 325& 976& 488& 244& 122& 61& 184& 92\\
46& 23& 70& 35& 106& 53& 160& 80& 40& 20\\
10& 5& 16& 8& 4& 2& 1& \\

379&&&&&&&&&\\
1138& 569& 1708& 854& 427& 1282& 641& 1924& 962& 481\\
1444& 722& 361& 1084& 542& 271& 814& 407& 1222& 611\\
1834& 917& 2752& 1376& 688& 344& 172& 86& 43& 130\\
65& 196& 98& 49& 148& 74& 37& 112& 56& 28\\
14& 7& 22& 11& 34& 17& 52& 26& 13& 40\\
20& 10& 5& 16& 8& 4& 2& 1& \\

380&&&&&&&&&\\
190& 95& 286& 143& 430& 215& 646& 323& 970& 485\\
1456& 728& 364& 182& 91& 274& 137& 412& 206& 103\\
310& 155& 466& 233& 700& 350& 175& 526& 263& 790\\
395& 1186& 593& 1780& 890& 445& 1336& 668& 334& 167\\
502& 251& 754& 377& 1132& 566& 283& 850& 425& 1276\\
638& 319& 958& 479& 1438& 719& 2158& 1079& 3238& 1619\\
4858& 2429& 7288& 3644& 1822& 911& 2734& 1367& 4102& 2051\\
6154& 3077& 9232& 4616& 2308& 1154& 577& 1732& 866& 433\\
1300& 650& 325& 976& 488& 244& 122& 61& 184& 92\\
46& 23& 70& 35& 106& 53& 160& 80& 40& 20\\
10& 5& 16& 8& 4& 2& 1& \\

381&&&&&&&&&\\
1144& 572& 286& 143& 430& 215& 646& 323& 970& 485\\
1456& 728& 364& 182& 91& 274& 137& 412& 206& 103\\
310& 155& 466& 233& 700& 350& 175& 526& 263& 790\\
395& 1186& 593& 1780& 890& 445& 1336& 668& 334& 167\\
502& 251& 754& 377& 1132& 566& 283& 850& 425& 1276\\
638& 319& 958& 479& 1438& 719& 2158& 1079& 3238& 1619\\
4858& 2429& 7288& 3644& 1822& 911& 2734& 1367& 4102& 2051\\
6154& 3077& 9232& 4616& 2308& 1154& 577& 1732& 866& 433\\
1300& 650& 325& 976& 488& 244& 122& 61& 184& 92\\
46& 23& 70& 35& 106& 53& 160& 80& 40& 20\\
10& 5& 16& 8& 4& 2& 1& \\

382&&&&&&&&&\\
191& 574& 287& 862& 431& 1294& 647& 1942& 971& 2914\\
1457& 4372& 2186& 1093& 3280& 1640& 820& 410& 205& 616\\
308& 154& 77& 232& 116& 58& 29& 88& 44& 22\\
11& 34& 17& 52& 26& 13& 40& 20& 10& 5\\
16& 8& 4& 2& 1& \\

383&&&&&&&&&\\
1150& 575& 1726& 863& 2590& 1295& 3886& 1943& 5830& 2915\\
8746& 4373& 13120& 6560& 3280& 1640& 820& 410& 205& 616\\
308& 154& 77& 232& 116& 58& 29& 88& 44& 22\\
11& 34& 17& 52& 26& 13& 40& 20& 10& 5\\
16& 8& 4& 2& 1& \\

384&&&&&&&&&\\
192& 96& 48& 24& 12& 6& 3& 10& 5& 16\\
8& 4& 2& 1& \\

385&&&&&&&&&\\
1156& 578& 289& 868& 434& 217& 652& 326& 163& 490\\
245& 736& 368& 184& 92& 46& 23& 70& 35& 106\\
53& 160& 80& 40& 20& 10& 5& 16& 8& 4\\
2& 1& \\

386&&&&&&&&&\\
193& 580& 290& 145& 436& 218& 109& 328& 164& 82\\
41& 124& 62& 31& 94& 47& 142& 71& 214& 107\\
322& 161& 484& 242& 121& 364& 182& 91& 274& 137\\
412& 206& 103& 310& 155& 466& 233& 700& 350& 175\\
526& 263& 790& 395& 1186& 593& 1780& 890& 445& 1336\\
668& 334& 167& 502& 251& 754& 377& 1132& 566& 283\\
850& 425& 1276& 638& 319& 958& 479& 1438& 719& 2158\\
1079& 3238& 1619& 4858& 2429& 7288& 3644& 1822& 911& 2734\\
1367& 4102& 2051& 6154& 3077& 9232& 4616& 2308& 1154& 577\\
1732& 866& 433& 1300& 650& 325& 976& 488& 244& 122\\
61& 184& 92& 46& 23& 70& 35& 106& 53& 160\\
80& 40& 20& 10& 5& 16& 8& 4& 2& 1\\

387&&&&&&&&&\\
1162& 581& 1744& 872& 436& 218& 109& 328& 164& 82\\
41& 124& 62& 31& 94& 47& 142& 71& 214& 107\\
322& 161& 484& 242& 121& 364& 182& 91& 274& 137\\
412& 206& 103& 310& 155& 466& 233& 700& 350& 175\\
526& 263& 790& 395& 1186& 593& 1780& 890& 445& 1336\\
668& 334& 167& 502& 251& 754& 377& 1132& 566& 283\\
850& 425& 1276& 638& 319& 958& 479& 1438& 719& 2158\\
1079& 3238& 1619& 4858& 2429& 7288& 3644& 1822& 911& 2734\\
1367& 4102& 2051& 6154& 3077& 9232& 4616& 2308& 1154& 577\\
1732& 866& 433& 1300& 650& 325& 976& 488& 244& 122\\
61& 184& 92& 46& 23& 70& 35& 106& 53& 160\\
80& 40& 20& 10& 5& 16& 8& 4& 2& 1\\

388&&&&&&&&&\\
194& 97& 292& 146& 73& 220& 110& 55& 166& 83\\
250& 125& 376& 188& 94& 47& 142& 71& 214& 107\\
322& 161& 484& 242& 121& 364& 182& 91& 274& 137\\
412& 206& 103& 310& 155& 466& 233& 700& 350& 175\\
526& 263& 790& 395& 1186& 593& 1780& 890& 445& 1336\\
668& 334& 167& 502& 251& 754& 377& 1132& 566& 283\\
850& 425& 1276& 638& 319& 958& 479& 1438& 719& 2158\\
1079& 3238& 1619& 4858& 2429& 7288& 3644& 1822& 911& 2734\\
1367& 4102& 2051& 6154& 3077& 9232& 4616& 2308& 1154& 577\\
1732& 866& 433& 1300& 650& 325& 976& 488& 244& 122\\
61& 184& 92& 46& 23& 70& 35& 106& 53& 160\\
80& 40& 20& 10& 5& 16& 8& 4& 2& 1\\

389&&&&&&&&&\\
1168& 584& 292& 146& 73& 220& 110& 55& 166& 83\\
250& 125& 376& 188& 94& 47& 142& 71& 214& 107\\
322& 161& 484& 242& 121& 364& 182& 91& 274& 137\\
412& 206& 103& 310& 155& 466& 233& 700& 350& 175\\
526& 263& 790& 395& 1186& 593& 1780& 890& 445& 1336\\
668& 334& 167& 502& 251& 754& 377& 1132& 566& 283\\
850& 425& 1276& 638& 319& 958& 479& 1438& 719& 2158\\
1079& 3238& 1619& 4858& 2429& 7288& 3644& 1822& 911& 2734\\
1367& 4102& 2051& 6154& 3077& 9232& 4616& 2308& 1154& 577\\
1732& 866& 433& 1300& 650& 325& 976& 488& 244& 122\\
61& 184& 92& 46& 23& 70& 35& 106& 53& 160\\
80& 40& 20& 10& 5& 16& 8& 4& 2& 1\\

390&&&&&&&&&\\
195& 586& 293& 880& 440& 220& 110& 55& 166& 83\\
250& 125& 376& 188& 94& 47& 142& 71& 214& 107\\
322& 161& 484& 242& 121& 364& 182& 91& 274& 137\\
412& 206& 103& 310& 155& 466& 233& 700& 350& 175\\
526& 263& 790& 395& 1186& 593& 1780& 890& 445& 1336\\
668& 334& 167& 502& 251& 754& 377& 1132& 566& 283\\
850& 425& 1276& 638& 319& 958& 479& 1438& 719& 2158\\
1079& 3238& 1619& 4858& 2429& 7288& 3644& 1822& 911& 2734\\
1367& 4102& 2051& 6154& 3077& 9232& 4616& 2308& 1154& 577\\
1732& 866& 433& 1300& 650& 325& 976& 488& 244& 122\\
61& 184& 92& 46& 23& 70& 35& 106& 53& 160\\
80& 40& 20& 10& 5& 16& 8& 4& 2& 1\\

391&&&&&&&&&\\
1174& 587& 1762& 881& 2644& 1322& 661& 1984& 992& 496\\
248& 124& 62& 31& 94& 47& 142& 71& 214& 107\\
322& 161& 484& 242& 121& 364& 182& 91& 274& 137\\
412& 206& 103& 310& 155& 466& 233& 700& 350& 175\\
526& 263& 790& 395& 1186& 593& 1780& 890& 445& 1336\\
668& 334& 167& 502& 251& 754& 377& 1132& 566& 283\\
850& 425& 1276& 638& 319& 958& 479& 1438& 719& 2158\\
1079& 3238& 1619& 4858& 2429& 7288& 3644& 1822& 911& 2734\\
1367& 4102& 2051& 6154& 3077& 9232& 4616& 2308& 1154& 577\\
1732& 866& 433& 1300& 650& 325& 976& 488& 244& 122\\
61& 184& 92& 46& 23& 70& 35& 106& 53& 160\\
80& 40& 20& 10& 5& 16& 8& 4& 2& 1\\

392&&&&&&&&&\\
196& 98& 49& 148& 74& 37& 112& 56& 28& 14\\
7& 22& 11& 34& 17& 52& 26& 13& 40& 20\\
10& 5& 16& 8& 4& 2& 1& \\

393&&&&&&&&&\\
1180& 590& 295& 886& 443& 1330& 665& 1996& 998& 499\\
1498& 749& 2248& 1124& 562& 281& 844& 422& 211& 634\\
317& 952& 476& 238& 119& 358& 179& 538& 269& 808\\
404& 202& 101& 304& 152& 76& 38& 19& 58& 29\\
88& 44& 22& 11& 34& 17& 52& 26& 13& 40\\
20& 10& 5& 16& 8& 4& 2& 1& \\

394&&&&&&&&&\\
197& 592& 296& 148& 74& 37& 112& 56& 28& 14\\
7& 22& 11& 34& 17& 52& 26& 13& 40& 20\\
10& 5& 16& 8& 4& 2& 1& \\

395&&&&&&&&&\\
1186& 593& 1780& 890& 445& 1336& 668& 334& 167& 502\\
251& 754& 377& 1132& 566& 283& 850& 425& 1276& 638\\
319& 958& 479& 1438& 719& 2158& 1079& 3238& 1619& 4858\\
2429& 7288& 3644& 1822& 911& 2734& 1367& 4102& 2051& 6154\\
3077& 9232& 4616& 2308& 1154& 577& 1732& 866& 433& 1300\\
650& 325& 976& 488& 244& 122& 61& 184& 92& 46\\
23& 70& 35& 106& 53& 160& 80& 40& 20& 10\\
5& 16& 8& 4& 2& 1& \\

396&&&&&&&&&\\
198& 99& 298& 149& 448& 224& 112& 56& 28& 14\\
7& 22& 11& 34& 17& 52& 26& 13& 40& 20\\
10& 5& 16& 8& 4& 2& 1& \\

397&&&&&&&&&\\
1192& 596& 298& 149& 448& 224& 112& 56& 28& 14\\
7& 22& 11& 34& 17& 52& 26& 13& 40& 20\\
10& 5& 16& 8& 4& 2& 1& \\

398&&&&&&&&&\\
199& 598& 299& 898& 449& 1348& 674& 337& 1012& 506\\
253& 760& 380& 190& 95& 286& 143& 430& 215& 646\\
323& 970& 485& 1456& 728& 364& 182& 91& 274& 137\\
412& 206& 103& 310& 155& 466& 233& 700& 350& 175\\
526& 263& 790& 395& 1186& 593& 1780& 890& 445& 1336\\
668& 334& 167& 502& 251& 754& 377& 1132& 566& 283\\
850& 425& 1276& 638& 319& 958& 479& 1438& 719& 2158\\
1079& 3238& 1619& 4858& 2429& 7288& 3644& 1822& 911& 2734\\
1367& 4102& 2051& 6154& 3077& 9232& 4616& 2308& 1154& 577\\
1732& 866& 433& 1300& 650& 325& 976& 488& 244& 122\\
61& 184& 92& 46& 23& 70& 35& 106& 53& 160\\
80& 40& 20& 10& 5& 16& 8& 4& 2& 1\\

399&&&&&&&&&\\
1198& 599& 1798& 899& 2698& 1349& 4048& 2024& 1012& 506\\
253& 760& 380& 190& 95& 286& 143& 430& 215& 646\\
323& 970& 485& 1456& 728& 364& 182& 91& 274& 137\\
412& 206& 103& 310& 155& 466& 233& 700& 350& 175\\
526& 263& 790& 395& 1186& 593& 1780& 890& 445& 1336\\
668& 334& 167& 502& 251& 754& 377& 1132& 566& 283\\
850& 425& 1276& 638& 319& 958& 479& 1438& 719& 2158\\
1079& 3238& 1619& 4858& 2429& 7288& 3644& 1822& 911& 2734\\
1367& 4102& 2051& 6154& 3077& 9232& 4616& 2308& 1154& 577\\
1732& 866& 433& 1300& 650& 325& 976& 488& 244& 122\\
61& 184& 92& 46& 23& 70& 35& 106& 53& 160\\
80& 40& 20& 10& 5& 16& 8& 4& 2& 1\\

400&&&&&&&&&\\
200& 100& 50& 25& 76& 38& 19& 58& 29& 88\\
44& 22& 11& 34& 17& 52& 26& 13& 40& 20\\
10& 5& 16& 8& 4& 2& 1& \\

401&&&&&&&&&\\
1204& 602& 301& 904& 452& 226& 113& 340& 170& 85\\
256& 128& 64& 32& 16& 8& 4& 2& 1& \\

402&&&&&&&&&\\
201& 604& 302& 151& 454& 227& 682& 341& 1024& 512\\
256& 128& 64& 32& 16& 8& 4& 2& 1& \\

403&&&&&&&&&\\
1210& 605& 1816& 908& 454& 227& 682& 341& 1024& 512\\
256& 128& 64& 32& 16& 8& 4& 2& 1& \\

404&&&&&&&&&\\
202& 101& 304& 152& 76& 38& 19& 58& 29& 88\\
44& 22& 11& 34& 17& 52& 26& 13& 40& 20\\
10& 5& 16& 8& 4& 2& 1& \\

405&&&&&&&&&\\
1216& 608& 304& 152& 76& 38& 19& 58& 29& 88\\
44& 22& 11& 34& 17& 52& 26& 13& 40& 20\\
10& 5& 16& 8& 4& 2& 1& \\

406&&&&&&&&&\\
203& 610& 305& 916& 458& 229& 688& 344& 172& 86\\
43& 130& 65& 196& 98& 49& 148& 74& 37& 112\\
56& 28& 14& 7& 22& 11& 34& 17& 52& 26\\
13& 40& 20& 10& 5& 16& 8& 4& 2& 1\\

407&&&&&&&&&\\
1222& 611& 1834& 917& 2752& 1376& 688& 344& 172& 86\\
43& 130& 65& 196& 98& 49& 148& 74& 37& 112\\
56& 28& 14& 7& 22& 11& 34& 17& 52& 26\\
13& 40& 20& 10& 5& 16& 8& 4& 2& 1\\

408&&&&&&&&&\\
204& 102& 51& 154& 77& 232& 116& 58& 29& 88\\
44& 22& 11& 34& 17& 52& 26& 13& 40& 20\\
10& 5& 16& 8& 4& 2& 1& \\

409&&&&&&&&&\\
1228& 614& 307& 922& 461& 1384& 692& 346& 173& 520\\
260& 130& 65& 196& 98& 49& 148& 74& 37& 112\\
56& 28& 14& 7& 22& 11& 34& 17& 52& 26\\
13& 40& 20& 10& 5& 16& 8& 4& 2& 1\\

410&&&&&&&&&\\
205& 616& 308& 154& 77& 232& 116& 58& 29& 88\\
44& 22& 11& 34& 17& 52& 26& 13& 40& 20\\
10& 5& 16& 8& 4& 2& 1& \\

411&&&&&&&&&\\
1234& 617& 1852& 926& 463& 1390& 695& 2086& 1043& 3130\\
1565& 4696& 2348& 1174& 587& 1762& 881& 2644& 1322& 661\\
1984& 992& 496& 248& 124& 62& 31& 94& 47& 142\\
71& 214& 107& 322& 161& 484& 242& 121& 364& 182\\
91& 274& 137& 412& 206& 103& 310& 155& 466& 233\\
700& 350& 175& 526& 263& 790& 395& 1186& 593& 1780\\
890& 445& 1336& 668& 334& 167& 502& 251& 754& 377\\
1132& 566& 283& 850& 425& 1276& 638& 319& 958& 479\\
1438& 719& 2158& 1079& 3238& 1619& 4858& 2429& 7288& 3644\\
1822& 911& 2734& 1367& 4102& 2051& 6154& 3077& 9232& 4616\\
2308& 1154& 577& 1732& 866& 433& 1300& 650& 325& 976\\
488& 244& 122& 61& 184& 92& 46& 23& 70& 35\\
106& 53& 160& 80& 40& 20& 10& 5& 16& 8\\
4& 2& 1& \\

412&&&&&&&&&\\
206& 103& 310& 155& 466& 233& 700& 350& 175& 526\\
263& 790& 395& 1186& 593& 1780& 890& 445& 1336& 668\\
334& 167& 502& 251& 754& 377& 1132& 566& 283& 850\\
425& 1276& 638& 319& 958& 479& 1438& 719& 2158& 1079\\
3238& 1619& 4858& 2429& 7288& 3644& 1822& 911& 2734& 1367\\
4102& 2051& 6154& 3077& 9232& 4616& 2308& 1154& 577& 1732\\
866& 433& 1300& 650& 325& 976& 488& 244& 122& 61\\
184& 92& 46& 23& 70& 35& 106& 53& 160& 80\\
40& 20& 10& 5& 16& 8& 4& 2& 1& \\

413&&&&&&&&&\\
1240& 620& 310& 155& 466& 233& 700& 350& 175& 526\\
263& 790& 395& 1186& 593& 1780& 890& 445& 1336& 668\\
334& 167& 502& 251& 754& 377& 1132& 566& 283& 850\\
425& 1276& 638& 319& 958& 479& 1438& 719& 2158& 1079\\
3238& 1619& 4858& 2429& 7288& 3644& 1822& 911& 2734& 1367\\
4102& 2051& 6154& 3077& 9232& 4616& 2308& 1154& 577& 1732\\
866& 433& 1300& 650& 325& 976& 488& 244& 122& 61\\
184& 92& 46& 23& 70& 35& 106& 53& 160& 80\\
40& 20& 10& 5& 16& 8& 4& 2& 1& \\

414&&&&&&&&&\\
207& 622& 311& 934& 467& 1402& 701& 2104& 1052& 526\\
263& 790& 395& 1186& 593& 1780& 890& 445& 1336& 668\\
334& 167& 502& 251& 754& 377& 1132& 566& 283& 850\\
425& 1276& 638& 319& 958& 479& 1438& 719& 2158& 1079\\
3238& 1619& 4858& 2429& 7288& 3644& 1822& 911& 2734& 1367\\
4102& 2051& 6154& 3077& 9232& 4616& 2308& 1154& 577& 1732\\
866& 433& 1300& 650& 325& 976& 488& 244& 122& 61\\
184& 92& 46& 23& 70& 35& 106& 53& 160& 80\\
40& 20& 10& 5& 16& 8& 4& 2& 1& \\

415&&&&&&&&&\\
1246& 623& 1870& 935& 2806& 1403& 4210& 2105& 6316& 3158\\
1579& 4738& 2369& 7108& 3554& 1777& 5332& 2666& 1333& 4000\\
2000& 1000& 500& 250& 125& 376& 188& 94& 47& 142\\
71& 214& 107& 322& 161& 484& 242& 121& 364& 182\\
91& 274& 137& 412& 206& 103& 310& 155& 466& 233\\
700& 350& 175& 526& 263& 790& 395& 1186& 593& 1780\\
890& 445& 1336& 668& 334& 167& 502& 251& 754& 377\\
1132& 566& 283& 850& 425& 1276& 638& 319& 958& 479\\
1438& 719& 2158& 1079& 3238& 1619& 4858& 2429& 7288& 3644\\
1822& 911& 2734& 1367& 4102& 2051& 6154& 3077& 9232& 4616\\
2308& 1154& 577& 1732& 866& 433& 1300& 650& 325& 976\\
488& 244& 122& 61& 184& 92& 46& 23& 70& 35\\
106& 53& 160& 80& 40& 20& 10& 5& 16& 8\\
4& 2& 1& \\

416&&&&&&&&&\\
208& 104& 52& 26& 13& 40& 20& 10& 5& 16\\
8& 4& 2& 1& \\

417&&&&&&&&&\\
1252& 626& 313& 940& 470& 235& 706& 353& 1060& 530\\
265& 796& 398& 199& 598& 299& 898& 449& 1348& 674\\
337& 1012& 506& 253& 760& 380& 190& 95& 286& 143\\
430& 215& 646& 323& 970& 485& 1456& 728& 364& 182\\
91& 274& 137& 412& 206& 103& 310& 155& 466& 233\\
700& 350& 175& 526& 263& 790& 395& 1186& 593& 1780\\
890& 445& 1336& 668& 334& 167& 502& 251& 754& 377\\
1132& 566& 283& 850& 425& 1276& 638& 319& 958& 479\\
1438& 719& 2158& 1079& 3238& 1619& 4858& 2429& 7288& 3644\\
1822& 911& 2734& 1367& 4102& 2051& 6154& 3077& 9232& 4616\\
2308& 1154& 577& 1732& 866& 433& 1300& 650& 325& 976\\
488& 244& 122& 61& 184& 92& 46& 23& 70& 35\\
106& 53& 160& 80& 40& 20& 10& 5& 16& 8\\
4& 2& 1& \\

418&&&&&&&&&\\
209& 628& 314& 157& 472& 236& 118& 59& 178& 89\\
268& 134& 67& 202& 101& 304& 152& 76& 38& 19\\
58& 29& 88& 44& 22& 11& 34& 17& 52& 26\\
13& 40& 20& 10& 5& 16& 8& 4& 2& 1\\

419&&&&&&&&&\\
1258& 629& 1888& 944& 472& 236& 118& 59& 178& 89\\
268& 134& 67& 202& 101& 304& 152& 76& 38& 19\\
58& 29& 88& 44& 22& 11& 34& 17& 52& 26\\
13& 40& 20& 10& 5& 16& 8& 4& 2& 1\\

420&&&&&&&&&\\
210& 105& 316& 158& 79& 238& 119& 358& 179& 538\\
269& 808& 404& 202& 101& 304& 152& 76& 38& 19\\
58& 29& 88& 44& 22& 11& 34& 17& 52& 26\\
13& 40& 20& 10& 5& 16& 8& 4& 2& 1\\

421&&&&&&&&&\\
1264& 632& 316& 158& 79& 238& 119& 358& 179& 538\\
269& 808& 404& 202& 101& 304& 152& 76& 38& 19\\
58& 29& 88& 44& 22& 11& 34& 17& 52& 26\\
13& 40& 20& 10& 5& 16& 8& 4& 2& 1\\

422&&&&&&&&&\\
211& 634& 317& 952& 476& 238& 119& 358& 179& 538\\
269& 808& 404& 202& 101& 304& 152& 76& 38& 19\\
58& 29& 88& 44& 22& 11& 34& 17& 52& 26\\
13& 40& 20& 10& 5& 16& 8& 4& 2& 1\\

423&&&&&&&&&\\
1270& 635& 1906& 953& 2860& 1430& 715& 2146& 1073& 3220\\
1610& 805& 2416& 1208& 604& 302& 151& 454& 227& 682\\
341& 1024& 512& 256& 128& 64& 32& 16& 8& 4\\
2& 1& \\

424&&&&&&&&&\\
212& 106& 53& 160& 80& 40& 20& 10& 5& 16\\
8& 4& 2& 1& \\

425&&&&&&&&&\\
1276& 638& 319& 958& 479& 1438& 719& 2158& 1079& 3238\\
1619& 4858& 2429& 7288& 3644& 1822& 911& 2734& 1367& 4102\\
2051& 6154& 3077& 9232& 4616& 2308& 1154& 577& 1732& 866\\
433& 1300& 650& 325& 976& 488& 244& 122& 61& 184\\
92& 46& 23& 70& 35& 106& 53& 160& 80& 40\\
20& 10& 5& 16& 8& 4& 2& 1& \\

426&&&&&&&&&\\
213& 640& 320& 160& 80& 40& 20& 10& 5& 16\\
8& 4& 2& 1& \\

427&&&&&&&&&\\
1282& 641& 1924& 962& 481& 1444& 722& 361& 1084& 542\\
271& 814& 407& 1222& 611& 1834& 917& 2752& 1376& 688\\
344& 172& 86& 43& 130& 65& 196& 98& 49& 148\\
74& 37& 112& 56& 28& 14& 7& 22& 11& 34\\
17& 52& 26& 13& 40& 20& 10& 5& 16& 8\\
4& 2& 1& \\

428&&&&&&&&&\\
214& 107& 322& 161& 484& 242& 121& 364& 182& 91\\
274& 137& 412& 206& 103& 310& 155& 466& 233& 700\\
350& 175& 526& 263& 790& 395& 1186& 593& 1780& 890\\
445& 1336& 668& 334& 167& 502& 251& 754& 377& 1132\\
566& 283& 850& 425& 1276& 638& 319& 958& 479& 1438\\
719& 2158& 1079& 3238& 1619& 4858& 2429& 7288& 3644& 1822\\
911& 2734& 1367& 4102& 2051& 6154& 3077& 9232& 4616& 2308\\
1154& 577& 1732& 866& 433& 1300& 650& 325& 976& 488\\
244& 122& 61& 184& 92& 46& 23& 70& 35& 106\\
53& 160& 80& 40& 20& 10& 5& 16& 8& 4\\
2& 1& \\

429&&&&&&&&&\\
1288& 644& 322& 161& 484& 242& 121& 364& 182& 91\\
274& 137& 412& 206& 103& 310& 155& 466& 233& 700\\
350& 175& 526& 263& 790& 395& 1186& 593& 1780& 890\\
445& 1336& 668& 334& 167& 502& 251& 754& 377& 1132\\
566& 283& 850& 425& 1276& 638& 319& 958& 479& 1438\\
719& 2158& 1079& 3238& 1619& 4858& 2429& 7288& 3644& 1822\\
911& 2734& 1367& 4102& 2051& 6154& 3077& 9232& 4616& 2308\\
1154& 577& 1732& 866& 433& 1300& 650& 325& 976& 488\\
244& 122& 61& 184& 92& 46& 23& 70& 35& 106\\
53& 160& 80& 40& 20& 10& 5& 16& 8& 4\\
2& 1& \\

430&&&&&&&&&\\
215& 646& 323& 970& 485& 1456& 728& 364& 182& 91\\
274& 137& 412& 206& 103& 310& 155& 466& 233& 700\\
350& 175& 526& 263& 790& 395& 1186& 593& 1780& 890\\
445& 1336& 668& 334& 167& 502& 251& 754& 377& 1132\\
566& 283& 850& 425& 1276& 638& 319& 958& 479& 1438\\
719& 2158& 1079& 3238& 1619& 4858& 2429& 7288& 3644& 1822\\
911& 2734& 1367& 4102& 2051& 6154& 3077& 9232& 4616& 2308\\
1154& 577& 1732& 866& 433& 1300& 650& 325& 976& 488\\
244& 122& 61& 184& 92& 46& 23& 70& 35& 106\\
53& 160& 80& 40& 20& 10& 5& 16& 8& 4\\
2& 1& \\

431&&&&&&&&&\\
1294& 647& 1942& 971& 2914& 1457& 4372& 2186& 1093& 3280\\
1640& 820& 410& 205& 616& 308& 154& 77& 232& 116\\
58& 29& 88& 44& 22& 11& 34& 17& 52& 26\\
13& 40& 20& 10& 5& 16& 8& 4& 2& 1\\

432&&&&&&&&&\\
216& 108& 54& 27& 82& 41& 124& 62& 31& 94\\
47& 142& 71& 214& 107& 322& 161& 484& 242& 121\\
364& 182& 91& 274& 137& 412& 206& 103& 310& 155\\
466& 233& 700& 350& 175& 526& 263& 790& 395& 1186\\
593& 1780& 890& 445& 1336& 668& 334& 167& 502& 251\\
754& 377& 1132& 566& 283& 850& 425& 1276& 638& 319\\
958& 479& 1438& 719& 2158& 1079& 3238& 1619& 4858& 2429\\
7288& 3644& 1822& 911& 2734& 1367& 4102& 2051& 6154& 3077\\
9232& 4616& 2308& 1154& 577& 1732& 866& 433& 1300& 650\\
325& 976& 488& 244& 122& 61& 184& 92& 46& 23\\
70& 35& 106& 53& 160& 80& 40& 20& 10& 5\\
16& 8& 4& 2& 1& \\

433&&&&&&&&&\\
1300& 650& 325& 976& 488& 244& 122& 61& 184& 92\\
46& 23& 70& 35& 106& 53& 160& 80& 40& 20\\
10& 5& 16& 8& 4& 2& 1& \\

434&&&&&&&&&\\
217& 652& 326& 163& 490& 245& 736& 368& 184& 92\\
46& 23& 70& 35& 106& 53& 160& 80& 40& 20\\
10& 5& 16& 8& 4& 2& 1& \\

435&&&&&&&&&\\
1306& 653& 1960& 980& 490& 245& 736& 368& 184& 92\\
46& 23& 70& 35& 106& 53& 160& 80& 40& 20\\
10& 5& 16& 8& 4& 2& 1& \\

436&&&&&&&&&\\
218& 109& 328& 164& 82& 41& 124& 62& 31& 94\\
47& 142& 71& 214& 107& 322& 161& 484& 242& 121\\
364& 182& 91& 274& 137& 412& 206& 103& 310& 155\\
466& 233& 700& 350& 175& 526& 263& 790& 395& 1186\\
593& 1780& 890& 445& 1336& 668& 334& 167& 502& 251\\
754& 377& 1132& 566& 283& 850& 425& 1276& 638& 319\\
958& 479& 1438& 719& 2158& 1079& 3238& 1619& 4858& 2429\\
7288& 3644& 1822& 911& 2734& 1367& 4102& 2051& 6154& 3077\\
9232& 4616& 2308& 1154& 577& 1732& 866& 433& 1300& 650\\
325& 976& 488& 244& 122& 61& 184& 92& 46& 23\\
70& 35& 106& 53& 160& 80& 40& 20& 10& 5\\
16& 8& 4& 2& 1& \\

437&&&&&&&&&\\
1312& 656& 328& 164& 82& 41& 124& 62& 31& 94\\
47& 142& 71& 214& 107& 322& 161& 484& 242& 121\\
364& 182& 91& 274& 137& 412& 206& 103& 310& 155\\
466& 233& 700& 350& 175& 526& 263& 790& 395& 1186\\
593& 1780& 890& 445& 1336& 668& 334& 167& 502& 251\\
754& 377& 1132& 566& 283& 850& 425& 1276& 638& 319\\
958& 479& 1438& 719& 2158& 1079& 3238& 1619& 4858& 2429\\
7288& 3644& 1822& 911& 2734& 1367& 4102& 2051& 6154& 3077\\
9232& 4616& 2308& 1154& 577& 1732& 866& 433& 1300& 650\\
325& 976& 488& 244& 122& 61& 184& 92& 46& 23\\
70& 35& 106& 53& 160& 80& 40& 20& 10& 5\\
16& 8& 4& 2& 1& \\

438&&&&&&&&&\\
219& 658& 329& 988& 494& 247& 742& 371& 1114& 557\\
1672& 836& 418& 209& 628& 314& 157& 472& 236& 118\\
59& 178& 89& 268& 134& 67& 202& 101& 304& 152\\
76& 38& 19& 58& 29& 88& 44& 22& 11& 34\\
17& 52& 26& 13& 40& 20& 10& 5& 16& 8\\
4& 2& 1& \\

439&&&&&&&&&\\
1318& 659& 1978& 989& 2968& 1484& 742& 371& 1114& 557\\
1672& 836& 418& 209& 628& 314& 157& 472& 236& 118\\
59& 178& 89& 268& 134& 67& 202& 101& 304& 152\\
76& 38& 19& 58& 29& 88& 44& 22& 11& 34\\
17& 52& 26& 13& 40& 20& 10& 5& 16& 8\\
4& 2& 1& \\

440&&&&&&&&&\\
220& 110& 55& 166& 83& 250& 125& 376& 188& 94\\
47& 142& 71& 214& 107& 322& 161& 484& 242& 121\\
364& 182& 91& 274& 137& 412& 206& 103& 310& 155\\
466& 233& 700& 350& 175& 526& 263& 790& 395& 1186\\
593& 1780& 890& 445& 1336& 668& 334& 167& 502& 251\\
754& 377& 1132& 566& 283& 850& 425& 1276& 638& 319\\
958& 479& 1438& 719& 2158& 1079& 3238& 1619& 4858& 2429\\
7288& 3644& 1822& 911& 2734& 1367& 4102& 2051& 6154& 3077\\
9232& 4616& 2308& 1154& 577& 1732& 866& 433& 1300& 650\\
325& 976& 488& 244& 122& 61& 184& 92& 46& 23\\
70& 35& 106& 53& 160& 80& 40& 20& 10& 5\\
16& 8& 4& 2& 1& \\

441&&&&&&&&&\\
1324& 662& 331& 994& 497& 1492& 746& 373& 1120& 560\\
280& 140& 70& 35& 106& 53& 160& 80& 40& 20\\
10& 5& 16& 8& 4& 2& 1& \\

442&&&&&&&&&\\
221& 664& 332& 166& 83& 250& 125& 376& 188& 94\\
47& 142& 71& 214& 107& 322& 161& 484& 242& 121\\
364& 182& 91& 274& 137& 412& 206& 103& 310& 155\\
466& 233& 700& 350& 175& 526& 263& 790& 395& 1186\\
593& 1780& 890& 445& 1336& 668& 334& 167& 502& 251\\
754& 377& 1132& 566& 283& 850& 425& 1276& 638& 319\\
958& 479& 1438& 719& 2158& 1079& 3238& 1619& 4858& 2429\\
7288& 3644& 1822& 911& 2734& 1367& 4102& 2051& 6154& 3077\\
9232& 4616& 2308& 1154& 577& 1732& 866& 433& 1300& 650\\
325& 976& 488& 244& 122& 61& 184& 92& 46& 23\\
70& 35& 106& 53& 160& 80& 40& 20& 10& 5\\
16& 8& 4& 2& 1& \\

443&&&&&&&&&\\
1330& 665& 1996& 998& 499& 1498& 749& 2248& 1124& 562\\
281& 844& 422& 211& 634& 317& 952& 476& 238& 119\\
358& 179& 538& 269& 808& 404& 202& 101& 304& 152\\
76& 38& 19& 58& 29& 88& 44& 22& 11& 34\\
17& 52& 26& 13& 40& 20& 10& 5& 16& 8\\
4& 2& 1& \\

444&&&&&&&&&\\
222& 111& 334& 167& 502& 251& 754& 377& 1132& 566\\
283& 850& 425& 1276& 638& 319& 958& 479& 1438& 719\\
2158& 1079& 3238& 1619& 4858& 2429& 7288& 3644& 1822& 911\\
2734& 1367& 4102& 2051& 6154& 3077& 9232& 4616& 2308& 1154\\
577& 1732& 866& 433& 1300& 650& 325& 976& 488& 244\\
122& 61& 184& 92& 46& 23& 70& 35& 106& 53\\
160& 80& 40& 20& 10& 5& 16& 8& 4& 2\\
1& \\

445&&&&&&&&&\\
1336& 668& 334& 167& 502& 251& 754& 377& 1132& 566\\
283& 850& 425& 1276& 638& 319& 958& 479& 1438& 719\\
2158& 1079& 3238& 1619& 4858& 2429& 7288& 3644& 1822& 911\\
2734& 1367& 4102& 2051& 6154& 3077& 9232& 4616& 2308& 1154\\
577& 1732& 866& 433& 1300& 650& 325& 976& 488& 244\\
122& 61& 184& 92& 46& 23& 70& 35& 106& 53\\
160& 80& 40& 20& 10& 5& 16& 8& 4& 2\\
1& \\

446&&&&&&&&&\\
223& 670& 335& 1006& 503& 1510& 755& 2266& 1133& 3400\\
1700& 850& 425& 1276& 638& 319& 958& 479& 1438& 719\\
2158& 1079& 3238& 1619& 4858& 2429& 7288& 3644& 1822& 911\\
2734& 1367& 4102& 2051& 6154& 3077& 9232& 4616& 2308& 1154\\
577& 1732& 866& 433& 1300& 650& 325& 976& 488& 244\\
122& 61& 184& 92& 46& 23& 70& 35& 106& 53\\
160& 80& 40& 20& 10& 5& 16& 8& 4& 2\\
1& \\

447&&&&&&&&&\\
1342& 671& 2014& 1007& 3022& 1511& 4534& 2267& 6802& 3401\\
10204& 5102& 2551& 7654& 3827& 11482& 5741& 17224& 8612& 4306\\
2153& 6460& 3230& 1615& 4846& 2423& 7270& 3635& 10906& 5453\\
16360& 8180& 4090& 2045& 6136& 3068& 1534& 767& 2302& 1151\\
3454& 1727& 5182& 2591& 7774& 3887& 11662& 5831& 17494& 8747\\
26242& 13121& 39364& 19682& 9841& 29524& 14762& 7381& 22144& 11072\\
5536& 2768& 1384& 692& 346& 173& 520& 260& 130& 65\\
196& 98& 49& 148& 74& 37& 112& 56& 28& 14\\
7& 22& 11& 34& 17& 52& 26& 13& 40& 20\\
10& 5& 16& 8& 4& 2& 1& \\

448&&&&&&&&&\\
224& 112& 56& 28& 14& 7& 22& 11& 34& 17\\
52& 26& 13& 40& 20& 10& 5& 16& 8& 4\\
2& 1& \\

449&&&&&&&&&\\
1348& 674& 337& 1012& 506& 253& 760& 380& 190& 95\\
286& 143& 430& 215& 646& 323& 970& 485& 1456& 728\\
364& 182& 91& 274& 137& 412& 206& 103& 310& 155\\
466& 233& 700& 350& 175& 526& 263& 790& 395& 1186\\
593& 1780& 890& 445& 1336& 668& 334& 167& 502& 251\\
754& 377& 1132& 566& 283& 850& 425& 1276& 638& 319\\
958& 479& 1438& 719& 2158& 1079& 3238& 1619& 4858& 2429\\
7288& 3644& 1822& 911& 2734& 1367& 4102& 2051& 6154& 3077\\
9232& 4616& 2308& 1154& 577& 1732& 866& 433& 1300& 650\\
325& 976& 488& 244& 122& 61& 184& 92& 46& 23\\
70& 35& 106& 53& 160& 80& 40& 20& 10& 5\\
16& 8& 4& 2& 1& \\

450&&&&&&&&&\\
225& 676& 338& 169& 508& 254& 127& 382& 191& 574\\
287& 862& 431& 1294& 647& 1942& 971& 2914& 1457& 4372\\
2186& 1093& 3280& 1640& 820& 410& 205& 616& 308& 154\\
77& 232& 116& 58& 29& 88& 44& 22& 11& 34\\
17& 52& 26& 13& 40& 20& 10& 5& 16& 8\\
4& 2& 1& \\

451&&&&&&&&&\\
1354& 677& 2032& 1016& 508& 254& 127& 382& 191& 574\\
287& 862& 431& 1294& 647& 1942& 971& 2914& 1457& 4372\\
2186& 1093& 3280& 1640& 820& 410& 205& 616& 308& 154\\
77& 232& 116& 58& 29& 88& 44& 22& 11& 34\\
17& 52& 26& 13& 40& 20& 10& 5& 16& 8\\
4& 2& 1& \\

452&&&&&&&&&\\
226& 113& 340& 170& 85& 256& 128& 64& 32& 16\\
8& 4& 2& 1& \\

453&&&&&&&&&\\
1360& 680& 340& 170& 85& 256& 128& 64& 32& 16\\
8& 4& 2& 1& \\

454&&&&&&&&&\\
227& 682& 341& 1024& 512& 256& 128& 64& 32& 16\\
8& 4& 2& 1& \\

455&&&&&&&&&\\
1366& 683& 2050& 1025& 3076& 1538& 769& 2308& 1154& 577\\
1732& 866& 433& 1300& 650& 325& 976& 488& 244& 122\\
61& 184& 92& 46& 23& 70& 35& 106& 53& 160\\
80& 40& 20& 10& 5& 16& 8& 4& 2& 1\\

456&&&&&&&&&\\
228& 114& 57& 172& 86& 43& 130& 65& 196& 98\\
49& 148& 74& 37& 112& 56& 28& 14& 7& 22\\
11& 34& 17& 52& 26& 13& 40& 20& 10& 5\\
16& 8& 4& 2& 1& \\

457&&&&&&&&&\\
1372& 686& 343& 1030& 515& 1546& 773& 2320& 1160& 580\\
290& 145& 436& 218& 109& 328& 164& 82& 41& 124\\
62& 31& 94& 47& 142& 71& 214& 107& 322& 161\\
484& 242& 121& 364& 182& 91& 274& 137& 412& 206\\
103& 310& 155& 466& 233& 700& 350& 175& 526& 263\\
790& 395& 1186& 593& 1780& 890& 445& 1336& 668& 334\\
167& 502& 251& 754& 377& 1132& 566& 283& 850& 425\\
1276& 638& 319& 958& 479& 1438& 719& 2158& 1079& 3238\\
1619& 4858& 2429& 7288& 3644& 1822& 911& 2734& 1367& 4102\\
2051& 6154& 3077& 9232& 4616& 2308& 1154& 577& 1732& 866\\
433& 1300& 650& 325& 976& 488& 244& 122& 61& 184\\
92& 46& 23& 70& 35& 106& 53& 160& 80& 40\\
20& 10& 5& 16& 8& 4& 2& 1& \\

458&&&&&&&&&\\
229& 688& 344& 172& 86& 43& 130& 65& 196& 98\\
49& 148& 74& 37& 112& 56& 28& 14& 7& 22\\
11& 34& 17& 52& 26& 13& 40& 20& 10& 5\\
16& 8& 4& 2& 1& \\

459&&&&&&&&&\\
1378& 689& 2068& 1034& 517& 1552& 776& 388& 194& 97\\
292& 146& 73& 220& 110& 55& 166& 83& 250& 125\\
376& 188& 94& 47& 142& 71& 214& 107& 322& 161\\
484& 242& 121& 364& 182& 91& 274& 137& 412& 206\\
103& 310& 155& 466& 233& 700& 350& 175& 526& 263\\
790& 395& 1186& 593& 1780& 890& 445& 1336& 668& 334\\
167& 502& 251& 754& 377& 1132& 566& 283& 850& 425\\
1276& 638& 319& 958& 479& 1438& 719& 2158& 1079& 3238\\
1619& 4858& 2429& 7288& 3644& 1822& 911& 2734& 1367& 4102\\
2051& 6154& 3077& 9232& 4616& 2308& 1154& 577& 1732& 866\\
433& 1300& 650& 325& 976& 488& 244& 122& 61& 184\\
92& 46& 23& 70& 35& 106& 53& 160& 80& 40\\
20& 10& 5& 16& 8& 4& 2& 1& \\

460&&&&&&&&&\\
230& 115& 346& 173& 520& 260& 130& 65& 196& 98\\
49& 148& 74& 37& 112& 56& 28& 14& 7& 22\\
11& 34& 17& 52& 26& 13& 40& 20& 10& 5\\
16& 8& 4& 2& 1& \\

461&&&&&&&&&\\
1384& 692& 346& 173& 520& 260& 130& 65& 196& 98\\
49& 148& 74& 37& 112& 56& 28& 14& 7& 22\\
11& 34& 17& 52& 26& 13& 40& 20& 10& 5\\
16& 8& 4& 2& 1& \\

462&&&&&&&&&\\
231& 694& 347& 1042& 521& 1564& 782& 391& 1174& 587\\
1762& 881& 2644& 1322& 661& 1984& 992& 496& 248& 124\\
62& 31& 94& 47& 142& 71& 214& 107& 322& 161\\
484& 242& 121& 364& 182& 91& 274& 137& 412& 206\\
103& 310& 155& 466& 233& 700& 350& 175& 526& 263\\
790& 395& 1186& 593& 1780& 890& 445& 1336& 668& 334\\
167& 502& 251& 754& 377& 1132& 566& 283& 850& 425\\
1276& 638& 319& 958& 479& 1438& 719& 2158& 1079& 3238\\
1619& 4858& 2429& 7288& 3644& 1822& 911& 2734& 1367& 4102\\
2051& 6154& 3077& 9232& 4616& 2308& 1154& 577& 1732& 866\\
433& 1300& 650& 325& 976& 488& 244& 122& 61& 184\\
92& 46& 23& 70& 35& 106& 53& 160& 80& 40\\
20& 10& 5& 16& 8& 4& 2& 1& \\

463&&&&&&&&&\\
1390& 695& 2086& 1043& 3130& 1565& 4696& 2348& 1174& 587\\
1762& 881& 2644& 1322& 661& 1984& 992& 496& 248& 124\\
62& 31& 94& 47& 142& 71& 214& 107& 322& 161\\
484& 242& 121& 364& 182& 91& 274& 137& 412& 206\\
103& 310& 155& 466& 233& 700& 350& 175& 526& 263\\
790& 395& 1186& 593& 1780& 890& 445& 1336& 668& 334\\
167& 502& 251& 754& 377& 1132& 566& 283& 850& 425\\
1276& 638& 319& 958& 479& 1438& 719& 2158& 1079& 3238\\
1619& 4858& 2429& 7288& 3644& 1822& 911& 2734& 1367& 4102\\
2051& 6154& 3077& 9232& 4616& 2308& 1154& 577& 1732& 866\\
433& 1300& 650& 325& 976& 488& 244& 122& 61& 184\\
92& 46& 23& 70& 35& 106& 53& 160& 80& 40\\
20& 10& 5& 16& 8& 4& 2& 1& \\

464&&&&&&&&&\\
232& 116& 58& 29& 88& 44& 22& 11& 34& 17\\
52& 26& 13& 40& 20& 10& 5& 16& 8& 4\\
2& 1& \\

465&&&&&&&&&\\
1396& 698& 349& 1048& 524& 262& 131& 394& 197& 592\\
296& 148& 74& 37& 112& 56& 28& 14& 7& 22\\
11& 34& 17& 52& 26& 13& 40& 20& 10& 5\\
16& 8& 4& 2& 1& \\

466&&&&&&&&&\\
233& 700& 350& 175& 526& 263& 790& 395& 1186& 593\\
1780& 890& 445& 1336& 668& 334& 167& 502& 251& 754\\
377& 1132& 566& 283& 850& 425& 1276& 638& 319& 958\\
479& 1438& 719& 2158& 1079& 3238& 1619& 4858& 2429& 7288\\
3644& 1822& 911& 2734& 1367& 4102& 2051& 6154& 3077& 9232\\
4616& 2308& 1154& 577& 1732& 866& 433& 1300& 650& 325\\
976& 488& 244& 122& 61& 184& 92& 46& 23& 70\\
35& 106& 53& 160& 80& 40& 20& 10& 5& 16\\
8& 4& 2& 1& \\

467&&&&&&&&&\\
1402& 701& 2104& 1052& 526& 263& 790& 395& 1186& 593\\
1780& 890& 445& 1336& 668& 334& 167& 502& 251& 754\\
377& 1132& 566& 283& 850& 425& 1276& 638& 319& 958\\
479& 1438& 719& 2158& 1079& 3238& 1619& 4858& 2429& 7288\\
3644& 1822& 911& 2734& 1367& 4102& 2051& 6154& 3077& 9232\\
4616& 2308& 1154& 577& 1732& 866& 433& 1300& 650& 325\\
976& 488& 244& 122& 61& 184& 92& 46& 23& 70\\
35& 106& 53& 160& 80& 40& 20& 10& 5& 16\\
8& 4& 2& 1& \\

468&&&&&&&&&\\
234& 117& 352& 176& 88& 44& 22& 11& 34& 17\\
52& 26& 13& 40& 20& 10& 5& 16& 8& 4\\
2& 1& \\

469&&&&&&&&&\\
1408& 704& 352& 176& 88& 44& 22& 11& 34& 17\\
52& 26& 13& 40& 20& 10& 5& 16& 8& 4\\
2& 1& \\

470&&&&&&&&&\\
235& 706& 353& 1060& 530& 265& 796& 398& 199& 598\\
299& 898& 449& 1348& 674& 337& 1012& 506& 253& 760\\
380& 190& 95& 286& 143& 430& 215& 646& 323& 970\\
485& 1456& 728& 364& 182& 91& 274& 137& 412& 206\\
103& 310& 155& 466& 233& 700& 350& 175& 526& 263\\
790& 395& 1186& 593& 1780& 890& 445& 1336& 668& 334\\
167& 502& 251& 754& 377& 1132& 566& 283& 850& 425\\
1276& 638& 319& 958& 479& 1438& 719& 2158& 1079& 3238\\
1619& 4858& 2429& 7288& 3644& 1822& 911& 2734& 1367& 4102\\
2051& 6154& 3077& 9232& 4616& 2308& 1154& 577& 1732& 866\\
433& 1300& 650& 325& 976& 488& 244& 122& 61& 184\\
92& 46& 23& 70& 35& 106& 53& 160& 80& 40\\
20& 10& 5& 16& 8& 4& 2& 1& \\

471&&&&&&&&&\\
1414& 707& 2122& 1061& 3184& 1592& 796& 398& 199& 598\\
299& 898& 449& 1348& 674& 337& 1012& 506& 253& 760\\
380& 190& 95& 286& 143& 430& 215& 646& 323& 970\\
485& 1456& 728& 364& 182& 91& 274& 137& 412& 206\\
103& 310& 155& 466& 233& 700& 350& 175& 526& 263\\
790& 395& 1186& 593& 1780& 890& 445& 1336& 668& 334\\
167& 502& 251& 754& 377& 1132& 566& 283& 850& 425\\
1276& 638& 319& 958& 479& 1438& 719& 2158& 1079& 3238\\
1619& 4858& 2429& 7288& 3644& 1822& 911& 2734& 1367& 4102\\
2051& 6154& 3077& 9232& 4616& 2308& 1154& 577& 1732& 866\\
433& 1300& 650& 325& 976& 488& 244& 122& 61& 184\\
92& 46& 23& 70& 35& 106& 53& 160& 80& 40\\
20& 10& 5& 16& 8& 4& 2& 1& \\

472&&&&&&&&&\\
236& 118& 59& 178& 89& 268& 134& 67& 202& 101\\
304& 152& 76& 38& 19& 58& 29& 88& 44& 22\\
11& 34& 17& 52& 26& 13& 40& 20& 10& 5\\
16& 8& 4& 2& 1& \\

473&&&&&&&&&\\
1420& 710& 355& 1066& 533& 1600& 800& 400& 200& 100\\
50& 25& 76& 38& 19& 58& 29& 88& 44& 22\\
11& 34& 17& 52& 26& 13& 40& 20& 10& 5\\
16& 8& 4& 2& 1& \\

474&&&&&&&&&\\
237& 712& 356& 178& 89& 268& 134& 67& 202& 101\\
304& 152& 76& 38& 19& 58& 29& 88& 44& 22\\
11& 34& 17& 52& 26& 13& 40& 20& 10& 5\\
16& 8& 4& 2& 1& \\

475&&&&&&&&&\\
1426& 713& 2140& 1070& 535& 1606& 803& 2410& 1205& 3616\\
1808& 904& 452& 226& 113& 340& 170& 85& 256& 128\\
64& 32& 16& 8& 4& 2& 1& \\

476&&&&&&&&&\\
238& 119& 358& 179& 538& 269& 808& 404& 202& 101\\
304& 152& 76& 38& 19& 58& 29& 88& 44& 22\\
11& 34& 17& 52& 26& 13& 40& 20& 10& 5\\
16& 8& 4& 2& 1& \\

477&&&&&&&&&\\
1432& 716& 358& 179& 538& 269& 808& 404& 202& 101\\
304& 152& 76& 38& 19& 58& 29& 88& 44& 22\\
11& 34& 17& 52& 26& 13& 40& 20& 10& 5\\
16& 8& 4& 2& 1& \\

478&&&&&&&&&\\
239& 718& 359& 1078& 539& 1618& 809& 2428& 1214& 607\\
1822& 911& 2734& 1367& 4102& 2051& 6154& 3077& 9232& 4616\\
2308& 1154& 577& 1732& 866& 433& 1300& 650& 325& 976\\
488& 244& 122& 61& 184& 92& 46& 23& 70& 35\\
106& 53& 160& 80& 40& 20& 10& 5& 16& 8\\
4& 2& 1& \\

479&&&&&&&&&\\
1438& 719& 2158& 1079& 3238& 1619& 4858& 2429& 7288& 3644\\
1822& 911& 2734& 1367& 4102& 2051& 6154& 3077& 9232& 4616\\
2308& 1154& 577& 1732& 866& 433& 1300& 650& 325& 976\\
488& 244& 122& 61& 184& 92& 46& 23& 70& 35\\
106& 53& 160& 80& 40& 20& 10& 5& 16& 8\\
4& 2& 1& \\

480&&&&&&&&&\\
240& 120& 60& 30& 15& 46& 23& 70& 35& 106\\
53& 160& 80& 40& 20& 10& 5& 16& 8& 4\\
2& 1& \\

481&&&&&&&&&\\
1444& 722& 361& 1084& 542& 271& 814& 407& 1222& 611\\
1834& 917& 2752& 1376& 688& 344& 172& 86& 43& 130\\
65& 196& 98& 49& 148& 74& 37& 112& 56& 28\\
14& 7& 22& 11& 34& 17& 52& 26& 13& 40\\
20& 10& 5& 16& 8& 4& 2& 1& \\

482&&&&&&&&&\\
241& 724& 362& 181& 544& 272& 136& 68& 34& 17\\
52& 26& 13& 40& 20& 10& 5& 16& 8& 4\\
2& 1& \\

483&&&&&&&&&\\
1450& 725& 2176& 1088& 544& 272& 136& 68& 34& 17\\
52& 26& 13& 40& 20& 10& 5& 16& 8& 4\\
2& 1& \\

484&&&&&&&&&\\
242& 121& 364& 182& 91& 274& 137& 412& 206& 103\\
310& 155& 466& 233& 700& 350& 175& 526& 263& 790\\
395& 1186& 593& 1780& 890& 445& 1336& 668& 334& 167\\
502& 251& 754& 377& 1132& 566& 283& 850& 425& 1276\\
638& 319& 958& 479& 1438& 719& 2158& 1079& 3238& 1619\\
4858& 2429& 7288& 3644& 1822& 911& 2734& 1367& 4102& 2051\\
6154& 3077& 9232& 4616& 2308& 1154& 577& 1732& 866& 433\\
1300& 650& 325& 976& 488& 244& 122& 61& 184& 92\\
46& 23& 70& 35& 106& 53& 160& 80& 40& 20\\
10& 5& 16& 8& 4& 2& 1& \\

485&&&&&&&&&\\
1456& 728& 364& 182& 91& 274& 137& 412& 206& 103\\
310& 155& 466& 233& 700& 350& 175& 526& 263& 790\\
395& 1186& 593& 1780& 890& 445& 1336& 668& 334& 167\\
502& 251& 754& 377& 1132& 566& 283& 850& 425& 1276\\
638& 319& 958& 479& 1438& 719& 2158& 1079& 3238& 1619\\
4858& 2429& 7288& 3644& 1822& 911& 2734& 1367& 4102& 2051\\
6154& 3077& 9232& 4616& 2308& 1154& 577& 1732& 866& 433\\
1300& 650& 325& 976& 488& 244& 122& 61& 184& 92\\
46& 23& 70& 35& 106& 53& 160& 80& 40& 20\\
10& 5& 16& 8& 4& 2& 1& \\

486&&&&&&&&&\\
243& 730& 365& 1096& 548& 274& 137& 412& 206& 103\\
310& 155& 466& 233& 700& 350& 175& 526& 263& 790\\
395& 1186& 593& 1780& 890& 445& 1336& 668& 334& 167\\
502& 251& 754& 377& 1132& 566& 283& 850& 425& 1276\\
638& 319& 958& 479& 1438& 719& 2158& 1079& 3238& 1619\\
4858& 2429& 7288& 3644& 1822& 911& 2734& 1367& 4102& 2051\\
6154& 3077& 9232& 4616& 2308& 1154& 577& 1732& 866& 433\\
1300& 650& 325& 976& 488& 244& 122& 61& 184& 92\\
46& 23& 70& 35& 106& 53& 160& 80& 40& 20\\
10& 5& 16& 8& 4& 2& 1& \\

487&&&&&&&&&\\
1462& 731& 2194& 1097& 3292& 1646& 823& 2470& 1235& 3706\\
1853& 5560& 2780& 1390& 695& 2086& 1043& 3130& 1565& 4696\\
2348& 1174& 587& 1762& 881& 2644& 1322& 661& 1984& 992\\
496& 248& 124& 62& 31& 94& 47& 142& 71& 214\\
107& 322& 161& 484& 242& 121& 364& 182& 91& 274\\
137& 412& 206& 103& 310& 155& 466& 233& 700& 350\\
175& 526& 263& 790& 395& 1186& 593& 1780& 890& 445\\
1336& 668& 334& 167& 502& 251& 754& 377& 1132& 566\\
283& 850& 425& 1276& 638& 319& 958& 479& 1438& 719\\
2158& 1079& 3238& 1619& 4858& 2429& 7288& 3644& 1822& 911\\
2734& 1367& 4102& 2051& 6154& 3077& 9232& 4616& 2308& 1154\\
577& 1732& 866& 433& 1300& 650& 325& 976& 488& 244\\
122& 61& 184& 92& 46& 23& 70& 35& 106& 53\\
160& 80& 40& 20& 10& 5& 16& 8& 4& 2\\
1& \\

488&&&&&&&&&\\
244& 122& 61& 184& 92& 46& 23& 70& 35& 106\\
53& 160& 80& 40& 20& 10& 5& 16& 8& 4\\
2& 1& \\

489&&&&&&&&&\\
1468& 734& 367& 1102& 551& 1654& 827& 2482& 1241& 3724\\
1862& 931& 2794& 1397& 4192& 2096& 1048& 524& 262& 131\\
394& 197& 592& 296& 148& 74& 37& 112& 56& 28\\
14& 7& 22& 11& 34& 17& 52& 26& 13& 40\\
20& 10& 5& 16& 8& 4& 2& 1& \\

490&&&&&&&&&\\
245& 736& 368& 184& 92& 46& 23& 70& 35& 106\\
53& 160& 80& 40& 20& 10& 5& 16& 8& 4\\
2& 1& \\

491&&&&&&&&&\\
1474& 737& 2212& 1106& 553& 1660& 830& 415& 1246& 623\\
1870& 935& 2806& 1403& 4210& 2105& 6316& 3158& 1579& 4738\\
2369& 7108& 3554& 1777& 5332& 2666& 1333& 4000& 2000& 1000\\
500& 250& 125& 376& 188& 94& 47& 142& 71& 214\\
107& 322& 161& 484& 242& 121& 364& 182& 91& 274\\
137& 412& 206& 103& 310& 155& 466& 233& 700& 350\\
175& 526& 263& 790& 395& 1186& 593& 1780& 890& 445\\
1336& 668& 334& 167& 502& 251& 754& 377& 1132& 566\\
283& 850& 425& 1276& 638& 319& 958& 479& 1438& 719\\
2158& 1079& 3238& 1619& 4858& 2429& 7288& 3644& 1822& 911\\
2734& 1367& 4102& 2051& 6154& 3077& 9232& 4616& 2308& 1154\\
577& 1732& 866& 433& 1300& 650& 325& 976& 488& 244\\
122& 61& 184& 92& 46& 23& 70& 35& 106& 53\\
160& 80& 40& 20& 10& 5& 16& 8& 4& 2\\
1& \\

492&&&&&&&&&\\
246& 123& 370& 185& 556& 278& 139& 418& 209& 628\\
314& 157& 472& 236& 118& 59& 178& 89& 268& 134\\
67& 202& 101& 304& 152& 76& 38& 19& 58& 29\\
88& 44& 22& 11& 34& 17& 52& 26& 13& 40\\
20& 10& 5& 16& 8& 4& 2& 1& \\

493&&&&&&&&&\\
1480& 740& 370& 185& 556& 278& 139& 418& 209& 628\\
314& 157& 472& 236& 118& 59& 178& 89& 268& 134\\
67& 202& 101& 304& 152& 76& 38& 19& 58& 29\\
88& 44& 22& 11& 34& 17& 52& 26& 13& 40\\
20& 10& 5& 16& 8& 4& 2& 1& \\

494&&&&&&&&&\\
247& 742& 371& 1114& 557& 1672& 836& 418& 209& 628\\
314& 157& 472& 236& 118& 59& 178& 89& 268& 134\\
67& 202& 101& 304& 152& 76& 38& 19& 58& 29\\
88& 44& 22& 11& 34& 17& 52& 26& 13& 40\\
20& 10& 5& 16& 8& 4& 2& 1& \\

495&&&&&&&&&\\
1486& 743& 2230& 1115& 3346& 1673& 5020& 2510& 1255& 3766\\
1883& 5650& 2825& 8476& 4238& 2119& 6358& 3179& 9538& 4769\\
14308& 7154& 3577& 10732& 5366& 2683& 8050& 4025& 12076& 6038\\
3019& 9058& 4529& 13588& 6794& 3397& 10192& 5096& 2548& 1274\\
637& 1912& 956& 478& 239& 718& 359& 1078& 539& 1618\\
809& 2428& 1214& 607& 1822& 911& 2734& 1367& 4102& 2051\\
6154& 3077& 9232& 4616& 2308& 1154& 577& 1732& 866& 433\\
1300& 650& 325& 976& 488& 244& 122& 61& 184& 92\\
46& 23& 70& 35& 106& 53& 160& 80& 40& 20\\
10& 5& 16& 8& 4& 2& 1& \\

496&&&&&&&&&\\
248& 124& 62& 31& 94& 47& 142& 71& 214& 107\\
322& 161& 484& 242& 121& 364& 182& 91& 274& 137\\
412& 206& 103& 310& 155& 466& 233& 700& 350& 175\\
526& 263& 790& 395& 1186& 593& 1780& 890& 445& 1336\\
668& 334& 167& 502& 251& 754& 377& 1132& 566& 283\\
850& 425& 1276& 638& 319& 958& 479& 1438& 719& 2158\\
1079& 3238& 1619& 4858& 2429& 7288& 3644& 1822& 911& 2734\\
1367& 4102& 2051& 6154& 3077& 9232& 4616& 2308& 1154& 577\\
1732& 866& 433& 1300& 650& 325& 976& 488& 244& 122\\
61& 184& 92& 46& 23& 70& 35& 106& 53& 160\\
80& 40& 20& 10& 5& 16& 8& 4& 2& 1\\

497&&&&&&&&&\\
1492& 746& 373& 1120& 560& 280& 140& 70& 35& 106\\
53& 160& 80& 40& 20& 10& 5& 16& 8& 4\\
2& 1& \\

498&&&&&&&&&\\
249& 748& 374& 187& 562& 281& 844& 422& 211& 634\\
317& 952& 476& 238& 119& 358& 179& 538& 269& 808\\
404& 202& 101& 304& 152& 76& 38& 19& 58& 29\\
88& 44& 22& 11& 34& 17& 52& 26& 13& 40\\
20& 10& 5& 16& 8& 4& 2& 1& \\

499&&&&&&&&&\\
1498& 749& 2248& 1124& 562& 281& 844& 422& 211& 634\\
317& 952& 476& 238& 119& 358& 179& 538& 269& 808\\
404& 202& 101& 304& 152& 76& 38& 19& 58& 29\\
88& 44& 22& 11& 34& 17& 52& 26& 13& 40\\
20& 10& 5& 16& 8& 4& 2& 1& \\

500&&&&&&&&&\\
250& 125& 376& 188& 94& 47& 142& 71& 214& 107\\
322& 161& 484& 242& 121& 364& 182& 91& 274& 137\\
412& 206& 103& 310& 155& 466& 233& 700& 350& 175\\
526& 263& 790& 395& 1186& 593& 1780& 890& 445& 1336\\
668& 334& 167& 502& 251& 754& 377& 1132& 566& 283\\
850& 425& 1276& 638& 319& 958& 479& 1438& 719& 2158\\
1079& 3238& 1619& 4858& 2429& 7288& 3644& 1822& 911& 2734\\
1367& 4102& 2051& 6154& 3077& 9232& 4616& 2308& 1154& 577\\
1732& 866& 433& 1300& 650& 325& 976& 488& 244& 122\\
61& 184& 92& 46& 23& 70& 35& 106& 53& 160\\
80& 40& 20& 10& 5& 16& 8& 4& 2& 1\\

501&&&&&&&&&\\
1504& 752& 376& 188& 94& 47& 142& 71& 214& 107\\
322& 161& 484& 242& 121& 364& 182& 91& 274& 137\\
412& 206& 103& 310& 155& 466& 233& 700& 350& 175\\
526& 263& 790& 395& 1186& 593& 1780& 890& 445& 1336\\
668& 334& 167& 502& 251& 754& 377& 1132& 566& 283\\
850& 425& 1276& 638& 319& 958& 479& 1438& 719& 2158\\
1079& 3238& 1619& 4858& 2429& 7288& 3644& 1822& 911& 2734\\
1367& 4102& 2051& 6154& 3077& 9232& 4616& 2308& 1154& 577\\
1732& 866& 433& 1300& 650& 325& 976& 488& 244& 122\\
61& 184& 92& 46& 23& 70& 35& 106& 53& 160\\
80& 40& 20& 10& 5& 16& 8& 4& 2& 1\\

502&&&&&&&&&\\
251& 754& 377& 1132& 566& 283& 850& 425& 1276& 638\\
319& 958& 479& 1438& 719& 2158& 1079& 3238& 1619& 4858\\
2429& 7288& 3644& 1822& 911& 2734& 1367& 4102& 2051& 6154\\
3077& 9232& 4616& 2308& 1154& 577& 1732& 866& 433& 1300\\
650& 325& 976& 488& 244& 122& 61& 184& 92& 46\\
23& 70& 35& 106& 53& 160& 80& 40& 20& 10\\
5& 16& 8& 4& 2& 1& \\

503&&&&&&&&&\\
1510& 755& 2266& 1133& 3400& 1700& 850& 425& 1276& 638\\
319& 958& 479& 1438& 719& 2158& 1079& 3238& 1619& 4858\\
2429& 7288& 3644& 1822& 911& 2734& 1367& 4102& 2051& 6154\\
3077& 9232& 4616& 2308& 1154& 577& 1732& 866& 433& 1300\\
650& 325& 976& 488& 244& 122& 61& 184& 92& 46\\
23& 70& 35& 106& 53& 160& 80& 40& 20& 10\\
5& 16& 8& 4& 2& 1& \\

504&&&&&&&&&\\
252& 126& 63& 190& 95& 286& 143& 430& 215& 646\\
323& 970& 485& 1456& 728& 364& 182& 91& 274& 137\\
412& 206& 103& 310& 155& 466& 233& 700& 350& 175\\
526& 263& 790& 395& 1186& 593& 1780& 890& 445& 1336\\
668& 334& 167& 502& 251& 754& 377& 1132& 566& 283\\
850& 425& 1276& 638& 319& 958& 479& 1438& 719& 2158\\
1079& 3238& 1619& 4858& 2429& 7288& 3644& 1822& 911& 2734\\
1367& 4102& 2051& 6154& 3077& 9232& 4616& 2308& 1154& 577\\
1732& 866& 433& 1300& 650& 325& 976& 488& 244& 122\\
61& 184& 92& 46& 23& 70& 35& 106& 53& 160\\
80& 40& 20& 10& 5& 16& 8& 4& 2& 1\\

505&&&&&&&&&\\
1516& 758& 379& 1138& 569& 1708& 854& 427& 1282& 641\\
1924& 962& 481& 1444& 722& 361& 1084& 542& 271& 814\\
407& 1222& 611& 1834& 917& 2752& 1376& 688& 344& 172\\
86& 43& 130& 65& 196& 98& 49& 148& 74& 37\\
112& 56& 28& 14& 7& 22& 11& 34& 17& 52\\
26& 13& 40& 20& 10& 5& 16& 8& 4& 2\\
1& \\

506&&&&&&&&&\\
253& 760& 380& 190& 95& 286& 143& 430& 215& 646\\
323& 970& 485& 1456& 728& 364& 182& 91& 274& 137\\
412& 206& 103& 310& 155& 466& 233& 700& 350& 175\\
526& 263& 790& 395& 1186& 593& 1780& 890& 445& 1336\\
668& 334& 167& 502& 251& 754& 377& 1132& 566& 283\\
850& 425& 1276& 638& 319& 958& 479& 1438& 719& 2158\\
1079& 3238& 1619& 4858& 2429& 7288& 3644& 1822& 911& 2734\\
1367& 4102& 2051& 6154& 3077& 9232& 4616& 2308& 1154& 577\\
1732& 866& 433& 1300& 650& 325& 976& 488& 244& 122\\
61& 184& 92& 46& 23& 70& 35& 106& 53& 160\\
80& 40& 20& 10& 5& 16& 8& 4& 2& 1\\

507&&&&&&&&&\\
1522& 761& 2284& 1142& 571& 1714& 857& 2572& 1286& 643\\
1930& 965& 2896& 1448& 724& 362& 181& 544& 272& 136\\
68& 34& 17& 52& 26& 13& 40& 20& 10& 5\\
16& 8& 4& 2& 1& \\

508&&&&&&&&&\\
254& 127& 382& 191& 574& 287& 862& 431& 1294& 647\\
1942& 971& 2914& 1457& 4372& 2186& 1093& 3280& 1640& 820\\
410& 205& 616& 308& 154& 77& 232& 116& 58& 29\\
88& 44& 22& 11& 34& 17& 52& 26& 13& 40\\
20& 10& 5& 16& 8& 4& 2& 1& \\

509&&&&&&&&&\\
1528& 764& 382& 191& 574& 287& 862& 431& 1294& 647\\
1942& 971& 2914& 1457& 4372& 2186& 1093& 3280& 1640& 820\\
410& 205& 616& 308& 154& 77& 232& 116& 58& 29\\
88& 44& 22& 11& 34& 17& 52& 26& 13& 40\\
20& 10& 5& 16& 8& 4& 2& 1& \\

510&&&&&&&&&\\
255& 766& 383& 1150& 575& 1726& 863& 2590& 1295& 3886\\
1943& 5830& 2915& 8746& 4373& 13120& 6560& 3280& 1640& 820\\
410& 205& 616& 308& 154& 77& 232& 116& 58& 29\\
88& 44& 22& 11& 34& 17& 52& 26& 13& 40\\
20& 10& 5& 16& 8& 4& 2& 1& \\

511&&&&&&&&&\\
1534& 767& 2302& 1151& 3454& 1727& 5182& 2591& 7774& 3887\\
11662& 5831& 17494& 8747& 26242& 13121& 39364& 19682& 9841& 29524\\
14762& 7381& 22144& 11072& 5536& 2768& 1384& 692& 346& 173\\
520& 260& 130& 65& 196& 98& 49& 148& 74& 37\\
112& 56& 28& 14& 7& 22& 11& 34& 17& 52\\
26& 13& 40& 20& 10& 5& 16& 8& 4& 2\\
1& \\

512&&&&&&&&&\\
256& 128& 64& 32& 16& 8& 4& 2& 1& \\

513&&&&&&&&&\\
1540& 770& 385& 1156& 578& 289& 868& 434& 217& 652\\
326& 163& 490& 245& 736& 368& 184& 92& 46& 23\\
70& 35& 106& 53& 160& 80& 40& 20& 10& 5\\
16& 8& 4& 2& 1& \\

514&&&&&&&&&\\
257& 772& 386& 193& 580& 290& 145& 436& 218& 109\\
328& 164& 82& 41& 124& 62& 31& 94& 47& 142\\
71& 214& 107& 322& 161& 484& 242& 121& 364& 182\\
91& 274& 137& 412& 206& 103& 310& 155& 466& 233\\
700& 350& 175& 526& 263& 790& 395& 1186& 593& 1780\\
890& 445& 1336& 668& 334& 167& 502& 251& 754& 377\\
1132& 566& 283& 850& 425& 1276& 638& 319& 958& 479\\
1438& 719& 2158& 1079& 3238& 1619& 4858& 2429& 7288& 3644\\
1822& 911& 2734& 1367& 4102& 2051& 6154& 3077& 9232& 4616\\
2308& 1154& 577& 1732& 866& 433& 1300& 650& 325& 976\\
488& 244& 122& 61& 184& 92& 46& 23& 70& 35\\
106& 53& 160& 80& 40& 20& 10& 5& 16& 8\\
4& 2& 1& \\

515&&&&&&&&&\\
1546& 773& 2320& 1160& 580& 290& 145& 436& 218& 109\\
328& 164& 82& 41& 124& 62& 31& 94& 47& 142\\
71& 214& 107& 322& 161& 484& 242& 121& 364& 182\\
91& 274& 137& 412& 206& 103& 310& 155& 466& 233\\
700& 350& 175& 526& 263& 790& 395& 1186& 593& 1780\\
890& 445& 1336& 668& 334& 167& 502& 251& 754& 377\\
1132& 566& 283& 850& 425& 1276& 638& 319& 958& 479\\
1438& 719& 2158& 1079& 3238& 1619& 4858& 2429& 7288& 3644\\
1822& 911& 2734& 1367& 4102& 2051& 6154& 3077& 9232& 4616\\
2308& 1154& 577& 1732& 866& 433& 1300& 650& 325& 976\\
488& 244& 122& 61& 184& 92& 46& 23& 70& 35\\
106& 53& 160& 80& 40& 20& 10& 5& 16& 8\\
4& 2& 1& \\

516&&&&&&&&&\\
258& 129& 388& 194& 97& 292& 146& 73& 220& 110\\
55& 166& 83& 250& 125& 376& 188& 94& 47& 142\\
71& 214& 107& 322& 161& 484& 242& 121& 364& 182\\
91& 274& 137& 412& 206& 103& 310& 155& 466& 233\\
700& 350& 175& 526& 263& 790& 395& 1186& 593& 1780\\
890& 445& 1336& 668& 334& 167& 502& 251& 754& 377\\
1132& 566& 283& 850& 425& 1276& 638& 319& 958& 479\\
1438& 719& 2158& 1079& 3238& 1619& 4858& 2429& 7288& 3644\\
1822& 911& 2734& 1367& 4102& 2051& 6154& 3077& 9232& 4616\\
2308& 1154& 577& 1732& 866& 433& 1300& 650& 325& 976\\
488& 244& 122& 61& 184& 92& 46& 23& 70& 35\\
106& 53& 160& 80& 40& 20& 10& 5& 16& 8\\
4& 2& 1& \\

517&&&&&&&&&\\
1552& 776& 388& 194& 97& 292& 146& 73& 220& 110\\
55& 166& 83& 250& 125& 376& 188& 94& 47& 142\\
71& 214& 107& 322& 161& 484& 242& 121& 364& 182\\
91& 274& 137& 412& 206& 103& 310& 155& 466& 233\\
700& 350& 175& 526& 263& 790& 395& 1186& 593& 1780\\
890& 445& 1336& 668& 334& 167& 502& 251& 754& 377\\
1132& 566& 283& 850& 425& 1276& 638& 319& 958& 479\\
1438& 719& 2158& 1079& 3238& 1619& 4858& 2429& 7288& 3644\\
1822& 911& 2734& 1367& 4102& 2051& 6154& 3077& 9232& 4616\\
2308& 1154& 577& 1732& 866& 433& 1300& 650& 325& 976\\
488& 244& 122& 61& 184& 92& 46& 23& 70& 35\\
106& 53& 160& 80& 40& 20& 10& 5& 16& 8\\
4& 2& 1& \\

518&&&&&&&&&\\
259& 778& 389& 1168& 584& 292& 146& 73& 220& 110\\
55& 166& 83& 250& 125& 376& 188& 94& 47& 142\\
71& 214& 107& 322& 161& 484& 242& 121& 364& 182\\
91& 274& 137& 412& 206& 103& 310& 155& 466& 233\\
700& 350& 175& 526& 263& 790& 395& 1186& 593& 1780\\
890& 445& 1336& 668& 334& 167& 502& 251& 754& 377\\
1132& 566& 283& 850& 425& 1276& 638& 319& 958& 479\\
1438& 719& 2158& 1079& 3238& 1619& 4858& 2429& 7288& 3644\\
1822& 911& 2734& 1367& 4102& 2051& 6154& 3077& 9232& 4616\\
2308& 1154& 577& 1732& 866& 433& 1300& 650& 325& 976\\
488& 244& 122& 61& 184& 92& 46& 23& 70& 35\\
106& 53& 160& 80& 40& 20& 10& 5& 16& 8\\
4& 2& 1& \\

519&&&&&&&&&\\
1558& 779& 2338& 1169& 3508& 1754& 877& 2632& 1316& 658\\
329& 988& 494& 247& 742& 371& 1114& 557& 1672& 836\\
418& 209& 628& 314& 157& 472& 236& 118& 59& 178\\
89& 268& 134& 67& 202& 101& 304& 152& 76& 38\\
19& 58& 29& 88& 44& 22& 11& 34& 17& 52\\
26& 13& 40& 20& 10& 5& 16& 8& 4& 2\\
1& \\

520&&&&&&&&&\\
260& 130& 65& 196& 98& 49& 148& 74& 37& 112\\
56& 28& 14& 7& 22& 11& 34& 17& 52& 26\\
13& 40& 20& 10& 5& 16& 8& 4& 2& 1\\

521&&&&&&&&&\\
1564& 782& 391& 1174& 587& 1762& 881& 2644& 1322& 661\\
1984& 992& 496& 248& 124& 62& 31& 94& 47& 142\\
71& 214& 107& 322& 161& 484& 242& 121& 364& 182\\
91& 274& 137& 412& 206& 103& 310& 155& 466& 233\\
700& 350& 175& 526& 263& 790& 395& 1186& 593& 1780\\
890& 445& 1336& 668& 334& 167& 502& 251& 754& 377\\
1132& 566& 283& 850& 425& 1276& 638& 319& 958& 479\\
1438& 719& 2158& 1079& 3238& 1619& 4858& 2429& 7288& 3644\\
1822& 911& 2734& 1367& 4102& 2051& 6154& 3077& 9232& 4616\\
2308& 1154& 577& 1732& 866& 433& 1300& 650& 325& 976\\
488& 244& 122& 61& 184& 92& 46& 23& 70& 35\\
106& 53& 160& 80& 40& 20& 10& 5& 16& 8\\
4& 2& 1& \\

522&&&&&&&&&\\
261& 784& 392& 196& 98& 49& 148& 74& 37& 112\\
56& 28& 14& 7& 22& 11& 34& 17& 52& 26\\
13& 40& 20& 10& 5& 16& 8& 4& 2& 1\\

523&&&&&&&&&\\
1570& 785& 2356& 1178& 589& 1768& 884& 442& 221& 664\\
332& 166& 83& 250& 125& 376& 188& 94& 47& 142\\
71& 214& 107& 322& 161& 484& 242& 121& 364& 182\\
91& 274& 137& 412& 206& 103& 310& 155& 466& 233\\
700& 350& 175& 526& 263& 790& 395& 1186& 593& 1780\\
890& 445& 1336& 668& 334& 167& 502& 251& 754& 377\\
1132& 566& 283& 850& 425& 1276& 638& 319& 958& 479\\
1438& 719& 2158& 1079& 3238& 1619& 4858& 2429& 7288& 3644\\
1822& 911& 2734& 1367& 4102& 2051& 6154& 3077& 9232& 4616\\
2308& 1154& 577& 1732& 866& 433& 1300& 650& 325& 976\\
488& 244& 122& 61& 184& 92& 46& 23& 70& 35\\
106& 53& 160& 80& 40& 20& 10& 5& 16& 8\\
4& 2& 1& \\

524&&&&&&&&&\\
262& 131& 394& 197& 592& 296& 148& 74& 37& 112\\
56& 28& 14& 7& 22& 11& 34& 17& 52& 26\\
13& 40& 20& 10& 5& 16& 8& 4& 2& 1\\

525&&&&&&&&&\\
1576& 788& 394& 197& 592& 296& 148& 74& 37& 112\\
56& 28& 14& 7& 22& 11& 34& 17& 52& 26\\
13& 40& 20& 10& 5& 16& 8& 4& 2& 1\\

526&&&&&&&&&\\
263& 790& 395& 1186& 593& 1780& 890& 445& 1336& 668\\
334& 167& 502& 251& 754& 377& 1132& 566& 283& 850\\
425& 1276& 638& 319& 958& 479& 1438& 719& 2158& 1079\\
3238& 1619& 4858& 2429& 7288& 3644& 1822& 911& 2734& 1367\\
4102& 2051& 6154& 3077& 9232& 4616& 2308& 1154& 577& 1732\\
866& 433& 1300& 650& 325& 976& 488& 244& 122& 61\\
184& 92& 46& 23& 70& 35& 106& 53& 160& 80\\
40& 20& 10& 5& 16& 8& 4& 2& 1& \\

527&&&&&&&&&\\
1582& 791& 2374& 1187& 3562& 1781& 5344& 2672& 1336& 668\\
334& 167& 502& 251& 754& 377& 1132& 566& 283& 850\\
425& 1276& 638& 319& 958& 479& 1438& 719& 2158& 1079\\
3238& 1619& 4858& 2429& 7288& 3644& 1822& 911& 2734& 1367\\
4102& 2051& 6154& 3077& 9232& 4616& 2308& 1154& 577& 1732\\
866& 433& 1300& 650& 325& 976& 488& 244& 122& 61\\
184& 92& 46& 23& 70& 35& 106& 53& 160& 80\\
40& 20& 10& 5& 16& 8& 4& 2& 1& \\

528&&&&&&&&&\\
264& 132& 66& 33& 100& 50& 25& 76& 38& 19\\
58& 29& 88& 44& 22& 11& 34& 17& 52& 26\\
13& 40& 20& 10& 5& 16& 8& 4& 2& 1\\

529&&&&&&&&&\\
1588& 794& 397& 1192& 596& 298& 149& 448& 224& 112\\
56& 28& 14& 7& 22& 11& 34& 17& 52& 26\\
13& 40& 20& 10& 5& 16& 8& 4& 2& 1\\

530&&&&&&&&&\\
265& 796& 398& 199& 598& 299& 898& 449& 1348& 674\\
337& 1012& 506& 253& 760& 380& 190& 95& 286& 143\\
430& 215& 646& 323& 970& 485& 1456& 728& 364& 182\\
91& 274& 137& 412& 206& 103& 310& 155& 466& 233\\
700& 350& 175& 526& 263& 790& 395& 1186& 593& 1780\\
890& 445& 1336& 668& 334& 167& 502& 251& 754& 377\\
1132& 566& 283& 850& 425& 1276& 638& 319& 958& 479\\
1438& 719& 2158& 1079& 3238& 1619& 4858& 2429& 7288& 3644\\
1822& 911& 2734& 1367& 4102& 2051& 6154& 3077& 9232& 4616\\
2308& 1154& 577& 1732& 866& 433& 1300& 650& 325& 976\\
488& 244& 122& 61& 184& 92& 46& 23& 70& 35\\
106& 53& 160& 80& 40& 20& 10& 5& 16& 8\\
4& 2& 1& \\

531&&&&&&&&&\\
1594& 797& 2392& 1196& 598& 299& 898& 449& 1348& 674\\
337& 1012& 506& 253& 760& 380& 190& 95& 286& 143\\
430& 215& 646& 323& 970& 485& 1456& 728& 364& 182\\
91& 274& 137& 412& 206& 103& 310& 155& 466& 233\\
700& 350& 175& 526& 263& 790& 395& 1186& 593& 1780\\
890& 445& 1336& 668& 334& 167& 502& 251& 754& 377\\
1132& 566& 283& 850& 425& 1276& 638& 319& 958& 479\\
1438& 719& 2158& 1079& 3238& 1619& 4858& 2429& 7288& 3644\\
1822& 911& 2734& 1367& 4102& 2051& 6154& 3077& 9232& 4616\\
2308& 1154& 577& 1732& 866& 433& 1300& 650& 325& 976\\
488& 244& 122& 61& 184& 92& 46& 23& 70& 35\\
106& 53& 160& 80& 40& 20& 10& 5& 16& 8\\
4& 2& 1& \\

532&&&&&&&&&\\
266& 133& 400& 200& 100& 50& 25& 76& 38& 19\\
58& 29& 88& 44& 22& 11& 34& 17& 52& 26\\
13& 40& 20& 10& 5& 16& 8& 4& 2& 1\\

533&&&&&&&&&\\
1600& 800& 400& 200& 100& 50& 25& 76& 38& 19\\
58& 29& 88& 44& 22& 11& 34& 17& 52& 26\\
13& 40& 20& 10& 5& 16& 8& 4& 2& 1\\

534&&&&&&&&&\\
267& 802& 401& 1204& 602& 301& 904& 452& 226& 113\\
340& 170& 85& 256& 128& 64& 32& 16& 8& 4\\
2& 1& \\

535&&&&&&&&&\\
1606& 803& 2410& 1205& 3616& 1808& 904& 452& 226& 113\\
340& 170& 85& 256& 128& 64& 32& 16& 8& 4\\
2& 1& \\

536&&&&&&&&&\\
268& 134& 67& 202& 101& 304& 152& 76& 38& 19\\
58& 29& 88& 44& 22& 11& 34& 17& 52& 26\\
13& 40& 20& 10& 5& 16& 8& 4& 2& 1\\

537&&&&&&&&&\\
1612& 806& 403& 1210& 605& 1816& 908& 454& 227& 682\\
341& 1024& 512& 256& 128& 64& 32& 16& 8& 4\\
2& 1& \\

538&&&&&&&&&\\
269& 808& 404& 202& 101& 304& 152& 76& 38& 19\\
58& 29& 88& 44& 22& 11& 34& 17& 52& 26\\
13& 40& 20& 10& 5& 16& 8& 4& 2& 1\\

539&&&&&&&&&\\
1618& 809& 2428& 1214& 607& 1822& 911& 2734& 1367& 4102\\
2051& 6154& 3077& 9232& 4616& 2308& 1154& 577& 1732& 866\\
433& 1300& 650& 325& 976& 488& 244& 122& 61& 184\\
92& 46& 23& 70& 35& 106& 53& 160& 80& 40\\
20& 10& 5& 16& 8& 4& 2& 1& \\

540&&&&&&&&&\\
270& 135& 406& 203& 610& 305& 916& 458& 229& 688\\
344& 172& 86& 43& 130& 65& 196& 98& 49& 148\\
74& 37& 112& 56& 28& 14& 7& 22& 11& 34\\
17& 52& 26& 13& 40& 20& 10& 5& 16& 8\\
4& 2& 1& \\

541&&&&&&&&&\\
1624& 812& 406& 203& 610& 305& 916& 458& 229& 688\\
344& 172& 86& 43& 130& 65& 196& 98& 49& 148\\
74& 37& 112& 56& 28& 14& 7& 22& 11& 34\\
17& 52& 26& 13& 40& 20& 10& 5& 16& 8\\
4& 2& 1& \\

542&&&&&&&&&\\
271& 814& 407& 1222& 611& 1834& 917& 2752& 1376& 688\\
344& 172& 86& 43& 130& 65& 196& 98& 49& 148\\
74& 37& 112& 56& 28& 14& 7& 22& 11& 34\\
17& 52& 26& 13& 40& 20& 10& 5& 16& 8\\
4& 2& 1& \\

543&&&&&&&&&\\
1630& 815& 2446& 1223& 3670& 1835& 5506& 2753& 8260& 4130\\
2065& 6196& 3098& 1549& 4648& 2324& 1162& 581& 1744& 872\\
436& 218& 109& 328& 164& 82& 41& 124& 62& 31\\
94& 47& 142& 71& 214& 107& 322& 161& 484& 242\\
121& 364& 182& 91& 274& 137& 412& 206& 103& 310\\
155& 466& 233& 700& 350& 175& 526& 263& 790& 395\\
1186& 593& 1780& 890& 445& 1336& 668& 334& 167& 502\\
251& 754& 377& 1132& 566& 283& 850& 425& 1276& 638\\
319& 958& 479& 1438& 719& 2158& 1079& 3238& 1619& 4858\\
2429& 7288& 3644& 1822& 911& 2734& 1367& 4102& 2051& 6154\\
3077& 9232& 4616& 2308& 1154& 577& 1732& 866& 433& 1300\\
650& 325& 976& 488& 244& 122& 61& 184& 92& 46\\
23& 70& 35& 106& 53& 160& 80& 40& 20& 10\\
5& 16& 8& 4& 2& 1& \\

544&&&&&&&&&\\
272& 136& 68& 34& 17& 52& 26& 13& 40& 20\\
10& 5& 16& 8& 4& 2& 1& \\

545&&&&&&&&&\\
1636& 818& 409& 1228& 614& 307& 922& 461& 1384& 692\\
346& 173& 520& 260& 130& 65& 196& 98& 49& 148\\
74& 37& 112& 56& 28& 14& 7& 22& 11& 34\\
17& 52& 26& 13& 40& 20& 10& 5& 16& 8\\
4& 2& 1& \\

546&&&&&&&&&\\
273& 820& 410& 205& 616& 308& 154& 77& 232& 116\\
58& 29& 88& 44& 22& 11& 34& 17& 52& 26\\
13& 40& 20& 10& 5& 16& 8& 4& 2& 1\\

547&&&&&&&&&\\
1642& 821& 2464& 1232& 616& 308& 154& 77& 232& 116\\
58& 29& 88& 44& 22& 11& 34& 17& 52& 26\\
13& 40& 20& 10& 5& 16& 8& 4& 2& 1\\

548&&&&&&&&&\\
274& 137& 412& 206& 103& 310& 155& 466& 233& 700\\
350& 175& 526& 263& 790& 395& 1186& 593& 1780& 890\\
445& 1336& 668& 334& 167& 502& 251& 754& 377& 1132\\
566& 283& 850& 425& 1276& 638& 319& 958& 479& 1438\\
719& 2158& 1079& 3238& 1619& 4858& 2429& 7288& 3644& 1822\\
911& 2734& 1367& 4102& 2051& 6154& 3077& 9232& 4616& 2308\\
1154& 577& 1732& 866& 433& 1300& 650& 325& 976& 488\\
244& 122& 61& 184& 92& 46& 23& 70& 35& 106\\
53& 160& 80& 40& 20& 10& 5& 16& 8& 4\\
2& 1& \\

549&&&&&&&&&\\
1648& 824& 412& 206& 103& 310& 155& 466& 233& 700\\
350& 175& 526& 263& 790& 395& 1186& 593& 1780& 890\\
445& 1336& 668& 334& 167& 502& 251& 754& 377& 1132\\
566& 283& 850& 425& 1276& 638& 319& 958& 479& 1438\\
719& 2158& 1079& 3238& 1619& 4858& 2429& 7288& 3644& 1822\\
911& 2734& 1367& 4102& 2051& 6154& 3077& 9232& 4616& 2308\\
1154& 577& 1732& 866& 433& 1300& 650& 325& 976& 488\\
244& 122& 61& 184& 92& 46& 23& 70& 35& 106\\
53& 160& 80& 40& 20& 10& 5& 16& 8& 4\\
2& 1& \\

550&&&&&&&&&\\
275& 826& 413& 1240& 620& 310& 155& 466& 233& 700\\
350& 175& 526& 263& 790& 395& 1186& 593& 1780& 890\\
445& 1336& 668& 334& 167& 502& 251& 754& 377& 1132\\
566& 283& 850& 425& 1276& 638& 319& 958& 479& 1438\\
719& 2158& 1079& 3238& 1619& 4858& 2429& 7288& 3644& 1822\\
911& 2734& 1367& 4102& 2051& 6154& 3077& 9232& 4616& 2308\\
1154& 577& 1732& 866& 433& 1300& 650& 325& 976& 488\\
244& 122& 61& 184& 92& 46& 23& 70& 35& 106\\
53& 160& 80& 40& 20& 10& 5& 16& 8& 4\\
2& 1& \\

551&&&&&&&&&\\
1654& 827& 2482& 1241& 3724& 1862& 931& 2794& 1397& 4192\\
2096& 1048& 524& 262& 131& 394& 197& 592& 296& 148\\
74& 37& 112& 56& 28& 14& 7& 22& 11& 34\\
17& 52& 26& 13& 40& 20& 10& 5& 16& 8\\
4& 2& 1& \\

552&&&&&&&&&\\
276& 138& 69& 208& 104& 52& 26& 13& 40& 20\\
10& 5& 16& 8& 4& 2& 1& \\

553&&&&&&&&&\\
1660& 830& 415& 1246& 623& 1870& 935& 2806& 1403& 4210\\
2105& 6316& 3158& 1579& 4738& 2369& 7108& 3554& 1777& 5332\\
2666& 1333& 4000& 2000& 1000& 500& 250& 125& 376& 188\\
94& 47& 142& 71& 214& 107& 322& 161& 484& 242\\
121& 364& 182& 91& 274& 137& 412& 206& 103& 310\\
155& 466& 233& 700& 350& 175& 526& 263& 790& 395\\
1186& 593& 1780& 890& 445& 1336& 668& 334& 167& 502\\
251& 754& 377& 1132& 566& 283& 850& 425& 1276& 638\\
319& 958& 479& 1438& 719& 2158& 1079& 3238& 1619& 4858\\
2429& 7288& 3644& 1822& 911& 2734& 1367& 4102& 2051& 6154\\
3077& 9232& 4616& 2308& 1154& 577& 1732& 866& 433& 1300\\
650& 325& 976& 488& 244& 122& 61& 184& 92& 46\\
23& 70& 35& 106& 53& 160& 80& 40& 20& 10\\
5& 16& 8& 4& 2& 1& \\

554&&&&&&&&&\\
277& 832& 416& 208& 104& 52& 26& 13& 40& 20\\
10& 5& 16& 8& 4& 2& 1& \\

555&&&&&&&&&\\
1666& 833& 2500& 1250& 625& 1876& 938& 469& 1408& 704\\
352& 176& 88& 44& 22& 11& 34& 17& 52& 26\\
13& 40& 20& 10& 5& 16& 8& 4& 2& 1\\

556&&&&&&&&&\\
278& 139& 418& 209& 628& 314& 157& 472& 236& 118\\
59& 178& 89& 268& 134& 67& 202& 101& 304& 152\\
76& 38& 19& 58& 29& 88& 44& 22& 11& 34\\
17& 52& 26& 13& 40& 20& 10& 5& 16& 8\\
4& 2& 1& \\

557&&&&&&&&&\\
1672& 836& 418& 209& 628& 314& 157& 472& 236& 118\\
59& 178& 89& 268& 134& 67& 202& 101& 304& 152\\
76& 38& 19& 58& 29& 88& 44& 22& 11& 34\\
17& 52& 26& 13& 40& 20& 10& 5& 16& 8\\
4& 2& 1& \\

558&&&&&&&&&\\
279& 838& 419& 1258& 629& 1888& 944& 472& 236& 118\\
59& 178& 89& 268& 134& 67& 202& 101& 304& 152\\
76& 38& 19& 58& 29& 88& 44& 22& 11& 34\\
17& 52& 26& 13& 40& 20& 10& 5& 16& 8\\
4& 2& 1& \\

559&&&&&&&&&\\
1678& 839& 2518& 1259& 3778& 1889& 5668& 2834& 1417& 4252\\
2126& 1063& 3190& 1595& 4786& 2393& 7180& 3590& 1795& 5386\\
2693& 8080& 4040& 2020& 1010& 505& 1516& 758& 379& 1138\\
569& 1708& 854& 427& 1282& 641& 1924& 962& 481& 1444\\
722& 361& 1084& 542& 271& 814& 407& 1222& 611& 1834\\
917& 2752& 1376& 688& 344& 172& 86& 43& 130& 65\\
196& 98& 49& 148& 74& 37& 112& 56& 28& 14\\
7& 22& 11& 34& 17& 52& 26& 13& 40& 20\\
10& 5& 16& 8& 4& 2& 1& \\

560&&&&&&&&&\\
280& 140& 70& 35& 106& 53& 160& 80& 40& 20\\
10& 5& 16& 8& 4& 2& 1& \\

561&&&&&&&&&\\
1684& 842& 421& 1264& 632& 316& 158& 79& 238& 119\\
358& 179& 538& 269& 808& 404& 202& 101& 304& 152\\
76& 38& 19& 58& 29& 88& 44& 22& 11& 34\\
17& 52& 26& 13& 40& 20& 10& 5& 16& 8\\
4& 2& 1& \\

562&&&&&&&&&\\
281& 844& 422& 211& 634& 317& 952& 476& 238& 119\\
358& 179& 538& 269& 808& 404& 202& 101& 304& 152\\
76& 38& 19& 58& 29& 88& 44& 22& 11& 34\\
17& 52& 26& 13& 40& 20& 10& 5& 16& 8\\
4& 2& 1& \\

563&&&&&&&&&\\
1690& 845& 2536& 1268& 634& 317& 952& 476& 238& 119\\
358& 179& 538& 269& 808& 404& 202& 101& 304& 152\\
76& 38& 19& 58& 29& 88& 44& 22& 11& 34\\
17& 52& 26& 13& 40& 20& 10& 5& 16& 8\\
4& 2& 1& \\

564&&&&&&&&&\\
282& 141& 424& 212& 106& 53& 160& 80& 40& 20\\
10& 5& 16& 8& 4& 2& 1& \\

565&&&&&&&&&\\
1696& 848& 424& 212& 106& 53& 160& 80& 40& 20\\
10& 5& 16& 8& 4& 2& 1& \\

566&&&&&&&&&\\
283& 850& 425& 1276& 638& 319& 958& 479& 1438& 719\\
2158& 1079& 3238& 1619& 4858& 2429& 7288& 3644& 1822& 911\\
2734& 1367& 4102& 2051& 6154& 3077& 9232& 4616& 2308& 1154\\
577& 1732& 866& 433& 1300& 650& 325& 976& 488& 244\\
122& 61& 184& 92& 46& 23& 70& 35& 106& 53\\
160& 80& 40& 20& 10& 5& 16& 8& 4& 2\\
1& \\

567&&&&&&&&&\\
1702& 851& 2554& 1277& 3832& 1916& 958& 479& 1438& 719\\
2158& 1079& 3238& 1619& 4858& 2429& 7288& 3644& 1822& 911\\
2734& 1367& 4102& 2051& 6154& 3077& 9232& 4616& 2308& 1154\\
577& 1732& 866& 433& 1300& 650& 325& 976& 488& 244\\
122& 61& 184& 92& 46& 23& 70& 35& 106& 53\\
160& 80& 40& 20& 10& 5& 16& 8& 4& 2\\
1& \\

568&&&&&&&&&\\
284& 142& 71& 214& 107& 322& 161& 484& 242& 121\\
364& 182& 91& 274& 137& 412& 206& 103& 310& 155\\
466& 233& 700& 350& 175& 526& 263& 790& 395& 1186\\
593& 1780& 890& 445& 1336& 668& 334& 167& 502& 251\\
754& 377& 1132& 566& 283& 850& 425& 1276& 638& 319\\
958& 479& 1438& 719& 2158& 1079& 3238& 1619& 4858& 2429\\
7288& 3644& 1822& 911& 2734& 1367& 4102& 2051& 6154& 3077\\
9232& 4616& 2308& 1154& 577& 1732& 866& 433& 1300& 650\\
325& 976& 488& 244& 122& 61& 184& 92& 46& 23\\
70& 35& 106& 53& 160& 80& 40& 20& 10& 5\\
16& 8& 4& 2& 1& \\

569&&&&&&&&&\\
1708& 854& 427& 1282& 641& 1924& 962& 481& 1444& 722\\
361& 1084& 542& 271& 814& 407& 1222& 611& 1834& 917\\
2752& 1376& 688& 344& 172& 86& 43& 130& 65& 196\\
98& 49& 148& 74& 37& 112& 56& 28& 14& 7\\
22& 11& 34& 17& 52& 26& 13& 40& 20& 10\\
5& 16& 8& 4& 2& 1& \\

570&&&&&&&&&\\
285& 856& 428& 214& 107& 322& 161& 484& 242& 121\\
364& 182& 91& 274& 137& 412& 206& 103& 310& 155\\
466& 233& 700& 350& 175& 526& 263& 790& 395& 1186\\
593& 1780& 890& 445& 1336& 668& 334& 167& 502& 251\\
754& 377& 1132& 566& 283& 850& 425& 1276& 638& 319\\
958& 479& 1438& 719& 2158& 1079& 3238& 1619& 4858& 2429\\
7288& 3644& 1822& 911& 2734& 1367& 4102& 2051& 6154& 3077\\
9232& 4616& 2308& 1154& 577& 1732& 866& 433& 1300& 650\\
325& 976& 488& 244& 122& 61& 184& 92& 46& 23\\
70& 35& 106& 53& 160& 80& 40& 20& 10& 5\\
16& 8& 4& 2& 1& \\

571&&&&&&&&&\\
1714& 857& 2572& 1286& 643& 1930& 965& 2896& 1448& 724\\
362& 181& 544& 272& 136& 68& 34& 17& 52& 26\\
13& 40& 20& 10& 5& 16& 8& 4& 2& 1\\

572&&&&&&&&&\\
286& 143& 430& 215& 646& 323& 970& 485& 1456& 728\\
364& 182& 91& 274& 137& 412& 206& 103& 310& 155\\
466& 233& 700& 350& 175& 526& 263& 790& 395& 1186\\
593& 1780& 890& 445& 1336& 668& 334& 167& 502& 251\\
754& 377& 1132& 566& 283& 850& 425& 1276& 638& 319\\
958& 479& 1438& 719& 2158& 1079& 3238& 1619& 4858& 2429\\
7288& 3644& 1822& 911& 2734& 1367& 4102& 2051& 6154& 3077\\
9232& 4616& 2308& 1154& 577& 1732& 866& 433& 1300& 650\\
325& 976& 488& 244& 122& 61& 184& 92& 46& 23\\
70& 35& 106& 53& 160& 80& 40& 20& 10& 5\\
16& 8& 4& 2& 1& \\

573&&&&&&&&&\\
1720& 860& 430& 215& 646& 323& 970& 485& 1456& 728\\
364& 182& 91& 274& 137& 412& 206& 103& 310& 155\\
466& 233& 700& 350& 175& 526& 263& 790& 395& 1186\\
593& 1780& 890& 445& 1336& 668& 334& 167& 502& 251\\
754& 377& 1132& 566& 283& 850& 425& 1276& 638& 319\\
958& 479& 1438& 719& 2158& 1079& 3238& 1619& 4858& 2429\\
7288& 3644& 1822& 911& 2734& 1367& 4102& 2051& 6154& 3077\\
9232& 4616& 2308& 1154& 577& 1732& 866& 433& 1300& 650\\
325& 976& 488& 244& 122& 61& 184& 92& 46& 23\\
70& 35& 106& 53& 160& 80& 40& 20& 10& 5\\
16& 8& 4& 2& 1& \\

574&&&&&&&&&\\
287& 862& 431& 1294& 647& 1942& 971& 2914& 1457& 4372\\
2186& 1093& 3280& 1640& 820& 410& 205& 616& 308& 154\\
77& 232& 116& 58& 29& 88& 44& 22& 11& 34\\
17& 52& 26& 13& 40& 20& 10& 5& 16& 8\\
4& 2& 1& \\

575&&&&&&&&&\\
1726& 863& 2590& 1295& 3886& 1943& 5830& 2915& 8746& 4373\\
13120& 6560& 3280& 1640& 820& 410& 205& 616& 308& 154\\
77& 232& 116& 58& 29& 88& 44& 22& 11& 34\\
17& 52& 26& 13& 40& 20& 10& 5& 16& 8\\
4& 2& 1& \\

576&&&&&&&&&\\
288& 144& 72& 36& 18& 9& 28& 14& 7& 22\\
11& 34& 17& 52& 26& 13& 40& 20& 10& 5\\
16& 8& 4& 2& 1& \\

577&&&&&&&&&\\
1732& 866& 433& 1300& 650& 325& 976& 488& 244& 122\\
61& 184& 92& 46& 23& 70& 35& 106& 53& 160\\
80& 40& 20& 10& 5& 16& 8& 4& 2& 1\\

578&&&&&&&&&\\
289& 868& 434& 217& 652& 326& 163& 490& 245& 736\\
368& 184& 92& 46& 23& 70& 35& 106& 53& 160\\
80& 40& 20& 10& 5& 16& 8& 4& 2& 1\\

579&&&&&&&&&\\
1738& 869& 2608& 1304& 652& 326& 163& 490& 245& 736\\
368& 184& 92& 46& 23& 70& 35& 106& 53& 160\\
80& 40& 20& 10& 5& 16& 8& 4& 2& 1\\

580&&&&&&&&&\\
290& 145& 436& 218& 109& 328& 164& 82& 41& 124\\
62& 31& 94& 47& 142& 71& 214& 107& 322& 161\\
484& 242& 121& 364& 182& 91& 274& 137& 412& 206\\
103& 310& 155& 466& 233& 700& 350& 175& 526& 263\\
790& 395& 1186& 593& 1780& 890& 445& 1336& 668& 334\\
167& 502& 251& 754& 377& 1132& 566& 283& 850& 425\\
1276& 638& 319& 958& 479& 1438& 719& 2158& 1079& 3238\\
1619& 4858& 2429& 7288& 3644& 1822& 911& 2734& 1367& 4102\\
2051& 6154& 3077& 9232& 4616& 2308& 1154& 577& 1732& 866\\
433& 1300& 650& 325& 976& 488& 244& 122& 61& 184\\
92& 46& 23& 70& 35& 106& 53& 160& 80& 40\\
20& 10& 5& 16& 8& 4& 2& 1& \\

581&&&&&&&&&\\
1744& 872& 436& 218& 109& 328& 164& 82& 41& 124\\
62& 31& 94& 47& 142& 71& 214& 107& 322& 161\\
484& 242& 121& 364& 182& 91& 274& 137& 412& 206\\
103& 310& 155& 466& 233& 700& 350& 175& 526& 263\\
790& 395& 1186& 593& 1780& 890& 445& 1336& 668& 334\\
167& 502& 251& 754& 377& 1132& 566& 283& 850& 425\\
1276& 638& 319& 958& 479& 1438& 719& 2158& 1079& 3238\\
1619& 4858& 2429& 7288& 3644& 1822& 911& 2734& 1367& 4102\\
2051& 6154& 3077& 9232& 4616& 2308& 1154& 577& 1732& 866\\
433& 1300& 650& 325& 976& 488& 244& 122& 61& 184\\
92& 46& 23& 70& 35& 106& 53& 160& 80& 40\\
20& 10& 5& 16& 8& 4& 2& 1& \\

582&&&&&&&&&\\
291& 874& 437& 1312& 656& 328& 164& 82& 41& 124\\
62& 31& 94& 47& 142& 71& 214& 107& 322& 161\\
484& 242& 121& 364& 182& 91& 274& 137& 412& 206\\
103& 310& 155& 466& 233& 700& 350& 175& 526& 263\\
790& 395& 1186& 593& 1780& 890& 445& 1336& 668& 334\\
167& 502& 251& 754& 377& 1132& 566& 283& 850& 425\\
1276& 638& 319& 958& 479& 1438& 719& 2158& 1079& 3238\\
1619& 4858& 2429& 7288& 3644& 1822& 911& 2734& 1367& 4102\\
2051& 6154& 3077& 9232& 4616& 2308& 1154& 577& 1732& 866\\
433& 1300& 650& 325& 976& 488& 244& 122& 61& 184\\
92& 46& 23& 70& 35& 106& 53& 160& 80& 40\\
20& 10& 5& 16& 8& 4& 2& 1& \\

583&&&&&&&&&\\
1750& 875& 2626& 1313& 3940& 1970& 985& 2956& 1478& 739\\
2218& 1109& 3328& 1664& 832& 416& 208& 104& 52& 26\\
13& 40& 20& 10& 5& 16& 8& 4& 2& 1\\

584&&&&&&&&&\\
292& 146& 73& 220& 110& 55& 166& 83& 250& 125\\
376& 188& 94& 47& 142& 71& 214& 107& 322& 161\\
484& 242& 121& 364& 182& 91& 274& 137& 412& 206\\
103& 310& 155& 466& 233& 700& 350& 175& 526& 263\\
790& 395& 1186& 593& 1780& 890& 445& 1336& 668& 334\\
167& 502& 251& 754& 377& 1132& 566& 283& 850& 425\\
1276& 638& 319& 958& 479& 1438& 719& 2158& 1079& 3238\\
1619& 4858& 2429& 7288& 3644& 1822& 911& 2734& 1367& 4102\\
2051& 6154& 3077& 9232& 4616& 2308& 1154& 577& 1732& 866\\
433& 1300& 650& 325& 976& 488& 244& 122& 61& 184\\
92& 46& 23& 70& 35& 106& 53& 160& 80& 40\\
20& 10& 5& 16& 8& 4& 2& 1& \\

585&&&&&&&&&\\
1756& 878& 439& 1318& 659& 1978& 989& 2968& 1484& 742\\
371& 1114& 557& 1672& 836& 418& 209& 628& 314& 157\\
472& 236& 118& 59& 178& 89& 268& 134& 67& 202\\
101& 304& 152& 76& 38& 19& 58& 29& 88& 44\\
22& 11& 34& 17& 52& 26& 13& 40& 20& 10\\
5& 16& 8& 4& 2& 1& \\

586&&&&&&&&&\\
293& 880& 440& 220& 110& 55& 166& 83& 250& 125\\
376& 188& 94& 47& 142& 71& 214& 107& 322& 161\\
484& 242& 121& 364& 182& 91& 274& 137& 412& 206\\
103& 310& 155& 466& 233& 700& 350& 175& 526& 263\\
790& 395& 1186& 593& 1780& 890& 445& 1336& 668& 334\\
167& 502& 251& 754& 377& 1132& 566& 283& 850& 425\\
1276& 638& 319& 958& 479& 1438& 719& 2158& 1079& 3238\\
1619& 4858& 2429& 7288& 3644& 1822& 911& 2734& 1367& 4102\\
2051& 6154& 3077& 9232& 4616& 2308& 1154& 577& 1732& 866\\
433& 1300& 650& 325& 976& 488& 244& 122& 61& 184\\
92& 46& 23& 70& 35& 106& 53& 160& 80& 40\\
20& 10& 5& 16& 8& 4& 2& 1& \\

587&&&&&&&&&\\
1762& 881& 2644& 1322& 661& 1984& 992& 496& 248& 124\\
62& 31& 94& 47& 142& 71& 214& 107& 322& 161\\
484& 242& 121& 364& 182& 91& 274& 137& 412& 206\\
103& 310& 155& 466& 233& 700& 350& 175& 526& 263\\
790& 395& 1186& 593& 1780& 890& 445& 1336& 668& 334\\
167& 502& 251& 754& 377& 1132& 566& 283& 850& 425\\
1276& 638& 319& 958& 479& 1438& 719& 2158& 1079& 3238\\
1619& 4858& 2429& 7288& 3644& 1822& 911& 2734& 1367& 4102\\
2051& 6154& 3077& 9232& 4616& 2308& 1154& 577& 1732& 866\\
433& 1300& 650& 325& 976& 488& 244& 122& 61& 184\\
92& 46& 23& 70& 35& 106& 53& 160& 80& 40\\
20& 10& 5& 16& 8& 4& 2& 1& \\

588&&&&&&&&&\\
294& 147& 442& 221& 664& 332& 166& 83& 250& 125\\
376& 188& 94& 47& 142& 71& 214& 107& 322& 161\\
484& 242& 121& 364& 182& 91& 274& 137& 412& 206\\
103& 310& 155& 466& 233& 700& 350& 175& 526& 263\\
790& 395& 1186& 593& 1780& 890& 445& 1336& 668& 334\\
167& 502& 251& 754& 377& 1132& 566& 283& 850& 425\\
1276& 638& 319& 958& 479& 1438& 719& 2158& 1079& 3238\\
1619& 4858& 2429& 7288& 3644& 1822& 911& 2734& 1367& 4102\\
2051& 6154& 3077& 9232& 4616& 2308& 1154& 577& 1732& 866\\
433& 1300& 650& 325& 976& 488& 244& 122& 61& 184\\
92& 46& 23& 70& 35& 106& 53& 160& 80& 40\\
20& 10& 5& 16& 8& 4& 2& 1& \\

589&&&&&&&&&\\
1768& 884& 442& 221& 664& 332& 166& 83& 250& 125\\
376& 188& 94& 47& 142& 71& 214& 107& 322& 161\\
484& 242& 121& 364& 182& 91& 274& 137& 412& 206\\
103& 310& 155& 466& 233& 700& 350& 175& 526& 263\\
790& 395& 1186& 593& 1780& 890& 445& 1336& 668& 334\\
167& 502& 251& 754& 377& 1132& 566& 283& 850& 425\\
1276& 638& 319& 958& 479& 1438& 719& 2158& 1079& 3238\\
1619& 4858& 2429& 7288& 3644& 1822& 911& 2734& 1367& 4102\\
2051& 6154& 3077& 9232& 4616& 2308& 1154& 577& 1732& 866\\
433& 1300& 650& 325& 976& 488& 244& 122& 61& 184\\
92& 46& 23& 70& 35& 106& 53& 160& 80& 40\\
20& 10& 5& 16& 8& 4& 2& 1& \\

590&&&&&&&&&\\
295& 886& 443& 1330& 665& 1996& 998& 499& 1498& 749\\
2248& 1124& 562& 281& 844& 422& 211& 634& 317& 952\\
476& 238& 119& 358& 179& 538& 269& 808& 404& 202\\
101& 304& 152& 76& 38& 19& 58& 29& 88& 44\\
22& 11& 34& 17& 52& 26& 13& 40& 20& 10\\
5& 16& 8& 4& 2& 1& \\

591&&&&&&&&&\\
1774& 887& 2662& 1331& 3994& 1997& 5992& 2996& 1498& 749\\
2248& 1124& 562& 281& 844& 422& 211& 634& 317& 952\\
476& 238& 119& 358& 179& 538& 269& 808& 404& 202\\
101& 304& 152& 76& 38& 19& 58& 29& 88& 44\\
22& 11& 34& 17& 52& 26& 13& 40& 20& 10\\
5& 16& 8& 4& 2& 1& \\

592&&&&&&&&&\\
296& 148& 74& 37& 112& 56& 28& 14& 7& 22\\
11& 34& 17& 52& 26& 13& 40& 20& 10& 5\\
16& 8& 4& 2& 1& \\

593&&&&&&&&&\\
1780& 890& 445& 1336& 668& 334& 167& 502& 251& 754\\
377& 1132& 566& 283& 850& 425& 1276& 638& 319& 958\\
479& 1438& 719& 2158& 1079& 3238& 1619& 4858& 2429& 7288\\
3644& 1822& 911& 2734& 1367& 4102& 2051& 6154& 3077& 9232\\
4616& 2308& 1154& 577& 1732& 866& 433& 1300& 650& 325\\
976& 488& 244& 122& 61& 184& 92& 46& 23& 70\\
35& 106& 53& 160& 80& 40& 20& 10& 5& 16\\
8& 4& 2& 1& \\

594&&&&&&&&&\\
297& 892& 446& 223& 670& 335& 1006& 503& 1510& 755\\
2266& 1133& 3400& 1700& 850& 425& 1276& 638& 319& 958\\
479& 1438& 719& 2158& 1079& 3238& 1619& 4858& 2429& 7288\\
3644& 1822& 911& 2734& 1367& 4102& 2051& 6154& 3077& 9232\\
4616& 2308& 1154& 577& 1732& 866& 433& 1300& 650& 325\\
976& 488& 244& 122& 61& 184& 92& 46& 23& 70\\
35& 106& 53& 160& 80& 40& 20& 10& 5& 16\\
8& 4& 2& 1& \\

595&&&&&&&&&\\
1786& 893& 2680& 1340& 670& 335& 1006& 503& 1510& 755\\
2266& 1133& 3400& 1700& 850& 425& 1276& 638& 319& 958\\
479& 1438& 719& 2158& 1079& 3238& 1619& 4858& 2429& 7288\\
3644& 1822& 911& 2734& 1367& 4102& 2051& 6154& 3077& 9232\\
4616& 2308& 1154& 577& 1732& 866& 433& 1300& 650& 325\\
976& 488& 244& 122& 61& 184& 92& 46& 23& 70\\
35& 106& 53& 160& 80& 40& 20& 10& 5& 16\\
8& 4& 2& 1& \\

596&&&&&&&&&\\
298& 149& 448& 224& 112& 56& 28& 14& 7& 22\\
11& 34& 17& 52& 26& 13& 40& 20& 10& 5\\
16& 8& 4& 2& 1& \\

597&&&&&&&&&\\
1792& 896& 448& 224& 112& 56& 28& 14& 7& 22\\
11& 34& 17& 52& 26& 13& 40& 20& 10& 5\\
16& 8& 4& 2& 1& \\

598&&&&&&&&&\\
299& 898& 449& 1348& 674& 337& 1012& 506& 253& 760\\
380& 190& 95& 286& 143& 430& 215& 646& 323& 970\\
485& 1456& 728& 364& 182& 91& 274& 137& 412& 206\\
103& 310& 155& 466& 233& 700& 350& 175& 526& 263\\
790& 395& 1186& 593& 1780& 890& 445& 1336& 668& 334\\
167& 502& 251& 754& 377& 1132& 566& 283& 850& 425\\
1276& 638& 319& 958& 479& 1438& 719& 2158& 1079& 3238\\
1619& 4858& 2429& 7288& 3644& 1822& 911& 2734& 1367& 4102\\
2051& 6154& 3077& 9232& 4616& 2308& 1154& 577& 1732& 866\\
433& 1300& 650& 325& 976& 488& 244& 122& 61& 184\\
92& 46& 23& 70& 35& 106& 53& 160& 80& 40\\
20& 10& 5& 16& 8& 4& 2& 1& \\

599&&&&&&&&&\\
1798& 899& 2698& 1349& 4048& 2024& 1012& 506& 253& 760\\
380& 190& 95& 286& 143& 430& 215& 646& 323& 970\\
485& 1456& 728& 364& 182& 91& 274& 137& 412& 206\\
103& 310& 155& 466& 233& 700& 350& 175& 526& 263\\
790& 395& 1186& 593& 1780& 890& 445& 1336& 668& 334\\
167& 502& 251& 754& 377& 1132& 566& 283& 850& 425\\
1276& 638& 319& 958& 479& 1438& 719& 2158& 1079& 3238\\
1619& 4858& 2429& 7288& 3644& 1822& 911& 2734& 1367& 4102\\
2051& 6154& 3077& 9232& 4616& 2308& 1154& 577& 1732& 866\\
433& 1300& 650& 325& 976& 488& 244& 122& 61& 184\\
92& 46& 23& 70& 35& 106& 53& 160& 80& 40\\
20& 10& 5& 16& 8& 4& 2& 1& \\

600&&&&&&&&&\\
300& 150& 75& 226& 113& 340& 170& 85& 256& 128\\
64& 32& 16& 8& 4& 2& 1& \\

601&&&&&&&&&\\
1804& 902& 451& 1354& 677& 2032& 1016& 508& 254& 127\\
382& 191& 574& 287& 862& 431& 1294& 647& 1942& 971\\
2914& 1457& 4372& 2186& 1093& 3280& 1640& 820& 410& 205\\
616& 308& 154& 77& 232& 116& 58& 29& 88& 44\\
22& 11& 34& 17& 52& 26& 13& 40& 20& 10\\
5& 16& 8& 4& 2& 1& \\

602&&&&&&&&&\\
301& 904& 452& 226& 113& 340& 170& 85& 256& 128\\
64& 32& 16& 8& 4& 2& 1& \\

603&&&&&&&&&\\
1810& 905& 2716& 1358& 679& 2038& 1019& 3058& 1529& 4588\\
2294& 1147& 3442& 1721& 5164& 2582& 1291& 3874& 1937& 5812\\
2906& 1453& 4360& 2180& 1090& 545& 1636& 818& 409& 1228\\
614& 307& 922& 461& 1384& 692& 346& 173& 520& 260\\
130& 65& 196& 98& 49& 148& 74& 37& 112& 56\\
28& 14& 7& 22& 11& 34& 17& 52& 26& 13\\
40& 20& 10& 5& 16& 8& 4& 2& 1& \\

604&&&&&&&&&\\
302& 151& 454& 227& 682& 341& 1024& 512& 256& 128\\
64& 32& 16& 8& 4& 2& 1& \\

605&&&&&&&&&\\
1816& 908& 454& 227& 682& 341& 1024& 512& 256& 128\\
64& 32& 16& 8& 4& 2& 1& \\

606&&&&&&&&&\\
303& 910& 455& 1366& 683& 2050& 1025& 3076& 1538& 769\\
2308& 1154& 577& 1732& 866& 433& 1300& 650& 325& 976\\
488& 244& 122& 61& 184& 92& 46& 23& 70& 35\\
106& 53& 160& 80& 40& 20& 10& 5& 16& 8\\
4& 2& 1& \\

607&&&&&&&&&\\
1822& 911& 2734& 1367& 4102& 2051& 6154& 3077& 9232& 4616\\
2308& 1154& 577& 1732& 866& 433& 1300& 650& 325& 976\\
488& 244& 122& 61& 184& 92& 46& 23& 70& 35\\
106& 53& 160& 80& 40& 20& 10& 5& 16& 8\\
4& 2& 1& \\

608&&&&&&&&&\\
304& 152& 76& 38& 19& 58& 29& 88& 44& 22\\
11& 34& 17& 52& 26& 13& 40& 20& 10& 5\\
16& 8& 4& 2& 1& \\

609&&&&&&&&&\\
1828& 914& 457& 1372& 686& 343& 1030& 515& 1546& 773\\
2320& 1160& 580& 290& 145& 436& 218& 109& 328& 164\\
82& 41& 124& 62& 31& 94& 47& 142& 71& 214\\
107& 322& 161& 484& 242& 121& 364& 182& 91& 274\\
137& 412& 206& 103& 310& 155& 466& 233& 700& 350\\
175& 526& 263& 790& 395& 1186& 593& 1780& 890& 445\\
1336& 668& 334& 167& 502& 251& 754& 377& 1132& 566\\
283& 850& 425& 1276& 638& 319& 958& 479& 1438& 719\\
2158& 1079& 3238& 1619& 4858& 2429& 7288& 3644& 1822& 911\\
2734& 1367& 4102& 2051& 6154& 3077& 9232& 4616& 2308& 1154\\
577& 1732& 866& 433& 1300& 650& 325& 976& 488& 244\\
122& 61& 184& 92& 46& 23& 70& 35& 106& 53\\
160& 80& 40& 20& 10& 5& 16& 8& 4& 2\\
1& \\

610&&&&&&&&&\\
305& 916& 458& 229& 688& 344& 172& 86& 43& 130\\
65& 196& 98& 49& 148& 74& 37& 112& 56& 28\\
14& 7& 22& 11& 34& 17& 52& 26& 13& 40\\
20& 10& 5& 16& 8& 4& 2& 1& \\

611&&&&&&&&&\\
1834& 917& 2752& 1376& 688& 344& 172& 86& 43& 130\\
65& 196& 98& 49& 148& 74& 37& 112& 56& 28\\
14& 7& 22& 11& 34& 17& 52& 26& 13& 40\\
20& 10& 5& 16& 8& 4& 2& 1& \\

612&&&&&&&&&\\
306& 153& 460& 230& 115& 346& 173& 520& 260& 130\\
65& 196& 98& 49& 148& 74& 37& 112& 56& 28\\
14& 7& 22& 11& 34& 17& 52& 26& 13& 40\\
20& 10& 5& 16& 8& 4& 2& 1& \\

613&&&&&&&&&\\
1840& 920& 460& 230& 115& 346& 173& 520& 260& 130\\
65& 196& 98& 49& 148& 74& 37& 112& 56& 28\\
14& 7& 22& 11& 34& 17& 52& 26& 13& 40\\
20& 10& 5& 16& 8& 4& 2& 1& \\

614&&&&&&&&&\\
307& 922& 461& 1384& 692& 346& 173& 520& 260& 130\\
65& 196& 98& 49& 148& 74& 37& 112& 56& 28\\
14& 7& 22& 11& 34& 17& 52& 26& 13& 40\\
20& 10& 5& 16& 8& 4& 2& 1& \\

615&&&&&&&&&\\
1846& 923& 2770& 1385& 4156& 2078& 1039& 3118& 1559& 4678\\
2339& 7018& 3509& 10528& 5264& 2632& 1316& 658& 329& 988\\
494& 247& 742& 371& 1114& 557& 1672& 836& 418& 209\\
628& 314& 157& 472& 236& 118& 59& 178& 89& 268\\
134& 67& 202& 101& 304& 152& 76& 38& 19& 58\\
29& 88& 44& 22& 11& 34& 17& 52& 26& 13\\
40& 20& 10& 5& 16& 8& 4& 2& 1& \\

616&&&&&&&&&\\
308& 154& 77& 232& 116& 58& 29& 88& 44& 22\\
11& 34& 17& 52& 26& 13& 40& 20& 10& 5\\
16& 8& 4& 2& 1& \\

617&&&&&&&&&\\
1852& 926& 463& 1390& 695& 2086& 1043& 3130& 1565& 4696\\
2348& 1174& 587& 1762& 881& 2644& 1322& 661& 1984& 992\\
496& 248& 124& 62& 31& 94& 47& 142& 71& 214\\
107& 322& 161& 484& 242& 121& 364& 182& 91& 274\\
137& 412& 206& 103& 310& 155& 466& 233& 700& 350\\
175& 526& 263& 790& 395& 1186& 593& 1780& 890& 445\\
1336& 668& 334& 167& 502& 251& 754& 377& 1132& 566\\
283& 850& 425& 1276& 638& 319& 958& 479& 1438& 719\\
2158& 1079& 3238& 1619& 4858& 2429& 7288& 3644& 1822& 911\\
2734& 1367& 4102& 2051& 6154& 3077& 9232& 4616& 2308& 1154\\
577& 1732& 866& 433& 1300& 650& 325& 976& 488& 244\\
122& 61& 184& 92& 46& 23& 70& 35& 106& 53\\
160& 80& 40& 20& 10& 5& 16& 8& 4& 2\\
1& \\

618&&&&&&&&&\\
309& 928& 464& 232& 116& 58& 29& 88& 44& 22\\
11& 34& 17& 52& 26& 13& 40& 20& 10& 5\\
16& 8& 4& 2& 1& \\

619&&&&&&&&&\\
1858& 929& 2788& 1394& 697& 2092& 1046& 523& 1570& 785\\
2356& 1178& 589& 1768& 884& 442& 221& 664& 332& 166\\
83& 250& 125& 376& 188& 94& 47& 142& 71& 214\\
107& 322& 161& 484& 242& 121& 364& 182& 91& 274\\
137& 412& 206& 103& 310& 155& 466& 233& 700& 350\\
175& 526& 263& 790& 395& 1186& 593& 1780& 890& 445\\
1336& 668& 334& 167& 502& 251& 754& 377& 1132& 566\\
283& 850& 425& 1276& 638& 319& 958& 479& 1438& 719\\
2158& 1079& 3238& 1619& 4858& 2429& 7288& 3644& 1822& 911\\
2734& 1367& 4102& 2051& 6154& 3077& 9232& 4616& 2308& 1154\\
577& 1732& 866& 433& 1300& 650& 325& 976& 488& 244\\
122& 61& 184& 92& 46& 23& 70& 35& 106& 53\\
160& 80& 40& 20& 10& 5& 16& 8& 4& 2\\
1& \\

620&&&&&&&&&\\
310& 155& 466& 233& 700& 350& 175& 526& 263& 790\\
395& 1186& 593& 1780& 890& 445& 1336& 668& 334& 167\\
502& 251& 754& 377& 1132& 566& 283& 850& 425& 1276\\
638& 319& 958& 479& 1438& 719& 2158& 1079& 3238& 1619\\
4858& 2429& 7288& 3644& 1822& 911& 2734& 1367& 4102& 2051\\
6154& 3077& 9232& 4616& 2308& 1154& 577& 1732& 866& 433\\
1300& 650& 325& 976& 488& 244& 122& 61& 184& 92\\
46& 23& 70& 35& 106& 53& 160& 80& 40& 20\\
10& 5& 16& 8& 4& 2& 1& \\

621&&&&&&&&&\\
1864& 932& 466& 233& 700& 350& 175& 526& 263& 790\\
395& 1186& 593& 1780& 890& 445& 1336& 668& 334& 167\\
502& 251& 754& 377& 1132& 566& 283& 850& 425& 1276\\
638& 319& 958& 479& 1438& 719& 2158& 1079& 3238& 1619\\
4858& 2429& 7288& 3644& 1822& 911& 2734& 1367& 4102& 2051\\
6154& 3077& 9232& 4616& 2308& 1154& 577& 1732& 866& 433\\
1300& 650& 325& 976& 488& 244& 122& 61& 184& 92\\
46& 23& 70& 35& 106& 53& 160& 80& 40& 20\\
10& 5& 16& 8& 4& 2& 1& \\

622&&&&&&&&&\\
311& 934& 467& 1402& 701& 2104& 1052& 526& 263& 790\\
395& 1186& 593& 1780& 890& 445& 1336& 668& 334& 167\\
502& 251& 754& 377& 1132& 566& 283& 850& 425& 1276\\
638& 319& 958& 479& 1438& 719& 2158& 1079& 3238& 1619\\
4858& 2429& 7288& 3644& 1822& 911& 2734& 1367& 4102& 2051\\
6154& 3077& 9232& 4616& 2308& 1154& 577& 1732& 866& 433\\
1300& 650& 325& 976& 488& 244& 122& 61& 184& 92\\
46& 23& 70& 35& 106& 53& 160& 80& 40& 20\\
10& 5& 16& 8& 4& 2& 1& \\

623&&&&&&&&&\\
1870& 935& 2806& 1403& 4210& 2105& 6316& 3158& 1579& 4738\\
2369& 7108& 3554& 1777& 5332& 2666& 1333& 4000& 2000& 1000\\
500& 250& 125& 376& 188& 94& 47& 142& 71& 214\\
107& 322& 161& 484& 242& 121& 364& 182& 91& 274\\
137& 412& 206& 103& 310& 155& 466& 233& 700& 350\\
175& 526& 263& 790& 395& 1186& 593& 1780& 890& 445\\
1336& 668& 334& 167& 502& 251& 754& 377& 1132& 566\\
283& 850& 425& 1276& 638& 319& 958& 479& 1438& 719\\
2158& 1079& 3238& 1619& 4858& 2429& 7288& 3644& 1822& 911\\
2734& 1367& 4102& 2051& 6154& 3077& 9232& 4616& 2308& 1154\\
577& 1732& 866& 433& 1300& 650& 325& 976& 488& 244\\
122& 61& 184& 92& 46& 23& 70& 35& 106& 53\\
160& 80& 40& 20& 10& 5& 16& 8& 4& 2\\
1& \\

624&&&&&&&&&\\
312& 156& 78& 39& 118& 59& 178& 89& 268& 134\\
67& 202& 101& 304& 152& 76& 38& 19& 58& 29\\
88& 44& 22& 11& 34& 17& 52& 26& 13& 40\\
20& 10& 5& 16& 8& 4& 2& 1& \\

625&&&&&&&&&\\
1876& 938& 469& 1408& 704& 352& 176& 88& 44& 22\\
11& 34& 17& 52& 26& 13& 40& 20& 10& 5\\
16& 8& 4& 2& 1& \\

626&&&&&&&&&\\
313& 940& 470& 235& 706& 353& 1060& 530& 265& 796\\
398& 199& 598& 299& 898& 449& 1348& 674& 337& 1012\\
506& 253& 760& 380& 190& 95& 286& 143& 430& 215\\
646& 323& 970& 485& 1456& 728& 364& 182& 91& 274\\
137& 412& 206& 103& 310& 155& 466& 233& 700& 350\\
175& 526& 263& 790& 395& 1186& 593& 1780& 890& 445\\
1336& 668& 334& 167& 502& 251& 754& 377& 1132& 566\\
283& 850& 425& 1276& 638& 319& 958& 479& 1438& 719\\
2158& 1079& 3238& 1619& 4858& 2429& 7288& 3644& 1822& 911\\
2734& 1367& 4102& 2051& 6154& 3077& 9232& 4616& 2308& 1154\\
577& 1732& 866& 433& 1300& 650& 325& 976& 488& 244\\
122& 61& 184& 92& 46& 23& 70& 35& 106& 53\\
160& 80& 40& 20& 10& 5& 16& 8& 4& 2\\
1& \\

627&&&&&&&&&\\
1882& 941& 2824& 1412& 706& 353& 1060& 530& 265& 796\\
398& 199& 598& 299& 898& 449& 1348& 674& 337& 1012\\
506& 253& 760& 380& 190& 95& 286& 143& 430& 215\\
646& 323& 970& 485& 1456& 728& 364& 182& 91& 274\\
137& 412& 206& 103& 310& 155& 466& 233& 700& 350\\
175& 526& 263& 790& 395& 1186& 593& 1780& 890& 445\\
1336& 668& 334& 167& 502& 251& 754& 377& 1132& 566\\
283& 850& 425& 1276& 638& 319& 958& 479& 1438& 719\\
2158& 1079& 3238& 1619& 4858& 2429& 7288& 3644& 1822& 911\\
2734& 1367& 4102& 2051& 6154& 3077& 9232& 4616& 2308& 1154\\
577& 1732& 866& 433& 1300& 650& 325& 976& 488& 244\\
122& 61& 184& 92& 46& 23& 70& 35& 106& 53\\
160& 80& 40& 20& 10& 5& 16& 8& 4& 2\\
1& \\

628&&&&&&&&&\\
314& 157& 472& 236& 118& 59& 178& 89& 268& 134\\
67& 202& 101& 304& 152& 76& 38& 19& 58& 29\\
88& 44& 22& 11& 34& 17& 52& 26& 13& 40\\
20& 10& 5& 16& 8& 4& 2& 1& \\

629&&&&&&&&&\\
1888& 944& 472& 236& 118& 59& 178& 89& 268& 134\\
67& 202& 101& 304& 152& 76& 38& 19& 58& 29\\
88& 44& 22& 11& 34& 17& 52& 26& 13& 40\\
20& 10& 5& 16& 8& 4& 2& 1& \\

630&&&&&&&&&\\
315& 946& 473& 1420& 710& 355& 1066& 533& 1600& 800\\
400& 200& 100& 50& 25& 76& 38& 19& 58& 29\\
88& 44& 22& 11& 34& 17& 52& 26& 13& 40\\
20& 10& 5& 16& 8& 4& 2& 1& \\

631&&&&&&&&&\\
1894& 947& 2842& 1421& 4264& 2132& 1066& 533& 1600& 800\\
400& 200& 100& 50& 25& 76& 38& 19& 58& 29\\
88& 44& 22& 11& 34& 17& 52& 26& 13& 40\\
20& 10& 5& 16& 8& 4& 2& 1& \\

632&&&&&&&&&\\
316& 158& 79& 238& 119& 358& 179& 538& 269& 808\\
404& 202& 101& 304& 152& 76& 38& 19& 58& 29\\
88& 44& 22& 11& 34& 17& 52& 26& 13& 40\\
20& 10& 5& 16& 8& 4& 2& 1& \\

633&&&&&&&&&\\
1900& 950& 475& 1426& 713& 2140& 1070& 535& 1606& 803\\
2410& 1205& 3616& 1808& 904& 452& 226& 113& 340& 170\\
85& 256& 128& 64& 32& 16& 8& 4& 2& 1\\

634&&&&&&&&&\\
317& 952& 476& 238& 119& 358& 179& 538& 269& 808\\
404& 202& 101& 304& 152& 76& 38& 19& 58& 29\\
88& 44& 22& 11& 34& 17& 52& 26& 13& 40\\
20& 10& 5& 16& 8& 4& 2& 1& \\

635&&&&&&&&&\\
1906& 953& 2860& 1430& 715& 2146& 1073& 3220& 1610& 805\\
2416& 1208& 604& 302& 151& 454& 227& 682& 341& 1024\\
512& 256& 128& 64& 32& 16& 8& 4& 2& 1\\

636&&&&&&&&&\\
318& 159& 478& 239& 718& 359& 1078& 539& 1618& 809\\
2428& 1214& 607& 1822& 911& 2734& 1367& 4102& 2051& 6154\\
3077& 9232& 4616& 2308& 1154& 577& 1732& 866& 433& 1300\\
650& 325& 976& 488& 244& 122& 61& 184& 92& 46\\
23& 70& 35& 106& 53& 160& 80& 40& 20& 10\\
5& 16& 8& 4& 2& 1& \\

637&&&&&&&&&\\
1912& 956& 478& 239& 718& 359& 1078& 539& 1618& 809\\
2428& 1214& 607& 1822& 911& 2734& 1367& 4102& 2051& 6154\\
3077& 9232& 4616& 2308& 1154& 577& 1732& 866& 433& 1300\\
650& 325& 976& 488& 244& 122& 61& 184& 92& 46\\
23& 70& 35& 106& 53& 160& 80& 40& 20& 10\\
5& 16& 8& 4& 2& 1& \\

638&&&&&&&&&\\
319& 958& 479& 1438& 719& 2158& 1079& 3238& 1619& 4858\\
2429& 7288& 3644& 1822& 911& 2734& 1367& 4102& 2051& 6154\\
3077& 9232& 4616& 2308& 1154& 577& 1732& 866& 433& 1300\\
650& 325& 976& 488& 244& 122& 61& 184& 92& 46\\
23& 70& 35& 106& 53& 160& 80& 40& 20& 10\\
5& 16& 8& 4& 2& 1& \\

639&&&&&&&&&\\
1918& 959& 2878& 1439& 4318& 2159& 6478& 3239& 9718& 4859\\
14578& 7289& 21868& 10934& 5467& 16402& 8201& 24604& 12302& 6151\\
18454& 9227& 27682& 13841& 41524& 20762& 10381& 31144& 15572& 7786\\
3893& 11680& 5840& 2920& 1460& 730& 365& 1096& 548& 274\\
137& 412& 206& 103& 310& 155& 466& 233& 700& 350\\
175& 526& 263& 790& 395& 1186& 593& 1780& 890& 445\\
1336& 668& 334& 167& 502& 251& 754& 377& 1132& 566\\
283& 850& 425& 1276& 638& 319& 958& 479& 1438& 719\\
2158& 1079& 3238& 1619& 4858& 2429& 7288& 3644& 1822& 911\\
2734& 1367& 4102& 2051& 6154& 3077& 9232& 4616& 2308& 1154\\
577& 1732& 866& 433& 1300& 650& 325& 976& 488& 244\\
122& 61& 184& 92& 46& 23& 70& 35& 106& 53\\
160& 80& 40& 20& 10& 5& 16& 8& 4& 2\\
1& \\

640&&&&&&&&&\\
320& 160& 80& 40& 20& 10& 5& 16& 8& 4\\
2& 1& \\

641&&&&&&&&&\\
1924& 962& 481& 1444& 722& 361& 1084& 542& 271& 814\\
407& 1222& 611& 1834& 917& 2752& 1376& 688& 344& 172\\
86& 43& 130& 65& 196& 98& 49& 148& 74& 37\\
112& 56& 28& 14& 7& 22& 11& 34& 17& 52\\
26& 13& 40& 20& 10& 5& 16& 8& 4& 2\\
1& \\

642&&&&&&&&&\\
321& 964& 482& 241& 724& 362& 181& 544& 272& 136\\
68& 34& 17& 52& 26& 13& 40& 20& 10& 5\\
16& 8& 4& 2& 1& \\

643&&&&&&&&&\\
1930& 965& 2896& 1448& 724& 362& 181& 544& 272& 136\\
68& 34& 17& 52& 26& 13& 40& 20& 10& 5\\
16& 8& 4& 2& 1& \\

644&&&&&&&&&\\
322& 161& 484& 242& 121& 364& 182& 91& 274& 137\\
412& 206& 103& 310& 155& 466& 233& 700& 350& 175\\
526& 263& 790& 395& 1186& 593& 1780& 890& 445& 1336\\
668& 334& 167& 502& 251& 754& 377& 1132& 566& 283\\
850& 425& 1276& 638& 319& 958& 479& 1438& 719& 2158\\
1079& 3238& 1619& 4858& 2429& 7288& 3644& 1822& 911& 2734\\
1367& 4102& 2051& 6154& 3077& 9232& 4616& 2308& 1154& 577\\
1732& 866& 433& 1300& 650& 325& 976& 488& 244& 122\\
61& 184& 92& 46& 23& 70& 35& 106& 53& 160\\
80& 40& 20& 10& 5& 16& 8& 4& 2& 1\\

645&&&&&&&&&\\
1936& 968& 484& 242& 121& 364& 182& 91& 274& 137\\
412& 206& 103& 310& 155& 466& 233& 700& 350& 175\\
526& 263& 790& 395& 1186& 593& 1780& 890& 445& 1336\\
668& 334& 167& 502& 251& 754& 377& 1132& 566& 283\\
850& 425& 1276& 638& 319& 958& 479& 1438& 719& 2158\\
1079& 3238& 1619& 4858& 2429& 7288& 3644& 1822& 911& 2734\\
1367& 4102& 2051& 6154& 3077& 9232& 4616& 2308& 1154& 577\\
1732& 866& 433& 1300& 650& 325& 976& 488& 244& 122\\
61& 184& 92& 46& 23& 70& 35& 106& 53& 160\\
80& 40& 20& 10& 5& 16& 8& 4& 2& 1\\

646&&&&&&&&&\\
323& 970& 485& 1456& 728& 364& 182& 91& 274& 137\\
412& 206& 103& 310& 155& 466& 233& 700& 350& 175\\
526& 263& 790& 395& 1186& 593& 1780& 890& 445& 1336\\
668& 334& 167& 502& 251& 754& 377& 1132& 566& 283\\
850& 425& 1276& 638& 319& 958& 479& 1438& 719& 2158\\
1079& 3238& 1619& 4858& 2429& 7288& 3644& 1822& 911& 2734\\
1367& 4102& 2051& 6154& 3077& 9232& 4616& 2308& 1154& 577\\
1732& 866& 433& 1300& 650& 325& 976& 488& 244& 122\\
61& 184& 92& 46& 23& 70& 35& 106& 53& 160\\
80& 40& 20& 10& 5& 16& 8& 4& 2& 1\\

647&&&&&&&&&\\
1942& 971& 2914& 1457& 4372& 2186& 1093& 3280& 1640& 820\\
410& 205& 616& 308& 154& 77& 232& 116& 58& 29\\
88& 44& 22& 11& 34& 17& 52& 26& 13& 40\\
20& 10& 5& 16& 8& 4& 2& 1& \\

648&&&&&&&&&\\
324& 162& 81& 244& 122& 61& 184& 92& 46& 23\\
70& 35& 106& 53& 160& 80& 40& 20& 10& 5\\
16& 8& 4& 2& 1& \\

649&&&&&&&&&\\
1948& 974& 487& 1462& 731& 2194& 1097& 3292& 1646& 823\\
2470& 1235& 3706& 1853& 5560& 2780& 1390& 695& 2086& 1043\\
3130& 1565& 4696& 2348& 1174& 587& 1762& 881& 2644& 1322\\
661& 1984& 992& 496& 248& 124& 62& 31& 94& 47\\
142& 71& 214& 107& 322& 161& 484& 242& 121& 364\\
182& 91& 274& 137& 412& 206& 103& 310& 155& 466\\
233& 700& 350& 175& 526& 263& 790& 395& 1186& 593\\
1780& 890& 445& 1336& 668& 334& 167& 502& 251& 754\\
377& 1132& 566& 283& 850& 425& 1276& 638& 319& 958\\
479& 1438& 719& 2158& 1079& 3238& 1619& 4858& 2429& 7288\\
3644& 1822& 911& 2734& 1367& 4102& 2051& 6154& 3077& 9232\\
4616& 2308& 1154& 577& 1732& 866& 433& 1300& 650& 325\\
976& 488& 244& 122& 61& 184& 92& 46& 23& 70\\
35& 106& 53& 160& 80& 40& 20& 10& 5& 16\\
8& 4& 2& 1& \\

650&&&&&&&&&\\
325& 976& 488& 244& 122& 61& 184& 92& 46& 23\\
70& 35& 106& 53& 160& 80& 40& 20& 10& 5\\
16& 8& 4& 2& 1& \\

651&&&&&&&&&\\
1954& 977& 2932& 1466& 733& 2200& 1100& 550& 275& 826\\
413& 1240& 620& 310& 155& 466& 233& 700& 350& 175\\
526& 263& 790& 395& 1186& 593& 1780& 890& 445& 1336\\
668& 334& 167& 502& 251& 754& 377& 1132& 566& 283\\
850& 425& 1276& 638& 319& 958& 479& 1438& 719& 2158\\
1079& 3238& 1619& 4858& 2429& 7288& 3644& 1822& 911& 2734\\
1367& 4102& 2051& 6154& 3077& 9232& 4616& 2308& 1154& 577\\
1732& 866& 433& 1300& 650& 325& 976& 488& 244& 122\\
61& 184& 92& 46& 23& 70& 35& 106& 53& 160\\
80& 40& 20& 10& 5& 16& 8& 4& 2& 1\\

652&&&&&&&&&\\
326& 163& 490& 245& 736& 368& 184& 92& 46& 23\\
70& 35& 106& 53& 160& 80& 40& 20& 10& 5\\
16& 8& 4& 2& 1& \\

653&&&&&&&&&\\
1960& 980& 490& 245& 736& 368& 184& 92& 46& 23\\
70& 35& 106& 53& 160& 80& 40& 20& 10& 5\\
16& 8& 4& 2& 1& \\

654&&&&&&&&&\\
327& 982& 491& 1474& 737& 2212& 1106& 553& 1660& 830\\
415& 1246& 623& 1870& 935& 2806& 1403& 4210& 2105& 6316\\
3158& 1579& 4738& 2369& 7108& 3554& 1777& 5332& 2666& 1333\\
4000& 2000& 1000& 500& 250& 125& 376& 188& 94& 47\\
142& 71& 214& 107& 322& 161& 484& 242& 121& 364\\
182& 91& 274& 137& 412& 206& 103& 310& 155& 466\\
233& 700& 350& 175& 526& 263& 790& 395& 1186& 593\\
1780& 890& 445& 1336& 668& 334& 167& 502& 251& 754\\
377& 1132& 566& 283& 850& 425& 1276& 638& 319& 958\\
479& 1438& 719& 2158& 1079& 3238& 1619& 4858& 2429& 7288\\
3644& 1822& 911& 2734& 1367& 4102& 2051& 6154& 3077& 9232\\
4616& 2308& 1154& 577& 1732& 866& 433& 1300& 650& 325\\
976& 488& 244& 122& 61& 184& 92& 46& 23& 70\\
35& 106& 53& 160& 80& 40& 20& 10& 5& 16\\
8& 4& 2& 1& \\

655&&&&&&&&&\\
1966& 983& 2950& 1475& 4426& 2213& 6640& 3320& 1660& 830\\
415& 1246& 623& 1870& 935& 2806& 1403& 4210& 2105& 6316\\
3158& 1579& 4738& 2369& 7108& 3554& 1777& 5332& 2666& 1333\\
4000& 2000& 1000& 500& 250& 125& 376& 188& 94& 47\\
142& 71& 214& 107& 322& 161& 484& 242& 121& 364\\
182& 91& 274& 137& 412& 206& 103& 310& 155& 466\\
233& 700& 350& 175& 526& 263& 790& 395& 1186& 593\\
1780& 890& 445& 1336& 668& 334& 167& 502& 251& 754\\
377& 1132& 566& 283& 850& 425& 1276& 638& 319& 958\\
479& 1438& 719& 2158& 1079& 3238& 1619& 4858& 2429& 7288\\
3644& 1822& 911& 2734& 1367& 4102& 2051& 6154& 3077& 9232\\
4616& 2308& 1154& 577& 1732& 866& 433& 1300& 650& 325\\
976& 488& 244& 122& 61& 184& 92& 46& 23& 70\\
35& 106& 53& 160& 80& 40& 20& 10& 5& 16\\
8& 4& 2& 1& \\

656&&&&&&&&&\\
328& 164& 82& 41& 124& 62& 31& 94& 47& 142\\
71& 214& 107& 322& 161& 484& 242& 121& 364& 182\\
91& 274& 137& 412& 206& 103& 310& 155& 466& 233\\
700& 350& 175& 526& 263& 790& 395& 1186& 593& 1780\\
890& 445& 1336& 668& 334& 167& 502& 251& 754& 377\\
1132& 566& 283& 850& 425& 1276& 638& 319& 958& 479\\
1438& 719& 2158& 1079& 3238& 1619& 4858& 2429& 7288& 3644\\
1822& 911& 2734& 1367& 4102& 2051& 6154& 3077& 9232& 4616\\
2308& 1154& 577& 1732& 866& 433& 1300& 650& 325& 976\\
488& 244& 122& 61& 184& 92& 46& 23& 70& 35\\
106& 53& 160& 80& 40& 20& 10& 5& 16& 8\\
4& 2& 1& \\

657&&&&&&&&&\\
1972& 986& 493& 1480& 740& 370& 185& 556& 278& 139\\
418& 209& 628& 314& 157& 472& 236& 118& 59& 178\\
89& 268& 134& 67& 202& 101& 304& 152& 76& 38\\
19& 58& 29& 88& 44& 22& 11& 34& 17& 52\\
26& 13& 40& 20& 10& 5& 16& 8& 4& 2\\
1& \\

658&&&&&&&&&\\
329& 988& 494& 247& 742& 371& 1114& 557& 1672& 836\\
418& 209& 628& 314& 157& 472& 236& 118& 59& 178\\
89& 268& 134& 67& 202& 101& 304& 152& 76& 38\\
19& 58& 29& 88& 44& 22& 11& 34& 17& 52\\
26& 13& 40& 20& 10& 5& 16& 8& 4& 2\\
1& \\

659&&&&&&&&&\\
1978& 989& 2968& 1484& 742& 371& 1114& 557& 1672& 836\\
418& 209& 628& 314& 157& 472& 236& 118& 59& 178\\
89& 268& 134& 67& 202& 101& 304& 152& 76& 38\\
19& 58& 29& 88& 44& 22& 11& 34& 17& 52\\
26& 13& 40& 20& 10& 5& 16& 8& 4& 2\\
1& \\

660&&&&&&&&&\\
330& 165& 496& 248& 124& 62& 31& 94& 47& 142\\
71& 214& 107& 322& 161& 484& 242& 121& 364& 182\\
91& 274& 137& 412& 206& 103& 310& 155& 466& 233\\
700& 350& 175& 526& 263& 790& 395& 1186& 593& 1780\\
890& 445& 1336& 668& 334& 167& 502& 251& 754& 377\\
1132& 566& 283& 850& 425& 1276& 638& 319& 958& 479\\
1438& 719& 2158& 1079& 3238& 1619& 4858& 2429& 7288& 3644\\
1822& 911& 2734& 1367& 4102& 2051& 6154& 3077& 9232& 4616\\
2308& 1154& 577& 1732& 866& 433& 1300& 650& 325& 976\\
488& 244& 122& 61& 184& 92& 46& 23& 70& 35\\
106& 53& 160& 80& 40& 20& 10& 5& 16& 8\\
4& 2& 1& \\

661&&&&&&&&&\\
1984& 992& 496& 248& 124& 62& 31& 94& 47& 142\\
71& 214& 107& 322& 161& 484& 242& 121& 364& 182\\
91& 274& 137& 412& 206& 103& 310& 155& 466& 233\\
700& 350& 175& 526& 263& 790& 395& 1186& 593& 1780\\
890& 445& 1336& 668& 334& 167& 502& 251& 754& 377\\
1132& 566& 283& 850& 425& 1276& 638& 319& 958& 479\\
1438& 719& 2158& 1079& 3238& 1619& 4858& 2429& 7288& 3644\\
1822& 911& 2734& 1367& 4102& 2051& 6154& 3077& 9232& 4616\\
2308& 1154& 577& 1732& 866& 433& 1300& 650& 325& 976\\
488& 244& 122& 61& 184& 92& 46& 23& 70& 35\\
106& 53& 160& 80& 40& 20& 10& 5& 16& 8\\
4& 2& 1& \\

662&&&&&&&&&\\
331& 994& 497& 1492& 746& 373& 1120& 560& 280& 140\\
70& 35& 106& 53& 160& 80& 40& 20& 10& 5\\
16& 8& 4& 2& 1& \\

663&&&&&&&&&\\
1990& 995& 2986& 1493& 4480& 2240& 1120& 560& 280& 140\\
70& 35& 106& 53& 160& 80& 40& 20& 10& 5\\
16& 8& 4& 2& 1& \\

664&&&&&&&&&\\
332& 166& 83& 250& 125& 376& 188& 94& 47& 142\\
71& 214& 107& 322& 161& 484& 242& 121& 364& 182\\
91& 274& 137& 412& 206& 103& 310& 155& 466& 233\\
700& 350& 175& 526& 263& 790& 395& 1186& 593& 1780\\
890& 445& 1336& 668& 334& 167& 502& 251& 754& 377\\
1132& 566& 283& 850& 425& 1276& 638& 319& 958& 479\\
1438& 719& 2158& 1079& 3238& 1619& 4858& 2429& 7288& 3644\\
1822& 911& 2734& 1367& 4102& 2051& 6154& 3077& 9232& 4616\\
2308& 1154& 577& 1732& 866& 433& 1300& 650& 325& 976\\
488& 244& 122& 61& 184& 92& 46& 23& 70& 35\\
106& 53& 160& 80& 40& 20& 10& 5& 16& 8\\
4& 2& 1& \\

665&&&&&&&&&\\
1996& 998& 499& 1498& 749& 2248& 1124& 562& 281& 844\\
422& 211& 634& 317& 952& 476& 238& 119& 358& 179\\
538& 269& 808& 404& 202& 101& 304& 152& 76& 38\\
19& 58& 29& 88& 44& 22& 11& 34& 17& 52\\
26& 13& 40& 20& 10& 5& 16& 8& 4& 2\\
1& \\

666&&&&&&&&&\\
333& 1000& 500& 250& 125& 376& 188& 94& 47& 142\\
71& 214& 107& 322& 161& 484& 242& 121& 364& 182\\
91& 274& 137& 412& 206& 103& 310& 155& 466& 233\\
700& 350& 175& 526& 263& 790& 395& 1186& 593& 1780\\
890& 445& 1336& 668& 334& 167& 502& 251& 754& 377\\
1132& 566& 283& 850& 425& 1276& 638& 319& 958& 479\\
1438& 719& 2158& 1079& 3238& 1619& 4858& 2429& 7288& 3644\\
1822& 911& 2734& 1367& 4102& 2051& 6154& 3077& 9232& 4616\\
2308& 1154& 577& 1732& 866& 433& 1300& 650& 325& 976\\
488& 244& 122& 61& 184& 92& 46& 23& 70& 35\\
106& 53& 160& 80& 40& 20& 10& 5& 16& 8\\
4& 2& 1& \\

667&&&&&&&&&\\
2002& 1001& 3004& 1502& 751& 2254& 1127& 3382& 1691& 5074\\
2537& 7612& 3806& 1903& 5710& 2855& 8566& 4283& 12850& 6425\\
19276& 9638& 4819& 14458& 7229& 21688& 10844& 5422& 2711& 8134\\
4067& 12202& 6101& 18304& 9152& 4576& 2288& 1144& 572& 286\\
143& 430& 215& 646& 323& 970& 485& 1456& 728& 364\\
182& 91& 274& 137& 412& 206& 103& 310& 155& 466\\
233& 700& 350& 175& 526& 263& 790& 395& 1186& 593\\
1780& 890& 445& 1336& 668& 334& 167& 502& 251& 754\\
377& 1132& 566& 283& 850& 425& 1276& 638& 319& 958\\
479& 1438& 719& 2158& 1079& 3238& 1619& 4858& 2429& 7288\\
3644& 1822& 911& 2734& 1367& 4102& 2051& 6154& 3077& 9232\\
4616& 2308& 1154& 577& 1732& 866& 433& 1300& 650& 325\\
976& 488& 244& 122& 61& 184& 92& 46& 23& 70\\
35& 106& 53& 160& 80& 40& 20& 10& 5& 16\\
8& 4& 2& 1& \\

668&&&&&&&&&\\
334& 167& 502& 251& 754& 377& 1132& 566& 283& 850\\
425& 1276& 638& 319& 958& 479& 1438& 719& 2158& 1079\\
3238& 1619& 4858& 2429& 7288& 3644& 1822& 911& 2734& 1367\\
4102& 2051& 6154& 3077& 9232& 4616& 2308& 1154& 577& 1732\\
866& 433& 1300& 650& 325& 976& 488& 244& 122& 61\\
184& 92& 46& 23& 70& 35& 106& 53& 160& 80\\
40& 20& 10& 5& 16& 8& 4& 2& 1& \\

669&&&&&&&&&\\
2008& 1004& 502& 251& 754& 377& 1132& 566& 283& 850\\
425& 1276& 638& 319& 958& 479& 1438& 719& 2158& 1079\\
3238& 1619& 4858& 2429& 7288& 3644& 1822& 911& 2734& 1367\\
4102& 2051& 6154& 3077& 9232& 4616& 2308& 1154& 577& 1732\\
866& 433& 1300& 650& 325& 976& 488& 244& 122& 61\\
184& 92& 46& 23& 70& 35& 106& 53& 160& 80\\
40& 20& 10& 5& 16& 8& 4& 2& 1& \\

670&&&&&&&&&\\
335& 1006& 503& 1510& 755& 2266& 1133& 3400& 1700& 850\\
425& 1276& 638& 319& 958& 479& 1438& 719& 2158& 1079\\
3238& 1619& 4858& 2429& 7288& 3644& 1822& 911& 2734& 1367\\
4102& 2051& 6154& 3077& 9232& 4616& 2308& 1154& 577& 1732\\
866& 433& 1300& 650& 325& 976& 488& 244& 122& 61\\
184& 92& 46& 23& 70& 35& 106& 53& 160& 80\\
40& 20& 10& 5& 16& 8& 4& 2& 1& \\

671&&&&&&&&&\\
2014& 1007& 3022& 1511& 4534& 2267& 6802& 3401& 10204& 5102\\
2551& 7654& 3827& 11482& 5741& 17224& 8612& 4306& 2153& 6460\\
3230& 1615& 4846& 2423& 7270& 3635& 10906& 5453& 16360& 8180\\
4090& 2045& 6136& 3068& 1534& 767& 2302& 1151& 3454& 1727\\
5182& 2591& 7774& 3887& 11662& 5831& 17494& 8747& 26242& 13121\\
39364& 19682& 9841& 29524& 14762& 7381& 22144& 11072& 5536& 2768\\
1384& 692& 346& 173& 520& 260& 130& 65& 196& 98\\
49& 148& 74& 37& 112& 56& 28& 14& 7& 22\\
11& 34& 17& 52& 26& 13& 40& 20& 10& 5\\
16& 8& 4& 2& 1& \\

672&&&&&&&&&\\
336& 168& 84& 42& 21& 64& 32& 16& 8& 4\\
2& 1& \\

673&&&&&&&&&\\
2020& 1010& 505& 1516& 758& 379& 1138& 569& 1708& 854\\
427& 1282& 641& 1924& 962& 481& 1444& 722& 361& 1084\\
542& 271& 814& 407& 1222& 611& 1834& 917& 2752& 1376\\
688& 344& 172& 86& 43& 130& 65& 196& 98& 49\\
148& 74& 37& 112& 56& 28& 14& 7& 22& 11\\
34& 17& 52& 26& 13& 40& 20& 10& 5& 16\\
8& 4& 2& 1& \\

674&&&&&&&&&\\
337& 1012& 506& 253& 760& 380& 190& 95& 286& 143\\
430& 215& 646& 323& 970& 485& 1456& 728& 364& 182\\
91& 274& 137& 412& 206& 103& 310& 155& 466& 233\\
700& 350& 175& 526& 263& 790& 395& 1186& 593& 1780\\
890& 445& 1336& 668& 334& 167& 502& 251& 754& 377\\
1132& 566& 283& 850& 425& 1276& 638& 319& 958& 479\\
1438& 719& 2158& 1079& 3238& 1619& 4858& 2429& 7288& 3644\\
1822& 911& 2734& 1367& 4102& 2051& 6154& 3077& 9232& 4616\\
2308& 1154& 577& 1732& 866& 433& 1300& 650& 325& 976\\
488& 244& 122& 61& 184& 92& 46& 23& 70& 35\\
106& 53& 160& 80& 40& 20& 10& 5& 16& 8\\
4& 2& 1& \\

675&&&&&&&&&\\
2026& 1013& 3040& 1520& 760& 380& 190& 95& 286& 143\\
430& 215& 646& 323& 970& 485& 1456& 728& 364& 182\\
91& 274& 137& 412& 206& 103& 310& 155& 466& 233\\
700& 350& 175& 526& 263& 790& 395& 1186& 593& 1780\\
890& 445& 1336& 668& 334& 167& 502& 251& 754& 377\\
1132& 566& 283& 850& 425& 1276& 638& 319& 958& 479\\
1438& 719& 2158& 1079& 3238& 1619& 4858& 2429& 7288& 3644\\
1822& 911& 2734& 1367& 4102& 2051& 6154& 3077& 9232& 4616\\
2308& 1154& 577& 1732& 866& 433& 1300& 650& 325& 976\\
488& 244& 122& 61& 184& 92& 46& 23& 70& 35\\
106& 53& 160& 80& 40& 20& 10& 5& 16& 8\\
4& 2& 1& \\

676&&&&&&&&&\\
338& 169& 508& 254& 127& 382& 191& 574& 287& 862\\
431& 1294& 647& 1942& 971& 2914& 1457& 4372& 2186& 1093\\
3280& 1640& 820& 410& 205& 616& 308& 154& 77& 232\\
116& 58& 29& 88& 44& 22& 11& 34& 17& 52\\
26& 13& 40& 20& 10& 5& 16& 8& 4& 2\\
1& \\

677&&&&&&&&&\\
2032& 1016& 508& 254& 127& 382& 191& 574& 287& 862\\
431& 1294& 647& 1942& 971& 2914& 1457& 4372& 2186& 1093\\
3280& 1640& 820& 410& 205& 616& 308& 154& 77& 232\\
116& 58& 29& 88& 44& 22& 11& 34& 17& 52\\
26& 13& 40& 20& 10& 5& 16& 8& 4& 2\\
1& \\

678&&&&&&&&&\\
339& 1018& 509& 1528& 764& 382& 191& 574& 287& 862\\
431& 1294& 647& 1942& 971& 2914& 1457& 4372& 2186& 1093\\
3280& 1640& 820& 410& 205& 616& 308& 154& 77& 232\\
116& 58& 29& 88& 44& 22& 11& 34& 17& 52\\
26& 13& 40& 20& 10& 5& 16& 8& 4& 2\\
1& \\

679&&&&&&&&&\\
2038& 1019& 3058& 1529& 4588& 2294& 1147& 3442& 1721& 5164\\
2582& 1291& 3874& 1937& 5812& 2906& 1453& 4360& 2180& 1090\\
545& 1636& 818& 409& 1228& 614& 307& 922& 461& 1384\\
692& 346& 173& 520& 260& 130& 65& 196& 98& 49\\
148& 74& 37& 112& 56& 28& 14& 7& 22& 11\\
34& 17& 52& 26& 13& 40& 20& 10& 5& 16\\
8& 4& 2& 1& \\

680&&&&&&&&&\\
340& 170& 85& 256& 128& 64& 32& 16& 8& 4\\
2& 1& \\

681&&&&&&&&&\\
2044& 1022& 511& 1534& 767& 2302& 1151& 3454& 1727& 5182\\
2591& 7774& 3887& 11662& 5831& 17494& 8747& 26242& 13121& 39364\\
19682& 9841& 29524& 14762& 7381& 22144& 11072& 5536& 2768& 1384\\
692& 346& 173& 520& 260& 130& 65& 196& 98& 49\\
148& 74& 37& 112& 56& 28& 14& 7& 22& 11\\
34& 17& 52& 26& 13& 40& 20& 10& 5& 16\\
8& 4& 2& 1& \\

682&&&&&&&&&\\
341& 1024& 512& 256& 128& 64& 32& 16& 8& 4\\
2& 1& \\

683&&&&&&&&&\\
2050& 1025& 3076& 1538& 769& 2308& 1154& 577& 1732& 866\\
433& 1300& 650& 325& 976& 488& 244& 122& 61& 184\\
92& 46& 23& 70& 35& 106& 53& 160& 80& 40\\
20& 10& 5& 16& 8& 4& 2& 1& \\

684&&&&&&&&&\\
342& 171& 514& 257& 772& 386& 193& 580& 290& 145\\
436& 218& 109& 328& 164& 82& 41& 124& 62& 31\\
94& 47& 142& 71& 214& 107& 322& 161& 484& 242\\
121& 364& 182& 91& 274& 137& 412& 206& 103& 310\\
155& 466& 233& 700& 350& 175& 526& 263& 790& 395\\
1186& 593& 1780& 890& 445& 1336& 668& 334& 167& 502\\
251& 754& 377& 1132& 566& 283& 850& 425& 1276& 638\\
319& 958& 479& 1438& 719& 2158& 1079& 3238& 1619& 4858\\
2429& 7288& 3644& 1822& 911& 2734& 1367& 4102& 2051& 6154\\
3077& 9232& 4616& 2308& 1154& 577& 1732& 866& 433& 1300\\
650& 325& 976& 488& 244& 122& 61& 184& 92& 46\\
23& 70& 35& 106& 53& 160& 80& 40& 20& 10\\
5& 16& 8& 4& 2& 1& \\

685&&&&&&&&&\\
2056& 1028& 514& 257& 772& 386& 193& 580& 290& 145\\
436& 218& 109& 328& 164& 82& 41& 124& 62& 31\\
94& 47& 142& 71& 214& 107& 322& 161& 484& 242\\
121& 364& 182& 91& 274& 137& 412& 206& 103& 310\\
155& 466& 233& 700& 350& 175& 526& 263& 790& 395\\
1186& 593& 1780& 890& 445& 1336& 668& 334& 167& 502\\
251& 754& 377& 1132& 566& 283& 850& 425& 1276& 638\\
319& 958& 479& 1438& 719& 2158& 1079& 3238& 1619& 4858\\
2429& 7288& 3644& 1822& 911& 2734& 1367& 4102& 2051& 6154\\
3077& 9232& 4616& 2308& 1154& 577& 1732& 866& 433& 1300\\
650& 325& 976& 488& 244& 122& 61& 184& 92& 46\\
23& 70& 35& 106& 53& 160& 80& 40& 20& 10\\
5& 16& 8& 4& 2& 1& \\

686&&&&&&&&&\\
343& 1030& 515& 1546& 773& 2320& 1160& 580& 290& 145\\
436& 218& 109& 328& 164& 82& 41& 124& 62& 31\\
94& 47& 142& 71& 214& 107& 322& 161& 484& 242\\
121& 364& 182& 91& 274& 137& 412& 206& 103& 310\\
155& 466& 233& 700& 350& 175& 526& 263& 790& 395\\
1186& 593& 1780& 890& 445& 1336& 668& 334& 167& 502\\
251& 754& 377& 1132& 566& 283& 850& 425& 1276& 638\\
319& 958& 479& 1438& 719& 2158& 1079& 3238& 1619& 4858\\
2429& 7288& 3644& 1822& 911& 2734& 1367& 4102& 2051& 6154\\
3077& 9232& 4616& 2308& 1154& 577& 1732& 866& 433& 1300\\
650& 325& 976& 488& 244& 122& 61& 184& 92& 46\\
23& 70& 35& 106& 53& 160& 80& 40& 20& 10\\
5& 16& 8& 4& 2& 1& \\

687&&&&&&&&&\\
2062& 1031& 3094& 1547& 4642& 2321& 6964& 3482& 1741& 5224\\
2612& 1306& 653& 1960& 980& 490& 245& 736& 368& 184\\
92& 46& 23& 70& 35& 106& 53& 160& 80& 40\\
20& 10& 5& 16& 8& 4& 2& 1& \\

688&&&&&&&&&\\
344& 172& 86& 43& 130& 65& 196& 98& 49& 148\\
74& 37& 112& 56& 28& 14& 7& 22& 11& 34\\
17& 52& 26& 13& 40& 20& 10& 5& 16& 8\\
4& 2& 1& \\

689&&&&&&&&&\\
2068& 1034& 517& 1552& 776& 388& 194& 97& 292& 146\\
73& 220& 110& 55& 166& 83& 250& 125& 376& 188\\
94& 47& 142& 71& 214& 107& 322& 161& 484& 242\\
121& 364& 182& 91& 274& 137& 412& 206& 103& 310\\
155& 466& 233& 700& 350& 175& 526& 263& 790& 395\\
1186& 593& 1780& 890& 445& 1336& 668& 334& 167& 502\\
251& 754& 377& 1132& 566& 283& 850& 425& 1276& 638\\
319& 958& 479& 1438& 719& 2158& 1079& 3238& 1619& 4858\\
2429& 7288& 3644& 1822& 911& 2734& 1367& 4102& 2051& 6154\\
3077& 9232& 4616& 2308& 1154& 577& 1732& 866& 433& 1300\\
650& 325& 976& 488& 244& 122& 61& 184& 92& 46\\
23& 70& 35& 106& 53& 160& 80& 40& 20& 10\\
5& 16& 8& 4& 2& 1& \\

690&&&&&&&&&\\
345& 1036& 518& 259& 778& 389& 1168& 584& 292& 146\\
73& 220& 110& 55& 166& 83& 250& 125& 376& 188\\
94& 47& 142& 71& 214& 107& 322& 161& 484& 242\\
121& 364& 182& 91& 274& 137& 412& 206& 103& 310\\
155& 466& 233& 700& 350& 175& 526& 263& 790& 395\\
1186& 593& 1780& 890& 445& 1336& 668& 334& 167& 502\\
251& 754& 377& 1132& 566& 283& 850& 425& 1276& 638\\
319& 958& 479& 1438& 719& 2158& 1079& 3238& 1619& 4858\\
2429& 7288& 3644& 1822& 911& 2734& 1367& 4102& 2051& 6154\\
3077& 9232& 4616& 2308& 1154& 577& 1732& 866& 433& 1300\\
650& 325& 976& 488& 244& 122& 61& 184& 92& 46\\
23& 70& 35& 106& 53& 160& 80& 40& 20& 10\\
5& 16& 8& 4& 2& 1& \\

691&&&&&&&&&\\
2074& 1037& 3112& 1556& 778& 389& 1168& 584& 292& 146\\
73& 220& 110& 55& 166& 83& 250& 125& 376& 188\\
94& 47& 142& 71& 214& 107& 322& 161& 484& 242\\
121& 364& 182& 91& 274& 137& 412& 206& 103& 310\\
155& 466& 233& 700& 350& 175& 526& 263& 790& 395\\
1186& 593& 1780& 890& 445& 1336& 668& 334& 167& 502\\
251& 754& 377& 1132& 566& 283& 850& 425& 1276& 638\\
319& 958& 479& 1438& 719& 2158& 1079& 3238& 1619& 4858\\
2429& 7288& 3644& 1822& 911& 2734& 1367& 4102& 2051& 6154\\
3077& 9232& 4616& 2308& 1154& 577& 1732& 866& 433& 1300\\
650& 325& 976& 488& 244& 122& 61& 184& 92& 46\\
23& 70& 35& 106& 53& 160& 80& 40& 20& 10\\
5& 16& 8& 4& 2& 1& \\

692&&&&&&&&&\\
346& 173& 520& 260& 130& 65& 196& 98& 49& 148\\
74& 37& 112& 56& 28& 14& 7& 22& 11& 34\\
17& 52& 26& 13& 40& 20& 10& 5& 16& 8\\
4& 2& 1& \\

693&&&&&&&&&\\
2080& 1040& 520& 260& 130& 65& 196& 98& 49& 148\\
74& 37& 112& 56& 28& 14& 7& 22& 11& 34\\
17& 52& 26& 13& 40& 20& 10& 5& 16& 8\\
4& 2& 1& \\

694&&&&&&&&&\\
347& 1042& 521& 1564& 782& 391& 1174& 587& 1762& 881\\
2644& 1322& 661& 1984& 992& 496& 248& 124& 62& 31\\
94& 47& 142& 71& 214& 107& 322& 161& 484& 242\\
121& 364& 182& 91& 274& 137& 412& 206& 103& 310\\
155& 466& 233& 700& 350& 175& 526& 263& 790& 395\\
1186& 593& 1780& 890& 445& 1336& 668& 334& 167& 502\\
251& 754& 377& 1132& 566& 283& 850& 425& 1276& 638\\
319& 958& 479& 1438& 719& 2158& 1079& 3238& 1619& 4858\\
2429& 7288& 3644& 1822& 911& 2734& 1367& 4102& 2051& 6154\\
3077& 9232& 4616& 2308& 1154& 577& 1732& 866& 433& 1300\\
650& 325& 976& 488& 244& 122& 61& 184& 92& 46\\
23& 70& 35& 106& 53& 160& 80& 40& 20& 10\\
5& 16& 8& 4& 2& 1& \\

695&&&&&&&&&\\
2086& 1043& 3130& 1565& 4696& 2348& 1174& 587& 1762& 881\\
2644& 1322& 661& 1984& 992& 496& 248& 124& 62& 31\\
94& 47& 142& 71& 214& 107& 322& 161& 484& 242\\
121& 364& 182& 91& 274& 137& 412& 206& 103& 310\\
155& 466& 233& 700& 350& 175& 526& 263& 790& 395\\
1186& 593& 1780& 890& 445& 1336& 668& 334& 167& 502\\
251& 754& 377& 1132& 566& 283& 850& 425& 1276& 638\\
319& 958& 479& 1438& 719& 2158& 1079& 3238& 1619& 4858\\
2429& 7288& 3644& 1822& 911& 2734& 1367& 4102& 2051& 6154\\
3077& 9232& 4616& 2308& 1154& 577& 1732& 866& 433& 1300\\
650& 325& 976& 488& 244& 122& 61& 184& 92& 46\\
23& 70& 35& 106& 53& 160& 80& 40& 20& 10\\
5& 16& 8& 4& 2& 1& \\

696&&&&&&&&&\\
348& 174& 87& 262& 131& 394& 197& 592& 296& 148\\
74& 37& 112& 56& 28& 14& 7& 22& 11& 34\\
17& 52& 26& 13& 40& 20& 10& 5& 16& 8\\
4& 2& 1& \\

697&&&&&&&&&\\
2092& 1046& 523& 1570& 785& 2356& 1178& 589& 1768& 884\\
442& 221& 664& 332& 166& 83& 250& 125& 376& 188\\
94& 47& 142& 71& 214& 107& 322& 161& 484& 242\\
121& 364& 182& 91& 274& 137& 412& 206& 103& 310\\
155& 466& 233& 700& 350& 175& 526& 263& 790& 395\\
1186& 593& 1780& 890& 445& 1336& 668& 334& 167& 502\\
251& 754& 377& 1132& 566& 283& 850& 425& 1276& 638\\
319& 958& 479& 1438& 719& 2158& 1079& 3238& 1619& 4858\\
2429& 7288& 3644& 1822& 911& 2734& 1367& 4102& 2051& 6154\\
3077& 9232& 4616& 2308& 1154& 577& 1732& 866& 433& 1300\\
650& 325& 976& 488& 244& 122& 61& 184& 92& 46\\
23& 70& 35& 106& 53& 160& 80& 40& 20& 10\\
5& 16& 8& 4& 2& 1& \\

698&&&&&&&&&\\
349& 1048& 524& 262& 131& 394& 197& 592& 296& 148\\
74& 37& 112& 56& 28& 14& 7& 22& 11& 34\\
17& 52& 26& 13& 40& 20& 10& 5& 16& 8\\
4& 2& 1& \\

699&&&&&&&&&\\
2098& 1049& 3148& 1574& 787& 2362& 1181& 3544& 1772& 886\\
443& 1330& 665& 1996& 998& 499& 1498& 749& 2248& 1124\\
562& 281& 844& 422& 211& 634& 317& 952& 476& 238\\
119& 358& 179& 538& 269& 808& 404& 202& 101& 304\\
152& 76& 38& 19& 58& 29& 88& 44& 22& 11\\
34& 17& 52& 26& 13& 40& 20& 10& 5& 16\\
8& 4& 2& 1& \\

700&&&&&&&&&\\
350& 175& 526& 263& 790& 395& 1186& 593& 1780& 890\\
445& 1336& 668& 334& 167& 502& 251& 754& 377& 1132\\
566& 283& 850& 425& 1276& 638& 319& 958& 479& 1438\\
719& 2158& 1079& 3238& 1619& 4858& 2429& 7288& 3644& 1822\\
911& 2734& 1367& 4102& 2051& 6154& 3077& 9232& 4616& 2308\\
1154& 577& 1732& 866& 433& 1300& 650& 325& 976& 488\\
244& 122& 61& 184& 92& 46& 23& 70& 35& 106\\
53& 160& 80& 40& 20& 10& 5& 16& 8& 4\\
2& 1& \\

701&&&&&&&&&\\
2104& 1052& 526& 263& 790& 395& 1186& 593& 1780& 890\\
445& 1336& 668& 334& 167& 502& 251& 754& 377& 1132\\
566& 283& 850& 425& 1276& 638& 319& 958& 479& 1438\\
719& 2158& 1079& 3238& 1619& 4858& 2429& 7288& 3644& 1822\\
911& 2734& 1367& 4102& 2051& 6154& 3077& 9232& 4616& 2308\\
1154& 577& 1732& 866& 433& 1300& 650& 325& 976& 488\\
244& 122& 61& 184& 92& 46& 23& 70& 35& 106\\
53& 160& 80& 40& 20& 10& 5& 16& 8& 4\\
2& 1& \\

702&&&&&&&&&\\
351& 1054& 527& 1582& 791& 2374& 1187& 3562& 1781& 5344\\
2672& 1336& 668& 334& 167& 502& 251& 754& 377& 1132\\
566& 283& 850& 425& 1276& 638& 319& 958& 479& 1438\\
719& 2158& 1079& 3238& 1619& 4858& 2429& 7288& 3644& 1822\\
911& 2734& 1367& 4102& 2051& 6154& 3077& 9232& 4616& 2308\\
1154& 577& 1732& 866& 433& 1300& 650& 325& 976& 488\\
244& 122& 61& 184& 92& 46& 23& 70& 35& 106\\
53& 160& 80& 40& 20& 10& 5& 16& 8& 4\\
2& 1& \\

703&&&&&&&&&\\
2110& 1055& 3166& 1583& 4750& 2375& 7126& 3563& 10690& 5345\\
16036& 8018& 4009& 12028& 6014& 3007& 9022& 4511& 13534& 6767\\
20302& 10151& 30454& 15227& 45682& 22841& 68524& 34262& 17131& 51394\\
25697& 77092& 38546& 19273& 57820& 28910& 14455& 43366& 21683& 65050\\
32525& 97576& 48788& 24394& 12197& 36592& 18296& 9148& 4574& 2287\\
6862& 3431& 10294& 5147& 15442& 7721& 23164& 11582& 5791& 17374\\
8687& 26062& 13031& 39094& 19547& 58642& 29321& 87964& 43982& 21991\\
65974& 32987& 98962& 49481& 148444& 74222& 37111& 111334& 55667& 167002\\
83501& 250504& 125252& 62626& 31313& 93940& 46970& 23485& 70456& 35228\\
17614& 8807& 26422& 13211& 39634& 19817& 59452& 29726& 14863& 44590\\
22295& 66886& 33443& 100330& 50165& 150496& 75248& 37624& 18812& 9406\\
4703& 14110& 7055& 21166& 10583& 31750& 15875& 47626& 23813& 71440\\
35720& 17860& 8930& 4465& 13396& 6698& 3349& 10048& 5024& 2512\\
1256& 628& 314& 157& 472& 236& 118& 59& 178& 89\\
268& 134& 67& 202& 101& 304& 152& 76& 38& 19\\
58& 29& 88& 44& 22& 11& 34& 17& 52& 26\\
13& 40& 20& 10& 5& 16& 8& 4& 2& 1\\

704&&&&&&&&&\\
352& 176& 88& 44& 22& 11& 34& 17& 52& 26\\
13& 40& 20& 10& 5& 16& 8& 4& 2& 1\\

705&&&&&&&&&\\
2116& 1058& 529& 1588& 794& 397& 1192& 596& 298& 149\\
448& 224& 112& 56& 28& 14& 7& 22& 11& 34\\
17& 52& 26& 13& 40& 20& 10& 5& 16& 8\\
4& 2& 1& \\

706&&&&&&&&&\\
353& 1060& 530& 265& 796& 398& 199& 598& 299& 898\\
449& 1348& 674& 337& 1012& 506& 253& 760& 380& 190\\
95& 286& 143& 430& 215& 646& 323& 970& 485& 1456\\
728& 364& 182& 91& 274& 137& 412& 206& 103& 310\\
155& 466& 233& 700& 350& 175& 526& 263& 790& 395\\
1186& 593& 1780& 890& 445& 1336& 668& 334& 167& 502\\
251& 754& 377& 1132& 566& 283& 850& 425& 1276& 638\\
319& 958& 479& 1438& 719& 2158& 1079& 3238& 1619& 4858\\
2429& 7288& 3644& 1822& 911& 2734& 1367& 4102& 2051& 6154\\
3077& 9232& 4616& 2308& 1154& 577& 1732& 866& 433& 1300\\
650& 325& 976& 488& 244& 122& 61& 184& 92& 46\\
23& 70& 35& 106& 53& 160& 80& 40& 20& 10\\
5& 16& 8& 4& 2& 1& \\

707&&&&&&&&&\\
2122& 1061& 3184& 1592& 796& 398& 199& 598& 299& 898\\
449& 1348& 674& 337& 1012& 506& 253& 760& 380& 190\\
95& 286& 143& 430& 215& 646& 323& 970& 485& 1456\\
728& 364& 182& 91& 274& 137& 412& 206& 103& 310\\
155& 466& 233& 700& 350& 175& 526& 263& 790& 395\\
1186& 593& 1780& 890& 445& 1336& 668& 334& 167& 502\\
251& 754& 377& 1132& 566& 283& 850& 425& 1276& 638\\
319& 958& 479& 1438& 719& 2158& 1079& 3238& 1619& 4858\\
2429& 7288& 3644& 1822& 911& 2734& 1367& 4102& 2051& 6154\\
3077& 9232& 4616& 2308& 1154& 577& 1732& 866& 433& 1300\\
650& 325& 976& 488& 244& 122& 61& 184& 92& 46\\
23& 70& 35& 106& 53& 160& 80& 40& 20& 10\\
5& 16& 8& 4& 2& 1& \\

708&&&&&&&&&\\
354& 177& 532& 266& 133& 400& 200& 100& 50& 25\\
76& 38& 19& 58& 29& 88& 44& 22& 11& 34\\
17& 52& 26& 13& 40& 20& 10& 5& 16& 8\\
4& 2& 1& \\

709&&&&&&&&&\\
2128& 1064& 532& 266& 133& 400& 200& 100& 50& 25\\
76& 38& 19& 58& 29& 88& 44& 22& 11& 34\\
17& 52& 26& 13& 40& 20& 10& 5& 16& 8\\
4& 2& 1& \\

710&&&&&&&&&\\
355& 1066& 533& 1600& 800& 400& 200& 100& 50& 25\\
76& 38& 19& 58& 29& 88& 44& 22& 11& 34\\
17& 52& 26& 13& 40& 20& 10& 5& 16& 8\\
4& 2& 1& \\

711&&&&&&&&&\\
2134& 1067& 3202& 1601& 4804& 2402& 1201& 3604& 1802& 901\\
2704& 1352& 676& 338& 169& 508& 254& 127& 382& 191\\
574& 287& 862& 431& 1294& 647& 1942& 971& 2914& 1457\\
4372& 2186& 1093& 3280& 1640& 820& 410& 205& 616& 308\\
154& 77& 232& 116& 58& 29& 88& 44& 22& 11\\
34& 17& 52& 26& 13& 40& 20& 10& 5& 16\\
8& 4& 2& 1& \\

712&&&&&&&&&\\
356& 178& 89& 268& 134& 67& 202& 101& 304& 152\\
76& 38& 19& 58& 29& 88& 44& 22& 11& 34\\
17& 52& 26& 13& 40& 20& 10& 5& 16& 8\\
4& 2& 1& \\

713&&&&&&&&&\\
2140& 1070& 535& 1606& 803& 2410& 1205& 3616& 1808& 904\\
452& 226& 113& 340& 170& 85& 256& 128& 64& 32\\
16& 8& 4& 2& 1& \\

714&&&&&&&&&\\
357& 1072& 536& 268& 134& 67& 202& 101& 304& 152\\
76& 38& 19& 58& 29& 88& 44& 22& 11& 34\\
17& 52& 26& 13& 40& 20& 10& 5& 16& 8\\
4& 2& 1& \\

715&&&&&&&&&\\
2146& 1073& 3220& 1610& 805& 2416& 1208& 604& 302& 151\\
454& 227& 682& 341& 1024& 512& 256& 128& 64& 32\\
16& 8& 4& 2& 1& \\

716&&&&&&&&&\\
358& 179& 538& 269& 808& 404& 202& 101& 304& 152\\
76& 38& 19& 58& 29& 88& 44& 22& 11& 34\\
17& 52& 26& 13& 40& 20& 10& 5& 16& 8\\
4& 2& 1& \\

717&&&&&&&&&\\
2152& 1076& 538& 269& 808& 404& 202& 101& 304& 152\\
76& 38& 19& 58& 29& 88& 44& 22& 11& 34\\
17& 52& 26& 13& 40& 20& 10& 5& 16& 8\\
4& 2& 1& \\

718&&&&&&&&&\\
359& 1078& 539& 1618& 809& 2428& 1214& 607& 1822& 911\\
2734& 1367& 4102& 2051& 6154& 3077& 9232& 4616& 2308& 1154\\
577& 1732& 866& 433& 1300& 650& 325& 976& 488& 244\\
122& 61& 184& 92& 46& 23& 70& 35& 106& 53\\
160& 80& 40& 20& 10& 5& 16& 8& 4& 2\\
1& \\

719&&&&&&&&&\\
2158& 1079& 3238& 1619& 4858& 2429& 7288& 3644& 1822& 911\\
2734& 1367& 4102& 2051& 6154& 3077& 9232& 4616& 2308& 1154\\
577& 1732& 866& 433& 1300& 650& 325& 976& 488& 244\\
122& 61& 184& 92& 46& 23& 70& 35& 106& 53\\
160& 80& 40& 20& 10& 5& 16& 8& 4& 2\\
1& \\

720&&&&&&&&&\\
360& 180& 90& 45& 136& 68& 34& 17& 52& 26\\
13& 40& 20& 10& 5& 16& 8& 4& 2& 1\\

721&&&&&&&&&\\
2164& 1082& 541& 1624& 812& 406& 203& 610& 305& 916\\
458& 229& 688& 344& 172& 86& 43& 130& 65& 196\\
98& 49& 148& 74& 37& 112& 56& 28& 14& 7\\
22& 11& 34& 17& 52& 26& 13& 40& 20& 10\\
5& 16& 8& 4& 2& 1& \\

722&&&&&&&&&\\
361& 1084& 542& 271& 814& 407& 1222& 611& 1834& 917\\
2752& 1376& 688& 344& 172& 86& 43& 130& 65& 196\\
98& 49& 148& 74& 37& 112& 56& 28& 14& 7\\
22& 11& 34& 17& 52& 26& 13& 40& 20& 10\\
5& 16& 8& 4& 2& 1& \\

723&&&&&&&&&\\
2170& 1085& 3256& 1628& 814& 407& 1222& 611& 1834& 917\\
2752& 1376& 688& 344& 172& 86& 43& 130& 65& 196\\
98& 49& 148& 74& 37& 112& 56& 28& 14& 7\\
22& 11& 34& 17& 52& 26& 13& 40& 20& 10\\
5& 16& 8& 4& 2& 1& \\

724&&&&&&&&&\\
362& 181& 544& 272& 136& 68& 34& 17& 52& 26\\
13& 40& 20& 10& 5& 16& 8& 4& 2& 1\\

725&&&&&&&&&\\
2176& 1088& 544& 272& 136& 68& 34& 17& 52& 26\\
13& 40& 20& 10& 5& 16& 8& 4& 2& 1\\

726&&&&&&&&&\\
363& 1090& 545& 1636& 818& 409& 1228& 614& 307& 922\\
461& 1384& 692& 346& 173& 520& 260& 130& 65& 196\\
98& 49& 148& 74& 37& 112& 56& 28& 14& 7\\
22& 11& 34& 17& 52& 26& 13& 40& 20& 10\\
5& 16& 8& 4& 2& 1& \\

727&&&&&&&&&\\
2182& 1091& 3274& 1637& 4912& 2456& 1228& 614& 307& 922\\
461& 1384& 692& 346& 173& 520& 260& 130& 65& 196\\
98& 49& 148& 74& 37& 112& 56& 28& 14& 7\\
22& 11& 34& 17& 52& 26& 13& 40& 20& 10\\
5& 16& 8& 4& 2& 1& \\

728&&&&&&&&&\\
364& 182& 91& 274& 137& 412& 206& 103& 310& 155\\
466& 233& 700& 350& 175& 526& 263& 790& 395& 1186\\
593& 1780& 890& 445& 1336& 668& 334& 167& 502& 251\\
754& 377& 1132& 566& 283& 850& 425& 1276& 638& 319\\
958& 479& 1438& 719& 2158& 1079& 3238& 1619& 4858& 2429\\
7288& 3644& 1822& 911& 2734& 1367& 4102& 2051& 6154& 3077\\
9232& 4616& 2308& 1154& 577& 1732& 866& 433& 1300& 650\\
325& 976& 488& 244& 122& 61& 184& 92& 46& 23\\
70& 35& 106& 53& 160& 80& 40& 20& 10& 5\\
16& 8& 4& 2& 1& \\

729&&&&&&&&&\\
2188& 1094& 547& 1642& 821& 2464& 1232& 616& 308& 154\\
77& 232& 116& 58& 29& 88& 44& 22& 11& 34\\
17& 52& 26& 13& 40& 20& 10& 5& 16& 8\\
4& 2& 1& \\

730&&&&&&&&&\\
365& 1096& 548& 274& 137& 412& 206& 103& 310& 155\\
466& 233& 700& 350& 175& 526& 263& 790& 395& 1186\\
593& 1780& 890& 445& 1336& 668& 334& 167& 502& 251\\
754& 377& 1132& 566& 283& 850& 425& 1276& 638& 319\\
958& 479& 1438& 719& 2158& 1079& 3238& 1619& 4858& 2429\\
7288& 3644& 1822& 911& 2734& 1367& 4102& 2051& 6154& 3077\\
9232& 4616& 2308& 1154& 577& 1732& 866& 433& 1300& 650\\
325& 976& 488& 244& 122& 61& 184& 92& 46& 23\\
70& 35& 106& 53& 160& 80& 40& 20& 10& 5\\
16& 8& 4& 2& 1& \\

731&&&&&&&&&\\
2194& 1097& 3292& 1646& 823& 2470& 1235& 3706& 1853& 5560\\
2780& 1390& 695& 2086& 1043& 3130& 1565& 4696& 2348& 1174\\
587& 1762& 881& 2644& 1322& 661& 1984& 992& 496& 248\\
124& 62& 31& 94& 47& 142& 71& 214& 107& 322\\
161& 484& 242& 121& 364& 182& 91& 274& 137& 412\\
206& 103& 310& 155& 466& 233& 700& 350& 175& 526\\
263& 790& 395& 1186& 593& 1780& 890& 445& 1336& 668\\
334& 167& 502& 251& 754& 377& 1132& 566& 283& 850\\
425& 1276& 638& 319& 958& 479& 1438& 719& 2158& 1079\\
3238& 1619& 4858& 2429& 7288& 3644& 1822& 911& 2734& 1367\\
4102& 2051& 6154& 3077& 9232& 4616& 2308& 1154& 577& 1732\\
866& 433& 1300& 650& 325& 976& 488& 244& 122& 61\\
184& 92& 46& 23& 70& 35& 106& 53& 160& 80\\
40& 20& 10& 5& 16& 8& 4& 2& 1& \\

732&&&&&&&&&\\
366& 183& 550& 275& 826& 413& 1240& 620& 310& 155\\
466& 233& 700& 350& 175& 526& 263& 790& 395& 1186\\
593& 1780& 890& 445& 1336& 668& 334& 167& 502& 251\\
754& 377& 1132& 566& 283& 850& 425& 1276& 638& 319\\
958& 479& 1438& 719& 2158& 1079& 3238& 1619& 4858& 2429\\
7288& 3644& 1822& 911& 2734& 1367& 4102& 2051& 6154& 3077\\
9232& 4616& 2308& 1154& 577& 1732& 866& 433& 1300& 650\\
325& 976& 488& 244& 122& 61& 184& 92& 46& 23\\
70& 35& 106& 53& 160& 80& 40& 20& 10& 5\\
16& 8& 4& 2& 1& \\

733&&&&&&&&&\\
2200& 1100& 550& 275& 826& 413& 1240& 620& 310& 155\\
466& 233& 700& 350& 175& 526& 263& 790& 395& 1186\\
593& 1780& 890& 445& 1336& 668& 334& 167& 502& 251\\
754& 377& 1132& 566& 283& 850& 425& 1276& 638& 319\\
958& 479& 1438& 719& 2158& 1079& 3238& 1619& 4858& 2429\\
7288& 3644& 1822& 911& 2734& 1367& 4102& 2051& 6154& 3077\\
9232& 4616& 2308& 1154& 577& 1732& 866& 433& 1300& 650\\
325& 976& 488& 244& 122& 61& 184& 92& 46& 23\\
70& 35& 106& 53& 160& 80& 40& 20& 10& 5\\
16& 8& 4& 2& 1& \\

734&&&&&&&&&\\
367& 1102& 551& 1654& 827& 2482& 1241& 3724& 1862& 931\\
2794& 1397& 4192& 2096& 1048& 524& 262& 131& 394& 197\\
592& 296& 148& 74& 37& 112& 56& 28& 14& 7\\
22& 11& 34& 17& 52& 26& 13& 40& 20& 10\\
5& 16& 8& 4& 2& 1& \\

735&&&&&&&&&\\
2206& 1103& 3310& 1655& 4966& 2483& 7450& 3725& 11176& 5588\\
2794& 1397& 4192& 2096& 1048& 524& 262& 131& 394& 197\\
592& 296& 148& 74& 37& 112& 56& 28& 14& 7\\
22& 11& 34& 17& 52& 26& 13& 40& 20& 10\\
5& 16& 8& 4& 2& 1& \\

736&&&&&&&&&\\
368& 184& 92& 46& 23& 70& 35& 106& 53& 160\\
80& 40& 20& 10& 5& 16& 8& 4& 2& 1\\

737&&&&&&&&&\\
2212& 1106& 553& 1660& 830& 415& 1246& 623& 1870& 935\\
2806& 1403& 4210& 2105& 6316& 3158& 1579& 4738& 2369& 7108\\
3554& 1777& 5332& 2666& 1333& 4000& 2000& 1000& 500& 250\\
125& 376& 188& 94& 47& 142& 71& 214& 107& 322\\
161& 484& 242& 121& 364& 182& 91& 274& 137& 412\\
206& 103& 310& 155& 466& 233& 700& 350& 175& 526\\
263& 790& 395& 1186& 593& 1780& 890& 445& 1336& 668\\
334& 167& 502& 251& 754& 377& 1132& 566& 283& 850\\
425& 1276& 638& 319& 958& 479& 1438& 719& 2158& 1079\\
3238& 1619& 4858& 2429& 7288& 3644& 1822& 911& 2734& 1367\\
4102& 2051& 6154& 3077& 9232& 4616& 2308& 1154& 577& 1732\\
866& 433& 1300& 650& 325& 976& 488& 244& 122& 61\\
184& 92& 46& 23& 70& 35& 106& 53& 160& 80\\
40& 20& 10& 5& 16& 8& 4& 2& 1& \\

738&&&&&&&&&\\
369& 1108& 554& 277& 832& 416& 208& 104& 52& 26\\
13& 40& 20& 10& 5& 16& 8& 4& 2& 1\\

739&&&&&&&&&\\
2218& 1109& 3328& 1664& 832& 416& 208& 104& 52& 26\\
13& 40& 20& 10& 5& 16& 8& 4& 2& 1\\

740&&&&&&&&&\\
370& 185& 556& 278& 139& 418& 209& 628& 314& 157\\
472& 236& 118& 59& 178& 89& 268& 134& 67& 202\\
101& 304& 152& 76& 38& 19& 58& 29& 88& 44\\
22& 11& 34& 17& 52& 26& 13& 40& 20& 10\\
5& 16& 8& 4& 2& 1& \\

741&&&&&&&&&\\
2224& 1112& 556& 278& 139& 418& 209& 628& 314& 157\\
472& 236& 118& 59& 178& 89& 268& 134& 67& 202\\
101& 304& 152& 76& 38& 19& 58& 29& 88& 44\\
22& 11& 34& 17& 52& 26& 13& 40& 20& 10\\
5& 16& 8& 4& 2& 1& \\

742&&&&&&&&&\\
371& 1114& 557& 1672& 836& 418& 209& 628& 314& 157\\
472& 236& 118& 59& 178& 89& 268& 134& 67& 202\\
101& 304& 152& 76& 38& 19& 58& 29& 88& 44\\
22& 11& 34& 17& 52& 26& 13& 40& 20& 10\\
5& 16& 8& 4& 2& 1& \\

743&&&&&&&&&\\
2230& 1115& 3346& 1673& 5020& 2510& 1255& 3766& 1883& 5650\\
2825& 8476& 4238& 2119& 6358& 3179& 9538& 4769& 14308& 7154\\
3577& 10732& 5366& 2683& 8050& 4025& 12076& 6038& 3019& 9058\\
4529& 13588& 6794& 3397& 10192& 5096& 2548& 1274& 637& 1912\\
956& 478& 239& 718& 359& 1078& 539& 1618& 809& 2428\\
1214& 607& 1822& 911& 2734& 1367& 4102& 2051& 6154& 3077\\
9232& 4616& 2308& 1154& 577& 1732& 866& 433& 1300& 650\\
325& 976& 488& 244& 122& 61& 184& 92& 46& 23\\
70& 35& 106& 53& 160& 80& 40& 20& 10& 5\\
16& 8& 4& 2& 1& \\

744&&&&&&&&&\\
372& 186& 93& 280& 140& 70& 35& 106& 53& 160\\
80& 40& 20& 10& 5& 16& 8& 4& 2& 1\\

745&&&&&&&&&\\
2236& 1118& 559& 1678& 839& 2518& 1259& 3778& 1889& 5668\\
2834& 1417& 4252& 2126& 1063& 3190& 1595& 4786& 2393& 7180\\
3590& 1795& 5386& 2693& 8080& 4040& 2020& 1010& 505& 1516\\
758& 379& 1138& 569& 1708& 854& 427& 1282& 641& 1924\\
962& 481& 1444& 722& 361& 1084& 542& 271& 814& 407\\
1222& 611& 1834& 917& 2752& 1376& 688& 344& 172& 86\\
43& 130& 65& 196& 98& 49& 148& 74& 37& 112\\
56& 28& 14& 7& 22& 11& 34& 17& 52& 26\\
13& 40& 20& 10& 5& 16& 8& 4& 2& 1\\

746&&&&&&&&&\\
373& 1120& 560& 280& 140& 70& 35& 106& 53& 160\\
80& 40& 20& 10& 5& 16& 8& 4& 2& 1\\

747&&&&&&&&&\\
2242& 1121& 3364& 1682& 841& 2524& 1262& 631& 1894& 947\\
2842& 1421& 4264& 2132& 1066& 533& 1600& 800& 400& 200\\
100& 50& 25& 76& 38& 19& 58& 29& 88& 44\\
22& 11& 34& 17& 52& 26& 13& 40& 20& 10\\
5& 16& 8& 4& 2& 1& \\

748&&&&&&&&&\\
374& 187& 562& 281& 844& 422& 211& 634& 317& 952\\
476& 238& 119& 358& 179& 538& 269& 808& 404& 202\\
101& 304& 152& 76& 38& 19& 58& 29& 88& 44\\
22& 11& 34& 17& 52& 26& 13& 40& 20& 10\\
5& 16& 8& 4& 2& 1& \\

749&&&&&&&&&\\
2248& 1124& 562& 281& 844& 422& 211& 634& 317& 952\\
476& 238& 119& 358& 179& 538& 269& 808& 404& 202\\
101& 304& 152& 76& 38& 19& 58& 29& 88& 44\\
22& 11& 34& 17& 52& 26& 13& 40& 20& 10\\
5& 16& 8& 4& 2& 1& \\

750&&&&&&&&&\\
375& 1126& 563& 1690& 845& 2536& 1268& 634& 317& 952\\
476& 238& 119& 358& 179& 538& 269& 808& 404& 202\\
101& 304& 152& 76& 38& 19& 58& 29& 88& 44\\
22& 11& 34& 17& 52& 26& 13& 40& 20& 10\\
5& 16& 8& 4& 2& 1& \\

751&&&&&&&&&\\
2254& 1127& 3382& 1691& 5074& 2537& 7612& 3806& 1903& 5710\\
2855& 8566& 4283& 12850& 6425& 19276& 9638& 4819& 14458& 7229\\
21688& 10844& 5422& 2711& 8134& 4067& 12202& 6101& 18304& 9152\\
4576& 2288& 1144& 572& 286& 143& 430& 215& 646& 323\\
970& 485& 1456& 728& 364& 182& 91& 274& 137& 412\\
206& 103& 310& 155& 466& 233& 700& 350& 175& 526\\
263& 790& 395& 1186& 593& 1780& 890& 445& 1336& 668\\
334& 167& 502& 251& 754& 377& 1132& 566& 283& 850\\
425& 1276& 638& 319& 958& 479& 1438& 719& 2158& 1079\\
3238& 1619& 4858& 2429& 7288& 3644& 1822& 911& 2734& 1367\\
4102& 2051& 6154& 3077& 9232& 4616& 2308& 1154& 577& 1732\\
866& 433& 1300& 650& 325& 976& 488& 244& 122& 61\\
184& 92& 46& 23& 70& 35& 106& 53& 160& 80\\
40& 20& 10& 5& 16& 8& 4& 2& 1& \\

752&&&&&&&&&\\
376& 188& 94& 47& 142& 71& 214& 107& 322& 161\\
484& 242& 121& 364& 182& 91& 274& 137& 412& 206\\
103& 310& 155& 466& 233& 700& 350& 175& 526& 263\\
790& 395& 1186& 593& 1780& 890& 445& 1336& 668& 334\\
167& 502& 251& 754& 377& 1132& 566& 283& 850& 425\\
1276& 638& 319& 958& 479& 1438& 719& 2158& 1079& 3238\\
1619& 4858& 2429& 7288& 3644& 1822& 911& 2734& 1367& 4102\\
2051& 6154& 3077& 9232& 4616& 2308& 1154& 577& 1732& 866\\
433& 1300& 650& 325& 976& 488& 244& 122& 61& 184\\
92& 46& 23& 70& 35& 106& 53& 160& 80& 40\\
20& 10& 5& 16& 8& 4& 2& 1& \\

753&&&&&&&&&\\
2260& 1130& 565& 1696& 848& 424& 212& 106& 53& 160\\
80& 40& 20& 10& 5& 16& 8& 4& 2& 1\\

754&&&&&&&&&\\
377& 1132& 566& 283& 850& 425& 1276& 638& 319& 958\\
479& 1438& 719& 2158& 1079& 3238& 1619& 4858& 2429& 7288\\
3644& 1822& 911& 2734& 1367& 4102& 2051& 6154& 3077& 9232\\
4616& 2308& 1154& 577& 1732& 866& 433& 1300& 650& 325\\
976& 488& 244& 122& 61& 184& 92& 46& 23& 70\\
35& 106& 53& 160& 80& 40& 20& 10& 5& 16\\
8& 4& 2& 1& \\

755&&&&&&&&&\\
2266& 1133& 3400& 1700& 850& 425& 1276& 638& 319& 958\\
479& 1438& 719& 2158& 1079& 3238& 1619& 4858& 2429& 7288\\
3644& 1822& 911& 2734& 1367& 4102& 2051& 6154& 3077& 9232\\
4616& 2308& 1154& 577& 1732& 866& 433& 1300& 650& 325\\
976& 488& 244& 122& 61& 184& 92& 46& 23& 70\\
35& 106& 53& 160& 80& 40& 20& 10& 5& 16\\
8& 4& 2& 1& \\

756&&&&&&&&&\\
378& 189& 568& 284& 142& 71& 214& 107& 322& 161\\
484& 242& 121& 364& 182& 91& 274& 137& 412& 206\\
103& 310& 155& 466& 233& 700& 350& 175& 526& 263\\
790& 395& 1186& 593& 1780& 890& 445& 1336& 668& 334\\
167& 502& 251& 754& 377& 1132& 566& 283& 850& 425\\
1276& 638& 319& 958& 479& 1438& 719& 2158& 1079& 3238\\
1619& 4858& 2429& 7288& 3644& 1822& 911& 2734& 1367& 4102\\
2051& 6154& 3077& 9232& 4616& 2308& 1154& 577& 1732& 866\\
433& 1300& 650& 325& 976& 488& 244& 122& 61& 184\\
92& 46& 23& 70& 35& 106& 53& 160& 80& 40\\
20& 10& 5& 16& 8& 4& 2& 1& \\

757&&&&&&&&&\\
2272& 1136& 568& 284& 142& 71& 214& 107& 322& 161\\
484& 242& 121& 364& 182& 91& 274& 137& 412& 206\\
103& 310& 155& 466& 233& 700& 350& 175& 526& 263\\
790& 395& 1186& 593& 1780& 890& 445& 1336& 668& 334\\
167& 502& 251& 754& 377& 1132& 566& 283& 850& 425\\
1276& 638& 319& 958& 479& 1438& 719& 2158& 1079& 3238\\
1619& 4858& 2429& 7288& 3644& 1822& 911& 2734& 1367& 4102\\
2051& 6154& 3077& 9232& 4616& 2308& 1154& 577& 1732& 866\\
433& 1300& 650& 325& 976& 488& 244& 122& 61& 184\\
92& 46& 23& 70& 35& 106& 53& 160& 80& 40\\
20& 10& 5& 16& 8& 4& 2& 1& \\

758&&&&&&&&&\\
379& 1138& 569& 1708& 854& 427& 1282& 641& 1924& 962\\
481& 1444& 722& 361& 1084& 542& 271& 814& 407& 1222\\
611& 1834& 917& 2752& 1376& 688& 344& 172& 86& 43\\
130& 65& 196& 98& 49& 148& 74& 37& 112& 56\\
28& 14& 7& 22& 11& 34& 17& 52& 26& 13\\
40& 20& 10& 5& 16& 8& 4& 2& 1& \\

759&&&&&&&&&\\
2278& 1139& 3418& 1709& 5128& 2564& 1282& 641& 1924& 962\\
481& 1444& 722& 361& 1084& 542& 271& 814& 407& 1222\\
611& 1834& 917& 2752& 1376& 688& 344& 172& 86& 43\\
130& 65& 196& 98& 49& 148& 74& 37& 112& 56\\
28& 14& 7& 22& 11& 34& 17& 52& 26& 13\\
40& 20& 10& 5& 16& 8& 4& 2& 1& \\

760&&&&&&&&&\\
380& 190& 95& 286& 143& 430& 215& 646& 323& 970\\
485& 1456& 728& 364& 182& 91& 274& 137& 412& 206\\
103& 310& 155& 466& 233& 700& 350& 175& 526& 263\\
790& 395& 1186& 593& 1780& 890& 445& 1336& 668& 334\\
167& 502& 251& 754& 377& 1132& 566& 283& 850& 425\\
1276& 638& 319& 958& 479& 1438& 719& 2158& 1079& 3238\\
1619& 4858& 2429& 7288& 3644& 1822& 911& 2734& 1367& 4102\\
2051& 6154& 3077& 9232& 4616& 2308& 1154& 577& 1732& 866\\
433& 1300& 650& 325& 976& 488& 244& 122& 61& 184\\
92& 46& 23& 70& 35& 106& 53& 160& 80& 40\\
20& 10& 5& 16& 8& 4& 2& 1& \\

761&&&&&&&&&\\
2284& 1142& 571& 1714& 857& 2572& 1286& 643& 1930& 965\\
2896& 1448& 724& 362& 181& 544& 272& 136& 68& 34\\
17& 52& 26& 13& 40& 20& 10& 5& 16& 8\\
4& 2& 1& \\

762&&&&&&&&&\\
381& 1144& 572& 286& 143& 430& 215& 646& 323& 970\\
485& 1456& 728& 364& 182& 91& 274& 137& 412& 206\\
103& 310& 155& 466& 233& 700& 350& 175& 526& 263\\
790& 395& 1186& 593& 1780& 890& 445& 1336& 668& 334\\
167& 502& 251& 754& 377& 1132& 566& 283& 850& 425\\
1276& 638& 319& 958& 479& 1438& 719& 2158& 1079& 3238\\
1619& 4858& 2429& 7288& 3644& 1822& 911& 2734& 1367& 4102\\
2051& 6154& 3077& 9232& 4616& 2308& 1154& 577& 1732& 866\\
433& 1300& 650& 325& 976& 488& 244& 122& 61& 184\\
92& 46& 23& 70& 35& 106& 53& 160& 80& 40\\
20& 10& 5& 16& 8& 4& 2& 1& \\

763&&&&&&&&&\\
2290& 1145& 3436& 1718& 859& 2578& 1289& 3868& 1934& 967\\
2902& 1451& 4354& 2177& 6532& 3266& 1633& 4900& 2450& 1225\\
3676& 1838& 919& 2758& 1379& 4138& 2069& 6208& 3104& 1552\\
776& 388& 194& 97& 292& 146& 73& 220& 110& 55\\
166& 83& 250& 125& 376& 188& 94& 47& 142& 71\\
214& 107& 322& 161& 484& 242& 121& 364& 182& 91\\
274& 137& 412& 206& 103& 310& 155& 466& 233& 700\\
350& 175& 526& 263& 790& 395& 1186& 593& 1780& 890\\
445& 1336& 668& 334& 167& 502& 251& 754& 377& 1132\\
566& 283& 850& 425& 1276& 638& 319& 958& 479& 1438\\
719& 2158& 1079& 3238& 1619& 4858& 2429& 7288& 3644& 1822\\
911& 2734& 1367& 4102& 2051& 6154& 3077& 9232& 4616& 2308\\
1154& 577& 1732& 866& 433& 1300& 650& 325& 976& 488\\
244& 122& 61& 184& 92& 46& 23& 70& 35& 106\\
53& 160& 80& 40& 20& 10& 5& 16& 8& 4\\
2& 1& \\

764&&&&&&&&&\\
382& 191& 574& 287& 862& 431& 1294& 647& 1942& 971\\
2914& 1457& 4372& 2186& 1093& 3280& 1640& 820& 410& 205\\
616& 308& 154& 77& 232& 116& 58& 29& 88& 44\\
22& 11& 34& 17& 52& 26& 13& 40& 20& 10\\
5& 16& 8& 4& 2& 1& \\

765&&&&&&&&&\\
2296& 1148& 574& 287& 862& 431& 1294& 647& 1942& 971\\
2914& 1457& 4372& 2186& 1093& 3280& 1640& 820& 410& 205\\
616& 308& 154& 77& 232& 116& 58& 29& 88& 44\\
22& 11& 34& 17& 52& 26& 13& 40& 20& 10\\
5& 16& 8& 4& 2& 1& \\

766&&&&&&&&&\\
383& 1150& 575& 1726& 863& 2590& 1295& 3886& 1943& 5830\\
2915& 8746& 4373& 13120& 6560& 3280& 1640& 820& 410& 205\\
616& 308& 154& 77& 232& 116& 58& 29& 88& 44\\
22& 11& 34& 17& 52& 26& 13& 40& 20& 10\\
5& 16& 8& 4& 2& 1& \\

767&&&&&&&&&\\
2302& 1151& 3454& 1727& 5182& 2591& 7774& 3887& 11662& 5831\\
17494& 8747& 26242& 13121& 39364& 19682& 9841& 29524& 14762& 7381\\
22144& 11072& 5536& 2768& 1384& 692& 346& 173& 520& 260\\
130& 65& 196& 98& 49& 148& 74& 37& 112& 56\\
28& 14& 7& 22& 11& 34& 17& 52& 26& 13\\
40& 20& 10& 5& 16& 8& 4& 2& 1& \\

768&&&&&&&&&\\
384& 192& 96& 48& 24& 12& 6& 3& 10& 5\\
16& 8& 4& 2& 1& \\

769&&&&&&&&&\\
2308& 1154& 577& 1732& 866& 433& 1300& 650& 325& 976\\
488& 244& 122& 61& 184& 92& 46& 23& 70& 35\\
106& 53& 160& 80& 40& 20& 10& 5& 16& 8\\
4& 2& 1& \\

770&&&&&&&&&\\
385& 1156& 578& 289& 868& 434& 217& 652& 326& 163\\
490& 245& 736& 368& 184& 92& 46& 23& 70& 35\\
106& 53& 160& 80& 40& 20& 10& 5& 16& 8\\
4& 2& 1& \\

771&&&&&&&&&\\
2314& 1157& 3472& 1736& 868& 434& 217& 652& 326& 163\\
490& 245& 736& 368& 184& 92& 46& 23& 70& 35\\
106& 53& 160& 80& 40& 20& 10& 5& 16& 8\\
4& 2& 1& \\

772&&&&&&&&&\\
386& 193& 580& 290& 145& 436& 218& 109& 328& 164\\
82& 41& 124& 62& 31& 94& 47& 142& 71& 214\\
107& 322& 161& 484& 242& 121& 364& 182& 91& 274\\
137& 412& 206& 103& 310& 155& 466& 233& 700& 350\\
175& 526& 263& 790& 395& 1186& 593& 1780& 890& 445\\
1336& 668& 334& 167& 502& 251& 754& 377& 1132& 566\\
283& 850& 425& 1276& 638& 319& 958& 479& 1438& 719\\
2158& 1079& 3238& 1619& 4858& 2429& 7288& 3644& 1822& 911\\
2734& 1367& 4102& 2051& 6154& 3077& 9232& 4616& 2308& 1154\\
577& 1732& 866& 433& 1300& 650& 325& 976& 488& 244\\
122& 61& 184& 92& 46& 23& 70& 35& 106& 53\\
160& 80& 40& 20& 10& 5& 16& 8& 4& 2\\
1& \\

773&&&&&&&&&\\
2320& 1160& 580& 290& 145& 436& 218& 109& 328& 164\\
82& 41& 124& 62& 31& 94& 47& 142& 71& 214\\
107& 322& 161& 484& 242& 121& 364& 182& 91& 274\\
137& 412& 206& 103& 310& 155& 466& 233& 700& 350\\
175& 526& 263& 790& 395& 1186& 593& 1780& 890& 445\\
1336& 668& 334& 167& 502& 251& 754& 377& 1132& 566\\
283& 850& 425& 1276& 638& 319& 958& 479& 1438& 719\\
2158& 1079& 3238& 1619& 4858& 2429& 7288& 3644& 1822& 911\\
2734& 1367& 4102& 2051& 6154& 3077& 9232& 4616& 2308& 1154\\
577& 1732& 866& 433& 1300& 650& 325& 976& 488& 244\\
122& 61& 184& 92& 46& 23& 70& 35& 106& 53\\
160& 80& 40& 20& 10& 5& 16& 8& 4& 2\\
1& \\

774&&&&&&&&&\\
387& 1162& 581& 1744& 872& 436& 218& 109& 328& 164\\
82& 41& 124& 62& 31& 94& 47& 142& 71& 214\\
107& 322& 161& 484& 242& 121& 364& 182& 91& 274\\
137& 412& 206& 103& 310& 155& 466& 233& 700& 350\\
175& 526& 263& 790& 395& 1186& 593& 1780& 890& 445\\
1336& 668& 334& 167& 502& 251& 754& 377& 1132& 566\\
283& 850& 425& 1276& 638& 319& 958& 479& 1438& 719\\
2158& 1079& 3238& 1619& 4858& 2429& 7288& 3644& 1822& 911\\
2734& 1367& 4102& 2051& 6154& 3077& 9232& 4616& 2308& 1154\\
577& 1732& 866& 433& 1300& 650& 325& 976& 488& 244\\
122& 61& 184& 92& 46& 23& 70& 35& 106& 53\\
160& 80& 40& 20& 10& 5& 16& 8& 4& 2\\
1& \\

775&&&&&&&&&\\
2326& 1163& 3490& 1745& 5236& 2618& 1309& 3928& 1964& 982\\
491& 1474& 737& 2212& 1106& 553& 1660& 830& 415& 1246\\
623& 1870& 935& 2806& 1403& 4210& 2105& 6316& 3158& 1579\\
4738& 2369& 7108& 3554& 1777& 5332& 2666& 1333& 4000& 2000\\
1000& 500& 250& 125& 376& 188& 94& 47& 142& 71\\
214& 107& 322& 161& 484& 242& 121& 364& 182& 91\\
274& 137& 412& 206& 103& 310& 155& 466& 233& 700\\
350& 175& 526& 263& 790& 395& 1186& 593& 1780& 890\\
445& 1336& 668& 334& 167& 502& 251& 754& 377& 1132\\
566& 283& 850& 425& 1276& 638& 319& 958& 479& 1438\\
719& 2158& 1079& 3238& 1619& 4858& 2429& 7288& 3644& 1822\\
911& 2734& 1367& 4102& 2051& 6154& 3077& 9232& 4616& 2308\\
1154& 577& 1732& 866& 433& 1300& 650& 325& 976& 488\\
244& 122& 61& 184& 92& 46& 23& 70& 35& 106\\
53& 160& 80& 40& 20& 10& 5& 16& 8& 4\\
2& 1& \\

776&&&&&&&&&\\
388& 194& 97& 292& 146& 73& 220& 110& 55& 166\\
83& 250& 125& 376& 188& 94& 47& 142& 71& 214\\
107& 322& 161& 484& 242& 121& 364& 182& 91& 274\\
137& 412& 206& 103& 310& 155& 466& 233& 700& 350\\
175& 526& 263& 790& 395& 1186& 593& 1780& 890& 445\\
1336& 668& 334& 167& 502& 251& 754& 377& 1132& 566\\
283& 850& 425& 1276& 638& 319& 958& 479& 1438& 719\\
2158& 1079& 3238& 1619& 4858& 2429& 7288& 3644& 1822& 911\\
2734& 1367& 4102& 2051& 6154& 3077& 9232& 4616& 2308& 1154\\
577& 1732& 866& 433& 1300& 650& 325& 976& 488& 244\\
122& 61& 184& 92& 46& 23& 70& 35& 106& 53\\
160& 80& 40& 20& 10& 5& 16& 8& 4& 2\\
1& \\

777&&&&&&&&&\\
2332& 1166& 583& 1750& 875& 2626& 1313& 3940& 1970& 985\\
2956& 1478& 739& 2218& 1109& 3328& 1664& 832& 416& 208\\
104& 52& 26& 13& 40& 20& 10& 5& 16& 8\\
4& 2& 1& \\

778&&&&&&&&&\\
389& 1168& 584& 292& 146& 73& 220& 110& 55& 166\\
83& 250& 125& 376& 188& 94& 47& 142& 71& 214\\
107& 322& 161& 484& 242& 121& 364& 182& 91& 274\\
137& 412& 206& 103& 310& 155& 466& 233& 700& 350\\
175& 526& 263& 790& 395& 1186& 593& 1780& 890& 445\\
1336& 668& 334& 167& 502& 251& 754& 377& 1132& 566\\
283& 850& 425& 1276& 638& 319& 958& 479& 1438& 719\\
2158& 1079& 3238& 1619& 4858& 2429& 7288& 3644& 1822& 911\\
2734& 1367& 4102& 2051& 6154& 3077& 9232& 4616& 2308& 1154\\
577& 1732& 866& 433& 1300& 650& 325& 976& 488& 244\\
122& 61& 184& 92& 46& 23& 70& 35& 106& 53\\
160& 80& 40& 20& 10& 5& 16& 8& 4& 2\\
1& \\

779&&&&&&&&&\\
2338& 1169& 3508& 1754& 877& 2632& 1316& 658& 329& 988\\
494& 247& 742& 371& 1114& 557& 1672& 836& 418& 209\\
628& 314& 157& 472& 236& 118& 59& 178& 89& 268\\
134& 67& 202& 101& 304& 152& 76& 38& 19& 58\\
29& 88& 44& 22& 11& 34& 17& 52& 26& 13\\
40& 20& 10& 5& 16& 8& 4& 2& 1& \\

780&&&&&&&&&\\
390& 195& 586& 293& 880& 440& 220& 110& 55& 166\\
83& 250& 125& 376& 188& 94& 47& 142& 71& 214\\
107& 322& 161& 484& 242& 121& 364& 182& 91& 274\\
137& 412& 206& 103& 310& 155& 466& 233& 700& 350\\
175& 526& 263& 790& 395& 1186& 593& 1780& 890& 445\\
1336& 668& 334& 167& 502& 251& 754& 377& 1132& 566\\
283& 850& 425& 1276& 638& 319& 958& 479& 1438& 719\\
2158& 1079& 3238& 1619& 4858& 2429& 7288& 3644& 1822& 911\\
2734& 1367& 4102& 2051& 6154& 3077& 9232& 4616& 2308& 1154\\
577& 1732& 866& 433& 1300& 650& 325& 976& 488& 244\\
122& 61& 184& 92& 46& 23& 70& 35& 106& 53\\
160& 80& 40& 20& 10& 5& 16& 8& 4& 2\\
1& \\

781&&&&&&&&&\\
2344& 1172& 586& 293& 880& 440& 220& 110& 55& 166\\
83& 250& 125& 376& 188& 94& 47& 142& 71& 214\\
107& 322& 161& 484& 242& 121& 364& 182& 91& 274\\
137& 412& 206& 103& 310& 155& 466& 233& 700& 350\\
175& 526& 263& 790& 395& 1186& 593& 1780& 890& 445\\
1336& 668& 334& 167& 502& 251& 754& 377& 1132& 566\\
283& 850& 425& 1276& 638& 319& 958& 479& 1438& 719\\
2158& 1079& 3238& 1619& 4858& 2429& 7288& 3644& 1822& 911\\
2734& 1367& 4102& 2051& 6154& 3077& 9232& 4616& 2308& 1154\\
577& 1732& 866& 433& 1300& 650& 325& 976& 488& 244\\
122& 61& 184& 92& 46& 23& 70& 35& 106& 53\\
160& 80& 40& 20& 10& 5& 16& 8& 4& 2\\
1& \\

782&&&&&&&&&\\
391& 1174& 587& 1762& 881& 2644& 1322& 661& 1984& 992\\
496& 248& 124& 62& 31& 94& 47& 142& 71& 214\\
107& 322& 161& 484& 242& 121& 364& 182& 91& 274\\
137& 412& 206& 103& 310& 155& 466& 233& 700& 350\\
175& 526& 263& 790& 395& 1186& 593& 1780& 890& 445\\
1336& 668& 334& 167& 502& 251& 754& 377& 1132& 566\\
283& 850& 425& 1276& 638& 319& 958& 479& 1438& 719\\
2158& 1079& 3238& 1619& 4858& 2429& 7288& 3644& 1822& 911\\
2734& 1367& 4102& 2051& 6154& 3077& 9232& 4616& 2308& 1154\\
577& 1732& 866& 433& 1300& 650& 325& 976& 488& 244\\
122& 61& 184& 92& 46& 23& 70& 35& 106& 53\\
160& 80& 40& 20& 10& 5& 16& 8& 4& 2\\
1& \\

783&&&&&&&&&\\
2350& 1175& 3526& 1763& 5290& 2645& 7936& 3968& 1984& 992\\
496& 248& 124& 62& 31& 94& 47& 142& 71& 214\\
107& 322& 161& 484& 242& 121& 364& 182& 91& 274\\
137& 412& 206& 103& 310& 155& 466& 233& 700& 350\\
175& 526& 263& 790& 395& 1186& 593& 1780& 890& 445\\
1336& 668& 334& 167& 502& 251& 754& 377& 1132& 566\\
283& 850& 425& 1276& 638& 319& 958& 479& 1438& 719\\
2158& 1079& 3238& 1619& 4858& 2429& 7288& 3644& 1822& 911\\
2734& 1367& 4102& 2051& 6154& 3077& 9232& 4616& 2308& 1154\\
577& 1732& 866& 433& 1300& 650& 325& 976& 488& 244\\
122& 61& 184& 92& 46& 23& 70& 35& 106& 53\\
160& 80& 40& 20& 10& 5& 16& 8& 4& 2\\
1& \\

784&&&&&&&&&\\
392& 196& 98& 49& 148& 74& 37& 112& 56& 28\\
14& 7& 22& 11& 34& 17& 52& 26& 13& 40\\
20& 10& 5& 16& 8& 4& 2& 1& \\

785&&&&&&&&&\\
2356& 1178& 589& 1768& 884& 442& 221& 664& 332& 166\\
83& 250& 125& 376& 188& 94& 47& 142& 71& 214\\
107& 322& 161& 484& 242& 121& 364& 182& 91& 274\\
137& 412& 206& 103& 310& 155& 466& 233& 700& 350\\
175& 526& 263& 790& 395& 1186& 593& 1780& 890& 445\\
1336& 668& 334& 167& 502& 251& 754& 377& 1132& 566\\
283& 850& 425& 1276& 638& 319& 958& 479& 1438& 719\\
2158& 1079& 3238& 1619& 4858& 2429& 7288& 3644& 1822& 911\\
2734& 1367& 4102& 2051& 6154& 3077& 9232& 4616& 2308& 1154\\
577& 1732& 866& 433& 1300& 650& 325& 976& 488& 244\\
122& 61& 184& 92& 46& 23& 70& 35& 106& 53\\
160& 80& 40& 20& 10& 5& 16& 8& 4& 2\\
1& \\

786&&&&&&&&&\\
393& 1180& 590& 295& 886& 443& 1330& 665& 1996& 998\\
499& 1498& 749& 2248& 1124& 562& 281& 844& 422& 211\\
634& 317& 952& 476& 238& 119& 358& 179& 538& 269\\
808& 404& 202& 101& 304& 152& 76& 38& 19& 58\\
29& 88& 44& 22& 11& 34& 17& 52& 26& 13\\
40& 20& 10& 5& 16& 8& 4& 2& 1& \\

787&&&&&&&&&\\
2362& 1181& 3544& 1772& 886& 443& 1330& 665& 1996& 998\\
499& 1498& 749& 2248& 1124& 562& 281& 844& 422& 211\\
634& 317& 952& 476& 238& 119& 358& 179& 538& 269\\
808& 404& 202& 101& 304& 152& 76& 38& 19& 58\\
29& 88& 44& 22& 11& 34& 17& 52& 26& 13\\
40& 20& 10& 5& 16& 8& 4& 2& 1& \\

788&&&&&&&&&\\
394& 197& 592& 296& 148& 74& 37& 112& 56& 28\\
14& 7& 22& 11& 34& 17& 52& 26& 13& 40\\
20& 10& 5& 16& 8& 4& 2& 1& \\

789&&&&&&&&&\\
2368& 1184& 592& 296& 148& 74& 37& 112& 56& 28\\
14& 7& 22& 11& 34& 17& 52& 26& 13& 40\\
20& 10& 5& 16& 8& 4& 2& 1& \\

790&&&&&&&&&\\
395& 1186& 593& 1780& 890& 445& 1336& 668& 334& 167\\
502& 251& 754& 377& 1132& 566& 283& 850& 425& 1276\\
638& 319& 958& 479& 1438& 719& 2158& 1079& 3238& 1619\\
4858& 2429& 7288& 3644& 1822& 911& 2734& 1367& 4102& 2051\\
6154& 3077& 9232& 4616& 2308& 1154& 577& 1732& 866& 433\\
1300& 650& 325& 976& 488& 244& 122& 61& 184& 92\\
46& 23& 70& 35& 106& 53& 160& 80& 40& 20\\
10& 5& 16& 8& 4& 2& 1& \\

791&&&&&&&&&\\
2374& 1187& 3562& 1781& 5344& 2672& 1336& 668& 334& 167\\
502& 251& 754& 377& 1132& 566& 283& 850& 425& 1276\\
638& 319& 958& 479& 1438& 719& 2158& 1079& 3238& 1619\\
4858& 2429& 7288& 3644& 1822& 911& 2734& 1367& 4102& 2051\\
6154& 3077& 9232& 4616& 2308& 1154& 577& 1732& 866& 433\\
1300& 650& 325& 976& 488& 244& 122& 61& 184& 92\\
46& 23& 70& 35& 106& 53& 160& 80& 40& 20\\
10& 5& 16& 8& 4& 2& 1& \\

792&&&&&&&&&\\
396& 198& 99& 298& 149& 448& 224& 112& 56& 28\\
14& 7& 22& 11& 34& 17& 52& 26& 13& 40\\
20& 10& 5& 16& 8& 4& 2& 1& \\

793&&&&&&&&&\\
2380& 1190& 595& 1786& 893& 2680& 1340& 670& 335& 1006\\
503& 1510& 755& 2266& 1133& 3400& 1700& 850& 425& 1276\\
638& 319& 958& 479& 1438& 719& 2158& 1079& 3238& 1619\\
4858& 2429& 7288& 3644& 1822& 911& 2734& 1367& 4102& 2051\\
6154& 3077& 9232& 4616& 2308& 1154& 577& 1732& 866& 433\\
1300& 650& 325& 976& 488& 244& 122& 61& 184& 92\\
46& 23& 70& 35& 106& 53& 160& 80& 40& 20\\
10& 5& 16& 8& 4& 2& 1& \\

794&&&&&&&&&\\
397& 1192& 596& 298& 149& 448& 224& 112& 56& 28\\
14& 7& 22& 11& 34& 17& 52& 26& 13& 40\\
20& 10& 5& 16& 8& 4& 2& 1& \\

795&&&&&&&&&\\
2386& 1193& 3580& 1790& 895& 2686& 1343& 4030& 2015& 6046\\
3023& 9070& 4535& 13606& 6803& 20410& 10205& 30616& 15308& 7654\\
3827& 11482& 5741& 17224& 8612& 4306& 2153& 6460& 3230& 1615\\
4846& 2423& 7270& 3635& 10906& 5453& 16360& 8180& 4090& 2045\\
6136& 3068& 1534& 767& 2302& 1151& 3454& 1727& 5182& 2591\\
7774& 3887& 11662& 5831& 17494& 8747& 26242& 13121& 39364& 19682\\
9841& 29524& 14762& 7381& 22144& 11072& 5536& 2768& 1384& 692\\
346& 173& 520& 260& 130& 65& 196& 98& 49& 148\\
74& 37& 112& 56& 28& 14& 7& 22& 11& 34\\
17& 52& 26& 13& 40& 20& 10& 5& 16& 8\\
4& 2& 1& \\

796&&&&&&&&&\\
398& 199& 598& 299& 898& 449& 1348& 674& 337& 1012\\
506& 253& 760& 380& 190& 95& 286& 143& 430& 215\\
646& 323& 970& 485& 1456& 728& 364& 182& 91& 274\\
137& 412& 206& 103& 310& 155& 466& 233& 700& 350\\
175& 526& 263& 790& 395& 1186& 593& 1780& 890& 445\\
1336& 668& 334& 167& 502& 251& 754& 377& 1132& 566\\
283& 850& 425& 1276& 638& 319& 958& 479& 1438& 719\\
2158& 1079& 3238& 1619& 4858& 2429& 7288& 3644& 1822& 911\\
2734& 1367& 4102& 2051& 6154& 3077& 9232& 4616& 2308& 1154\\
577& 1732& 866& 433& 1300& 650& 325& 976& 488& 244\\
122& 61& 184& 92& 46& 23& 70& 35& 106& 53\\
160& 80& 40& 20& 10& 5& 16& 8& 4& 2\\
1& \\

797&&&&&&&&&\\
2392& 1196& 598& 299& 898& 449& 1348& 674& 337& 1012\\
506& 253& 760& 380& 190& 95& 286& 143& 430& 215\\
646& 323& 970& 485& 1456& 728& 364& 182& 91& 274\\
137& 412& 206& 103& 310& 155& 466& 233& 700& 350\\
175& 526& 263& 790& 395& 1186& 593& 1780& 890& 445\\
1336& 668& 334& 167& 502& 251& 754& 377& 1132& 566\\
283& 850& 425& 1276& 638& 319& 958& 479& 1438& 719\\
2158& 1079& 3238& 1619& 4858& 2429& 7288& 3644& 1822& 911\\
2734& 1367& 4102& 2051& 6154& 3077& 9232& 4616& 2308& 1154\\
577& 1732& 866& 433& 1300& 650& 325& 976& 488& 244\\
122& 61& 184& 92& 46& 23& 70& 35& 106& 53\\
160& 80& 40& 20& 10& 5& 16& 8& 4& 2\\
1& \\

798&&&&&&&&&\\
399& 1198& 599& 1798& 899& 2698& 1349& 4048& 2024& 1012\\
506& 253& 760& 380& 190& 95& 286& 143& 430& 215\\
646& 323& 970& 485& 1456& 728& 364& 182& 91& 274\\
137& 412& 206& 103& 310& 155& 466& 233& 700& 350\\
175& 526& 263& 790& 395& 1186& 593& 1780& 890& 445\\
1336& 668& 334& 167& 502& 251& 754& 377& 1132& 566\\
283& 850& 425& 1276& 638& 319& 958& 479& 1438& 719\\
2158& 1079& 3238& 1619& 4858& 2429& 7288& 3644& 1822& 911\\
2734& 1367& 4102& 2051& 6154& 3077& 9232& 4616& 2308& 1154\\
577& 1732& 866& 433& 1300& 650& 325& 976& 488& 244\\
122& 61& 184& 92& 46& 23& 70& 35& 106& 53\\
160& 80& 40& 20& 10& 5& 16& 8& 4& 2\\
1& \\

799&&&&&&&&&\\
2398& 1199& 3598& 1799& 5398& 2699& 8098& 4049& 12148& 6074\\
3037& 9112& 4556& 2278& 1139& 3418& 1709& 5128& 2564& 1282\\
641& 1924& 962& 481& 1444& 722& 361& 1084& 542& 271\\
814& 407& 1222& 611& 1834& 917& 2752& 1376& 688& 344\\
172& 86& 43& 130& 65& 196& 98& 49& 148& 74\\
37& 112& 56& 28& 14& 7& 22& 11& 34& 17\\
52& 26& 13& 40& 20& 10& 5& 16& 8& 4\\
2& 1& \\

800&&&&&&&&&\\
400& 200& 100& 50& 25& 76& 38& 19& 58& 29\\
88& 44& 22& 11& 34& 17& 52& 26& 13& 40\\
20& 10& 5& 16& 8& 4& 2& 1& \\

801&&&&&&&&&\\
2404& 1202& 601& 1804& 902& 451& 1354& 677& 2032& 1016\\
508& 254& 127& 382& 191& 574& 287& 862& 431& 1294\\
647& 1942& 971& 2914& 1457& 4372& 2186& 1093& 3280& 1640\\
820& 410& 205& 616& 308& 154& 77& 232& 116& 58\\
29& 88& 44& 22& 11& 34& 17& 52& 26& 13\\
40& 20& 10& 5& 16& 8& 4& 2& 1& \\

802&&&&&&&&&\\
401& 1204& 602& 301& 904& 452& 226& 113& 340& 170\\
85& 256& 128& 64& 32& 16& 8& 4& 2& 1\\

803&&&&&&&&&\\
2410& 1205& 3616& 1808& 904& 452& 226& 113& 340& 170\\
85& 256& 128& 64& 32& 16& 8& 4& 2& 1\\

804&&&&&&&&&\\
402& 201& 604& 302& 151& 454& 227& 682& 341& 1024\\
512& 256& 128& 64& 32& 16& 8& 4& 2& 1\\

805&&&&&&&&&\\
2416& 1208& 604& 302& 151& 454& 227& 682& 341& 1024\\
512& 256& 128& 64& 32& 16& 8& 4& 2& 1\\

806&&&&&&&&&\\
403& 1210& 605& 1816& 908& 454& 227& 682& 341& 1024\\
512& 256& 128& 64& 32& 16& 8& 4& 2& 1\\

807&&&&&&&&&\\
2422& 1211& 3634& 1817& 5452& 2726& 1363& 4090& 2045& 6136\\
3068& 1534& 767& 2302& 1151& 3454& 1727& 5182& 2591& 7774\\
3887& 11662& 5831& 17494& 8747& 26242& 13121& 39364& 19682& 9841\\
29524& 14762& 7381& 22144& 11072& 5536& 2768& 1384& 692& 346\\
173& 520& 260& 130& 65& 196& 98& 49& 148& 74\\
37& 112& 56& 28& 14& 7& 22& 11& 34& 17\\
52& 26& 13& 40& 20& 10& 5& 16& 8& 4\\
2& 1& \\

808&&&&&&&&&\\
404& 202& 101& 304& 152& 76& 38& 19& 58& 29\\
88& 44& 22& 11& 34& 17& 52& 26& 13& 40\\
20& 10& 5& 16& 8& 4& 2& 1& \\

809&&&&&&&&&\\
2428& 1214& 607& 1822& 911& 2734& 1367& 4102& 2051& 6154\\
3077& 9232& 4616& 2308& 1154& 577& 1732& 866& 433& 1300\\
650& 325& 976& 488& 244& 122& 61& 184& 92& 46\\
23& 70& 35& 106& 53& 160& 80& 40& 20& 10\\
5& 16& 8& 4& 2& 1& \\

810&&&&&&&&&\\
405& 1216& 608& 304& 152& 76& 38& 19& 58& 29\\
88& 44& 22& 11& 34& 17& 52& 26& 13& 40\\
20& 10& 5& 16& 8& 4& 2& 1& \\

811&&&&&&&&&\\
2434& 1217& 3652& 1826& 913& 2740& 1370& 685& 2056& 1028\\
514& 257& 772& 386& 193& 580& 290& 145& 436& 218\\
109& 328& 164& 82& 41& 124& 62& 31& 94& 47\\
142& 71& 214& 107& 322& 161& 484& 242& 121& 364\\
182& 91& 274& 137& 412& 206& 103& 310& 155& 466\\
233& 700& 350& 175& 526& 263& 790& 395& 1186& 593\\
1780& 890& 445& 1336& 668& 334& 167& 502& 251& 754\\
377& 1132& 566& 283& 850& 425& 1276& 638& 319& 958\\
479& 1438& 719& 2158& 1079& 3238& 1619& 4858& 2429& 7288\\
3644& 1822& 911& 2734& 1367& 4102& 2051& 6154& 3077& 9232\\
4616& 2308& 1154& 577& 1732& 866& 433& 1300& 650& 325\\
976& 488& 244& 122& 61& 184& 92& 46& 23& 70\\
35& 106& 53& 160& 80& 40& 20& 10& 5& 16\\
8& 4& 2& 1& \\

812&&&&&&&&&\\
406& 203& 610& 305& 916& 458& 229& 688& 344& 172\\
86& 43& 130& 65& 196& 98& 49& 148& 74& 37\\
112& 56& 28& 14& 7& 22& 11& 34& 17& 52\\
26& 13& 40& 20& 10& 5& 16& 8& 4& 2\\
1& \\

813&&&&&&&&&\\
2440& 1220& 610& 305& 916& 458& 229& 688& 344& 172\\
86& 43& 130& 65& 196& 98& 49& 148& 74& 37\\
112& 56& 28& 14& 7& 22& 11& 34& 17& 52\\
26& 13& 40& 20& 10& 5& 16& 8& 4& 2\\
1& \\

814&&&&&&&&&\\
407& 1222& 611& 1834& 917& 2752& 1376& 688& 344& 172\\
86& 43& 130& 65& 196& 98& 49& 148& 74& 37\\
112& 56& 28& 14& 7& 22& 11& 34& 17& 52\\
26& 13& 40& 20& 10& 5& 16& 8& 4& 2\\
1& \\

815&&&&&&&&&\\
2446& 1223& 3670& 1835& 5506& 2753& 8260& 4130& 2065& 6196\\
3098& 1549& 4648& 2324& 1162& 581& 1744& 872& 436& 218\\
109& 328& 164& 82& 41& 124& 62& 31& 94& 47\\
142& 71& 214& 107& 322& 161& 484& 242& 121& 364\\
182& 91& 274& 137& 412& 206& 103& 310& 155& 466\\
233& 700& 350& 175& 526& 263& 790& 395& 1186& 593\\
1780& 890& 445& 1336& 668& 334& 167& 502& 251& 754\\
377& 1132& 566& 283& 850& 425& 1276& 638& 319& 958\\
479& 1438& 719& 2158& 1079& 3238& 1619& 4858& 2429& 7288\\
3644& 1822& 911& 2734& 1367& 4102& 2051& 6154& 3077& 9232\\
4616& 2308& 1154& 577& 1732& 866& 433& 1300& 650& 325\\
976& 488& 244& 122& 61& 184& 92& 46& 23& 70\\
35& 106& 53& 160& 80& 40& 20& 10& 5& 16\\
8& 4& 2& 1& \\

816&&&&&&&&&\\
408& 204& 102& 51& 154& 77& 232& 116& 58& 29\\
88& 44& 22& 11& 34& 17& 52& 26& 13& 40\\
20& 10& 5& 16& 8& 4& 2& 1& \\

817&&&&&&&&&\\
2452& 1226& 613& 1840& 920& 460& 230& 115& 346& 173\\
520& 260& 130& 65& 196& 98& 49& 148& 74& 37\\
112& 56& 28& 14& 7& 22& 11& 34& 17& 52\\
26& 13& 40& 20& 10& 5& 16& 8& 4& 2\\
1& \\

818&&&&&&&&&\\
409& 1228& 614& 307& 922& 461& 1384& 692& 346& 173\\
520& 260& 130& 65& 196& 98& 49& 148& 74& 37\\
112& 56& 28& 14& 7& 22& 11& 34& 17& 52\\
26& 13& 40& 20& 10& 5& 16& 8& 4& 2\\
1& \\

819&&&&&&&&&\\
2458& 1229& 3688& 1844& 922& 461& 1384& 692& 346& 173\\
520& 260& 130& 65& 196& 98& 49& 148& 74& 37\\
112& 56& 28& 14& 7& 22& 11& 34& 17& 52\\
26& 13& 40& 20& 10& 5& 16& 8& 4& 2\\
1& \\

820&&&&&&&&&\\
410& 205& 616& 308& 154& 77& 232& 116& 58& 29\\
88& 44& 22& 11& 34& 17& 52& 26& 13& 40\\
20& 10& 5& 16& 8& 4& 2& 1& \\

821&&&&&&&&&\\
2464& 1232& 616& 308& 154& 77& 232& 116& 58& 29\\
88& 44& 22& 11& 34& 17& 52& 26& 13& 40\\
20& 10& 5& 16& 8& 4& 2& 1& \\

822&&&&&&&&&\\
411& 1234& 617& 1852& 926& 463& 1390& 695& 2086& 1043\\
3130& 1565& 4696& 2348& 1174& 587& 1762& 881& 2644& 1322\\
661& 1984& 992& 496& 248& 124& 62& 31& 94& 47\\
142& 71& 214& 107& 322& 161& 484& 242& 121& 364\\
182& 91& 274& 137& 412& 206& 103& 310& 155& 466\\
233& 700& 350& 175& 526& 263& 790& 395& 1186& 593\\
1780& 890& 445& 1336& 668& 334& 167& 502& 251& 754\\
377& 1132& 566& 283& 850& 425& 1276& 638& 319& 958\\
479& 1438& 719& 2158& 1079& 3238& 1619& 4858& 2429& 7288\\
3644& 1822& 911& 2734& 1367& 4102& 2051& 6154& 3077& 9232\\
4616& 2308& 1154& 577& 1732& 866& 433& 1300& 650& 325\\
976& 488& 244& 122& 61& 184& 92& 46& 23& 70\\
35& 106& 53& 160& 80& 40& 20& 10& 5& 16\\
8& 4& 2& 1& \\

823&&&&&&&&&\\
2470& 1235& 3706& 1853& 5560& 2780& 1390& 695& 2086& 1043\\
3130& 1565& 4696& 2348& 1174& 587& 1762& 881& 2644& 1322\\
661& 1984& 992& 496& 248& 124& 62& 31& 94& 47\\
142& 71& 214& 107& 322& 161& 484& 242& 121& 364\\
182& 91& 274& 137& 412& 206& 103& 310& 155& 466\\
233& 700& 350& 175& 526& 263& 790& 395& 1186& 593\\
1780& 890& 445& 1336& 668& 334& 167& 502& 251& 754\\
377& 1132& 566& 283& 850& 425& 1276& 638& 319& 958\\
479& 1438& 719& 2158& 1079& 3238& 1619& 4858& 2429& 7288\\
3644& 1822& 911& 2734& 1367& 4102& 2051& 6154& 3077& 9232\\
4616& 2308& 1154& 577& 1732& 866& 433& 1300& 650& 325\\
976& 488& 244& 122& 61& 184& 92& 46& 23& 70\\
35& 106& 53& 160& 80& 40& 20& 10& 5& 16\\
8& 4& 2& 1& \\

824&&&&&&&&&\\
412& 206& 103& 310& 155& 466& 233& 700& 350& 175\\
526& 263& 790& 395& 1186& 593& 1780& 890& 445& 1336\\
668& 334& 167& 502& 251& 754& 377& 1132& 566& 283\\
850& 425& 1276& 638& 319& 958& 479& 1438& 719& 2158\\
1079& 3238& 1619& 4858& 2429& 7288& 3644& 1822& 911& 2734\\
1367& 4102& 2051& 6154& 3077& 9232& 4616& 2308& 1154& 577\\
1732& 866& 433& 1300& 650& 325& 976& 488& 244& 122\\
61& 184& 92& 46& 23& 70& 35& 106& 53& 160\\
80& 40& 20& 10& 5& 16& 8& 4& 2& 1\\

825&&&&&&&&&\\
2476& 1238& 619& 1858& 929& 2788& 1394& 697& 2092& 1046\\
523& 1570& 785& 2356& 1178& 589& 1768& 884& 442& 221\\
664& 332& 166& 83& 250& 125& 376& 188& 94& 47\\
142& 71& 214& 107& 322& 161& 484& 242& 121& 364\\
182& 91& 274& 137& 412& 206& 103& 310& 155& 466\\
233& 700& 350& 175& 526& 263& 790& 395& 1186& 593\\
1780& 890& 445& 1336& 668& 334& 167& 502& 251& 754\\
377& 1132& 566& 283& 850& 425& 1276& 638& 319& 958\\
479& 1438& 719& 2158& 1079& 3238& 1619& 4858& 2429& 7288\\
3644& 1822& 911& 2734& 1367& 4102& 2051& 6154& 3077& 9232\\
4616& 2308& 1154& 577& 1732& 866& 433& 1300& 650& 325\\
976& 488& 244& 122& 61& 184& 92& 46& 23& 70\\
35& 106& 53& 160& 80& 40& 20& 10& 5& 16\\
8& 4& 2& 1& \\

826&&&&&&&&&\\
413& 1240& 620& 310& 155& 466& 233& 700& 350& 175\\
526& 263& 790& 395& 1186& 593& 1780& 890& 445& 1336\\
668& 334& 167& 502& 251& 754& 377& 1132& 566& 283\\
850& 425& 1276& 638& 319& 958& 479& 1438& 719& 2158\\
1079& 3238& 1619& 4858& 2429& 7288& 3644& 1822& 911& 2734\\
1367& 4102& 2051& 6154& 3077& 9232& 4616& 2308& 1154& 577\\
1732& 866& 433& 1300& 650& 325& 976& 488& 244& 122\\
61& 184& 92& 46& 23& 70& 35& 106& 53& 160\\
80& 40& 20& 10& 5& 16& 8& 4& 2& 1\\

827&&&&&&&&&\\
2482& 1241& 3724& 1862& 931& 2794& 1397& 4192& 2096& 1048\\
524& 262& 131& 394& 197& 592& 296& 148& 74& 37\\
112& 56& 28& 14& 7& 22& 11& 34& 17& 52\\
26& 13& 40& 20& 10& 5& 16& 8& 4& 2\\
1& \\

828&&&&&&&&&\\
414& 207& 622& 311& 934& 467& 1402& 701& 2104& 1052\\
526& 263& 790& 395& 1186& 593& 1780& 890& 445& 1336\\
668& 334& 167& 502& 251& 754& 377& 1132& 566& 283\\
850& 425& 1276& 638& 319& 958& 479& 1438& 719& 2158\\
1079& 3238& 1619& 4858& 2429& 7288& 3644& 1822& 911& 2734\\
1367& 4102& 2051& 6154& 3077& 9232& 4616& 2308& 1154& 577\\
1732& 866& 433& 1300& 650& 325& 976& 488& 244& 122\\
61& 184& 92& 46& 23& 70& 35& 106& 53& 160\\
80& 40& 20& 10& 5& 16& 8& 4& 2& 1\\

829&&&&&&&&&\\
2488& 1244& 622& 311& 934& 467& 1402& 701& 2104& 1052\\
526& 263& 790& 395& 1186& 593& 1780& 890& 445& 1336\\
668& 334& 167& 502& 251& 754& 377& 1132& 566& 283\\
850& 425& 1276& 638& 319& 958& 479& 1438& 719& 2158\\
1079& 3238& 1619& 4858& 2429& 7288& 3644& 1822& 911& 2734\\
1367& 4102& 2051& 6154& 3077& 9232& 4616& 2308& 1154& 577\\
1732& 866& 433& 1300& 650& 325& 976& 488& 244& 122\\
61& 184& 92& 46& 23& 70& 35& 106& 53& 160\\
80& 40& 20& 10& 5& 16& 8& 4& 2& 1\\

830&&&&&&&&&\\
415& 1246& 623& 1870& 935& 2806& 1403& 4210& 2105& 6316\\
3158& 1579& 4738& 2369& 7108& 3554& 1777& 5332& 2666& 1333\\
4000& 2000& 1000& 500& 250& 125& 376& 188& 94& 47\\
142& 71& 214& 107& 322& 161& 484& 242& 121& 364\\
182& 91& 274& 137& 412& 206& 103& 310& 155& 466\\
233& 700& 350& 175& 526& 263& 790& 395& 1186& 593\\
1780& 890& 445& 1336& 668& 334& 167& 502& 251& 754\\
377& 1132& 566& 283& 850& 425& 1276& 638& 319& 958\\
479& 1438& 719& 2158& 1079& 3238& 1619& 4858& 2429& 7288\\
3644& 1822& 911& 2734& 1367& 4102& 2051& 6154& 3077& 9232\\
4616& 2308& 1154& 577& 1732& 866& 433& 1300& 650& 325\\
976& 488& 244& 122& 61& 184& 92& 46& 23& 70\\
35& 106& 53& 160& 80& 40& 20& 10& 5& 16\\
8& 4& 2& 1& \\

831&&&&&&&&&\\
2494& 1247& 3742& 1871& 5614& 2807& 8422& 4211& 12634& 6317\\
18952& 9476& 4738& 2369& 7108& 3554& 1777& 5332& 2666& 1333\\
4000& 2000& 1000& 500& 250& 125& 376& 188& 94& 47\\
142& 71& 214& 107& 322& 161& 484& 242& 121& 364\\
182& 91& 274& 137& 412& 206& 103& 310& 155& 466\\
233& 700& 350& 175& 526& 263& 790& 395& 1186& 593\\
1780& 890& 445& 1336& 668& 334& 167& 502& 251& 754\\
377& 1132& 566& 283& 850& 425& 1276& 638& 319& 958\\
479& 1438& 719& 2158& 1079& 3238& 1619& 4858& 2429& 7288\\
3644& 1822& 911& 2734& 1367& 4102& 2051& 6154& 3077& 9232\\
4616& 2308& 1154& 577& 1732& 866& 433& 1300& 650& 325\\
976& 488& 244& 122& 61& 184& 92& 46& 23& 70\\
35& 106& 53& 160& 80& 40& 20& 10& 5& 16\\
8& 4& 2& 1& \\

832&&&&&&&&&\\
416& 208& 104& 52& 26& 13& 40& 20& 10& 5\\
16& 8& 4& 2& 1& \\

833&&&&&&&&&\\
2500& 1250& 625& 1876& 938& 469& 1408& 704& 352& 176\\
88& 44& 22& 11& 34& 17& 52& 26& 13& 40\\
20& 10& 5& 16& 8& 4& 2& 1& \\

834&&&&&&&&&\\
417& 1252& 626& 313& 940& 470& 235& 706& 353& 1060\\
530& 265& 796& 398& 199& 598& 299& 898& 449& 1348\\
674& 337& 1012& 506& 253& 760& 380& 190& 95& 286\\
143& 430& 215& 646& 323& 970& 485& 1456& 728& 364\\
182& 91& 274& 137& 412& 206& 103& 310& 155& 466\\
233& 700& 350& 175& 526& 263& 790& 395& 1186& 593\\
1780& 890& 445& 1336& 668& 334& 167& 502& 251& 754\\
377& 1132& 566& 283& 850& 425& 1276& 638& 319& 958\\
479& 1438& 719& 2158& 1079& 3238& 1619& 4858& 2429& 7288\\
3644& 1822& 911& 2734& 1367& 4102& 2051& 6154& 3077& 9232\\
4616& 2308& 1154& 577& 1732& 866& 433& 1300& 650& 325\\
976& 488& 244& 122& 61& 184& 92& 46& 23& 70\\
35& 106& 53& 160& 80& 40& 20& 10& 5& 16\\
8& 4& 2& 1& \\

835&&&&&&&&&\\
2506& 1253& 3760& 1880& 940& 470& 235& 706& 353& 1060\\
530& 265& 796& 398& 199& 598& 299& 898& 449& 1348\\
674& 337& 1012& 506& 253& 760& 380& 190& 95& 286\\
143& 430& 215& 646& 323& 970& 485& 1456& 728& 364\\
182& 91& 274& 137& 412& 206& 103& 310& 155& 466\\
233& 700& 350& 175& 526& 263& 790& 395& 1186& 593\\
1780& 890& 445& 1336& 668& 334& 167& 502& 251& 754\\
377& 1132& 566& 283& 850& 425& 1276& 638& 319& 958\\
479& 1438& 719& 2158& 1079& 3238& 1619& 4858& 2429& 7288\\
3644& 1822& 911& 2734& 1367& 4102& 2051& 6154& 3077& 9232\\
4616& 2308& 1154& 577& 1732& 866& 433& 1300& 650& 325\\
976& 488& 244& 122& 61& 184& 92& 46& 23& 70\\
35& 106& 53& 160& 80& 40& 20& 10& 5& 16\\
8& 4& 2& 1& \\

836&&&&&&&&&\\
418& 209& 628& 314& 157& 472& 236& 118& 59& 178\\
89& 268& 134& 67& 202& 101& 304& 152& 76& 38\\
19& 58& 29& 88& 44& 22& 11& 34& 17& 52\\
26& 13& 40& 20& 10& 5& 16& 8& 4& 2\\
1& \\

837&&&&&&&&&\\
2512& 1256& 628& 314& 157& 472& 236& 118& 59& 178\\
89& 268& 134& 67& 202& 101& 304& 152& 76& 38\\
19& 58& 29& 88& 44& 22& 11& 34& 17& 52\\
26& 13& 40& 20& 10& 5& 16& 8& 4& 2\\
1& \\

838&&&&&&&&&\\
419& 1258& 629& 1888& 944& 472& 236& 118& 59& 178\\
89& 268& 134& 67& 202& 101& 304& 152& 76& 38\\
19& 58& 29& 88& 44& 22& 11& 34& 17& 52\\
26& 13& 40& 20& 10& 5& 16& 8& 4& 2\\
1& \\

839&&&&&&&&&\\
2518& 1259& 3778& 1889& 5668& 2834& 1417& 4252& 2126& 1063\\
3190& 1595& 4786& 2393& 7180& 3590& 1795& 5386& 2693& 8080\\
4040& 2020& 1010& 505& 1516& 758& 379& 1138& 569& 1708\\
854& 427& 1282& 641& 1924& 962& 481& 1444& 722& 361\\
1084& 542& 271& 814& 407& 1222& 611& 1834& 917& 2752\\
1376& 688& 344& 172& 86& 43& 130& 65& 196& 98\\
49& 148& 74& 37& 112& 56& 28& 14& 7& 22\\
11& 34& 17& 52& 26& 13& 40& 20& 10& 5\\
16& 8& 4& 2& 1& \\

840&&&&&&&&&\\
420& 210& 105& 316& 158& 79& 238& 119& 358& 179\\
538& 269& 808& 404& 202& 101& 304& 152& 76& 38\\
19& 58& 29& 88& 44& 22& 11& 34& 17& 52\\
26& 13& 40& 20& 10& 5& 16& 8& 4& 2\\
1& \\

841&&&&&&&&&\\
2524& 1262& 631& 1894& 947& 2842& 1421& 4264& 2132& 1066\\
533& 1600& 800& 400& 200& 100& 50& 25& 76& 38\\
19& 58& 29& 88& 44& 22& 11& 34& 17& 52\\
26& 13& 40& 20& 10& 5& 16& 8& 4& 2\\
1& \\

842&&&&&&&&&\\
421& 1264& 632& 316& 158& 79& 238& 119& 358& 179\\
538& 269& 808& 404& 202& 101& 304& 152& 76& 38\\
19& 58& 29& 88& 44& 22& 11& 34& 17& 52\\
26& 13& 40& 20& 10& 5& 16& 8& 4& 2\\
1& \\

843&&&&&&&&&\\
2530& 1265& 3796& 1898& 949& 2848& 1424& 712& 356& 178\\
89& 268& 134& 67& 202& 101& 304& 152& 76& 38\\
19& 58& 29& 88& 44& 22& 11& 34& 17& 52\\
26& 13& 40& 20& 10& 5& 16& 8& 4& 2\\
1& \\

844&&&&&&&&&\\
422& 211& 634& 317& 952& 476& 238& 119& 358& 179\\
538& 269& 808& 404& 202& 101& 304& 152& 76& 38\\
19& 58& 29& 88& 44& 22& 11& 34& 17& 52\\
26& 13& 40& 20& 10& 5& 16& 8& 4& 2\\
1& \\

845&&&&&&&&&\\
2536& 1268& 634& 317& 952& 476& 238& 119& 358& 179\\
538& 269& 808& 404& 202& 101& 304& 152& 76& 38\\
19& 58& 29& 88& 44& 22& 11& 34& 17& 52\\
26& 13& 40& 20& 10& 5& 16& 8& 4& 2\\
1& \\

846&&&&&&&&&\\
423& 1270& 635& 1906& 953& 2860& 1430& 715& 2146& 1073\\
3220& 1610& 805& 2416& 1208& 604& 302& 151& 454& 227\\
682& 341& 1024& 512& 256& 128& 64& 32& 16& 8\\
4& 2& 1& \\

847&&&&&&&&&\\
2542& 1271& 3814& 1907& 5722& 2861& 8584& 4292& 2146& 1073\\
3220& 1610& 805& 2416& 1208& 604& 302& 151& 454& 227\\
682& 341& 1024& 512& 256& 128& 64& 32& 16& 8\\
4& 2& 1& \\

848&&&&&&&&&\\
424& 212& 106& 53& 160& 80& 40& 20& 10& 5\\
16& 8& 4& 2& 1& \\

849&&&&&&&&&\\
2548& 1274& 637& 1912& 956& 478& 239& 718& 359& 1078\\
539& 1618& 809& 2428& 1214& 607& 1822& 911& 2734& 1367\\
4102& 2051& 6154& 3077& 9232& 4616& 2308& 1154& 577& 1732\\
866& 433& 1300& 650& 325& 976& 488& 244& 122& 61\\
184& 92& 46& 23& 70& 35& 106& 53& 160& 80\\
40& 20& 10& 5& 16& 8& 4& 2& 1& \\

850&&&&&&&&&\\
425& 1276& 638& 319& 958& 479& 1438& 719& 2158& 1079\\
3238& 1619& 4858& 2429& 7288& 3644& 1822& 911& 2734& 1367\\
4102& 2051& 6154& 3077& 9232& 4616& 2308& 1154& 577& 1732\\
866& 433& 1300& 650& 325& 976& 488& 244& 122& 61\\
184& 92& 46& 23& 70& 35& 106& 53& 160& 80\\
40& 20& 10& 5& 16& 8& 4& 2& 1& \\

851&&&&&&&&&\\
2554& 1277& 3832& 1916& 958& 479& 1438& 719& 2158& 1079\\
3238& 1619& 4858& 2429& 7288& 3644& 1822& 911& 2734& 1367\\
4102& 2051& 6154& 3077& 9232& 4616& 2308& 1154& 577& 1732\\
866& 433& 1300& 650& 325& 976& 488& 244& 122& 61\\
184& 92& 46& 23& 70& 35& 106& 53& 160& 80\\
40& 20& 10& 5& 16& 8& 4& 2& 1& \\

852&&&&&&&&&\\
426& 213& 640& 320& 160& 80& 40& 20& 10& 5\\
16& 8& 4& 2& 1& \\

853&&&&&&&&&\\
2560& 1280& 640& 320& 160& 80& 40& 20& 10& 5\\
16& 8& 4& 2& 1& \\

854&&&&&&&&&\\
427& 1282& 641& 1924& 962& 481& 1444& 722& 361& 1084\\
542& 271& 814& 407& 1222& 611& 1834& 917& 2752& 1376\\
688& 344& 172& 86& 43& 130& 65& 196& 98& 49\\
148& 74& 37& 112& 56& 28& 14& 7& 22& 11\\
34& 17& 52& 26& 13& 40& 20& 10& 5& 16\\
8& 4& 2& 1& \\

855&&&&&&&&&\\
2566& 1283& 3850& 1925& 5776& 2888& 1444& 722& 361& 1084\\
542& 271& 814& 407& 1222& 611& 1834& 917& 2752& 1376\\
688& 344& 172& 86& 43& 130& 65& 196& 98& 49\\
148& 74& 37& 112& 56& 28& 14& 7& 22& 11\\
34& 17& 52& 26& 13& 40& 20& 10& 5& 16\\
8& 4& 2& 1& \\

856&&&&&&&&&\\
428& 214& 107& 322& 161& 484& 242& 121& 364& 182\\
91& 274& 137& 412& 206& 103& 310& 155& 466& 233\\
700& 350& 175& 526& 263& 790& 395& 1186& 593& 1780\\
890& 445& 1336& 668& 334& 167& 502& 251& 754& 377\\
1132& 566& 283& 850& 425& 1276& 638& 319& 958& 479\\
1438& 719& 2158& 1079& 3238& 1619& 4858& 2429& 7288& 3644\\
1822& 911& 2734& 1367& 4102& 2051& 6154& 3077& 9232& 4616\\
2308& 1154& 577& 1732& 866& 433& 1300& 650& 325& 976\\
488& 244& 122& 61& 184& 92& 46& 23& 70& 35\\
106& 53& 160& 80& 40& 20& 10& 5& 16& 8\\
4& 2& 1& \\

857&&&&&&&&&\\
2572& 1286& 643& 1930& 965& 2896& 1448& 724& 362& 181\\
544& 272& 136& 68& 34& 17& 52& 26& 13& 40\\
20& 10& 5& 16& 8& 4& 2& 1& \\

858&&&&&&&&&\\
429& 1288& 644& 322& 161& 484& 242& 121& 364& 182\\
91& 274& 137& 412& 206& 103& 310& 155& 466& 233\\
700& 350& 175& 526& 263& 790& 395& 1186& 593& 1780\\
890& 445& 1336& 668& 334& 167& 502& 251& 754& 377\\
1132& 566& 283& 850& 425& 1276& 638& 319& 958& 479\\
1438& 719& 2158& 1079& 3238& 1619& 4858& 2429& 7288& 3644\\
1822& 911& 2734& 1367& 4102& 2051& 6154& 3077& 9232& 4616\\
2308& 1154& 577& 1732& 866& 433& 1300& 650& 325& 976\\
488& 244& 122& 61& 184& 92& 46& 23& 70& 35\\
106& 53& 160& 80& 40& 20& 10& 5& 16& 8\\
4& 2& 1& \\

859&&&&&&&&&\\
2578& 1289& 3868& 1934& 967& 2902& 1451& 4354& 2177& 6532\\
3266& 1633& 4900& 2450& 1225& 3676& 1838& 919& 2758& 1379\\
4138& 2069& 6208& 3104& 1552& 776& 388& 194& 97& 292\\
146& 73& 220& 110& 55& 166& 83& 250& 125& 376\\
188& 94& 47& 142& 71& 214& 107& 322& 161& 484\\
242& 121& 364& 182& 91& 274& 137& 412& 206& 103\\
310& 155& 466& 233& 700& 350& 175& 526& 263& 790\\
395& 1186& 593& 1780& 890& 445& 1336& 668& 334& 167\\
502& 251& 754& 377& 1132& 566& 283& 850& 425& 1276\\
638& 319& 958& 479& 1438& 719& 2158& 1079& 3238& 1619\\
4858& 2429& 7288& 3644& 1822& 911& 2734& 1367& 4102& 2051\\
6154& 3077& 9232& 4616& 2308& 1154& 577& 1732& 866& 433\\
1300& 650& 325& 976& 488& 244& 122& 61& 184& 92\\
46& 23& 70& 35& 106& 53& 160& 80& 40& 20\\
10& 5& 16& 8& 4& 2& 1& \\

860&&&&&&&&&\\
430& 215& 646& 323& 970& 485& 1456& 728& 364& 182\\
91& 274& 137& 412& 206& 103& 310& 155& 466& 233\\
700& 350& 175& 526& 263& 790& 395& 1186& 593& 1780\\
890& 445& 1336& 668& 334& 167& 502& 251& 754& 377\\
1132& 566& 283& 850& 425& 1276& 638& 319& 958& 479\\
1438& 719& 2158& 1079& 3238& 1619& 4858& 2429& 7288& 3644\\
1822& 911& 2734& 1367& 4102& 2051& 6154& 3077& 9232& 4616\\
2308& 1154& 577& 1732& 866& 433& 1300& 650& 325& 976\\
488& 244& 122& 61& 184& 92& 46& 23& 70& 35\\
106& 53& 160& 80& 40& 20& 10& 5& 16& 8\\
4& 2& 1& \\

861&&&&&&&&&\\
2584& 1292& 646& 323& 970& 485& 1456& 728& 364& 182\\
91& 274& 137& 412& 206& 103& 310& 155& 466& 233\\
700& 350& 175& 526& 263& 790& 395& 1186& 593& 1780\\
890& 445& 1336& 668& 334& 167& 502& 251& 754& 377\\
1132& 566& 283& 850& 425& 1276& 638& 319& 958& 479\\
1438& 719& 2158& 1079& 3238& 1619& 4858& 2429& 7288& 3644\\
1822& 911& 2734& 1367& 4102& 2051& 6154& 3077& 9232& 4616\\
2308& 1154& 577& 1732& 866& 433& 1300& 650& 325& 976\\
488& 244& 122& 61& 184& 92& 46& 23& 70& 35\\
106& 53& 160& 80& 40& 20& 10& 5& 16& 8\\
4& 2& 1& \\

862&&&&&&&&&\\
431& 1294& 647& 1942& 971& 2914& 1457& 4372& 2186& 1093\\
3280& 1640& 820& 410& 205& 616& 308& 154& 77& 232\\
116& 58& 29& 88& 44& 22& 11& 34& 17& 52\\
26& 13& 40& 20& 10& 5& 16& 8& 4& 2\\
1& \\

863&&&&&&&&&\\
2590& 1295& 3886& 1943& 5830& 2915& 8746& 4373& 13120& 6560\\
3280& 1640& 820& 410& 205& 616& 308& 154& 77& 232\\
116& 58& 29& 88& 44& 22& 11& 34& 17& 52\\
26& 13& 40& 20& 10& 5& 16& 8& 4& 2\\
1& \\

864&&&&&&&&&\\
432& 216& 108& 54& 27& 82& 41& 124& 62& 31\\
94& 47& 142& 71& 214& 107& 322& 161& 484& 242\\
121& 364& 182& 91& 274& 137& 412& 206& 103& 310\\
155& 466& 233& 700& 350& 175& 526& 263& 790& 395\\
1186& 593& 1780& 890& 445& 1336& 668& 334& 167& 502\\
251& 754& 377& 1132& 566& 283& 850& 425& 1276& 638\\
319& 958& 479& 1438& 719& 2158& 1079& 3238& 1619& 4858\\
2429& 7288& 3644& 1822& 911& 2734& 1367& 4102& 2051& 6154\\
3077& 9232& 4616& 2308& 1154& 577& 1732& 866& 433& 1300\\
650& 325& 976& 488& 244& 122& 61& 184& 92& 46\\
23& 70& 35& 106& 53& 160& 80& 40& 20& 10\\
5& 16& 8& 4& 2& 1& \\

865&&&&&&&&&\\
2596& 1298& 649& 1948& 974& 487& 1462& 731& 2194& 1097\\
3292& 1646& 823& 2470& 1235& 3706& 1853& 5560& 2780& 1390\\
695& 2086& 1043& 3130& 1565& 4696& 2348& 1174& 587& 1762\\
881& 2644& 1322& 661& 1984& 992& 496& 248& 124& 62\\
31& 94& 47& 142& 71& 214& 107& 322& 161& 484\\
242& 121& 364& 182& 91& 274& 137& 412& 206& 103\\
310& 155& 466& 233& 700& 350& 175& 526& 263& 790\\
395& 1186& 593& 1780& 890& 445& 1336& 668& 334& 167\\
502& 251& 754& 377& 1132& 566& 283& 850& 425& 1276\\
638& 319& 958& 479& 1438& 719& 2158& 1079& 3238& 1619\\
4858& 2429& 7288& 3644& 1822& 911& 2734& 1367& 4102& 2051\\
6154& 3077& 9232& 4616& 2308& 1154& 577& 1732& 866& 433\\
1300& 650& 325& 976& 488& 244& 122& 61& 184& 92\\
46& 23& 70& 35& 106& 53& 160& 80& 40& 20\\
10& 5& 16& 8& 4& 2& 1& \\

866&&&&&&&&&\\
433& 1300& 650& 325& 976& 488& 244& 122& 61& 184\\
92& 46& 23& 70& 35& 106& 53& 160& 80& 40\\
20& 10& 5& 16& 8& 4& 2& 1& \\

867&&&&&&&&&\\
2602& 1301& 3904& 1952& 976& 488& 244& 122& 61& 184\\
92& 46& 23& 70& 35& 106& 53& 160& 80& 40\\
20& 10& 5& 16& 8& 4& 2& 1& \\

868&&&&&&&&&\\
434& 217& 652& 326& 163& 490& 245& 736& 368& 184\\
92& 46& 23& 70& 35& 106& 53& 160& 80& 40\\
20& 10& 5& 16& 8& 4& 2& 1& \\

869&&&&&&&&&\\
2608& 1304& 652& 326& 163& 490& 245& 736& 368& 184\\
92& 46& 23& 70& 35& 106& 53& 160& 80& 40\\
20& 10& 5& 16& 8& 4& 2& 1& \\

870&&&&&&&&&\\
435& 1306& 653& 1960& 980& 490& 245& 736& 368& 184\\
92& 46& 23& 70& 35& 106& 53& 160& 80& 40\\
20& 10& 5& 16& 8& 4& 2& 1& \\

871&&&&&&&&&\\
2614& 1307& 3922& 1961& 5884& 2942& 1471& 4414& 2207& 6622\\
3311& 9934& 4967& 14902& 7451& 22354& 11177& 33532& 16766& 8383\\
25150& 12575& 37726& 18863& 56590& 28295& 84886& 42443& 127330& 63665\\
190996& 95498& 47749& 143248& 71624& 35812& 17906& 8953& 26860& 13430\\
6715& 20146& 10073& 30220& 15110& 7555& 22666& 11333& 34000& 17000\\
8500& 4250& 2125& 6376& 3188& 1594& 797& 2392& 1196& 598\\
299& 898& 449& 1348& 674& 337& 1012& 506& 253& 760\\
380& 190& 95& 286& 143& 430& 215& 646& 323& 970\\
485& 1456& 728& 364& 182& 91& 274& 137& 412& 206\\
103& 310& 155& 466& 233& 700& 350& 175& 526& 263\\
790& 395& 1186& 593& 1780& 890& 445& 1336& 668& 334\\
167& 502& 251& 754& 377& 1132& 566& 283& 850& 425\\
1276& 638& 319& 958& 479& 1438& 719& 2158& 1079& 3238\\
1619& 4858& 2429& 7288& 3644& 1822& 911& 2734& 1367& 4102\\
2051& 6154& 3077& 9232& 4616& 2308& 1154& 577& 1732& 866\\
433& 1300& 650& 325& 976& 488& 244& 122& 61& 184\\
92& 46& 23& 70& 35& 106& 53& 160& 80& 40\\
20& 10& 5& 16& 8& 4& 2& 1& \\

872&&&&&&&&&\\
436& 218& 109& 328& 164& 82& 41& 124& 62& 31\\
94& 47& 142& 71& 214& 107& 322& 161& 484& 242\\
121& 364& 182& 91& 274& 137& 412& 206& 103& 310\\
155& 466& 233& 700& 350& 175& 526& 263& 790& 395\\
1186& 593& 1780& 890& 445& 1336& 668& 334& 167& 502\\
251& 754& 377& 1132& 566& 283& 850& 425& 1276& 638\\
319& 958& 479& 1438& 719& 2158& 1079& 3238& 1619& 4858\\
2429& 7288& 3644& 1822& 911& 2734& 1367& 4102& 2051& 6154\\
3077& 9232& 4616& 2308& 1154& 577& 1732& 866& 433& 1300\\
650& 325& 976& 488& 244& 122& 61& 184& 92& 46\\
23& 70& 35& 106& 53& 160& 80& 40& 20& 10\\
5& 16& 8& 4& 2& 1& \\

873&&&&&&&&&\\
2620& 1310& 655& 1966& 983& 2950& 1475& 4426& 2213& 6640\\
3320& 1660& 830& 415& 1246& 623& 1870& 935& 2806& 1403\\
4210& 2105& 6316& 3158& 1579& 4738& 2369& 7108& 3554& 1777\\
5332& 2666& 1333& 4000& 2000& 1000& 500& 250& 125& 376\\
188& 94& 47& 142& 71& 214& 107& 322& 161& 484\\
242& 121& 364& 182& 91& 274& 137& 412& 206& 103\\
310& 155& 466& 233& 700& 350& 175& 526& 263& 790\\
395& 1186& 593& 1780& 890& 445& 1336& 668& 334& 167\\
502& 251& 754& 377& 1132& 566& 283& 850& 425& 1276\\
638& 319& 958& 479& 1438& 719& 2158& 1079& 3238& 1619\\
4858& 2429& 7288& 3644& 1822& 911& 2734& 1367& 4102& 2051\\
6154& 3077& 9232& 4616& 2308& 1154& 577& 1732& 866& 433\\
1300& 650& 325& 976& 488& 244& 122& 61& 184& 92\\
46& 23& 70& 35& 106& 53& 160& 80& 40& 20\\
10& 5& 16& 8& 4& 2& 1& \\

874&&&&&&&&&\\
437& 1312& 656& 328& 164& 82& 41& 124& 62& 31\\
94& 47& 142& 71& 214& 107& 322& 161& 484& 242\\
121& 364& 182& 91& 274& 137& 412& 206& 103& 310\\
155& 466& 233& 700& 350& 175& 526& 263& 790& 395\\
1186& 593& 1780& 890& 445& 1336& 668& 334& 167& 502\\
251& 754& 377& 1132& 566& 283& 850& 425& 1276& 638\\
319& 958& 479& 1438& 719& 2158& 1079& 3238& 1619& 4858\\
2429& 7288& 3644& 1822& 911& 2734& 1367& 4102& 2051& 6154\\
3077& 9232& 4616& 2308& 1154& 577& 1732& 866& 433& 1300\\
650& 325& 976& 488& 244& 122& 61& 184& 92& 46\\
23& 70& 35& 106& 53& 160& 80& 40& 20& 10\\
5& 16& 8& 4& 2& 1& \\

875&&&&&&&&&\\
2626& 1313& 3940& 1970& 985& 2956& 1478& 739& 2218& 1109\\
3328& 1664& 832& 416& 208& 104& 52& 26& 13& 40\\
20& 10& 5& 16& 8& 4& 2& 1& \\

876&&&&&&&&&\\
438& 219& 658& 329& 988& 494& 247& 742& 371& 1114\\
557& 1672& 836& 418& 209& 628& 314& 157& 472& 236\\
118& 59& 178& 89& 268& 134& 67& 202& 101& 304\\
152& 76& 38& 19& 58& 29& 88& 44& 22& 11\\
34& 17& 52& 26& 13& 40& 20& 10& 5& 16\\
8& 4& 2& 1& \\

877&&&&&&&&&\\
2632& 1316& 658& 329& 988& 494& 247& 742& 371& 1114\\
557& 1672& 836& 418& 209& 628& 314& 157& 472& 236\\
118& 59& 178& 89& 268& 134& 67& 202& 101& 304\\
152& 76& 38& 19& 58& 29& 88& 44& 22& 11\\
34& 17& 52& 26& 13& 40& 20& 10& 5& 16\\
8& 4& 2& 1& \\

878&&&&&&&&&\\
439& 1318& 659& 1978& 989& 2968& 1484& 742& 371& 1114\\
557& 1672& 836& 418& 209& 628& 314& 157& 472& 236\\
118& 59& 178& 89& 268& 134& 67& 202& 101& 304\\
152& 76& 38& 19& 58& 29& 88& 44& 22& 11\\
34& 17& 52& 26& 13& 40& 20& 10& 5& 16\\
8& 4& 2& 1& \\

879&&&&&&&&&\\
2638& 1319& 3958& 1979& 5938& 2969& 8908& 4454& 2227& 6682\\
3341& 10024& 5012& 2506& 1253& 3760& 1880& 940& 470& 235\\
706& 353& 1060& 530& 265& 796& 398& 199& 598& 299\\
898& 449& 1348& 674& 337& 1012& 506& 253& 760& 380\\
190& 95& 286& 143& 430& 215& 646& 323& 970& 485\\
1456& 728& 364& 182& 91& 274& 137& 412& 206& 103\\
310& 155& 466& 233& 700& 350& 175& 526& 263& 790\\
395& 1186& 593& 1780& 890& 445& 1336& 668& 334& 167\\
502& 251& 754& 377& 1132& 566& 283& 850& 425& 1276\\
638& 319& 958& 479& 1438& 719& 2158& 1079& 3238& 1619\\
4858& 2429& 7288& 3644& 1822& 911& 2734& 1367& 4102& 2051\\
6154& 3077& 9232& 4616& 2308& 1154& 577& 1732& 866& 433\\
1300& 650& 325& 976& 488& 244& 122& 61& 184& 92\\
46& 23& 70& 35& 106& 53& 160& 80& 40& 20\\
10& 5& 16& 8& 4& 2& 1& \\

880&&&&&&&&&\\
440& 220& 110& 55& 166& 83& 250& 125& 376& 188\\
94& 47& 142& 71& 214& 107& 322& 161& 484& 242\\
121& 364& 182& 91& 274& 137& 412& 206& 103& 310\\
155& 466& 233& 700& 350& 175& 526& 263& 790& 395\\
1186& 593& 1780& 890& 445& 1336& 668& 334& 167& 502\\
251& 754& 377& 1132& 566& 283& 850& 425& 1276& 638\\
319& 958& 479& 1438& 719& 2158& 1079& 3238& 1619& 4858\\
2429& 7288& 3644& 1822& 911& 2734& 1367& 4102& 2051& 6154\\
3077& 9232& 4616& 2308& 1154& 577& 1732& 866& 433& 1300\\
650& 325& 976& 488& 244& 122& 61& 184& 92& 46\\
23& 70& 35& 106& 53& 160& 80& 40& 20& 10\\
5& 16& 8& 4& 2& 1& \\

881&&&&&&&&&\\
2644& 1322& 661& 1984& 992& 496& 248& 124& 62& 31\\
94& 47& 142& 71& 214& 107& 322& 161& 484& 242\\
121& 364& 182& 91& 274& 137& 412& 206& 103& 310\\
155& 466& 233& 700& 350& 175& 526& 263& 790& 395\\
1186& 593& 1780& 890& 445& 1336& 668& 334& 167& 502\\
251& 754& 377& 1132& 566& 283& 850& 425& 1276& 638\\
319& 958& 479& 1438& 719& 2158& 1079& 3238& 1619& 4858\\
2429& 7288& 3644& 1822& 911& 2734& 1367& 4102& 2051& 6154\\
3077& 9232& 4616& 2308& 1154& 577& 1732& 866& 433& 1300\\
650& 325& 976& 488& 244& 122& 61& 184& 92& 46\\
23& 70& 35& 106& 53& 160& 80& 40& 20& 10\\
5& 16& 8& 4& 2& 1& \\

882&&&&&&&&&\\
441& 1324& 662& 331& 994& 497& 1492& 746& 373& 1120\\
560& 280& 140& 70& 35& 106& 53& 160& 80& 40\\
20& 10& 5& 16& 8& 4& 2& 1& \\

883&&&&&&&&&\\
2650& 1325& 3976& 1988& 994& 497& 1492& 746& 373& 1120\\
560& 280& 140& 70& 35& 106& 53& 160& 80& 40\\
20& 10& 5& 16& 8& 4& 2& 1& \\

884&&&&&&&&&\\
442& 221& 664& 332& 166& 83& 250& 125& 376& 188\\
94& 47& 142& 71& 214& 107& 322& 161& 484& 242\\
121& 364& 182& 91& 274& 137& 412& 206& 103& 310\\
155& 466& 233& 700& 350& 175& 526& 263& 790& 395\\
1186& 593& 1780& 890& 445& 1336& 668& 334& 167& 502\\
251& 754& 377& 1132& 566& 283& 850& 425& 1276& 638\\
319& 958& 479& 1438& 719& 2158& 1079& 3238& 1619& 4858\\
2429& 7288& 3644& 1822& 911& 2734& 1367& 4102& 2051& 6154\\
3077& 9232& 4616& 2308& 1154& 577& 1732& 866& 433& 1300\\
650& 325& 976& 488& 244& 122& 61& 184& 92& 46\\
23& 70& 35& 106& 53& 160& 80& 40& 20& 10\\
5& 16& 8& 4& 2& 1& \\

885&&&&&&&&&\\
2656& 1328& 664& 332& 166& 83& 250& 125& 376& 188\\
94& 47& 142& 71& 214& 107& 322& 161& 484& 242\\
121& 364& 182& 91& 274& 137& 412& 206& 103& 310\\
155& 466& 233& 700& 350& 175& 526& 263& 790& 395\\
1186& 593& 1780& 890& 445& 1336& 668& 334& 167& 502\\
251& 754& 377& 1132& 566& 283& 850& 425& 1276& 638\\
319& 958& 479& 1438& 719& 2158& 1079& 3238& 1619& 4858\\
2429& 7288& 3644& 1822& 911& 2734& 1367& 4102& 2051& 6154\\
3077& 9232& 4616& 2308& 1154& 577& 1732& 866& 433& 1300\\
650& 325& 976& 488& 244& 122& 61& 184& 92& 46\\
23& 70& 35& 106& 53& 160& 80& 40& 20& 10\\
5& 16& 8& 4& 2& 1& \\

886&&&&&&&&&\\
443& 1330& 665& 1996& 998& 499& 1498& 749& 2248& 1124\\
562& 281& 844& 422& 211& 634& 317& 952& 476& 238\\
119& 358& 179& 538& 269& 808& 404& 202& 101& 304\\
152& 76& 38& 19& 58& 29& 88& 44& 22& 11\\
34& 17& 52& 26& 13& 40& 20& 10& 5& 16\\
8& 4& 2& 1& \\

887&&&&&&&&&\\
2662& 1331& 3994& 1997& 5992& 2996& 1498& 749& 2248& 1124\\
562& 281& 844& 422& 211& 634& 317& 952& 476& 238\\
119& 358& 179& 538& 269& 808& 404& 202& 101& 304\\
152& 76& 38& 19& 58& 29& 88& 44& 22& 11\\
34& 17& 52& 26& 13& 40& 20& 10& 5& 16\\
8& 4& 2& 1& \\

888&&&&&&&&&\\
444& 222& 111& 334& 167& 502& 251& 754& 377& 1132\\
566& 283& 850& 425& 1276& 638& 319& 958& 479& 1438\\
719& 2158& 1079& 3238& 1619& 4858& 2429& 7288& 3644& 1822\\
911& 2734& 1367& 4102& 2051& 6154& 3077& 9232& 4616& 2308\\
1154& 577& 1732& 866& 433& 1300& 650& 325& 976& 488\\
244& 122& 61& 184& 92& 46& 23& 70& 35& 106\\
53& 160& 80& 40& 20& 10& 5& 16& 8& 4\\
2& 1& \\

889&&&&&&&&&\\
2668& 1334& 667& 2002& 1001& 3004& 1502& 751& 2254& 1127\\
3382& 1691& 5074& 2537& 7612& 3806& 1903& 5710& 2855& 8566\\
4283& 12850& 6425& 19276& 9638& 4819& 14458& 7229& 21688& 10844\\
5422& 2711& 8134& 4067& 12202& 6101& 18304& 9152& 4576& 2288\\
1144& 572& 286& 143& 430& 215& 646& 323& 970& 485\\
1456& 728& 364& 182& 91& 274& 137& 412& 206& 103\\
310& 155& 466& 233& 700& 350& 175& 526& 263& 790\\
395& 1186& 593& 1780& 890& 445& 1336& 668& 334& 167\\
502& 251& 754& 377& 1132& 566& 283& 850& 425& 1276\\
638& 319& 958& 479& 1438& 719& 2158& 1079& 3238& 1619\\
4858& 2429& 7288& 3644& 1822& 911& 2734& 1367& 4102& 2051\\
6154& 3077& 9232& 4616& 2308& 1154& 577& 1732& 866& 433\\
1300& 650& 325& 976& 488& 244& 122& 61& 184& 92\\
46& 23& 70& 35& 106& 53& 160& 80& 40& 20\\
10& 5& 16& 8& 4& 2& 1& \\

890&&&&&&&&&\\
445& 1336& 668& 334& 167& 502& 251& 754& 377& 1132\\
566& 283& 850& 425& 1276& 638& 319& 958& 479& 1438\\
719& 2158& 1079& 3238& 1619& 4858& 2429& 7288& 3644& 1822\\
911& 2734& 1367& 4102& 2051& 6154& 3077& 9232& 4616& 2308\\
1154& 577& 1732& 866& 433& 1300& 650& 325& 976& 488\\
244& 122& 61& 184& 92& 46& 23& 70& 35& 106\\
53& 160& 80& 40& 20& 10& 5& 16& 8& 4\\
2& 1& \\

891&&&&&&&&&\\
2674& 1337& 4012& 2006& 1003& 3010& 1505& 4516& 2258& 1129\\
3388& 1694& 847& 2542& 1271& 3814& 1907& 5722& 2861& 8584\\
4292& 2146& 1073& 3220& 1610& 805& 2416& 1208& 604& 302\\
151& 454& 227& 682& 341& 1024& 512& 256& 128& 64\\
32& 16& 8& 4& 2& 1& \\

892&&&&&&&&&\\
446& 223& 670& 335& 1006& 503& 1510& 755& 2266& 1133\\
3400& 1700& 850& 425& 1276& 638& 319& 958& 479& 1438\\
719& 2158& 1079& 3238& 1619& 4858& 2429& 7288& 3644& 1822\\
911& 2734& 1367& 4102& 2051& 6154& 3077& 9232& 4616& 2308\\
1154& 577& 1732& 866& 433& 1300& 650& 325& 976& 488\\
244& 122& 61& 184& 92& 46& 23& 70& 35& 106\\
53& 160& 80& 40& 20& 10& 5& 16& 8& 4\\
2& 1& \\

893&&&&&&&&&\\
2680& 1340& 670& 335& 1006& 503& 1510& 755& 2266& 1133\\
3400& 1700& 850& 425& 1276& 638& 319& 958& 479& 1438\\
719& 2158& 1079& 3238& 1619& 4858& 2429& 7288& 3644& 1822\\
911& 2734& 1367& 4102& 2051& 6154& 3077& 9232& 4616& 2308\\
1154& 577& 1732& 866& 433& 1300& 650& 325& 976& 488\\
244& 122& 61& 184& 92& 46& 23& 70& 35& 106\\
53& 160& 80& 40& 20& 10& 5& 16& 8& 4\\
2& 1& \\

894&&&&&&&&&\\
447& 1342& 671& 2014& 1007& 3022& 1511& 4534& 2267& 6802\\
3401& 10204& 5102& 2551& 7654& 3827& 11482& 5741& 17224& 8612\\
4306& 2153& 6460& 3230& 1615& 4846& 2423& 7270& 3635& 10906\\
5453& 16360& 8180& 4090& 2045& 6136& 3068& 1534& 767& 2302\\
1151& 3454& 1727& 5182& 2591& 7774& 3887& 11662& 5831& 17494\\
8747& 26242& 13121& 39364& 19682& 9841& 29524& 14762& 7381& 22144\\
11072& 5536& 2768& 1384& 692& 346& 173& 520& 260& 130\\
65& 196& 98& 49& 148& 74& 37& 112& 56& 28\\
14& 7& 22& 11& 34& 17& 52& 26& 13& 40\\
20& 10& 5& 16& 8& 4& 2& 1& \\

895&&&&&&&&&\\
2686& 1343& 4030& 2015& 6046& 3023& 9070& 4535& 13606& 6803\\
20410& 10205& 30616& 15308& 7654& 3827& 11482& 5741& 17224& 8612\\
4306& 2153& 6460& 3230& 1615& 4846& 2423& 7270& 3635& 10906\\
5453& 16360& 8180& 4090& 2045& 6136& 3068& 1534& 767& 2302\\
1151& 3454& 1727& 5182& 2591& 7774& 3887& 11662& 5831& 17494\\
8747& 26242& 13121& 39364& 19682& 9841& 29524& 14762& 7381& 22144\\
11072& 5536& 2768& 1384& 692& 346& 173& 520& 260& 130\\
65& 196& 98& 49& 148& 74& 37& 112& 56& 28\\
14& 7& 22& 11& 34& 17& 52& 26& 13& 40\\
20& 10& 5& 16& 8& 4& 2& 1& \\

896&&&&&&&&&\\
448& 224& 112& 56& 28& 14& 7& 22& 11& 34\\
17& 52& 26& 13& 40& 20& 10& 5& 16& 8\\
4& 2& 1& \\

897&&&&&&&&&\\
2692& 1346& 673& 2020& 1010& 505& 1516& 758& 379& 1138\\
569& 1708& 854& 427& 1282& 641& 1924& 962& 481& 1444\\
722& 361& 1084& 542& 271& 814& 407& 1222& 611& 1834\\
917& 2752& 1376& 688& 344& 172& 86& 43& 130& 65\\
196& 98& 49& 148& 74& 37& 112& 56& 28& 14\\
7& 22& 11& 34& 17& 52& 26& 13& 40& 20\\
10& 5& 16& 8& 4& 2& 1& \\

898&&&&&&&&&\\
449& 1348& 674& 337& 1012& 506& 253& 760& 380& 190\\
95& 286& 143& 430& 215& 646& 323& 970& 485& 1456\\
728& 364& 182& 91& 274& 137& 412& 206& 103& 310\\
155& 466& 233& 700& 350& 175& 526& 263& 790& 395\\
1186& 593& 1780& 890& 445& 1336& 668& 334& 167& 502\\
251& 754& 377& 1132& 566& 283& 850& 425& 1276& 638\\
319& 958& 479& 1438& 719& 2158& 1079& 3238& 1619& 4858\\
2429& 7288& 3644& 1822& 911& 2734& 1367& 4102& 2051& 6154\\
3077& 9232& 4616& 2308& 1154& 577& 1732& 866& 433& 1300\\
650& 325& 976& 488& 244& 122& 61& 184& 92& 46\\
23& 70& 35& 106& 53& 160& 80& 40& 20& 10\\
5& 16& 8& 4& 2& 1& \\

899&&&&&&&&&\\
2698& 1349& 4048& 2024& 1012& 506& 253& 760& 380& 190\\
95& 286& 143& 430& 215& 646& 323& 970& 485& 1456\\
728& 364& 182& 91& 274& 137& 412& 206& 103& 310\\
155& 466& 233& 700& 350& 175& 526& 263& 790& 395\\
1186& 593& 1780& 890& 445& 1336& 668& 334& 167& 502\\
251& 754& 377& 1132& 566& 283& 850& 425& 1276& 638\\
319& 958& 479& 1438& 719& 2158& 1079& 3238& 1619& 4858\\
2429& 7288& 3644& 1822& 911& 2734& 1367& 4102& 2051& 6154\\
3077& 9232& 4616& 2308& 1154& 577& 1732& 866& 433& 1300\\
650& 325& 976& 488& 244& 122& 61& 184& 92& 46\\
23& 70& 35& 106& 53& 160& 80& 40& 20& 10\\
5& 16& 8& 4& 2& 1& \\

900&&&&&&&&&\\
450& 225& 676& 338& 169& 508& 254& 127& 382& 191\\
574& 287& 862& 431& 1294& 647& 1942& 971& 2914& 1457\\
4372& 2186& 1093& 3280& 1640& 820& 410& 205& 616& 308\\
154& 77& 232& 116& 58& 29& 88& 44& 22& 11\\
34& 17& 52& 26& 13& 40& 20& 10& 5& 16\\
8& 4& 2& 1& \\

901&&&&&&&&&\\
2704& 1352& 676& 338& 169& 508& 254& 127& 382& 191\\
574& 287& 862& 431& 1294& 647& 1942& 971& 2914& 1457\\
4372& 2186& 1093& 3280& 1640& 820& 410& 205& 616& 308\\
154& 77& 232& 116& 58& 29& 88& 44& 22& 11\\
34& 17& 52& 26& 13& 40& 20& 10& 5& 16\\
8& 4& 2& 1& \\

902&&&&&&&&&\\
451& 1354& 677& 2032& 1016& 508& 254& 127& 382& 191\\
574& 287& 862& 431& 1294& 647& 1942& 971& 2914& 1457\\
4372& 2186& 1093& 3280& 1640& 820& 410& 205& 616& 308\\
154& 77& 232& 116& 58& 29& 88& 44& 22& 11\\
34& 17& 52& 26& 13& 40& 20& 10& 5& 16\\
8& 4& 2& 1& \\

903&&&&&&&&&\\
2710& 1355& 4066& 2033& 6100& 3050& 1525& 4576& 2288& 1144\\
572& 286& 143& 430& 215& 646& 323& 970& 485& 1456\\
728& 364& 182& 91& 274& 137& 412& 206& 103& 310\\
155& 466& 233& 700& 350& 175& 526& 263& 790& 395\\
1186& 593& 1780& 890& 445& 1336& 668& 334& 167& 502\\
251& 754& 377& 1132& 566& 283& 850& 425& 1276& 638\\
319& 958& 479& 1438& 719& 2158& 1079& 3238& 1619& 4858\\
2429& 7288& 3644& 1822& 911& 2734& 1367& 4102& 2051& 6154\\
3077& 9232& 4616& 2308& 1154& 577& 1732& 866& 433& 1300\\
650& 325& 976& 488& 244& 122& 61& 184& 92& 46\\
23& 70& 35& 106& 53& 160& 80& 40& 20& 10\\
5& 16& 8& 4& 2& 1& \\

904&&&&&&&&&\\
452& 226& 113& 340& 170& 85& 256& 128& 64& 32\\
16& 8& 4& 2& 1& \\

905&&&&&&&&&\\
2716& 1358& 679& 2038& 1019& 3058& 1529& 4588& 2294& 1147\\
3442& 1721& 5164& 2582& 1291& 3874& 1937& 5812& 2906& 1453\\
4360& 2180& 1090& 545& 1636& 818& 409& 1228& 614& 307\\
922& 461& 1384& 692& 346& 173& 520& 260& 130& 65\\
196& 98& 49& 148& 74& 37& 112& 56& 28& 14\\
7& 22& 11& 34& 17& 52& 26& 13& 40& 20\\
10& 5& 16& 8& 4& 2& 1& \\

906&&&&&&&&&\\
453& 1360& 680& 340& 170& 85& 256& 128& 64& 32\\
16& 8& 4& 2& 1& \\

907&&&&&&&&&\\
2722& 1361& 4084& 2042& 1021& 3064& 1532& 766& 383& 1150\\
575& 1726& 863& 2590& 1295& 3886& 1943& 5830& 2915& 8746\\
4373& 13120& 6560& 3280& 1640& 820& 410& 205& 616& 308\\
154& 77& 232& 116& 58& 29& 88& 44& 22& 11\\
34& 17& 52& 26& 13& 40& 20& 10& 5& 16\\
8& 4& 2& 1& \\

908&&&&&&&&&\\
454& 227& 682& 341& 1024& 512& 256& 128& 64& 32\\
16& 8& 4& 2& 1& \\

909&&&&&&&&&\\
2728& 1364& 682& 341& 1024& 512& 256& 128& 64& 32\\
16& 8& 4& 2& 1& \\

910&&&&&&&&&\\
455& 1366& 683& 2050& 1025& 3076& 1538& 769& 2308& 1154\\
577& 1732& 866& 433& 1300& 650& 325& 976& 488& 244\\
122& 61& 184& 92& 46& 23& 70& 35& 106& 53\\
160& 80& 40& 20& 10& 5& 16& 8& 4& 2\\
1& \\

911&&&&&&&&&\\
2734& 1367& 4102& 2051& 6154& 3077& 9232& 4616& 2308& 1154\\
577& 1732& 866& 433& 1300& 650& 325& 976& 488& 244\\
122& 61& 184& 92& 46& 23& 70& 35& 106& 53\\
160& 80& 40& 20& 10& 5& 16& 8& 4& 2\\
1& \\

912&&&&&&&&&\\
456& 228& 114& 57& 172& 86& 43& 130& 65& 196\\
98& 49& 148& 74& 37& 112& 56& 28& 14& 7\\
22& 11& 34& 17& 52& 26& 13& 40& 20& 10\\
5& 16& 8& 4& 2& 1& \\

913&&&&&&&&&\\
2740& 1370& 685& 2056& 1028& 514& 257& 772& 386& 193\\
580& 290& 145& 436& 218& 109& 328& 164& 82& 41\\
124& 62& 31& 94& 47& 142& 71& 214& 107& 322\\
161& 484& 242& 121& 364& 182& 91& 274& 137& 412\\
206& 103& 310& 155& 466& 233& 700& 350& 175& 526\\
263& 790& 395& 1186& 593& 1780& 890& 445& 1336& 668\\
334& 167& 502& 251& 754& 377& 1132& 566& 283& 850\\
425& 1276& 638& 319& 958& 479& 1438& 719& 2158& 1079\\
3238& 1619& 4858& 2429& 7288& 3644& 1822& 911& 2734& 1367\\
4102& 2051& 6154& 3077& 9232& 4616& 2308& 1154& 577& 1732\\
866& 433& 1300& 650& 325& 976& 488& 244& 122& 61\\
184& 92& 46& 23& 70& 35& 106& 53& 160& 80\\
40& 20& 10& 5& 16& 8& 4& 2& 1& \\

914&&&&&&&&&\\
457& 1372& 686& 343& 1030& 515& 1546& 773& 2320& 1160\\
580& 290& 145& 436& 218& 109& 328& 164& 82& 41\\
124& 62& 31& 94& 47& 142& 71& 214& 107& 322\\
161& 484& 242& 121& 364& 182& 91& 274& 137& 412\\
206& 103& 310& 155& 466& 233& 700& 350& 175& 526\\
263& 790& 395& 1186& 593& 1780& 890& 445& 1336& 668\\
334& 167& 502& 251& 754& 377& 1132& 566& 283& 850\\
425& 1276& 638& 319& 958& 479& 1438& 719& 2158& 1079\\
3238& 1619& 4858& 2429& 7288& 3644& 1822& 911& 2734& 1367\\
4102& 2051& 6154& 3077& 9232& 4616& 2308& 1154& 577& 1732\\
866& 433& 1300& 650& 325& 976& 488& 244& 122& 61\\
184& 92& 46& 23& 70& 35& 106& 53& 160& 80\\
40& 20& 10& 5& 16& 8& 4& 2& 1& \\

915&&&&&&&&&\\
2746& 1373& 4120& 2060& 1030& 515& 1546& 773& 2320& 1160\\
580& 290& 145& 436& 218& 109& 328& 164& 82& 41\\
124& 62& 31& 94& 47& 142& 71& 214& 107& 322\\
161& 484& 242& 121& 364& 182& 91& 274& 137& 412\\
206& 103& 310& 155& 466& 233& 700& 350& 175& 526\\
263& 790& 395& 1186& 593& 1780& 890& 445& 1336& 668\\
334& 167& 502& 251& 754& 377& 1132& 566& 283& 850\\
425& 1276& 638& 319& 958& 479& 1438& 719& 2158& 1079\\
3238& 1619& 4858& 2429& 7288& 3644& 1822& 911& 2734& 1367\\
4102& 2051& 6154& 3077& 9232& 4616& 2308& 1154& 577& 1732\\
866& 433& 1300& 650& 325& 976& 488& 244& 122& 61\\
184& 92& 46& 23& 70& 35& 106& 53& 160& 80\\
40& 20& 10& 5& 16& 8& 4& 2& 1& \\

916&&&&&&&&&\\
458& 229& 688& 344& 172& 86& 43& 130& 65& 196\\
98& 49& 148& 74& 37& 112& 56& 28& 14& 7\\
22& 11& 34& 17& 52& 26& 13& 40& 20& 10\\
5& 16& 8& 4& 2& 1& \\

917&&&&&&&&&\\
2752& 1376& 688& 344& 172& 86& 43& 130& 65& 196\\
98& 49& 148& 74& 37& 112& 56& 28& 14& 7\\
22& 11& 34& 17& 52& 26& 13& 40& 20& 10\\
5& 16& 8& 4& 2& 1& \\

918&&&&&&&&&\\
459& 1378& 689& 2068& 1034& 517& 1552& 776& 388& 194\\
97& 292& 146& 73& 220& 110& 55& 166& 83& 250\\
125& 376& 188& 94& 47& 142& 71& 214& 107& 322\\
161& 484& 242& 121& 364& 182& 91& 274& 137& 412\\
206& 103& 310& 155& 466& 233& 700& 350& 175& 526\\
263& 790& 395& 1186& 593& 1780& 890& 445& 1336& 668\\
334& 167& 502& 251& 754& 377& 1132& 566& 283& 850\\
425& 1276& 638& 319& 958& 479& 1438& 719& 2158& 1079\\
3238& 1619& 4858& 2429& 7288& 3644& 1822& 911& 2734& 1367\\
4102& 2051& 6154& 3077& 9232& 4616& 2308& 1154& 577& 1732\\
866& 433& 1300& 650& 325& 976& 488& 244& 122& 61\\
184& 92& 46& 23& 70& 35& 106& 53& 160& 80\\
40& 20& 10& 5& 16& 8& 4& 2& 1& \\

919&&&&&&&&&\\
2758& 1379& 4138& 2069& 6208& 3104& 1552& 776& 388& 194\\
97& 292& 146& 73& 220& 110& 55& 166& 83& 250\\
125& 376& 188& 94& 47& 142& 71& 214& 107& 322\\
161& 484& 242& 121& 364& 182& 91& 274& 137& 412\\
206& 103& 310& 155& 466& 233& 700& 350& 175& 526\\
263& 790& 395& 1186& 593& 1780& 890& 445& 1336& 668\\
334& 167& 502& 251& 754& 377& 1132& 566& 283& 850\\
425& 1276& 638& 319& 958& 479& 1438& 719& 2158& 1079\\
3238& 1619& 4858& 2429& 7288& 3644& 1822& 911& 2734& 1367\\
4102& 2051& 6154& 3077& 9232& 4616& 2308& 1154& 577& 1732\\
866& 433& 1300& 650& 325& 976& 488& 244& 122& 61\\
184& 92& 46& 23& 70& 35& 106& 53& 160& 80\\
40& 20& 10& 5& 16& 8& 4& 2& 1& \\

920&&&&&&&&&\\
460& 230& 115& 346& 173& 520& 260& 130& 65& 196\\
98& 49& 148& 74& 37& 112& 56& 28& 14& 7\\
22& 11& 34& 17& 52& 26& 13& 40& 20& 10\\
5& 16& 8& 4& 2& 1& \\

921&&&&&&&&&\\
2764& 1382& 691& 2074& 1037& 3112& 1556& 778& 389& 1168\\
584& 292& 146& 73& 220& 110& 55& 166& 83& 250\\
125& 376& 188& 94& 47& 142& 71& 214& 107& 322\\
161& 484& 242& 121& 364& 182& 91& 274& 137& 412\\
206& 103& 310& 155& 466& 233& 700& 350& 175& 526\\
263& 790& 395& 1186& 593& 1780& 890& 445& 1336& 668\\
334& 167& 502& 251& 754& 377& 1132& 566& 283& 850\\
425& 1276& 638& 319& 958& 479& 1438& 719& 2158& 1079\\
3238& 1619& 4858& 2429& 7288& 3644& 1822& 911& 2734& 1367\\
4102& 2051& 6154& 3077& 9232& 4616& 2308& 1154& 577& 1732\\
866& 433& 1300& 650& 325& 976& 488& 244& 122& 61\\
184& 92& 46& 23& 70& 35& 106& 53& 160& 80\\
40& 20& 10& 5& 16& 8& 4& 2& 1& \\

922&&&&&&&&&\\
461& 1384& 692& 346& 173& 520& 260& 130& 65& 196\\
98& 49& 148& 74& 37& 112& 56& 28& 14& 7\\
22& 11& 34& 17& 52& 26& 13& 40& 20& 10\\
5& 16& 8& 4& 2& 1& \\

923&&&&&&&&&\\
2770& 1385& 4156& 2078& 1039& 3118& 1559& 4678& 2339& 7018\\
3509& 10528& 5264& 2632& 1316& 658& 329& 988& 494& 247\\
742& 371& 1114& 557& 1672& 836& 418& 209& 628& 314\\
157& 472& 236& 118& 59& 178& 89& 268& 134& 67\\
202& 101& 304& 152& 76& 38& 19& 58& 29& 88\\
44& 22& 11& 34& 17& 52& 26& 13& 40& 20\\
10& 5& 16& 8& 4& 2& 1& \\

924&&&&&&&&&\\
462& 231& 694& 347& 1042& 521& 1564& 782& 391& 1174\\
587& 1762& 881& 2644& 1322& 661& 1984& 992& 496& 248\\
124& 62& 31& 94& 47& 142& 71& 214& 107& 322\\
161& 484& 242& 121& 364& 182& 91& 274& 137& 412\\
206& 103& 310& 155& 466& 233& 700& 350& 175& 526\\
263& 790& 395& 1186& 593& 1780& 890& 445& 1336& 668\\
334& 167& 502& 251& 754& 377& 1132& 566& 283& 850\\
425& 1276& 638& 319& 958& 479& 1438& 719& 2158& 1079\\
3238& 1619& 4858& 2429& 7288& 3644& 1822& 911& 2734& 1367\\
4102& 2051& 6154& 3077& 9232& 4616& 2308& 1154& 577& 1732\\
866& 433& 1300& 650& 325& 976& 488& 244& 122& 61\\
184& 92& 46& 23& 70& 35& 106& 53& 160& 80\\
40& 20& 10& 5& 16& 8& 4& 2& 1& \\

925&&&&&&&&&\\
2776& 1388& 694& 347& 1042& 521& 1564& 782& 391& 1174\\
587& 1762& 881& 2644& 1322& 661& 1984& 992& 496& 248\\
124& 62& 31& 94& 47& 142& 71& 214& 107& 322\\
161& 484& 242& 121& 364& 182& 91& 274& 137& 412\\
206& 103& 310& 155& 466& 233& 700& 350& 175& 526\\
263& 790& 395& 1186& 593& 1780& 890& 445& 1336& 668\\
334& 167& 502& 251& 754& 377& 1132& 566& 283& 850\\
425& 1276& 638& 319& 958& 479& 1438& 719& 2158& 1079\\
3238& 1619& 4858& 2429& 7288& 3644& 1822& 911& 2734& 1367\\
4102& 2051& 6154& 3077& 9232& 4616& 2308& 1154& 577& 1732\\
866& 433& 1300& 650& 325& 976& 488& 244& 122& 61\\
184& 92& 46& 23& 70& 35& 106& 53& 160& 80\\
40& 20& 10& 5& 16& 8& 4& 2& 1& \\

926&&&&&&&&&\\
463& 1390& 695& 2086& 1043& 3130& 1565& 4696& 2348& 1174\\
587& 1762& 881& 2644& 1322& 661& 1984& 992& 496& 248\\
124& 62& 31& 94& 47& 142& 71& 214& 107& 322\\
161& 484& 242& 121& 364& 182& 91& 274& 137& 412\\
206& 103& 310& 155& 466& 233& 700& 350& 175& 526\\
263& 790& 395& 1186& 593& 1780& 890& 445& 1336& 668\\
334& 167& 502& 251& 754& 377& 1132& 566& 283& 850\\
425& 1276& 638& 319& 958& 479& 1438& 719& 2158& 1079\\
3238& 1619& 4858& 2429& 7288& 3644& 1822& 911& 2734& 1367\\
4102& 2051& 6154& 3077& 9232& 4616& 2308& 1154& 577& 1732\\
866& 433& 1300& 650& 325& 976& 488& 244& 122& 61\\
184& 92& 46& 23& 70& 35& 106& 53& 160& 80\\
40& 20& 10& 5& 16& 8& 4& 2& 1& \\

927&&&&&&&&&\\
2782& 1391& 4174& 2087& 6262& 3131& 9394& 4697& 14092& 7046\\
3523& 10570& 5285& 15856& 7928& 3964& 1982& 991& 2974& 1487\\
4462& 2231& 6694& 3347& 10042& 5021& 15064& 7532& 3766& 1883\\
5650& 2825& 8476& 4238& 2119& 6358& 3179& 9538& 4769& 14308\\
7154& 3577& 10732& 5366& 2683& 8050& 4025& 12076& 6038& 3019\\
9058& 4529& 13588& 6794& 3397& 10192& 5096& 2548& 1274& 637\\
1912& 956& 478& 239& 718& 359& 1078& 539& 1618& 809\\
2428& 1214& 607& 1822& 911& 2734& 1367& 4102& 2051& 6154\\
3077& 9232& 4616& 2308& 1154& 577& 1732& 866& 433& 1300\\
650& 325& 976& 488& 244& 122& 61& 184& 92& 46\\
23& 70& 35& 106& 53& 160& 80& 40& 20& 10\\
5& 16& 8& 4& 2& 1& \\

928&&&&&&&&&\\
464& 232& 116& 58& 29& 88& 44& 22& 11& 34\\
17& 52& 26& 13& 40& 20& 10& 5& 16& 8\\
4& 2& 1& \\

929&&&&&&&&&\\
2788& 1394& 697& 2092& 1046& 523& 1570& 785& 2356& 1178\\
589& 1768& 884& 442& 221& 664& 332& 166& 83& 250\\
125& 376& 188& 94& 47& 142& 71& 214& 107& 322\\
161& 484& 242& 121& 364& 182& 91& 274& 137& 412\\
206& 103& 310& 155& 466& 233& 700& 350& 175& 526\\
263& 790& 395& 1186& 593& 1780& 890& 445& 1336& 668\\
334& 167& 502& 251& 754& 377& 1132& 566& 283& 850\\
425& 1276& 638& 319& 958& 479& 1438& 719& 2158& 1079\\
3238& 1619& 4858& 2429& 7288& 3644& 1822& 911& 2734& 1367\\
4102& 2051& 6154& 3077& 9232& 4616& 2308& 1154& 577& 1732\\
866& 433& 1300& 650& 325& 976& 488& 244& 122& 61\\
184& 92& 46& 23& 70& 35& 106& 53& 160& 80\\
40& 20& 10& 5& 16& 8& 4& 2& 1& \\

930&&&&&&&&&\\
465& 1396& 698& 349& 1048& 524& 262& 131& 394& 197\\
592& 296& 148& 74& 37& 112& 56& 28& 14& 7\\
22& 11& 34& 17& 52& 26& 13& 40& 20& 10\\
5& 16& 8& 4& 2& 1& \\

931&&&&&&&&&\\
2794& 1397& 4192& 2096& 1048& 524& 262& 131& 394& 197\\
592& 296& 148& 74& 37& 112& 56& 28& 14& 7\\
22& 11& 34& 17& 52& 26& 13& 40& 20& 10\\
5& 16& 8& 4& 2& 1& \\

932&&&&&&&&&\\
466& 233& 700& 350& 175& 526& 263& 790& 395& 1186\\
593& 1780& 890& 445& 1336& 668& 334& 167& 502& 251\\
754& 377& 1132& 566& 283& 850& 425& 1276& 638& 319\\
958& 479& 1438& 719& 2158& 1079& 3238& 1619& 4858& 2429\\
7288& 3644& 1822& 911& 2734& 1367& 4102& 2051& 6154& 3077\\
9232& 4616& 2308& 1154& 577& 1732& 866& 433& 1300& 650\\
325& 976& 488& 244& 122& 61& 184& 92& 46& 23\\
70& 35& 106& 53& 160& 80& 40& 20& 10& 5\\
16& 8& 4& 2& 1& \\

933&&&&&&&&&\\
2800& 1400& 700& 350& 175& 526& 263& 790& 395& 1186\\
593& 1780& 890& 445& 1336& 668& 334& 167& 502& 251\\
754& 377& 1132& 566& 283& 850& 425& 1276& 638& 319\\
958& 479& 1438& 719& 2158& 1079& 3238& 1619& 4858& 2429\\
7288& 3644& 1822& 911& 2734& 1367& 4102& 2051& 6154& 3077\\
9232& 4616& 2308& 1154& 577& 1732& 866& 433& 1300& 650\\
325& 976& 488& 244& 122& 61& 184& 92& 46& 23\\
70& 35& 106& 53& 160& 80& 40& 20& 10& 5\\
16& 8& 4& 2& 1& \\

934&&&&&&&&&\\
467& 1402& 701& 2104& 1052& 526& 263& 790& 395& 1186\\
593& 1780& 890& 445& 1336& 668& 334& 167& 502& 251\\
754& 377& 1132& 566& 283& 850& 425& 1276& 638& 319\\
958& 479& 1438& 719& 2158& 1079& 3238& 1619& 4858& 2429\\
7288& 3644& 1822& 911& 2734& 1367& 4102& 2051& 6154& 3077\\
9232& 4616& 2308& 1154& 577& 1732& 866& 433& 1300& 650\\
325& 976& 488& 244& 122& 61& 184& 92& 46& 23\\
70& 35& 106& 53& 160& 80& 40& 20& 10& 5\\
16& 8& 4& 2& 1& \\

935&&&&&&&&&\\
2806& 1403& 4210& 2105& 6316& 3158& 1579& 4738& 2369& 7108\\
3554& 1777& 5332& 2666& 1333& 4000& 2000& 1000& 500& 250\\
125& 376& 188& 94& 47& 142& 71& 214& 107& 322\\
161& 484& 242& 121& 364& 182& 91& 274& 137& 412\\
206& 103& 310& 155& 466& 233& 700& 350& 175& 526\\
263& 790& 395& 1186& 593& 1780& 890& 445& 1336& 668\\
334& 167& 502& 251& 754& 377& 1132& 566& 283& 850\\
425& 1276& 638& 319& 958& 479& 1438& 719& 2158& 1079\\
3238& 1619& 4858& 2429& 7288& 3644& 1822& 911& 2734& 1367\\
4102& 2051& 6154& 3077& 9232& 4616& 2308& 1154& 577& 1732\\
866& 433& 1300& 650& 325& 976& 488& 244& 122& 61\\
184& 92& 46& 23& 70& 35& 106& 53& 160& 80\\
40& 20& 10& 5& 16& 8& 4& 2& 1& \\

936&&&&&&&&&\\
468& 234& 117& 352& 176& 88& 44& 22& 11& 34\\
17& 52& 26& 13& 40& 20& 10& 5& 16& 8\\
4& 2& 1& \\

937&&&&&&&&&\\
2812& 1406& 703& 2110& 1055& 3166& 1583& 4750& 2375& 7126\\
3563& 10690& 5345& 16036& 8018& 4009& 12028& 6014& 3007& 9022\\
4511& 13534& 6767& 20302& 10151& 30454& 15227& 45682& 22841& 68524\\
34262& 17131& 51394& 25697& 77092& 38546& 19273& 57820& 28910& 14455\\
43366& 21683& 65050& 32525& 97576& 48788& 24394& 12197& 36592& 18296\\
9148& 4574& 2287& 6862& 3431& 10294& 5147& 15442& 7721& 23164\\
11582& 5791& 17374& 8687& 26062& 13031& 39094& 19547& 58642& 29321\\
87964& 43982& 21991& 65974& 32987& 98962& 49481& 148444& 74222& 37111\\
111334& 55667& 167002& 83501& 250504& 125252& 62626& 31313& 93940& 46970\\
23485& 70456& 35228& 17614& 8807& 26422& 13211& 39634& 19817& 59452\\
29726& 14863& 44590& 22295& 66886& 33443& 100330& 50165& 150496& 75248\\
37624& 18812& 9406& 4703& 14110& 7055& 21166& 10583& 31750& 15875\\
47626& 23813& 71440& 35720& 17860& 8930& 4465& 13396& 6698& 3349\\
10048& 5024& 2512& 1256& 628& 314& 157& 472& 236& 118\\
59& 178& 89& 268& 134& 67& 202& 101& 304& 152\\
76& 38& 19& 58& 29& 88& 44& 22& 11& 34\\
17& 52& 26& 13& 40& 20& 10& 5& 16& 8\\
4& 2& 1& \\

938&&&&&&&&&\\
469& 1408& 704& 352& 176& 88& 44& 22& 11& 34\\
17& 52& 26& 13& 40& 20& 10& 5& 16& 8\\
4& 2& 1& \\

939&&&&&&&&&\\
2818& 1409& 4228& 2114& 1057& 3172& 1586& 793& 2380& 1190\\
595& 1786& 893& 2680& 1340& 670& 335& 1006& 503& 1510\\
755& 2266& 1133& 3400& 1700& 850& 425& 1276& 638& 319\\
958& 479& 1438& 719& 2158& 1079& 3238& 1619& 4858& 2429\\
7288& 3644& 1822& 911& 2734& 1367& 4102& 2051& 6154& 3077\\
9232& 4616& 2308& 1154& 577& 1732& 866& 433& 1300& 650\\
325& 976& 488& 244& 122& 61& 184& 92& 46& 23\\
70& 35& 106& 53& 160& 80& 40& 20& 10& 5\\
16& 8& 4& 2& 1& \\

940&&&&&&&&&\\
470& 235& 706& 353& 1060& 530& 265& 796& 398& 199\\
598& 299& 898& 449& 1348& 674& 337& 1012& 506& 253\\
760& 380& 190& 95& 286& 143& 430& 215& 646& 323\\
970& 485& 1456& 728& 364& 182& 91& 274& 137& 412\\
206& 103& 310& 155& 466& 233& 700& 350& 175& 526\\
263& 790& 395& 1186& 593& 1780& 890& 445& 1336& 668\\
334& 167& 502& 251& 754& 377& 1132& 566& 283& 850\\
425& 1276& 638& 319& 958& 479& 1438& 719& 2158& 1079\\
3238& 1619& 4858& 2429& 7288& 3644& 1822& 911& 2734& 1367\\
4102& 2051& 6154& 3077& 9232& 4616& 2308& 1154& 577& 1732\\
866& 433& 1300& 650& 325& 976& 488& 244& 122& 61\\
184& 92& 46& 23& 70& 35& 106& 53& 160& 80\\
40& 20& 10& 5& 16& 8& 4& 2& 1& \\

941&&&&&&&&&\\
2824& 1412& 706& 353& 1060& 530& 265& 796& 398& 199\\
598& 299& 898& 449& 1348& 674& 337& 1012& 506& 253\\
760& 380& 190& 95& 286& 143& 430& 215& 646& 323\\
970& 485& 1456& 728& 364& 182& 91& 274& 137& 412\\
206& 103& 310& 155& 466& 233& 700& 350& 175& 526\\
263& 790& 395& 1186& 593& 1780& 890& 445& 1336& 668\\
334& 167& 502& 251& 754& 377& 1132& 566& 283& 850\\
425& 1276& 638& 319& 958& 479& 1438& 719& 2158& 1079\\
3238& 1619& 4858& 2429& 7288& 3644& 1822& 911& 2734& 1367\\
4102& 2051& 6154& 3077& 9232& 4616& 2308& 1154& 577& 1732\\
866& 433& 1300& 650& 325& 976& 488& 244& 122& 61\\
184& 92& 46& 23& 70& 35& 106& 53& 160& 80\\
40& 20& 10& 5& 16& 8& 4& 2& 1& \\

942&&&&&&&&&\\
471& 1414& 707& 2122& 1061& 3184& 1592& 796& 398& 199\\
598& 299& 898& 449& 1348& 674& 337& 1012& 506& 253\\
760& 380& 190& 95& 286& 143& 430& 215& 646& 323\\
970& 485& 1456& 728& 364& 182& 91& 274& 137& 412\\
206& 103& 310& 155& 466& 233& 700& 350& 175& 526\\
263& 790& 395& 1186& 593& 1780& 890& 445& 1336& 668\\
334& 167& 502& 251& 754& 377& 1132& 566& 283& 850\\
425& 1276& 638& 319& 958& 479& 1438& 719& 2158& 1079\\
3238& 1619& 4858& 2429& 7288& 3644& 1822& 911& 2734& 1367\\
4102& 2051& 6154& 3077& 9232& 4616& 2308& 1154& 577& 1732\\
866& 433& 1300& 650& 325& 976& 488& 244& 122& 61\\
184& 92& 46& 23& 70& 35& 106& 53& 160& 80\\
40& 20& 10& 5& 16& 8& 4& 2& 1& \\

943&&&&&&&&&\\
2830& 1415& 4246& 2123& 6370& 3185& 9556& 4778& 2389& 7168\\
3584& 1792& 896& 448& 224& 112& 56& 28& 14& 7\\
22& 11& 34& 17& 52& 26& 13& 40& 20& 10\\
5& 16& 8& 4& 2& 1& \\

944&&&&&&&&&\\
472& 236& 118& 59& 178& 89& 268& 134& 67& 202\\
101& 304& 152& 76& 38& 19& 58& 29& 88& 44\\
22& 11& 34& 17& 52& 26& 13& 40& 20& 10\\
5& 16& 8& 4& 2& 1& \\

945&&&&&&&&&\\
2836& 1418& 709& 2128& 1064& 532& 266& 133& 400& 200\\
100& 50& 25& 76& 38& 19& 58& 29& 88& 44\\
22& 11& 34& 17& 52& 26& 13& 40& 20& 10\\
5& 16& 8& 4& 2& 1& \\

946&&&&&&&&&\\
473& 1420& 710& 355& 1066& 533& 1600& 800& 400& 200\\
100& 50& 25& 76& 38& 19& 58& 29& 88& 44\\
22& 11& 34& 17& 52& 26& 13& 40& 20& 10\\
5& 16& 8& 4& 2& 1& \\

947&&&&&&&&&\\
2842& 1421& 4264& 2132& 1066& 533& 1600& 800& 400& 200\\
100& 50& 25& 76& 38& 19& 58& 29& 88& 44\\
22& 11& 34& 17& 52& 26& 13& 40& 20& 10\\
5& 16& 8& 4& 2& 1& \\

948&&&&&&&&&\\
474& 237& 712& 356& 178& 89& 268& 134& 67& 202\\
101& 304& 152& 76& 38& 19& 58& 29& 88& 44\\
22& 11& 34& 17& 52& 26& 13& 40& 20& 10\\
5& 16& 8& 4& 2& 1& \\

949&&&&&&&&&\\
2848& 1424& 712& 356& 178& 89& 268& 134& 67& 202\\
101& 304& 152& 76& 38& 19& 58& 29& 88& 44\\
22& 11& 34& 17& 52& 26& 13& 40& 20& 10\\
5& 16& 8& 4& 2& 1& \\

950&&&&&&&&&\\
475& 1426& 713& 2140& 1070& 535& 1606& 803& 2410& 1205\\
3616& 1808& 904& 452& 226& 113& 340& 170& 85& 256\\
128& 64& 32& 16& 8& 4& 2& 1& \\

951&&&&&&&&&\\
2854& 1427& 4282& 2141& 6424& 3212& 1606& 803& 2410& 1205\\
3616& 1808& 904& 452& 226& 113& 340& 170& 85& 256\\
128& 64& 32& 16& 8& 4& 2& 1& \\

952&&&&&&&&&\\
476& 238& 119& 358& 179& 538& 269& 808& 404& 202\\
101& 304& 152& 76& 38& 19& 58& 29& 88& 44\\
22& 11& 34& 17& 52& 26& 13& 40& 20& 10\\
5& 16& 8& 4& 2& 1& \\

953&&&&&&&&&\\
2860& 1430& 715& 2146& 1073& 3220& 1610& 805& 2416& 1208\\
604& 302& 151& 454& 227& 682& 341& 1024& 512& 256\\
128& 64& 32& 16& 8& 4& 2& 1& \\

954&&&&&&&&&\\
477& 1432& 716& 358& 179& 538& 269& 808& 404& 202\\
101& 304& 152& 76& 38& 19& 58& 29& 88& 44\\
22& 11& 34& 17& 52& 26& 13& 40& 20& 10\\
5& 16& 8& 4& 2& 1& \\

955&&&&&&&&&\\
2866& 1433& 4300& 2150& 1075& 3226& 1613& 4840& 2420& 1210\\
605& 1816& 908& 454& 227& 682& 341& 1024& 512& 256\\
128& 64& 32& 16& 8& 4& 2& 1& \\

956&&&&&&&&&\\
478& 239& 718& 359& 1078& 539& 1618& 809& 2428& 1214\\
607& 1822& 911& 2734& 1367& 4102& 2051& 6154& 3077& 9232\\
4616& 2308& 1154& 577& 1732& 866& 433& 1300& 650& 325\\
976& 488& 244& 122& 61& 184& 92& 46& 23& 70\\
35& 106& 53& 160& 80& 40& 20& 10& 5& 16\\
8& 4& 2& 1& \\

957&&&&&&&&&\\
2872& 1436& 718& 359& 1078& 539& 1618& 809& 2428& 1214\\
607& 1822& 911& 2734& 1367& 4102& 2051& 6154& 3077& 9232\\
4616& 2308& 1154& 577& 1732& 866& 433& 1300& 650& 325\\
976& 488& 244& 122& 61& 184& 92& 46& 23& 70\\
35& 106& 53& 160& 80& 40& 20& 10& 5& 16\\
8& 4& 2& 1& \\

958&&&&&&&&&\\
479& 1438& 719& 2158& 1079& 3238& 1619& 4858& 2429& 7288\\
3644& 1822& 911& 2734& 1367& 4102& 2051& 6154& 3077& 9232\\
4616& 2308& 1154& 577& 1732& 866& 433& 1300& 650& 325\\
976& 488& 244& 122& 61& 184& 92& 46& 23& 70\\
35& 106& 53& 160& 80& 40& 20& 10& 5& 16\\
8& 4& 2& 1& \\

959&&&&&&&&&\\
2878& 1439& 4318& 2159& 6478& 3239& 9718& 4859& 14578& 7289\\
21868& 10934& 5467& 16402& 8201& 24604& 12302& 6151& 18454& 9227\\
27682& 13841& 41524& 20762& 10381& 31144& 15572& 7786& 3893& 11680\\
5840& 2920& 1460& 730& 365& 1096& 548& 274& 137& 412\\
206& 103& 310& 155& 466& 233& 700& 350& 175& 526\\
263& 790& 395& 1186& 593& 1780& 890& 445& 1336& 668\\
334& 167& 502& 251& 754& 377& 1132& 566& 283& 850\\
425& 1276& 638& 319& 958& 479& 1438& 719& 2158& 1079\\
3238& 1619& 4858& 2429& 7288& 3644& 1822& 911& 2734& 1367\\
4102& 2051& 6154& 3077& 9232& 4616& 2308& 1154& 577& 1732\\
866& 433& 1300& 650& 325& 976& 488& 244& 122& 61\\
184& 92& 46& 23& 70& 35& 106& 53& 160& 80\\
40& 20& 10& 5& 16& 8& 4& 2& 1& \\

960&&&&&&&&&\\
480& 240& 120& 60& 30& 15& 46& 23& 70& 35\\
106& 53& 160& 80& 40& 20& 10& 5& 16& 8\\
4& 2& 1& \\

961&&&&&&&&&\\
2884& 1442& 721& 2164& 1082& 541& 1624& 812& 406& 203\\
610& 305& 916& 458& 229& 688& 344& 172& 86& 43\\
130& 65& 196& 98& 49& 148& 74& 37& 112& 56\\
28& 14& 7& 22& 11& 34& 17& 52& 26& 13\\
40& 20& 10& 5& 16& 8& 4& 2& 1& \\

962&&&&&&&&&\\
481& 1444& 722& 361& 1084& 542& 271& 814& 407& 1222\\
611& 1834& 917& 2752& 1376& 688& 344& 172& 86& 43\\
130& 65& 196& 98& 49& 148& 74& 37& 112& 56\\
28& 14& 7& 22& 11& 34& 17& 52& 26& 13\\
40& 20& 10& 5& 16& 8& 4& 2& 1& \\

963&&&&&&&&&\\
2890& 1445& 4336& 2168& 1084& 542& 271& 814& 407& 1222\\
611& 1834& 917& 2752& 1376& 688& 344& 172& 86& 43\\
130& 65& 196& 98& 49& 148& 74& 37& 112& 56\\
28& 14& 7& 22& 11& 34& 17& 52& 26& 13\\
40& 20& 10& 5& 16& 8& 4& 2& 1& \\

964&&&&&&&&&\\
482& 241& 724& 362& 181& 544& 272& 136& 68& 34\\
17& 52& 26& 13& 40& 20& 10& 5& 16& 8\\
4& 2& 1& \\

965&&&&&&&&&\\
2896& 1448& 724& 362& 181& 544& 272& 136& 68& 34\\
17& 52& 26& 13& 40& 20& 10& 5& 16& 8\\
4& 2& 1& \\

966&&&&&&&&&\\
483& 1450& 725& 2176& 1088& 544& 272& 136& 68& 34\\
17& 52& 26& 13& 40& 20& 10& 5& 16& 8\\
4& 2& 1& \\

967&&&&&&&&&\\
2902& 1451& 4354& 2177& 6532& 3266& 1633& 4900& 2450& 1225\\
3676& 1838& 919& 2758& 1379& 4138& 2069& 6208& 3104& 1552\\
776& 388& 194& 97& 292& 146& 73& 220& 110& 55\\
166& 83& 250& 125& 376& 188& 94& 47& 142& 71\\
214& 107& 322& 161& 484& 242& 121& 364& 182& 91\\
274& 137& 412& 206& 103& 310& 155& 466& 233& 700\\
350& 175& 526& 263& 790& 395& 1186& 593& 1780& 890\\
445& 1336& 668& 334& 167& 502& 251& 754& 377& 1132\\
566& 283& 850& 425& 1276& 638& 319& 958& 479& 1438\\
719& 2158& 1079& 3238& 1619& 4858& 2429& 7288& 3644& 1822\\
911& 2734& 1367& 4102& 2051& 6154& 3077& 9232& 4616& 2308\\
1154& 577& 1732& 866& 433& 1300& 650& 325& 976& 488\\
244& 122& 61& 184& 92& 46& 23& 70& 35& 106\\
53& 160& 80& 40& 20& 10& 5& 16& 8& 4\\
2& 1& \\

968&&&&&&&&&\\
484& 242& 121& 364& 182& 91& 274& 137& 412& 206\\
103& 310& 155& 466& 233& 700& 350& 175& 526& 263\\
790& 395& 1186& 593& 1780& 890& 445& 1336& 668& 334\\
167& 502& 251& 754& 377& 1132& 566& 283& 850& 425\\
1276& 638& 319& 958& 479& 1438& 719& 2158& 1079& 3238\\
1619& 4858& 2429& 7288& 3644& 1822& 911& 2734& 1367& 4102\\
2051& 6154& 3077& 9232& 4616& 2308& 1154& 577& 1732& 866\\
433& 1300& 650& 325& 976& 488& 244& 122& 61& 184\\
92& 46& 23& 70& 35& 106& 53& 160& 80& 40\\
20& 10& 5& 16& 8& 4& 2& 1& \\

969&&&&&&&&&\\
2908& 1454& 727& 2182& 1091& 3274& 1637& 4912& 2456& 1228\\
614& 307& 922& 461& 1384& 692& 346& 173& 520& 260\\
130& 65& 196& 98& 49& 148& 74& 37& 112& 56\\
28& 14& 7& 22& 11& 34& 17& 52& 26& 13\\
40& 20& 10& 5& 16& 8& 4& 2& 1& \\

970&&&&&&&&&\\
485& 1456& 728& 364& 182& 91& 274& 137& 412& 206\\
103& 310& 155& 466& 233& 700& 350& 175& 526& 263\\
790& 395& 1186& 593& 1780& 890& 445& 1336& 668& 334\\
167& 502& 251& 754& 377& 1132& 566& 283& 850& 425\\
1276& 638& 319& 958& 479& 1438& 719& 2158& 1079& 3238\\
1619& 4858& 2429& 7288& 3644& 1822& 911& 2734& 1367& 4102\\
2051& 6154& 3077& 9232& 4616& 2308& 1154& 577& 1732& 866\\
433& 1300& 650& 325& 976& 488& 244& 122& 61& 184\\
92& 46& 23& 70& 35& 106& 53& 160& 80& 40\\
20& 10& 5& 16& 8& 4& 2& 1& \\

971&&&&&&&&&\\
2914& 1457& 4372& 2186& 1093& 3280& 1640& 820& 410& 205\\
616& 308& 154& 77& 232& 116& 58& 29& 88& 44\\
22& 11& 34& 17& 52& 26& 13& 40& 20& 10\\
5& 16& 8& 4& 2& 1& \\

972&&&&&&&&&\\
486& 243& 730& 365& 1096& 548& 274& 137& 412& 206\\
103& 310& 155& 466& 233& 700& 350& 175& 526& 263\\
790& 395& 1186& 593& 1780& 890& 445& 1336& 668& 334\\
167& 502& 251& 754& 377& 1132& 566& 283& 850& 425\\
1276& 638& 319& 958& 479& 1438& 719& 2158& 1079& 3238\\
1619& 4858& 2429& 7288& 3644& 1822& 911& 2734& 1367& 4102\\
2051& 6154& 3077& 9232& 4616& 2308& 1154& 577& 1732& 866\\
433& 1300& 650& 325& 976& 488& 244& 122& 61& 184\\
92& 46& 23& 70& 35& 106& 53& 160& 80& 40\\
20& 10& 5& 16& 8& 4& 2& 1& \\

973&&&&&&&&&\\
2920& 1460& 730& 365& 1096& 548& 274& 137& 412& 206\\
103& 310& 155& 466& 233& 700& 350& 175& 526& 263\\
790& 395& 1186& 593& 1780& 890& 445& 1336& 668& 334\\
167& 502& 251& 754& 377& 1132& 566& 283& 850& 425\\
1276& 638& 319& 958& 479& 1438& 719& 2158& 1079& 3238\\
1619& 4858& 2429& 7288& 3644& 1822& 911& 2734& 1367& 4102\\
2051& 6154& 3077& 9232& 4616& 2308& 1154& 577& 1732& 866\\
433& 1300& 650& 325& 976& 488& 244& 122& 61& 184\\
92& 46& 23& 70& 35& 106& 53& 160& 80& 40\\
20& 10& 5& 16& 8& 4& 2& 1& \\

974&&&&&&&&&\\
487& 1462& 731& 2194& 1097& 3292& 1646& 823& 2470& 1235\\
3706& 1853& 5560& 2780& 1390& 695& 2086& 1043& 3130& 1565\\
4696& 2348& 1174& 587& 1762& 881& 2644& 1322& 661& 1984\\
992& 496& 248& 124& 62& 31& 94& 47& 142& 71\\
214& 107& 322& 161& 484& 242& 121& 364& 182& 91\\
274& 137& 412& 206& 103& 310& 155& 466& 233& 700\\
350& 175& 526& 263& 790& 395& 1186& 593& 1780& 890\\
445& 1336& 668& 334& 167& 502& 251& 754& 377& 1132\\
566& 283& 850& 425& 1276& 638& 319& 958& 479& 1438\\
719& 2158& 1079& 3238& 1619& 4858& 2429& 7288& 3644& 1822\\
911& 2734& 1367& 4102& 2051& 6154& 3077& 9232& 4616& 2308\\
1154& 577& 1732& 866& 433& 1300& 650& 325& 976& 488\\
244& 122& 61& 184& 92& 46& 23& 70& 35& 106\\
53& 160& 80& 40& 20& 10& 5& 16& 8& 4\\
2& 1& \\

975&&&&&&&&&\\
2926& 1463& 4390& 2195& 6586& 3293& 9880& 4940& 2470& 1235\\
3706& 1853& 5560& 2780& 1390& 695& 2086& 1043& 3130& 1565\\
4696& 2348& 1174& 587& 1762& 881& 2644& 1322& 661& 1984\\
992& 496& 248& 124& 62& 31& 94& 47& 142& 71\\
214& 107& 322& 161& 484& 242& 121& 364& 182& 91\\
274& 137& 412& 206& 103& 310& 155& 466& 233& 700\\
350& 175& 526& 263& 790& 395& 1186& 593& 1780& 890\\
445& 1336& 668& 334& 167& 502& 251& 754& 377& 1132\\
566& 283& 850& 425& 1276& 638& 319& 958& 479& 1438\\
719& 2158& 1079& 3238& 1619& 4858& 2429& 7288& 3644& 1822\\
911& 2734& 1367& 4102& 2051& 6154& 3077& 9232& 4616& 2308\\
1154& 577& 1732& 866& 433& 1300& 650& 325& 976& 488\\
244& 122& 61& 184& 92& 46& 23& 70& 35& 106\\
53& 160& 80& 40& 20& 10& 5& 16& 8& 4\\
2& 1& \\

976&&&&&&&&&\\
488& 244& 122& 61& 184& 92& 46& 23& 70& 35\\
106& 53& 160& 80& 40& 20& 10& 5& 16& 8\\
4& 2& 1& \\

977&&&&&&&&&\\
2932& 1466& 733& 2200& 1100& 550& 275& 826& 413& 1240\\
620& 310& 155& 466& 233& 700& 350& 175& 526& 263\\
790& 395& 1186& 593& 1780& 890& 445& 1336& 668& 334\\
167& 502& 251& 754& 377& 1132& 566& 283& 850& 425\\
1276& 638& 319& 958& 479& 1438& 719& 2158& 1079& 3238\\
1619& 4858& 2429& 7288& 3644& 1822& 911& 2734& 1367& 4102\\
2051& 6154& 3077& 9232& 4616& 2308& 1154& 577& 1732& 866\\
433& 1300& 650& 325& 976& 488& 244& 122& 61& 184\\
92& 46& 23& 70& 35& 106& 53& 160& 80& 40\\
20& 10& 5& 16& 8& 4& 2& 1& \\

978&&&&&&&&&\\
489& 1468& 734& 367& 1102& 551& 1654& 827& 2482& 1241\\
3724& 1862& 931& 2794& 1397& 4192& 2096& 1048& 524& 262\\
131& 394& 197& 592& 296& 148& 74& 37& 112& 56\\
28& 14& 7& 22& 11& 34& 17& 52& 26& 13\\
40& 20& 10& 5& 16& 8& 4& 2& 1& \\

979&&&&&&&&&\\
2938& 1469& 4408& 2204& 1102& 551& 1654& 827& 2482& 1241\\
3724& 1862& 931& 2794& 1397& 4192& 2096& 1048& 524& 262\\
131& 394& 197& 592& 296& 148& 74& 37& 112& 56\\
28& 14& 7& 22& 11& 34& 17& 52& 26& 13\\
40& 20& 10& 5& 16& 8& 4& 2& 1& \\

980&&&&&&&&&\\
490& 245& 736& 368& 184& 92& 46& 23& 70& 35\\
106& 53& 160& 80& 40& 20& 10& 5& 16& 8\\
4& 2& 1& \\

981&&&&&&&&&\\
2944& 1472& 736& 368& 184& 92& 46& 23& 70& 35\\
106& 53& 160& 80& 40& 20& 10& 5& 16& 8\\
4& 2& 1& \\

982&&&&&&&&&\\
491& 1474& 737& 2212& 1106& 553& 1660& 830& 415& 1246\\
623& 1870& 935& 2806& 1403& 4210& 2105& 6316& 3158& 1579\\
4738& 2369& 7108& 3554& 1777& 5332& 2666& 1333& 4000& 2000\\
1000& 500& 250& 125& 376& 188& 94& 47& 142& 71\\
214& 107& 322& 161& 484& 242& 121& 364& 182& 91\\
274& 137& 412& 206& 103& 310& 155& 466& 233& 700\\
350& 175& 526& 263& 790& 395& 1186& 593& 1780& 890\\
445& 1336& 668& 334& 167& 502& 251& 754& 377& 1132\\
566& 283& 850& 425& 1276& 638& 319& 958& 479& 1438\\
719& 2158& 1079& 3238& 1619& 4858& 2429& 7288& 3644& 1822\\
911& 2734& 1367& 4102& 2051& 6154& 3077& 9232& 4616& 2308\\
1154& 577& 1732& 866& 433& 1300& 650& 325& 976& 488\\
244& 122& 61& 184& 92& 46& 23& 70& 35& 106\\
53& 160& 80& 40& 20& 10& 5& 16& 8& 4\\
2& 1& \\

983&&&&&&&&&\\
2950& 1475& 4426& 2213& 6640& 3320& 1660& 830& 415& 1246\\
623& 1870& 935& 2806& 1403& 4210& 2105& 6316& 3158& 1579\\
4738& 2369& 7108& 3554& 1777& 5332& 2666& 1333& 4000& 2000\\
1000& 500& 250& 125& 376& 188& 94& 47& 142& 71\\
214& 107& 322& 161& 484& 242& 121& 364& 182& 91\\
274& 137& 412& 206& 103& 310& 155& 466& 233& 700\\
350& 175& 526& 263& 790& 395& 1186& 593& 1780& 890\\
445& 1336& 668& 334& 167& 502& 251& 754& 377& 1132\\
566& 283& 850& 425& 1276& 638& 319& 958& 479& 1438\\
719& 2158& 1079& 3238& 1619& 4858& 2429& 7288& 3644& 1822\\
911& 2734& 1367& 4102& 2051& 6154& 3077& 9232& 4616& 2308\\
1154& 577& 1732& 866& 433& 1300& 650& 325& 976& 488\\
244& 122& 61& 184& 92& 46& 23& 70& 35& 106\\
53& 160& 80& 40& 20& 10& 5& 16& 8& 4\\
2& 1& \\

984&&&&&&&&&\\
492& 246& 123& 370& 185& 556& 278& 139& 418& 209\\
628& 314& 157& 472& 236& 118& 59& 178& 89& 268\\
134& 67& 202& 101& 304& 152& 76& 38& 19& 58\\
29& 88& 44& 22& 11& 34& 17& 52& 26& 13\\
40& 20& 10& 5& 16& 8& 4& 2& 1& \\

985&&&&&&&&&\\
2956& 1478& 739& 2218& 1109& 3328& 1664& 832& 416& 208\\
104& 52& 26& 13& 40& 20& 10& 5& 16& 8\\
4& 2& 1& \\

986&&&&&&&&&\\
493& 1480& 740& 370& 185& 556& 278& 139& 418& 209\\
628& 314& 157& 472& 236& 118& 59& 178& 89& 268\\
134& 67& 202& 101& 304& 152& 76& 38& 19& 58\\
29& 88& 44& 22& 11& 34& 17& 52& 26& 13\\
40& 20& 10& 5& 16& 8& 4& 2& 1& \\

987&&&&&&&&&\\
2962& 1481& 4444& 2222& 1111& 3334& 1667& 5002& 2501& 7504\\
3752& 1876& 938& 469& 1408& 704& 352& 176& 88& 44\\
22& 11& 34& 17& 52& 26& 13& 40& 20& 10\\
5& 16& 8& 4& 2& 1& \\

988&&&&&&&&&\\
494& 247& 742& 371& 1114& 557& 1672& 836& 418& 209\\
628& 314& 157& 472& 236& 118& 59& 178& 89& 268\\
134& 67& 202& 101& 304& 152& 76& 38& 19& 58\\
29& 88& 44& 22& 11& 34& 17& 52& 26& 13\\
40& 20& 10& 5& 16& 8& 4& 2& 1& \\

989&&&&&&&&&\\
2968& 1484& 742& 371& 1114& 557& 1672& 836& 418& 209\\
628& 314& 157& 472& 236& 118& 59& 178& 89& 268\\
134& 67& 202& 101& 304& 152& 76& 38& 19& 58\\
29& 88& 44& 22& 11& 34& 17& 52& 26& 13\\
40& 20& 10& 5& 16& 8& 4& 2& 1& \\

990&&&&&&&&&\\
495& 1486& 743& 2230& 1115& 3346& 1673& 5020& 2510& 1255\\
3766& 1883& 5650& 2825& 8476& 4238& 2119& 6358& 3179& 9538\\
4769& 14308& 7154& 3577& 10732& 5366& 2683& 8050& 4025& 12076\\
6038& 3019& 9058& 4529& 13588& 6794& 3397& 10192& 5096& 2548\\
1274& 637& 1912& 956& 478& 239& 718& 359& 1078& 539\\
1618& 809& 2428& 1214& 607& 1822& 911& 2734& 1367& 4102\\
2051& 6154& 3077& 9232& 4616& 2308& 1154& 577& 1732& 866\\
433& 1300& 650& 325& 976& 488& 244& 122& 61& 184\\
92& 46& 23& 70& 35& 106& 53& 160& 80& 40\\
20& 10& 5& 16& 8& 4& 2& 1& \\

991&&&&&&&&&\\
2974& 1487& 4462& 2231& 6694& 3347& 10042& 5021& 15064& 7532\\
3766& 1883& 5650& 2825& 8476& 4238& 2119& 6358& 3179& 9538\\
4769& 14308& 7154& 3577& 10732& 5366& 2683& 8050& 4025& 12076\\
6038& 3019& 9058& 4529& 13588& 6794& 3397& 10192& 5096& 2548\\
1274& 637& 1912& 956& 478& 239& 718& 359& 1078& 539\\
1618& 809& 2428& 1214& 607& 1822& 911& 2734& 1367& 4102\\
2051& 6154& 3077& 9232& 4616& 2308& 1154& 577& 1732& 866\\
433& 1300& 650& 325& 976& 488& 244& 122& 61& 184\\
92& 46& 23& 70& 35& 106& 53& 160& 80& 40\\
20& 10& 5& 16& 8& 4& 2& 1& \\

992&&&&&&&&&\\
496& 248& 124& 62& 31& 94& 47& 142& 71& 214\\
107& 322& 161& 484& 242& 121& 364& 182& 91& 274\\
137& 412& 206& 103& 310& 155& 466& 233& 700& 350\\
175& 526& 263& 790& 395& 1186& 593& 1780& 890& 445\\
1336& 668& 334& 167& 502& 251& 754& 377& 1132& 566\\
283& 850& 425& 1276& 638& 319& 958& 479& 1438& 719\\
2158& 1079& 3238& 1619& 4858& 2429& 7288& 3644& 1822& 911\\
2734& 1367& 4102& 2051& 6154& 3077& 9232& 4616& 2308& 1154\\
577& 1732& 866& 433& 1300& 650& 325& 976& 488& 244\\
122& 61& 184& 92& 46& 23& 70& 35& 106& 53\\
160& 80& 40& 20& 10& 5& 16& 8& 4& 2\\
1& \\

993&&&&&&&&&\\
2980& 1490& 745& 2236& 1118& 559& 1678& 839& 2518& 1259\\
3778& 1889& 5668& 2834& 1417& 4252& 2126& 1063& 3190& 1595\\
4786& 2393& 7180& 3590& 1795& 5386& 2693& 8080& 4040& 2020\\
1010& 505& 1516& 758& 379& 1138& 569& 1708& 854& 427\\
1282& 641& 1924& 962& 481& 1444& 722& 361& 1084& 542\\
271& 814& 407& 1222& 611& 1834& 917& 2752& 1376& 688\\
344& 172& 86& 43& 130& 65& 196& 98& 49& 148\\
74& 37& 112& 56& 28& 14& 7& 22& 11& 34\\
17& 52& 26& 13& 40& 20& 10& 5& 16& 8\\
4& 2& 1& \\

994&&&&&&&&&\\
497& 1492& 746& 373& 1120& 560& 280& 140& 70& 35\\
106& 53& 160& 80& 40& 20& 10& 5& 16& 8\\
4& 2& 1& \\

995&&&&&&&&&\\
2986& 1493& 4480& 2240& 1120& 560& 280& 140& 70& 35\\
106& 53& 160& 80& 40& 20& 10& 5& 16& 8\\
4& 2& 1& \\

996&&&&&&&&&\\
498& 249& 748& 374& 187& 562& 281& 844& 422& 211\\
634& 317& 952& 476& 238& 119& 358& 179& 538& 269\\
808& 404& 202& 101& 304& 152& 76& 38& 19& 58\\
29& 88& 44& 22& 11& 34& 17& 52& 26& 13\\
40& 20& 10& 5& 16& 8& 4& 2& 1& \\

997&&&&&&&&&\\
2992& 1496& 748& 374& 187& 562& 281& 844& 422& 211\\
634& 317& 952& 476& 238& 119& 358& 179& 538& 269\\
808& 404& 202& 101& 304& 152& 76& 38& 19& 58\\
29& 88& 44& 22& 11& 34& 17& 52& 26& 13\\
40& 20& 10& 5& 16& 8& 4& 2& 1& \\

998&&&&&&&&&\\
499& 1498& 749& 2248& 1124& 562& 281& 844& 422& 211\\
634& 317& 952& 476& 238& 119& 358& 179& 538& 269\\
808& 404& 202& 101& 304& 152& 76& 38& 19& 58\\
29& 88& 44& 22& 11& 34& 17& 52& 26& 13\\
40& 20& 10& 5& 16& 8& 4& 2& 1& \\

999&&&&&&&&&\\
2998& 1499& 4498& 2249& 6748& 3374& 1687& 5062& 2531& 7594\\
3797& 11392& 5696& 2848& 1424& 712& 356& 178& 89& 268\\
134& 67& 202& 101& 304& 152& 76& 38& 19& 58\\
29& 88& 44& 22& 11& 34& 17& 52& 26& 13\\
40& 20& 10& 5& 16& 8& 4& 2& 1& \\

1000&&&&&&&&&\\
500& 250& 125& 376& 188& 94& 47& 142& 71& 214\\
107& 322& 161& 484& 242& 121& 364& 182& 91& 274\\
137& 412& 206& 103& 310& 155& 466& 233& 700& 350\\
175& 526& 263& 790& 395& 1186& 593& 1780& 890& 445\\
1336& 668& 334& 167& 502& 251& 754& 377& 1132& 566\\
283& 850& 425& 1276& 638& 319& 958& 479& 1438& 719\\
2158& 1079& 3238& 1619& 4858& 2429& 7288& 3644& 1822& 911\\
2734& 1367& 4102& 2051& 6154& 3077& 9232& 4616& 2308& 1154\\
577& 1732& 866& 433& 1300& 650& 325& 976& 488& 244\\
122& 61& 184& 92& 46& 23& 70& 35& 106& 53\\
160& 80& 40& 20& 10& 5& 16& 8& 4& 2\\
1& \\

1001&&&&&&&&&\\
3004& 1502& 751& 2254& 1127& 3382& 1691& 5074& 2537& 7612\\
3806& 1903& 5710& 2855& 8566& 4283& 12850& 6425& 19276& 9638\\
4819& 14458& 7229& 21688& 10844& 5422& 2711& 8134& 4067& 12202\\
6101& 18304& 9152& 4576& 2288& 1144& 572& 286& 143& 430\\
215& 646& 323& 970& 485& 1456& 728& 364& 182& 91\\
274& 137& 412& 206& 103& 310& 155& 466& 233& 700\\
350& 175& 526& 263& 790& 395& 1186& 593& 1780& 890\\
445& 1336& 668& 334& 167& 502& 251& 754& 377& 1132\\
566& 283& 850& 425& 1276& 638& 319& 958& 479& 1438\\
719& 2158& 1079& 3238& 1619& 4858& 2429& 7288& 3644& 1822\\
911& 2734& 1367& 4102& 2051& 6154& 3077& 9232& 4616& 2308\\
1154& 577& 1732& 866& 433& 1300& 650& 325& 976& 488\\
244& 122& 61& 184& 92& 46& 23& 70& 35& 106\\
53& 160& 80& 40& 20& 10& 5& 16& 8& 4\\
2& 1& \\

1002&&&&&&&&&\\
501& 1504& 752& 376& 188& 94& 47& 142& 71& 214\\
107& 322& 161& 484& 242& 121& 364& 182& 91& 274\\
137& 412& 206& 103& 310& 155& 466& 233& 700& 350\\
175& 526& 263& 790& 395& 1186& 593& 1780& 890& 445\\
1336& 668& 334& 167& 502& 251& 754& 377& 1132& 566\\
283& 850& 425& 1276& 638& 319& 958& 479& 1438& 719\\
2158& 1079& 3238& 1619& 4858& 2429& 7288& 3644& 1822& 911\\
2734& 1367& 4102& 2051& 6154& 3077& 9232& 4616& 2308& 1154\\
577& 1732& 866& 433& 1300& 650& 325& 976& 488& 244\\
122& 61& 184& 92& 46& 23& 70& 35& 106& 53\\
160& 80& 40& 20& 10& 5& 16& 8& 4& 2\\
1& \\

1003&&&&&&&&&\\
3010& 1505& 4516& 2258& 1129& 3388& 1694& 847& 2542& 1271\\
3814& 1907& 5722& 2861& 8584& 4292& 2146& 1073& 3220& 1610\\
805& 2416& 1208& 604& 302& 151& 454& 227& 682& 341\\
1024& 512& 256& 128& 64& 32& 16& 8& 4& 2\\
1& \\

1004&&&&&&&&&\\
502& 251& 754& 377& 1132& 566& 283& 850& 425& 1276\\
638& 319& 958& 479& 1438& 719& 2158& 1079& 3238& 1619\\
4858& 2429& 7288& 3644& 1822& 911& 2734& 1367& 4102& 2051\\
6154& 3077& 9232& 4616& 2308& 1154& 577& 1732& 866& 433\\
1300& 650& 325& 976& 488& 244& 122& 61& 184& 92\\
46& 23& 70& 35& 106& 53& 160& 80& 40& 20\\
10& 5& 16& 8& 4& 2& 1& \\

1005&&&&&&&&&\\
3016& 1508& 754& 377& 1132& 566& 283& 850& 425& 1276\\
638& 319& 958& 479& 1438& 719& 2158& 1079& 3238& 1619\\
4858& 2429& 7288& 3644& 1822& 911& 2734& 1367& 4102& 2051\\
6154& 3077& 9232& 4616& 2308& 1154& 577& 1732& 866& 433\\
1300& 650& 325& 976& 488& 244& 122& 61& 184& 92\\
46& 23& 70& 35& 106& 53& 160& 80& 40& 20\\
10& 5& 16& 8& 4& 2& 1& \\

1006&&&&&&&&&\\
503& 1510& 755& 2266& 1133& 3400& 1700& 850& 425& 1276\\
638& 319& 958& 479& 1438& 719& 2158& 1079& 3238& 1619\\
4858& 2429& 7288& 3644& 1822& 911& 2734& 1367& 4102& 2051\\
6154& 3077& 9232& 4616& 2308& 1154& 577& 1732& 866& 433\\
1300& 650& 325& 976& 488& 244& 122& 61& 184& 92\\
46& 23& 70& 35& 106& 53& 160& 80& 40& 20\\
10& 5& 16& 8& 4& 2& 1& \\

1007&&&&&&&&&\\
3022& 1511& 4534& 2267& 6802& 3401& 10204& 5102& 2551& 7654\\
3827& 11482& 5741& 17224& 8612& 4306& 2153& 6460& 3230& 1615\\
4846& 2423& 7270& 3635& 10906& 5453& 16360& 8180& 4090& 2045\\
6136& 3068& 1534& 767& 2302& 1151& 3454& 1727& 5182& 2591\\
7774& 3887& 11662& 5831& 17494& 8747& 26242& 13121& 39364& 19682\\
9841& 29524& 14762& 7381& 22144& 11072& 5536& 2768& 1384& 692\\
346& 173& 520& 260& 130& 65& 196& 98& 49& 148\\
74& 37& 112& 56& 28& 14& 7& 22& 11& 34\\
17& 52& 26& 13& 40& 20& 10& 5& 16& 8\\
4& 2& 1& \\

1008&&&&&&&&&\\
504& 252& 126& 63& 190& 95& 286& 143& 430& 215\\
646& 323& 970& 485& 1456& 728& 364& 182& 91& 274\\
137& 412& 206& 103& 310& 155& 466& 233& 700& 350\\
175& 526& 263& 790& 395& 1186& 593& 1780& 890& 445\\
1336& 668& 334& 167& 502& 251& 754& 377& 1132& 566\\
283& 850& 425& 1276& 638& 319& 958& 479& 1438& 719\\
2158& 1079& 3238& 1619& 4858& 2429& 7288& 3644& 1822& 911\\
2734& 1367& 4102& 2051& 6154& 3077& 9232& 4616& 2308& 1154\\
577& 1732& 866& 433& 1300& 650& 325& 976& 488& 244\\
122& 61& 184& 92& 46& 23& 70& 35& 106& 53\\
160& 80& 40& 20& 10& 5& 16& 8& 4& 2\\
1& \\

1009&&&&&&&&&\\
3028& 1514& 757& 2272& 1136& 568& 284& 142& 71& 214\\
107& 322& 161& 484& 242& 121& 364& 182& 91& 274\\
137& 412& 206& 103& 310& 155& 466& 233& 700& 350\\
175& 526& 263& 790& 395& 1186& 593& 1780& 890& 445\\
1336& 668& 334& 167& 502& 251& 754& 377& 1132& 566\\
283& 850& 425& 1276& 638& 319& 958& 479& 1438& 719\\
2158& 1079& 3238& 1619& 4858& 2429& 7288& 3644& 1822& 911\\
2734& 1367& 4102& 2051& 6154& 3077& 9232& 4616& 2308& 1154\\
577& 1732& 866& 433& 1300& 650& 325& 976& 488& 244\\
122& 61& 184& 92& 46& 23& 70& 35& 106& 53\\
160& 80& 40& 20& 10& 5& 16& 8& 4& 2\\
1& \\

1010&&&&&&&&&\\
505& 1516& 758& 379& 1138& 569& 1708& 854& 427& 1282\\
641& 1924& 962& 481& 1444& 722& 361& 1084& 542& 271\\
814& 407& 1222& 611& 1834& 917& 2752& 1376& 688& 344\\
172& 86& 43& 130& 65& 196& 98& 49& 148& 74\\
37& 112& 56& 28& 14& 7& 22& 11& 34& 17\\
52& 26& 13& 40& 20& 10& 5& 16& 8& 4\\
2& 1& \\

1011&&&&&&&&&\\
3034& 1517& 4552& 2276& 1138& 569& 1708& 854& 427& 1282\\
641& 1924& 962& 481& 1444& 722& 361& 1084& 542& 271\\
814& 407& 1222& 611& 1834& 917& 2752& 1376& 688& 344\\
172& 86& 43& 130& 65& 196& 98& 49& 148& 74\\
37& 112& 56& 28& 14& 7& 22& 11& 34& 17\\
52& 26& 13& 40& 20& 10& 5& 16& 8& 4\\
2& 1& \\

1012&&&&&&&&&\\
506& 253& 760& 380& 190& 95& 286& 143& 430& 215\\
646& 323& 970& 485& 1456& 728& 364& 182& 91& 274\\
137& 412& 206& 103& 310& 155& 466& 233& 700& 350\\
175& 526& 263& 790& 395& 1186& 593& 1780& 890& 445\\
1336& 668& 334& 167& 502& 251& 754& 377& 1132& 566\\
283& 850& 425& 1276& 638& 319& 958& 479& 1438& 719\\
2158& 1079& 3238& 1619& 4858& 2429& 7288& 3644& 1822& 911\\
2734& 1367& 4102& 2051& 6154& 3077& 9232& 4616& 2308& 1154\\
577& 1732& 866& 433& 1300& 650& 325& 976& 488& 244\\
122& 61& 184& 92& 46& 23& 70& 35& 106& 53\\
160& 80& 40& 20& 10& 5& 16& 8& 4& 2\\
1& \\

1013&&&&&&&&&\\
3040& 1520& 760& 380& 190& 95& 286& 143& 430& 215\\
646& 323& 970& 485& 1456& 728& 364& 182& 91& 274\\
137& 412& 206& 103& 310& 155& 466& 233& 700& 350\\
175& 526& 263& 790& 395& 1186& 593& 1780& 890& 445\\
1336& 668& 334& 167& 502& 251& 754& 377& 1132& 566\\
283& 850& 425& 1276& 638& 319& 958& 479& 1438& 719\\
2158& 1079& 3238& 1619& 4858& 2429& 7288& 3644& 1822& 911\\
2734& 1367& 4102& 2051& 6154& 3077& 9232& 4616& 2308& 1154\\
577& 1732& 866& 433& 1300& 650& 325& 976& 488& 244\\
122& 61& 184& 92& 46& 23& 70& 35& 106& 53\\
160& 80& 40& 20& 10& 5& 16& 8& 4& 2\\
1& \\

1014&&&&&&&&&\\
507& 1522& 761& 2284& 1142& 571& 1714& 857& 2572& 1286\\
643& 1930& 965& 2896& 1448& 724& 362& 181& 544& 272\\
136& 68& 34& 17& 52& 26& 13& 40& 20& 10\\
5& 16& 8& 4& 2& 1& \\

1015&&&&&&&&&\\
3046& 1523& 4570& 2285& 6856& 3428& 1714& 857& 2572& 1286\\
643& 1930& 965& 2896& 1448& 724& 362& 181& 544& 272\\
136& 68& 34& 17& 52& 26& 13& 40& 20& 10\\
5& 16& 8& 4& 2& 1& \\

1016&&&&&&&&&\\
508& 254& 127& 382& 191& 574& 287& 862& 431& 1294\\
647& 1942& 971& 2914& 1457& 4372& 2186& 1093& 3280& 1640\\
820& 410& 205& 616& 308& 154& 77& 232& 116& 58\\
29& 88& 44& 22& 11& 34& 17& 52& 26& 13\\
40& 20& 10& 5& 16& 8& 4& 2& 1& \\

1017&&&&&&&&&\\
3052& 1526& 763& 2290& 1145& 3436& 1718& 859& 2578& 1289\\
3868& 1934& 967& 2902& 1451& 4354& 2177& 6532& 3266& 1633\\
4900& 2450& 1225& 3676& 1838& 919& 2758& 1379& 4138& 2069\\
6208& 3104& 1552& 776& 388& 194& 97& 292& 146& 73\\
220& 110& 55& 166& 83& 250& 125& 376& 188& 94\\
47& 142& 71& 214& 107& 322& 161& 484& 242& 121\\
364& 182& 91& 274& 137& 412& 206& 103& 310& 155\\
466& 233& 700& 350& 175& 526& 263& 790& 395& 1186\\
593& 1780& 890& 445& 1336& 668& 334& 167& 502& 251\\
754& 377& 1132& 566& 283& 850& 425& 1276& 638& 319\\
958& 479& 1438& 719& 2158& 1079& 3238& 1619& 4858& 2429\\
7288& 3644& 1822& 911& 2734& 1367& 4102& 2051& 6154& 3077\\
9232& 4616& 2308& 1154& 577& 1732& 866& 433& 1300& 650\\
325& 976& 488& 244& 122& 61& 184& 92& 46& 23\\
70& 35& 106& 53& 160& 80& 40& 20& 10& 5\\
16& 8& 4& 2& 1& \\

1018&&&&&&&&&\\
509& 1528& 764& 382& 191& 574& 287& 862& 431& 1294\\
647& 1942& 971& 2914& 1457& 4372& 2186& 1093& 3280& 1640\\
820& 410& 205& 616& 308& 154& 77& 232& 116& 58\\
29& 88& 44& 22& 11& 34& 17& 52& 26& 13\\
40& 20& 10& 5& 16& 8& 4& 2& 1& \\

1019&&&&&&&&&\\
3058& 1529& 4588& 2294& 1147& 3442& 1721& 5164& 2582& 1291\\
3874& 1937& 5812& 2906& 1453& 4360& 2180& 1090& 545& 1636\\
818& 409& 1228& 614& 307& 922& 461& 1384& 692& 346\\
173& 520& 260& 130& 65& 196& 98& 49& 148& 74\\
37& 112& 56& 28& 14& 7& 22& 11& 34& 17\\
52& 26& 13& 40& 20& 10& 5& 16& 8& 4\\
2& 1& \\

1020&&&&&&&&&\\
510& 255& 766& 383& 1150& 575& 1726& 863& 2590& 1295\\
3886& 1943& 5830& 2915& 8746& 4373& 13120& 6560& 3280& 1640\\
820& 410& 205& 616& 308& 154& 77& 232& 116& 58\\
29& 88& 44& 22& 11& 34& 17& 52& 26& 13\\
40& 20& 10& 5& 16& 8& 4& 2& 1& \\

1021&&&&&&&&&\\
3064& 1532& 766& 383& 1150& 575& 1726& 863& 2590& 1295\\
3886& 1943& 5830& 2915& 8746& 4373& 13120& 6560& 3280& 1640\\
820& 410& 205& 616& 308& 154& 77& 232& 116& 58\\
29& 88& 44& 22& 11& 34& 17& 52& 26& 13\\
40& 20& 10& 5& 16& 8& 4& 2& 1& \\

1022&&&&&&&&&\\
511& 1534& 767& 2302& 1151& 3454& 1727& 5182& 2591& 7774\\
3887& 11662& 5831& 17494& 8747& 26242& 13121& 39364& 19682& 9841\\
29524& 14762& 7381& 22144& 11072& 5536& 2768& 1384& 692& 346\\
173& 520& 260& 130& 65& 196& 98& 49& 148& 74\\
37& 112& 56& 28& 14& 7& 22& 11& 34& 17\\
52& 26& 13& 40& 20& 10& 5& 16& 8& 4\\
2& 1& \\

1023&&&&&&&&&\\
3070& 1535& 4606& 2303& 6910& 3455& 10366& 5183& 15550& 7775\\
23326& 11663& 34990& 17495& 52486& 26243& 78730& 39365& 118096& 59048\\
29524& 14762& 7381& 22144& 11072& 5536& 2768& 1384& 692& 346\\
173& 520& 260& 130& 65& 196& 98& 49& 148& 74\\
37& 112& 56& 28& 14& 7& 22& 11& 34& 17\\
52& 26& 13& 40& 20& 10& 5& 16& 8& 4\\
2& 1& \\

1024&&&&&&&&&\\
512& 256& 128& 64& 32& 16& 8& 4& 2& 1\\

1025&&&&&&&&&\\
3076& 1538& 769& 2308& 1154& 577& 1732& 866& 433& 1300\\
650& 325& 976& 488& 244& 122& 61& 184& 92& 46\\
23& 70& 35& 106& 53& 160& 80& 40& 20& 10\\
5& 16& 8& 4& 2& 1& \\

1026&&&&&&&&&\\
513& 1540& 770& 385& 1156& 578& 289& 868& 434& 217\\
652& 326& 163& 490& 245& 736& 368& 184& 92& 46\\
23& 70& 35& 106& 53& 160& 80& 40& 20& 10\\
5& 16& 8& 4& 2& 1& \\

1027&&&&&&&&&\\
3082& 1541& 4624& 2312& 1156& 578& 289& 868& 434& 217\\
652& 326& 163& 490& 245& 736& 368& 184& 92& 46\\
23& 70& 35& 106& 53& 160& 80& 40& 20& 10\\
5& 16& 8& 4& 2& 1& \\

1028&&&&&&&&&\\
514& 257& 772& 386& 193& 580& 290& 145& 436& 218\\
109& 328& 164& 82& 41& 124& 62& 31& 94& 47\\
142& 71& 214& 107& 322& 161& 484& 242& 121& 364\\
182& 91& 274& 137& 412& 206& 103& 310& 155& 466\\
233& 700& 350& 175& 526& 263& 790& 395& 1186& 593\\
1780& 890& 445& 1336& 668& 334& 167& 502& 251& 754\\
377& 1132& 566& 283& 850& 425& 1276& 638& 319& 958\\
479& 1438& 719& 2158& 1079& 3238& 1619& 4858& 2429& 7288\\
3644& 1822& 911& 2734& 1367& 4102& 2051& 6154& 3077& 9232\\
4616& 2308& 1154& 577& 1732& 866& 433& 1300& 650& 325\\
976& 488& 244& 122& 61& 184& 92& 46& 23& 70\\
35& 106& 53& 160& 80& 40& 20& 10& 5& 16\\
8& 4& 2& 1& \\

1029&&&&&&&&&\\
3088& 1544& 772& 386& 193& 580& 290& 145& 436& 218\\
109& 328& 164& 82& 41& 124& 62& 31& 94& 47\\
142& 71& 214& 107& 322& 161& 484& 242& 121& 364\\
182& 91& 274& 137& 412& 206& 103& 310& 155& 466\\
233& 700& 350& 175& 526& 263& 790& 395& 1186& 593\\
1780& 890& 445& 1336& 668& 334& 167& 502& 251& 754\\
377& 1132& 566& 283& 850& 425& 1276& 638& 319& 958\\
479& 1438& 719& 2158& 1079& 3238& 1619& 4858& 2429& 7288\\
3644& 1822& 911& 2734& 1367& 4102& 2051& 6154& 3077& 9232\\
4616& 2308& 1154& 577& 1732& 866& 433& 1300& 650& 325\\
976& 488& 244& 122& 61& 184& 92& 46& 23& 70\\
35& 106& 53& 160& 80& 40& 20& 10& 5& 16\\
8& 4& 2& 1& \\

1030&&&&&&&&&\\
515& 1546& 773& 2320& 1160& 580& 290& 145& 436& 218\\
109& 328& 164& 82& 41& 124& 62& 31& 94& 47\\
142& 71& 214& 107& 322& 161& 484& 242& 121& 364\\
182& 91& 274& 137& 412& 206& 103& 310& 155& 466\\
233& 700& 350& 175& 526& 263& 790& 395& 1186& 593\\
1780& 890& 445& 1336& 668& 334& 167& 502& 251& 754\\
377& 1132& 566& 283& 850& 425& 1276& 638& 319& 958\\
479& 1438& 719& 2158& 1079& 3238& 1619& 4858& 2429& 7288\\
3644& 1822& 911& 2734& 1367& 4102& 2051& 6154& 3077& 9232\\
4616& 2308& 1154& 577& 1732& 866& 433& 1300& 650& 325\\
976& 488& 244& 122& 61& 184& 92& 46& 23& 70\\
35& 106& 53& 160& 80& 40& 20& 10& 5& 16\\
8& 4& 2& 1& \\

1031&&&&&&&&&\\
3094& 1547& 4642& 2321& 6964& 3482& 1741& 5224& 2612& 1306\\
653& 1960& 980& 490& 245& 736& 368& 184& 92& 46\\
23& 70& 35& 106& 53& 160& 80& 40& 20& 10\\
5& 16& 8& 4& 2& 1& \\

1032&&&&&&&&&\\
516& 258& 129& 388& 194& 97& 292& 146& 73& 220\\
110& 55& 166& 83& 250& 125& 376& 188& 94& 47\\
142& 71& 214& 107& 322& 161& 484& 242& 121& 364\\
182& 91& 274& 137& 412& 206& 103& 310& 155& 466\\
233& 700& 350& 175& 526& 263& 790& 395& 1186& 593\\
1780& 890& 445& 1336& 668& 334& 167& 502& 251& 754\\
377& 1132& 566& 283& 850& 425& 1276& 638& 319& 958\\
479& 1438& 719& 2158& 1079& 3238& 1619& 4858& 2429& 7288\\
3644& 1822& 911& 2734& 1367& 4102& 2051& 6154& 3077& 9232\\
4616& 2308& 1154& 577& 1732& 866& 433& 1300& 650& 325\\
976& 488& 244& 122& 61& 184& 92& 46& 23& 70\\
35& 106& 53& 160& 80& 40& 20& 10& 5& 16\\
8& 4& 2& 1& \\

1033&&&&&&&&&\\
3100& 1550& 775& 2326& 1163& 3490& 1745& 5236& 2618& 1309\\
3928& 1964& 982& 491& 1474& 737& 2212& 1106& 553& 1660\\
830& 415& 1246& 623& 1870& 935& 2806& 1403& 4210& 2105\\
6316& 3158& 1579& 4738& 2369& 7108& 3554& 1777& 5332& 2666\\
1333& 4000& 2000& 1000& 500& 250& 125& 376& 188& 94\\
47& 142& 71& 214& 107& 322& 161& 484& 242& 121\\
364& 182& 91& 274& 137& 412& 206& 103& 310& 155\\
466& 233& 700& 350& 175& 526& 263& 790& 395& 1186\\
593& 1780& 890& 445& 1336& 668& 334& 167& 502& 251\\
754& 377& 1132& 566& 283& 850& 425& 1276& 638& 319\\
958& 479& 1438& 719& 2158& 1079& 3238& 1619& 4858& 2429\\
7288& 3644& 1822& 911& 2734& 1367& 4102& 2051& 6154& 3077\\
9232& 4616& 2308& 1154& 577& 1732& 866& 433& 1300& 650\\
325& 976& 488& 244& 122& 61& 184& 92& 46& 23\\
70& 35& 106& 53& 160& 80& 40& 20& 10& 5\\
16& 8& 4& 2& 1& \\

1034&&&&&&&&&\\
517& 1552& 776& 388& 194& 97& 292& 146& 73& 220\\
110& 55& 166& 83& 250& 125& 376& 188& 94& 47\\
142& 71& 214& 107& 322& 161& 484& 242& 121& 364\\
182& 91& 274& 137& 412& 206& 103& 310& 155& 466\\
233& 700& 350& 175& 526& 263& 790& 395& 1186& 593\\
1780& 890& 445& 1336& 668& 334& 167& 502& 251& 754\\
377& 1132& 566& 283& 850& 425& 1276& 638& 319& 958\\
479& 1438& 719& 2158& 1079& 3238& 1619& 4858& 2429& 7288\\
3644& 1822& 911& 2734& 1367& 4102& 2051& 6154& 3077& 9232\\
4616& 2308& 1154& 577& 1732& 866& 433& 1300& 650& 325\\
976& 488& 244& 122& 61& 184& 92& 46& 23& 70\\
35& 106& 53& 160& 80& 40& 20& 10& 5& 16\\
8& 4& 2& 1& \\

1035&&&&&&&&&\\
3106& 1553& 4660& 2330& 1165& 3496& 1748& 874& 437& 1312\\
656& 328& 164& 82& 41& 124& 62& 31& 94& 47\\
142& 71& 214& 107& 322& 161& 484& 242& 121& 364\\
182& 91& 274& 137& 412& 206& 103& 310& 155& 466\\
233& 700& 350& 175& 526& 263& 790& 395& 1186& 593\\
1780& 890& 445& 1336& 668& 334& 167& 502& 251& 754\\
377& 1132& 566& 283& 850& 425& 1276& 638& 319& 958\\
479& 1438& 719& 2158& 1079& 3238& 1619& 4858& 2429& 7288\\
3644& 1822& 911& 2734& 1367& 4102& 2051& 6154& 3077& 9232\\
4616& 2308& 1154& 577& 1732& 866& 433& 1300& 650& 325\\
976& 488& 244& 122& 61& 184& 92& 46& 23& 70\\
35& 106& 53& 160& 80& 40& 20& 10& 5& 16\\
8& 4& 2& 1& \\

1036&&&&&&&&&\\
518& 259& 778& 389& 1168& 584& 292& 146& 73& 220\\
110& 55& 166& 83& 250& 125& 376& 188& 94& 47\\
142& 71& 214& 107& 322& 161& 484& 242& 121& 364\\
182& 91& 274& 137& 412& 206& 103& 310& 155& 466\\
233& 700& 350& 175& 526& 263& 790& 395& 1186& 593\\
1780& 890& 445& 1336& 668& 334& 167& 502& 251& 754\\
377& 1132& 566& 283& 850& 425& 1276& 638& 319& 958\\
479& 1438& 719& 2158& 1079& 3238& 1619& 4858& 2429& 7288\\
3644& 1822& 911& 2734& 1367& 4102& 2051& 6154& 3077& 9232\\
4616& 2308& 1154& 577& 1732& 866& 433& 1300& 650& 325\\
976& 488& 244& 122& 61& 184& 92& 46& 23& 70\\
35& 106& 53& 160& 80& 40& 20& 10& 5& 16\\
8& 4& 2& 1& \\

1037&&&&&&&&&\\
3112& 1556& 778& 389& 1168& 584& 292& 146& 73& 220\\
110& 55& 166& 83& 250& 125& 376& 188& 94& 47\\
142& 71& 214& 107& 322& 161& 484& 242& 121& 364\\
182& 91& 274& 137& 412& 206& 103& 310& 155& 466\\
233& 700& 350& 175& 526& 263& 790& 395& 1186& 593\\
1780& 890& 445& 1336& 668& 334& 167& 502& 251& 754\\
377& 1132& 566& 283& 850& 425& 1276& 638& 319& 958\\
479& 1438& 719& 2158& 1079& 3238& 1619& 4858& 2429& 7288\\
3644& 1822& 911& 2734& 1367& 4102& 2051& 6154& 3077& 9232\\
4616& 2308& 1154& 577& 1732& 866& 433& 1300& 650& 325\\
976& 488& 244& 122& 61& 184& 92& 46& 23& 70\\
35& 106& 53& 160& 80& 40& 20& 10& 5& 16\\
8& 4& 2& 1& \\

1038&&&&&&&&&\\
519& 1558& 779& 2338& 1169& 3508& 1754& 877& 2632& 1316\\
658& 329& 988& 494& 247& 742& 371& 1114& 557& 1672\\
836& 418& 209& 628& 314& 157& 472& 236& 118& 59\\
178& 89& 268& 134& 67& 202& 101& 304& 152& 76\\
38& 19& 58& 29& 88& 44& 22& 11& 34& 17\\
52& 26& 13& 40& 20& 10& 5& 16& 8& 4\\
2& 1& \\

1039&&&&&&&&&\\
3118& 1559& 4678& 2339& 7018& 3509& 10528& 5264& 2632& 1316\\
658& 329& 988& 494& 247& 742& 371& 1114& 557& 1672\\
836& 418& 209& 628& 314& 157& 472& 236& 118& 59\\
178& 89& 268& 134& 67& 202& 101& 304& 152& 76\\
38& 19& 58& 29& 88& 44& 22& 11& 34& 17\\
52& 26& 13& 40& 20& 10& 5& 16& 8& 4\\
2& 1& \\

1040&&&&&&&&&\\
520& 260& 130& 65& 196& 98& 49& 148& 74& 37\\
112& 56& 28& 14& 7& 22& 11& 34& 17& 52\\
26& 13& 40& 20& 10& 5& 16& 8& 4& 2\\
1& \\

1041&&&&&&&&&\\
3124& 1562& 781& 2344& 1172& 586& 293& 880& 440& 220\\
110& 55& 166& 83& 250& 125& 376& 188& 94& 47\\
142& 71& 214& 107& 322& 161& 484& 242& 121& 364\\
182& 91& 274& 137& 412& 206& 103& 310& 155& 466\\
233& 700& 350& 175& 526& 263& 790& 395& 1186& 593\\
1780& 890& 445& 1336& 668& 334& 167& 502& 251& 754\\
377& 1132& 566& 283& 850& 425& 1276& 638& 319& 958\\
479& 1438& 719& 2158& 1079& 3238& 1619& 4858& 2429& 7288\\
3644& 1822& 911& 2734& 1367& 4102& 2051& 6154& 3077& 9232\\
4616& 2308& 1154& 577& 1732& 866& 433& 1300& 650& 325\\
976& 488& 244& 122& 61& 184& 92& 46& 23& 70\\
35& 106& 53& 160& 80& 40& 20& 10& 5& 16\\
8& 4& 2& 1& \\

1042&&&&&&&&&\\
521& 1564& 782& 391& 1174& 587& 1762& 881& 2644& 1322\\
661& 1984& 992& 496& 248& 124& 62& 31& 94& 47\\
142& 71& 214& 107& 322& 161& 484& 242& 121& 364\\
182& 91& 274& 137& 412& 206& 103& 310& 155& 466\\
233& 700& 350& 175& 526& 263& 790& 395& 1186& 593\\
1780& 890& 445& 1336& 668& 334& 167& 502& 251& 754\\
377& 1132& 566& 283& 850& 425& 1276& 638& 319& 958\\
479& 1438& 719& 2158& 1079& 3238& 1619& 4858& 2429& 7288\\
3644& 1822& 911& 2734& 1367& 4102& 2051& 6154& 3077& 9232\\
4616& 2308& 1154& 577& 1732& 866& 433& 1300& 650& 325\\
976& 488& 244& 122& 61& 184& 92& 46& 23& 70\\
35& 106& 53& 160& 80& 40& 20& 10& 5& 16\\
8& 4& 2& 1& \\

1043&&&&&&&&&\\
3130& 1565& 4696& 2348& 1174& 587& 1762& 881& 2644& 1322\\
661& 1984& 992& 496& 248& 124& 62& 31& 94& 47\\
142& 71& 214& 107& 322& 161& 484& 242& 121& 364\\
182& 91& 274& 137& 412& 206& 103& 310& 155& 466\\
233& 700& 350& 175& 526& 263& 790& 395& 1186& 593\\
1780& 890& 445& 1336& 668& 334& 167& 502& 251& 754\\
377& 1132& 566& 283& 850& 425& 1276& 638& 319& 958\\
479& 1438& 719& 2158& 1079& 3238& 1619& 4858& 2429& 7288\\
3644& 1822& 911& 2734& 1367& 4102& 2051& 6154& 3077& 9232\\
4616& 2308& 1154& 577& 1732& 866& 433& 1300& 650& 325\\
976& 488& 244& 122& 61& 184& 92& 46& 23& 70\\
35& 106& 53& 160& 80& 40& 20& 10& 5& 16\\
8& 4& 2& 1& \\

1044&&&&&&&&&\\
522& 261& 784& 392& 196& 98& 49& 148& 74& 37\\
112& 56& 28& 14& 7& 22& 11& 34& 17& 52\\
26& 13& 40& 20& 10& 5& 16& 8& 4& 2\\
1& \\

1045&&&&&&&&&\\
3136& 1568& 784& 392& 196& 98& 49& 148& 74& 37\\
112& 56& 28& 14& 7& 22& 11& 34& 17& 52\\
26& 13& 40& 20& 10& 5& 16& 8& 4& 2\\
1& \\

1046&&&&&&&&&\\
523& 1570& 785& 2356& 1178& 589& 1768& 884& 442& 221\\
664& 332& 166& 83& 250& 125& 376& 188& 94& 47\\
142& 71& 214& 107& 322& 161& 484& 242& 121& 364\\
182& 91& 274& 137& 412& 206& 103& 310& 155& 466\\
233& 700& 350& 175& 526& 263& 790& 395& 1186& 593\\
1780& 890& 445& 1336& 668& 334& 167& 502& 251& 754\\
377& 1132& 566& 283& 850& 425& 1276& 638& 319& 958\\
479& 1438& 719& 2158& 1079& 3238& 1619& 4858& 2429& 7288\\
3644& 1822& 911& 2734& 1367& 4102& 2051& 6154& 3077& 9232\\
4616& 2308& 1154& 577& 1732& 866& 433& 1300& 650& 325\\
976& 488& 244& 122& 61& 184& 92& 46& 23& 70\\
35& 106& 53& 160& 80& 40& 20& 10& 5& 16\\
8& 4& 2& 1& \\

1047&&&&&&&&&\\
3142& 1571& 4714& 2357& 7072& 3536& 1768& 884& 442& 221\\
664& 332& 166& 83& 250& 125& 376& 188& 94& 47\\
142& 71& 214& 107& 322& 161& 484& 242& 121& 364\\
182& 91& 274& 137& 412& 206& 103& 310& 155& 466\\
233& 700& 350& 175& 526& 263& 790& 395& 1186& 593\\
1780& 890& 445& 1336& 668& 334& 167& 502& 251& 754\\
377& 1132& 566& 283& 850& 425& 1276& 638& 319& 958\\
479& 1438& 719& 2158& 1079& 3238& 1619& 4858& 2429& 7288\\
3644& 1822& 911& 2734& 1367& 4102& 2051& 6154& 3077& 9232\\
4616& 2308& 1154& 577& 1732& 866& 433& 1300& 650& 325\\
976& 488& 244& 122& 61& 184& 92& 46& 23& 70\\
35& 106& 53& 160& 80& 40& 20& 10& 5& 16\\
8& 4& 2& 1& \\

1048&&&&&&&&&\\
524& 262& 131& 394& 197& 592& 296& 148& 74& 37\\
112& 56& 28& 14& 7& 22& 11& 34& 17& 52\\
26& 13& 40& 20& 10& 5& 16& 8& 4& 2\\
1& \\

1049&&&&&&&&&\\
3148& 1574& 787& 2362& 1181& 3544& 1772& 886& 443& 1330\\
665& 1996& 998& 499& 1498& 749& 2248& 1124& 562& 281\\
844& 422& 211& 634& 317& 952& 476& 238& 119& 358\\
179& 538& 269& 808& 404& 202& 101& 304& 152& 76\\
38& 19& 58& 29& 88& 44& 22& 11& 34& 17\\
52& 26& 13& 40& 20& 10& 5& 16& 8& 4\\
2& 1& \\

1050&&&&&&&&&\\
525& 1576& 788& 394& 197& 592& 296& 148& 74& 37\\
112& 56& 28& 14& 7& 22& 11& 34& 17& 52\\
26& 13& 40& 20& 10& 5& 16& 8& 4& 2\\
1& \\

1051&&&&&&&&&\\
3154& 1577& 4732& 2366& 1183& 3550& 1775& 5326& 2663& 7990\\
3995& 11986& 5993& 17980& 8990& 4495& 13486& 6743& 20230& 10115\\
30346& 15173& 45520& 22760& 11380& 5690& 2845& 8536& 4268& 2134\\
1067& 3202& 1601& 4804& 2402& 1201& 3604& 1802& 901& 2704\\
1352& 676& 338& 169& 508& 254& 127& 382& 191& 574\\
287& 862& 431& 1294& 647& 1942& 971& 2914& 1457& 4372\\
2186& 1093& 3280& 1640& 820& 410& 205& 616& 308& 154\\
77& 232& 116& 58& 29& 88& 44& 22& 11& 34\\
17& 52& 26& 13& 40& 20& 10& 5& 16& 8\\
4& 2& 1& \\

1052&&&&&&&&&\\
526& 263& 790& 395& 1186& 593& 1780& 890& 445& 1336\\
668& 334& 167& 502& 251& 754& 377& 1132& 566& 283\\
850& 425& 1276& 638& 319& 958& 479& 1438& 719& 2158\\
1079& 3238& 1619& 4858& 2429& 7288& 3644& 1822& 911& 2734\\
1367& 4102& 2051& 6154& 3077& 9232& 4616& 2308& 1154& 577\\
1732& 866& 433& 1300& 650& 325& 976& 488& 244& 122\\
61& 184& 92& 46& 23& 70& 35& 106& 53& 160\\
80& 40& 20& 10& 5& 16& 8& 4& 2& 1\\

1053&&&&&&&&&\\
3160& 1580& 790& 395& 1186& 593& 1780& 890& 445& 1336\\
668& 334& 167& 502& 251& 754& 377& 1132& 566& 283\\
850& 425& 1276& 638& 319& 958& 479& 1438& 719& 2158\\
1079& 3238& 1619& 4858& 2429& 7288& 3644& 1822& 911& 2734\\
1367& 4102& 2051& 6154& 3077& 9232& 4616& 2308& 1154& 577\\
1732& 866& 433& 1300& 650& 325& 976& 488& 244& 122\\
61& 184& 92& 46& 23& 70& 35& 106& 53& 160\\
80& 40& 20& 10& 5& 16& 8& 4& 2& 1\\

1054&&&&&&&&&\\
527& 1582& 791& 2374& 1187& 3562& 1781& 5344& 2672& 1336\\
668& 334& 167& 502& 251& 754& 377& 1132& 566& 283\\
850& 425& 1276& 638& 319& 958& 479& 1438& 719& 2158\\
1079& 3238& 1619& 4858& 2429& 7288& 3644& 1822& 911& 2734\\
1367& 4102& 2051& 6154& 3077& 9232& 4616& 2308& 1154& 577\\
1732& 866& 433& 1300& 650& 325& 976& 488& 244& 122\\
61& 184& 92& 46& 23& 70& 35& 106& 53& 160\\
80& 40& 20& 10& 5& 16& 8& 4& 2& 1\\

1055&&&&&&&&&\\
3166& 1583& 4750& 2375& 7126& 3563& 10690& 5345& 16036& 8018\\
4009& 12028& 6014& 3007& 9022& 4511& 13534& 6767& 20302& 10151\\
30454& 15227& 45682& 22841& 68524& 34262& 17131& 51394& 25697& 77092\\
38546& 19273& 57820& 28910& 14455& 43366& 21683& 65050& 32525& 97576\\
48788& 24394& 12197& 36592& 18296& 9148& 4574& 2287& 6862& 3431\\
10294& 5147& 15442& 7721& 23164& 11582& 5791& 17374& 8687& 26062\\
13031& 39094& 19547& 58642& 29321& 87964& 43982& 21991& 65974& 32987\\
98962& 49481& 148444& 74222& 37111& 111334& 55667& 167002& 83501& 250504\\
125252& 62626& 31313& 93940& 46970& 23485& 70456& 35228& 17614& 8807\\
26422& 13211& 39634& 19817& 59452& 29726& 14863& 44590& 22295& 66886\\
33443& 100330& 50165& 150496& 75248& 37624& 18812& 9406& 4703& 14110\\
7055& 21166& 10583& 31750& 15875& 47626& 23813& 71440& 35720& 17860\\
8930& 4465& 13396& 6698& 3349& 10048& 5024& 2512& 1256& 628\\
314& 157& 472& 236& 118& 59& 178& 89& 268& 134\\
67& 202& 101& 304& 152& 76& 38& 19& 58& 29\\
88& 44& 22& 11& 34& 17& 52& 26& 13& 40\\
20& 10& 5& 16& 8& 4& 2& 1& \\

1056&&&&&&&&&\\
528& 264& 132& 66& 33& 100& 50& 25& 76& 38\\
19& 58& 29& 88& 44& 22& 11& 34& 17& 52\\
26& 13& 40& 20& 10& 5& 16& 8& 4& 2\\
1& \\

1057&&&&&&&&&\\
3172& 1586& 793& 2380& 1190& 595& 1786& 893& 2680& 1340\\
670& 335& 1006& 503& 1510& 755& 2266& 1133& 3400& 1700\\
850& 425& 1276& 638& 319& 958& 479& 1438& 719& 2158\\
1079& 3238& 1619& 4858& 2429& 7288& 3644& 1822& 911& 2734\\
1367& 4102& 2051& 6154& 3077& 9232& 4616& 2308& 1154& 577\\
1732& 866& 433& 1300& 650& 325& 976& 488& 244& 122\\
61& 184& 92& 46& 23& 70& 35& 106& 53& 160\\
80& 40& 20& 10& 5& 16& 8& 4& 2& 1\\

1058&&&&&&&&&\\
529& 1588& 794& 397& 1192& 596& 298& 149& 448& 224\\
112& 56& 28& 14& 7& 22& 11& 34& 17& 52\\
26& 13& 40& 20& 10& 5& 16& 8& 4& 2\\
1& \\

1059&&&&&&&&&\\
3178& 1589& 4768& 2384& 1192& 596& 298& 149& 448& 224\\
112& 56& 28& 14& 7& 22& 11& 34& 17& 52\\
26& 13& 40& 20& 10& 5& 16& 8& 4& 2\\
1& \\

1060&&&&&&&&&\\
530& 265& 796& 398& 199& 598& 299& 898& 449& 1348\\
674& 337& 1012& 506& 253& 760& 380& 190& 95& 286\\
143& 430& 215& 646& 323& 970& 485& 1456& 728& 364\\
182& 91& 274& 137& 412& 206& 103& 310& 155& 466\\
233& 700& 350& 175& 526& 263& 790& 395& 1186& 593\\
1780& 890& 445& 1336& 668& 334& 167& 502& 251& 754\\
377& 1132& 566& 283& 850& 425& 1276& 638& 319& 958\\
479& 1438& 719& 2158& 1079& 3238& 1619& 4858& 2429& 7288\\
3644& 1822& 911& 2734& 1367& 4102& 2051& 6154& 3077& 9232\\
4616& 2308& 1154& 577& 1732& 866& 433& 1300& 650& 325\\
976& 488& 244& 122& 61& 184& 92& 46& 23& 70\\
35& 106& 53& 160& 80& 40& 20& 10& 5& 16\\
8& 4& 2& 1& \\

1061&&&&&&&&&\\
3184& 1592& 796& 398& 199& 598& 299& 898& 449& 1348\\
674& 337& 1012& 506& 253& 760& 380& 190& 95& 286\\
143& 430& 215& 646& 323& 970& 485& 1456& 728& 364\\
182& 91& 274& 137& 412& 206& 103& 310& 155& 466\\
233& 700& 350& 175& 526& 263& 790& 395& 1186& 593\\
1780& 890& 445& 1336& 668& 334& 167& 502& 251& 754\\
377& 1132& 566& 283& 850& 425& 1276& 638& 319& 958\\
479& 1438& 719& 2158& 1079& 3238& 1619& 4858& 2429& 7288\\
3644& 1822& 911& 2734& 1367& 4102& 2051& 6154& 3077& 9232\\
4616& 2308& 1154& 577& 1732& 866& 433& 1300& 650& 325\\
976& 488& 244& 122& 61& 184& 92& 46& 23& 70\\
35& 106& 53& 160& 80& 40& 20& 10& 5& 16\\
8& 4& 2& 1& \\

1062&&&&&&&&&\\
531& 1594& 797& 2392& 1196& 598& 299& 898& 449& 1348\\
674& 337& 1012& 506& 253& 760& 380& 190& 95& 286\\
143& 430& 215& 646& 323& 970& 485& 1456& 728& 364\\
182& 91& 274& 137& 412& 206& 103& 310& 155& 466\\
233& 700& 350& 175& 526& 263& 790& 395& 1186& 593\\
1780& 890& 445& 1336& 668& 334& 167& 502& 251& 754\\
377& 1132& 566& 283& 850& 425& 1276& 638& 319& 958\\
479& 1438& 719& 2158& 1079& 3238& 1619& 4858& 2429& 7288\\
3644& 1822& 911& 2734& 1367& 4102& 2051& 6154& 3077& 9232\\
4616& 2308& 1154& 577& 1732& 866& 433& 1300& 650& 325\\
976& 488& 244& 122& 61& 184& 92& 46& 23& 70\\
35& 106& 53& 160& 80& 40& 20& 10& 5& 16\\
8& 4& 2& 1& \\

1063&&&&&&&&&\\
3190& 1595& 4786& 2393& 7180& 3590& 1795& 5386& 2693& 8080\\
4040& 2020& 1010& 505& 1516& 758& 379& 1138& 569& 1708\\
854& 427& 1282& 641& 1924& 962& 481& 1444& 722& 361\\
1084& 542& 271& 814& 407& 1222& 611& 1834& 917& 2752\\
1376& 688& 344& 172& 86& 43& 130& 65& 196& 98\\
49& 148& 74& 37& 112& 56& 28& 14& 7& 22\\
11& 34& 17& 52& 26& 13& 40& 20& 10& 5\\
16& 8& 4& 2& 1& \\

1064&&&&&&&&&\\
532& 266& 133& 400& 200& 100& 50& 25& 76& 38\\
19& 58& 29& 88& 44& 22& 11& 34& 17& 52\\
26& 13& 40& 20& 10& 5& 16& 8& 4& 2\\
1& \\

1065&&&&&&&&&\\
3196& 1598& 799& 2398& 1199& 3598& 1799& 5398& 2699& 8098\\
4049& 12148& 6074& 3037& 9112& 4556& 2278& 1139& 3418& 1709\\
5128& 2564& 1282& 641& 1924& 962& 481& 1444& 722& 361\\
1084& 542& 271& 814& 407& 1222& 611& 1834& 917& 2752\\
1376& 688& 344& 172& 86& 43& 130& 65& 196& 98\\
49& 148& 74& 37& 112& 56& 28& 14& 7& 22\\
11& 34& 17& 52& 26& 13& 40& 20& 10& 5\\
16& 8& 4& 2& 1& \\

1066&&&&&&&&&\\
533& 1600& 800& 400& 200& 100& 50& 25& 76& 38\\
19& 58& 29& 88& 44& 22& 11& 34& 17& 52\\
26& 13& 40& 20& 10& 5& 16& 8& 4& 2\\
1& \\

1067&&&&&&&&&\\
3202& 1601& 4804& 2402& 1201& 3604& 1802& 901& 2704& 1352\\
676& 338& 169& 508& 254& 127& 382& 191& 574& 287\\
862& 431& 1294& 647& 1942& 971& 2914& 1457& 4372& 2186\\
1093& 3280& 1640& 820& 410& 205& 616& 308& 154& 77\\
232& 116& 58& 29& 88& 44& 22& 11& 34& 17\\
52& 26& 13& 40& 20& 10& 5& 16& 8& 4\\
2& 1& \\

1068&&&&&&&&&\\
534& 267& 802& 401& 1204& 602& 301& 904& 452& 226\\
113& 340& 170& 85& 256& 128& 64& 32& 16& 8\\
4& 2& 1& \\

1069&&&&&&&&&\\
3208& 1604& 802& 401& 1204& 602& 301& 904& 452& 226\\
113& 340& 170& 85& 256& 128& 64& 32& 16& 8\\
4& 2& 1& \\

1070&&&&&&&&&\\
535& 1606& 803& 2410& 1205& 3616& 1808& 904& 452& 226\\
113& 340& 170& 85& 256& 128& 64& 32& 16& 8\\
4& 2& 1& \\

1071&&&&&&&&&\\
3214& 1607& 4822& 2411& 7234& 3617& 10852& 5426& 2713& 8140\\
4070& 2035& 6106& 3053& 9160& 4580& 2290& 1145& 3436& 1718\\
859& 2578& 1289& 3868& 1934& 967& 2902& 1451& 4354& 2177\\
6532& 3266& 1633& 4900& 2450& 1225& 3676& 1838& 919& 2758\\
1379& 4138& 2069& 6208& 3104& 1552& 776& 388& 194& 97\\
292& 146& 73& 220& 110& 55& 166& 83& 250& 125\\
376& 188& 94& 47& 142& 71& 214& 107& 322& 161\\
484& 242& 121& 364& 182& 91& 274& 137& 412& 206\\
103& 310& 155& 466& 233& 700& 350& 175& 526& 263\\
790& 395& 1186& 593& 1780& 890& 445& 1336& 668& 334\\
167& 502& 251& 754& 377& 1132& 566& 283& 850& 425\\
1276& 638& 319& 958& 479& 1438& 719& 2158& 1079& 3238\\
1619& 4858& 2429& 7288& 3644& 1822& 911& 2734& 1367& 4102\\
2051& 6154& 3077& 9232& 4616& 2308& 1154& 577& 1732& 866\\
433& 1300& 650& 325& 976& 488& 244& 122& 61& 184\\
92& 46& 23& 70& 35& 106& 53& 160& 80& 40\\
20& 10& 5& 16& 8& 4& 2& 1& \\

1072&&&&&&&&&\\
536& 268& 134& 67& 202& 101& 304& 152& 76& 38\\
19& 58& 29& 88& 44& 22& 11& 34& 17& 52\\
26& 13& 40& 20& 10& 5& 16& 8& 4& 2\\
1& \\

1073&&&&&&&&&\\
3220& 1610& 805& 2416& 1208& 604& 302& 151& 454& 227\\
682& 341& 1024& 512& 256& 128& 64& 32& 16& 8\\
4& 2& 1& \\

1074&&&&&&&&&\\
537& 1612& 806& 403& 1210& 605& 1816& 908& 454& 227\\
682& 341& 1024& 512& 256& 128& 64& 32& 16& 8\\
4& 2& 1& \\

1075&&&&&&&&&\\
3226& 1613& 4840& 2420& 1210& 605& 1816& 908& 454& 227\\
682& 341& 1024& 512& 256& 128& 64& 32& 16& 8\\
4& 2& 1& \\

1076&&&&&&&&&\\
538& 269& 808& 404& 202& 101& 304& 152& 76& 38\\
19& 58& 29& 88& 44& 22& 11& 34& 17& 52\\
26& 13& 40& 20& 10& 5& 16& 8& 4& 2\\
1& \\

1077&&&&&&&&&\\
3232& 1616& 808& 404& 202& 101& 304& 152& 76& 38\\
19& 58& 29& 88& 44& 22& 11& 34& 17& 52\\
26& 13& 40& 20& 10& 5& 16& 8& 4& 2\\
1& \\

1078&&&&&&&&&\\
539& 1618& 809& 2428& 1214& 607& 1822& 911& 2734& 1367\\
4102& 2051& 6154& 3077& 9232& 4616& 2308& 1154& 577& 1732\\
866& 433& 1300& 650& 325& 976& 488& 244& 122& 61\\
184& 92& 46& 23& 70& 35& 106& 53& 160& 80\\
40& 20& 10& 5& 16& 8& 4& 2& 1& \\

1079&&&&&&&&&\\
3238& 1619& 4858& 2429& 7288& 3644& 1822& 911& 2734& 1367\\
4102& 2051& 6154& 3077& 9232& 4616& 2308& 1154& 577& 1732\\
866& 433& 1300& 650& 325& 976& 488& 244& 122& 61\\
184& 92& 46& 23& 70& 35& 106& 53& 160& 80\\
40& 20& 10& 5& 16& 8& 4& 2& 1& \\

1080&&&&&&&&&\\
540& 270& 135& 406& 203& 610& 305& 916& 458& 229\\
688& 344& 172& 86& 43& 130& 65& 196& 98& 49\\
148& 74& 37& 112& 56& 28& 14& 7& 22& 11\\
34& 17& 52& 26& 13& 40& 20& 10& 5& 16\\
8& 4& 2& 1& \\

1081&&&&&&&&&\\
3244& 1622& 811& 2434& 1217& 3652& 1826& 913& 2740& 1370\\
685& 2056& 1028& 514& 257& 772& 386& 193& 580& 290\\
145& 436& 218& 109& 328& 164& 82& 41& 124& 62\\
31& 94& 47& 142& 71& 214& 107& 322& 161& 484\\
242& 121& 364& 182& 91& 274& 137& 412& 206& 103\\
310& 155& 466& 233& 700& 350& 175& 526& 263& 790\\
395& 1186& 593& 1780& 890& 445& 1336& 668& 334& 167\\
502& 251& 754& 377& 1132& 566& 283& 850& 425& 1276\\
638& 319& 958& 479& 1438& 719& 2158& 1079& 3238& 1619\\
4858& 2429& 7288& 3644& 1822& 911& 2734& 1367& 4102& 2051\\
6154& 3077& 9232& 4616& 2308& 1154& 577& 1732& 866& 433\\
1300& 650& 325& 976& 488& 244& 122& 61& 184& 92\\
46& 23& 70& 35& 106& 53& 160& 80& 40& 20\\
10& 5& 16& 8& 4& 2& 1& \\

1082&&&&&&&&&\\
541& 1624& 812& 406& 203& 610& 305& 916& 458& 229\\
688& 344& 172& 86& 43& 130& 65& 196& 98& 49\\
148& 74& 37& 112& 56& 28& 14& 7& 22& 11\\
34& 17& 52& 26& 13& 40& 20& 10& 5& 16\\
8& 4& 2& 1& \\

1083&&&&&&&&&\\
3250& 1625& 4876& 2438& 1219& 3658& 1829& 5488& 2744& 1372\\
686& 343& 1030& 515& 1546& 773& 2320& 1160& 580& 290\\
145& 436& 218& 109& 328& 164& 82& 41& 124& 62\\
31& 94& 47& 142& 71& 214& 107& 322& 161& 484\\
242& 121& 364& 182& 91& 274& 137& 412& 206& 103\\
310& 155& 466& 233& 700& 350& 175& 526& 263& 790\\
395& 1186& 593& 1780& 890& 445& 1336& 668& 334& 167\\
502& 251& 754& 377& 1132& 566& 283& 850& 425& 1276\\
638& 319& 958& 479& 1438& 719& 2158& 1079& 3238& 1619\\
4858& 2429& 7288& 3644& 1822& 911& 2734& 1367& 4102& 2051\\
6154& 3077& 9232& 4616& 2308& 1154& 577& 1732& 866& 433\\
1300& 650& 325& 976& 488& 244& 122& 61& 184& 92\\
46& 23& 70& 35& 106& 53& 160& 80& 40& 20\\
10& 5& 16& 8& 4& 2& 1& \\

1084&&&&&&&&&\\
542& 271& 814& 407& 1222& 611& 1834& 917& 2752& 1376\\
688& 344& 172& 86& 43& 130& 65& 196& 98& 49\\
148& 74& 37& 112& 56& 28& 14& 7& 22& 11\\
34& 17& 52& 26& 13& 40& 20& 10& 5& 16\\
8& 4& 2& 1& \\

1085&&&&&&&&&\\
3256& 1628& 814& 407& 1222& 611& 1834& 917& 2752& 1376\\
688& 344& 172& 86& 43& 130& 65& 196& 98& 49\\
148& 74& 37& 112& 56& 28& 14& 7& 22& 11\\
34& 17& 52& 26& 13& 40& 20& 10& 5& 16\\
8& 4& 2& 1& \\

1086&&&&&&&&&\\
543& 1630& 815& 2446& 1223& 3670& 1835& 5506& 2753& 8260\\
4130& 2065& 6196& 3098& 1549& 4648& 2324& 1162& 581& 1744\\
872& 436& 218& 109& 328& 164& 82& 41& 124& 62\\
31& 94& 47& 142& 71& 214& 107& 322& 161& 484\\
242& 121& 364& 182& 91& 274& 137& 412& 206& 103\\
310& 155& 466& 233& 700& 350& 175& 526& 263& 790\\
395& 1186& 593& 1780& 890& 445& 1336& 668& 334& 167\\
502& 251& 754& 377& 1132& 566& 283& 850& 425& 1276\\
638& 319& 958& 479& 1438& 719& 2158& 1079& 3238& 1619\\
4858& 2429& 7288& 3644& 1822& 911& 2734& 1367& 4102& 2051\\
6154& 3077& 9232& 4616& 2308& 1154& 577& 1732& 866& 433\\
1300& 650& 325& 976& 488& 244& 122& 61& 184& 92\\
46& 23& 70& 35& 106& 53& 160& 80& 40& 20\\
10& 5& 16& 8& 4& 2& 1& \\

1087&&&&&&&&&\\
3262& 1631& 4894& 2447& 7342& 3671& 11014& 5507& 16522& 8261\\
24784& 12392& 6196& 3098& 1549& 4648& 2324& 1162& 581& 1744\\
872& 436& 218& 109& 328& 164& 82& 41& 124& 62\\
31& 94& 47& 142& 71& 214& 107& 322& 161& 484\\
242& 121& 364& 182& 91& 274& 137& 412& 206& 103\\
310& 155& 466& 233& 700& 350& 175& 526& 263& 790\\
395& 1186& 593& 1780& 890& 445& 1336& 668& 334& 167\\
502& 251& 754& 377& 1132& 566& 283& 850& 425& 1276\\
638& 319& 958& 479& 1438& 719& 2158& 1079& 3238& 1619\\
4858& 2429& 7288& 3644& 1822& 911& 2734& 1367& 4102& 2051\\
6154& 3077& 9232& 4616& 2308& 1154& 577& 1732& 866& 433\\
1300& 650& 325& 976& 488& 244& 122& 61& 184& 92\\
46& 23& 70& 35& 106& 53& 160& 80& 40& 20\\
10& 5& 16& 8& 4& 2& 1& \\

1088&&&&&&&&&\\
544& 272& 136& 68& 34& 17& 52& 26& 13& 40\\
20& 10& 5& 16& 8& 4& 2& 1& \\

1089&&&&&&&&&\\
3268& 1634& 817& 2452& 1226& 613& 1840& 920& 460& 230\\
115& 346& 173& 520& 260& 130& 65& 196& 98& 49\\
148& 74& 37& 112& 56& 28& 14& 7& 22& 11\\
34& 17& 52& 26& 13& 40& 20& 10& 5& 16\\
8& 4& 2& 1& \\

1090&&&&&&&&&\\
545& 1636& 818& 409& 1228& 614& 307& 922& 461& 1384\\
692& 346& 173& 520& 260& 130& 65& 196& 98& 49\\
148& 74& 37& 112& 56& 28& 14& 7& 22& 11\\
34& 17& 52& 26& 13& 40& 20& 10& 5& 16\\
8& 4& 2& 1& \\

1091&&&&&&&&&\\
3274& 1637& 4912& 2456& 1228& 614& 307& 922& 461& 1384\\
692& 346& 173& 520& 260& 130& 65& 196& 98& 49\\
148& 74& 37& 112& 56& 28& 14& 7& 22& 11\\
34& 17& 52& 26& 13& 40& 20& 10& 5& 16\\
8& 4& 2& 1& \\

1092&&&&&&&&&\\
546& 273& 820& 410& 205& 616& 308& 154& 77& 232\\
116& 58& 29& 88& 44& 22& 11& 34& 17& 52\\
26& 13& 40& 20& 10& 5& 16& 8& 4& 2\\
1& \\

1093&&&&&&&&&\\
3280& 1640& 820& 410& 205& 616& 308& 154& 77& 232\\
116& 58& 29& 88& 44& 22& 11& 34& 17& 52\\
26& 13& 40& 20& 10& 5& 16& 8& 4& 2\\
1& \\

1094&&&&&&&&&\\
547& 1642& 821& 2464& 1232& 616& 308& 154& 77& 232\\
116& 58& 29& 88& 44& 22& 11& 34& 17& 52\\
26& 13& 40& 20& 10& 5& 16& 8& 4& 2\\
1& \\

1095&&&&&&&&&\\
3286& 1643& 4930& 2465& 7396& 3698& 1849& 5548& 2774& 1387\\
4162& 2081& 6244& 3122& 1561& 4684& 2342& 1171& 3514& 1757\\
5272& 2636& 1318& 659& 1978& 989& 2968& 1484& 742& 371\\
1114& 557& 1672& 836& 418& 209& 628& 314& 157& 472\\
236& 118& 59& 178& 89& 268& 134& 67& 202& 101\\
304& 152& 76& 38& 19& 58& 29& 88& 44& 22\\
11& 34& 17& 52& 26& 13& 40& 20& 10& 5\\
16& 8& 4& 2& 1& \\

1096&&&&&&&&&\\
548& 274& 137& 412& 206& 103& 310& 155& 466& 233\\
700& 350& 175& 526& 263& 790& 395& 1186& 593& 1780\\
890& 445& 1336& 668& 334& 167& 502& 251& 754& 377\\
1132& 566& 283& 850& 425& 1276& 638& 319& 958& 479\\
1438& 719& 2158& 1079& 3238& 1619& 4858& 2429& 7288& 3644\\
1822& 911& 2734& 1367& 4102& 2051& 6154& 3077& 9232& 4616\\
2308& 1154& 577& 1732& 866& 433& 1300& 650& 325& 976\\
488& 244& 122& 61& 184& 92& 46& 23& 70& 35\\
106& 53& 160& 80& 40& 20& 10& 5& 16& 8\\
4& 2& 1& \\

1097&&&&&&&&&\\
3292& 1646& 823& 2470& 1235& 3706& 1853& 5560& 2780& 1390\\
695& 2086& 1043& 3130& 1565& 4696& 2348& 1174& 587& 1762\\
881& 2644& 1322& 661& 1984& 992& 496& 248& 124& 62\\
31& 94& 47& 142& 71& 214& 107& 322& 161& 484\\
242& 121& 364& 182& 91& 274& 137& 412& 206& 103\\
310& 155& 466& 233& 700& 350& 175& 526& 263& 790\\
395& 1186& 593& 1780& 890& 445& 1336& 668& 334& 167\\
502& 251& 754& 377& 1132& 566& 283& 850& 425& 1276\\
638& 319& 958& 479& 1438& 719& 2158& 1079& 3238& 1619\\
4858& 2429& 7288& 3644& 1822& 911& 2734& 1367& 4102& 2051\\
6154& 3077& 9232& 4616& 2308& 1154& 577& 1732& 866& 433\\
1300& 650& 325& 976& 488& 244& 122& 61& 184& 92\\
46& 23& 70& 35& 106& 53& 160& 80& 40& 20\\
10& 5& 16& 8& 4& 2& 1& \\

1098&&&&&&&&&\\
549& 1648& 824& 412& 206& 103& 310& 155& 466& 233\\
700& 350& 175& 526& 263& 790& 395& 1186& 593& 1780\\
890& 445& 1336& 668& 334& 167& 502& 251& 754& 377\\
1132& 566& 283& 850& 425& 1276& 638& 319& 958& 479\\
1438& 719& 2158& 1079& 3238& 1619& 4858& 2429& 7288& 3644\\
1822& 911& 2734& 1367& 4102& 2051& 6154& 3077& 9232& 4616\\
2308& 1154& 577& 1732& 866& 433& 1300& 650& 325& 976\\
488& 244& 122& 61& 184& 92& 46& 23& 70& 35\\
106& 53& 160& 80& 40& 20& 10& 5& 16& 8\\
4& 2& 1& \\

1099&&&&&&&&&\\
3298& 1649& 4948& 2474& 1237& 3712& 1856& 928& 464& 232\\
116& 58& 29& 88& 44& 22& 11& 34& 17& 52\\
26& 13& 40& 20& 10& 5& 16& 8& 4& 2\\
1& \\

1100&&&&&&&&&\\
550& 275& 826& 413& 1240& 620& 310& 155& 466& 233\\
700& 350& 175& 526& 263& 790& 395& 1186& 593& 1780\\
890& 445& 1336& 668& 334& 167& 502& 251& 754& 377\\
1132& 566& 283& 850& 425& 1276& 638& 319& 958& 479\\
1438& 719& 2158& 1079& 3238& 1619& 4858& 2429& 7288& 3644\\
1822& 911& 2734& 1367& 4102& 2051& 6154& 3077& 9232& 4616\\
2308& 1154& 577& 1732& 866& 433& 1300& 650& 325& 976\\
488& 244& 122& 61& 184& 92& 46& 23& 70& 35\\
106& 53& 160& 80& 40& 20& 10& 5& 16& 8\\
4& 2& 1& \\

1101&&&&&&&&&\\
3304& 1652& 826& 413& 1240& 620& 310& 155& 466& 233\\
700& 350& 175& 526& 263& 790& 395& 1186& 593& 1780\\
890& 445& 1336& 668& 334& 167& 502& 251& 754& 377\\
1132& 566& 283& 850& 425& 1276& 638& 319& 958& 479\\
1438& 719& 2158& 1079& 3238& 1619& 4858& 2429& 7288& 3644\\
1822& 911& 2734& 1367& 4102& 2051& 6154& 3077& 9232& 4616\\
2308& 1154& 577& 1732& 866& 433& 1300& 650& 325& 976\\
488& 244& 122& 61& 184& 92& 46& 23& 70& 35\\
106& 53& 160& 80& 40& 20& 10& 5& 16& 8\\
4& 2& 1& \\

1102&&&&&&&&&\\
551& 1654& 827& 2482& 1241& 3724& 1862& 931& 2794& 1397\\
4192& 2096& 1048& 524& 262& 131& 394& 197& 592& 296\\
148& 74& 37& 112& 56& 28& 14& 7& 22& 11\\
34& 17& 52& 26& 13& 40& 20& 10& 5& 16\\
8& 4& 2& 1& \\

1103&&&&&&&&&\\
3310& 1655& 4966& 2483& 7450& 3725& 11176& 5588& 2794& 1397\\
4192& 2096& 1048& 524& 262& 131& 394& 197& 592& 296\\
148& 74& 37& 112& 56& 28& 14& 7& 22& 11\\
34& 17& 52& 26& 13& 40& 20& 10& 5& 16\\
8& 4& 2& 1& \\

1104&&&&&&&&&\\
552& 276& 138& 69& 208& 104& 52& 26& 13& 40\\
20& 10& 5& 16& 8& 4& 2& 1& \\

1105&&&&&&&&&\\
3316& 1658& 829& 2488& 1244& 622& 311& 934& 467& 1402\\
701& 2104& 1052& 526& 263& 790& 395& 1186& 593& 1780\\
890& 445& 1336& 668& 334& 167& 502& 251& 754& 377\\
1132& 566& 283& 850& 425& 1276& 638& 319& 958& 479\\
1438& 719& 2158& 1079& 3238& 1619& 4858& 2429& 7288& 3644\\
1822& 911& 2734& 1367& 4102& 2051& 6154& 3077& 9232& 4616\\
2308& 1154& 577& 1732& 866& 433& 1300& 650& 325& 976\\
488& 244& 122& 61& 184& 92& 46& 23& 70& 35\\
106& 53& 160& 80& 40& 20& 10& 5& 16& 8\\
4& 2& 1& \\

1106&&&&&&&&&\\
553& 1660& 830& 415& 1246& 623& 1870& 935& 2806& 1403\\
4210& 2105& 6316& 3158& 1579& 4738& 2369& 7108& 3554& 1777\\
5332& 2666& 1333& 4000& 2000& 1000& 500& 250& 125& 376\\
188& 94& 47& 142& 71& 214& 107& 322& 161& 484\\
242& 121& 364& 182& 91& 274& 137& 412& 206& 103\\
310& 155& 466& 233& 700& 350& 175& 526& 263& 790\\
395& 1186& 593& 1780& 890& 445& 1336& 668& 334& 167\\
502& 251& 754& 377& 1132& 566& 283& 850& 425& 1276\\
638& 319& 958& 479& 1438& 719& 2158& 1079& 3238& 1619\\
4858& 2429& 7288& 3644& 1822& 911& 2734& 1367& 4102& 2051\\
6154& 3077& 9232& 4616& 2308& 1154& 577& 1732& 866& 433\\
1300& 650& 325& 976& 488& 244& 122& 61& 184& 92\\
46& 23& 70& 35& 106& 53& 160& 80& 40& 20\\
10& 5& 16& 8& 4& 2& 1& \\

1107&&&&&&&&&\\
3322& 1661& 4984& 2492& 1246& 623& 1870& 935& 2806& 1403\\
4210& 2105& 6316& 3158& 1579& 4738& 2369& 7108& 3554& 1777\\
5332& 2666& 1333& 4000& 2000& 1000& 500& 250& 125& 376\\
188& 94& 47& 142& 71& 214& 107& 322& 161& 484\\
242& 121& 364& 182& 91& 274& 137& 412& 206& 103\\
310& 155& 466& 233& 700& 350& 175& 526& 263& 790\\
395& 1186& 593& 1780& 890& 445& 1336& 668& 334& 167\\
502& 251& 754& 377& 1132& 566& 283& 850& 425& 1276\\
638& 319& 958& 479& 1438& 719& 2158& 1079& 3238& 1619\\
4858& 2429& 7288& 3644& 1822& 911& 2734& 1367& 4102& 2051\\
6154& 3077& 9232& 4616& 2308& 1154& 577& 1732& 866& 433\\
1300& 650& 325& 976& 488& 244& 122& 61& 184& 92\\
46& 23& 70& 35& 106& 53& 160& 80& 40& 20\\
10& 5& 16& 8& 4& 2& 1& \\

1108&&&&&&&&&\\
554& 277& 832& 416& 208& 104& 52& 26& 13& 40\\
20& 10& 5& 16& 8& 4& 2& 1& \\

1109&&&&&&&&&\\
3328& 1664& 832& 416& 208& 104& 52& 26& 13& 40\\
20& 10& 5& 16& 8& 4& 2& 1& \\

1110&&&&&&&&&\\
555& 1666& 833& 2500& 1250& 625& 1876& 938& 469& 1408\\
704& 352& 176& 88& 44& 22& 11& 34& 17& 52\\
26& 13& 40& 20& 10& 5& 16& 8& 4& 2\\
1& \\

1111&&&&&&&&&\\
3334& 1667& 5002& 2501& 7504& 3752& 1876& 938& 469& 1408\\
704& 352& 176& 88& 44& 22& 11& 34& 17& 52\\
26& 13& 40& 20& 10& 5& 16& 8& 4& 2\\
1& \\

1112&&&&&&&&&\\
556& 278& 139& 418& 209& 628& 314& 157& 472& 236\\
118& 59& 178& 89& 268& 134& 67& 202& 101& 304\\
152& 76& 38& 19& 58& 29& 88& 44& 22& 11\\
34& 17& 52& 26& 13& 40& 20& 10& 5& 16\\
8& 4& 2& 1& \\

1113&&&&&&&&&\\
3340& 1670& 835& 2506& 1253& 3760& 1880& 940& 470& 235\\
706& 353& 1060& 530& 265& 796& 398& 199& 598& 299\\
898& 449& 1348& 674& 337& 1012& 506& 253& 760& 380\\
190& 95& 286& 143& 430& 215& 646& 323& 970& 485\\
1456& 728& 364& 182& 91& 274& 137& 412& 206& 103\\
310& 155& 466& 233& 700& 350& 175& 526& 263& 790\\
395& 1186& 593& 1780& 890& 445& 1336& 668& 334& 167\\
502& 251& 754& 377& 1132& 566& 283& 850& 425& 1276\\
638& 319& 958& 479& 1438& 719& 2158& 1079& 3238& 1619\\
4858& 2429& 7288& 3644& 1822& 911& 2734& 1367& 4102& 2051\\
6154& 3077& 9232& 4616& 2308& 1154& 577& 1732& 866& 433\\
1300& 650& 325& 976& 488& 244& 122& 61& 184& 92\\
46& 23& 70& 35& 106& 53& 160& 80& 40& 20\\
10& 5& 16& 8& 4& 2& 1& \\

1114&&&&&&&&&\\
557& 1672& 836& 418& 209& 628& 314& 157& 472& 236\\
118& 59& 178& 89& 268& 134& 67& 202& 101& 304\\
152& 76& 38& 19& 58& 29& 88& 44& 22& 11\\
34& 17& 52& 26& 13& 40& 20& 10& 5& 16\\
8& 4& 2& 1& \\

1115&&&&&&&&&\\
3346& 1673& 5020& 2510& 1255& 3766& 1883& 5650& 2825& 8476\\
4238& 2119& 6358& 3179& 9538& 4769& 14308& 7154& 3577& 10732\\
5366& 2683& 8050& 4025& 12076& 6038& 3019& 9058& 4529& 13588\\
6794& 3397& 10192& 5096& 2548& 1274& 637& 1912& 956& 478\\
239& 718& 359& 1078& 539& 1618& 809& 2428& 1214& 607\\
1822& 911& 2734& 1367& 4102& 2051& 6154& 3077& 9232& 4616\\
2308& 1154& 577& 1732& 866& 433& 1300& 650& 325& 976\\
488& 244& 122& 61& 184& 92& 46& 23& 70& 35\\
106& 53& 160& 80& 40& 20& 10& 5& 16& 8\\
4& 2& 1& \\

1116&&&&&&&&&\\
558& 279& 838& 419& 1258& 629& 1888& 944& 472& 236\\
118& 59& 178& 89& 268& 134& 67& 202& 101& 304\\
152& 76& 38& 19& 58& 29& 88& 44& 22& 11\\
34& 17& 52& 26& 13& 40& 20& 10& 5& 16\\
8& 4& 2& 1& \\

1117&&&&&&&&&\\
3352& 1676& 838& 419& 1258& 629& 1888& 944& 472& 236\\
118& 59& 178& 89& 268& 134& 67& 202& 101& 304\\
152& 76& 38& 19& 58& 29& 88& 44& 22& 11\\
34& 17& 52& 26& 13& 40& 20& 10& 5& 16\\
8& 4& 2& 1& \\

1118&&&&&&&&&\\
559& 1678& 839& 2518& 1259& 3778& 1889& 5668& 2834& 1417\\
4252& 2126& 1063& 3190& 1595& 4786& 2393& 7180& 3590& 1795\\
5386& 2693& 8080& 4040& 2020& 1010& 505& 1516& 758& 379\\
1138& 569& 1708& 854& 427& 1282& 641& 1924& 962& 481\\
1444& 722& 361& 1084& 542& 271& 814& 407& 1222& 611\\
1834& 917& 2752& 1376& 688& 344& 172& 86& 43& 130\\
65& 196& 98& 49& 148& 74& 37& 112& 56& 28\\
14& 7& 22& 11& 34& 17& 52& 26& 13& 40\\
20& 10& 5& 16& 8& 4& 2& 1& \\

1119&&&&&&&&&\\
3358& 1679& 5038& 2519& 7558& 3779& 11338& 5669& 17008& 8504\\
4252& 2126& 1063& 3190& 1595& 4786& 2393& 7180& 3590& 1795\\
5386& 2693& 8080& 4040& 2020& 1010& 505& 1516& 758& 379\\
1138& 569& 1708& 854& 427& 1282& 641& 1924& 962& 481\\
1444& 722& 361& 1084& 542& 271& 814& 407& 1222& 611\\
1834& 917& 2752& 1376& 688& 344& 172& 86& 43& 130\\
65& 196& 98& 49& 148& 74& 37& 112& 56& 28\\
14& 7& 22& 11& 34& 17& 52& 26& 13& 40\\
20& 10& 5& 16& 8& 4& 2& 1& \\

1120&&&&&&&&&\\
560& 280& 140& 70& 35& 106& 53& 160& 80& 40\\
20& 10& 5& 16& 8& 4& 2& 1& \\

1121&&&&&&&&&\\
3364& 1682& 841& 2524& 1262& 631& 1894& 947& 2842& 1421\\
4264& 2132& 1066& 533& 1600& 800& 400& 200& 100& 50\\
25& 76& 38& 19& 58& 29& 88& 44& 22& 11\\
34& 17& 52& 26& 13& 40& 20& 10& 5& 16\\
8& 4& 2& 1& \\

1122&&&&&&&&&\\
561& 1684& 842& 421& 1264& 632& 316& 158& 79& 238\\
119& 358& 179& 538& 269& 808& 404& 202& 101& 304\\
152& 76& 38& 19& 58& 29& 88& 44& 22& 11\\
34& 17& 52& 26& 13& 40& 20& 10& 5& 16\\
8& 4& 2& 1& \\

1123&&&&&&&&&\\
3370& 1685& 5056& 2528& 1264& 632& 316& 158& 79& 238\\
119& 358& 179& 538& 269& 808& 404& 202& 101& 304\\
152& 76& 38& 19& 58& 29& 88& 44& 22& 11\\
34& 17& 52& 26& 13& 40& 20& 10& 5& 16\\
8& 4& 2& 1& \\

1124&&&&&&&&&\\
562& 281& 844& 422& 211& 634& 317& 952& 476& 238\\
119& 358& 179& 538& 269& 808& 404& 202& 101& 304\\
152& 76& 38& 19& 58& 29& 88& 44& 22& 11\\
34& 17& 52& 26& 13& 40& 20& 10& 5& 16\\
8& 4& 2& 1& \\

1125&&&&&&&&&\\
3376& 1688& 844& 422& 211& 634& 317& 952& 476& 238\\
119& 358& 179& 538& 269& 808& 404& 202& 101& 304\\
152& 76& 38& 19& 58& 29& 88& 44& 22& 11\\
34& 17& 52& 26& 13& 40& 20& 10& 5& 16\\
8& 4& 2& 1& \\

1126&&&&&&&&&\\
563& 1690& 845& 2536& 1268& 634& 317& 952& 476& 238\\
119& 358& 179& 538& 269& 808& 404& 202& 101& 304\\
152& 76& 38& 19& 58& 29& 88& 44& 22& 11\\
34& 17& 52& 26& 13& 40& 20& 10& 5& 16\\
8& 4& 2& 1& \\

1127&&&&&&&&&\\
3382& 1691& 5074& 2537& 7612& 3806& 1903& 5710& 2855& 8566\\
4283& 12850& 6425& 19276& 9638& 4819& 14458& 7229& 21688& 10844\\
5422& 2711& 8134& 4067& 12202& 6101& 18304& 9152& 4576& 2288\\
1144& 572& 286& 143& 430& 215& 646& 323& 970& 485\\
1456& 728& 364& 182& 91& 274& 137& 412& 206& 103\\
310& 155& 466& 233& 700& 350& 175& 526& 263& 790\\
395& 1186& 593& 1780& 890& 445& 1336& 668& 334& 167\\
502& 251& 754& 377& 1132& 566& 283& 850& 425& 1276\\
638& 319& 958& 479& 1438& 719& 2158& 1079& 3238& 1619\\
4858& 2429& 7288& 3644& 1822& 911& 2734& 1367& 4102& 2051\\
6154& 3077& 9232& 4616& 2308& 1154& 577& 1732& 866& 433\\
1300& 650& 325& 976& 488& 244& 122& 61& 184& 92\\
46& 23& 70& 35& 106& 53& 160& 80& 40& 20\\
10& 5& 16& 8& 4& 2& 1& \\

1128&&&&&&&&&\\
564& 282& 141& 424& 212& 106& 53& 160& 80& 40\\
20& 10& 5& 16& 8& 4& 2& 1& \\

1129&&&&&&&&&\\
3388& 1694& 847& 2542& 1271& 3814& 1907& 5722& 2861& 8584\\
4292& 2146& 1073& 3220& 1610& 805& 2416& 1208& 604& 302\\
151& 454& 227& 682& 341& 1024& 512& 256& 128& 64\\
32& 16& 8& 4& 2& 1& \\

1130&&&&&&&&&\\
565& 1696& 848& 424& 212& 106& 53& 160& 80& 40\\
20& 10& 5& 16& 8& 4& 2& 1& \\

1131&&&&&&&&&\\
3394& 1697& 5092& 2546& 1273& 3820& 1910& 955& 2866& 1433\\
4300& 2150& 1075& 3226& 1613& 4840& 2420& 1210& 605& 1816\\
908& 454& 227& 682& 341& 1024& 512& 256& 128& 64\\
32& 16& 8& 4& 2& 1& \\

1132&&&&&&&&&\\
566& 283& 850& 425& 1276& 638& 319& 958& 479& 1438\\
719& 2158& 1079& 3238& 1619& 4858& 2429& 7288& 3644& 1822\\
911& 2734& 1367& 4102& 2051& 6154& 3077& 9232& 4616& 2308\\
1154& 577& 1732& 866& 433& 1300& 650& 325& 976& 488\\
244& 122& 61& 184& 92& 46& 23& 70& 35& 106\\
53& 160& 80& 40& 20& 10& 5& 16& 8& 4\\
2& 1& \\

1133&&&&&&&&&\\
3400& 1700& 850& 425& 1276& 638& 319& 958& 479& 1438\\
719& 2158& 1079& 3238& 1619& 4858& 2429& 7288& 3644& 1822\\
911& 2734& 1367& 4102& 2051& 6154& 3077& 9232& 4616& 2308\\
1154& 577& 1732& 866& 433& 1300& 650& 325& 976& 488\\
244& 122& 61& 184& 92& 46& 23& 70& 35& 106\\
53& 160& 80& 40& 20& 10& 5& 16& 8& 4\\
2& 1& \\

1134&&&&&&&&&\\
567& 1702& 851& 2554& 1277& 3832& 1916& 958& 479& 1438\\
719& 2158& 1079& 3238& 1619& 4858& 2429& 7288& 3644& 1822\\
911& 2734& 1367& 4102& 2051& 6154& 3077& 9232& 4616& 2308\\
1154& 577& 1732& 866& 433& 1300& 650& 325& 976& 488\\
244& 122& 61& 184& 92& 46& 23& 70& 35& 106\\
53& 160& 80& 40& 20& 10& 5& 16& 8& 4\\
2& 1& \\

1135&&&&&&&&&\\
3406& 1703& 5110& 2555& 7666& 3833& 11500& 5750& 2875& 8626\\
4313& 12940& 6470& 3235& 9706& 4853& 14560& 7280& 3640& 1820\\
910& 455& 1366& 683& 2050& 1025& 3076& 1538& 769& 2308\\
1154& 577& 1732& 866& 433& 1300& 650& 325& 976& 488\\
244& 122& 61& 184& 92& 46& 23& 70& 35& 106\\
53& 160& 80& 40& 20& 10& 5& 16& 8& 4\\
2& 1& \\

1136&&&&&&&&&\\
568& 284& 142& 71& 214& 107& 322& 161& 484& 242\\
121& 364& 182& 91& 274& 137& 412& 206& 103& 310\\
155& 466& 233& 700& 350& 175& 526& 263& 790& 395\\
1186& 593& 1780& 890& 445& 1336& 668& 334& 167& 502\\
251& 754& 377& 1132& 566& 283& 850& 425& 1276& 638\\
319& 958& 479& 1438& 719& 2158& 1079& 3238& 1619& 4858\\
2429& 7288& 3644& 1822& 911& 2734& 1367& 4102& 2051& 6154\\
3077& 9232& 4616& 2308& 1154& 577& 1732& 866& 433& 1300\\
650& 325& 976& 488& 244& 122& 61& 184& 92& 46\\
23& 70& 35& 106& 53& 160& 80& 40& 20& 10\\
5& 16& 8& 4& 2& 1& \\

1137&&&&&&&&&\\
3412& 1706& 853& 2560& 1280& 640& 320& 160& 80& 40\\
20& 10& 5& 16& 8& 4& 2& 1& \\

1138&&&&&&&&&\\
569& 1708& 854& 427& 1282& 641& 1924& 962& 481& 1444\\
722& 361& 1084& 542& 271& 814& 407& 1222& 611& 1834\\
917& 2752& 1376& 688& 344& 172& 86& 43& 130& 65\\
196& 98& 49& 148& 74& 37& 112& 56& 28& 14\\
7& 22& 11& 34& 17& 52& 26& 13& 40& 20\\
10& 5& 16& 8& 4& 2& 1& \\

1139&&&&&&&&&\\
3418& 1709& 5128& 2564& 1282& 641& 1924& 962& 481& 1444\\
722& 361& 1084& 542& 271& 814& 407& 1222& 611& 1834\\
917& 2752& 1376& 688& 344& 172& 86& 43& 130& 65\\
196& 98& 49& 148& 74& 37& 112& 56& 28& 14\\
7& 22& 11& 34& 17& 52& 26& 13& 40& 20\\
10& 5& 16& 8& 4& 2& 1& \\

1140&&&&&&&&&\\
570& 285& 856& 428& 214& 107& 322& 161& 484& 242\\
121& 364& 182& 91& 274& 137& 412& 206& 103& 310\\
155& 466& 233& 700& 350& 175& 526& 263& 790& 395\\
1186& 593& 1780& 890& 445& 1336& 668& 334& 167& 502\\
251& 754& 377& 1132& 566& 283& 850& 425& 1276& 638\\
319& 958& 479& 1438& 719& 2158& 1079& 3238& 1619& 4858\\
2429& 7288& 3644& 1822& 911& 2734& 1367& 4102& 2051& 6154\\
3077& 9232& 4616& 2308& 1154& 577& 1732& 866& 433& 1300\\
650& 325& 976& 488& 244& 122& 61& 184& 92& 46\\
23& 70& 35& 106& 53& 160& 80& 40& 20& 10\\
5& 16& 8& 4& 2& 1& \\

1141&&&&&&&&&\\
3424& 1712& 856& 428& 214& 107& 322& 161& 484& 242\\
121& 364& 182& 91& 274& 137& 412& 206& 103& 310\\
155& 466& 233& 700& 350& 175& 526& 263& 790& 395\\
1186& 593& 1780& 890& 445& 1336& 668& 334& 167& 502\\
251& 754& 377& 1132& 566& 283& 850& 425& 1276& 638\\
319& 958& 479& 1438& 719& 2158& 1079& 3238& 1619& 4858\\
2429& 7288& 3644& 1822& 911& 2734& 1367& 4102& 2051& 6154\\
3077& 9232& 4616& 2308& 1154& 577& 1732& 866& 433& 1300\\
650& 325& 976& 488& 244& 122& 61& 184& 92& 46\\
23& 70& 35& 106& 53& 160& 80& 40& 20& 10\\
5& 16& 8& 4& 2& 1& \\

1142&&&&&&&&&\\
571& 1714& 857& 2572& 1286& 643& 1930& 965& 2896& 1448\\
724& 362& 181& 544& 272& 136& 68& 34& 17& 52\\
26& 13& 40& 20& 10& 5& 16& 8& 4& 2\\
1& \\

1143&&&&&&&&&\\
3430& 1715& 5146& 2573& 7720& 3860& 1930& 965& 2896& 1448\\
724& 362& 181& 544& 272& 136& 68& 34& 17& 52\\
26& 13& 40& 20& 10& 5& 16& 8& 4& 2\\
1& \\

1144&&&&&&&&&\\
572& 286& 143& 430& 215& 646& 323& 970& 485& 1456\\
728& 364& 182& 91& 274& 137& 412& 206& 103& 310\\
155& 466& 233& 700& 350& 175& 526& 263& 790& 395\\
1186& 593& 1780& 890& 445& 1336& 668& 334& 167& 502\\
251& 754& 377& 1132& 566& 283& 850& 425& 1276& 638\\
319& 958& 479& 1438& 719& 2158& 1079& 3238& 1619& 4858\\
2429& 7288& 3644& 1822& 911& 2734& 1367& 4102& 2051& 6154\\
3077& 9232& 4616& 2308& 1154& 577& 1732& 866& 433& 1300\\
650& 325& 976& 488& 244& 122& 61& 184& 92& 46\\
23& 70& 35& 106& 53& 160& 80& 40& 20& 10\\
5& 16& 8& 4& 2& 1& \\

1145&&&&&&&&&\\
3436& 1718& 859& 2578& 1289& 3868& 1934& 967& 2902& 1451\\
4354& 2177& 6532& 3266& 1633& 4900& 2450& 1225& 3676& 1838\\
919& 2758& 1379& 4138& 2069& 6208& 3104& 1552& 776& 388\\
194& 97& 292& 146& 73& 220& 110& 55& 166& 83\\
250& 125& 376& 188& 94& 47& 142& 71& 214& 107\\
322& 161& 484& 242& 121& 364& 182& 91& 274& 137\\
412& 206& 103& 310& 155& 466& 233& 700& 350& 175\\
526& 263& 790& 395& 1186& 593& 1780& 890& 445& 1336\\
668& 334& 167& 502& 251& 754& 377& 1132& 566& 283\\
850& 425& 1276& 638& 319& 958& 479& 1438& 719& 2158\\
1079& 3238& 1619& 4858& 2429& 7288& 3644& 1822& 911& 2734\\
1367& 4102& 2051& 6154& 3077& 9232& 4616& 2308& 1154& 577\\
1732& 866& 433& 1300& 650& 325& 976& 488& 244& 122\\
61& 184& 92& 46& 23& 70& 35& 106& 53& 160\\
80& 40& 20& 10& 5& 16& 8& 4& 2& 1\\

1146&&&&&&&&&\\
573& 1720& 860& 430& 215& 646& 323& 970& 485& 1456\\
728& 364& 182& 91& 274& 137& 412& 206& 103& 310\\
155& 466& 233& 700& 350& 175& 526& 263& 790& 395\\
1186& 593& 1780& 890& 445& 1336& 668& 334& 167& 502\\
251& 754& 377& 1132& 566& 283& 850& 425& 1276& 638\\
319& 958& 479& 1438& 719& 2158& 1079& 3238& 1619& 4858\\
2429& 7288& 3644& 1822& 911& 2734& 1367& 4102& 2051& 6154\\
3077& 9232& 4616& 2308& 1154& 577& 1732& 866& 433& 1300\\
650& 325& 976& 488& 244& 122& 61& 184& 92& 46\\
23& 70& 35& 106& 53& 160& 80& 40& 20& 10\\
5& 16& 8& 4& 2& 1& \\

1147&&&&&&&&&\\
3442& 1721& 5164& 2582& 1291& 3874& 1937& 5812& 2906& 1453\\
4360& 2180& 1090& 545& 1636& 818& 409& 1228& 614& 307\\
922& 461& 1384& 692& 346& 173& 520& 260& 130& 65\\
196& 98& 49& 148& 74& 37& 112& 56& 28& 14\\
7& 22& 11& 34& 17& 52& 26& 13& 40& 20\\
10& 5& 16& 8& 4& 2& 1& \\

1148&&&&&&&&&\\
574& 287& 862& 431& 1294& 647& 1942& 971& 2914& 1457\\
4372& 2186& 1093& 3280& 1640& 820& 410& 205& 616& 308\\
154& 77& 232& 116& 58& 29& 88& 44& 22& 11\\
34& 17& 52& 26& 13& 40& 20& 10& 5& 16\\
8& 4& 2& 1& \\

1149&&&&&&&&&\\
3448& 1724& 862& 431& 1294& 647& 1942& 971& 2914& 1457\\
4372& 2186& 1093& 3280& 1640& 820& 410& 205& 616& 308\\
154& 77& 232& 116& 58& 29& 88& 44& 22& 11\\
34& 17& 52& 26& 13& 40& 20& 10& 5& 16\\
8& 4& 2& 1& \\

1150&&&&&&&&&\\
575& 1726& 863& 2590& 1295& 3886& 1943& 5830& 2915& 8746\\
4373& 13120& 6560& 3280& 1640& 820& 410& 205& 616& 308\\
154& 77& 232& 116& 58& 29& 88& 44& 22& 11\\
34& 17& 52& 26& 13& 40& 20& 10& 5& 16\\
8& 4& 2& 1& \\

1151&&&&&&&&&\\
3454& 1727& 5182& 2591& 7774& 3887& 11662& 5831& 17494& 8747\\
26242& 13121& 39364& 19682& 9841& 29524& 14762& 7381& 22144& 11072\\
5536& 2768& 1384& 692& 346& 173& 520& 260& 130& 65\\
196& 98& 49& 148& 74& 37& 112& 56& 28& 14\\
7& 22& 11& 34& 17& 52& 26& 13& 40& 20\\
10& 5& 16& 8& 4& 2& 1& \\

1152&&&&&&&&&\\
576& 288& 144& 72& 36& 18& 9& 28& 14& 7\\
22& 11& 34& 17& 52& 26& 13& 40& 20& 10\\
5& 16& 8& 4& 2& 1& \\

1153&&&&&&&&&\\
3460& 1730& 865& 2596& 1298& 649& 1948& 974& 487& 1462\\
731& 2194& 1097& 3292& 1646& 823& 2470& 1235& 3706& 1853\\
5560& 2780& 1390& 695& 2086& 1043& 3130& 1565& 4696& 2348\\
1174& 587& 1762& 881& 2644& 1322& 661& 1984& 992& 496\\
248& 124& 62& 31& 94& 47& 142& 71& 214& 107\\
322& 161& 484& 242& 121& 364& 182& 91& 274& 137\\
412& 206& 103& 310& 155& 466& 233& 700& 350& 175\\
526& 263& 790& 395& 1186& 593& 1780& 890& 445& 1336\\
668& 334& 167& 502& 251& 754& 377& 1132& 566& 283\\
850& 425& 1276& 638& 319& 958& 479& 1438& 719& 2158\\
1079& 3238& 1619& 4858& 2429& 7288& 3644& 1822& 911& 2734\\
1367& 4102& 2051& 6154& 3077& 9232& 4616& 2308& 1154& 577\\
1732& 866& 433& 1300& 650& 325& 976& 488& 244& 122\\
61& 184& 92& 46& 23& 70& 35& 106& 53& 160\\
80& 40& 20& 10& 5& 16& 8& 4& 2& 1\\

1154&&&&&&&&&\\
577& 1732& 866& 433& 1300& 650& 325& 976& 488& 244\\
122& 61& 184& 92& 46& 23& 70& 35& 106& 53\\
160& 80& 40& 20& 10& 5& 16& 8& 4& 2\\
1& \\

1155&&&&&&&&&\\
3466& 1733& 5200& 2600& 1300& 650& 325& 976& 488& 244\\
122& 61& 184& 92& 46& 23& 70& 35& 106& 53\\
160& 80& 40& 20& 10& 5& 16& 8& 4& 2\\
1& \\

1156&&&&&&&&&\\
578& 289& 868& 434& 217& 652& 326& 163& 490& 245\\
736& 368& 184& 92& 46& 23& 70& 35& 106& 53\\
160& 80& 40& 20& 10& 5& 16& 8& 4& 2\\
1& \\

1157&&&&&&&&&\\
3472& 1736& 868& 434& 217& 652& 326& 163& 490& 245\\
736& 368& 184& 92& 46& 23& 70& 35& 106& 53\\
160& 80& 40& 20& 10& 5& 16& 8& 4& 2\\
1& \\

1158&&&&&&&&&\\
579& 1738& 869& 2608& 1304& 652& 326& 163& 490& 245\\
736& 368& 184& 92& 46& 23& 70& 35& 106& 53\\
160& 80& 40& 20& 10& 5& 16& 8& 4& 2\\
1& \\

1159&&&&&&&&&\\
3478& 1739& 5218& 2609& 7828& 3914& 1957& 5872& 2936& 1468\\
734& 367& 1102& 551& 1654& 827& 2482& 1241& 3724& 1862\\
931& 2794& 1397& 4192& 2096& 1048& 524& 262& 131& 394\\
197& 592& 296& 148& 74& 37& 112& 56& 28& 14\\
7& 22& 11& 34& 17& 52& 26& 13& 40& 20\\
10& 5& 16& 8& 4& 2& 1& \\

1160&&&&&&&&&\\
580& 290& 145& 436& 218& 109& 328& 164& 82& 41\\
124& 62& 31& 94& 47& 142& 71& 214& 107& 322\\
161& 484& 242& 121& 364& 182& 91& 274& 137& 412\\
206& 103& 310& 155& 466& 233& 700& 350& 175& 526\\
263& 790& 395& 1186& 593& 1780& 890& 445& 1336& 668\\
334& 167& 502& 251& 754& 377& 1132& 566& 283& 850\\
425& 1276& 638& 319& 958& 479& 1438& 719& 2158& 1079\\
3238& 1619& 4858& 2429& 7288& 3644& 1822& 911& 2734& 1367\\
4102& 2051& 6154& 3077& 9232& 4616& 2308& 1154& 577& 1732\\
866& 433& 1300& 650& 325& 976& 488& 244& 122& 61\\
184& 92& 46& 23& 70& 35& 106& 53& 160& 80\\
40& 20& 10& 5& 16& 8& 4& 2& 1& \\

1161&&&&&&&&&\\
3484& 1742& 871& 2614& 1307& 3922& 1961& 5884& 2942& 1471\\
4414& 2207& 6622& 3311& 9934& 4967& 14902& 7451& 22354& 11177\\
33532& 16766& 8383& 25150& 12575& 37726& 18863& 56590& 28295& 84886\\
42443& 127330& 63665& 190996& 95498& 47749& 143248& 71624& 35812& 17906\\
8953& 26860& 13430& 6715& 20146& 10073& 30220& 15110& 7555& 22666\\
11333& 34000& 17000& 8500& 4250& 2125& 6376& 3188& 1594& 797\\
2392& 1196& 598& 299& 898& 449& 1348& 674& 337& 1012\\
506& 253& 760& 380& 190& 95& 286& 143& 430& 215\\
646& 323& 970& 485& 1456& 728& 364& 182& 91& 274\\
137& 412& 206& 103& 310& 155& 466& 233& 700& 350\\
175& 526& 263& 790& 395& 1186& 593& 1780& 890& 445\\
1336& 668& 334& 167& 502& 251& 754& 377& 1132& 566\\
283& 850& 425& 1276& 638& 319& 958& 479& 1438& 719\\
2158& 1079& 3238& 1619& 4858& 2429& 7288& 3644& 1822& 911\\
2734& 1367& 4102& 2051& 6154& 3077& 9232& 4616& 2308& 1154\\
577& 1732& 866& 433& 1300& 650& 325& 976& 488& 244\\
122& 61& 184& 92& 46& 23& 70& 35& 106& 53\\
160& 80& 40& 20& 10& 5& 16& 8& 4& 2\\
1& \\

1162&&&&&&&&&\\
581& 1744& 872& 436& 218& 109& 328& 164& 82& 41\\
124& 62& 31& 94& 47& 142& 71& 214& 107& 322\\
161& 484& 242& 121& 364& 182& 91& 274& 137& 412\\
206& 103& 310& 155& 466& 233& 700& 350& 175& 526\\
263& 790& 395& 1186& 593& 1780& 890& 445& 1336& 668\\
334& 167& 502& 251& 754& 377& 1132& 566& 283& 850\\
425& 1276& 638& 319& 958& 479& 1438& 719& 2158& 1079\\
3238& 1619& 4858& 2429& 7288& 3644& 1822& 911& 2734& 1367\\
4102& 2051& 6154& 3077& 9232& 4616& 2308& 1154& 577& 1732\\
866& 433& 1300& 650& 325& 976& 488& 244& 122& 61\\
184& 92& 46& 23& 70& 35& 106& 53& 160& 80\\
40& 20& 10& 5& 16& 8& 4& 2& 1& \\

1163&&&&&&&&&\\
3490& 1745& 5236& 2618& 1309& 3928& 1964& 982& 491& 1474\\
737& 2212& 1106& 553& 1660& 830& 415& 1246& 623& 1870\\
935& 2806& 1403& 4210& 2105& 6316& 3158& 1579& 4738& 2369\\
7108& 3554& 1777& 5332& 2666& 1333& 4000& 2000& 1000& 500\\
250& 125& 376& 188& 94& 47& 142& 71& 214& 107\\
322& 161& 484& 242& 121& 364& 182& 91& 274& 137\\
412& 206& 103& 310& 155& 466& 233& 700& 350& 175\\
526& 263& 790& 395& 1186& 593& 1780& 890& 445& 1336\\
668& 334& 167& 502& 251& 754& 377& 1132& 566& 283\\
850& 425& 1276& 638& 319& 958& 479& 1438& 719& 2158\\
1079& 3238& 1619& 4858& 2429& 7288& 3644& 1822& 911& 2734\\
1367& 4102& 2051& 6154& 3077& 9232& 4616& 2308& 1154& 577\\
1732& 866& 433& 1300& 650& 325& 976& 488& 244& 122\\
61& 184& 92& 46& 23& 70& 35& 106& 53& 160\\
80& 40& 20& 10& 5& 16& 8& 4& 2& 1\\

1164&&&&&&&&&\\
582& 291& 874& 437& 1312& 656& 328& 164& 82& 41\\
124& 62& 31& 94& 47& 142& 71& 214& 107& 322\\
161& 484& 242& 121& 364& 182& 91& 274& 137& 412\\
206& 103& 310& 155& 466& 233& 700& 350& 175& 526\\
263& 790& 395& 1186& 593& 1780& 890& 445& 1336& 668\\
334& 167& 502& 251& 754& 377& 1132& 566& 283& 850\\
425& 1276& 638& 319& 958& 479& 1438& 719& 2158& 1079\\
3238& 1619& 4858& 2429& 7288& 3644& 1822& 911& 2734& 1367\\
4102& 2051& 6154& 3077& 9232& 4616& 2308& 1154& 577& 1732\\
866& 433& 1300& 650& 325& 976& 488& 244& 122& 61\\
184& 92& 46& 23& 70& 35& 106& 53& 160& 80\\
40& 20& 10& 5& 16& 8& 4& 2& 1& \\

1165&&&&&&&&&\\
3496& 1748& 874& 437& 1312& 656& 328& 164& 82& 41\\
124& 62& 31& 94& 47& 142& 71& 214& 107& 322\\
161& 484& 242& 121& 364& 182& 91& 274& 137& 412\\
206& 103& 310& 155& 466& 233& 700& 350& 175& 526\\
263& 790& 395& 1186& 593& 1780& 890& 445& 1336& 668\\
334& 167& 502& 251& 754& 377& 1132& 566& 283& 850\\
425& 1276& 638& 319& 958& 479& 1438& 719& 2158& 1079\\
3238& 1619& 4858& 2429& 7288& 3644& 1822& 911& 2734& 1367\\
4102& 2051& 6154& 3077& 9232& 4616& 2308& 1154& 577& 1732\\
866& 433& 1300& 650& 325& 976& 488& 244& 122& 61\\
184& 92& 46& 23& 70& 35& 106& 53& 160& 80\\
40& 20& 10& 5& 16& 8& 4& 2& 1& \\

1166&&&&&&&&&\\
583& 1750& 875& 2626& 1313& 3940& 1970& 985& 2956& 1478\\
739& 2218& 1109& 3328& 1664& 832& 416& 208& 104& 52\\
26& 13& 40& 20& 10& 5& 16& 8& 4& 2\\
1& \\

1167&&&&&&&&&\\
3502& 1751& 5254& 2627& 7882& 3941& 11824& 5912& 2956& 1478\\
739& 2218& 1109& 3328& 1664& 832& 416& 208& 104& 52\\
26& 13& 40& 20& 10& 5& 16& 8& 4& 2\\
1& \\

1168&&&&&&&&&\\
584& 292& 146& 73& 220& 110& 55& 166& 83& 250\\
125& 376& 188& 94& 47& 142& 71& 214& 107& 322\\
161& 484& 242& 121& 364& 182& 91& 274& 137& 412\\
206& 103& 310& 155& 466& 233& 700& 350& 175& 526\\
263& 790& 395& 1186& 593& 1780& 890& 445& 1336& 668\\
334& 167& 502& 251& 754& 377& 1132& 566& 283& 850\\
425& 1276& 638& 319& 958& 479& 1438& 719& 2158& 1079\\
3238& 1619& 4858& 2429& 7288& 3644& 1822& 911& 2734& 1367\\
4102& 2051& 6154& 3077& 9232& 4616& 2308& 1154& 577& 1732\\
866& 433& 1300& 650& 325& 976& 488& 244& 122& 61\\
184& 92& 46& 23& 70& 35& 106& 53& 160& 80\\
40& 20& 10& 5& 16& 8& 4& 2& 1& \\

1169&&&&&&&&&\\
3508& 1754& 877& 2632& 1316& 658& 329& 988& 494& 247\\
742& 371& 1114& 557& 1672& 836& 418& 209& 628& 314\\
157& 472& 236& 118& 59& 178& 89& 268& 134& 67\\
202& 101& 304& 152& 76& 38& 19& 58& 29& 88\\
44& 22& 11& 34& 17& 52& 26& 13& 40& 20\\
10& 5& 16& 8& 4& 2& 1& \\

1170&&&&&&&&&\\
585& 1756& 878& 439& 1318& 659& 1978& 989& 2968& 1484\\
742& 371& 1114& 557& 1672& 836& 418& 209& 628& 314\\
157& 472& 236& 118& 59& 178& 89& 268& 134& 67\\
202& 101& 304& 152& 76& 38& 19& 58& 29& 88\\
44& 22& 11& 34& 17& 52& 26& 13& 40& 20\\
10& 5& 16& 8& 4& 2& 1& \\

1171&&&&&&&&&\\
3514& 1757& 5272& 2636& 1318& 659& 1978& 989& 2968& 1484\\
742& 371& 1114& 557& 1672& 836& 418& 209& 628& 314\\
157& 472& 236& 118& 59& 178& 89& 268& 134& 67\\
202& 101& 304& 152& 76& 38& 19& 58& 29& 88\\
44& 22& 11& 34& 17& 52& 26& 13& 40& 20\\
10& 5& 16& 8& 4& 2& 1& \\

1172&&&&&&&&&\\
586& 293& 880& 440& 220& 110& 55& 166& 83& 250\\
125& 376& 188& 94& 47& 142& 71& 214& 107& 322\\
161& 484& 242& 121& 364& 182& 91& 274& 137& 412\\
206& 103& 310& 155& 466& 233& 700& 350& 175& 526\\
263& 790& 395& 1186& 593& 1780& 890& 445& 1336& 668\\
334& 167& 502& 251& 754& 377& 1132& 566& 283& 850\\
425& 1276& 638& 319& 958& 479& 1438& 719& 2158& 1079\\
3238& 1619& 4858& 2429& 7288& 3644& 1822& 911& 2734& 1367\\
4102& 2051& 6154& 3077& 9232& 4616& 2308& 1154& 577& 1732\\
866& 433& 1300& 650& 325& 976& 488& 244& 122& 61\\
184& 92& 46& 23& 70& 35& 106& 53& 160& 80\\
40& 20& 10& 5& 16& 8& 4& 2& 1& \\

1173&&&&&&&&&\\
3520& 1760& 880& 440& 220& 110& 55& 166& 83& 250\\
125& 376& 188& 94& 47& 142& 71& 214& 107& 322\\
161& 484& 242& 121& 364& 182& 91& 274& 137& 412\\
206& 103& 310& 155& 466& 233& 700& 350& 175& 526\\
263& 790& 395& 1186& 593& 1780& 890& 445& 1336& 668\\
334& 167& 502& 251& 754& 377& 1132& 566& 283& 850\\
425& 1276& 638& 319& 958& 479& 1438& 719& 2158& 1079\\
3238& 1619& 4858& 2429& 7288& 3644& 1822& 911& 2734& 1367\\
4102& 2051& 6154& 3077& 9232& 4616& 2308& 1154& 577& 1732\\
866& 433& 1300& 650& 325& 976& 488& 244& 122& 61\\
184& 92& 46& 23& 70& 35& 106& 53& 160& 80\\
40& 20& 10& 5& 16& 8& 4& 2& 1& \\

1174&&&&&&&&&\\
587& 1762& 881& 2644& 1322& 661& 1984& 992& 496& 248\\
124& 62& 31& 94& 47& 142& 71& 214& 107& 322\\
161& 484& 242& 121& 364& 182& 91& 274& 137& 412\\
206& 103& 310& 155& 466& 233& 700& 350& 175& 526\\
263& 790& 395& 1186& 593& 1780& 890& 445& 1336& 668\\
334& 167& 502& 251& 754& 377& 1132& 566& 283& 850\\
425& 1276& 638& 319& 958& 479& 1438& 719& 2158& 1079\\
3238& 1619& 4858& 2429& 7288& 3644& 1822& 911& 2734& 1367\\
4102& 2051& 6154& 3077& 9232& 4616& 2308& 1154& 577& 1732\\
866& 433& 1300& 650& 325& 976& 488& 244& 122& 61\\
184& 92& 46& 23& 70& 35& 106& 53& 160& 80\\
40& 20& 10& 5& 16& 8& 4& 2& 1& \\

1175&&&&&&&&&\\
3526& 1763& 5290& 2645& 7936& 3968& 1984& 992& 496& 248\\
124& 62& 31& 94& 47& 142& 71& 214& 107& 322\\
161& 484& 242& 121& 364& 182& 91& 274& 137& 412\\
206& 103& 310& 155& 466& 233& 700& 350& 175& 526\\
263& 790& 395& 1186& 593& 1780& 890& 445& 1336& 668\\
334& 167& 502& 251& 754& 377& 1132& 566& 283& 850\\
425& 1276& 638& 319& 958& 479& 1438& 719& 2158& 1079\\
3238& 1619& 4858& 2429& 7288& 3644& 1822& 911& 2734& 1367\\
4102& 2051& 6154& 3077& 9232& 4616& 2308& 1154& 577& 1732\\
866& 433& 1300& 650& 325& 976& 488& 244& 122& 61\\
184& 92& 46& 23& 70& 35& 106& 53& 160& 80\\
40& 20& 10& 5& 16& 8& 4& 2& 1& \\

1176&&&&&&&&&\\
588& 294& 147& 442& 221& 664& 332& 166& 83& 250\\
125& 376& 188& 94& 47& 142& 71& 214& 107& 322\\
161& 484& 242& 121& 364& 182& 91& 274& 137& 412\\
206& 103& 310& 155& 466& 233& 700& 350& 175& 526\\
263& 790& 395& 1186& 593& 1780& 890& 445& 1336& 668\\
334& 167& 502& 251& 754& 377& 1132& 566& 283& 850\\
425& 1276& 638& 319& 958& 479& 1438& 719& 2158& 1079\\
3238& 1619& 4858& 2429& 7288& 3644& 1822& 911& 2734& 1367\\
4102& 2051& 6154& 3077& 9232& 4616& 2308& 1154& 577& 1732\\
866& 433& 1300& 650& 325& 976& 488& 244& 122& 61\\
184& 92& 46& 23& 70& 35& 106& 53& 160& 80\\
40& 20& 10& 5& 16& 8& 4& 2& 1& \\

1177&&&&&&&&&\\
3532& 1766& 883& 2650& 1325& 3976& 1988& 994& 497& 1492\\
746& 373& 1120& 560& 280& 140& 70& 35& 106& 53\\
160& 80& 40& 20& 10& 5& 16& 8& 4& 2\\
1& \\

1178&&&&&&&&&\\
589& 1768& 884& 442& 221& 664& 332& 166& 83& 250\\
125& 376& 188& 94& 47& 142& 71& 214& 107& 322\\
161& 484& 242& 121& 364& 182& 91& 274& 137& 412\\
206& 103& 310& 155& 466& 233& 700& 350& 175& 526\\
263& 790& 395& 1186& 593& 1780& 890& 445& 1336& 668\\
334& 167& 502& 251& 754& 377& 1132& 566& 283& 850\\
425& 1276& 638& 319& 958& 479& 1438& 719& 2158& 1079\\
3238& 1619& 4858& 2429& 7288& 3644& 1822& 911& 2734& 1367\\
4102& 2051& 6154& 3077& 9232& 4616& 2308& 1154& 577& 1732\\
866& 433& 1300& 650& 325& 976& 488& 244& 122& 61\\
184& 92& 46& 23& 70& 35& 106& 53& 160& 80\\
40& 20& 10& 5& 16& 8& 4& 2& 1& \\

1179&&&&&&&&&\\
3538& 1769& 5308& 2654& 1327& 3982& 1991& 5974& 2987& 8962\\
4481& 13444& 6722& 3361& 10084& 5042& 2521& 7564& 3782& 1891\\
5674& 2837& 8512& 4256& 2128& 1064& 532& 266& 133& 400\\
200& 100& 50& 25& 76& 38& 19& 58& 29& 88\\
44& 22& 11& 34& 17& 52& 26& 13& 40& 20\\
10& 5& 16& 8& 4& 2& 1& \\

1180&&&&&&&&&\\
590& 295& 886& 443& 1330& 665& 1996& 998& 499& 1498\\
749& 2248& 1124& 562& 281& 844& 422& 211& 634& 317\\
952& 476& 238& 119& 358& 179& 538& 269& 808& 404\\
202& 101& 304& 152& 76& 38& 19& 58& 29& 88\\
44& 22& 11& 34& 17& 52& 26& 13& 40& 20\\
10& 5& 16& 8& 4& 2& 1& \\

1181&&&&&&&&&\\
3544& 1772& 886& 443& 1330& 665& 1996& 998& 499& 1498\\
749& 2248& 1124& 562& 281& 844& 422& 211& 634& 317\\
952& 476& 238& 119& 358& 179& 538& 269& 808& 404\\
202& 101& 304& 152& 76& 38& 19& 58& 29& 88\\
44& 22& 11& 34& 17& 52& 26& 13& 40& 20\\
10& 5& 16& 8& 4& 2& 1& \\

1182&&&&&&&&&\\
591& 1774& 887& 2662& 1331& 3994& 1997& 5992& 2996& 1498\\
749& 2248& 1124& 562& 281& 844& 422& 211& 634& 317\\
952& 476& 238& 119& 358& 179& 538& 269& 808& 404\\
202& 101& 304& 152& 76& 38& 19& 58& 29& 88\\
44& 22& 11& 34& 17& 52& 26& 13& 40& 20\\
10& 5& 16& 8& 4& 2& 1& \\

1183&&&&&&&&&\\
3550& 1775& 5326& 2663& 7990& 3995& 11986& 5993& 17980& 8990\\
4495& 13486& 6743& 20230& 10115& 30346& 15173& 45520& 22760& 11380\\
5690& 2845& 8536& 4268& 2134& 1067& 3202& 1601& 4804& 2402\\
1201& 3604& 1802& 901& 2704& 1352& 676& 338& 169& 508\\
254& 127& 382& 191& 574& 287& 862& 431& 1294& 647\\
1942& 971& 2914& 1457& 4372& 2186& 1093& 3280& 1640& 820\\
410& 205& 616& 308& 154& 77& 232& 116& 58& 29\\
88& 44& 22& 11& 34& 17& 52& 26& 13& 40\\
20& 10& 5& 16& 8& 4& 2& 1& \\

1184&&&&&&&&&\\
592& 296& 148& 74& 37& 112& 56& 28& 14& 7\\
22& 11& 34& 17& 52& 26& 13& 40& 20& 10\\
5& 16& 8& 4& 2& 1& \\

1185&&&&&&&&&\\
3556& 1778& 889& 2668& 1334& 667& 2002& 1001& 3004& 1502\\
751& 2254& 1127& 3382& 1691& 5074& 2537& 7612& 3806& 1903\\
5710& 2855& 8566& 4283& 12850& 6425& 19276& 9638& 4819& 14458\\
7229& 21688& 10844& 5422& 2711& 8134& 4067& 12202& 6101& 18304\\
9152& 4576& 2288& 1144& 572& 286& 143& 430& 215& 646\\
323& 970& 485& 1456& 728& 364& 182& 91& 274& 137\\
412& 206& 103& 310& 155& 466& 233& 700& 350& 175\\
526& 263& 790& 395& 1186& 593& 1780& 890& 445& 1336\\
668& 334& 167& 502& 251& 754& 377& 1132& 566& 283\\
850& 425& 1276& 638& 319& 958& 479& 1438& 719& 2158\\
1079& 3238& 1619& 4858& 2429& 7288& 3644& 1822& 911& 2734\\
1367& 4102& 2051& 6154& 3077& 9232& 4616& 2308& 1154& 577\\
1732& 866& 433& 1300& 650& 325& 976& 488& 244& 122\\
61& 184& 92& 46& 23& 70& 35& 106& 53& 160\\
80& 40& 20& 10& 5& 16& 8& 4& 2& 1\\

1186&&&&&&&&&\\
593& 1780& 890& 445& 1336& 668& 334& 167& 502& 251\\
754& 377& 1132& 566& 283& 850& 425& 1276& 638& 319\\
958& 479& 1438& 719& 2158& 1079& 3238& 1619& 4858& 2429\\
7288& 3644& 1822& 911& 2734& 1367& 4102& 2051& 6154& 3077\\
9232& 4616& 2308& 1154& 577& 1732& 866& 433& 1300& 650\\
325& 976& 488& 244& 122& 61& 184& 92& 46& 23\\
70& 35& 106& 53& 160& 80& 40& 20& 10& 5\\
16& 8& 4& 2& 1& \\

1187&&&&&&&&&\\
3562& 1781& 5344& 2672& 1336& 668& 334& 167& 502& 251\\
754& 377& 1132& 566& 283& 850& 425& 1276& 638& 319\\
958& 479& 1438& 719& 2158& 1079& 3238& 1619& 4858& 2429\\
7288& 3644& 1822& 911& 2734& 1367& 4102& 2051& 6154& 3077\\
9232& 4616& 2308& 1154& 577& 1732& 866& 433& 1300& 650\\
325& 976& 488& 244& 122& 61& 184& 92& 46& 23\\
70& 35& 106& 53& 160& 80& 40& 20& 10& 5\\
16& 8& 4& 2& 1& \\

1188&&&&&&&&&\\
594& 297& 892& 446& 223& 670& 335& 1006& 503& 1510\\
755& 2266& 1133& 3400& 1700& 850& 425& 1276& 638& 319\\
958& 479& 1438& 719& 2158& 1079& 3238& 1619& 4858& 2429\\
7288& 3644& 1822& 911& 2734& 1367& 4102& 2051& 6154& 3077\\
9232& 4616& 2308& 1154& 577& 1732& 866& 433& 1300& 650\\
325& 976& 488& 244& 122& 61& 184& 92& 46& 23\\
70& 35& 106& 53& 160& 80& 40& 20& 10& 5\\
16& 8& 4& 2& 1& \\

1189&&&&&&&&&\\
3568& 1784& 892& 446& 223& 670& 335& 1006& 503& 1510\\
755& 2266& 1133& 3400& 1700& 850& 425& 1276& 638& 319\\
958& 479& 1438& 719& 2158& 1079& 3238& 1619& 4858& 2429\\
7288& 3644& 1822& 911& 2734& 1367& 4102& 2051& 6154& 3077\\
9232& 4616& 2308& 1154& 577& 1732& 866& 433& 1300& 650\\
325& 976& 488& 244& 122& 61& 184& 92& 46& 23\\
70& 35& 106& 53& 160& 80& 40& 20& 10& 5\\
16& 8& 4& 2& 1& \\

1190&&&&&&&&&\\
595& 1786& 893& 2680& 1340& 670& 335& 1006& 503& 1510\\
755& 2266& 1133& 3400& 1700& 850& 425& 1276& 638& 319\\
958& 479& 1438& 719& 2158& 1079& 3238& 1619& 4858& 2429\\
7288& 3644& 1822& 911& 2734& 1367& 4102& 2051& 6154& 3077\\
9232& 4616& 2308& 1154& 577& 1732& 866& 433& 1300& 650\\
325& 976& 488& 244& 122& 61& 184& 92& 46& 23\\
70& 35& 106& 53& 160& 80& 40& 20& 10& 5\\
16& 8& 4& 2& 1& \\

1191&&&&&&&&&\\
3574& 1787& 5362& 2681& 8044& 4022& 2011& 6034& 3017& 9052\\
4526& 2263& 6790& 3395& 10186& 5093& 15280& 7640& 3820& 1910\\
955& 2866& 1433& 4300& 2150& 1075& 3226& 1613& 4840& 2420\\
1210& 605& 1816& 908& 454& 227& 682& 341& 1024& 512\\
256& 128& 64& 32& 16& 8& 4& 2& 1& \\

1192&&&&&&&&&\\
596& 298& 149& 448& 224& 112& 56& 28& 14& 7\\
22& 11& 34& 17& 52& 26& 13& 40& 20& 10\\
5& 16& 8& 4& 2& 1& \\

1193&&&&&&&&&\\
3580& 1790& 895& 2686& 1343& 4030& 2015& 6046& 3023& 9070\\
4535& 13606& 6803& 20410& 10205& 30616& 15308& 7654& 3827& 11482\\
5741& 17224& 8612& 4306& 2153& 6460& 3230& 1615& 4846& 2423\\
7270& 3635& 10906& 5453& 16360& 8180& 4090& 2045& 6136& 3068\\
1534& 767& 2302& 1151& 3454& 1727& 5182& 2591& 7774& 3887\\
11662& 5831& 17494& 8747& 26242& 13121& 39364& 19682& 9841& 29524\\
14762& 7381& 22144& 11072& 5536& 2768& 1384& 692& 346& 173\\
520& 260& 130& 65& 196& 98& 49& 148& 74& 37\\
112& 56& 28& 14& 7& 22& 11& 34& 17& 52\\
26& 13& 40& 20& 10& 5& 16& 8& 4& 2\\
1& \\

1194&&&&&&&&&\\
597& 1792& 896& 448& 224& 112& 56& 28& 14& 7\\
22& 11& 34& 17& 52& 26& 13& 40& 20& 10\\
5& 16& 8& 4& 2& 1& \\

1195&&&&&&&&&\\
3586& 1793& 5380& 2690& 1345& 4036& 2018& 1009& 3028& 1514\\
757& 2272& 1136& 568& 284& 142& 71& 214& 107& 322\\
161& 484& 242& 121& 364& 182& 91& 274& 137& 412\\
206& 103& 310& 155& 466& 233& 700& 350& 175& 526\\
263& 790& 395& 1186& 593& 1780& 890& 445& 1336& 668\\
334& 167& 502& 251& 754& 377& 1132& 566& 283& 850\\
425& 1276& 638& 319& 958& 479& 1438& 719& 2158& 1079\\
3238& 1619& 4858& 2429& 7288& 3644& 1822& 911& 2734& 1367\\
4102& 2051& 6154& 3077& 9232& 4616& 2308& 1154& 577& 1732\\
866& 433& 1300& 650& 325& 976& 488& 244& 122& 61\\
184& 92& 46& 23& 70& 35& 106& 53& 160& 80\\
40& 20& 10& 5& 16& 8& 4& 2& 1& \\

1196&&&&&&&&&\\
598& 299& 898& 449& 1348& 674& 337& 1012& 506& 253\\
760& 380& 190& 95& 286& 143& 430& 215& 646& 323\\
970& 485& 1456& 728& 364& 182& 91& 274& 137& 412\\
206& 103& 310& 155& 466& 233& 700& 350& 175& 526\\
263& 790& 395& 1186& 593& 1780& 890& 445& 1336& 668\\
334& 167& 502& 251& 754& 377& 1132& 566& 283& 850\\
425& 1276& 638& 319& 958& 479& 1438& 719& 2158& 1079\\
3238& 1619& 4858& 2429& 7288& 3644& 1822& 911& 2734& 1367\\
4102& 2051& 6154& 3077& 9232& 4616& 2308& 1154& 577& 1732\\
866& 433& 1300& 650& 325& 976& 488& 244& 122& 61\\
184& 92& 46& 23& 70& 35& 106& 53& 160& 80\\
40& 20& 10& 5& 16& 8& 4& 2& 1& \\

1197&&&&&&&&&\\
3592& 1796& 898& 449& 1348& 674& 337& 1012& 506& 253\\
760& 380& 190& 95& 286& 143& 430& 215& 646& 323\\
970& 485& 1456& 728& 364& 182& 91& 274& 137& 412\\
206& 103& 310& 155& 466& 233& 700& 350& 175& 526\\
263& 790& 395& 1186& 593& 1780& 890& 445& 1336& 668\\
334& 167& 502& 251& 754& 377& 1132& 566& 283& 850\\
425& 1276& 638& 319& 958& 479& 1438& 719& 2158& 1079\\
3238& 1619& 4858& 2429& 7288& 3644& 1822& 911& 2734& 1367\\
4102& 2051& 6154& 3077& 9232& 4616& 2308& 1154& 577& 1732\\
866& 433& 1300& 650& 325& 976& 488& 244& 122& 61\\
184& 92& 46& 23& 70& 35& 106& 53& 160& 80\\
40& 20& 10& 5& 16& 8& 4& 2& 1& \\

1198&&&&&&&&&\\
599& 1798& 899& 2698& 1349& 4048& 2024& 1012& 506& 253\\
760& 380& 190& 95& 286& 143& 430& 215& 646& 323\\
970& 485& 1456& 728& 364& 182& 91& 274& 137& 412\\
206& 103& 310& 155& 466& 233& 700& 350& 175& 526\\
263& 790& 395& 1186& 593& 1780& 890& 445& 1336& 668\\
334& 167& 502& 251& 754& 377& 1132& 566& 283& 850\\
425& 1276& 638& 319& 958& 479& 1438& 719& 2158& 1079\\
3238& 1619& 4858& 2429& 7288& 3644& 1822& 911& 2734& 1367\\
4102& 2051& 6154& 3077& 9232& 4616& 2308& 1154& 577& 1732\\
866& 433& 1300& 650& 325& 976& 488& 244& 122& 61\\
184& 92& 46& 23& 70& 35& 106& 53& 160& 80\\
40& 20& 10& 5& 16& 8& 4& 2& 1& \\

1199&&&&&&&&&\\
3598& 1799& 5398& 2699& 8098& 4049& 12148& 6074& 3037& 9112\\
4556& 2278& 1139& 3418& 1709& 5128& 2564& 1282& 641& 1924\\
962& 481& 1444& 722& 361& 1084& 542& 271& 814& 407\\
1222& 611& 1834& 917& 2752& 1376& 688& 344& 172& 86\\
43& 130& 65& 196& 98& 49& 148& 74& 37& 112\\
56& 28& 14& 7& 22& 11& 34& 17& 52& 26\\
13& 40& 20& 10& 5& 16& 8& 4& 2& 1\\

1200&&&&&&&&&\\
600& 300& 150& 75& 226& 113& 340& 170& 85& 256\\
128& 64& 32& 16& 8& 4& 2& 1& \\

1201&&&&&&&&&\\
3604& 1802& 901& 2704& 1352& 676& 338& 169& 508& 254\\
127& 382& 191& 574& 287& 862& 431& 1294& 647& 1942\\
971& 2914& 1457& 4372& 2186& 1093& 3280& 1640& 820& 410\\
205& 616& 308& 154& 77& 232& 116& 58& 29& 88\\
44& 22& 11& 34& 17& 52& 26& 13& 40& 20\\
10& 5& 16& 8& 4& 2& 1& \\

1202&&&&&&&&&\\
601& 1804& 902& 451& 1354& 677& 2032& 1016& 508& 254\\
127& 382& 191& 574& 287& 862& 431& 1294& 647& 1942\\
971& 2914& 1457& 4372& 2186& 1093& 3280& 1640& 820& 410\\
205& 616& 308& 154& 77& 232& 116& 58& 29& 88\\
44& 22& 11& 34& 17& 52& 26& 13& 40& 20\\
10& 5& 16& 8& 4& 2& 1& \\

1203&&&&&&&&&\\
3610& 1805& 5416& 2708& 1354& 677& 2032& 1016& 508& 254\\
127& 382& 191& 574& 287& 862& 431& 1294& 647& 1942\\
971& 2914& 1457& 4372& 2186& 1093& 3280& 1640& 820& 410\\
205& 616& 308& 154& 77& 232& 116& 58& 29& 88\\
44& 22& 11& 34& 17& 52& 26& 13& 40& 20\\
10& 5& 16& 8& 4& 2& 1& \\

1204&&&&&&&&&\\
602& 301& 904& 452& 226& 113& 340& 170& 85& 256\\
128& 64& 32& 16& 8& 4& 2& 1& \\

1205&&&&&&&&&\\
3616& 1808& 904& 452& 226& 113& 340& 170& 85& 256\\
128& 64& 32& 16& 8& 4& 2& 1& \\

1206&&&&&&&&&\\
603& 1810& 905& 2716& 1358& 679& 2038& 1019& 3058& 1529\\
4588& 2294& 1147& 3442& 1721& 5164& 2582& 1291& 3874& 1937\\
5812& 2906& 1453& 4360& 2180& 1090& 545& 1636& 818& 409\\
1228& 614& 307& 922& 461& 1384& 692& 346& 173& 520\\
260& 130& 65& 196& 98& 49& 148& 74& 37& 112\\
56& 28& 14& 7& 22& 11& 34& 17& 52& 26\\
13& 40& 20& 10& 5& 16& 8& 4& 2& 1\\

1207&&&&&&&&&\\
3622& 1811& 5434& 2717& 8152& 4076& 2038& 1019& 3058& 1529\\
4588& 2294& 1147& 3442& 1721& 5164& 2582& 1291& 3874& 1937\\
5812& 2906& 1453& 4360& 2180& 1090& 545& 1636& 818& 409\\
1228& 614& 307& 922& 461& 1384& 692& 346& 173& 520\\
260& 130& 65& 196& 98& 49& 148& 74& 37& 112\\
56& 28& 14& 7& 22& 11& 34& 17& 52& 26\\
13& 40& 20& 10& 5& 16& 8& 4& 2& 1\\

1208&&&&&&&&&\\
604& 302& 151& 454& 227& 682& 341& 1024& 512& 256\\
128& 64& 32& 16& 8& 4& 2& 1& \\

1209&&&&&&&&&\\
3628& 1814& 907& 2722& 1361& 4084& 2042& 1021& 3064& 1532\\
766& 383& 1150& 575& 1726& 863& 2590& 1295& 3886& 1943\\
5830& 2915& 8746& 4373& 13120& 6560& 3280& 1640& 820& 410\\
205& 616& 308& 154& 77& 232& 116& 58& 29& 88\\
44& 22& 11& 34& 17& 52& 26& 13& 40& 20\\
10& 5& 16& 8& 4& 2& 1& \\

1210&&&&&&&&&\\
605& 1816& 908& 454& 227& 682& 341& 1024& 512& 256\\
128& 64& 32& 16& 8& 4& 2& 1& \\

1211&&&&&&&&&\\
3634& 1817& 5452& 2726& 1363& 4090& 2045& 6136& 3068& 1534\\
767& 2302& 1151& 3454& 1727& 5182& 2591& 7774& 3887& 11662\\
5831& 17494& 8747& 26242& 13121& 39364& 19682& 9841& 29524& 14762\\
7381& 22144& 11072& 5536& 2768& 1384& 692& 346& 173& 520\\
260& 130& 65& 196& 98& 49& 148& 74& 37& 112\\
56& 28& 14& 7& 22& 11& 34& 17& 52& 26\\
13& 40& 20& 10& 5& 16& 8& 4& 2& 1\\

1212&&&&&&&&&\\
606& 303& 910& 455& 1366& 683& 2050& 1025& 3076& 1538\\
769& 2308& 1154& 577& 1732& 866& 433& 1300& 650& 325\\
976& 488& 244& 122& 61& 184& 92& 46& 23& 70\\
35& 106& 53& 160& 80& 40& 20& 10& 5& 16\\
8& 4& 2& 1& \\

1213&&&&&&&&&\\
3640& 1820& 910& 455& 1366& 683& 2050& 1025& 3076& 1538\\
769& 2308& 1154& 577& 1732& 866& 433& 1300& 650& 325\\
976& 488& 244& 122& 61& 184& 92& 46& 23& 70\\
35& 106& 53& 160& 80& 40& 20& 10& 5& 16\\
8& 4& 2& 1& \\

1214&&&&&&&&&\\
607& 1822& 911& 2734& 1367& 4102& 2051& 6154& 3077& 9232\\
4616& 2308& 1154& 577& 1732& 866& 433& 1300& 650& 325\\
976& 488& 244& 122& 61& 184& 92& 46& 23& 70\\
35& 106& 53& 160& 80& 40& 20& 10& 5& 16\\
8& 4& 2& 1& \\

1215&&&&&&&&&\\
3646& 1823& 5470& 2735& 8206& 4103& 12310& 6155& 18466& 9233\\
27700& 13850& 6925& 20776& 10388& 5194& 2597& 7792& 3896& 1948\\
974& 487& 1462& 731& 2194& 1097& 3292& 1646& 823& 2470\\
1235& 3706& 1853& 5560& 2780& 1390& 695& 2086& 1043& 3130\\
1565& 4696& 2348& 1174& 587& 1762& 881& 2644& 1322& 661\\
1984& 992& 496& 248& 124& 62& 31& 94& 47& 142\\
71& 214& 107& 322& 161& 484& 242& 121& 364& 182\\
91& 274& 137& 412& 206& 103& 310& 155& 466& 233\\
700& 350& 175& 526& 263& 790& 395& 1186& 593& 1780\\
890& 445& 1336& 668& 334& 167& 502& 251& 754& 377\\
1132& 566& 283& 850& 425& 1276& 638& 319& 958& 479\\
1438& 719& 2158& 1079& 3238& 1619& 4858& 2429& 7288& 3644\\
1822& 911& 2734& 1367& 4102& 2051& 6154& 3077& 9232& 4616\\
2308& 1154& 577& 1732& 866& 433& 1300& 650& 325& 976\\
488& 244& 122& 61& 184& 92& 46& 23& 70& 35\\
106& 53& 160& 80& 40& 20& 10& 5& 16& 8\\
4& 2& 1& \\

1216&&&&&&&&&\\
608& 304& 152& 76& 38& 19& 58& 29& 88& 44\\
22& 11& 34& 17& 52& 26& 13& 40& 20& 10\\
5& 16& 8& 4& 2& 1& \\

1217&&&&&&&&&\\
3652& 1826& 913& 2740& 1370& 685& 2056& 1028& 514& 257\\
772& 386& 193& 580& 290& 145& 436& 218& 109& 328\\
164& 82& 41& 124& 62& 31& 94& 47& 142& 71\\
214& 107& 322& 161& 484& 242& 121& 364& 182& 91\\
274& 137& 412& 206& 103& 310& 155& 466& 233& 700\\
350& 175& 526& 263& 790& 395& 1186& 593& 1780& 890\\
445& 1336& 668& 334& 167& 502& 251& 754& 377& 1132\\
566& 283& 850& 425& 1276& 638& 319& 958& 479& 1438\\
719& 2158& 1079& 3238& 1619& 4858& 2429& 7288& 3644& 1822\\
911& 2734& 1367& 4102& 2051& 6154& 3077& 9232& 4616& 2308\\
1154& 577& 1732& 866& 433& 1300& 650& 325& 976& 488\\
244& 122& 61& 184& 92& 46& 23& 70& 35& 106\\
53& 160& 80& 40& 20& 10& 5& 16& 8& 4\\
2& 1& \\

1218&&&&&&&&&\\
609& 1828& 914& 457& 1372& 686& 343& 1030& 515& 1546\\
773& 2320& 1160& 580& 290& 145& 436& 218& 109& 328\\
164& 82& 41& 124& 62& 31& 94& 47& 142& 71\\
214& 107& 322& 161& 484& 242& 121& 364& 182& 91\\
274& 137& 412& 206& 103& 310& 155& 466& 233& 700\\
350& 175& 526& 263& 790& 395& 1186& 593& 1780& 890\\
445& 1336& 668& 334& 167& 502& 251& 754& 377& 1132\\
566& 283& 850& 425& 1276& 638& 319& 958& 479& 1438\\
719& 2158& 1079& 3238& 1619& 4858& 2429& 7288& 3644& 1822\\
911& 2734& 1367& 4102& 2051& 6154& 3077& 9232& 4616& 2308\\
1154& 577& 1732& 866& 433& 1300& 650& 325& 976& 488\\
244& 122& 61& 184& 92& 46& 23& 70& 35& 106\\
53& 160& 80& 40& 20& 10& 5& 16& 8& 4\\
2& 1& \\

1219&&&&&&&&&\\
3658& 1829& 5488& 2744& 1372& 686& 343& 1030& 515& 1546\\
773& 2320& 1160& 580& 290& 145& 436& 218& 109& 328\\
164& 82& 41& 124& 62& 31& 94& 47& 142& 71\\
214& 107& 322& 161& 484& 242& 121& 364& 182& 91\\
274& 137& 412& 206& 103& 310& 155& 466& 233& 700\\
350& 175& 526& 263& 790& 395& 1186& 593& 1780& 890\\
445& 1336& 668& 334& 167& 502& 251& 754& 377& 1132\\
566& 283& 850& 425& 1276& 638& 319& 958& 479& 1438\\
719& 2158& 1079& 3238& 1619& 4858& 2429& 7288& 3644& 1822\\
911& 2734& 1367& 4102& 2051& 6154& 3077& 9232& 4616& 2308\\
1154& 577& 1732& 866& 433& 1300& 650& 325& 976& 488\\
244& 122& 61& 184& 92& 46& 23& 70& 35& 106\\
53& 160& 80& 40& 20& 10& 5& 16& 8& 4\\
2& 1& \\

1220&&&&&&&&&\\
610& 305& 916& 458& 229& 688& 344& 172& 86& 43\\
130& 65& 196& 98& 49& 148& 74& 37& 112& 56\\
28& 14& 7& 22& 11& 34& 17& 52& 26& 13\\
40& 20& 10& 5& 16& 8& 4& 2& 1& \\

1221&&&&&&&&&\\
3664& 1832& 916& 458& 229& 688& 344& 172& 86& 43\\
130& 65& 196& 98& 49& 148& 74& 37& 112& 56\\
28& 14& 7& 22& 11& 34& 17& 52& 26& 13\\
40& 20& 10& 5& 16& 8& 4& 2& 1& \\

1222&&&&&&&&&\\
611& 1834& 917& 2752& 1376& 688& 344& 172& 86& 43\\
130& 65& 196& 98& 49& 148& 74& 37& 112& 56\\
28& 14& 7& 22& 11& 34& 17& 52& 26& 13\\
40& 20& 10& 5& 16& 8& 4& 2& 1& \\

1223&&&&&&&&&\\
3670& 1835& 5506& 2753& 8260& 4130& 2065& 6196& 3098& 1549\\
4648& 2324& 1162& 581& 1744& 872& 436& 218& 109& 328\\
164& 82& 41& 124& 62& 31& 94& 47& 142& 71\\
214& 107& 322& 161& 484& 242& 121& 364& 182& 91\\
274& 137& 412& 206& 103& 310& 155& 466& 233& 700\\
350& 175& 526& 263& 790& 395& 1186& 593& 1780& 890\\
445& 1336& 668& 334& 167& 502& 251& 754& 377& 1132\\
566& 283& 850& 425& 1276& 638& 319& 958& 479& 1438\\
719& 2158& 1079& 3238& 1619& 4858& 2429& 7288& 3644& 1822\\
911& 2734& 1367& 4102& 2051& 6154& 3077& 9232& 4616& 2308\\
1154& 577& 1732& 866& 433& 1300& 650& 325& 976& 488\\
244& 122& 61& 184& 92& 46& 23& 70& 35& 106\\
53& 160& 80& 40& 20& 10& 5& 16& 8& 4\\
2& 1& \\

1224&&&&&&&&&\\
612& 306& 153& 460& 230& 115& 346& 173& 520& 260\\
130& 65& 196& 98& 49& 148& 74& 37& 112& 56\\
28& 14& 7& 22& 11& 34& 17& 52& 26& 13\\
40& 20& 10& 5& 16& 8& 4& 2& 1& \\

1225&&&&&&&&&\\
3676& 1838& 919& 2758& 1379& 4138& 2069& 6208& 3104& 1552\\
776& 388& 194& 97& 292& 146& 73& 220& 110& 55\\
166& 83& 250& 125& 376& 188& 94& 47& 142& 71\\
214& 107& 322& 161& 484& 242& 121& 364& 182& 91\\
274& 137& 412& 206& 103& 310& 155& 466& 233& 700\\
350& 175& 526& 263& 790& 395& 1186& 593& 1780& 890\\
445& 1336& 668& 334& 167& 502& 251& 754& 377& 1132\\
566& 283& 850& 425& 1276& 638& 319& 958& 479& 1438\\
719& 2158& 1079& 3238& 1619& 4858& 2429& 7288& 3644& 1822\\
911& 2734& 1367& 4102& 2051& 6154& 3077& 9232& 4616& 2308\\
1154& 577& 1732& 866& 433& 1300& 650& 325& 976& 488\\
244& 122& 61& 184& 92& 46& 23& 70& 35& 106\\
53& 160& 80& 40& 20& 10& 5& 16& 8& 4\\
2& 1& \\

1226&&&&&&&&&\\
613& 1840& 920& 460& 230& 115& 346& 173& 520& 260\\
130& 65& 196& 98& 49& 148& 74& 37& 112& 56\\
28& 14& 7& 22& 11& 34& 17& 52& 26& 13\\
40& 20& 10& 5& 16& 8& 4& 2& 1& \\

1227&&&&&&&&&\\
3682& 1841& 5524& 2762& 1381& 4144& 2072& 1036& 518& 259\\
778& 389& 1168& 584& 292& 146& 73& 220& 110& 55\\
166& 83& 250& 125& 376& 188& 94& 47& 142& 71\\
214& 107& 322& 161& 484& 242& 121& 364& 182& 91\\
274& 137& 412& 206& 103& 310& 155& 466& 233& 700\\
350& 175& 526& 263& 790& 395& 1186& 593& 1780& 890\\
445& 1336& 668& 334& 167& 502& 251& 754& 377& 1132\\
566& 283& 850& 425& 1276& 638& 319& 958& 479& 1438\\
719& 2158& 1079& 3238& 1619& 4858& 2429& 7288& 3644& 1822\\
911& 2734& 1367& 4102& 2051& 6154& 3077& 9232& 4616& 2308\\
1154& 577& 1732& 866& 433& 1300& 650& 325& 976& 488\\
244& 122& 61& 184& 92& 46& 23& 70& 35& 106\\
53& 160& 80& 40& 20& 10& 5& 16& 8& 4\\
2& 1& \\

1228&&&&&&&&&\\
614& 307& 922& 461& 1384& 692& 346& 173& 520& 260\\
130& 65& 196& 98& 49& 148& 74& 37& 112& 56\\
28& 14& 7& 22& 11& 34& 17& 52& 26& 13\\
40& 20& 10& 5& 16& 8& 4& 2& 1& \\

1229&&&&&&&&&\\
3688& 1844& 922& 461& 1384& 692& 346& 173& 520& 260\\
130& 65& 196& 98& 49& 148& 74& 37& 112& 56\\
28& 14& 7& 22& 11& 34& 17& 52& 26& 13\\
40& 20& 10& 5& 16& 8& 4& 2& 1& \\

1230&&&&&&&&&\\
615& 1846& 923& 2770& 1385& 4156& 2078& 1039& 3118& 1559\\
4678& 2339& 7018& 3509& 10528& 5264& 2632& 1316& 658& 329\\
988& 494& 247& 742& 371& 1114& 557& 1672& 836& 418\\
209& 628& 314& 157& 472& 236& 118& 59& 178& 89\\
268& 134& 67& 202& 101& 304& 152& 76& 38& 19\\
58& 29& 88& 44& 22& 11& 34& 17& 52& 26\\
13& 40& 20& 10& 5& 16& 8& 4& 2& 1\\

1231&&&&&&&&&\\
3694& 1847& 5542& 2771& 8314& 4157& 12472& 6236& 3118& 1559\\
4678& 2339& 7018& 3509& 10528& 5264& 2632& 1316& 658& 329\\
988& 494& 247& 742& 371& 1114& 557& 1672& 836& 418\\
209& 628& 314& 157& 472& 236& 118& 59& 178& 89\\
268& 134& 67& 202& 101& 304& 152& 76& 38& 19\\
58& 29& 88& 44& 22& 11& 34& 17& 52& 26\\
13& 40& 20& 10& 5& 16& 8& 4& 2& 1\\

1232&&&&&&&&&\\
616& 308& 154& 77& 232& 116& 58& 29& 88& 44\\
22& 11& 34& 17& 52& 26& 13& 40& 20& 10\\
5& 16& 8& 4& 2& 1& \\

1233&&&&&&&&&\\
3700& 1850& 925& 2776& 1388& 694& 347& 1042& 521& 1564\\
782& 391& 1174& 587& 1762& 881& 2644& 1322& 661& 1984\\
992& 496& 248& 124& 62& 31& 94& 47& 142& 71\\
214& 107& 322& 161& 484& 242& 121& 364& 182& 91\\
274& 137& 412& 206& 103& 310& 155& 466& 233& 700\\
350& 175& 526& 263& 790& 395& 1186& 593& 1780& 890\\
445& 1336& 668& 334& 167& 502& 251& 754& 377& 1132\\
566& 283& 850& 425& 1276& 638& 319& 958& 479& 1438\\
719& 2158& 1079& 3238& 1619& 4858& 2429& 7288& 3644& 1822\\
911& 2734& 1367& 4102& 2051& 6154& 3077& 9232& 4616& 2308\\
1154& 577& 1732& 866& 433& 1300& 650& 325& 976& 488\\
244& 122& 61& 184& 92& 46& 23& 70& 35& 106\\
53& 160& 80& 40& 20& 10& 5& 16& 8& 4\\
2& 1& \\

1234&&&&&&&&&\\
617& 1852& 926& 463& 1390& 695& 2086& 1043& 3130& 1565\\
4696& 2348& 1174& 587& 1762& 881& 2644& 1322& 661& 1984\\
992& 496& 248& 124& 62& 31& 94& 47& 142& 71\\
214& 107& 322& 161& 484& 242& 121& 364& 182& 91\\
274& 137& 412& 206& 103& 310& 155& 466& 233& 700\\
350& 175& 526& 263& 790& 395& 1186& 593& 1780& 890\\
445& 1336& 668& 334& 167& 502& 251& 754& 377& 1132\\
566& 283& 850& 425& 1276& 638& 319& 958& 479& 1438\\
719& 2158& 1079& 3238& 1619& 4858& 2429& 7288& 3644& 1822\\
911& 2734& 1367& 4102& 2051& 6154& 3077& 9232& 4616& 2308\\
1154& 577& 1732& 866& 433& 1300& 650& 325& 976& 488\\
244& 122& 61& 184& 92& 46& 23& 70& 35& 106\\
53& 160& 80& 40& 20& 10& 5& 16& 8& 4\\
2& 1& \\

1235&&&&&&&&&\\
3706& 1853& 5560& 2780& 1390& 695& 2086& 1043& 3130& 1565\\
4696& 2348& 1174& 587& 1762& 881& 2644& 1322& 661& 1984\\
992& 496& 248& 124& 62& 31& 94& 47& 142& 71\\
214& 107& 322& 161& 484& 242& 121& 364& 182& 91\\
274& 137& 412& 206& 103& 310& 155& 466& 233& 700\\
350& 175& 526& 263& 790& 395& 1186& 593& 1780& 890\\
445& 1336& 668& 334& 167& 502& 251& 754& 377& 1132\\
566& 283& 850& 425& 1276& 638& 319& 958& 479& 1438\\
719& 2158& 1079& 3238& 1619& 4858& 2429& 7288& 3644& 1822\\
911& 2734& 1367& 4102& 2051& 6154& 3077& 9232& 4616& 2308\\
1154& 577& 1732& 866& 433& 1300& 650& 325& 976& 488\\
244& 122& 61& 184& 92& 46& 23& 70& 35& 106\\
53& 160& 80& 40& 20& 10& 5& 16& 8& 4\\
2& 1& \\

1236&&&&&&&&&\\
618& 309& 928& 464& 232& 116& 58& 29& 88& 44\\
22& 11& 34& 17& 52& 26& 13& 40& 20& 10\\
5& 16& 8& 4& 2& 1& \\

1237&&&&&&&&&\\
3712& 1856& 928& 464& 232& 116& 58& 29& 88& 44\\
22& 11& 34& 17& 52& 26& 13& 40& 20& 10\\
5& 16& 8& 4& 2& 1& \\

1238&&&&&&&&&\\
619& 1858& 929& 2788& 1394& 697& 2092& 1046& 523& 1570\\
785& 2356& 1178& 589& 1768& 884& 442& 221& 664& 332\\
166& 83& 250& 125& 376& 188& 94& 47& 142& 71\\
214& 107& 322& 161& 484& 242& 121& 364& 182& 91\\
274& 137& 412& 206& 103& 310& 155& 466& 233& 700\\
350& 175& 526& 263& 790& 395& 1186& 593& 1780& 890\\
445& 1336& 668& 334& 167& 502& 251& 754& 377& 1132\\
566& 283& 850& 425& 1276& 638& 319& 958& 479& 1438\\
719& 2158& 1079& 3238& 1619& 4858& 2429& 7288& 3644& 1822\\
911& 2734& 1367& 4102& 2051& 6154& 3077& 9232& 4616& 2308\\
1154& 577& 1732& 866& 433& 1300& 650& 325& 976& 488\\
244& 122& 61& 184& 92& 46& 23& 70& 35& 106\\
53& 160& 80& 40& 20& 10& 5& 16& 8& 4\\
2& 1& \\

1239&&&&&&&&&\\
3718& 1859& 5578& 2789& 8368& 4184& 2092& 1046& 523& 1570\\
785& 2356& 1178& 589& 1768& 884& 442& 221& 664& 332\\
166& 83& 250& 125& 376& 188& 94& 47& 142& 71\\
214& 107& 322& 161& 484& 242& 121& 364& 182& 91\\
274& 137& 412& 206& 103& 310& 155& 466& 233& 700\\
350& 175& 526& 263& 790& 395& 1186& 593& 1780& 890\\
445& 1336& 668& 334& 167& 502& 251& 754& 377& 1132\\
566& 283& 850& 425& 1276& 638& 319& 958& 479& 1438\\
719& 2158& 1079& 3238& 1619& 4858& 2429& 7288& 3644& 1822\\
911& 2734& 1367& 4102& 2051& 6154& 3077& 9232& 4616& 2308\\
1154& 577& 1732& 866& 433& 1300& 650& 325& 976& 488\\
244& 122& 61& 184& 92& 46& 23& 70& 35& 106\\
53& 160& 80& 40& 20& 10& 5& 16& 8& 4\\
2& 1& \\

1240&&&&&&&&&\\
620& 310& 155& 466& 233& 700& 350& 175& 526& 263\\
790& 395& 1186& 593& 1780& 890& 445& 1336& 668& 334\\
167& 502& 251& 754& 377& 1132& 566& 283& 850& 425\\
1276& 638& 319& 958& 479& 1438& 719& 2158& 1079& 3238\\
1619& 4858& 2429& 7288& 3644& 1822& 911& 2734& 1367& 4102\\
2051& 6154& 3077& 9232& 4616& 2308& 1154& 577& 1732& 866\\
433& 1300& 650& 325& 976& 488& 244& 122& 61& 184\\
92& 46& 23& 70& 35& 106& 53& 160& 80& 40\\
20& 10& 5& 16& 8& 4& 2& 1& \\

1241&&&&&&&&&\\
3724& 1862& 931& 2794& 1397& 4192& 2096& 1048& 524& 262\\
131& 394& 197& 592& 296& 148& 74& 37& 112& 56\\
28& 14& 7& 22& 11& 34& 17& 52& 26& 13\\
40& 20& 10& 5& 16& 8& 4& 2& 1& \\

1242&&&&&&&&&\\
621& 1864& 932& 466& 233& 700& 350& 175& 526& 263\\
790& 395& 1186& 593& 1780& 890& 445& 1336& 668& 334\\
167& 502& 251& 754& 377& 1132& 566& 283& 850& 425\\
1276& 638& 319& 958& 479& 1438& 719& 2158& 1079& 3238\\
1619& 4858& 2429& 7288& 3644& 1822& 911& 2734& 1367& 4102\\
2051& 6154& 3077& 9232& 4616& 2308& 1154& 577& 1732& 866\\
433& 1300& 650& 325& 976& 488& 244& 122& 61& 184\\
92& 46& 23& 70& 35& 106& 53& 160& 80& 40\\
20& 10& 5& 16& 8& 4& 2& 1& \\

1243&&&&&&&&&\\
3730& 1865& 5596& 2798& 1399& 4198& 2099& 6298& 3149& 9448\\
4724& 2362& 1181& 3544& 1772& 886& 443& 1330& 665& 1996\\
998& 499& 1498& 749& 2248& 1124& 562& 281& 844& 422\\
211& 634& 317& 952& 476& 238& 119& 358& 179& 538\\
269& 808& 404& 202& 101& 304& 152& 76& 38& 19\\
58& 29& 88& 44& 22& 11& 34& 17& 52& 26\\
13& 40& 20& 10& 5& 16& 8& 4& 2& 1\\

1244&&&&&&&&&\\
622& 311& 934& 467& 1402& 701& 2104& 1052& 526& 263\\
790& 395& 1186& 593& 1780& 890& 445& 1336& 668& 334\\
167& 502& 251& 754& 377& 1132& 566& 283& 850& 425\\
1276& 638& 319& 958& 479& 1438& 719& 2158& 1079& 3238\\
1619& 4858& 2429& 7288& 3644& 1822& 911& 2734& 1367& 4102\\
2051& 6154& 3077& 9232& 4616& 2308& 1154& 577& 1732& 866\\
433& 1300& 650& 325& 976& 488& 244& 122& 61& 184\\
92& 46& 23& 70& 35& 106& 53& 160& 80& 40\\
20& 10& 5& 16& 8& 4& 2& 1& \\

1245&&&&&&&&&\\
3736& 1868& 934& 467& 1402& 701& 2104& 1052& 526& 263\\
790& 395& 1186& 593& 1780& 890& 445& 1336& 668& 334\\
167& 502& 251& 754& 377& 1132& 566& 283& 850& 425\\
1276& 638& 319& 958& 479& 1438& 719& 2158& 1079& 3238\\
1619& 4858& 2429& 7288& 3644& 1822& 911& 2734& 1367& 4102\\
2051& 6154& 3077& 9232& 4616& 2308& 1154& 577& 1732& 866\\
433& 1300& 650& 325& 976& 488& 244& 122& 61& 184\\
92& 46& 23& 70& 35& 106& 53& 160& 80& 40\\
20& 10& 5& 16& 8& 4& 2& 1& \\

1246&&&&&&&&&\\
623& 1870& 935& 2806& 1403& 4210& 2105& 6316& 3158& 1579\\
4738& 2369& 7108& 3554& 1777& 5332& 2666& 1333& 4000& 2000\\
1000& 500& 250& 125& 376& 188& 94& 47& 142& 71\\
214& 107& 322& 161& 484& 242& 121& 364& 182& 91\\
274& 137& 412& 206& 103& 310& 155& 466& 233& 700\\
350& 175& 526& 263& 790& 395& 1186& 593& 1780& 890\\
445& 1336& 668& 334& 167& 502& 251& 754& 377& 1132\\
566& 283& 850& 425& 1276& 638& 319& 958& 479& 1438\\
719& 2158& 1079& 3238& 1619& 4858& 2429& 7288& 3644& 1822\\
911& 2734& 1367& 4102& 2051& 6154& 3077& 9232& 4616& 2308\\
1154& 577& 1732& 866& 433& 1300& 650& 325& 976& 488\\
244& 122& 61& 184& 92& 46& 23& 70& 35& 106\\
53& 160& 80& 40& 20& 10& 5& 16& 8& 4\\
2& 1& \\

1247&&&&&&&&&\\
3742& 1871& 5614& 2807& 8422& 4211& 12634& 6317& 18952& 9476\\
4738& 2369& 7108& 3554& 1777& 5332& 2666& 1333& 4000& 2000\\
1000& 500& 250& 125& 376& 188& 94& 47& 142& 71\\
214& 107& 322& 161& 484& 242& 121& 364& 182& 91\\
274& 137& 412& 206& 103& 310& 155& 466& 233& 700\\
350& 175& 526& 263& 790& 395& 1186& 593& 1780& 890\\
445& 1336& 668& 334& 167& 502& 251& 754& 377& 1132\\
566& 283& 850& 425& 1276& 638& 319& 958& 479& 1438\\
719& 2158& 1079& 3238& 1619& 4858& 2429& 7288& 3644& 1822\\
911& 2734& 1367& 4102& 2051& 6154& 3077& 9232& 4616& 2308\\
1154& 577& 1732& 866& 433& 1300& 650& 325& 976& 488\\
244& 122& 61& 184& 92& 46& 23& 70& 35& 106\\
53& 160& 80& 40& 20& 10& 5& 16& 8& 4\\
2& 1& \\

1248&&&&&&&&&\\
624& 312& 156& 78& 39& 118& 59& 178& 89& 268\\
134& 67& 202& 101& 304& 152& 76& 38& 19& 58\\
29& 88& 44& 22& 11& 34& 17& 52& 26& 13\\
40& 20& 10& 5& 16& 8& 4& 2& 1& \\

1249&&&&&&&&&\\
3748& 1874& 937& 2812& 1406& 703& 2110& 1055& 3166& 1583\\
4750& 2375& 7126& 3563& 10690& 5345& 16036& 8018& 4009& 12028\\
6014& 3007& 9022& 4511& 13534& 6767& 20302& 10151& 30454& 15227\\
45682& 22841& 68524& 34262& 17131& 51394& 25697& 77092& 38546& 19273\\
57820& 28910& 14455& 43366& 21683& 65050& 32525& 97576& 48788& 24394\\
12197& 36592& 18296& 9148& 4574& 2287& 6862& 3431& 10294& 5147\\
15442& 7721& 23164& 11582& 5791& 17374& 8687& 26062& 13031& 39094\\
19547& 58642& 29321& 87964& 43982& 21991& 65974& 32987& 98962& 49481\\
148444& 74222& 37111& 111334& 55667& 167002& 83501& 250504& 125252& 62626\\
31313& 93940& 46970& 23485& 70456& 35228& 17614& 8807& 26422& 13211\\
39634& 19817& 59452& 29726& 14863& 44590& 22295& 66886& 33443& 100330\\
50165& 150496& 75248& 37624& 18812& 9406& 4703& 14110& 7055& 21166\\
10583& 31750& 15875& 47626& 23813& 71440& 35720& 17860& 8930& 4465\\
13396& 6698& 3349& 10048& 5024& 2512& 1256& 628& 314& 157\\
472& 236& 118& 59& 178& 89& 268& 134& 67& 202\\
101& 304& 152& 76& 38& 19& 58& 29& 88& 44\\
22& 11& 34& 17& 52& 26& 13& 40& 20& 10\\
5& 16& 8& 4& 2& 1& \\

1250&&&&&&&&&\\
625& 1876& 938& 469& 1408& 704& 352& 176& 88& 44\\
22& 11& 34& 17& 52& 26& 13& 40& 20& 10\\
5& 16& 8& 4& 2& 1& \\

1251&&&&&&&&&\\
3754& 1877& 5632& 2816& 1408& 704& 352& 176& 88& 44\\
22& 11& 34& 17& 52& 26& 13& 40& 20& 10\\
5& 16& 8& 4& 2& 1& \\

1252&&&&&&&&&\\
626& 313& 940& 470& 235& 706& 353& 1060& 530& 265\\
796& 398& 199& 598& 299& 898& 449& 1348& 674& 337\\
1012& 506& 253& 760& 380& 190& 95& 286& 143& 430\\
215& 646& 323& 970& 485& 1456& 728& 364& 182& 91\\
274& 137& 412& 206& 103& 310& 155& 466& 233& 700\\
350& 175& 526& 263& 790& 395& 1186& 593& 1780& 890\\
445& 1336& 668& 334& 167& 502& 251& 754& 377& 1132\\
566& 283& 850& 425& 1276& 638& 319& 958& 479& 1438\\
719& 2158& 1079& 3238& 1619& 4858& 2429& 7288& 3644& 1822\\
911& 2734& 1367& 4102& 2051& 6154& 3077& 9232& 4616& 2308\\
1154& 577& 1732& 866& 433& 1300& 650& 325& 976& 488\\
244& 122& 61& 184& 92& 46& 23& 70& 35& 106\\
53& 160& 80& 40& 20& 10& 5& 16& 8& 4\\
2& 1& \\

1253&&&&&&&&&\\
3760& 1880& 940& 470& 235& 706& 353& 1060& 530& 265\\
796& 398& 199& 598& 299& 898& 449& 1348& 674& 337\\
1012& 506& 253& 760& 380& 190& 95& 286& 143& 430\\
215& 646& 323& 970& 485& 1456& 728& 364& 182& 91\\
274& 137& 412& 206& 103& 310& 155& 466& 233& 700\\
350& 175& 526& 263& 790& 395& 1186& 593& 1780& 890\\
445& 1336& 668& 334& 167& 502& 251& 754& 377& 1132\\
566& 283& 850& 425& 1276& 638& 319& 958& 479& 1438\\
719& 2158& 1079& 3238& 1619& 4858& 2429& 7288& 3644& 1822\\
911& 2734& 1367& 4102& 2051& 6154& 3077& 9232& 4616& 2308\\
1154& 577& 1732& 866& 433& 1300& 650& 325& 976& 488\\
244& 122& 61& 184& 92& 46& 23& 70& 35& 106\\
53& 160& 80& 40& 20& 10& 5& 16& 8& 4\\
2& 1& \\

1254&&&&&&&&&\\
627& 1882& 941& 2824& 1412& 706& 353& 1060& 530& 265\\
796& 398& 199& 598& 299& 898& 449& 1348& 674& 337\\
1012& 506& 253& 760& 380& 190& 95& 286& 143& 430\\
215& 646& 323& 970& 485& 1456& 728& 364& 182& 91\\
274& 137& 412& 206& 103& 310& 155& 466& 233& 700\\
350& 175& 526& 263& 790& 395& 1186& 593& 1780& 890\\
445& 1336& 668& 334& 167& 502& 251& 754& 377& 1132\\
566& 283& 850& 425& 1276& 638& 319& 958& 479& 1438\\
719& 2158& 1079& 3238& 1619& 4858& 2429& 7288& 3644& 1822\\
911& 2734& 1367& 4102& 2051& 6154& 3077& 9232& 4616& 2308\\
1154& 577& 1732& 866& 433& 1300& 650& 325& 976& 488\\
244& 122& 61& 184& 92& 46& 23& 70& 35& 106\\
53& 160& 80& 40& 20& 10& 5& 16& 8& 4\\
2& 1& \\

1255&&&&&&&&&\\
3766& 1883& 5650& 2825& 8476& 4238& 2119& 6358& 3179& 9538\\
4769& 14308& 7154& 3577& 10732& 5366& 2683& 8050& 4025& 12076\\
6038& 3019& 9058& 4529& 13588& 6794& 3397& 10192& 5096& 2548\\
1274& 637& 1912& 956& 478& 239& 718& 359& 1078& 539\\
1618& 809& 2428& 1214& 607& 1822& 911& 2734& 1367& 4102\\
2051& 6154& 3077& 9232& 4616& 2308& 1154& 577& 1732& 866\\
433& 1300& 650& 325& 976& 488& 244& 122& 61& 184\\
92& 46& 23& 70& 35& 106& 53& 160& 80& 40\\
20& 10& 5& 16& 8& 4& 2& 1& \\

1256&&&&&&&&&\\
628& 314& 157& 472& 236& 118& 59& 178& 89& 268\\
134& 67& 202& 101& 304& 152& 76& 38& 19& 58\\
29& 88& 44& 22& 11& 34& 17& 52& 26& 13\\
40& 20& 10& 5& 16& 8& 4& 2& 1& \\

1257&&&&&&&&&\\
3772& 1886& 943& 2830& 1415& 4246& 2123& 6370& 3185& 9556\\
4778& 2389& 7168& 3584& 1792& 896& 448& 224& 112& 56\\
28& 14& 7& 22& 11& 34& 17& 52& 26& 13\\
40& 20& 10& 5& 16& 8& 4& 2& 1& \\

1258&&&&&&&&&\\
629& 1888& 944& 472& 236& 118& 59& 178& 89& 268\\
134& 67& 202& 101& 304& 152& 76& 38& 19& 58\\
29& 88& 44& 22& 11& 34& 17& 52& 26& 13\\
40& 20& 10& 5& 16& 8& 4& 2& 1& \\

1259&&&&&&&&&\\
3778& 1889& 5668& 2834& 1417& 4252& 2126& 1063& 3190& 1595\\
4786& 2393& 7180& 3590& 1795& 5386& 2693& 8080& 4040& 2020\\
1010& 505& 1516& 758& 379& 1138& 569& 1708& 854& 427\\
1282& 641& 1924& 962& 481& 1444& 722& 361& 1084& 542\\
271& 814& 407& 1222& 611& 1834& 917& 2752& 1376& 688\\
344& 172& 86& 43& 130& 65& 196& 98& 49& 148\\
74& 37& 112& 56& 28& 14& 7& 22& 11& 34\\
17& 52& 26& 13& 40& 20& 10& 5& 16& 8\\
4& 2& 1& \\

1260&&&&&&&&&\\
630& 315& 946& 473& 1420& 710& 355& 1066& 533& 1600\\
800& 400& 200& 100& 50& 25& 76& 38& 19& 58\\
29& 88& 44& 22& 11& 34& 17& 52& 26& 13\\
40& 20& 10& 5& 16& 8& 4& 2& 1& \\

1261&&&&&&&&&\\
3784& 1892& 946& 473& 1420& 710& 355& 1066& 533& 1600\\
800& 400& 200& 100& 50& 25& 76& 38& 19& 58\\
29& 88& 44& 22& 11& 34& 17& 52& 26& 13\\
40& 20& 10& 5& 16& 8& 4& 2& 1& \\

1262&&&&&&&&&\\
631& 1894& 947& 2842& 1421& 4264& 2132& 1066& 533& 1600\\
800& 400& 200& 100& 50& 25& 76& 38& 19& 58\\
29& 88& 44& 22& 11& 34& 17& 52& 26& 13\\
40& 20& 10& 5& 16& 8& 4& 2& 1& \\

1263&&&&&&&&&\\
3790& 1895& 5686& 2843& 8530& 4265& 12796& 6398& 3199& 9598\\
4799& 14398& 7199& 21598& 10799& 32398& 16199& 48598& 24299& 72898\\
36449& 109348& 54674& 27337& 82012& 41006& 20503& 61510& 30755& 92266\\
46133& 138400& 69200& 34600& 17300& 8650& 4325& 12976& 6488& 3244\\
1622& 811& 2434& 1217& 3652& 1826& 913& 2740& 1370& 685\\
2056& 1028& 514& 257& 772& 386& 193& 580& 290& 145\\
436& 218& 109& 328& 164& 82& 41& 124& 62& 31\\
94& 47& 142& 71& 214& 107& 322& 161& 484& 242\\
121& 364& 182& 91& 274& 137& 412& 206& 103& 310\\
155& 466& 233& 700& 350& 175& 526& 263& 790& 395\\
1186& 593& 1780& 890& 445& 1336& 668& 334& 167& 502\\
251& 754& 377& 1132& 566& 283& 850& 425& 1276& 638\\
319& 958& 479& 1438& 719& 2158& 1079& 3238& 1619& 4858\\
2429& 7288& 3644& 1822& 911& 2734& 1367& 4102& 2051& 6154\\
3077& 9232& 4616& 2308& 1154& 577& 1732& 866& 433& 1300\\
650& 325& 976& 488& 244& 122& 61& 184& 92& 46\\
23& 70& 35& 106& 53& 160& 80& 40& 20& 10\\
5& 16& 8& 4& 2& 1& \\

1264&&&&&&&&&\\
632& 316& 158& 79& 238& 119& 358& 179& 538& 269\\
808& 404& 202& 101& 304& 152& 76& 38& 19& 58\\
29& 88& 44& 22& 11& 34& 17& 52& 26& 13\\
40& 20& 10& 5& 16& 8& 4& 2& 1& \\

1265&&&&&&&&&\\
3796& 1898& 949& 2848& 1424& 712& 356& 178& 89& 268\\
134& 67& 202& 101& 304& 152& 76& 38& 19& 58\\
29& 88& 44& 22& 11& 34& 17& 52& 26& 13\\
40& 20& 10& 5& 16& 8& 4& 2& 1& \\

1266&&&&&&&&&\\
633& 1900& 950& 475& 1426& 713& 2140& 1070& 535& 1606\\
803& 2410& 1205& 3616& 1808& 904& 452& 226& 113& 340\\
170& 85& 256& 128& 64& 32& 16& 8& 4& 2\\
1& \\

1267&&&&&&&&&\\
3802& 1901& 5704& 2852& 1426& 713& 2140& 1070& 535& 1606\\
803& 2410& 1205& 3616& 1808& 904& 452& 226& 113& 340\\
170& 85& 256& 128& 64& 32& 16& 8& 4& 2\\
1& \\

1268&&&&&&&&&\\
634& 317& 952& 476& 238& 119& 358& 179& 538& 269\\
808& 404& 202& 101& 304& 152& 76& 38& 19& 58\\
29& 88& 44& 22& 11& 34& 17& 52& 26& 13\\
40& 20& 10& 5& 16& 8& 4& 2& 1& \\

1269&&&&&&&&&\\
3808& 1904& 952& 476& 238& 119& 358& 179& 538& 269\\
808& 404& 202& 101& 304& 152& 76& 38& 19& 58\\
29& 88& 44& 22& 11& 34& 17& 52& 26& 13\\
40& 20& 10& 5& 16& 8& 4& 2& 1& \\

1270&&&&&&&&&\\
635& 1906& 953& 2860& 1430& 715& 2146& 1073& 3220& 1610\\
805& 2416& 1208& 604& 302& 151& 454& 227& 682& 341\\
1024& 512& 256& 128& 64& 32& 16& 8& 4& 2\\
1& \\

1271&&&&&&&&&\\
3814& 1907& 5722& 2861& 8584& 4292& 2146& 1073& 3220& 1610\\
805& 2416& 1208& 604& 302& 151& 454& 227& 682& 341\\
1024& 512& 256& 128& 64& 32& 16& 8& 4& 2\\
1& \\

1272&&&&&&&&&\\
636& 318& 159& 478& 239& 718& 359& 1078& 539& 1618\\
809& 2428& 1214& 607& 1822& 911& 2734& 1367& 4102& 2051\\
6154& 3077& 9232& 4616& 2308& 1154& 577& 1732& 866& 433\\
1300& 650& 325& 976& 488& 244& 122& 61& 184& 92\\
46& 23& 70& 35& 106& 53& 160& 80& 40& 20\\
10& 5& 16& 8& 4& 2& 1& \\

1273&&&&&&&&&\\
3820& 1910& 955& 2866& 1433& 4300& 2150& 1075& 3226& 1613\\
4840& 2420& 1210& 605& 1816& 908& 454& 227& 682& 341\\
1024& 512& 256& 128& 64& 32& 16& 8& 4& 2\\
1& \\

1274&&&&&&&&&\\
637& 1912& 956& 478& 239& 718& 359& 1078& 539& 1618\\
809& 2428& 1214& 607& 1822& 911& 2734& 1367& 4102& 2051\\
6154& 3077& 9232& 4616& 2308& 1154& 577& 1732& 866& 433\\
1300& 650& 325& 976& 488& 244& 122& 61& 184& 92\\
46& 23& 70& 35& 106& 53& 160& 80& 40& 20\\
10& 5& 16& 8& 4& 2& 1& \\

1275&&&&&&&&&\\
3826& 1913& 5740& 2870& 1435& 4306& 2153& 6460& 3230& 1615\\
4846& 2423& 7270& 3635& 10906& 5453& 16360& 8180& 4090& 2045\\
6136& 3068& 1534& 767& 2302& 1151& 3454& 1727& 5182& 2591\\
7774& 3887& 11662& 5831& 17494& 8747& 26242& 13121& 39364& 19682\\
9841& 29524& 14762& 7381& 22144& 11072& 5536& 2768& 1384& 692\\
346& 173& 520& 260& 130& 65& 196& 98& 49& 148\\
74& 37& 112& 56& 28& 14& 7& 22& 11& 34\\
17& 52& 26& 13& 40& 20& 10& 5& 16& 8\\
4& 2& 1& \\

1276&&&&&&&&&\\
638& 319& 958& 479& 1438& 719& 2158& 1079& 3238& 1619\\
4858& 2429& 7288& 3644& 1822& 911& 2734& 1367& 4102& 2051\\
6154& 3077& 9232& 4616& 2308& 1154& 577& 1732& 866& 433\\
1300& 650& 325& 976& 488& 244& 122& 61& 184& 92\\
46& 23& 70& 35& 106& 53& 160& 80& 40& 20\\
10& 5& 16& 8& 4& 2& 1& \\

1277&&&&&&&&&\\
3832& 1916& 958& 479& 1438& 719& 2158& 1079& 3238& 1619\\
4858& 2429& 7288& 3644& 1822& 911& 2734& 1367& 4102& 2051\\
6154& 3077& 9232& 4616& 2308& 1154& 577& 1732& 866& 433\\
1300& 650& 325& 976& 488& 244& 122& 61& 184& 92\\
46& 23& 70& 35& 106& 53& 160& 80& 40& 20\\
10& 5& 16& 8& 4& 2& 1& \\

1278&&&&&&&&&\\
639& 1918& 959& 2878& 1439& 4318& 2159& 6478& 3239& 9718\\
4859& 14578& 7289& 21868& 10934& 5467& 16402& 8201& 24604& 12302\\
6151& 18454& 9227& 27682& 13841& 41524& 20762& 10381& 31144& 15572\\
7786& 3893& 11680& 5840& 2920& 1460& 730& 365& 1096& 548\\
274& 137& 412& 206& 103& 310& 155& 466& 233& 700\\
350& 175& 526& 263& 790& 395& 1186& 593& 1780& 890\\
445& 1336& 668& 334& 167& 502& 251& 754& 377& 1132\\
566& 283& 850& 425& 1276& 638& 319& 958& 479& 1438\\
719& 2158& 1079& 3238& 1619& 4858& 2429& 7288& 3644& 1822\\
911& 2734& 1367& 4102& 2051& 6154& 3077& 9232& 4616& 2308\\
1154& 577& 1732& 866& 433& 1300& 650& 325& 976& 488\\
244& 122& 61& 184& 92& 46& 23& 70& 35& 106\\
53& 160& 80& 40& 20& 10& 5& 16& 8& 4\\
2& 1& \\

1279&&&&&&&&&\\
3838& 1919& 5758& 2879& 8638& 4319& 12958& 6479& 19438& 9719\\
29158& 14579& 43738& 21869& 65608& 32804& 16402& 8201& 24604& 12302\\
6151& 18454& 9227& 27682& 13841& 41524& 20762& 10381& 31144& 15572\\
7786& 3893& 11680& 5840& 2920& 1460& 730& 365& 1096& 548\\
274& 137& 412& 206& 103& 310& 155& 466& 233& 700\\
350& 175& 526& 263& 790& 395& 1186& 593& 1780& 890\\
445& 1336& 668& 334& 167& 502& 251& 754& 377& 1132\\
566& 283& 850& 425& 1276& 638& 319& 958& 479& 1438\\
719& 2158& 1079& 3238& 1619& 4858& 2429& 7288& 3644& 1822\\
911& 2734& 1367& 4102& 2051& 6154& 3077& 9232& 4616& 2308\\
1154& 577& 1732& 866& 433& 1300& 650& 325& 976& 488\\
244& 122& 61& 184& 92& 46& 23& 70& 35& 106\\
53& 160& 80& 40& 20& 10& 5& 16& 8& 4\\
2& 1& \\

1280&&&&&&&&&\\
640& 320& 160& 80& 40& 20& 10& 5& 16& 8\\
4& 2& 1& \\

1281&&&&&&&&&\\
3844& 1922& 961& 2884& 1442& 721& 2164& 1082& 541& 1624\\
812& 406& 203& 610& 305& 916& 458& 229& 688& 344\\
172& 86& 43& 130& 65& 196& 98& 49& 148& 74\\
37& 112& 56& 28& 14& 7& 22& 11& 34& 17\\
52& 26& 13& 40& 20& 10& 5& 16& 8& 4\\
2& 1& \\

1282&&&&&&&&&\\
641& 1924& 962& 481& 1444& 722& 361& 1084& 542& 271\\
814& 407& 1222& 611& 1834& 917& 2752& 1376& 688& 344\\
172& 86& 43& 130& 65& 196& 98& 49& 148& 74\\
37& 112& 56& 28& 14& 7& 22& 11& 34& 17\\
52& 26& 13& 40& 20& 10& 5& 16& 8& 4\\
2& 1& \\

1283&&&&&&&&&\\
3850& 1925& 5776& 2888& 1444& 722& 361& 1084& 542& 271\\
814& 407& 1222& 611& 1834& 917& 2752& 1376& 688& 344\\
172& 86& 43& 130& 65& 196& 98& 49& 148& 74\\
37& 112& 56& 28& 14& 7& 22& 11& 34& 17\\
52& 26& 13& 40& 20& 10& 5& 16& 8& 4\\
2& 1& \\

1284&&&&&&&&&\\
642& 321& 964& 482& 241& 724& 362& 181& 544& 272\\
136& 68& 34& 17& 52& 26& 13& 40& 20& 10\\
5& 16& 8& 4& 2& 1& \\

1285&&&&&&&&&\\
3856& 1928& 964& 482& 241& 724& 362& 181& 544& 272\\
136& 68& 34& 17& 52& 26& 13& 40& 20& 10\\
5& 16& 8& 4& 2& 1& \\

1286&&&&&&&&&\\
643& 1930& 965& 2896& 1448& 724& 362& 181& 544& 272\\
136& 68& 34& 17& 52& 26& 13& 40& 20& 10\\
5& 16& 8& 4& 2& 1& \\

1287&&&&&&&&&\\
3862& 1931& 5794& 2897& 8692& 4346& 2173& 6520& 3260& 1630\\
815& 2446& 1223& 3670& 1835& 5506& 2753& 8260& 4130& 2065\\
6196& 3098& 1549& 4648& 2324& 1162& 581& 1744& 872& 436\\
218& 109& 328& 164& 82& 41& 124& 62& 31& 94\\
47& 142& 71& 214& 107& 322& 161& 484& 242& 121\\
364& 182& 91& 274& 137& 412& 206& 103& 310& 155\\
466& 233& 700& 350& 175& 526& 263& 790& 395& 1186\\
593& 1780& 890& 445& 1336& 668& 334& 167& 502& 251\\
754& 377& 1132& 566& 283& 850& 425& 1276& 638& 319\\
958& 479& 1438& 719& 2158& 1079& 3238& 1619& 4858& 2429\\
7288& 3644& 1822& 911& 2734& 1367& 4102& 2051& 6154& 3077\\
9232& 4616& 2308& 1154& 577& 1732& 866& 433& 1300& 650\\
325& 976& 488& 244& 122& 61& 184& 92& 46& 23\\
70& 35& 106& 53& 160& 80& 40& 20& 10& 5\\
16& 8& 4& 2& 1& \\

1288&&&&&&&&&\\
644& 322& 161& 484& 242& 121& 364& 182& 91& 274\\
137& 412& 206& 103& 310& 155& 466& 233& 700& 350\\
175& 526& 263& 790& 395& 1186& 593& 1780& 890& 445\\
1336& 668& 334& 167& 502& 251& 754& 377& 1132& 566\\
283& 850& 425& 1276& 638& 319& 958& 479& 1438& 719\\
2158& 1079& 3238& 1619& 4858& 2429& 7288& 3644& 1822& 911\\
2734& 1367& 4102& 2051& 6154& 3077& 9232& 4616& 2308& 1154\\
577& 1732& 866& 433& 1300& 650& 325& 976& 488& 244\\
122& 61& 184& 92& 46& 23& 70& 35& 106& 53\\
160& 80& 40& 20& 10& 5& 16& 8& 4& 2\\
1& \\

1289&&&&&&&&&\\
3868& 1934& 967& 2902& 1451& 4354& 2177& 6532& 3266& 1633\\
4900& 2450& 1225& 3676& 1838& 919& 2758& 1379& 4138& 2069\\
6208& 3104& 1552& 776& 388& 194& 97& 292& 146& 73\\
220& 110& 55& 166& 83& 250& 125& 376& 188& 94\\
47& 142& 71& 214& 107& 322& 161& 484& 242& 121\\
364& 182& 91& 274& 137& 412& 206& 103& 310& 155\\
466& 233& 700& 350& 175& 526& 263& 790& 395& 1186\\
593& 1780& 890& 445& 1336& 668& 334& 167& 502& 251\\
754& 377& 1132& 566& 283& 850& 425& 1276& 638& 319\\
958& 479& 1438& 719& 2158& 1079& 3238& 1619& 4858& 2429\\
7288& 3644& 1822& 911& 2734& 1367& 4102& 2051& 6154& 3077\\
9232& 4616& 2308& 1154& 577& 1732& 866& 433& 1300& 650\\
325& 976& 488& 244& 122& 61& 184& 92& 46& 23\\
70& 35& 106& 53& 160& 80& 40& 20& 10& 5\\
16& 8& 4& 2& 1& \\

1290&&&&&&&&&\\
645& 1936& 968& 484& 242& 121& 364& 182& 91& 274\\
137& 412& 206& 103& 310& 155& 466& 233& 700& 350\\
175& 526& 263& 790& 395& 1186& 593& 1780& 890& 445\\
1336& 668& 334& 167& 502& 251& 754& 377& 1132& 566\\
283& 850& 425& 1276& 638& 319& 958& 479& 1438& 719\\
2158& 1079& 3238& 1619& 4858& 2429& 7288& 3644& 1822& 911\\
2734& 1367& 4102& 2051& 6154& 3077& 9232& 4616& 2308& 1154\\
577& 1732& 866& 433& 1300& 650& 325& 976& 488& 244\\
122& 61& 184& 92& 46& 23& 70& 35& 106& 53\\
160& 80& 40& 20& 10& 5& 16& 8& 4& 2\\
1& \\

1291&&&&&&&&&\\
3874& 1937& 5812& 2906& 1453& 4360& 2180& 1090& 545& 1636\\
818& 409& 1228& 614& 307& 922& 461& 1384& 692& 346\\
173& 520& 260& 130& 65& 196& 98& 49& 148& 74\\
37& 112& 56& 28& 14& 7& 22& 11& 34& 17\\
52& 26& 13& 40& 20& 10& 5& 16& 8& 4\\
2& 1& \\

1292&&&&&&&&&\\
646& 323& 970& 485& 1456& 728& 364& 182& 91& 274\\
137& 412& 206& 103& 310& 155& 466& 233& 700& 350\\
175& 526& 263& 790& 395& 1186& 593& 1780& 890& 445\\
1336& 668& 334& 167& 502& 251& 754& 377& 1132& 566\\
283& 850& 425& 1276& 638& 319& 958& 479& 1438& 719\\
2158& 1079& 3238& 1619& 4858& 2429& 7288& 3644& 1822& 911\\
2734& 1367& 4102& 2051& 6154& 3077& 9232& 4616& 2308& 1154\\
577& 1732& 866& 433& 1300& 650& 325& 976& 488& 244\\
122& 61& 184& 92& 46& 23& 70& 35& 106& 53\\
160& 80& 40& 20& 10& 5& 16& 8& 4& 2\\
1& \\

1293&&&&&&&&&\\
3880& 1940& 970& 485& 1456& 728& 364& 182& 91& 274\\
137& 412& 206& 103& 310& 155& 466& 233& 700& 350\\
175& 526& 263& 790& 395& 1186& 593& 1780& 890& 445\\
1336& 668& 334& 167& 502& 251& 754& 377& 1132& 566\\
283& 850& 425& 1276& 638& 319& 958& 479& 1438& 719\\
2158& 1079& 3238& 1619& 4858& 2429& 7288& 3644& 1822& 911\\
2734& 1367& 4102& 2051& 6154& 3077& 9232& 4616& 2308& 1154\\
577& 1732& 866& 433& 1300& 650& 325& 976& 488& 244\\
122& 61& 184& 92& 46& 23& 70& 35& 106& 53\\
160& 80& 40& 20& 10& 5& 16& 8& 4& 2\\
1& \\

1294&&&&&&&&&\\
647& 1942& 971& 2914& 1457& 4372& 2186& 1093& 3280& 1640\\
820& 410& 205& 616& 308& 154& 77& 232& 116& 58\\
29& 88& 44& 22& 11& 34& 17& 52& 26& 13\\
40& 20& 10& 5& 16& 8& 4& 2& 1& \\

1295&&&&&&&&&\\
3886& 1943& 5830& 2915& 8746& 4373& 13120& 6560& 3280& 1640\\
820& 410& 205& 616& 308& 154& 77& 232& 116& 58\\
29& 88& 44& 22& 11& 34& 17& 52& 26& 13\\
40& 20& 10& 5& 16& 8& 4& 2& 1& \\

1296&&&&&&&&&\\
648& 324& 162& 81& 244& 122& 61& 184& 92& 46\\
23& 70& 35& 106& 53& 160& 80& 40& 20& 10\\
5& 16& 8& 4& 2& 1& \\

1297&&&&&&&&&\\
3892& 1946& 973& 2920& 1460& 730& 365& 1096& 548& 274\\
137& 412& 206& 103& 310& 155& 466& 233& 700& 350\\
175& 526& 263& 790& 395& 1186& 593& 1780& 890& 445\\
1336& 668& 334& 167& 502& 251& 754& 377& 1132& 566\\
283& 850& 425& 1276& 638& 319& 958& 479& 1438& 719\\
2158& 1079& 3238& 1619& 4858& 2429& 7288& 3644& 1822& 911\\
2734& 1367& 4102& 2051& 6154& 3077& 9232& 4616& 2308& 1154\\
577& 1732& 866& 433& 1300& 650& 325& 976& 488& 244\\
122& 61& 184& 92& 46& 23& 70& 35& 106& 53\\
160& 80& 40& 20& 10& 5& 16& 8& 4& 2\\
1& \\

1298&&&&&&&&&\\
649& 1948& 974& 487& 1462& 731& 2194& 1097& 3292& 1646\\
823& 2470& 1235& 3706& 1853& 5560& 2780& 1390& 695& 2086\\
1043& 3130& 1565& 4696& 2348& 1174& 587& 1762& 881& 2644\\
1322& 661& 1984& 992& 496& 248& 124& 62& 31& 94\\
47& 142& 71& 214& 107& 322& 161& 484& 242& 121\\
364& 182& 91& 274& 137& 412& 206& 103& 310& 155\\
466& 233& 700& 350& 175& 526& 263& 790& 395& 1186\\
593& 1780& 890& 445& 1336& 668& 334& 167& 502& 251\\
754& 377& 1132& 566& 283& 850& 425& 1276& 638& 319\\
958& 479& 1438& 719& 2158& 1079& 3238& 1619& 4858& 2429\\
7288& 3644& 1822& 911& 2734& 1367& 4102& 2051& 6154& 3077\\
9232& 4616& 2308& 1154& 577& 1732& 866& 433& 1300& 650\\
325& 976& 488& 244& 122& 61& 184& 92& 46& 23\\
70& 35& 106& 53& 160& 80& 40& 20& 10& 5\\
16& 8& 4& 2& 1& \\

1299&&&&&&&&&\\
3898& 1949& 5848& 2924& 1462& 731& 2194& 1097& 3292& 1646\\
823& 2470& 1235& 3706& 1853& 5560& 2780& 1390& 695& 2086\\
1043& 3130& 1565& 4696& 2348& 1174& 587& 1762& 881& 2644\\
1322& 661& 1984& 992& 496& 248& 124& 62& 31& 94\\
47& 142& 71& 214& 107& 322& 161& 484& 242& 121\\
364& 182& 91& 274& 137& 412& 206& 103& 310& 155\\
466& 233& 700& 350& 175& 526& 263& 790& 395& 1186\\
593& 1780& 890& 445& 1336& 668& 334& 167& 502& 251\\
754& 377& 1132& 566& 283& 850& 425& 1276& 638& 319\\
958& 479& 1438& 719& 2158& 1079& 3238& 1619& 4858& 2429\\
7288& 3644& 1822& 911& 2734& 1367& 4102& 2051& 6154& 3077\\
9232& 4616& 2308& 1154& 577& 1732& 866& 433& 1300& 650\\
325& 976& 488& 244& 122& 61& 184& 92& 46& 23\\
70& 35& 106& 53& 160& 80& 40& 20& 10& 5\\
16& 8& 4& 2& 1& \\

1300&&&&&&&&&\\
650& 325& 976& 488& 244& 122& 61& 184& 92& 46\\
23& 70& 35& 106& 53& 160& 80& 40& 20& 10\\
5& 16& 8& 4& 2& 1& \\

1301&&&&&&&&&\\
3904& 1952& 976& 488& 244& 122& 61& 184& 92& 46\\
23& 70& 35& 106& 53& 160& 80& 40& 20& 10\\
5& 16& 8& 4& 2& 1& \\

1302&&&&&&&&&\\
651& 1954& 977& 2932& 1466& 733& 2200& 1100& 550& 275\\
826& 413& 1240& 620& 310& 155& 466& 233& 700& 350\\
175& 526& 263& 790& 395& 1186& 593& 1780& 890& 445\\
1336& 668& 334& 167& 502& 251& 754& 377& 1132& 566\\
283& 850& 425& 1276& 638& 319& 958& 479& 1438& 719\\
2158& 1079& 3238& 1619& 4858& 2429& 7288& 3644& 1822& 911\\
2734& 1367& 4102& 2051& 6154& 3077& 9232& 4616& 2308& 1154\\
577& 1732& 866& 433& 1300& 650& 325& 976& 488& 244\\
122& 61& 184& 92& 46& 23& 70& 35& 106& 53\\
160& 80& 40& 20& 10& 5& 16& 8& 4& 2\\
1& \\

1303&&&&&&&&&\\
3910& 1955& 5866& 2933& 8800& 4400& 2200& 1100& 550& 275\\
826& 413& 1240& 620& 310& 155& 466& 233& 700& 350\\
175& 526& 263& 790& 395& 1186& 593& 1780& 890& 445\\
1336& 668& 334& 167& 502& 251& 754& 377& 1132& 566\\
283& 850& 425& 1276& 638& 319& 958& 479& 1438& 719\\
2158& 1079& 3238& 1619& 4858& 2429& 7288& 3644& 1822& 911\\
2734& 1367& 4102& 2051& 6154& 3077& 9232& 4616& 2308& 1154\\
577& 1732& 866& 433& 1300& 650& 325& 976& 488& 244\\
122& 61& 184& 92& 46& 23& 70& 35& 106& 53\\
160& 80& 40& 20& 10& 5& 16& 8& 4& 2\\
1& \\

1304&&&&&&&&&\\
652& 326& 163& 490& 245& 736& 368& 184& 92& 46\\
23& 70& 35& 106& 53& 160& 80& 40& 20& 10\\
5& 16& 8& 4& 2& 1& \\

1305&&&&&&&&&\\
3916& 1958& 979& 2938& 1469& 4408& 2204& 1102& 551& 1654\\
827& 2482& 1241& 3724& 1862& 931& 2794& 1397& 4192& 2096\\
1048& 524& 262& 131& 394& 197& 592& 296& 148& 74\\
37& 112& 56& 28& 14& 7& 22& 11& 34& 17\\
52& 26& 13& 40& 20& 10& 5& 16& 8& 4\\
2& 1& \\

1306&&&&&&&&&\\
653& 1960& 980& 490& 245& 736& 368& 184& 92& 46\\
23& 70& 35& 106& 53& 160& 80& 40& 20& 10\\
5& 16& 8& 4& 2& 1& \\

1307&&&&&&&&&\\
3922& 1961& 5884& 2942& 1471& 4414& 2207& 6622& 3311& 9934\\
4967& 14902& 7451& 22354& 11177& 33532& 16766& 8383& 25150& 12575\\
37726& 18863& 56590& 28295& 84886& 42443& 127330& 63665& 190996& 95498\\
47749& 143248& 71624& 35812& 17906& 8953& 26860& 13430& 6715& 20146\\
10073& 30220& 15110& 7555& 22666& 11333& 34000& 17000& 8500& 4250\\
2125& 6376& 3188& 1594& 797& 2392& 1196& 598& 299& 898\\
449& 1348& 674& 337& 1012& 506& 253& 760& 380& 190\\
95& 286& 143& 430& 215& 646& 323& 970& 485& 1456\\
728& 364& 182& 91& 274& 137& 412& 206& 103& 310\\
155& 466& 233& 700& 350& 175& 526& 263& 790& 395\\
1186& 593& 1780& 890& 445& 1336& 668& 334& 167& 502\\
251& 754& 377& 1132& 566& 283& 850& 425& 1276& 638\\
319& 958& 479& 1438& 719& 2158& 1079& 3238& 1619& 4858\\
2429& 7288& 3644& 1822& 911& 2734& 1367& 4102& 2051& 6154\\
3077& 9232& 4616& 2308& 1154& 577& 1732& 866& 433& 1300\\
650& 325& 976& 488& 244& 122& 61& 184& 92& 46\\
23& 70& 35& 106& 53& 160& 80& 40& 20& 10\\
5& 16& 8& 4& 2& 1& \\

1308&&&&&&&&&\\
654& 327& 982& 491& 1474& 737& 2212& 1106& 553& 1660\\
830& 415& 1246& 623& 1870& 935& 2806& 1403& 4210& 2105\\
6316& 3158& 1579& 4738& 2369& 7108& 3554& 1777& 5332& 2666\\
1333& 4000& 2000& 1000& 500& 250& 125& 376& 188& 94\\
47& 142& 71& 214& 107& 322& 161& 484& 242& 121\\
364& 182& 91& 274& 137& 412& 206& 103& 310& 155\\
466& 233& 700& 350& 175& 526& 263& 790& 395& 1186\\
593& 1780& 890& 445& 1336& 668& 334& 167& 502& 251\\
754& 377& 1132& 566& 283& 850& 425& 1276& 638& 319\\
958& 479& 1438& 719& 2158& 1079& 3238& 1619& 4858& 2429\\
7288& 3644& 1822& 911& 2734& 1367& 4102& 2051& 6154& 3077\\
9232& 4616& 2308& 1154& 577& 1732& 866& 433& 1300& 650\\
325& 976& 488& 244& 122& 61& 184& 92& 46& 23\\
70& 35& 106& 53& 160& 80& 40& 20& 10& 5\\
16& 8& 4& 2& 1& \\

1309&&&&&&&&&\\
3928& 1964& 982& 491& 1474& 737& 2212& 1106& 553& 1660\\
830& 415& 1246& 623& 1870& 935& 2806& 1403& 4210& 2105\\
6316& 3158& 1579& 4738& 2369& 7108& 3554& 1777& 5332& 2666\\
1333& 4000& 2000& 1000& 500& 250& 125& 376& 188& 94\\
47& 142& 71& 214& 107& 322& 161& 484& 242& 121\\
364& 182& 91& 274& 137& 412& 206& 103& 310& 155\\
466& 233& 700& 350& 175& 526& 263& 790& 395& 1186\\
593& 1780& 890& 445& 1336& 668& 334& 167& 502& 251\\
754& 377& 1132& 566& 283& 850& 425& 1276& 638& 319\\
958& 479& 1438& 719& 2158& 1079& 3238& 1619& 4858& 2429\\
7288& 3644& 1822& 911& 2734& 1367& 4102& 2051& 6154& 3077\\
9232& 4616& 2308& 1154& 577& 1732& 866& 433& 1300& 650\\
325& 976& 488& 244& 122& 61& 184& 92& 46& 23\\
70& 35& 106& 53& 160& 80& 40& 20& 10& 5\\
16& 8& 4& 2& 1& \\

1310&&&&&&&&&\\
655& 1966& 983& 2950& 1475& 4426& 2213& 6640& 3320& 1660\\
830& 415& 1246& 623& 1870& 935& 2806& 1403& 4210& 2105\\
6316& 3158& 1579& 4738& 2369& 7108& 3554& 1777& 5332& 2666\\
1333& 4000& 2000& 1000& 500& 250& 125& 376& 188& 94\\
47& 142& 71& 214& 107& 322& 161& 484& 242& 121\\
364& 182& 91& 274& 137& 412& 206& 103& 310& 155\\
466& 233& 700& 350& 175& 526& 263& 790& 395& 1186\\
593& 1780& 890& 445& 1336& 668& 334& 167& 502& 251\\
754& 377& 1132& 566& 283& 850& 425& 1276& 638& 319\\
958& 479& 1438& 719& 2158& 1079& 3238& 1619& 4858& 2429\\
7288& 3644& 1822& 911& 2734& 1367& 4102& 2051& 6154& 3077\\
9232& 4616& 2308& 1154& 577& 1732& 866& 433& 1300& 650\\
325& 976& 488& 244& 122& 61& 184& 92& 46& 23\\
70& 35& 106& 53& 160& 80& 40& 20& 10& 5\\
16& 8& 4& 2& 1& \\

1311&&&&&&&&&\\
3934& 1967& 5902& 2951& 8854& 4427& 13282& 6641& 19924& 9962\\
4981& 14944& 7472& 3736& 1868& 934& 467& 1402& 701& 2104\\
1052& 526& 263& 790& 395& 1186& 593& 1780& 890& 445\\
1336& 668& 334& 167& 502& 251& 754& 377& 1132& 566\\
283& 850& 425& 1276& 638& 319& 958& 479& 1438& 719\\
2158& 1079& 3238& 1619& 4858& 2429& 7288& 3644& 1822& 911\\
2734& 1367& 4102& 2051& 6154& 3077& 9232& 4616& 2308& 1154\\
577& 1732& 866& 433& 1300& 650& 325& 976& 488& 244\\
122& 61& 184& 92& 46& 23& 70& 35& 106& 53\\
160& 80& 40& 20& 10& 5& 16& 8& 4& 2\\
1& \\

1312&&&&&&&&&\\
656& 328& 164& 82& 41& 124& 62& 31& 94& 47\\
142& 71& 214& 107& 322& 161& 484& 242& 121& 364\\
182& 91& 274& 137& 412& 206& 103& 310& 155& 466\\
233& 700& 350& 175& 526& 263& 790& 395& 1186& 593\\
1780& 890& 445& 1336& 668& 334& 167& 502& 251& 754\\
377& 1132& 566& 283& 850& 425& 1276& 638& 319& 958\\
479& 1438& 719& 2158& 1079& 3238& 1619& 4858& 2429& 7288\\
3644& 1822& 911& 2734& 1367& 4102& 2051& 6154& 3077& 9232\\
4616& 2308& 1154& 577& 1732& 866& 433& 1300& 650& 325\\
976& 488& 244& 122& 61& 184& 92& 46& 23& 70\\
35& 106& 53& 160& 80& 40& 20& 10& 5& 16\\
8& 4& 2& 1& \\

1313&&&&&&&&&\\
3940& 1970& 985& 2956& 1478& 739& 2218& 1109& 3328& 1664\\
832& 416& 208& 104& 52& 26& 13& 40& 20& 10\\
5& 16& 8& 4& 2& 1& \\

1314&&&&&&&&&\\
657& 1972& 986& 493& 1480& 740& 370& 185& 556& 278\\
139& 418& 209& 628& 314& 157& 472& 236& 118& 59\\
178& 89& 268& 134& 67& 202& 101& 304& 152& 76\\
38& 19& 58& 29& 88& 44& 22& 11& 34& 17\\
52& 26& 13& 40& 20& 10& 5& 16& 8& 4\\
2& 1& \\

1315&&&&&&&&&\\
3946& 1973& 5920& 2960& 1480& 740& 370& 185& 556& 278\\
139& 418& 209& 628& 314& 157& 472& 236& 118& 59\\
178& 89& 268& 134& 67& 202& 101& 304& 152& 76\\
38& 19& 58& 29& 88& 44& 22& 11& 34& 17\\
52& 26& 13& 40& 20& 10& 5& 16& 8& 4\\
2& 1& \\

1316&&&&&&&&&\\
658& 329& 988& 494& 247& 742& 371& 1114& 557& 1672\\
836& 418& 209& 628& 314& 157& 472& 236& 118& 59\\
178& 89& 268& 134& 67& 202& 101& 304& 152& 76\\
38& 19& 58& 29& 88& 44& 22& 11& 34& 17\\
52& 26& 13& 40& 20& 10& 5& 16& 8& 4\\
2& 1& \\

1317&&&&&&&&&\\
3952& 1976& 988& 494& 247& 742& 371& 1114& 557& 1672\\
836& 418& 209& 628& 314& 157& 472& 236& 118& 59\\
178& 89& 268& 134& 67& 202& 101& 304& 152& 76\\
38& 19& 58& 29& 88& 44& 22& 11& 34& 17\\
52& 26& 13& 40& 20& 10& 5& 16& 8& 4\\
2& 1& \\

1318&&&&&&&&&\\
659& 1978& 989& 2968& 1484& 742& 371& 1114& 557& 1672\\
836& 418& 209& 628& 314& 157& 472& 236& 118& 59\\
178& 89& 268& 134& 67& 202& 101& 304& 152& 76\\
38& 19& 58& 29& 88& 44& 22& 11& 34& 17\\
52& 26& 13& 40& 20& 10& 5& 16& 8& 4\\
2& 1& \\

1319&&&&&&&&&\\
3958& 1979& 5938& 2969& 8908& 4454& 2227& 6682& 3341& 10024\\
5012& 2506& 1253& 3760& 1880& 940& 470& 235& 706& 353\\
1060& 530& 265& 796& 398& 199& 598& 299& 898& 449\\
1348& 674& 337& 1012& 506& 253& 760& 380& 190& 95\\
286& 143& 430& 215& 646& 323& 970& 485& 1456& 728\\
364& 182& 91& 274& 137& 412& 206& 103& 310& 155\\
466& 233& 700& 350& 175& 526& 263& 790& 395& 1186\\
593& 1780& 890& 445& 1336& 668& 334& 167& 502& 251\\
754& 377& 1132& 566& 283& 850& 425& 1276& 638& 319\\
958& 479& 1438& 719& 2158& 1079& 3238& 1619& 4858& 2429\\
7288& 3644& 1822& 911& 2734& 1367& 4102& 2051& 6154& 3077\\
9232& 4616& 2308& 1154& 577& 1732& 866& 433& 1300& 650\\
325& 976& 488& 244& 122& 61& 184& 92& 46& 23\\
70& 35& 106& 53& 160& 80& 40& 20& 10& 5\\
16& 8& 4& 2& 1& \\

1320&&&&&&&&&\\
660& 330& 165& 496& 248& 124& 62& 31& 94& 47\\
142& 71& 214& 107& 322& 161& 484& 242& 121& 364\\
182& 91& 274& 137& 412& 206& 103& 310& 155& 466\\
233& 700& 350& 175& 526& 263& 790& 395& 1186& 593\\
1780& 890& 445& 1336& 668& 334& 167& 502& 251& 754\\
377& 1132& 566& 283& 850& 425& 1276& 638& 319& 958\\
479& 1438& 719& 2158& 1079& 3238& 1619& 4858& 2429& 7288\\
3644& 1822& 911& 2734& 1367& 4102& 2051& 6154& 3077& 9232\\
4616& 2308& 1154& 577& 1732& 866& 433& 1300& 650& 325\\
976& 488& 244& 122& 61& 184& 92& 46& 23& 70\\
35& 106& 53& 160& 80& 40& 20& 10& 5& 16\\
8& 4& 2& 1& \\

1321&&&&&&&&&\\
3964& 1982& 991& 2974& 1487& 4462& 2231& 6694& 3347& 10042\\
5021& 15064& 7532& 3766& 1883& 5650& 2825& 8476& 4238& 2119\\
6358& 3179& 9538& 4769& 14308& 7154& 3577& 10732& 5366& 2683\\
8050& 4025& 12076& 6038& 3019& 9058& 4529& 13588& 6794& 3397\\
10192& 5096& 2548& 1274& 637& 1912& 956& 478& 239& 718\\
359& 1078& 539& 1618& 809& 2428& 1214& 607& 1822& 911\\
2734& 1367& 4102& 2051& 6154& 3077& 9232& 4616& 2308& 1154\\
577& 1732& 866& 433& 1300& 650& 325& 976& 488& 244\\
122& 61& 184& 92& 46& 23& 70& 35& 106& 53\\
160& 80& 40& 20& 10& 5& 16& 8& 4& 2\\
1& \\

1322&&&&&&&&&\\
661& 1984& 992& 496& 248& 124& 62& 31& 94& 47\\
142& 71& 214& 107& 322& 161& 484& 242& 121& 364\\
182& 91& 274& 137& 412& 206& 103& 310& 155& 466\\
233& 700& 350& 175& 526& 263& 790& 395& 1186& 593\\
1780& 890& 445& 1336& 668& 334& 167& 502& 251& 754\\
377& 1132& 566& 283& 850& 425& 1276& 638& 319& 958\\
479& 1438& 719& 2158& 1079& 3238& 1619& 4858& 2429& 7288\\
3644& 1822& 911& 2734& 1367& 4102& 2051& 6154& 3077& 9232\\
4616& 2308& 1154& 577& 1732& 866& 433& 1300& 650& 325\\
976& 488& 244& 122& 61& 184& 92& 46& 23& 70\\
35& 106& 53& 160& 80& 40& 20& 10& 5& 16\\
8& 4& 2& 1& \\

1323&&&&&&&&&\\
3970& 1985& 5956& 2978& 1489& 4468& 2234& 1117& 3352& 1676\\
838& 419& 1258& 629& 1888& 944& 472& 236& 118& 59\\
178& 89& 268& 134& 67& 202& 101& 304& 152& 76\\
38& 19& 58& 29& 88& 44& 22& 11& 34& 17\\
52& 26& 13& 40& 20& 10& 5& 16& 8& 4\\
2& 1& \\

1324&&&&&&&&&\\
662& 331& 994& 497& 1492& 746& 373& 1120& 560& 280\\
140& 70& 35& 106& 53& 160& 80& 40& 20& 10\\
5& 16& 8& 4& 2& 1& \\

1325&&&&&&&&&\\
3976& 1988& 994& 497& 1492& 746& 373& 1120& 560& 280\\
140& 70& 35& 106& 53& 160& 80& 40& 20& 10\\
5& 16& 8& 4& 2& 1& \\

1326&&&&&&&&&\\
663& 1990& 995& 2986& 1493& 4480& 2240& 1120& 560& 280\\
140& 70& 35& 106& 53& 160& 80& 40& 20& 10\\
5& 16& 8& 4& 2& 1& \\

1327&&&&&&&&&\\
3982& 1991& 5974& 2987& 8962& 4481& 13444& 6722& 3361& 10084\\
5042& 2521& 7564& 3782& 1891& 5674& 2837& 8512& 4256& 2128\\
1064& 532& 266& 133& 400& 200& 100& 50& 25& 76\\
38& 19& 58& 29& 88& 44& 22& 11& 34& 17\\
52& 26& 13& 40& 20& 10& 5& 16& 8& 4\\
2& 1& \\

1328&&&&&&&&&\\
664& 332& 166& 83& 250& 125& 376& 188& 94& 47\\
142& 71& 214& 107& 322& 161& 484& 242& 121& 364\\
182& 91& 274& 137& 412& 206& 103& 310& 155& 466\\
233& 700& 350& 175& 526& 263& 790& 395& 1186& 593\\
1780& 890& 445& 1336& 668& 334& 167& 502& 251& 754\\
377& 1132& 566& 283& 850& 425& 1276& 638& 319& 958\\
479& 1438& 719& 2158& 1079& 3238& 1619& 4858& 2429& 7288\\
3644& 1822& 911& 2734& 1367& 4102& 2051& 6154& 3077& 9232\\
4616& 2308& 1154& 577& 1732& 866& 433& 1300& 650& 325\\
976& 488& 244& 122& 61& 184& 92& 46& 23& 70\\
35& 106& 53& 160& 80& 40& 20& 10& 5& 16\\
8& 4& 2& 1& \\

1329&&&&&&&&&\\
3988& 1994& 997& 2992& 1496& 748& 374& 187& 562& 281\\
844& 422& 211& 634& 317& 952& 476& 238& 119& 358\\
179& 538& 269& 808& 404& 202& 101& 304& 152& 76\\
38& 19& 58& 29& 88& 44& 22& 11& 34& 17\\
52& 26& 13& 40& 20& 10& 5& 16& 8& 4\\
2& 1& \\

1330&&&&&&&&&\\
665& 1996& 998& 499& 1498& 749& 2248& 1124& 562& 281\\
844& 422& 211& 634& 317& 952& 476& 238& 119& 358\\
179& 538& 269& 808& 404& 202& 101& 304& 152& 76\\
38& 19& 58& 29& 88& 44& 22& 11& 34& 17\\
52& 26& 13& 40& 20& 10& 5& 16& 8& 4\\
2& 1& \\

1331&&&&&&&&&\\
3994& 1997& 5992& 2996& 1498& 749& 2248& 1124& 562& 281\\
844& 422& 211& 634& 317& 952& 476& 238& 119& 358\\
179& 538& 269& 808& 404& 202& 101& 304& 152& 76\\
38& 19& 58& 29& 88& 44& 22& 11& 34& 17\\
52& 26& 13& 40& 20& 10& 5& 16& 8& 4\\
2& 1& \\

1332&&&&&&&&&\\
666& 333& 1000& 500& 250& 125& 376& 188& 94& 47\\
142& 71& 214& 107& 322& 161& 484& 242& 121& 364\\
182& 91& 274& 137& 412& 206& 103& 310& 155& 466\\
233& 700& 350& 175& 526& 263& 790& 395& 1186& 593\\
1780& 890& 445& 1336& 668& 334& 167& 502& 251& 754\\
377& 1132& 566& 283& 850& 425& 1276& 638& 319& 958\\
479& 1438& 719& 2158& 1079& 3238& 1619& 4858& 2429& 7288\\
3644& 1822& 911& 2734& 1367& 4102& 2051& 6154& 3077& 9232\\
4616& 2308& 1154& 577& 1732& 866& 433& 1300& 650& 325\\
976& 488& 244& 122& 61& 184& 92& 46& 23& 70\\
35& 106& 53& 160& 80& 40& 20& 10& 5& 16\\
8& 4& 2& 1& \\

1333&&&&&&&&&\\
4000& 2000& 1000& 500& 250& 125& 376& 188& 94& 47\\
142& 71& 214& 107& 322& 161& 484& 242& 121& 364\\
182& 91& 274& 137& 412& 206& 103& 310& 155& 466\\
233& 700& 350& 175& 526& 263& 790& 395& 1186& 593\\
1780& 890& 445& 1336& 668& 334& 167& 502& 251& 754\\
377& 1132& 566& 283& 850& 425& 1276& 638& 319& 958\\
479& 1438& 719& 2158& 1079& 3238& 1619& 4858& 2429& 7288\\
3644& 1822& 911& 2734& 1367& 4102& 2051& 6154& 3077& 9232\\
4616& 2308& 1154& 577& 1732& 866& 433& 1300& 650& 325\\
976& 488& 244& 122& 61& 184& 92& 46& 23& 70\\
35& 106& 53& 160& 80& 40& 20& 10& 5& 16\\
8& 4& 2& 1& \\

1334&&&&&&&&&\\
667& 2002& 1001& 3004& 1502& 751& 2254& 1127& 3382& 1691\\
5074& 2537& 7612& 3806& 1903& 5710& 2855& 8566& 4283& 12850\\
6425& 19276& 9638& 4819& 14458& 7229& 21688& 10844& 5422& 2711\\
8134& 4067& 12202& 6101& 18304& 9152& 4576& 2288& 1144& 572\\
286& 143& 430& 215& 646& 323& 970& 485& 1456& 728\\
364& 182& 91& 274& 137& 412& 206& 103& 310& 155\\
466& 233& 700& 350& 175& 526& 263& 790& 395& 1186\\
593& 1780& 890& 445& 1336& 668& 334& 167& 502& 251\\
754& 377& 1132& 566& 283& 850& 425& 1276& 638& 319\\
958& 479& 1438& 719& 2158& 1079& 3238& 1619& 4858& 2429\\
7288& 3644& 1822& 911& 2734& 1367& 4102& 2051& 6154& 3077\\
9232& 4616& 2308& 1154& 577& 1732& 866& 433& 1300& 650\\
325& 976& 488& 244& 122& 61& 184& 92& 46& 23\\
70& 35& 106& 53& 160& 80& 40& 20& 10& 5\\
16& 8& 4& 2& 1& \\

1335&&&&&&&&&\\
4006& 2003& 6010& 3005& 9016& 4508& 2254& 1127& 3382& 1691\\
5074& 2537& 7612& 3806& 1903& 5710& 2855& 8566& 4283& 12850\\
6425& 19276& 9638& 4819& 14458& 7229& 21688& 10844& 5422& 2711\\
8134& 4067& 12202& 6101& 18304& 9152& 4576& 2288& 1144& 572\\
286& 143& 430& 215& 646& 323& 970& 485& 1456& 728\\
364& 182& 91& 274& 137& 412& 206& 103& 310& 155\\
466& 233& 700& 350& 175& 526& 263& 790& 395& 1186\\
593& 1780& 890& 445& 1336& 668& 334& 167& 502& 251\\
754& 377& 1132& 566& 283& 850& 425& 1276& 638& 319\\
958& 479& 1438& 719& 2158& 1079& 3238& 1619& 4858& 2429\\
7288& 3644& 1822& 911& 2734& 1367& 4102& 2051& 6154& 3077\\
9232& 4616& 2308& 1154& 577& 1732& 866& 433& 1300& 650\\
325& 976& 488& 244& 122& 61& 184& 92& 46& 23\\
70& 35& 106& 53& 160& 80& 40& 20& 10& 5\\
16& 8& 4& 2& 1& \\

1336&&&&&&&&&\\
668& 334& 167& 502& 251& 754& 377& 1132& 566& 283\\
850& 425& 1276& 638& 319& 958& 479& 1438& 719& 2158\\
1079& 3238& 1619& 4858& 2429& 7288& 3644& 1822& 911& 2734\\
1367& 4102& 2051& 6154& 3077& 9232& 4616& 2308& 1154& 577\\
1732& 866& 433& 1300& 650& 325& 976& 488& 244& 122\\
61& 184& 92& 46& 23& 70& 35& 106& 53& 160\\
80& 40& 20& 10& 5& 16& 8& 4& 2& 1\\

1337&&&&&&&&&\\
4012& 2006& 1003& 3010& 1505& 4516& 2258& 1129& 3388& 1694\\
847& 2542& 1271& 3814& 1907& 5722& 2861& 8584& 4292& 2146\\
1073& 3220& 1610& 805& 2416& 1208& 604& 302& 151& 454\\
227& 682& 341& 1024& 512& 256& 128& 64& 32& 16\\
8& 4& 2& 1& \\

1338&&&&&&&&&\\
669& 2008& 1004& 502& 251& 754& 377& 1132& 566& 283\\
850& 425& 1276& 638& 319& 958& 479& 1438& 719& 2158\\
1079& 3238& 1619& 4858& 2429& 7288& 3644& 1822& 911& 2734\\
1367& 4102& 2051& 6154& 3077& 9232& 4616& 2308& 1154& 577\\
1732& 866& 433& 1300& 650& 325& 976& 488& 244& 122\\
61& 184& 92& 46& 23& 70& 35& 106& 53& 160\\
80& 40& 20& 10& 5& 16& 8& 4& 2& 1\\

1339&&&&&&&&&\\
4018& 2009& 6028& 3014& 1507& 4522& 2261& 6784& 3392& 1696\\
848& 424& 212& 106& 53& 160& 80& 40& 20& 10\\
5& 16& 8& 4& 2& 1& \\

1340&&&&&&&&&\\
670& 335& 1006& 503& 1510& 755& 2266& 1133& 3400& 1700\\
850& 425& 1276& 638& 319& 958& 479& 1438& 719& 2158\\
1079& 3238& 1619& 4858& 2429& 7288& 3644& 1822& 911& 2734\\
1367& 4102& 2051& 6154& 3077& 9232& 4616& 2308& 1154& 577\\
1732& 866& 433& 1300& 650& 325& 976& 488& 244& 122\\
61& 184& 92& 46& 23& 70& 35& 106& 53& 160\\
80& 40& 20& 10& 5& 16& 8& 4& 2& 1\\

1341&&&&&&&&&\\
4024& 2012& 1006& 503& 1510& 755& 2266& 1133& 3400& 1700\\
850& 425& 1276& 638& 319& 958& 479& 1438& 719& 2158\\
1079& 3238& 1619& 4858& 2429& 7288& 3644& 1822& 911& 2734\\
1367& 4102& 2051& 6154& 3077& 9232& 4616& 2308& 1154& 577\\
1732& 866& 433& 1300& 650& 325& 976& 488& 244& 122\\
61& 184& 92& 46& 23& 70& 35& 106& 53& 160\\
80& 40& 20& 10& 5& 16& 8& 4& 2& 1\\

1342&&&&&&&&&\\
671& 2014& 1007& 3022& 1511& 4534& 2267& 6802& 3401& 10204\\
5102& 2551& 7654& 3827& 11482& 5741& 17224& 8612& 4306& 2153\\
6460& 3230& 1615& 4846& 2423& 7270& 3635& 10906& 5453& 16360\\
8180& 4090& 2045& 6136& 3068& 1534& 767& 2302& 1151& 3454\\
1727& 5182& 2591& 7774& 3887& 11662& 5831& 17494& 8747& 26242\\
13121& 39364& 19682& 9841& 29524& 14762& 7381& 22144& 11072& 5536\\
2768& 1384& 692& 346& 173& 520& 260& 130& 65& 196\\
98& 49& 148& 74& 37& 112& 56& 28& 14& 7\\
22& 11& 34& 17& 52& 26& 13& 40& 20& 10\\
5& 16& 8& 4& 2& 1& \\

1343&&&&&&&&&\\
4030& 2015& 6046& 3023& 9070& 4535& 13606& 6803& 20410& 10205\\
30616& 15308& 7654& 3827& 11482& 5741& 17224& 8612& 4306& 2153\\
6460& 3230& 1615& 4846& 2423& 7270& 3635& 10906& 5453& 16360\\
8180& 4090& 2045& 6136& 3068& 1534& 767& 2302& 1151& 3454\\
1727& 5182& 2591& 7774& 3887& 11662& 5831& 17494& 8747& 26242\\
13121& 39364& 19682& 9841& 29524& 14762& 7381& 22144& 11072& 5536\\
2768& 1384& 692& 346& 173& 520& 260& 130& 65& 196\\
98& 49& 148& 74& 37& 112& 56& 28& 14& 7\\
22& 11& 34& 17& 52& 26& 13& 40& 20& 10\\
5& 16& 8& 4& 2& 1& \\

1344&&&&&&&&&\\
672& 336& 168& 84& 42& 21& 64& 32& 16& 8\\
4& 2& 1& \\

1345&&&&&&&&&\\
4036& 2018& 1009& 3028& 1514& 757& 2272& 1136& 568& 284\\
142& 71& 214& 107& 322& 161& 484& 242& 121& 364\\
182& 91& 274& 137& 412& 206& 103& 310& 155& 466\\
233& 700& 350& 175& 526& 263& 790& 395& 1186& 593\\
1780& 890& 445& 1336& 668& 334& 167& 502& 251& 754\\
377& 1132& 566& 283& 850& 425& 1276& 638& 319& 958\\
479& 1438& 719& 2158& 1079& 3238& 1619& 4858& 2429& 7288\\
3644& 1822& 911& 2734& 1367& 4102& 2051& 6154& 3077& 9232\\
4616& 2308& 1154& 577& 1732& 866& 433& 1300& 650& 325\\
976& 488& 244& 122& 61& 184& 92& 46& 23& 70\\
35& 106& 53& 160& 80& 40& 20& 10& 5& 16\\
8& 4& 2& 1& \\

1346&&&&&&&&&\\
673& 2020& 1010& 505& 1516& 758& 379& 1138& 569& 1708\\
854& 427& 1282& 641& 1924& 962& 481& 1444& 722& 361\\
1084& 542& 271& 814& 407& 1222& 611& 1834& 917& 2752\\
1376& 688& 344& 172& 86& 43& 130& 65& 196& 98\\
49& 148& 74& 37& 112& 56& 28& 14& 7& 22\\
11& 34& 17& 52& 26& 13& 40& 20& 10& 5\\
16& 8& 4& 2& 1& \\

1347&&&&&&&&&\\
4042& 2021& 6064& 3032& 1516& 758& 379& 1138& 569& 1708\\
854& 427& 1282& 641& 1924& 962& 481& 1444& 722& 361\\
1084& 542& 271& 814& 407& 1222& 611& 1834& 917& 2752\\
1376& 688& 344& 172& 86& 43& 130& 65& 196& 98\\
49& 148& 74& 37& 112& 56& 28& 14& 7& 22\\
11& 34& 17& 52& 26& 13& 40& 20& 10& 5\\
16& 8& 4& 2& 1& \\

1348&&&&&&&&&\\
674& 337& 1012& 506& 253& 760& 380& 190& 95& 286\\
143& 430& 215& 646& 323& 970& 485& 1456& 728& 364\\
182& 91& 274& 137& 412& 206& 103& 310& 155& 466\\
233& 700& 350& 175& 526& 263& 790& 395& 1186& 593\\
1780& 890& 445& 1336& 668& 334& 167& 502& 251& 754\\
377& 1132& 566& 283& 850& 425& 1276& 638& 319& 958\\
479& 1438& 719& 2158& 1079& 3238& 1619& 4858& 2429& 7288\\
3644& 1822& 911& 2734& 1367& 4102& 2051& 6154& 3077& 9232\\
4616& 2308& 1154& 577& 1732& 866& 433& 1300& 650& 325\\
976& 488& 244& 122& 61& 184& 92& 46& 23& 70\\
35& 106& 53& 160& 80& 40& 20& 10& 5& 16\\
8& 4& 2& 1& \\

1349&&&&&&&&&\\
4048& 2024& 1012& 506& 253& 760& 380& 190& 95& 286\\
143& 430& 215& 646& 323& 970& 485& 1456& 728& 364\\
182& 91& 274& 137& 412& 206& 103& 310& 155& 466\\
233& 700& 350& 175& 526& 263& 790& 395& 1186& 593\\
1780& 890& 445& 1336& 668& 334& 167& 502& 251& 754\\
377& 1132& 566& 283& 850& 425& 1276& 638& 319& 958\\
479& 1438& 719& 2158& 1079& 3238& 1619& 4858& 2429& 7288\\
3644& 1822& 911& 2734& 1367& 4102& 2051& 6154& 3077& 9232\\
4616& 2308& 1154& 577& 1732& 866& 433& 1300& 650& 325\\
976& 488& 244& 122& 61& 184& 92& 46& 23& 70\\
35& 106& 53& 160& 80& 40& 20& 10& 5& 16\\
8& 4& 2& 1& \\

1350&&&&&&&&&\\
675& 2026& 1013& 3040& 1520& 760& 380& 190& 95& 286\\
143& 430& 215& 646& 323& 970& 485& 1456& 728& 364\\
182& 91& 274& 137& 412& 206& 103& 310& 155& 466\\
233& 700& 350& 175& 526& 263& 790& 395& 1186& 593\\
1780& 890& 445& 1336& 668& 334& 167& 502& 251& 754\\
377& 1132& 566& 283& 850& 425& 1276& 638& 319& 958\\
479& 1438& 719& 2158& 1079& 3238& 1619& 4858& 2429& 7288\\
3644& 1822& 911& 2734& 1367& 4102& 2051& 6154& 3077& 9232\\
4616& 2308& 1154& 577& 1732& 866& 433& 1300& 650& 325\\
976& 488& 244& 122& 61& 184& 92& 46& 23& 70\\
35& 106& 53& 160& 80& 40& 20& 10& 5& 16\\
8& 4& 2& 1& \\

1351&&&&&&&&&\\
4054& 2027& 6082& 3041& 9124& 4562& 2281& 6844& 3422& 1711\\
5134& 2567& 7702& 3851& 11554& 5777& 17332& 8666& 4333& 13000\\
6500& 3250& 1625& 4876& 2438& 1219& 3658& 1829& 5488& 2744\\
1372& 686& 343& 1030& 515& 1546& 773& 2320& 1160& 580\\
290& 145& 436& 218& 109& 328& 164& 82& 41& 124\\
62& 31& 94& 47& 142& 71& 214& 107& 322& 161\\
484& 242& 121& 364& 182& 91& 274& 137& 412& 206\\
103& 310& 155& 466& 233& 700& 350& 175& 526& 263\\
790& 395& 1186& 593& 1780& 890& 445& 1336& 668& 334\\
167& 502& 251& 754& 377& 1132& 566& 283& 850& 425\\
1276& 638& 319& 958& 479& 1438& 719& 2158& 1079& 3238\\
1619& 4858& 2429& 7288& 3644& 1822& 911& 2734& 1367& 4102\\
2051& 6154& 3077& 9232& 4616& 2308& 1154& 577& 1732& 866\\
433& 1300& 650& 325& 976& 488& 244& 122& 61& 184\\
92& 46& 23& 70& 35& 106& 53& 160& 80& 40\\
20& 10& 5& 16& 8& 4& 2& 1& \\

1352&&&&&&&&&\\
676& 338& 169& 508& 254& 127& 382& 191& 574& 287\\
862& 431& 1294& 647& 1942& 971& 2914& 1457& 4372& 2186\\
1093& 3280& 1640& 820& 410& 205& 616& 308& 154& 77\\
232& 116& 58& 29& 88& 44& 22& 11& 34& 17\\
52& 26& 13& 40& 20& 10& 5& 16& 8& 4\\
2& 1& \\

1353&&&&&&&&&\\
4060& 2030& 1015& 3046& 1523& 4570& 2285& 6856& 3428& 1714\\
857& 2572& 1286& 643& 1930& 965& 2896& 1448& 724& 362\\
181& 544& 272& 136& 68& 34& 17& 52& 26& 13\\
40& 20& 10& 5& 16& 8& 4& 2& 1& \\

1354&&&&&&&&&\\
677& 2032& 1016& 508& 254& 127& 382& 191& 574& 287\\
862& 431& 1294& 647& 1942& 971& 2914& 1457& 4372& 2186\\
1093& 3280& 1640& 820& 410& 205& 616& 308& 154& 77\\
232& 116& 58& 29& 88& 44& 22& 11& 34& 17\\
52& 26& 13& 40& 20& 10& 5& 16& 8& 4\\
2& 1& \\

1355&&&&&&&&&\\
4066& 2033& 6100& 3050& 1525& 4576& 2288& 1144& 572& 286\\
143& 430& 215& 646& 323& 970& 485& 1456& 728& 364\\
182& 91& 274& 137& 412& 206& 103& 310& 155& 466\\
233& 700& 350& 175& 526& 263& 790& 395& 1186& 593\\
1780& 890& 445& 1336& 668& 334& 167& 502& 251& 754\\
377& 1132& 566& 283& 850& 425& 1276& 638& 319& 958\\
479& 1438& 719& 2158& 1079& 3238& 1619& 4858& 2429& 7288\\
3644& 1822& 911& 2734& 1367& 4102& 2051& 6154& 3077& 9232\\
4616& 2308& 1154& 577& 1732& 866& 433& 1300& 650& 325\\
976& 488& 244& 122& 61& 184& 92& 46& 23& 70\\
35& 106& 53& 160& 80& 40& 20& 10& 5& 16\\
8& 4& 2& 1& \\

1356&&&&&&&&&\\
678& 339& 1018& 509& 1528& 764& 382& 191& 574& 287\\
862& 431& 1294& 647& 1942& 971& 2914& 1457& 4372& 2186\\
1093& 3280& 1640& 820& 410& 205& 616& 308& 154& 77\\
232& 116& 58& 29& 88& 44& 22& 11& 34& 17\\
52& 26& 13& 40& 20& 10& 5& 16& 8& 4\\
2& 1& \\

1357&&&&&&&&&\\
4072& 2036& 1018& 509& 1528& 764& 382& 191& 574& 287\\
862& 431& 1294& 647& 1942& 971& 2914& 1457& 4372& 2186\\
1093& 3280& 1640& 820& 410& 205& 616& 308& 154& 77\\
232& 116& 58& 29& 88& 44& 22& 11& 34& 17\\
52& 26& 13& 40& 20& 10& 5& 16& 8& 4\\
2& 1& \\

1358&&&&&&&&&\\
679& 2038& 1019& 3058& 1529& 4588& 2294& 1147& 3442& 1721\\
5164& 2582& 1291& 3874& 1937& 5812& 2906& 1453& 4360& 2180\\
1090& 545& 1636& 818& 409& 1228& 614& 307& 922& 461\\
1384& 692& 346& 173& 520& 260& 130& 65& 196& 98\\
49& 148& 74& 37& 112& 56& 28& 14& 7& 22\\
11& 34& 17& 52& 26& 13& 40& 20& 10& 5\\
16& 8& 4& 2& 1& \\

1359&&&&&&&&&\\
4078& 2039& 6118& 3059& 9178& 4589& 13768& 6884& 3442& 1721\\
5164& 2582& 1291& 3874& 1937& 5812& 2906& 1453& 4360& 2180\\
1090& 545& 1636& 818& 409& 1228& 614& 307& 922& 461\\
1384& 692& 346& 173& 520& 260& 130& 65& 196& 98\\
49& 148& 74& 37& 112& 56& 28& 14& 7& 22\\
11& 34& 17& 52& 26& 13& 40& 20& 10& 5\\
16& 8& 4& 2& 1& \\

1360&&&&&&&&&\\
680& 340& 170& 85& 256& 128& 64& 32& 16& 8\\
4& 2& 1& \\

1361&&&&&&&&&\\
4084& 2042& 1021& 3064& 1532& 766& 383& 1150& 575& 1726\\
863& 2590& 1295& 3886& 1943& 5830& 2915& 8746& 4373& 13120\\
6560& 3280& 1640& 820& 410& 205& 616& 308& 154& 77\\
232& 116& 58& 29& 88& 44& 22& 11& 34& 17\\
52& 26& 13& 40& 20& 10& 5& 16& 8& 4\\
2& 1& \\

1362&&&&&&&&&\\
681& 2044& 1022& 511& 1534& 767& 2302& 1151& 3454& 1727\\
5182& 2591& 7774& 3887& 11662& 5831& 17494& 8747& 26242& 13121\\
39364& 19682& 9841& 29524& 14762& 7381& 22144& 11072& 5536& 2768\\
1384& 692& 346& 173& 520& 260& 130& 65& 196& 98\\
49& 148& 74& 37& 112& 56& 28& 14& 7& 22\\
11& 34& 17& 52& 26& 13& 40& 20& 10& 5\\
16& 8& 4& 2& 1& \\

1363&&&&&&&&&\\
4090& 2045& 6136& 3068& 1534& 767& 2302& 1151& 3454& 1727\\
5182& 2591& 7774& 3887& 11662& 5831& 17494& 8747& 26242& 13121\\
39364& 19682& 9841& 29524& 14762& 7381& 22144& 11072& 5536& 2768\\
1384& 692& 346& 173& 520& 260& 130& 65& 196& 98\\
49& 148& 74& 37& 112& 56& 28& 14& 7& 22\\
11& 34& 17& 52& 26& 13& 40& 20& 10& 5\\
16& 8& 4& 2& 1& \\

1364&&&&&&&&&\\
682& 341& 1024& 512& 256& 128& 64& 32& 16& 8\\
4& 2& 1& \\

1365&&&&&&&&&\\
4096& 2048& 1024& 512& 256& 128& 64& 32& 16& 8\\
4& 2& 1& \\

1366&&&&&&&&&\\
683& 2050& 1025& 3076& 1538& 769& 2308& 1154& 577& 1732\\
866& 433& 1300& 650& 325& 976& 488& 244& 122& 61\\
184& 92& 46& 23& 70& 35& 106& 53& 160& 80\\
40& 20& 10& 5& 16& 8& 4& 2& 1& \\

1367&&&&&&&&&\\
4102& 2051& 6154& 3077& 9232& 4616& 2308& 1154& 577& 1732\\
866& 433& 1300& 650& 325& 976& 488& 244& 122& 61\\
184& 92& 46& 23& 70& 35& 106& 53& 160& 80\\
40& 20& 10& 5& 16& 8& 4& 2& 1& \\

1368&&&&&&&&&\\
684& 342& 171& 514& 257& 772& 386& 193& 580& 290\\
145& 436& 218& 109& 328& 164& 82& 41& 124& 62\\
31& 94& 47& 142& 71& 214& 107& 322& 161& 484\\
242& 121& 364& 182& 91& 274& 137& 412& 206& 103\\
310& 155& 466& 233& 700& 350& 175& 526& 263& 790\\
395& 1186& 593& 1780& 890& 445& 1336& 668& 334& 167\\
502& 251& 754& 377& 1132& 566& 283& 850& 425& 1276\\
638& 319& 958& 479& 1438& 719& 2158& 1079& 3238& 1619\\
4858& 2429& 7288& 3644& 1822& 911& 2734& 1367& 4102& 2051\\
6154& 3077& 9232& 4616& 2308& 1154& 577& 1732& 866& 433\\
1300& 650& 325& 976& 488& 244& 122& 61& 184& 92\\
46& 23& 70& 35& 106& 53& 160& 80& 40& 20\\
10& 5& 16& 8& 4& 2& 1& \\

1369&&&&&&&&&\\
4108& 2054& 1027& 3082& 1541& 4624& 2312& 1156& 578& 289\\
868& 434& 217& 652& 326& 163& 490& 245& 736& 368\\
184& 92& 46& 23& 70& 35& 106& 53& 160& 80\\
40& 20& 10& 5& 16& 8& 4& 2& 1& \\

1370&&&&&&&&&\\
685& 2056& 1028& 514& 257& 772& 386& 193& 580& 290\\
145& 436& 218& 109& 328& 164& 82& 41& 124& 62\\
31& 94& 47& 142& 71& 214& 107& 322& 161& 484\\
242& 121& 364& 182& 91& 274& 137& 412& 206& 103\\
310& 155& 466& 233& 700& 350& 175& 526& 263& 790\\
395& 1186& 593& 1780& 890& 445& 1336& 668& 334& 167\\
502& 251& 754& 377& 1132& 566& 283& 850& 425& 1276\\
638& 319& 958& 479& 1438& 719& 2158& 1079& 3238& 1619\\
4858& 2429& 7288& 3644& 1822& 911& 2734& 1367& 4102& 2051\\
6154& 3077& 9232& 4616& 2308& 1154& 577& 1732& 866& 433\\
1300& 650& 325& 976& 488& 244& 122& 61& 184& 92\\
46& 23& 70& 35& 106& 53& 160& 80& 40& 20\\
10& 5& 16& 8& 4& 2& 1& \\

1371&&&&&&&&&\\
4114& 2057& 6172& 3086& 1543& 4630& 2315& 6946& 3473& 10420\\
5210& 2605& 7816& 3908& 1954& 977& 2932& 1466& 733& 2200\\
1100& 550& 275& 826& 413& 1240& 620& 310& 155& 466\\
233& 700& 350& 175& 526& 263& 790& 395& 1186& 593\\
1780& 890& 445& 1336& 668& 334& 167& 502& 251& 754\\
377& 1132& 566& 283& 850& 425& 1276& 638& 319& 958\\
479& 1438& 719& 2158& 1079& 3238& 1619& 4858& 2429& 7288\\
3644& 1822& 911& 2734& 1367& 4102& 2051& 6154& 3077& 9232\\
4616& 2308& 1154& 577& 1732& 866& 433& 1300& 650& 325\\
976& 488& 244& 122& 61& 184& 92& 46& 23& 70\\
35& 106& 53& 160& 80& 40& 20& 10& 5& 16\\
8& 4& 2& 1& \\

1372&&&&&&&&&\\
686& 343& 1030& 515& 1546& 773& 2320& 1160& 580& 290\\
145& 436& 218& 109& 328& 164& 82& 41& 124& 62\\
31& 94& 47& 142& 71& 214& 107& 322& 161& 484\\
242& 121& 364& 182& 91& 274& 137& 412& 206& 103\\
310& 155& 466& 233& 700& 350& 175& 526& 263& 790\\
395& 1186& 593& 1780& 890& 445& 1336& 668& 334& 167\\
502& 251& 754& 377& 1132& 566& 283& 850& 425& 1276\\
638& 319& 958& 479& 1438& 719& 2158& 1079& 3238& 1619\\
4858& 2429& 7288& 3644& 1822& 911& 2734& 1367& 4102& 2051\\
6154& 3077& 9232& 4616& 2308& 1154& 577& 1732& 866& 433\\
1300& 650& 325& 976& 488& 244& 122& 61& 184& 92\\
46& 23& 70& 35& 106& 53& 160& 80& 40& 20\\
10& 5& 16& 8& 4& 2& 1& \\

1373&&&&&&&&&\\
4120& 2060& 1030& 515& 1546& 773& 2320& 1160& 580& 290\\
145& 436& 218& 109& 328& 164& 82& 41& 124& 62\\
31& 94& 47& 142& 71& 214& 107& 322& 161& 484\\
242& 121& 364& 182& 91& 274& 137& 412& 206& 103\\
310& 155& 466& 233& 700& 350& 175& 526& 263& 790\\
395& 1186& 593& 1780& 890& 445& 1336& 668& 334& 167\\
502& 251& 754& 377& 1132& 566& 283& 850& 425& 1276\\
638& 319& 958& 479& 1438& 719& 2158& 1079& 3238& 1619\\
4858& 2429& 7288& 3644& 1822& 911& 2734& 1367& 4102& 2051\\
6154& 3077& 9232& 4616& 2308& 1154& 577& 1732& 866& 433\\
1300& 650& 325& 976& 488& 244& 122& 61& 184& 92\\
46& 23& 70& 35& 106& 53& 160& 80& 40& 20\\
10& 5& 16& 8& 4& 2& 1& \\

1374&&&&&&&&&\\
687& 2062& 1031& 3094& 1547& 4642& 2321& 6964& 3482& 1741\\
5224& 2612& 1306& 653& 1960& 980& 490& 245& 736& 368\\
184& 92& 46& 23& 70& 35& 106& 53& 160& 80\\
40& 20& 10& 5& 16& 8& 4& 2& 1& \\

1375&&&&&&&&&\\
4126& 2063& 6190& 3095& 9286& 4643& 13930& 6965& 20896& 10448\\
5224& 2612& 1306& 653& 1960& 980& 490& 245& 736& 368\\
184& 92& 46& 23& 70& 35& 106& 53& 160& 80\\
40& 20& 10& 5& 16& 8& 4& 2& 1& \\

1376&&&&&&&&&\\
688& 344& 172& 86& 43& 130& 65& 196& 98& 49\\
148& 74& 37& 112& 56& 28& 14& 7& 22& 11\\
34& 17& 52& 26& 13& 40& 20& 10& 5& 16\\
8& 4& 2& 1& \\

1377&&&&&&&&&\\
4132& 2066& 1033& 3100& 1550& 775& 2326& 1163& 3490& 1745\\
5236& 2618& 1309& 3928& 1964& 982& 491& 1474& 737& 2212\\
1106& 553& 1660& 830& 415& 1246& 623& 1870& 935& 2806\\
1403& 4210& 2105& 6316& 3158& 1579& 4738& 2369& 7108& 3554\\
1777& 5332& 2666& 1333& 4000& 2000& 1000& 500& 250& 125\\
376& 188& 94& 47& 142& 71& 214& 107& 322& 161\\
484& 242& 121& 364& 182& 91& 274& 137& 412& 206\\
103& 310& 155& 466& 233& 700& 350& 175& 526& 263\\
790& 395& 1186& 593& 1780& 890& 445& 1336& 668& 334\\
167& 502& 251& 754& 377& 1132& 566& 283& 850& 425\\
1276& 638& 319& 958& 479& 1438& 719& 2158& 1079& 3238\\
1619& 4858& 2429& 7288& 3644& 1822& 911& 2734& 1367& 4102\\
2051& 6154& 3077& 9232& 4616& 2308& 1154& 577& 1732& 866\\
433& 1300& 650& 325& 976& 488& 244& 122& 61& 184\\
92& 46& 23& 70& 35& 106& 53& 160& 80& 40\\
20& 10& 5& 16& 8& 4& 2& 1& \\

1378&&&&&&&&&\\
689& 2068& 1034& 517& 1552& 776& 388& 194& 97& 292\\
146& 73& 220& 110& 55& 166& 83& 250& 125& 376\\
188& 94& 47& 142& 71& 214& 107& 322& 161& 484\\
242& 121& 364& 182& 91& 274& 137& 412& 206& 103\\
310& 155& 466& 233& 700& 350& 175& 526& 263& 790\\
395& 1186& 593& 1780& 890& 445& 1336& 668& 334& 167\\
502& 251& 754& 377& 1132& 566& 283& 850& 425& 1276\\
638& 319& 958& 479& 1438& 719& 2158& 1079& 3238& 1619\\
4858& 2429& 7288& 3644& 1822& 911& 2734& 1367& 4102& 2051\\
6154& 3077& 9232& 4616& 2308& 1154& 577& 1732& 866& 433\\
1300& 650& 325& 976& 488& 244& 122& 61& 184& 92\\
46& 23& 70& 35& 106& 53& 160& 80& 40& 20\\
10& 5& 16& 8& 4& 2& 1& \\

1379&&&&&&&&&\\
4138& 2069& 6208& 3104& 1552& 776& 388& 194& 97& 292\\
146& 73& 220& 110& 55& 166& 83& 250& 125& 376\\
188& 94& 47& 142& 71& 214& 107& 322& 161& 484\\
242& 121& 364& 182& 91& 274& 137& 412& 206& 103\\
310& 155& 466& 233& 700& 350& 175& 526& 263& 790\\
395& 1186& 593& 1780& 890& 445& 1336& 668& 334& 167\\
502& 251& 754& 377& 1132& 566& 283& 850& 425& 1276\\
638& 319& 958& 479& 1438& 719& 2158& 1079& 3238& 1619\\
4858& 2429& 7288& 3644& 1822& 911& 2734& 1367& 4102& 2051\\
6154& 3077& 9232& 4616& 2308& 1154& 577& 1732& 866& 433\\
1300& 650& 325& 976& 488& 244& 122& 61& 184& 92\\
46& 23& 70& 35& 106& 53& 160& 80& 40& 20\\
10& 5& 16& 8& 4& 2& 1& \\

1380&&&&&&&&&\\
690& 345& 1036& 518& 259& 778& 389& 1168& 584& 292\\
146& 73& 220& 110& 55& 166& 83& 250& 125& 376\\
188& 94& 47& 142& 71& 214& 107& 322& 161& 484\\
242& 121& 364& 182& 91& 274& 137& 412& 206& 103\\
310& 155& 466& 233& 700& 350& 175& 526& 263& 790\\
395& 1186& 593& 1780& 890& 445& 1336& 668& 334& 167\\
502& 251& 754& 377& 1132& 566& 283& 850& 425& 1276\\
638& 319& 958& 479& 1438& 719& 2158& 1079& 3238& 1619\\
4858& 2429& 7288& 3644& 1822& 911& 2734& 1367& 4102& 2051\\
6154& 3077& 9232& 4616& 2308& 1154& 577& 1732& 866& 433\\
1300& 650& 325& 976& 488& 244& 122& 61& 184& 92\\
46& 23& 70& 35& 106& 53& 160& 80& 40& 20\\
10& 5& 16& 8& 4& 2& 1& \\

1381&&&&&&&&&\\
4144& 2072& 1036& 518& 259& 778& 389& 1168& 584& 292\\
146& 73& 220& 110& 55& 166& 83& 250& 125& 376\\
188& 94& 47& 142& 71& 214& 107& 322& 161& 484\\
242& 121& 364& 182& 91& 274& 137& 412& 206& 103\\
310& 155& 466& 233& 700& 350& 175& 526& 263& 790\\
395& 1186& 593& 1780& 890& 445& 1336& 668& 334& 167\\
502& 251& 754& 377& 1132& 566& 283& 850& 425& 1276\\
638& 319& 958& 479& 1438& 719& 2158& 1079& 3238& 1619\\
4858& 2429& 7288& 3644& 1822& 911& 2734& 1367& 4102& 2051\\
6154& 3077& 9232& 4616& 2308& 1154& 577& 1732& 866& 433\\
1300& 650& 325& 976& 488& 244& 122& 61& 184& 92\\
46& 23& 70& 35& 106& 53& 160& 80& 40& 20\\
10& 5& 16& 8& 4& 2& 1& \\

1382&&&&&&&&&\\
691& 2074& 1037& 3112& 1556& 778& 389& 1168& 584& 292\\
146& 73& 220& 110& 55& 166& 83& 250& 125& 376\\
188& 94& 47& 142& 71& 214& 107& 322& 161& 484\\
242& 121& 364& 182& 91& 274& 137& 412& 206& 103\\
310& 155& 466& 233& 700& 350& 175& 526& 263& 790\\
395& 1186& 593& 1780& 890& 445& 1336& 668& 334& 167\\
502& 251& 754& 377& 1132& 566& 283& 850& 425& 1276\\
638& 319& 958& 479& 1438& 719& 2158& 1079& 3238& 1619\\
4858& 2429& 7288& 3644& 1822& 911& 2734& 1367& 4102& 2051\\
6154& 3077& 9232& 4616& 2308& 1154& 577& 1732& 866& 433\\
1300& 650& 325& 976& 488& 244& 122& 61& 184& 92\\
46& 23& 70& 35& 106& 53& 160& 80& 40& 20\\
10& 5& 16& 8& 4& 2& 1& \\

1383&&&&&&&&&\\
4150& 2075& 6226& 3113& 9340& 4670& 2335& 7006& 3503& 10510\\
5255& 15766& 7883& 23650& 11825& 35476& 17738& 8869& 26608& 13304\\
6652& 3326& 1663& 4990& 2495& 7486& 3743& 11230& 5615& 16846\\
8423& 25270& 12635& 37906& 18953& 56860& 28430& 14215& 42646& 21323\\
63970& 31985& 95956& 47978& 23989& 71968& 35984& 17992& 8996& 4498\\
2249& 6748& 3374& 1687& 5062& 2531& 7594& 3797& 11392& 5696\\
2848& 1424& 712& 356& 178& 89& 268& 134& 67& 202\\
101& 304& 152& 76& 38& 19& 58& 29& 88& 44\\
22& 11& 34& 17& 52& 26& 13& 40& 20& 10\\
5& 16& 8& 4& 2& 1& \\

1384&&&&&&&&&\\
692& 346& 173& 520& 260& 130& 65& 196& 98& 49\\
148& 74& 37& 112& 56& 28& 14& 7& 22& 11\\
34& 17& 52& 26& 13& 40& 20& 10& 5& 16\\
8& 4& 2& 1& \\

1385&&&&&&&&&\\
4156& 2078& 1039& 3118& 1559& 4678& 2339& 7018& 3509& 10528\\
5264& 2632& 1316& 658& 329& 988& 494& 247& 742& 371\\
1114& 557& 1672& 836& 418& 209& 628& 314& 157& 472\\
236& 118& 59& 178& 89& 268& 134& 67& 202& 101\\
304& 152& 76& 38& 19& 58& 29& 88& 44& 22\\
11& 34& 17& 52& 26& 13& 40& 20& 10& 5\\
16& 8& 4& 2& 1& \\

1386&&&&&&&&&\\
693& 2080& 1040& 520& 260& 130& 65& 196& 98& 49\\
148& 74& 37& 112& 56& 28& 14& 7& 22& 11\\
34& 17& 52& 26& 13& 40& 20& 10& 5& 16\\
8& 4& 2& 1& \\

1387&&&&&&&&&\\
4162& 2081& 6244& 3122& 1561& 4684& 2342& 1171& 3514& 1757\\
5272& 2636& 1318& 659& 1978& 989& 2968& 1484& 742& 371\\
1114& 557& 1672& 836& 418& 209& 628& 314& 157& 472\\
236& 118& 59& 178& 89& 268& 134& 67& 202& 101\\
304& 152& 76& 38& 19& 58& 29& 88& 44& 22\\
11& 34& 17& 52& 26& 13& 40& 20& 10& 5\\
16& 8& 4& 2& 1& \\

1388&&&&&&&&&\\
694& 347& 1042& 521& 1564& 782& 391& 1174& 587& 1762\\
881& 2644& 1322& 661& 1984& 992& 496& 248& 124& 62\\
31& 94& 47& 142& 71& 214& 107& 322& 161& 484\\
242& 121& 364& 182& 91& 274& 137& 412& 206& 103\\
310& 155& 466& 233& 700& 350& 175& 526& 263& 790\\
395& 1186& 593& 1780& 890& 445& 1336& 668& 334& 167\\
502& 251& 754& 377& 1132& 566& 283& 850& 425& 1276\\
638& 319& 958& 479& 1438& 719& 2158& 1079& 3238& 1619\\
4858& 2429& 7288& 3644& 1822& 911& 2734& 1367& 4102& 2051\\
6154& 3077& 9232& 4616& 2308& 1154& 577& 1732& 866& 433\\
1300& 650& 325& 976& 488& 244& 122& 61& 184& 92\\
46& 23& 70& 35& 106& 53& 160& 80& 40& 20\\
10& 5& 16& 8& 4& 2& 1& \\

1389&&&&&&&&&\\
4168& 2084& 1042& 521& 1564& 782& 391& 1174& 587& 1762\\
881& 2644& 1322& 661& 1984& 992& 496& 248& 124& 62\\
31& 94& 47& 142& 71& 214& 107& 322& 161& 484\\
242& 121& 364& 182& 91& 274& 137& 412& 206& 103\\
310& 155& 466& 233& 700& 350& 175& 526& 263& 790\\
395& 1186& 593& 1780& 890& 445& 1336& 668& 334& 167\\
502& 251& 754& 377& 1132& 566& 283& 850& 425& 1276\\
638& 319& 958& 479& 1438& 719& 2158& 1079& 3238& 1619\\
4858& 2429& 7288& 3644& 1822& 911& 2734& 1367& 4102& 2051\\
6154& 3077& 9232& 4616& 2308& 1154& 577& 1732& 866& 433\\
1300& 650& 325& 976& 488& 244& 122& 61& 184& 92\\
46& 23& 70& 35& 106& 53& 160& 80& 40& 20\\
10& 5& 16& 8& 4& 2& 1& \\

1390&&&&&&&&&\\
695& 2086& 1043& 3130& 1565& 4696& 2348& 1174& 587& 1762\\
881& 2644& 1322& 661& 1984& 992& 496& 248& 124& 62\\
31& 94& 47& 142& 71& 214& 107& 322& 161& 484\\
242& 121& 364& 182& 91& 274& 137& 412& 206& 103\\
310& 155& 466& 233& 700& 350& 175& 526& 263& 790\\
395& 1186& 593& 1780& 890& 445& 1336& 668& 334& 167\\
502& 251& 754& 377& 1132& 566& 283& 850& 425& 1276\\
638& 319& 958& 479& 1438& 719& 2158& 1079& 3238& 1619\\
4858& 2429& 7288& 3644& 1822& 911& 2734& 1367& 4102& 2051\\
6154& 3077& 9232& 4616& 2308& 1154& 577& 1732& 866& 433\\
1300& 650& 325& 976& 488& 244& 122& 61& 184& 92\\
46& 23& 70& 35& 106& 53& 160& 80& 40& 20\\
10& 5& 16& 8& 4& 2& 1& \\

1391&&&&&&&&&\\
4174& 2087& 6262& 3131& 9394& 4697& 14092& 7046& 3523& 10570\\
5285& 15856& 7928& 3964& 1982& 991& 2974& 1487& 4462& 2231\\
6694& 3347& 10042& 5021& 15064& 7532& 3766& 1883& 5650& 2825\\
8476& 4238& 2119& 6358& 3179& 9538& 4769& 14308& 7154& 3577\\
10732& 5366& 2683& 8050& 4025& 12076& 6038& 3019& 9058& 4529\\
13588& 6794& 3397& 10192& 5096& 2548& 1274& 637& 1912& 956\\
478& 239& 718& 359& 1078& 539& 1618& 809& 2428& 1214\\
607& 1822& 911& 2734& 1367& 4102& 2051& 6154& 3077& 9232\\
4616& 2308& 1154& 577& 1732& 866& 433& 1300& 650& 325\\
976& 488& 244& 122& 61& 184& 92& 46& 23& 70\\
35& 106& 53& 160& 80& 40& 20& 10& 5& 16\\
8& 4& 2& 1& \\

1392&&&&&&&&&\\
696& 348& 174& 87& 262& 131& 394& 197& 592& 296\\
148& 74& 37& 112& 56& 28& 14& 7& 22& 11\\
34& 17& 52& 26& 13& 40& 20& 10& 5& 16\\
8& 4& 2& 1& \\

1393&&&&&&&&&\\
4180& 2090& 1045& 3136& 1568& 784& 392& 196& 98& 49\\
148& 74& 37& 112& 56& 28& 14& 7& 22& 11\\
34& 17& 52& 26& 13& 40& 20& 10& 5& 16\\
8& 4& 2& 1& \\

1394&&&&&&&&&\\
697& 2092& 1046& 523& 1570& 785& 2356& 1178& 589& 1768\\
884& 442& 221& 664& 332& 166& 83& 250& 125& 376\\
188& 94& 47& 142& 71& 214& 107& 322& 161& 484\\
242& 121& 364& 182& 91& 274& 137& 412& 206& 103\\
310& 155& 466& 233& 700& 350& 175& 526& 263& 790\\
395& 1186& 593& 1780& 890& 445& 1336& 668& 334& 167\\
502& 251& 754& 377& 1132& 566& 283& 850& 425& 1276\\
638& 319& 958& 479& 1438& 719& 2158& 1079& 3238& 1619\\
4858& 2429& 7288& 3644& 1822& 911& 2734& 1367& 4102& 2051\\
6154& 3077& 9232& 4616& 2308& 1154& 577& 1732& 866& 433\\
1300& 650& 325& 976& 488& 244& 122& 61& 184& 92\\
46& 23& 70& 35& 106& 53& 160& 80& 40& 20\\
10& 5& 16& 8& 4& 2& 1& \\

1395&&&&&&&&&\\
4186& 2093& 6280& 3140& 1570& 785& 2356& 1178& 589& 1768\\
884& 442& 221& 664& 332& 166& 83& 250& 125& 376\\
188& 94& 47& 142& 71& 214& 107& 322& 161& 484\\
242& 121& 364& 182& 91& 274& 137& 412& 206& 103\\
310& 155& 466& 233& 700& 350& 175& 526& 263& 790\\
395& 1186& 593& 1780& 890& 445& 1336& 668& 334& 167\\
502& 251& 754& 377& 1132& 566& 283& 850& 425& 1276\\
638& 319& 958& 479& 1438& 719& 2158& 1079& 3238& 1619\\
4858& 2429& 7288& 3644& 1822& 911& 2734& 1367& 4102& 2051\\
6154& 3077& 9232& 4616& 2308& 1154& 577& 1732& 866& 433\\
1300& 650& 325& 976& 488& 244& 122& 61& 184& 92\\
46& 23& 70& 35& 106& 53& 160& 80& 40& 20\\
10& 5& 16& 8& 4& 2& 1& \\

1396&&&&&&&&&\\
698& 349& 1048& 524& 262& 131& 394& 197& 592& 296\\
148& 74& 37& 112& 56& 28& 14& 7& 22& 11\\
34& 17& 52& 26& 13& 40& 20& 10& 5& 16\\
8& 4& 2& 1& \\

1397&&&&&&&&&\\
4192& 2096& 1048& 524& 262& 131& 394& 197& 592& 296\\
148& 74& 37& 112& 56& 28& 14& 7& 22& 11\\
34& 17& 52& 26& 13& 40& 20& 10& 5& 16\\
8& 4& 2& 1& \\

1398&&&&&&&&&\\
699& 2098& 1049& 3148& 1574& 787& 2362& 1181& 3544& 1772\\
886& 443& 1330& 665& 1996& 998& 499& 1498& 749& 2248\\
1124& 562& 281& 844& 422& 211& 634& 317& 952& 476\\
238& 119& 358& 179& 538& 269& 808& 404& 202& 101\\
304& 152& 76& 38& 19& 58& 29& 88& 44& 22\\
11& 34& 17& 52& 26& 13& 40& 20& 10& 5\\
16& 8& 4& 2& 1& \\

1399&&&&&&&&&\\
4198& 2099& 6298& 3149& 9448& 4724& 2362& 1181& 3544& 1772\\
886& 443& 1330& 665& 1996& 998& 499& 1498& 749& 2248\\
1124& 562& 281& 844& 422& 211& 634& 317& 952& 476\\
238& 119& 358& 179& 538& 269& 808& 404& 202& 101\\
304& 152& 76& 38& 19& 58& 29& 88& 44& 22\\
11& 34& 17& 52& 26& 13& 40& 20& 10& 5\\
16& 8& 4& 2& 1& \\

1400&&&&&&&&&\\
700& 350& 175& 526& 263& 790& 395& 1186& 593& 1780\\
890& 445& 1336& 668& 334& 167& 502& 251& 754& 377\\
1132& 566& 283& 850& 425& 1276& 638& 319& 958& 479\\
1438& 719& 2158& 1079& 3238& 1619& 4858& 2429& 7288& 3644\\
1822& 911& 2734& 1367& 4102& 2051& 6154& 3077& 9232& 4616\\
2308& 1154& 577& 1732& 866& 433& 1300& 650& 325& 976\\
488& 244& 122& 61& 184& 92& 46& 23& 70& 35\\
106& 53& 160& 80& 40& 20& 10& 5& 16& 8\\
4& 2& 1& \\

1401&&&&&&&&&\\
4204& 2102& 1051& 3154& 1577& 4732& 2366& 1183& 3550& 1775\\
5326& 2663& 7990& 3995& 11986& 5993& 17980& 8990& 4495& 13486\\
6743& 20230& 10115& 30346& 15173& 45520& 22760& 11380& 5690& 2845\\
8536& 4268& 2134& 1067& 3202& 1601& 4804& 2402& 1201& 3604\\
1802& 901& 2704& 1352& 676& 338& 169& 508& 254& 127\\
382& 191& 574& 287& 862& 431& 1294& 647& 1942& 971\\
2914& 1457& 4372& 2186& 1093& 3280& 1640& 820& 410& 205\\
616& 308& 154& 77& 232& 116& 58& 29& 88& 44\\
22& 11& 34& 17& 52& 26& 13& 40& 20& 10\\
5& 16& 8& 4& 2& 1& \\

1402&&&&&&&&&\\
701& 2104& 1052& 526& 263& 790& 395& 1186& 593& 1780\\
890& 445& 1336& 668& 334& 167& 502& 251& 754& 377\\
1132& 566& 283& 850& 425& 1276& 638& 319& 958& 479\\
1438& 719& 2158& 1079& 3238& 1619& 4858& 2429& 7288& 3644\\
1822& 911& 2734& 1367& 4102& 2051& 6154& 3077& 9232& 4616\\
2308& 1154& 577& 1732& 866& 433& 1300& 650& 325& 976\\
488& 244& 122& 61& 184& 92& 46& 23& 70& 35\\
106& 53& 160& 80& 40& 20& 10& 5& 16& 8\\
4& 2& 1& \\

1403&&&&&&&&&\\
4210& 2105& 6316& 3158& 1579& 4738& 2369& 7108& 3554& 1777\\
5332& 2666& 1333& 4000& 2000& 1000& 500& 250& 125& 376\\
188& 94& 47& 142& 71& 214& 107& 322& 161& 484\\
242& 121& 364& 182& 91& 274& 137& 412& 206& 103\\
310& 155& 466& 233& 700& 350& 175& 526& 263& 790\\
395& 1186& 593& 1780& 890& 445& 1336& 668& 334& 167\\
502& 251& 754& 377& 1132& 566& 283& 850& 425& 1276\\
638& 319& 958& 479& 1438& 719& 2158& 1079& 3238& 1619\\
4858& 2429& 7288& 3644& 1822& 911& 2734& 1367& 4102& 2051\\
6154& 3077& 9232& 4616& 2308& 1154& 577& 1732& 866& 433\\
1300& 650& 325& 976& 488& 244& 122& 61& 184& 92\\
46& 23& 70& 35& 106& 53& 160& 80& 40& 20\\
10& 5& 16& 8& 4& 2& 1& \\

1404&&&&&&&&&\\
702& 351& 1054& 527& 1582& 791& 2374& 1187& 3562& 1781\\
5344& 2672& 1336& 668& 334& 167& 502& 251& 754& 377\\
1132& 566& 283& 850& 425& 1276& 638& 319& 958& 479\\
1438& 719& 2158& 1079& 3238& 1619& 4858& 2429& 7288& 3644\\
1822& 911& 2734& 1367& 4102& 2051& 6154& 3077& 9232& 4616\\
2308& 1154& 577& 1732& 866& 433& 1300& 650& 325& 976\\
488& 244& 122& 61& 184& 92& 46& 23& 70& 35\\
106& 53& 160& 80& 40& 20& 10& 5& 16& 8\\
4& 2& 1& \\

1405&&&&&&&&&\\
4216& 2108& 1054& 527& 1582& 791& 2374& 1187& 3562& 1781\\
5344& 2672& 1336& 668& 334& 167& 502& 251& 754& 377\\
1132& 566& 283& 850& 425& 1276& 638& 319& 958& 479\\
1438& 719& 2158& 1079& 3238& 1619& 4858& 2429& 7288& 3644\\
1822& 911& 2734& 1367& 4102& 2051& 6154& 3077& 9232& 4616\\
2308& 1154& 577& 1732& 866& 433& 1300& 650& 325& 976\\
488& 244& 122& 61& 184& 92& 46& 23& 70& 35\\
106& 53& 160& 80& 40& 20& 10& 5& 16& 8\\
4& 2& 1& \\

1406&&&&&&&&&\\
703& 2110& 1055& 3166& 1583& 4750& 2375& 7126& 3563& 10690\\
5345& 16036& 8018& 4009& 12028& 6014& 3007& 9022& 4511& 13534\\
6767& 20302& 10151& 30454& 15227& 45682& 22841& 68524& 34262& 17131\\
51394& 25697& 77092& 38546& 19273& 57820& 28910& 14455& 43366& 21683\\
65050& 32525& 97576& 48788& 24394& 12197& 36592& 18296& 9148& 4574\\
2287& 6862& 3431& 10294& 5147& 15442& 7721& 23164& 11582& 5791\\
17374& 8687& 26062& 13031& 39094& 19547& 58642& 29321& 87964& 43982\\
21991& 65974& 32987& 98962& 49481& 148444& 74222& 37111& 111334& 55667\\
167002& 83501& 250504& 125252& 62626& 31313& 93940& 46970& 23485& 70456\\
35228& 17614& 8807& 26422& 13211& 39634& 19817& 59452& 29726& 14863\\
44590& 22295& 66886& 33443& 100330& 50165& 150496& 75248& 37624& 18812\\
9406& 4703& 14110& 7055& 21166& 10583& 31750& 15875& 47626& 23813\\
71440& 35720& 17860& 8930& 4465& 13396& 6698& 3349& 10048& 5024\\
2512& 1256& 628& 314& 157& 472& 236& 118& 59& 178\\
89& 268& 134& 67& 202& 101& 304& 152& 76& 38\\
19& 58& 29& 88& 44& 22& 11& 34& 17& 52\\
26& 13& 40& 20& 10& 5& 16& 8& 4& 2\\
1& \\

1407&&&&&&&&&\\
4222& 2111& 6334& 3167& 9502& 4751& 14254& 7127& 21382& 10691\\
32074& 16037& 48112& 24056& 12028& 6014& 3007& 9022& 4511& 13534\\
6767& 20302& 10151& 30454& 15227& 45682& 22841& 68524& 34262& 17131\\
51394& 25697& 77092& 38546& 19273& 57820& 28910& 14455& 43366& 21683\\
65050& 32525& 97576& 48788& 24394& 12197& 36592& 18296& 9148& 4574\\
2287& 6862& 3431& 10294& 5147& 15442& 7721& 23164& 11582& 5791\\
17374& 8687& 26062& 13031& 39094& 19547& 58642& 29321& 87964& 43982\\
21991& 65974& 32987& 98962& 49481& 148444& 74222& 37111& 111334& 55667\\
167002& 83501& 250504& 125252& 62626& 31313& 93940& 46970& 23485& 70456\\
35228& 17614& 8807& 26422& 13211& 39634& 19817& 59452& 29726& 14863\\
44590& 22295& 66886& 33443& 100330& 50165& 150496& 75248& 37624& 18812\\
9406& 4703& 14110& 7055& 21166& 10583& 31750& 15875& 47626& 23813\\
71440& 35720& 17860& 8930& 4465& 13396& 6698& 3349& 10048& 5024\\
2512& 1256& 628& 314& 157& 472& 236& 118& 59& 178\\
89& 268& 134& 67& 202& 101& 304& 152& 76& 38\\
19& 58& 29& 88& 44& 22& 11& 34& 17& 52\\
26& 13& 40& 20& 10& 5& 16& 8& 4& 2\\
1& \\

1408&&&&&&&&&\\
704& 352& 176& 88& 44& 22& 11& 34& 17& 52\\
26& 13& 40& 20& 10& 5& 16& 8& 4& 2\\
1& \\

1409&&&&&&&&&\\
4228& 2114& 1057& 3172& 1586& 793& 2380& 1190& 595& 1786\\
893& 2680& 1340& 670& 335& 1006& 503& 1510& 755& 2266\\
1133& 3400& 1700& 850& 425& 1276& 638& 319& 958& 479\\
1438& 719& 2158& 1079& 3238& 1619& 4858& 2429& 7288& 3644\\
1822& 911& 2734& 1367& 4102& 2051& 6154& 3077& 9232& 4616\\
2308& 1154& 577& 1732& 866& 433& 1300& 650& 325& 976\\
488& 244& 122& 61& 184& 92& 46& 23& 70& 35\\
106& 53& 160& 80& 40& 20& 10& 5& 16& 8\\
4& 2& 1& \\

1410&&&&&&&&&\\
705& 2116& 1058& 529& 1588& 794& 397& 1192& 596& 298\\
149& 448& 224& 112& 56& 28& 14& 7& 22& 11\\
34& 17& 52& 26& 13& 40& 20& 10& 5& 16\\
8& 4& 2& 1& \\

1411&&&&&&&&&\\
4234& 2117& 6352& 3176& 1588& 794& 397& 1192& 596& 298\\
149& 448& 224& 112& 56& 28& 14& 7& 22& 11\\
34& 17& 52& 26& 13& 40& 20& 10& 5& 16\\
8& 4& 2& 1& \\

1412&&&&&&&&&\\
706& 353& 1060& 530& 265& 796& 398& 199& 598& 299\\
898& 449& 1348& 674& 337& 1012& 506& 253& 760& 380\\
190& 95& 286& 143& 430& 215& 646& 323& 970& 485\\
1456& 728& 364& 182& 91& 274& 137& 412& 206& 103\\
310& 155& 466& 233& 700& 350& 175& 526& 263& 790\\
395& 1186& 593& 1780& 890& 445& 1336& 668& 334& 167\\
502& 251& 754& 377& 1132& 566& 283& 850& 425& 1276\\
638& 319& 958& 479& 1438& 719& 2158& 1079& 3238& 1619\\
4858& 2429& 7288& 3644& 1822& 911& 2734& 1367& 4102& 2051\\
6154& 3077& 9232& 4616& 2308& 1154& 577& 1732& 866& 433\\
1300& 650& 325& 976& 488& 244& 122& 61& 184& 92\\
46& 23& 70& 35& 106& 53& 160& 80& 40& 20\\
10& 5& 16& 8& 4& 2& 1& \\

1413&&&&&&&&&\\
4240& 2120& 1060& 530& 265& 796& 398& 199& 598& 299\\
898& 449& 1348& 674& 337& 1012& 506& 253& 760& 380\\
190& 95& 286& 143& 430& 215& 646& 323& 970& 485\\
1456& 728& 364& 182& 91& 274& 137& 412& 206& 103\\
310& 155& 466& 233& 700& 350& 175& 526& 263& 790\\
395& 1186& 593& 1780& 890& 445& 1336& 668& 334& 167\\
502& 251& 754& 377& 1132& 566& 283& 850& 425& 1276\\
638& 319& 958& 479& 1438& 719& 2158& 1079& 3238& 1619\\
4858& 2429& 7288& 3644& 1822& 911& 2734& 1367& 4102& 2051\\
6154& 3077& 9232& 4616& 2308& 1154& 577& 1732& 866& 433\\
1300& 650& 325& 976& 488& 244& 122& 61& 184& 92\\
46& 23& 70& 35& 106& 53& 160& 80& 40& 20\\
10& 5& 16& 8& 4& 2& 1& \\

1414&&&&&&&&&\\
707& 2122& 1061& 3184& 1592& 796& 398& 199& 598& 299\\
898& 449& 1348& 674& 337& 1012& 506& 253& 760& 380\\
190& 95& 286& 143& 430& 215& 646& 323& 970& 485\\
1456& 728& 364& 182& 91& 274& 137& 412& 206& 103\\
310& 155& 466& 233& 700& 350& 175& 526& 263& 790\\
395& 1186& 593& 1780& 890& 445& 1336& 668& 334& 167\\
502& 251& 754& 377& 1132& 566& 283& 850& 425& 1276\\
638& 319& 958& 479& 1438& 719& 2158& 1079& 3238& 1619\\
4858& 2429& 7288& 3644& 1822& 911& 2734& 1367& 4102& 2051\\
6154& 3077& 9232& 4616& 2308& 1154& 577& 1732& 866& 433\\
1300& 650& 325& 976& 488& 244& 122& 61& 184& 92\\
46& 23& 70& 35& 106& 53& 160& 80& 40& 20\\
10& 5& 16& 8& 4& 2& 1& \\

1415&&&&&&&&&\\
4246& 2123& 6370& 3185& 9556& 4778& 2389& 7168& 3584& 1792\\
896& 448& 224& 112& 56& 28& 14& 7& 22& 11\\
34& 17& 52& 26& 13& 40& 20& 10& 5& 16\\
8& 4& 2& 1& \\

1416&&&&&&&&&\\
708& 354& 177& 532& 266& 133& 400& 200& 100& 50\\
25& 76& 38& 19& 58& 29& 88& 44& 22& 11\\
34& 17& 52& 26& 13& 40& 20& 10& 5& 16\\
8& 4& 2& 1& \\

1417&&&&&&&&&\\
4252& 2126& 1063& 3190& 1595& 4786& 2393& 7180& 3590& 1795\\
5386& 2693& 8080& 4040& 2020& 1010& 505& 1516& 758& 379\\
1138& 569& 1708& 854& 427& 1282& 641& 1924& 962& 481\\
1444& 722& 361& 1084& 542& 271& 814& 407& 1222& 611\\
1834& 917& 2752& 1376& 688& 344& 172& 86& 43& 130\\
65& 196& 98& 49& 148& 74& 37& 112& 56& 28\\
14& 7& 22& 11& 34& 17& 52& 26& 13& 40\\
20& 10& 5& 16& 8& 4& 2& 1& \\

1418&&&&&&&&&\\
709& 2128& 1064& 532& 266& 133& 400& 200& 100& 50\\
25& 76& 38& 19& 58& 29& 88& 44& 22& 11\\
34& 17& 52& 26& 13& 40& 20& 10& 5& 16\\
8& 4& 2& 1& \\

1419&&&&&&&&&\\
4258& 2129& 6388& 3194& 1597& 4792& 2396& 1198& 599& 1798\\
899& 2698& 1349& 4048& 2024& 1012& 506& 253& 760& 380\\
190& 95& 286& 143& 430& 215& 646& 323& 970& 485\\
1456& 728& 364& 182& 91& 274& 137& 412& 206& 103\\
310& 155& 466& 233& 700& 350& 175& 526& 263& 790\\
395& 1186& 593& 1780& 890& 445& 1336& 668& 334& 167\\
502& 251& 754& 377& 1132& 566& 283& 850& 425& 1276\\
638& 319& 958& 479& 1438& 719& 2158& 1079& 3238& 1619\\
4858& 2429& 7288& 3644& 1822& 911& 2734& 1367& 4102& 2051\\
6154& 3077& 9232& 4616& 2308& 1154& 577& 1732& 866& 433\\
1300& 650& 325& 976& 488& 244& 122& 61& 184& 92\\
46& 23& 70& 35& 106& 53& 160& 80& 40& 20\\
10& 5& 16& 8& 4& 2& 1& \\

1420&&&&&&&&&\\
710& 355& 1066& 533& 1600& 800& 400& 200& 100& 50\\
25& 76& 38& 19& 58& 29& 88& 44& 22& 11\\
34& 17& 52& 26& 13& 40& 20& 10& 5& 16\\
8& 4& 2& 1& \\

1421&&&&&&&&&\\
4264& 2132& 1066& 533& 1600& 800& 400& 200& 100& 50\\
25& 76& 38& 19& 58& 29& 88& 44& 22& 11\\
34& 17& 52& 26& 13& 40& 20& 10& 5& 16\\
8& 4& 2& 1& \\

1422&&&&&&&&&\\
711& 2134& 1067& 3202& 1601& 4804& 2402& 1201& 3604& 1802\\
901& 2704& 1352& 676& 338& 169& 508& 254& 127& 382\\
191& 574& 287& 862& 431& 1294& 647& 1942& 971& 2914\\
1457& 4372& 2186& 1093& 3280& 1640& 820& 410& 205& 616\\
308& 154& 77& 232& 116& 58& 29& 88& 44& 22\\
11& 34& 17& 52& 26& 13& 40& 20& 10& 5\\
16& 8& 4& 2& 1& \\

1423&&&&&&&&&\\
4270& 2135& 6406& 3203& 9610& 4805& 14416& 7208& 3604& 1802\\
901& 2704& 1352& 676& 338& 169& 508& 254& 127& 382\\
191& 574& 287& 862& 431& 1294& 647& 1942& 971& 2914\\
1457& 4372& 2186& 1093& 3280& 1640& 820& 410& 205& 616\\
308& 154& 77& 232& 116& 58& 29& 88& 44& 22\\
11& 34& 17& 52& 26& 13& 40& 20& 10& 5\\
16& 8& 4& 2& 1& \\

1424&&&&&&&&&\\
712& 356& 178& 89& 268& 134& 67& 202& 101& 304\\
152& 76& 38& 19& 58& 29& 88& 44& 22& 11\\
34& 17& 52& 26& 13& 40& 20& 10& 5& 16\\
8& 4& 2& 1& \\

1425&&&&&&&&&\\
4276& 2138& 1069& 3208& 1604& 802& 401& 1204& 602& 301\\
904& 452& 226& 113& 340& 170& 85& 256& 128& 64\\
32& 16& 8& 4& 2& 1& \\

1426&&&&&&&&&\\
713& 2140& 1070& 535& 1606& 803& 2410& 1205& 3616& 1808\\
904& 452& 226& 113& 340& 170& 85& 256& 128& 64\\
32& 16& 8& 4& 2& 1& \\

1427&&&&&&&&&\\
4282& 2141& 6424& 3212& 1606& 803& 2410& 1205& 3616& 1808\\
904& 452& 226& 113& 340& 170& 85& 256& 128& 64\\
32& 16& 8& 4& 2& 1& \\

1428&&&&&&&&&\\
714& 357& 1072& 536& 268& 134& 67& 202& 101& 304\\
152& 76& 38& 19& 58& 29& 88& 44& 22& 11\\
34& 17& 52& 26& 13& 40& 20& 10& 5& 16\\
8& 4& 2& 1& \\

1429&&&&&&&&&\\
4288& 2144& 1072& 536& 268& 134& 67& 202& 101& 304\\
152& 76& 38& 19& 58& 29& 88& 44& 22& 11\\
34& 17& 52& 26& 13& 40& 20& 10& 5& 16\\
8& 4& 2& 1& \\

1430&&&&&&&&&\\
715& 2146& 1073& 3220& 1610& 805& 2416& 1208& 604& 302\\
151& 454& 227& 682& 341& 1024& 512& 256& 128& 64\\
32& 16& 8& 4& 2& 1& \\

1431&&&&&&&&&\\
4294& 2147& 6442& 3221& 9664& 4832& 2416& 1208& 604& 302\\
151& 454& 227& 682& 341& 1024& 512& 256& 128& 64\\
32& 16& 8& 4& 2& 1& \\

1432&&&&&&&&&\\
716& 358& 179& 538& 269& 808& 404& 202& 101& 304\\
152& 76& 38& 19& 58& 29& 88& 44& 22& 11\\
34& 17& 52& 26& 13& 40& 20& 10& 5& 16\\
8& 4& 2& 1& \\

1433&&&&&&&&&\\
4300& 2150& 1075& 3226& 1613& 4840& 2420& 1210& 605& 1816\\
908& 454& 227& 682& 341& 1024& 512& 256& 128& 64\\
32& 16& 8& 4& 2& 1& \\

1434&&&&&&&&&\\
717& 2152& 1076& 538& 269& 808& 404& 202& 101& 304\\
152& 76& 38& 19& 58& 29& 88& 44& 22& 11\\
34& 17& 52& 26& 13& 40& 20& 10& 5& 16\\
8& 4& 2& 1& \\

1435&&&&&&&&&\\
4306& 2153& 6460& 3230& 1615& 4846& 2423& 7270& 3635& 10906\\
5453& 16360& 8180& 4090& 2045& 6136& 3068& 1534& 767& 2302\\
1151& 3454& 1727& 5182& 2591& 7774& 3887& 11662& 5831& 17494\\
8747& 26242& 13121& 39364& 19682& 9841& 29524& 14762& 7381& 22144\\
11072& 5536& 2768& 1384& 692& 346& 173& 520& 260& 130\\
65& 196& 98& 49& 148& 74& 37& 112& 56& 28\\
14& 7& 22& 11& 34& 17& 52& 26& 13& 40\\
20& 10& 5& 16& 8& 4& 2& 1& \\

1436&&&&&&&&&\\
718& 359& 1078& 539& 1618& 809& 2428& 1214& 607& 1822\\
911& 2734& 1367& 4102& 2051& 6154& 3077& 9232& 4616& 2308\\
1154& 577& 1732& 866& 433& 1300& 650& 325& 976& 488\\
244& 122& 61& 184& 92& 46& 23& 70& 35& 106\\
53& 160& 80& 40& 20& 10& 5& 16& 8& 4\\
2& 1& \\

1437&&&&&&&&&\\
4312& 2156& 1078& 539& 1618& 809& 2428& 1214& 607& 1822\\
911& 2734& 1367& 4102& 2051& 6154& 3077& 9232& 4616& 2308\\
1154& 577& 1732& 866& 433& 1300& 650& 325& 976& 488\\
244& 122& 61& 184& 92& 46& 23& 70& 35& 106\\
53& 160& 80& 40& 20& 10& 5& 16& 8& 4\\
2& 1& \\

1438&&&&&&&&&\\
719& 2158& 1079& 3238& 1619& 4858& 2429& 7288& 3644& 1822\\
911& 2734& 1367& 4102& 2051& 6154& 3077& 9232& 4616& 2308\\
1154& 577& 1732& 866& 433& 1300& 650& 325& 976& 488\\
244& 122& 61& 184& 92& 46& 23& 70& 35& 106\\
53& 160& 80& 40& 20& 10& 5& 16& 8& 4\\
2& 1& \\

1439&&&&&&&&&\\
4318& 2159& 6478& 3239& 9718& 4859& 14578& 7289& 21868& 10934\\
5467& 16402& 8201& 24604& 12302& 6151& 18454& 9227& 27682& 13841\\
41524& 20762& 10381& 31144& 15572& 7786& 3893& 11680& 5840& 2920\\
1460& 730& 365& 1096& 548& 274& 137& 412& 206& 103\\
310& 155& 466& 233& 700& 350& 175& 526& 263& 790\\
395& 1186& 593& 1780& 890& 445& 1336& 668& 334& 167\\
502& 251& 754& 377& 1132& 566& 283& 850& 425& 1276\\
638& 319& 958& 479& 1438& 719& 2158& 1079& 3238& 1619\\
4858& 2429& 7288& 3644& 1822& 911& 2734& 1367& 4102& 2051\\
6154& 3077& 9232& 4616& 2308& 1154& 577& 1732& 866& 433\\
1300& 650& 325& 976& 488& 244& 122& 61& 184& 92\\
46& 23& 70& 35& 106& 53& 160& 80& 40& 20\\
10& 5& 16& 8& 4& 2& 1& \\

1440&&&&&&&&&\\
720& 360& 180& 90& 45& 136& 68& 34& 17& 52\\
26& 13& 40& 20& 10& 5& 16& 8& 4& 2\\
1& \\

1441&&&&&&&&&\\
4324& 2162& 1081& 3244& 1622& 811& 2434& 1217& 3652& 1826\\
913& 2740& 1370& 685& 2056& 1028& 514& 257& 772& 386\\
193& 580& 290& 145& 436& 218& 109& 328& 164& 82\\
41& 124& 62& 31& 94& 47& 142& 71& 214& 107\\
322& 161& 484& 242& 121& 364& 182& 91& 274& 137\\
412& 206& 103& 310& 155& 466& 233& 700& 350& 175\\
526& 263& 790& 395& 1186& 593& 1780& 890& 445& 1336\\
668& 334& 167& 502& 251& 754& 377& 1132& 566& 283\\
850& 425& 1276& 638& 319& 958& 479& 1438& 719& 2158\\
1079& 3238& 1619& 4858& 2429& 7288& 3644& 1822& 911& 2734\\
1367& 4102& 2051& 6154& 3077& 9232& 4616& 2308& 1154& 577\\
1732& 866& 433& 1300& 650& 325& 976& 488& 244& 122\\
61& 184& 92& 46& 23& 70& 35& 106& 53& 160\\
80& 40& 20& 10& 5& 16& 8& 4& 2& 1\\

1442&&&&&&&&&\\
721& 2164& 1082& 541& 1624& 812& 406& 203& 610& 305\\
916& 458& 229& 688& 344& 172& 86& 43& 130& 65\\
196& 98& 49& 148& 74& 37& 112& 56& 28& 14\\
7& 22& 11& 34& 17& 52& 26& 13& 40& 20\\
10& 5& 16& 8& 4& 2& 1& \\

1443&&&&&&&&&\\
4330& 2165& 6496& 3248& 1624& 812& 406& 203& 610& 305\\
916& 458& 229& 688& 344& 172& 86& 43& 130& 65\\
196& 98& 49& 148& 74& 37& 112& 56& 28& 14\\
7& 22& 11& 34& 17& 52& 26& 13& 40& 20\\
10& 5& 16& 8& 4& 2& 1& \\

1444&&&&&&&&&\\
722& 361& 1084& 542& 271& 814& 407& 1222& 611& 1834\\
917& 2752& 1376& 688& 344& 172& 86& 43& 130& 65\\
196& 98& 49& 148& 74& 37& 112& 56& 28& 14\\
7& 22& 11& 34& 17& 52& 26& 13& 40& 20\\
10& 5& 16& 8& 4& 2& 1& \\

1445&&&&&&&&&\\
4336& 2168& 1084& 542& 271& 814& 407& 1222& 611& 1834\\
917& 2752& 1376& 688& 344& 172& 86& 43& 130& 65\\
196& 98& 49& 148& 74& 37& 112& 56& 28& 14\\
7& 22& 11& 34& 17& 52& 26& 13& 40& 20\\
10& 5& 16& 8& 4& 2& 1& \\

1446&&&&&&&&&\\
723& 2170& 1085& 3256& 1628& 814& 407& 1222& 611& 1834\\
917& 2752& 1376& 688& 344& 172& 86& 43& 130& 65\\
196& 98& 49& 148& 74& 37& 112& 56& 28& 14\\
7& 22& 11& 34& 17& 52& 26& 13& 40& 20\\
10& 5& 16& 8& 4& 2& 1& \\

1447&&&&&&&&&\\
4342& 2171& 6514& 3257& 9772& 4886& 2443& 7330& 3665& 10996\\
5498& 2749& 8248& 4124& 2062& 1031& 3094& 1547& 4642& 2321\\
6964& 3482& 1741& 5224& 2612& 1306& 653& 1960& 980& 490\\
245& 736& 368& 184& 92& 46& 23& 70& 35& 106\\
53& 160& 80& 40& 20& 10& 5& 16& 8& 4\\
2& 1& \\

1448&&&&&&&&&\\
724& 362& 181& 544& 272& 136& 68& 34& 17& 52\\
26& 13& 40& 20& 10& 5& 16& 8& 4& 2\\
1& \\

1449&&&&&&&&&\\
4348& 2174& 1087& 3262& 1631& 4894& 2447& 7342& 3671& 11014\\
5507& 16522& 8261& 24784& 12392& 6196& 3098& 1549& 4648& 2324\\
1162& 581& 1744& 872& 436& 218& 109& 328& 164& 82\\
41& 124& 62& 31& 94& 47& 142& 71& 214& 107\\
322& 161& 484& 242& 121& 364& 182& 91& 274& 137\\
412& 206& 103& 310& 155& 466& 233& 700& 350& 175\\
526& 263& 790& 395& 1186& 593& 1780& 890& 445& 1336\\
668& 334& 167& 502& 251& 754& 377& 1132& 566& 283\\
850& 425& 1276& 638& 319& 958& 479& 1438& 719& 2158\\
1079& 3238& 1619& 4858& 2429& 7288& 3644& 1822& 911& 2734\\
1367& 4102& 2051& 6154& 3077& 9232& 4616& 2308& 1154& 577\\
1732& 866& 433& 1300& 650& 325& 976& 488& 244& 122\\
61& 184& 92& 46& 23& 70& 35& 106& 53& 160\\
80& 40& 20& 10& 5& 16& 8& 4& 2& 1\\

1450&&&&&&&&&\\
725& 2176& 1088& 544& 272& 136& 68& 34& 17& 52\\
26& 13& 40& 20& 10& 5& 16& 8& 4& 2\\
1& \\

1451&&&&&&&&&\\
4354& 2177& 6532& 3266& 1633& 4900& 2450& 1225& 3676& 1838\\
919& 2758& 1379& 4138& 2069& 6208& 3104& 1552& 776& 388\\
194& 97& 292& 146& 73& 220& 110& 55& 166& 83\\
250& 125& 376& 188& 94& 47& 142& 71& 214& 107\\
322& 161& 484& 242& 121& 364& 182& 91& 274& 137\\
412& 206& 103& 310& 155& 466& 233& 700& 350& 175\\
526& 263& 790& 395& 1186& 593& 1780& 890& 445& 1336\\
668& 334& 167& 502& 251& 754& 377& 1132& 566& 283\\
850& 425& 1276& 638& 319& 958& 479& 1438& 719& 2158\\
1079& 3238& 1619& 4858& 2429& 7288& 3644& 1822& 911& 2734\\
1367& 4102& 2051& 6154& 3077& 9232& 4616& 2308& 1154& 577\\
1732& 866& 433& 1300& 650& 325& 976& 488& 244& 122\\
61& 184& 92& 46& 23& 70& 35& 106& 53& 160\\
80& 40& 20& 10& 5& 16& 8& 4& 2& 1\\

1452&&&&&&&&&\\
726& 363& 1090& 545& 1636& 818& 409& 1228& 614& 307\\
922& 461& 1384& 692& 346& 173& 520& 260& 130& 65\\
196& 98& 49& 148& 74& 37& 112& 56& 28& 14\\
7& 22& 11& 34& 17& 52& 26& 13& 40& 20\\
10& 5& 16& 8& 4& 2& 1& \\

1453&&&&&&&&&\\
4360& 2180& 1090& 545& 1636& 818& 409& 1228& 614& 307\\
922& 461& 1384& 692& 346& 173& 520& 260& 130& 65\\
196& 98& 49& 148& 74& 37& 112& 56& 28& 14\\
7& 22& 11& 34& 17& 52& 26& 13& 40& 20\\
10& 5& 16& 8& 4& 2& 1& \\

1454&&&&&&&&&\\
727& 2182& 1091& 3274& 1637& 4912& 2456& 1228& 614& 307\\
922& 461& 1384& 692& 346& 173& 520& 260& 130& 65\\
196& 98& 49& 148& 74& 37& 112& 56& 28& 14\\
7& 22& 11& 34& 17& 52& 26& 13& 40& 20\\
10& 5& 16& 8& 4& 2& 1& \\

1455&&&&&&&&&\\
4366& 2183& 6550& 3275& 9826& 4913& 14740& 7370& 3685& 11056\\
5528& 2764& 1382& 691& 2074& 1037& 3112& 1556& 778& 389\\
1168& 584& 292& 146& 73& 220& 110& 55& 166& 83\\
250& 125& 376& 188& 94& 47& 142& 71& 214& 107\\
322& 161& 484& 242& 121& 364& 182& 91& 274& 137\\
412& 206& 103& 310& 155& 466& 233& 700& 350& 175\\
526& 263& 790& 395& 1186& 593& 1780& 890& 445& 1336\\
668& 334& 167& 502& 251& 754& 377& 1132& 566& 283\\
850& 425& 1276& 638& 319& 958& 479& 1438& 719& 2158\\
1079& 3238& 1619& 4858& 2429& 7288& 3644& 1822& 911& 2734\\
1367& 4102& 2051& 6154& 3077& 9232& 4616& 2308& 1154& 577\\
1732& 866& 433& 1300& 650& 325& 976& 488& 244& 122\\
61& 184& 92& 46& 23& 70& 35& 106& 53& 160\\
80& 40& 20& 10& 5& 16& 8& 4& 2& 1\\

1456&&&&&&&&&\\
728& 364& 182& 91& 274& 137& 412& 206& 103& 310\\
155& 466& 233& 700& 350& 175& 526& 263& 790& 395\\
1186& 593& 1780& 890& 445& 1336& 668& 334& 167& 502\\
251& 754& 377& 1132& 566& 283& 850& 425& 1276& 638\\
319& 958& 479& 1438& 719& 2158& 1079& 3238& 1619& 4858\\
2429& 7288& 3644& 1822& 911& 2734& 1367& 4102& 2051& 6154\\
3077& 9232& 4616& 2308& 1154& 577& 1732& 866& 433& 1300\\
650& 325& 976& 488& 244& 122& 61& 184& 92& 46\\
23& 70& 35& 106& 53& 160& 80& 40& 20& 10\\
5& 16& 8& 4& 2& 1& \\

1457&&&&&&&&&\\
4372& 2186& 1093& 3280& 1640& 820& 410& 205& 616& 308\\
154& 77& 232& 116& 58& 29& 88& 44& 22& 11\\
34& 17& 52& 26& 13& 40& 20& 10& 5& 16\\
8& 4& 2& 1& \\

1458&&&&&&&&&\\
729& 2188& 1094& 547& 1642& 821& 2464& 1232& 616& 308\\
154& 77& 232& 116& 58& 29& 88& 44& 22& 11\\
34& 17& 52& 26& 13& 40& 20& 10& 5& 16\\
8& 4& 2& 1& \\

1459&&&&&&&&&\\
4378& 2189& 6568& 3284& 1642& 821& 2464& 1232& 616& 308\\
154& 77& 232& 116& 58& 29& 88& 44& 22& 11\\
34& 17& 52& 26& 13& 40& 20& 10& 5& 16\\
8& 4& 2& 1& \\

1460&&&&&&&&&\\
730& 365& 1096& 548& 274& 137& 412& 206& 103& 310\\
155& 466& 233& 700& 350& 175& 526& 263& 790& 395\\
1186& 593& 1780& 890& 445& 1336& 668& 334& 167& 502\\
251& 754& 377& 1132& 566& 283& 850& 425& 1276& 638\\
319& 958& 479& 1438& 719& 2158& 1079& 3238& 1619& 4858\\
2429& 7288& 3644& 1822& 911& 2734& 1367& 4102& 2051& 6154\\
3077& 9232& 4616& 2308& 1154& 577& 1732& 866& 433& 1300\\
650& 325& 976& 488& 244& 122& 61& 184& 92& 46\\
23& 70& 35& 106& 53& 160& 80& 40& 20& 10\\
5& 16& 8& 4& 2& 1& \\

1461&&&&&&&&&\\
4384& 2192& 1096& 548& 274& 137& 412& 206& 103& 310\\
155& 466& 233& 700& 350& 175& 526& 263& 790& 395\\
1186& 593& 1780& 890& 445& 1336& 668& 334& 167& 502\\
251& 754& 377& 1132& 566& 283& 850& 425& 1276& 638\\
319& 958& 479& 1438& 719& 2158& 1079& 3238& 1619& 4858\\
2429& 7288& 3644& 1822& 911& 2734& 1367& 4102& 2051& 6154\\
3077& 9232& 4616& 2308& 1154& 577& 1732& 866& 433& 1300\\
650& 325& 976& 488& 244& 122& 61& 184& 92& 46\\
23& 70& 35& 106& 53& 160& 80& 40& 20& 10\\
5& 16& 8& 4& 2& 1& \\

1462&&&&&&&&&\\
731& 2194& 1097& 3292& 1646& 823& 2470& 1235& 3706& 1853\\
5560& 2780& 1390& 695& 2086& 1043& 3130& 1565& 4696& 2348\\
1174& 587& 1762& 881& 2644& 1322& 661& 1984& 992& 496\\
248& 124& 62& 31& 94& 47& 142& 71& 214& 107\\
322& 161& 484& 242& 121& 364& 182& 91& 274& 137\\
412& 206& 103& 310& 155& 466& 233& 700& 350& 175\\
526& 263& 790& 395& 1186& 593& 1780& 890& 445& 1336\\
668& 334& 167& 502& 251& 754& 377& 1132& 566& 283\\
850& 425& 1276& 638& 319& 958& 479& 1438& 719& 2158\\
1079& 3238& 1619& 4858& 2429& 7288& 3644& 1822& 911& 2734\\
1367& 4102& 2051& 6154& 3077& 9232& 4616& 2308& 1154& 577\\
1732& 866& 433& 1300& 650& 325& 976& 488& 244& 122\\
61& 184& 92& 46& 23& 70& 35& 106& 53& 160\\
80& 40& 20& 10& 5& 16& 8& 4& 2& 1\\

1463&&&&&&&&&\\
4390& 2195& 6586& 3293& 9880& 4940& 2470& 1235& 3706& 1853\\
5560& 2780& 1390& 695& 2086& 1043& 3130& 1565& 4696& 2348\\
1174& 587& 1762& 881& 2644& 1322& 661& 1984& 992& 496\\
248& 124& 62& 31& 94& 47& 142& 71& 214& 107\\
322& 161& 484& 242& 121& 364& 182& 91& 274& 137\\
412& 206& 103& 310& 155& 466& 233& 700& 350& 175\\
526& 263& 790& 395& 1186& 593& 1780& 890& 445& 1336\\
668& 334& 167& 502& 251& 754& 377& 1132& 566& 283\\
850& 425& 1276& 638& 319& 958& 479& 1438& 719& 2158\\
1079& 3238& 1619& 4858& 2429& 7288& 3644& 1822& 911& 2734\\
1367& 4102& 2051& 6154& 3077& 9232& 4616& 2308& 1154& 577\\
1732& 866& 433& 1300& 650& 325& 976& 488& 244& 122\\
61& 184& 92& 46& 23& 70& 35& 106& 53& 160\\
80& 40& 20& 10& 5& 16& 8& 4& 2& 1\\

1464&&&&&&&&&\\
732& 366& 183& 550& 275& 826& 413& 1240& 620& 310\\
155& 466& 233& 700& 350& 175& 526& 263& 790& 395\\
1186& 593& 1780& 890& 445& 1336& 668& 334& 167& 502\\
251& 754& 377& 1132& 566& 283& 850& 425& 1276& 638\\
319& 958& 479& 1438& 719& 2158& 1079& 3238& 1619& 4858\\
2429& 7288& 3644& 1822& 911& 2734& 1367& 4102& 2051& 6154\\
3077& 9232& 4616& 2308& 1154& 577& 1732& 866& 433& 1300\\
650& 325& 976& 488& 244& 122& 61& 184& 92& 46\\
23& 70& 35& 106& 53& 160& 80& 40& 20& 10\\
5& 16& 8& 4& 2& 1& \\

1465&&&&&&&&&\\
4396& 2198& 1099& 3298& 1649& 4948& 2474& 1237& 3712& 1856\\
928& 464& 232& 116& 58& 29& 88& 44& 22& 11\\
34& 17& 52& 26& 13& 40& 20& 10& 5& 16\\
8& 4& 2& 1& \\

1466&&&&&&&&&\\
733& 2200& 1100& 550& 275& 826& 413& 1240& 620& 310\\
155& 466& 233& 700& 350& 175& 526& 263& 790& 395\\
1186& 593& 1780& 890& 445& 1336& 668& 334& 167& 502\\
251& 754& 377& 1132& 566& 283& 850& 425& 1276& 638\\
319& 958& 479& 1438& 719& 2158& 1079& 3238& 1619& 4858\\
2429& 7288& 3644& 1822& 911& 2734& 1367& 4102& 2051& 6154\\
3077& 9232& 4616& 2308& 1154& 577& 1732& 866& 433& 1300\\
650& 325& 976& 488& 244& 122& 61& 184& 92& 46\\
23& 70& 35& 106& 53& 160& 80& 40& 20& 10\\
5& 16& 8& 4& 2& 1& \\

1467&&&&&&&&&\\
4402& 2201& 6604& 3302& 1651& 4954& 2477& 7432& 3716& 1858\\
929& 2788& 1394& 697& 2092& 1046& 523& 1570& 785& 2356\\
1178& 589& 1768& 884& 442& 221& 664& 332& 166& 83\\
250& 125& 376& 188& 94& 47& 142& 71& 214& 107\\
322& 161& 484& 242& 121& 364& 182& 91& 274& 137\\
412& 206& 103& 310& 155& 466& 233& 700& 350& 175\\
526& 263& 790& 395& 1186& 593& 1780& 890& 445& 1336\\
668& 334& 167& 502& 251& 754& 377& 1132& 566& 283\\
850& 425& 1276& 638& 319& 958& 479& 1438& 719& 2158\\
1079& 3238& 1619& 4858& 2429& 7288& 3644& 1822& 911& 2734\\
1367& 4102& 2051& 6154& 3077& 9232& 4616& 2308& 1154& 577\\
1732& 866& 433& 1300& 650& 325& 976& 488& 244& 122\\
61& 184& 92& 46& 23& 70& 35& 106& 53& 160\\
80& 40& 20& 10& 5& 16& 8& 4& 2& 1\\

1468&&&&&&&&&\\
734& 367& 1102& 551& 1654& 827& 2482& 1241& 3724& 1862\\
931& 2794& 1397& 4192& 2096& 1048& 524& 262& 131& 394\\
197& 592& 296& 148& 74& 37& 112& 56& 28& 14\\
7& 22& 11& 34& 17& 52& 26& 13& 40& 20\\
10& 5& 16& 8& 4& 2& 1& \\

1469&&&&&&&&&\\
4408& 2204& 1102& 551& 1654& 827& 2482& 1241& 3724& 1862\\
931& 2794& 1397& 4192& 2096& 1048& 524& 262& 131& 394\\
197& 592& 296& 148& 74& 37& 112& 56& 28& 14\\
7& 22& 11& 34& 17& 52& 26& 13& 40& 20\\
10& 5& 16& 8& 4& 2& 1& \\

1470&&&&&&&&&\\
735& 2206& 1103& 3310& 1655& 4966& 2483& 7450& 3725& 11176\\
5588& 2794& 1397& 4192& 2096& 1048& 524& 262& 131& 394\\
197& 592& 296& 148& 74& 37& 112& 56& 28& 14\\
7& 22& 11& 34& 17& 52& 26& 13& 40& 20\\
10& 5& 16& 8& 4& 2& 1& \\

1471&&&&&&&&&\\
4414& 2207& 6622& 3311& 9934& 4967& 14902& 7451& 22354& 11177\\
33532& 16766& 8383& 25150& 12575& 37726& 18863& 56590& 28295& 84886\\
42443& 127330& 63665& 190996& 95498& 47749& 143248& 71624& 35812& 17906\\
8953& 26860& 13430& 6715& 20146& 10073& 30220& 15110& 7555& 22666\\
11333& 34000& 17000& 8500& 4250& 2125& 6376& 3188& 1594& 797\\
2392& 1196& 598& 299& 898& 449& 1348& 674& 337& 1012\\
506& 253& 760& 380& 190& 95& 286& 143& 430& 215\\
646& 323& 970& 485& 1456& 728& 364& 182& 91& 274\\
137& 412& 206& 103& 310& 155& 466& 233& 700& 350\\
175& 526& 263& 790& 395& 1186& 593& 1780& 890& 445\\
1336& 668& 334& 167& 502& 251& 754& 377& 1132& 566\\
283& 850& 425& 1276& 638& 319& 958& 479& 1438& 719\\
2158& 1079& 3238& 1619& 4858& 2429& 7288& 3644& 1822& 911\\
2734& 1367& 4102& 2051& 6154& 3077& 9232& 4616& 2308& 1154\\
577& 1732& 866& 433& 1300& 650& 325& 976& 488& 244\\
122& 61& 184& 92& 46& 23& 70& 35& 106& 53\\
160& 80& 40& 20& 10& 5& 16& 8& 4& 2\\
1& \\

1472&&&&&&&&&\\
736& 368& 184& 92& 46& 23& 70& 35& 106& 53\\
160& 80& 40& 20& 10& 5& 16& 8& 4& 2\\
1& \\

1473&&&&&&&&&\\
4420& 2210& 1105& 3316& 1658& 829& 2488& 1244& 622& 311\\
934& 467& 1402& 701& 2104& 1052& 526& 263& 790& 395\\
1186& 593& 1780& 890& 445& 1336& 668& 334& 167& 502\\
251& 754& 377& 1132& 566& 283& 850& 425& 1276& 638\\
319& 958& 479& 1438& 719& 2158& 1079& 3238& 1619& 4858\\
2429& 7288& 3644& 1822& 911& 2734& 1367& 4102& 2051& 6154\\
3077& 9232& 4616& 2308& 1154& 577& 1732& 866& 433& 1300\\
650& 325& 976& 488& 244& 122& 61& 184& 92& 46\\
23& 70& 35& 106& 53& 160& 80& 40& 20& 10\\
5& 16& 8& 4& 2& 1& \\

1474&&&&&&&&&\\
737& 2212& 1106& 553& 1660& 830& 415& 1246& 623& 1870\\
935& 2806& 1403& 4210& 2105& 6316& 3158& 1579& 4738& 2369\\
7108& 3554& 1777& 5332& 2666& 1333& 4000& 2000& 1000& 500\\
250& 125& 376& 188& 94& 47& 142& 71& 214& 107\\
322& 161& 484& 242& 121& 364& 182& 91& 274& 137\\
412& 206& 103& 310& 155& 466& 233& 700& 350& 175\\
526& 263& 790& 395& 1186& 593& 1780& 890& 445& 1336\\
668& 334& 167& 502& 251& 754& 377& 1132& 566& 283\\
850& 425& 1276& 638& 319& 958& 479& 1438& 719& 2158\\
1079& 3238& 1619& 4858& 2429& 7288& 3644& 1822& 911& 2734\\
1367& 4102& 2051& 6154& 3077& 9232& 4616& 2308& 1154& 577\\
1732& 866& 433& 1300& 650& 325& 976& 488& 244& 122\\
61& 184& 92& 46& 23& 70& 35& 106& 53& 160\\
80& 40& 20& 10& 5& 16& 8& 4& 2& 1\\

1475&&&&&&&&&\\
4426& 2213& 6640& 3320& 1660& 830& 415& 1246& 623& 1870\\
935& 2806& 1403& 4210& 2105& 6316& 3158& 1579& 4738& 2369\\
7108& 3554& 1777& 5332& 2666& 1333& 4000& 2000& 1000& 500\\
250& 125& 376& 188& 94& 47& 142& 71& 214& 107\\
322& 161& 484& 242& 121& 364& 182& 91& 274& 137\\
412& 206& 103& 310& 155& 466& 233& 700& 350& 175\\
526& 263& 790& 395& 1186& 593& 1780& 890& 445& 1336\\
668& 334& 167& 502& 251& 754& 377& 1132& 566& 283\\
850& 425& 1276& 638& 319& 958& 479& 1438& 719& 2158\\
1079& 3238& 1619& 4858& 2429& 7288& 3644& 1822& 911& 2734\\
1367& 4102& 2051& 6154& 3077& 9232& 4616& 2308& 1154& 577\\
1732& 866& 433& 1300& 650& 325& 976& 488& 244& 122\\
61& 184& 92& 46& 23& 70& 35& 106& 53& 160\\
80& 40& 20& 10& 5& 16& 8& 4& 2& 1\\

1476&&&&&&&&&\\
738& 369& 1108& 554& 277& 832& 416& 208& 104& 52\\
26& 13& 40& 20& 10& 5& 16& 8& 4& 2\\
1& \\

1477&&&&&&&&&\\
4432& 2216& 1108& 554& 277& 832& 416& 208& 104& 52\\
26& 13& 40& 20& 10& 5& 16& 8& 4& 2\\
1& \\

1478&&&&&&&&&\\
739& 2218& 1109& 3328& 1664& 832& 416& 208& 104& 52\\
26& 13& 40& 20& 10& 5& 16& 8& 4& 2\\
1& \\

1479&&&&&&&&&\\
4438& 2219& 6658& 3329& 9988& 4994& 2497& 7492& 3746& 1873\\
5620& 2810& 1405& 4216& 2108& 1054& 527& 1582& 791& 2374\\
1187& 3562& 1781& 5344& 2672& 1336& 668& 334& 167& 502\\
251& 754& 377& 1132& 566& 283& 850& 425& 1276& 638\\
319& 958& 479& 1438& 719& 2158& 1079& 3238& 1619& 4858\\
2429& 7288& 3644& 1822& 911& 2734& 1367& 4102& 2051& 6154\\
3077& 9232& 4616& 2308& 1154& 577& 1732& 866& 433& 1300\\
650& 325& 976& 488& 244& 122& 61& 184& 92& 46\\
23& 70& 35& 106& 53& 160& 80& 40& 20& 10\\
5& 16& 8& 4& 2& 1& \\

1480&&&&&&&&&\\
740& 370& 185& 556& 278& 139& 418& 209& 628& 314\\
157& 472& 236& 118& 59& 178& 89& 268& 134& 67\\
202& 101& 304& 152& 76& 38& 19& 58& 29& 88\\
44& 22& 11& 34& 17& 52& 26& 13& 40& 20\\
10& 5& 16& 8& 4& 2& 1& \\

1481&&&&&&&&&\\
4444& 2222& 1111& 3334& 1667& 5002& 2501& 7504& 3752& 1876\\
938& 469& 1408& 704& 352& 176& 88& 44& 22& 11\\
34& 17& 52& 26& 13& 40& 20& 10& 5& 16\\
8& 4& 2& 1& \\

1482&&&&&&&&&\\
741& 2224& 1112& 556& 278& 139& 418& 209& 628& 314\\
157& 472& 236& 118& 59& 178& 89& 268& 134& 67\\
202& 101& 304& 152& 76& 38& 19& 58& 29& 88\\
44& 22& 11& 34& 17& 52& 26& 13& 40& 20\\
10& 5& 16& 8& 4& 2& 1& \\

1483&&&&&&&&&\\
4450& 2225& 6676& 3338& 1669& 5008& 2504& 1252& 626& 313\\
940& 470& 235& 706& 353& 1060& 530& 265& 796& 398\\
199& 598& 299& 898& 449& 1348& 674& 337& 1012& 506\\
253& 760& 380& 190& 95& 286& 143& 430& 215& 646\\
323& 970& 485& 1456& 728& 364& 182& 91& 274& 137\\
412& 206& 103& 310& 155& 466& 233& 700& 350& 175\\
526& 263& 790& 395& 1186& 593& 1780& 890& 445& 1336\\
668& 334& 167& 502& 251& 754& 377& 1132& 566& 283\\
850& 425& 1276& 638& 319& 958& 479& 1438& 719& 2158\\
1079& 3238& 1619& 4858& 2429& 7288& 3644& 1822& 911& 2734\\
1367& 4102& 2051& 6154& 3077& 9232& 4616& 2308& 1154& 577\\
1732& 866& 433& 1300& 650& 325& 976& 488& 244& 122\\
61& 184& 92& 46& 23& 70& 35& 106& 53& 160\\
80& 40& 20& 10& 5& 16& 8& 4& 2& 1\\

1484&&&&&&&&&\\
742& 371& 1114& 557& 1672& 836& 418& 209& 628& 314\\
157& 472& 236& 118& 59& 178& 89& 268& 134& 67\\
202& 101& 304& 152& 76& 38& 19& 58& 29& 88\\
44& 22& 11& 34& 17& 52& 26& 13& 40& 20\\
10& 5& 16& 8& 4& 2& 1& \\

1485&&&&&&&&&\\
4456& 2228& 1114& 557& 1672& 836& 418& 209& 628& 314\\
157& 472& 236& 118& 59& 178& 89& 268& 134& 67\\
202& 101& 304& 152& 76& 38& 19& 58& 29& 88\\
44& 22& 11& 34& 17& 52& 26& 13& 40& 20\\
10& 5& 16& 8& 4& 2& 1& \\

1486&&&&&&&&&\\
743& 2230& 1115& 3346& 1673& 5020& 2510& 1255& 3766& 1883\\
5650& 2825& 8476& 4238& 2119& 6358& 3179& 9538& 4769& 14308\\
7154& 3577& 10732& 5366& 2683& 8050& 4025& 12076& 6038& 3019\\
9058& 4529& 13588& 6794& 3397& 10192& 5096& 2548& 1274& 637\\
1912& 956& 478& 239& 718& 359& 1078& 539& 1618& 809\\
2428& 1214& 607& 1822& 911& 2734& 1367& 4102& 2051& 6154\\
3077& 9232& 4616& 2308& 1154& 577& 1732& 866& 433& 1300\\
650& 325& 976& 488& 244& 122& 61& 184& 92& 46\\
23& 70& 35& 106& 53& 160& 80& 40& 20& 10\\
5& 16& 8& 4& 2& 1& \\

1487&&&&&&&&&\\
4462& 2231& 6694& 3347& 10042& 5021& 15064& 7532& 3766& 1883\\
5650& 2825& 8476& 4238& 2119& 6358& 3179& 9538& 4769& 14308\\
7154& 3577& 10732& 5366& 2683& 8050& 4025& 12076& 6038& 3019\\
9058& 4529& 13588& 6794& 3397& 10192& 5096& 2548& 1274& 637\\
1912& 956& 478& 239& 718& 359& 1078& 539& 1618& 809\\
2428& 1214& 607& 1822& 911& 2734& 1367& 4102& 2051& 6154\\
3077& 9232& 4616& 2308& 1154& 577& 1732& 866& 433& 1300\\
650& 325& 976& 488& 244& 122& 61& 184& 92& 46\\
23& 70& 35& 106& 53& 160& 80& 40& 20& 10\\
5& 16& 8& 4& 2& 1& \\

1488&&&&&&&&&\\
744& 372& 186& 93& 280& 140& 70& 35& 106& 53\\
160& 80& 40& 20& 10& 5& 16& 8& 4& 2\\
1& \\

1489&&&&&&&&&\\
4468& 2234& 1117& 3352& 1676& 838& 419& 1258& 629& 1888\\
944& 472& 236& 118& 59& 178& 89& 268& 134& 67\\
202& 101& 304& 152& 76& 38& 19& 58& 29& 88\\
44& 22& 11& 34& 17& 52& 26& 13& 40& 20\\
10& 5& 16& 8& 4& 2& 1& \\

1490&&&&&&&&&\\
745& 2236& 1118& 559& 1678& 839& 2518& 1259& 3778& 1889\\
5668& 2834& 1417& 4252& 2126& 1063& 3190& 1595& 4786& 2393\\
7180& 3590& 1795& 5386& 2693& 8080& 4040& 2020& 1010& 505\\
1516& 758& 379& 1138& 569& 1708& 854& 427& 1282& 641\\
1924& 962& 481& 1444& 722& 361& 1084& 542& 271& 814\\
407& 1222& 611& 1834& 917& 2752& 1376& 688& 344& 172\\
86& 43& 130& 65& 196& 98& 49& 148& 74& 37\\
112& 56& 28& 14& 7& 22& 11& 34& 17& 52\\
26& 13& 40& 20& 10& 5& 16& 8& 4& 2\\
1& \\

1491&&&&&&&&&\\
4474& 2237& 6712& 3356& 1678& 839& 2518& 1259& 3778& 1889\\
5668& 2834& 1417& 4252& 2126& 1063& 3190& 1595& 4786& 2393\\
7180& 3590& 1795& 5386& 2693& 8080& 4040& 2020& 1010& 505\\
1516& 758& 379& 1138& 569& 1708& 854& 427& 1282& 641\\
1924& 962& 481& 1444& 722& 361& 1084& 542& 271& 814\\
407& 1222& 611& 1834& 917& 2752& 1376& 688& 344& 172\\
86& 43& 130& 65& 196& 98& 49& 148& 74& 37\\
112& 56& 28& 14& 7& 22& 11& 34& 17& 52\\
26& 13& 40& 20& 10& 5& 16& 8& 4& 2\\
1& \\

1492&&&&&&&&&\\
746& 373& 1120& 560& 280& 140& 70& 35& 106& 53\\
160& 80& 40& 20& 10& 5& 16& 8& 4& 2\\
1& \\

1493&&&&&&&&&\\
4480& 2240& 1120& 560& 280& 140& 70& 35& 106& 53\\
160& 80& 40& 20& 10& 5& 16& 8& 4& 2\\
1& \\

1494&&&&&&&&&\\
747& 2242& 1121& 3364& 1682& 841& 2524& 1262& 631& 1894\\
947& 2842& 1421& 4264& 2132& 1066& 533& 1600& 800& 400\\
200& 100& 50& 25& 76& 38& 19& 58& 29& 88\\
44& 22& 11& 34& 17& 52& 26& 13& 40& 20\\
10& 5& 16& 8& 4& 2& 1& \\

1495&&&&&&&&&\\
4486& 2243& 6730& 3365& 10096& 5048& 2524& 1262& 631& 1894\\
947& 2842& 1421& 4264& 2132& 1066& 533& 1600& 800& 400\\
200& 100& 50& 25& 76& 38& 19& 58& 29& 88\\
44& 22& 11& 34& 17& 52& 26& 13& 40& 20\\
10& 5& 16& 8& 4& 2& 1& \\

1496&&&&&&&&&\\
748& 374& 187& 562& 281& 844& 422& 211& 634& 317\\
952& 476& 238& 119& 358& 179& 538& 269& 808& 404\\
202& 101& 304& 152& 76& 38& 19& 58& 29& 88\\
44& 22& 11& 34& 17& 52& 26& 13& 40& 20\\
10& 5& 16& 8& 4& 2& 1& \\

1497&&&&&&&&&\\
4492& 2246& 1123& 3370& 1685& 5056& 2528& 1264& 632& 316\\
158& 79& 238& 119& 358& 179& 538& 269& 808& 404\\
202& 101& 304& 152& 76& 38& 19& 58& 29& 88\\
44& 22& 11& 34& 17& 52& 26& 13& 40& 20\\
10& 5& 16& 8& 4& 2& 1& \\

1498&&&&&&&&&\\
749& 2248& 1124& 562& 281& 844& 422& 211& 634& 317\\
952& 476& 238& 119& 358& 179& 538& 269& 808& 404\\
202& 101& 304& 152& 76& 38& 19& 58& 29& 88\\
44& 22& 11& 34& 17& 52& 26& 13& 40& 20\\
10& 5& 16& 8& 4& 2& 1& \\

1499&&&&&&&&&\\
4498& 2249& 6748& 3374& 1687& 5062& 2531& 7594& 3797& 11392\\
5696& 2848& 1424& 712& 356& 178& 89& 268& 134& 67\\
202& 101& 304& 152& 76& 38& 19& 58& 29& 88\\
44& 22& 11& 34& 17& 52& 26& 13& 40& 20\\
10& 5& 16& 8& 4& 2& 1& \\

1500&&&&&&&&&\\
750& 375& 1126& 563& 1690& 845& 2536& 1268& 634& 317\\
952& 476& 238& 119& 358& 179& 538& 269& 808& 404\\
202& 101& 304& 152& 76& 38& 19& 58& 29& 88\\
44& 22& 11& 34& 17& 52& 26& 13& 40& 20\\
10& 5& 16& 8& 4& 2& 1& \\

1501&&&&&&&&&\\
4504& 2252& 1126& 563& 1690& 845& 2536& 1268& 634& 317\\
952& 476& 238& 119& 358& 179& 538& 269& 808& 404\\
202& 101& 304& 152& 76& 38& 19& 58& 29& 88\\
44& 22& 11& 34& 17& 52& 26& 13& 40& 20\\
10& 5& 16& 8& 4& 2& 1& \\

1502&&&&&&&&&\\
751& 2254& 1127& 3382& 1691& 5074& 2537& 7612& 3806& 1903\\
5710& 2855& 8566& 4283& 12850& 6425& 19276& 9638& 4819& 14458\\
7229& 21688& 10844& 5422& 2711& 8134& 4067& 12202& 6101& 18304\\
9152& 4576& 2288& 1144& 572& 286& 143& 430& 215& 646\\
323& 970& 485& 1456& 728& 364& 182& 91& 274& 137\\
412& 206& 103& 310& 155& 466& 233& 700& 350& 175\\
526& 263& 790& 395& 1186& 593& 1780& 890& 445& 1336\\
668& 334& 167& 502& 251& 754& 377& 1132& 566& 283\\
850& 425& 1276& 638& 319& 958& 479& 1438& 719& 2158\\
1079& 3238& 1619& 4858& 2429& 7288& 3644& 1822& 911& 2734\\
1367& 4102& 2051& 6154& 3077& 9232& 4616& 2308& 1154& 577\\
1732& 866& 433& 1300& 650& 325& 976& 488& 244& 122\\
61& 184& 92& 46& 23& 70& 35& 106& 53& 160\\
80& 40& 20& 10& 5& 16& 8& 4& 2& 1\\

1503&&&&&&&&&\\
4510& 2255& 6766& 3383& 10150& 5075& 15226& 7613& 22840& 11420\\
5710& 2855& 8566& 4283& 12850& 6425& 19276& 9638& 4819& 14458\\
7229& 21688& 10844& 5422& 2711& 8134& 4067& 12202& 6101& 18304\\
9152& 4576& 2288& 1144& 572& 286& 143& 430& 215& 646\\
323& 970& 485& 1456& 728& 364& 182& 91& 274& 137\\
412& 206& 103& 310& 155& 466& 233& 700& 350& 175\\
526& 263& 790& 395& 1186& 593& 1780& 890& 445& 1336\\
668& 334& 167& 502& 251& 754& 377& 1132& 566& 283\\
850& 425& 1276& 638& 319& 958& 479& 1438& 719& 2158\\
1079& 3238& 1619& 4858& 2429& 7288& 3644& 1822& 911& 2734\\
1367& 4102& 2051& 6154& 3077& 9232& 4616& 2308& 1154& 577\\
1732& 866& 433& 1300& 650& 325& 976& 488& 244& 122\\
61& 184& 92& 46& 23& 70& 35& 106& 53& 160\\
80& 40& 20& 10& 5& 16& 8& 4& 2& 1\\

1504&&&&&&&&&\\
752& 376& 188& 94& 47& 142& 71& 214& 107& 322\\
161& 484& 242& 121& 364& 182& 91& 274& 137& 412\\
206& 103& 310& 155& 466& 233& 700& 350& 175& 526\\
263& 790& 395& 1186& 593& 1780& 890& 445& 1336& 668\\
334& 167& 502& 251& 754& 377& 1132& 566& 283& 850\\
425& 1276& 638& 319& 958& 479& 1438& 719& 2158& 1079\\
3238& 1619& 4858& 2429& 7288& 3644& 1822& 911& 2734& 1367\\
4102& 2051& 6154& 3077& 9232& 4616& 2308& 1154& 577& 1732\\
866& 433& 1300& 650& 325& 976& 488& 244& 122& 61\\
184& 92& 46& 23& 70& 35& 106& 53& 160& 80\\
40& 20& 10& 5& 16& 8& 4& 2& 1& \\

1505&&&&&&&&&\\
4516& 2258& 1129& 3388& 1694& 847& 2542& 1271& 3814& 1907\\
5722& 2861& 8584& 4292& 2146& 1073& 3220& 1610& 805& 2416\\
1208& 604& 302& 151& 454& 227& 682& 341& 1024& 512\\
256& 128& 64& 32& 16& 8& 4& 2& 1& \\

1506&&&&&&&&&\\
753& 2260& 1130& 565& 1696& 848& 424& 212& 106& 53\\
160& 80& 40& 20& 10& 5& 16& 8& 4& 2\\
1& \\

1507&&&&&&&&&\\
4522& 2261& 6784& 3392& 1696& 848& 424& 212& 106& 53\\
160& 80& 40& 20& 10& 5& 16& 8& 4& 2\\
1& \\

1508&&&&&&&&&\\
754& 377& 1132& 566& 283& 850& 425& 1276& 638& 319\\
958& 479& 1438& 719& 2158& 1079& 3238& 1619& 4858& 2429\\
7288& 3644& 1822& 911& 2734& 1367& 4102& 2051& 6154& 3077\\
9232& 4616& 2308& 1154& 577& 1732& 866& 433& 1300& 650\\
325& 976& 488& 244& 122& 61& 184& 92& 46& 23\\
70& 35& 106& 53& 160& 80& 40& 20& 10& 5\\
16& 8& 4& 2& 1& \\

1509&&&&&&&&&\\
4528& 2264& 1132& 566& 283& 850& 425& 1276& 638& 319\\
958& 479& 1438& 719& 2158& 1079& 3238& 1619& 4858& 2429\\
7288& 3644& 1822& 911& 2734& 1367& 4102& 2051& 6154& 3077\\
9232& 4616& 2308& 1154& 577& 1732& 866& 433& 1300& 650\\
325& 976& 488& 244& 122& 61& 184& 92& 46& 23\\
70& 35& 106& 53& 160& 80& 40& 20& 10& 5\\
16& 8& 4& 2& 1& \\

1510&&&&&&&&&\\
755& 2266& 1133& 3400& 1700& 850& 425& 1276& 638& 319\\
958& 479& 1438& 719& 2158& 1079& 3238& 1619& 4858& 2429\\
7288& 3644& 1822& 911& 2734& 1367& 4102& 2051& 6154& 3077\\
9232& 4616& 2308& 1154& 577& 1732& 866& 433& 1300& 650\\
325& 976& 488& 244& 122& 61& 184& 92& 46& 23\\
70& 35& 106& 53& 160& 80& 40& 20& 10& 5\\
16& 8& 4& 2& 1& \\

1511&&&&&&&&&\\
4534& 2267& 6802& 3401& 10204& 5102& 2551& 7654& 3827& 11482\\
5741& 17224& 8612& 4306& 2153& 6460& 3230& 1615& 4846& 2423\\
7270& 3635& 10906& 5453& 16360& 8180& 4090& 2045& 6136& 3068\\
1534& 767& 2302& 1151& 3454& 1727& 5182& 2591& 7774& 3887\\
11662& 5831& 17494& 8747& 26242& 13121& 39364& 19682& 9841& 29524\\
14762& 7381& 22144& 11072& 5536& 2768& 1384& 692& 346& 173\\
520& 260& 130& 65& 196& 98& 49& 148& 74& 37\\
112& 56& 28& 14& 7& 22& 11& 34& 17& 52\\
26& 13& 40& 20& 10& 5& 16& 8& 4& 2\\
1& \\

1512&&&&&&&&&\\
756& 378& 189& 568& 284& 142& 71& 214& 107& 322\\
161& 484& 242& 121& 364& 182& 91& 274& 137& 412\\
206& 103& 310& 155& 466& 233& 700& 350& 175& 526\\
263& 790& 395& 1186& 593& 1780& 890& 445& 1336& 668\\
334& 167& 502& 251& 754& 377& 1132& 566& 283& 850\\
425& 1276& 638& 319& 958& 479& 1438& 719& 2158& 1079\\
3238& 1619& 4858& 2429& 7288& 3644& 1822& 911& 2734& 1367\\
4102& 2051& 6154& 3077& 9232& 4616& 2308& 1154& 577& 1732\\
866& 433& 1300& 650& 325& 976& 488& 244& 122& 61\\
184& 92& 46& 23& 70& 35& 106& 53& 160& 80\\
40& 20& 10& 5& 16& 8& 4& 2& 1& \\

1513&&&&&&&&&\\
4540& 2270& 1135& 3406& 1703& 5110& 2555& 7666& 3833& 11500\\
5750& 2875& 8626& 4313& 12940& 6470& 3235& 9706& 4853& 14560\\
7280& 3640& 1820& 910& 455& 1366& 683& 2050& 1025& 3076\\
1538& 769& 2308& 1154& 577& 1732& 866& 433& 1300& 650\\
325& 976& 488& 244& 122& 61& 184& 92& 46& 23\\
70& 35& 106& 53& 160& 80& 40& 20& 10& 5\\
16& 8& 4& 2& 1& \\

1514&&&&&&&&&\\
757& 2272& 1136& 568& 284& 142& 71& 214& 107& 322\\
161& 484& 242& 121& 364& 182& 91& 274& 137& 412\\
206& 103& 310& 155& 466& 233& 700& 350& 175& 526\\
263& 790& 395& 1186& 593& 1780& 890& 445& 1336& 668\\
334& 167& 502& 251& 754& 377& 1132& 566& 283& 850\\
425& 1276& 638& 319& 958& 479& 1438& 719& 2158& 1079\\
3238& 1619& 4858& 2429& 7288& 3644& 1822& 911& 2734& 1367\\
4102& 2051& 6154& 3077& 9232& 4616& 2308& 1154& 577& 1732\\
866& 433& 1300& 650& 325& 976& 488& 244& 122& 61\\
184& 92& 46& 23& 70& 35& 106& 53& 160& 80\\
40& 20& 10& 5& 16& 8& 4& 2& 1& \\

1515&&&&&&&&&\\
4546& 2273& 6820& 3410& 1705& 5116& 2558& 1279& 3838& 1919\\
5758& 2879& 8638& 4319& 12958& 6479& 19438& 9719& 29158& 14579\\
43738& 21869& 65608& 32804& 16402& 8201& 24604& 12302& 6151& 18454\\
9227& 27682& 13841& 41524& 20762& 10381& 31144& 15572& 7786& 3893\\
11680& 5840& 2920& 1460& 730& 365& 1096& 548& 274& 137\\
412& 206& 103& 310& 155& 466& 233& 700& 350& 175\\
526& 263& 790& 395& 1186& 593& 1780& 890& 445& 1336\\
668& 334& 167& 502& 251& 754& 377& 1132& 566& 283\\
850& 425& 1276& 638& 319& 958& 479& 1438& 719& 2158\\
1079& 3238& 1619& 4858& 2429& 7288& 3644& 1822& 911& 2734\\
1367& 4102& 2051& 6154& 3077& 9232& 4616& 2308& 1154& 577\\
1732& 866& 433& 1300& 650& 325& 976& 488& 244& 122\\
61& 184& 92& 46& 23& 70& 35& 106& 53& 160\\
80& 40& 20& 10& 5& 16& 8& 4& 2& 1\\

1516&&&&&&&&&\\
758& 379& 1138& 569& 1708& 854& 427& 1282& 641& 1924\\
962& 481& 1444& 722& 361& 1084& 542& 271& 814& 407\\
1222& 611& 1834& 917& 2752& 1376& 688& 344& 172& 86\\
43& 130& 65& 196& 98& 49& 148& 74& 37& 112\\
56& 28& 14& 7& 22& 11& 34& 17& 52& 26\\
13& 40& 20& 10& 5& 16& 8& 4& 2& 1\\

1517&&&&&&&&&\\
4552& 2276& 1138& 569& 1708& 854& 427& 1282& 641& 1924\\
962& 481& 1444& 722& 361& 1084& 542& 271& 814& 407\\
1222& 611& 1834& 917& 2752& 1376& 688& 344& 172& 86\\
43& 130& 65& 196& 98& 49& 148& 74& 37& 112\\
56& 28& 14& 7& 22& 11& 34& 17& 52& 26\\
13& 40& 20& 10& 5& 16& 8& 4& 2& 1\\

1518&&&&&&&&&\\
759& 2278& 1139& 3418& 1709& 5128& 2564& 1282& 641& 1924\\
962& 481& 1444& 722& 361& 1084& 542& 271& 814& 407\\
1222& 611& 1834& 917& 2752& 1376& 688& 344& 172& 86\\
43& 130& 65& 196& 98& 49& 148& 74& 37& 112\\
56& 28& 14& 7& 22& 11& 34& 17& 52& 26\\
13& 40& 20& 10& 5& 16& 8& 4& 2& 1\\

1519&&&&&&&&&\\
4558& 2279& 6838& 3419& 10258& 5129& 15388& 7694& 3847& 11542\\
5771& 17314& 8657& 25972& 12986& 6493& 19480& 9740& 4870& 2435\\
7306& 3653& 10960& 5480& 2740& 1370& 685& 2056& 1028& 514\\
257& 772& 386& 193& 580& 290& 145& 436& 218& 109\\
328& 164& 82& 41& 124& 62& 31& 94& 47& 142\\
71& 214& 107& 322& 161& 484& 242& 121& 364& 182\\
91& 274& 137& 412& 206& 103& 310& 155& 466& 233\\
700& 350& 175& 526& 263& 790& 395& 1186& 593& 1780\\
890& 445& 1336& 668& 334& 167& 502& 251& 754& 377\\
1132& 566& 283& 850& 425& 1276& 638& 319& 958& 479\\
1438& 719& 2158& 1079& 3238& 1619& 4858& 2429& 7288& 3644\\
1822& 911& 2734& 1367& 4102& 2051& 6154& 3077& 9232& 4616\\
2308& 1154& 577& 1732& 866& 433& 1300& 650& 325& 976\\
488& 244& 122& 61& 184& 92& 46& 23& 70& 35\\
106& 53& 160& 80& 40& 20& 10& 5& 16& 8\\
4& 2& 1& \\

1520&&&&&&&&&\\
760& 380& 190& 95& 286& 143& 430& 215& 646& 323\\
970& 485& 1456& 728& 364& 182& 91& 274& 137& 412\\
206& 103& 310& 155& 466& 233& 700& 350& 175& 526\\
263& 790& 395& 1186& 593& 1780& 890& 445& 1336& 668\\
334& 167& 502& 251& 754& 377& 1132& 566& 283& 850\\
425& 1276& 638& 319& 958& 479& 1438& 719& 2158& 1079\\
3238& 1619& 4858& 2429& 7288& 3644& 1822& 911& 2734& 1367\\
4102& 2051& 6154& 3077& 9232& 4616& 2308& 1154& 577& 1732\\
866& 433& 1300& 650& 325& 976& 488& 244& 122& 61\\
184& 92& 46& 23& 70& 35& 106& 53& 160& 80\\
40& 20& 10& 5& 16& 8& 4& 2& 1& \\

1521&&&&&&&&&\\
4564& 2282& 1141& 3424& 1712& 856& 428& 214& 107& 322\\
161& 484& 242& 121& 364& 182& 91& 274& 137& 412\\
206& 103& 310& 155& 466& 233& 700& 350& 175& 526\\
263& 790& 395& 1186& 593& 1780& 890& 445& 1336& 668\\
334& 167& 502& 251& 754& 377& 1132& 566& 283& 850\\
425& 1276& 638& 319& 958& 479& 1438& 719& 2158& 1079\\
3238& 1619& 4858& 2429& 7288& 3644& 1822& 911& 2734& 1367\\
4102& 2051& 6154& 3077& 9232& 4616& 2308& 1154& 577& 1732\\
866& 433& 1300& 650& 325& 976& 488& 244& 122& 61\\
184& 92& 46& 23& 70& 35& 106& 53& 160& 80\\
40& 20& 10& 5& 16& 8& 4& 2& 1& \\

1522&&&&&&&&&\\
761& 2284& 1142& 571& 1714& 857& 2572& 1286& 643& 1930\\
965& 2896& 1448& 724& 362& 181& 544& 272& 136& 68\\
34& 17& 52& 26& 13& 40& 20& 10& 5& 16\\
8& 4& 2& 1& \\

1523&&&&&&&&&\\
4570& 2285& 6856& 3428& 1714& 857& 2572& 1286& 643& 1930\\
965& 2896& 1448& 724& 362& 181& 544& 272& 136& 68\\
34& 17& 52& 26& 13& 40& 20& 10& 5& 16\\
8& 4& 2& 1& \\

1524&&&&&&&&&\\
762& 381& 1144& 572& 286& 143& 430& 215& 646& 323\\
970& 485& 1456& 728& 364& 182& 91& 274& 137& 412\\
206& 103& 310& 155& 466& 233& 700& 350& 175& 526\\
263& 790& 395& 1186& 593& 1780& 890& 445& 1336& 668\\
334& 167& 502& 251& 754& 377& 1132& 566& 283& 850\\
425& 1276& 638& 319& 958& 479& 1438& 719& 2158& 1079\\
3238& 1619& 4858& 2429& 7288& 3644& 1822& 911& 2734& 1367\\
4102& 2051& 6154& 3077& 9232& 4616& 2308& 1154& 577& 1732\\
866& 433& 1300& 650& 325& 976& 488& 244& 122& 61\\
184& 92& 46& 23& 70& 35& 106& 53& 160& 80\\
40& 20& 10& 5& 16& 8& 4& 2& 1& \\

1525&&&&&&&&&\\
4576& 2288& 1144& 572& 286& 143& 430& 215& 646& 323\\
970& 485& 1456& 728& 364& 182& 91& 274& 137& 412\\
206& 103& 310& 155& 466& 233& 700& 350& 175& 526\\
263& 790& 395& 1186& 593& 1780& 890& 445& 1336& 668\\
334& 167& 502& 251& 754& 377& 1132& 566& 283& 850\\
425& 1276& 638& 319& 958& 479& 1438& 719& 2158& 1079\\
3238& 1619& 4858& 2429& 7288& 3644& 1822& 911& 2734& 1367\\
4102& 2051& 6154& 3077& 9232& 4616& 2308& 1154& 577& 1732\\
866& 433& 1300& 650& 325& 976& 488& 244& 122& 61\\
184& 92& 46& 23& 70& 35& 106& 53& 160& 80\\
40& 20& 10& 5& 16& 8& 4& 2& 1& \\

1526&&&&&&&&&\\
763& 2290& 1145& 3436& 1718& 859& 2578& 1289& 3868& 1934\\
967& 2902& 1451& 4354& 2177& 6532& 3266& 1633& 4900& 2450\\
1225& 3676& 1838& 919& 2758& 1379& 4138& 2069& 6208& 3104\\
1552& 776& 388& 194& 97& 292& 146& 73& 220& 110\\
55& 166& 83& 250& 125& 376& 188& 94& 47& 142\\
71& 214& 107& 322& 161& 484& 242& 121& 364& 182\\
91& 274& 137& 412& 206& 103& 310& 155& 466& 233\\
700& 350& 175& 526& 263& 790& 395& 1186& 593& 1780\\
890& 445& 1336& 668& 334& 167& 502& 251& 754& 377\\
1132& 566& 283& 850& 425& 1276& 638& 319& 958& 479\\
1438& 719& 2158& 1079& 3238& 1619& 4858& 2429& 7288& 3644\\
1822& 911& 2734& 1367& 4102& 2051& 6154& 3077& 9232& 4616\\
2308& 1154& 577& 1732& 866& 433& 1300& 650& 325& 976\\
488& 244& 122& 61& 184& 92& 46& 23& 70& 35\\
106& 53& 160& 80& 40& 20& 10& 5& 16& 8\\
4& 2& 1& \\

1527&&&&&&&&&\\
4582& 2291& 6874& 3437& 10312& 5156& 2578& 1289& 3868& 1934\\
967& 2902& 1451& 4354& 2177& 6532& 3266& 1633& 4900& 2450\\
1225& 3676& 1838& 919& 2758& 1379& 4138& 2069& 6208& 3104\\
1552& 776& 388& 194& 97& 292& 146& 73& 220& 110\\
55& 166& 83& 250& 125& 376& 188& 94& 47& 142\\
71& 214& 107& 322& 161& 484& 242& 121& 364& 182\\
91& 274& 137& 412& 206& 103& 310& 155& 466& 233\\
700& 350& 175& 526& 263& 790& 395& 1186& 593& 1780\\
890& 445& 1336& 668& 334& 167& 502& 251& 754& 377\\
1132& 566& 283& 850& 425& 1276& 638& 319& 958& 479\\
1438& 719& 2158& 1079& 3238& 1619& 4858& 2429& 7288& 3644\\
1822& 911& 2734& 1367& 4102& 2051& 6154& 3077& 9232& 4616\\
2308& 1154& 577& 1732& 866& 433& 1300& 650& 325& 976\\
488& 244& 122& 61& 184& 92& 46& 23& 70& 35\\
106& 53& 160& 80& 40& 20& 10& 5& 16& 8\\
4& 2& 1& \\

1528&&&&&&&&&\\
764& 382& 191& 574& 287& 862& 431& 1294& 647& 1942\\
971& 2914& 1457& 4372& 2186& 1093& 3280& 1640& 820& 410\\
205& 616& 308& 154& 77& 232& 116& 58& 29& 88\\
44& 22& 11& 34& 17& 52& 26& 13& 40& 20\\
10& 5& 16& 8& 4& 2& 1& \\

1529&&&&&&&&&\\
4588& 2294& 1147& 3442& 1721& 5164& 2582& 1291& 3874& 1937\\
5812& 2906& 1453& 4360& 2180& 1090& 545& 1636& 818& 409\\
1228& 614& 307& 922& 461& 1384& 692& 346& 173& 520\\
260& 130& 65& 196& 98& 49& 148& 74& 37& 112\\
56& 28& 14& 7& 22& 11& 34& 17& 52& 26\\
13& 40& 20& 10& 5& 16& 8& 4& 2& 1\\

1530&&&&&&&&&\\
765& 2296& 1148& 574& 287& 862& 431& 1294& 647& 1942\\
971& 2914& 1457& 4372& 2186& 1093& 3280& 1640& 820& 410\\
205& 616& 308& 154& 77& 232& 116& 58& 29& 88\\
44& 22& 11& 34& 17& 52& 26& 13& 40& 20\\
10& 5& 16& 8& 4& 2& 1& \\

1531&&&&&&&&&\\
4594& 2297& 6892& 3446& 1723& 5170& 2585& 7756& 3878& 1939\\
5818& 2909& 8728& 4364& 2182& 1091& 3274& 1637& 4912& 2456\\
1228& 614& 307& 922& 461& 1384& 692& 346& 173& 520\\
260& 130& 65& 196& 98& 49& 148& 74& 37& 112\\
56& 28& 14& 7& 22& 11& 34& 17& 52& 26\\
13& 40& 20& 10& 5& 16& 8& 4& 2& 1\\

1532&&&&&&&&&\\
766& 383& 1150& 575& 1726& 863& 2590& 1295& 3886& 1943\\
5830& 2915& 8746& 4373& 13120& 6560& 3280& 1640& 820& 410\\
205& 616& 308& 154& 77& 232& 116& 58& 29& 88\\
44& 22& 11& 34& 17& 52& 26& 13& 40& 20\\
10& 5& 16& 8& 4& 2& 1& \\

1533&&&&&&&&&\\
4600& 2300& 1150& 575& 1726& 863& 2590& 1295& 3886& 1943\\
5830& 2915& 8746& 4373& 13120& 6560& 3280& 1640& 820& 410\\
205& 616& 308& 154& 77& 232& 116& 58& 29& 88\\
44& 22& 11& 34& 17& 52& 26& 13& 40& 20\\
10& 5& 16& 8& 4& 2& 1& \\

1534&&&&&&&&&\\
767& 2302& 1151& 3454& 1727& 5182& 2591& 7774& 3887& 11662\\
5831& 17494& 8747& 26242& 13121& 39364& 19682& 9841& 29524& 14762\\
7381& 22144& 11072& 5536& 2768& 1384& 692& 346& 173& 520\\
260& 130& 65& 196& 98& 49& 148& 74& 37& 112\\
56& 28& 14& 7& 22& 11& 34& 17& 52& 26\\
13& 40& 20& 10& 5& 16& 8& 4& 2& 1\\

1535&&&&&&&&&\\
4606& 2303& 6910& 3455& 10366& 5183& 15550& 7775& 23326& 11663\\
34990& 17495& 52486& 26243& 78730& 39365& 118096& 59048& 29524& 14762\\
7381& 22144& 11072& 5536& 2768& 1384& 692& 346& 173& 520\\
260& 130& 65& 196& 98& 49& 148& 74& 37& 112\\
56& 28& 14& 7& 22& 11& 34& 17& 52& 26\\
13& 40& 20& 10& 5& 16& 8& 4& 2& 1\\

1536&&&&&&&&&\\
768& 384& 192& 96& 48& 24& 12& 6& 3& 10\\
5& 16& 8& 4& 2& 1& \\

1537&&&&&&&&&\\
4612& 2306& 1153& 3460& 1730& 865& 2596& 1298& 649& 1948\\
974& 487& 1462& 731& 2194& 1097& 3292& 1646& 823& 2470\\
1235& 3706& 1853& 5560& 2780& 1390& 695& 2086& 1043& 3130\\
1565& 4696& 2348& 1174& 587& 1762& 881& 2644& 1322& 661\\
1984& 992& 496& 248& 124& 62& 31& 94& 47& 142\\
71& 214& 107& 322& 161& 484& 242& 121& 364& 182\\
91& 274& 137& 412& 206& 103& 310& 155& 466& 233\\
700& 350& 175& 526& 263& 790& 395& 1186& 593& 1780\\
890& 445& 1336& 668& 334& 167& 502& 251& 754& 377\\
1132& 566& 283& 850& 425& 1276& 638& 319& 958& 479\\
1438& 719& 2158& 1079& 3238& 1619& 4858& 2429& 7288& 3644\\
1822& 911& 2734& 1367& 4102& 2051& 6154& 3077& 9232& 4616\\
2308& 1154& 577& 1732& 866& 433& 1300& 650& 325& 976\\
488& 244& 122& 61& 184& 92& 46& 23& 70& 35\\
106& 53& 160& 80& 40& 20& 10& 5& 16& 8\\
4& 2& 1& \\

1538&&&&&&&&&\\
769& 2308& 1154& 577& 1732& 866& 433& 1300& 650& 325\\
976& 488& 244& 122& 61& 184& 92& 46& 23& 70\\
35& 106& 53& 160& 80& 40& 20& 10& 5& 16\\
8& 4& 2& 1& \\

1539&&&&&&&&&\\
4618& 2309& 6928& 3464& 1732& 866& 433& 1300& 650& 325\\
976& 488& 244& 122& 61& 184& 92& 46& 23& 70\\
35& 106& 53& 160& 80& 40& 20& 10& 5& 16\\
8& 4& 2& 1& \\

1540&&&&&&&&&\\
770& 385& 1156& 578& 289& 868& 434& 217& 652& 326\\
163& 490& 245& 736& 368& 184& 92& 46& 23& 70\\
35& 106& 53& 160& 80& 40& 20& 10& 5& 16\\
8& 4& 2& 1& \\

1541&&&&&&&&&\\
4624& 2312& 1156& 578& 289& 868& 434& 217& 652& 326\\
163& 490& 245& 736& 368& 184& 92& 46& 23& 70\\
35& 106& 53& 160& 80& 40& 20& 10& 5& 16\\
8& 4& 2& 1& \\

1542&&&&&&&&&\\
771& 2314& 1157& 3472& 1736& 868& 434& 217& 652& 326\\
163& 490& 245& 736& 368& 184& 92& 46& 23& 70\\
35& 106& 53& 160& 80& 40& 20& 10& 5& 16\\
8& 4& 2& 1& \\

1543&&&&&&&&&\\
4630& 2315& 6946& 3473& 10420& 5210& 2605& 7816& 3908& 1954\\
977& 2932& 1466& 733& 2200& 1100& 550& 275& 826& 413\\
1240& 620& 310& 155& 466& 233& 700& 350& 175& 526\\
263& 790& 395& 1186& 593& 1780& 890& 445& 1336& 668\\
334& 167& 502& 251& 754& 377& 1132& 566& 283& 850\\
425& 1276& 638& 319& 958& 479& 1438& 719& 2158& 1079\\
3238& 1619& 4858& 2429& 7288& 3644& 1822& 911& 2734& 1367\\
4102& 2051& 6154& 3077& 9232& 4616& 2308& 1154& 577& 1732\\
866& 433& 1300& 650& 325& 976& 488& 244& 122& 61\\
184& 92& 46& 23& 70& 35& 106& 53& 160& 80\\
40& 20& 10& 5& 16& 8& 4& 2& 1& \\

1544&&&&&&&&&\\
772& 386& 193& 580& 290& 145& 436& 218& 109& 328\\
164& 82& 41& 124& 62& 31& 94& 47& 142& 71\\
214& 107& 322& 161& 484& 242& 121& 364& 182& 91\\
274& 137& 412& 206& 103& 310& 155& 466& 233& 700\\
350& 175& 526& 263& 790& 395& 1186& 593& 1780& 890\\
445& 1336& 668& 334& 167& 502& 251& 754& 377& 1132\\
566& 283& 850& 425& 1276& 638& 319& 958& 479& 1438\\
719& 2158& 1079& 3238& 1619& 4858& 2429& 7288& 3644& 1822\\
911& 2734& 1367& 4102& 2051& 6154& 3077& 9232& 4616& 2308\\
1154& 577& 1732& 866& 433& 1300& 650& 325& 976& 488\\
244& 122& 61& 184& 92& 46& 23& 70& 35& 106\\
53& 160& 80& 40& 20& 10& 5& 16& 8& 4\\
2& 1& \\

1545&&&&&&&&&\\
4636& 2318& 1159& 3478& 1739& 5218& 2609& 7828& 3914& 1957\\
5872& 2936& 1468& 734& 367& 1102& 551& 1654& 827& 2482\\
1241& 3724& 1862& 931& 2794& 1397& 4192& 2096& 1048& 524\\
262& 131& 394& 197& 592& 296& 148& 74& 37& 112\\
56& 28& 14& 7& 22& 11& 34& 17& 52& 26\\
13& 40& 20& 10& 5& 16& 8& 4& 2& 1\\

1546&&&&&&&&&\\
773& 2320& 1160& 580& 290& 145& 436& 218& 109& 328\\
164& 82& 41& 124& 62& 31& 94& 47& 142& 71\\
214& 107& 322& 161& 484& 242& 121& 364& 182& 91\\
274& 137& 412& 206& 103& 310& 155& 466& 233& 700\\
350& 175& 526& 263& 790& 395& 1186& 593& 1780& 890\\
445& 1336& 668& 334& 167& 502& 251& 754& 377& 1132\\
566& 283& 850& 425& 1276& 638& 319& 958& 479& 1438\\
719& 2158& 1079& 3238& 1619& 4858& 2429& 7288& 3644& 1822\\
911& 2734& 1367& 4102& 2051& 6154& 3077& 9232& 4616& 2308\\
1154& 577& 1732& 866& 433& 1300& 650& 325& 976& 488\\
244& 122& 61& 184& 92& 46& 23& 70& 35& 106\\
53& 160& 80& 40& 20& 10& 5& 16& 8& 4\\
2& 1& \\

1547&&&&&&&&&\\
4642& 2321& 6964& 3482& 1741& 5224& 2612& 1306& 653& 1960\\
980& 490& 245& 736& 368& 184& 92& 46& 23& 70\\
35& 106& 53& 160& 80& 40& 20& 10& 5& 16\\
8& 4& 2& 1& \\

1548&&&&&&&&&\\
774& 387& 1162& 581& 1744& 872& 436& 218& 109& 328\\
164& 82& 41& 124& 62& 31& 94& 47& 142& 71\\
214& 107& 322& 161& 484& 242& 121& 364& 182& 91\\
274& 137& 412& 206& 103& 310& 155& 466& 233& 700\\
350& 175& 526& 263& 790& 395& 1186& 593& 1780& 890\\
445& 1336& 668& 334& 167& 502& 251& 754& 377& 1132\\
566& 283& 850& 425& 1276& 638& 319& 958& 479& 1438\\
719& 2158& 1079& 3238& 1619& 4858& 2429& 7288& 3644& 1822\\
911& 2734& 1367& 4102& 2051& 6154& 3077& 9232& 4616& 2308\\
1154& 577& 1732& 866& 433& 1300& 650& 325& 976& 488\\
244& 122& 61& 184& 92& 46& 23& 70& 35& 106\\
53& 160& 80& 40& 20& 10& 5& 16& 8& 4\\
2& 1& \\

1549&&&&&&&&&\\
4648& 2324& 1162& 581& 1744& 872& 436& 218& 109& 328\\
164& 82& 41& 124& 62& 31& 94& 47& 142& 71\\
214& 107& 322& 161& 484& 242& 121& 364& 182& 91\\
274& 137& 412& 206& 103& 310& 155& 466& 233& 700\\
350& 175& 526& 263& 790& 395& 1186& 593& 1780& 890\\
445& 1336& 668& 334& 167& 502& 251& 754& 377& 1132\\
566& 283& 850& 425& 1276& 638& 319& 958& 479& 1438\\
719& 2158& 1079& 3238& 1619& 4858& 2429& 7288& 3644& 1822\\
911& 2734& 1367& 4102& 2051& 6154& 3077& 9232& 4616& 2308\\
1154& 577& 1732& 866& 433& 1300& 650& 325& 976& 488\\
244& 122& 61& 184& 92& 46& 23& 70& 35& 106\\
53& 160& 80& 40& 20& 10& 5& 16& 8& 4\\
2& 1& \\

1550&&&&&&&&&\\
775& 2326& 1163& 3490& 1745& 5236& 2618& 1309& 3928& 1964\\
982& 491& 1474& 737& 2212& 1106& 553& 1660& 830& 415\\
1246& 623& 1870& 935& 2806& 1403& 4210& 2105& 6316& 3158\\
1579& 4738& 2369& 7108& 3554& 1777& 5332& 2666& 1333& 4000\\
2000& 1000& 500& 250& 125& 376& 188& 94& 47& 142\\
71& 214& 107& 322& 161& 484& 242& 121& 364& 182\\
91& 274& 137& 412& 206& 103& 310& 155& 466& 233\\
700& 350& 175& 526& 263& 790& 395& 1186& 593& 1780\\
890& 445& 1336& 668& 334& 167& 502& 251& 754& 377\\
1132& 566& 283& 850& 425& 1276& 638& 319& 958& 479\\
1438& 719& 2158& 1079& 3238& 1619& 4858& 2429& 7288& 3644\\
1822& 911& 2734& 1367& 4102& 2051& 6154& 3077& 9232& 4616\\
2308& 1154& 577& 1732& 866& 433& 1300& 650& 325& 976\\
488& 244& 122& 61& 184& 92& 46& 23& 70& 35\\
106& 53& 160& 80& 40& 20& 10& 5& 16& 8\\
4& 2& 1& \\

1551&&&&&&&&&\\
4654& 2327& 6982& 3491& 10474& 5237& 15712& 7856& 3928& 1964\\
982& 491& 1474& 737& 2212& 1106& 553& 1660& 830& 415\\
1246& 623& 1870& 935& 2806& 1403& 4210& 2105& 6316& 3158\\
1579& 4738& 2369& 7108& 3554& 1777& 5332& 2666& 1333& 4000\\
2000& 1000& 500& 250& 125& 376& 188& 94& 47& 142\\
71& 214& 107& 322& 161& 484& 242& 121& 364& 182\\
91& 274& 137& 412& 206& 103& 310& 155& 466& 233\\
700& 350& 175& 526& 263& 790& 395& 1186& 593& 1780\\
890& 445& 1336& 668& 334& 167& 502& 251& 754& 377\\
1132& 566& 283& 850& 425& 1276& 638& 319& 958& 479\\
1438& 719& 2158& 1079& 3238& 1619& 4858& 2429& 7288& 3644\\
1822& 911& 2734& 1367& 4102& 2051& 6154& 3077& 9232& 4616\\
2308& 1154& 577& 1732& 866& 433& 1300& 650& 325& 976\\
488& 244& 122& 61& 184& 92& 46& 23& 70& 35\\
106& 53& 160& 80& 40& 20& 10& 5& 16& 8\\
4& 2& 1& \\

1552&&&&&&&&&\\
776& 388& 194& 97& 292& 146& 73& 220& 110& 55\\
166& 83& 250& 125& 376& 188& 94& 47& 142& 71\\
214& 107& 322& 161& 484& 242& 121& 364& 182& 91\\
274& 137& 412& 206& 103& 310& 155& 466& 233& 700\\
350& 175& 526& 263& 790& 395& 1186& 593& 1780& 890\\
445& 1336& 668& 334& 167& 502& 251& 754& 377& 1132\\
566& 283& 850& 425& 1276& 638& 319& 958& 479& 1438\\
719& 2158& 1079& 3238& 1619& 4858& 2429& 7288& 3644& 1822\\
911& 2734& 1367& 4102& 2051& 6154& 3077& 9232& 4616& 2308\\
1154& 577& 1732& 866& 433& 1300& 650& 325& 976& 488\\
244& 122& 61& 184& 92& 46& 23& 70& 35& 106\\
53& 160& 80& 40& 20& 10& 5& 16& 8& 4\\
2& 1& \\

1553&&&&&&&&&\\
4660& 2330& 1165& 3496& 1748& 874& 437& 1312& 656& 328\\
164& 82& 41& 124& 62& 31& 94& 47& 142& 71\\
214& 107& 322& 161& 484& 242& 121& 364& 182& 91\\
274& 137& 412& 206& 103& 310& 155& 466& 233& 700\\
350& 175& 526& 263& 790& 395& 1186& 593& 1780& 890\\
445& 1336& 668& 334& 167& 502& 251& 754& 377& 1132\\
566& 283& 850& 425& 1276& 638& 319& 958& 479& 1438\\
719& 2158& 1079& 3238& 1619& 4858& 2429& 7288& 3644& 1822\\
911& 2734& 1367& 4102& 2051& 6154& 3077& 9232& 4616& 2308\\
1154& 577& 1732& 866& 433& 1300& 650& 325& 976& 488\\
244& 122& 61& 184& 92& 46& 23& 70& 35& 106\\
53& 160& 80& 40& 20& 10& 5& 16& 8& 4\\
2& 1& \\

1554&&&&&&&&&\\
777& 2332& 1166& 583& 1750& 875& 2626& 1313& 3940& 1970\\
985& 2956& 1478& 739& 2218& 1109& 3328& 1664& 832& 416\\
208& 104& 52& 26& 13& 40& 20& 10& 5& 16\\
8& 4& 2& 1& \\

1555&&&&&&&&&\\
4666& 2333& 7000& 3500& 1750& 875& 2626& 1313& 3940& 1970\\
985& 2956& 1478& 739& 2218& 1109& 3328& 1664& 832& 416\\
208& 104& 52& 26& 13& 40& 20& 10& 5& 16\\
8& 4& 2& 1& \\

1556&&&&&&&&&\\
778& 389& 1168& 584& 292& 146& 73& 220& 110& 55\\
166& 83& 250& 125& 376& 188& 94& 47& 142& 71\\
214& 107& 322& 161& 484& 242& 121& 364& 182& 91\\
274& 137& 412& 206& 103& 310& 155& 466& 233& 700\\
350& 175& 526& 263& 790& 395& 1186& 593& 1780& 890\\
445& 1336& 668& 334& 167& 502& 251& 754& 377& 1132\\
566& 283& 850& 425& 1276& 638& 319& 958& 479& 1438\\
719& 2158& 1079& 3238& 1619& 4858& 2429& 7288& 3644& 1822\\
911& 2734& 1367& 4102& 2051& 6154& 3077& 9232& 4616& 2308\\
1154& 577& 1732& 866& 433& 1300& 650& 325& 976& 488\\
244& 122& 61& 184& 92& 46& 23& 70& 35& 106\\
53& 160& 80& 40& 20& 10& 5& 16& 8& 4\\
2& 1& \\

1557&&&&&&&&&\\
4672& 2336& 1168& 584& 292& 146& 73& 220& 110& 55\\
166& 83& 250& 125& 376& 188& 94& 47& 142& 71\\
214& 107& 322& 161& 484& 242& 121& 364& 182& 91\\
274& 137& 412& 206& 103& 310& 155& 466& 233& 700\\
350& 175& 526& 263& 790& 395& 1186& 593& 1780& 890\\
445& 1336& 668& 334& 167& 502& 251& 754& 377& 1132\\
566& 283& 850& 425& 1276& 638& 319& 958& 479& 1438\\
719& 2158& 1079& 3238& 1619& 4858& 2429& 7288& 3644& 1822\\
911& 2734& 1367& 4102& 2051& 6154& 3077& 9232& 4616& 2308\\
1154& 577& 1732& 866& 433& 1300& 650& 325& 976& 488\\
244& 122& 61& 184& 92& 46& 23& 70& 35& 106\\
53& 160& 80& 40& 20& 10& 5& 16& 8& 4\\
2& 1& \\

1558&&&&&&&&&\\
779& 2338& 1169& 3508& 1754& 877& 2632& 1316& 658& 329\\
988& 494& 247& 742& 371& 1114& 557& 1672& 836& 418\\
209& 628& 314& 157& 472& 236& 118& 59& 178& 89\\
268& 134& 67& 202& 101& 304& 152& 76& 38& 19\\
58& 29& 88& 44& 22& 11& 34& 17& 52& 26\\
13& 40& 20& 10& 5& 16& 8& 4& 2& 1\\

1559&&&&&&&&&\\
4678& 2339& 7018& 3509& 10528& 5264& 2632& 1316& 658& 329\\
988& 494& 247& 742& 371& 1114& 557& 1672& 836& 418\\
209& 628& 314& 157& 472& 236& 118& 59& 178& 89\\
268& 134& 67& 202& 101& 304& 152& 76& 38& 19\\
58& 29& 88& 44& 22& 11& 34& 17& 52& 26\\
13& 40& 20& 10& 5& 16& 8& 4& 2& 1\\

1560&&&&&&&&&\\
780& 390& 195& 586& 293& 880& 440& 220& 110& 55\\
166& 83& 250& 125& 376& 188& 94& 47& 142& 71\\
214& 107& 322& 161& 484& 242& 121& 364& 182& 91\\
274& 137& 412& 206& 103& 310& 155& 466& 233& 700\\
350& 175& 526& 263& 790& 395& 1186& 593& 1780& 890\\
445& 1336& 668& 334& 167& 502& 251& 754& 377& 1132\\
566& 283& 850& 425& 1276& 638& 319& 958& 479& 1438\\
719& 2158& 1079& 3238& 1619& 4858& 2429& 7288& 3644& 1822\\
911& 2734& 1367& 4102& 2051& 6154& 3077& 9232& 4616& 2308\\
1154& 577& 1732& 866& 433& 1300& 650& 325& 976& 488\\
244& 122& 61& 184& 92& 46& 23& 70& 35& 106\\
53& 160& 80& 40& 20& 10& 5& 16& 8& 4\\
2& 1& \\

1561&&&&&&&&&\\
4684& 2342& 1171& 3514& 1757& 5272& 2636& 1318& 659& 1978\\
989& 2968& 1484& 742& 371& 1114& 557& 1672& 836& 418\\
209& 628& 314& 157& 472& 236& 118& 59& 178& 89\\
268& 134& 67& 202& 101& 304& 152& 76& 38& 19\\
58& 29& 88& 44& 22& 11& 34& 17& 52& 26\\
13& 40& 20& 10& 5& 16& 8& 4& 2& 1\\

1562&&&&&&&&&\\
781& 2344& 1172& 586& 293& 880& 440& 220& 110& 55\\
166& 83& 250& 125& 376& 188& 94& 47& 142& 71\\
214& 107& 322& 161& 484& 242& 121& 364& 182& 91\\
274& 137& 412& 206& 103& 310& 155& 466& 233& 700\\
350& 175& 526& 263& 790& 395& 1186& 593& 1780& 890\\
445& 1336& 668& 334& 167& 502& 251& 754& 377& 1132\\
566& 283& 850& 425& 1276& 638& 319& 958& 479& 1438\\
719& 2158& 1079& 3238& 1619& 4858& 2429& 7288& 3644& 1822\\
911& 2734& 1367& 4102& 2051& 6154& 3077& 9232& 4616& 2308\\
1154& 577& 1732& 866& 433& 1300& 650& 325& 976& 488\\
244& 122& 61& 184& 92& 46& 23& 70& 35& 106\\
53& 160& 80& 40& 20& 10& 5& 16& 8& 4\\
2& 1& \\

1563&&&&&&&&&\\
4690& 2345& 7036& 3518& 1759& 5278& 2639& 7918& 3959& 11878\\
5939& 17818& 8909& 26728& 13364& 6682& 3341& 10024& 5012& 2506\\
1253& 3760& 1880& 940& 470& 235& 706& 353& 1060& 530\\
265& 796& 398& 199& 598& 299& 898& 449& 1348& 674\\
337& 1012& 506& 253& 760& 380& 190& 95& 286& 143\\
430& 215& 646& 323& 970& 485& 1456& 728& 364& 182\\
91& 274& 137& 412& 206& 103& 310& 155& 466& 233\\
700& 350& 175& 526& 263& 790& 395& 1186& 593& 1780\\
890& 445& 1336& 668& 334& 167& 502& 251& 754& 377\\
1132& 566& 283& 850& 425& 1276& 638& 319& 958& 479\\
1438& 719& 2158& 1079& 3238& 1619& 4858& 2429& 7288& 3644\\
1822& 911& 2734& 1367& 4102& 2051& 6154& 3077& 9232& 4616\\
2308& 1154& 577& 1732& 866& 433& 1300& 650& 325& 976\\
488& 244& 122& 61& 184& 92& 46& 23& 70& 35\\
106& 53& 160& 80& 40& 20& 10& 5& 16& 8\\
4& 2& 1& \\

1564&&&&&&&&&\\
782& 391& 1174& 587& 1762& 881& 2644& 1322& 661& 1984\\
992& 496& 248& 124& 62& 31& 94& 47& 142& 71\\
214& 107& 322& 161& 484& 242& 121& 364& 182& 91\\
274& 137& 412& 206& 103& 310& 155& 466& 233& 700\\
350& 175& 526& 263& 790& 395& 1186& 593& 1780& 890\\
445& 1336& 668& 334& 167& 502& 251& 754& 377& 1132\\
566& 283& 850& 425& 1276& 638& 319& 958& 479& 1438\\
719& 2158& 1079& 3238& 1619& 4858& 2429& 7288& 3644& 1822\\
911& 2734& 1367& 4102& 2051& 6154& 3077& 9232& 4616& 2308\\
1154& 577& 1732& 866& 433& 1300& 650& 325& 976& 488\\
244& 122& 61& 184& 92& 46& 23& 70& 35& 106\\
53& 160& 80& 40& 20& 10& 5& 16& 8& 4\\
2& 1& \\

1565&&&&&&&&&\\
4696& 2348& 1174& 587& 1762& 881& 2644& 1322& 661& 1984\\
992& 496& 248& 124& 62& 31& 94& 47& 142& 71\\
214& 107& 322& 161& 484& 242& 121& 364& 182& 91\\
274& 137& 412& 206& 103& 310& 155& 466& 233& 700\\
350& 175& 526& 263& 790& 395& 1186& 593& 1780& 890\\
445& 1336& 668& 334& 167& 502& 251& 754& 377& 1132\\
566& 283& 850& 425& 1276& 638& 319& 958& 479& 1438\\
719& 2158& 1079& 3238& 1619& 4858& 2429& 7288& 3644& 1822\\
911& 2734& 1367& 4102& 2051& 6154& 3077& 9232& 4616& 2308\\
1154& 577& 1732& 866& 433& 1300& 650& 325& 976& 488\\
244& 122& 61& 184& 92& 46& 23& 70& 35& 106\\
53& 160& 80& 40& 20& 10& 5& 16& 8& 4\\
2& 1& \\

1566&&&&&&&&&\\
783& 2350& 1175& 3526& 1763& 5290& 2645& 7936& 3968& 1984\\
992& 496& 248& 124& 62& 31& 94& 47& 142& 71\\
214& 107& 322& 161& 484& 242& 121& 364& 182& 91\\
274& 137& 412& 206& 103& 310& 155& 466& 233& 700\\
350& 175& 526& 263& 790& 395& 1186& 593& 1780& 890\\
445& 1336& 668& 334& 167& 502& 251& 754& 377& 1132\\
566& 283& 850& 425& 1276& 638& 319& 958& 479& 1438\\
719& 2158& 1079& 3238& 1619& 4858& 2429& 7288& 3644& 1822\\
911& 2734& 1367& 4102& 2051& 6154& 3077& 9232& 4616& 2308\\
1154& 577& 1732& 866& 433& 1300& 650& 325& 976& 488\\
244& 122& 61& 184& 92& 46& 23& 70& 35& 106\\
53& 160& 80& 40& 20& 10& 5& 16& 8& 4\\
2& 1& \\

1567&&&&&&&&&\\
4702& 2351& 7054& 3527& 10582& 5291& 15874& 7937& 23812& 11906\\
5953& 17860& 8930& 4465& 13396& 6698& 3349& 10048& 5024& 2512\\
1256& 628& 314& 157& 472& 236& 118& 59& 178& 89\\
268& 134& 67& 202& 101& 304& 152& 76& 38& 19\\
58& 29& 88& 44& 22& 11& 34& 17& 52& 26\\
13& 40& 20& 10& 5& 16& 8& 4& 2& 1\\

1568&&&&&&&&&\\
784& 392& 196& 98& 49& 148& 74& 37& 112& 56\\
28& 14& 7& 22& 11& 34& 17& 52& 26& 13\\
40& 20& 10& 5& 16& 8& 4& 2& 1& \\

1569&&&&&&&&&\\
4708& 2354& 1177& 3532& 1766& 883& 2650& 1325& 3976& 1988\\
994& 497& 1492& 746& 373& 1120& 560& 280& 140& 70\\
35& 106& 53& 160& 80& 40& 20& 10& 5& 16\\
8& 4& 2& 1& \\

1570&&&&&&&&&\\
785& 2356& 1178& 589& 1768& 884& 442& 221& 664& 332\\
166& 83& 250& 125& 376& 188& 94& 47& 142& 71\\
214& 107& 322& 161& 484& 242& 121& 364& 182& 91\\
274& 137& 412& 206& 103& 310& 155& 466& 233& 700\\
350& 175& 526& 263& 790& 395& 1186& 593& 1780& 890\\
445& 1336& 668& 334& 167& 502& 251& 754& 377& 1132\\
566& 283& 850& 425& 1276& 638& 319& 958& 479& 1438\\
719& 2158& 1079& 3238& 1619& 4858& 2429& 7288& 3644& 1822\\
911& 2734& 1367& 4102& 2051& 6154& 3077& 9232& 4616& 2308\\
1154& 577& 1732& 866& 433& 1300& 650& 325& 976& 488\\
244& 122& 61& 184& 92& 46& 23& 70& 35& 106\\
53& 160& 80& 40& 20& 10& 5& 16& 8& 4\\
2& 1& \\

1571&&&&&&&&&\\
4714& 2357& 7072& 3536& 1768& 884& 442& 221& 664& 332\\
166& 83& 250& 125& 376& 188& 94& 47& 142& 71\\
214& 107& 322& 161& 484& 242& 121& 364& 182& 91\\
274& 137& 412& 206& 103& 310& 155& 466& 233& 700\\
350& 175& 526& 263& 790& 395& 1186& 593& 1780& 890\\
445& 1336& 668& 334& 167& 502& 251& 754& 377& 1132\\
566& 283& 850& 425& 1276& 638& 319& 958& 479& 1438\\
719& 2158& 1079& 3238& 1619& 4858& 2429& 7288& 3644& 1822\\
911& 2734& 1367& 4102& 2051& 6154& 3077& 9232& 4616& 2308\\
1154& 577& 1732& 866& 433& 1300& 650& 325& 976& 488\\
244& 122& 61& 184& 92& 46& 23& 70& 35& 106\\
53& 160& 80& 40& 20& 10& 5& 16& 8& 4\\
2& 1& \\

1572&&&&&&&&&\\
786& 393& 1180& 590& 295& 886& 443& 1330& 665& 1996\\
998& 499& 1498& 749& 2248& 1124& 562& 281& 844& 422\\
211& 634& 317& 952& 476& 238& 119& 358& 179& 538\\
269& 808& 404& 202& 101& 304& 152& 76& 38& 19\\
58& 29& 88& 44& 22& 11& 34& 17& 52& 26\\
13& 40& 20& 10& 5& 16& 8& 4& 2& 1\\

1573&&&&&&&&&\\
4720& 2360& 1180& 590& 295& 886& 443& 1330& 665& 1996\\
998& 499& 1498& 749& 2248& 1124& 562& 281& 844& 422\\
211& 634& 317& 952& 476& 238& 119& 358& 179& 538\\
269& 808& 404& 202& 101& 304& 152& 76& 38& 19\\
58& 29& 88& 44& 22& 11& 34& 17& 52& 26\\
13& 40& 20& 10& 5& 16& 8& 4& 2& 1\\

1574&&&&&&&&&\\
787& 2362& 1181& 3544& 1772& 886& 443& 1330& 665& 1996\\
998& 499& 1498& 749& 2248& 1124& 562& 281& 844& 422\\
211& 634& 317& 952& 476& 238& 119& 358& 179& 538\\
269& 808& 404& 202& 101& 304& 152& 76& 38& 19\\
58& 29& 88& 44& 22& 11& 34& 17& 52& 26\\
13& 40& 20& 10& 5& 16& 8& 4& 2& 1\\

1575&&&&&&&&&\\
4726& 2363& 7090& 3545& 10636& 5318& 2659& 7978& 3989& 11968\\
5984& 2992& 1496& 748& 374& 187& 562& 281& 844& 422\\
211& 634& 317& 952& 476& 238& 119& 358& 179& 538\\
269& 808& 404& 202& 101& 304& 152& 76& 38& 19\\
58& 29& 88& 44& 22& 11& 34& 17& 52& 26\\
13& 40& 20& 10& 5& 16& 8& 4& 2& 1\\

1576&&&&&&&&&\\
788& 394& 197& 592& 296& 148& 74& 37& 112& 56\\
28& 14& 7& 22& 11& 34& 17& 52& 26& 13\\
40& 20& 10& 5& 16& 8& 4& 2& 1& \\

1577&&&&&&&&&\\
4732& 2366& 1183& 3550& 1775& 5326& 2663& 7990& 3995& 11986\\
5993& 17980& 8990& 4495& 13486& 6743& 20230& 10115& 30346& 15173\\
45520& 22760& 11380& 5690& 2845& 8536& 4268& 2134& 1067& 3202\\
1601& 4804& 2402& 1201& 3604& 1802& 901& 2704& 1352& 676\\
338& 169& 508& 254& 127& 382& 191& 574& 287& 862\\
431& 1294& 647& 1942& 971& 2914& 1457& 4372& 2186& 1093\\
3280& 1640& 820& 410& 205& 616& 308& 154& 77& 232\\
116& 58& 29& 88& 44& 22& 11& 34& 17& 52\\
26& 13& 40& 20& 10& 5& 16& 8& 4& 2\\
1& \\

1578&&&&&&&&&\\
789& 2368& 1184& 592& 296& 148& 74& 37& 112& 56\\
28& 14& 7& 22& 11& 34& 17& 52& 26& 13\\
40& 20& 10& 5& 16& 8& 4& 2& 1& \\

1579&&&&&&&&&\\
4738& 2369& 7108& 3554& 1777& 5332& 2666& 1333& 4000& 2000\\
1000& 500& 250& 125& 376& 188& 94& 47& 142& 71\\
214& 107& 322& 161& 484& 242& 121& 364& 182& 91\\
274& 137& 412& 206& 103& 310& 155& 466& 233& 700\\
350& 175& 526& 263& 790& 395& 1186& 593& 1780& 890\\
445& 1336& 668& 334& 167& 502& 251& 754& 377& 1132\\
566& 283& 850& 425& 1276& 638& 319& 958& 479& 1438\\
719& 2158& 1079& 3238& 1619& 4858& 2429& 7288& 3644& 1822\\
911& 2734& 1367& 4102& 2051& 6154& 3077& 9232& 4616& 2308\\
1154& 577& 1732& 866& 433& 1300& 650& 325& 976& 488\\
244& 122& 61& 184& 92& 46& 23& 70& 35& 106\\
53& 160& 80& 40& 20& 10& 5& 16& 8& 4\\
2& 1& \\

1580&&&&&&&&&\\
790& 395& 1186& 593& 1780& 890& 445& 1336& 668& 334\\
167& 502& 251& 754& 377& 1132& 566& 283& 850& 425\\
1276& 638& 319& 958& 479& 1438& 719& 2158& 1079& 3238\\
1619& 4858& 2429& 7288& 3644& 1822& 911& 2734& 1367& 4102\\
2051& 6154& 3077& 9232& 4616& 2308& 1154& 577& 1732& 866\\
433& 1300& 650& 325& 976& 488& 244& 122& 61& 184\\
92& 46& 23& 70& 35& 106& 53& 160& 80& 40\\
20& 10& 5& 16& 8& 4& 2& 1& \\

1581&&&&&&&&&\\
4744& 2372& 1186& 593& 1780& 890& 445& 1336& 668& 334\\
167& 502& 251& 754& 377& 1132& 566& 283& 850& 425\\
1276& 638& 319& 958& 479& 1438& 719& 2158& 1079& 3238\\
1619& 4858& 2429& 7288& 3644& 1822& 911& 2734& 1367& 4102\\
2051& 6154& 3077& 9232& 4616& 2308& 1154& 577& 1732& 866\\
433& 1300& 650& 325& 976& 488& 244& 122& 61& 184\\
92& 46& 23& 70& 35& 106& 53& 160& 80& 40\\
20& 10& 5& 16& 8& 4& 2& 1& \\

1582&&&&&&&&&\\
791& 2374& 1187& 3562& 1781& 5344& 2672& 1336& 668& 334\\
167& 502& 251& 754& 377& 1132& 566& 283& 850& 425\\
1276& 638& 319& 958& 479& 1438& 719& 2158& 1079& 3238\\
1619& 4858& 2429& 7288& 3644& 1822& 911& 2734& 1367& 4102\\
2051& 6154& 3077& 9232& 4616& 2308& 1154& 577& 1732& 866\\
433& 1300& 650& 325& 976& 488& 244& 122& 61& 184\\
92& 46& 23& 70& 35& 106& 53& 160& 80& 40\\
20& 10& 5& 16& 8& 4& 2& 1& \\

1583&&&&&&&&&\\
4750& 2375& 7126& 3563& 10690& 5345& 16036& 8018& 4009& 12028\\
6014& 3007& 9022& 4511& 13534& 6767& 20302& 10151& 30454& 15227\\
45682& 22841& 68524& 34262& 17131& 51394& 25697& 77092& 38546& 19273\\
57820& 28910& 14455& 43366& 21683& 65050& 32525& 97576& 48788& 24394\\
12197& 36592& 18296& 9148& 4574& 2287& 6862& 3431& 10294& 5147\\
15442& 7721& 23164& 11582& 5791& 17374& 8687& 26062& 13031& 39094\\
19547& 58642& 29321& 87964& 43982& 21991& 65974& 32987& 98962& 49481\\
148444& 74222& 37111& 111334& 55667& 167002& 83501& 250504& 125252& 62626\\
31313& 93940& 46970& 23485& 70456& 35228& 17614& 8807& 26422& 13211\\
39634& 19817& 59452& 29726& 14863& 44590& 22295& 66886& 33443& 100330\\
50165& 150496& 75248& 37624& 18812& 9406& 4703& 14110& 7055& 21166\\
10583& 31750& 15875& 47626& 23813& 71440& 35720& 17860& 8930& 4465\\
13396& 6698& 3349& 10048& 5024& 2512& 1256& 628& 314& 157\\
472& 236& 118& 59& 178& 89& 268& 134& 67& 202\\
101& 304& 152& 76& 38& 19& 58& 29& 88& 44\\
22& 11& 34& 17& 52& 26& 13& 40& 20& 10\\
5& 16& 8& 4& 2& 1& \\

1584&&&&&&&&&\\
792& 396& 198& 99& 298& 149& 448& 224& 112& 56\\
28& 14& 7& 22& 11& 34& 17& 52& 26& 13\\
40& 20& 10& 5& 16& 8& 4& 2& 1& \\

1585&&&&&&&&&\\
4756& 2378& 1189& 3568& 1784& 892& 446& 223& 670& 335\\
1006& 503& 1510& 755& 2266& 1133& 3400& 1700& 850& 425\\
1276& 638& 319& 958& 479& 1438& 719& 2158& 1079& 3238\\
1619& 4858& 2429& 7288& 3644& 1822& 911& 2734& 1367& 4102\\
2051& 6154& 3077& 9232& 4616& 2308& 1154& 577& 1732& 866\\
433& 1300& 650& 325& 976& 488& 244& 122& 61& 184\\
92& 46& 23& 70& 35& 106& 53& 160& 80& 40\\
20& 10& 5& 16& 8& 4& 2& 1& \\

1586&&&&&&&&&\\
793& 2380& 1190& 595& 1786& 893& 2680& 1340& 670& 335\\
1006& 503& 1510& 755& 2266& 1133& 3400& 1700& 850& 425\\
1276& 638& 319& 958& 479& 1438& 719& 2158& 1079& 3238\\
1619& 4858& 2429& 7288& 3644& 1822& 911& 2734& 1367& 4102\\
2051& 6154& 3077& 9232& 4616& 2308& 1154& 577& 1732& 866\\
433& 1300& 650& 325& 976& 488& 244& 122& 61& 184\\
92& 46& 23& 70& 35& 106& 53& 160& 80& 40\\
20& 10& 5& 16& 8& 4& 2& 1& \\

1587&&&&&&&&&\\
4762& 2381& 7144& 3572& 1786& 893& 2680& 1340& 670& 335\\
1006& 503& 1510& 755& 2266& 1133& 3400& 1700& 850& 425\\
1276& 638& 319& 958& 479& 1438& 719& 2158& 1079& 3238\\
1619& 4858& 2429& 7288& 3644& 1822& 911& 2734& 1367& 4102\\
2051& 6154& 3077& 9232& 4616& 2308& 1154& 577& 1732& 866\\
433& 1300& 650& 325& 976& 488& 244& 122& 61& 184\\
92& 46& 23& 70& 35& 106& 53& 160& 80& 40\\
20& 10& 5& 16& 8& 4& 2& 1& \\

1588&&&&&&&&&\\
794& 397& 1192& 596& 298& 149& 448& 224& 112& 56\\
28& 14& 7& 22& 11& 34& 17& 52& 26& 13\\
40& 20& 10& 5& 16& 8& 4& 2& 1& \\

1589&&&&&&&&&\\
4768& 2384& 1192& 596& 298& 149& 448& 224& 112& 56\\
28& 14& 7& 22& 11& 34& 17& 52& 26& 13\\
40& 20& 10& 5& 16& 8& 4& 2& 1& \\

1590&&&&&&&&&\\
795& 2386& 1193& 3580& 1790& 895& 2686& 1343& 4030& 2015\\
6046& 3023& 9070& 4535& 13606& 6803& 20410& 10205& 30616& 15308\\
7654& 3827& 11482& 5741& 17224& 8612& 4306& 2153& 6460& 3230\\
1615& 4846& 2423& 7270& 3635& 10906& 5453& 16360& 8180& 4090\\
2045& 6136& 3068& 1534& 767& 2302& 1151& 3454& 1727& 5182\\
2591& 7774& 3887& 11662& 5831& 17494& 8747& 26242& 13121& 39364\\
19682& 9841& 29524& 14762& 7381& 22144& 11072& 5536& 2768& 1384\\
692& 346& 173& 520& 260& 130& 65& 196& 98& 49\\
148& 74& 37& 112& 56& 28& 14& 7& 22& 11\\
34& 17& 52& 26& 13& 40& 20& 10& 5& 16\\
8& 4& 2& 1& \\

1591&&&&&&&&&\\
4774& 2387& 7162& 3581& 10744& 5372& 2686& 1343& 4030& 2015\\
6046& 3023& 9070& 4535& 13606& 6803& 20410& 10205& 30616& 15308\\
7654& 3827& 11482& 5741& 17224& 8612& 4306& 2153& 6460& 3230\\
1615& 4846& 2423& 7270& 3635& 10906& 5453& 16360& 8180& 4090\\
2045& 6136& 3068& 1534& 767& 2302& 1151& 3454& 1727& 5182\\
2591& 7774& 3887& 11662& 5831& 17494& 8747& 26242& 13121& 39364\\
19682& 9841& 29524& 14762& 7381& 22144& 11072& 5536& 2768& 1384\\
692& 346& 173& 520& 260& 130& 65& 196& 98& 49\\
148& 74& 37& 112& 56& 28& 14& 7& 22& 11\\
34& 17& 52& 26& 13& 40& 20& 10& 5& 16\\
8& 4& 2& 1& \\

1592&&&&&&&&&\\
796& 398& 199& 598& 299& 898& 449& 1348& 674& 337\\
1012& 506& 253& 760& 380& 190& 95& 286& 143& 430\\
215& 646& 323& 970& 485& 1456& 728& 364& 182& 91\\
274& 137& 412& 206& 103& 310& 155& 466& 233& 700\\
350& 175& 526& 263& 790& 395& 1186& 593& 1780& 890\\
445& 1336& 668& 334& 167& 502& 251& 754& 377& 1132\\
566& 283& 850& 425& 1276& 638& 319& 958& 479& 1438\\
719& 2158& 1079& 3238& 1619& 4858& 2429& 7288& 3644& 1822\\
911& 2734& 1367& 4102& 2051& 6154& 3077& 9232& 4616& 2308\\
1154& 577& 1732& 866& 433& 1300& 650& 325& 976& 488\\
244& 122& 61& 184& 92& 46& 23& 70& 35& 106\\
53& 160& 80& 40& 20& 10& 5& 16& 8& 4\\
2& 1& \\

1593&&&&&&&&&\\
4780& 2390& 1195& 3586& 1793& 5380& 2690& 1345& 4036& 2018\\
1009& 3028& 1514& 757& 2272& 1136& 568& 284& 142& 71\\
214& 107& 322& 161& 484& 242& 121& 364& 182& 91\\
274& 137& 412& 206& 103& 310& 155& 466& 233& 700\\
350& 175& 526& 263& 790& 395& 1186& 593& 1780& 890\\
445& 1336& 668& 334& 167& 502& 251& 754& 377& 1132\\
566& 283& 850& 425& 1276& 638& 319& 958& 479& 1438\\
719& 2158& 1079& 3238& 1619& 4858& 2429& 7288& 3644& 1822\\
911& 2734& 1367& 4102& 2051& 6154& 3077& 9232& 4616& 2308\\
1154& 577& 1732& 866& 433& 1300& 650& 325& 976& 488\\
244& 122& 61& 184& 92& 46& 23& 70& 35& 106\\
53& 160& 80& 40& 20& 10& 5& 16& 8& 4\\
2& 1& \\

1594&&&&&&&&&\\
797& 2392& 1196& 598& 299& 898& 449& 1348& 674& 337\\
1012& 506& 253& 760& 380& 190& 95& 286& 143& 430\\
215& 646& 323& 970& 485& 1456& 728& 364& 182& 91\\
274& 137& 412& 206& 103& 310& 155& 466& 233& 700\\
350& 175& 526& 263& 790& 395& 1186& 593& 1780& 890\\
445& 1336& 668& 334& 167& 502& 251& 754& 377& 1132\\
566& 283& 850& 425& 1276& 638& 319& 958& 479& 1438\\
719& 2158& 1079& 3238& 1619& 4858& 2429& 7288& 3644& 1822\\
911& 2734& 1367& 4102& 2051& 6154& 3077& 9232& 4616& 2308\\
1154& 577& 1732& 866& 433& 1300& 650& 325& 976& 488\\
244& 122& 61& 184& 92& 46& 23& 70& 35& 106\\
53& 160& 80& 40& 20& 10& 5& 16& 8& 4\\
2& 1& \\

1595&&&&&&&&&\\
4786& 2393& 7180& 3590& 1795& 5386& 2693& 8080& 4040& 2020\\
1010& 505& 1516& 758& 379& 1138& 569& 1708& 854& 427\\
1282& 641& 1924& 962& 481& 1444& 722& 361& 1084& 542\\
271& 814& 407& 1222& 611& 1834& 917& 2752& 1376& 688\\
344& 172& 86& 43& 130& 65& 196& 98& 49& 148\\
74& 37& 112& 56& 28& 14& 7& 22& 11& 34\\
17& 52& 26& 13& 40& 20& 10& 5& 16& 8\\
4& 2& 1& \\

1596&&&&&&&&&\\
798& 399& 1198& 599& 1798& 899& 2698& 1349& 4048& 2024\\
1012& 506& 253& 760& 380& 190& 95& 286& 143& 430\\
215& 646& 323& 970& 485& 1456& 728& 364& 182& 91\\
274& 137& 412& 206& 103& 310& 155& 466& 233& 700\\
350& 175& 526& 263& 790& 395& 1186& 593& 1780& 890\\
445& 1336& 668& 334& 167& 502& 251& 754& 377& 1132\\
566& 283& 850& 425& 1276& 638& 319& 958& 479& 1438\\
719& 2158& 1079& 3238& 1619& 4858& 2429& 7288& 3644& 1822\\
911& 2734& 1367& 4102& 2051& 6154& 3077& 9232& 4616& 2308\\
1154& 577& 1732& 866& 433& 1300& 650& 325& 976& 488\\
244& 122& 61& 184& 92& 46& 23& 70& 35& 106\\
53& 160& 80& 40& 20& 10& 5& 16& 8& 4\\
2& 1& \\

1597&&&&&&&&&\\
4792& 2396& 1198& 599& 1798& 899& 2698& 1349& 4048& 2024\\
1012& 506& 253& 760& 380& 190& 95& 286& 143& 430\\
215& 646& 323& 970& 485& 1456& 728& 364& 182& 91\\
274& 137& 412& 206& 103& 310& 155& 466& 233& 700\\
350& 175& 526& 263& 790& 395& 1186& 593& 1780& 890\\
445& 1336& 668& 334& 167& 502& 251& 754& 377& 1132\\
566& 283& 850& 425& 1276& 638& 319& 958& 479& 1438\\
719& 2158& 1079& 3238& 1619& 4858& 2429& 7288& 3644& 1822\\
911& 2734& 1367& 4102& 2051& 6154& 3077& 9232& 4616& 2308\\
1154& 577& 1732& 866& 433& 1300& 650& 325& 976& 488\\
244& 122& 61& 184& 92& 46& 23& 70& 35& 106\\
53& 160& 80& 40& 20& 10& 5& 16& 8& 4\\
2& 1& \\

1598&&&&&&&&&\\
799& 2398& 1199& 3598& 1799& 5398& 2699& 8098& 4049& 12148\\
6074& 3037& 9112& 4556& 2278& 1139& 3418& 1709& 5128& 2564\\
1282& 641& 1924& 962& 481& 1444& 722& 361& 1084& 542\\
271& 814& 407& 1222& 611& 1834& 917& 2752& 1376& 688\\
344& 172& 86& 43& 130& 65& 196& 98& 49& 148\\
74& 37& 112& 56& 28& 14& 7& 22& 11& 34\\
17& 52& 26& 13& 40& 20& 10& 5& 16& 8\\
4& 2& 1& \\

1599&&&&&&&&&\\
4798& 2399& 7198& 3599& 10798& 5399& 16198& 8099& 24298& 12149\\
36448& 18224& 9112& 4556& 2278& 1139& 3418& 1709& 5128& 2564\\
1282& 641& 1924& 962& 481& 1444& 722& 361& 1084& 542\\
271& 814& 407& 1222& 611& 1834& 917& 2752& 1376& 688\\
344& 172& 86& 43& 130& 65& 196& 98& 49& 148\\
74& 37& 112& 56& 28& 14& 7& 22& 11& 34\\
17& 52& 26& 13& 40& 20& 10& 5& 16& 8\\
4& 2& 1& \\

1600&&&&&&&&&\\
800& 400& 200& 100& 50& 25& 76& 38& 19& 58\\
29& 88& 44& 22& 11& 34& 17& 52& 26& 13\\
40& 20& 10& 5& 16& 8& 4& 2& 1& \\

1601&&&&&&&&&\\
4804& 2402& 1201& 3604& 1802& 901& 2704& 1352& 676& 338\\
169& 508& 254& 127& 382& 191& 574& 287& 862& 431\\
1294& 647& 1942& 971& 2914& 1457& 4372& 2186& 1093& 3280\\
1640& 820& 410& 205& 616& 308& 154& 77& 232& 116\\
58& 29& 88& 44& 22& 11& 34& 17& 52& 26\\
13& 40& 20& 10& 5& 16& 8& 4& 2& 1\\

1602&&&&&&&&&\\
801& 2404& 1202& 601& 1804& 902& 451& 1354& 677& 2032\\
1016& 508& 254& 127& 382& 191& 574& 287& 862& 431\\
1294& 647& 1942& 971& 2914& 1457& 4372& 2186& 1093& 3280\\
1640& 820& 410& 205& 616& 308& 154& 77& 232& 116\\
58& 29& 88& 44& 22& 11& 34& 17& 52& 26\\
13& 40& 20& 10& 5& 16& 8& 4& 2& 1\\

1603&&&&&&&&&\\
4810& 2405& 7216& 3608& 1804& 902& 451& 1354& 677& 2032\\
1016& 508& 254& 127& 382& 191& 574& 287& 862& 431\\
1294& 647& 1942& 971& 2914& 1457& 4372& 2186& 1093& 3280\\
1640& 820& 410& 205& 616& 308& 154& 77& 232& 116\\
58& 29& 88& 44& 22& 11& 34& 17& 52& 26\\
13& 40& 20& 10& 5& 16& 8& 4& 2& 1\\

1604&&&&&&&&&\\
802& 401& 1204& 602& 301& 904& 452& 226& 113& 340\\
170& 85& 256& 128& 64& 32& 16& 8& 4& 2\\
1& \\

1605&&&&&&&&&\\
4816& 2408& 1204& 602& 301& 904& 452& 226& 113& 340\\
170& 85& 256& 128& 64& 32& 16& 8& 4& 2\\
1& \\

1606&&&&&&&&&\\
803& 2410& 1205& 3616& 1808& 904& 452& 226& 113& 340\\
170& 85& 256& 128& 64& 32& 16& 8& 4& 2\\
1& \\

1607&&&&&&&&&\\
4822& 2411& 7234& 3617& 10852& 5426& 2713& 8140& 4070& 2035\\
6106& 3053& 9160& 4580& 2290& 1145& 3436& 1718& 859& 2578\\
1289& 3868& 1934& 967& 2902& 1451& 4354& 2177& 6532& 3266\\
1633& 4900& 2450& 1225& 3676& 1838& 919& 2758& 1379& 4138\\
2069& 6208& 3104& 1552& 776& 388& 194& 97& 292& 146\\
73& 220& 110& 55& 166& 83& 250& 125& 376& 188\\
94& 47& 142& 71& 214& 107& 322& 161& 484& 242\\
121& 364& 182& 91& 274& 137& 412& 206& 103& 310\\
155& 466& 233& 700& 350& 175& 526& 263& 790& 395\\
1186& 593& 1780& 890& 445& 1336& 668& 334& 167& 502\\
251& 754& 377& 1132& 566& 283& 850& 425& 1276& 638\\
319& 958& 479& 1438& 719& 2158& 1079& 3238& 1619& 4858\\
2429& 7288& 3644& 1822& 911& 2734& 1367& 4102& 2051& 6154\\
3077& 9232& 4616& 2308& 1154& 577& 1732& 866& 433& 1300\\
650& 325& 976& 488& 244& 122& 61& 184& 92& 46\\
23& 70& 35& 106& 53& 160& 80& 40& 20& 10\\
5& 16& 8& 4& 2& 1& \\

1608&&&&&&&&&\\
804& 402& 201& 604& 302& 151& 454& 227& 682& 341\\
1024& 512& 256& 128& 64& 32& 16& 8& 4& 2\\
1& \\

1609&&&&&&&&&\\
4828& 2414& 1207& 3622& 1811& 5434& 2717& 8152& 4076& 2038\\
1019& 3058& 1529& 4588& 2294& 1147& 3442& 1721& 5164& 2582\\
1291& 3874& 1937& 5812& 2906& 1453& 4360& 2180& 1090& 545\\
1636& 818& 409& 1228& 614& 307& 922& 461& 1384& 692\\
346& 173& 520& 260& 130& 65& 196& 98& 49& 148\\
74& 37& 112& 56& 28& 14& 7& 22& 11& 34\\
17& 52& 26& 13& 40& 20& 10& 5& 16& 8\\
4& 2& 1& \\

1610&&&&&&&&&\\
805& 2416& 1208& 604& 302& 151& 454& 227& 682& 341\\
1024& 512& 256& 128& 64& 32& 16& 8& 4& 2\\
1& \\

1611&&&&&&&&&\\
4834& 2417& 7252& 3626& 1813& 5440& 2720& 1360& 680& 340\\
170& 85& 256& 128& 64& 32& 16& 8& 4& 2\\
1& \\

1612&&&&&&&&&\\
806& 403& 1210& 605& 1816& 908& 454& 227& 682& 341\\
1024& 512& 256& 128& 64& 32& 16& 8& 4& 2\\
1& \\

1613&&&&&&&&&\\
4840& 2420& 1210& 605& 1816& 908& 454& 227& 682& 341\\
1024& 512& 256& 128& 64& 32& 16& 8& 4& 2\\
1& \\

1614&&&&&&&&&\\
807& 2422& 1211& 3634& 1817& 5452& 2726& 1363& 4090& 2045\\
6136& 3068& 1534& 767& 2302& 1151& 3454& 1727& 5182& 2591\\
7774& 3887& 11662& 5831& 17494& 8747& 26242& 13121& 39364& 19682\\
9841& 29524& 14762& 7381& 22144& 11072& 5536& 2768& 1384& 692\\
346& 173& 520& 260& 130& 65& 196& 98& 49& 148\\
74& 37& 112& 56& 28& 14& 7& 22& 11& 34\\
17& 52& 26& 13& 40& 20& 10& 5& 16& 8\\
4& 2& 1& \\

1615&&&&&&&&&\\
4846& 2423& 7270& 3635& 10906& 5453& 16360& 8180& 4090& 2045\\
6136& 3068& 1534& 767& 2302& 1151& 3454& 1727& 5182& 2591\\
7774& 3887& 11662& 5831& 17494& 8747& 26242& 13121& 39364& 19682\\
9841& 29524& 14762& 7381& 22144& 11072& 5536& 2768& 1384& 692\\
346& 173& 520& 260& 130& 65& 196& 98& 49& 148\\
74& 37& 112& 56& 28& 14& 7& 22& 11& 34\\
17& 52& 26& 13& 40& 20& 10& 5& 16& 8\\
4& 2& 1& \\

1616&&&&&&&&&\\
808& 404& 202& 101& 304& 152& 76& 38& 19& 58\\
29& 88& 44& 22& 11& 34& 17& 52& 26& 13\\
40& 20& 10& 5& 16& 8& 4& 2& 1& \\

1617&&&&&&&&&\\
4852& 2426& 1213& 3640& 1820& 910& 455& 1366& 683& 2050\\
1025& 3076& 1538& 769& 2308& 1154& 577& 1732& 866& 433\\
1300& 650& 325& 976& 488& 244& 122& 61& 184& 92\\
46& 23& 70& 35& 106& 53& 160& 80& 40& 20\\
10& 5& 16& 8& 4& 2& 1& \\

1618&&&&&&&&&\\
809& 2428& 1214& 607& 1822& 911& 2734& 1367& 4102& 2051\\
6154& 3077& 9232& 4616& 2308& 1154& 577& 1732& 866& 433\\
1300& 650& 325& 976& 488& 244& 122& 61& 184& 92\\
46& 23& 70& 35& 106& 53& 160& 80& 40& 20\\
10& 5& 16& 8& 4& 2& 1& \\

1619&&&&&&&&&\\
4858& 2429& 7288& 3644& 1822& 911& 2734& 1367& 4102& 2051\\
6154& 3077& 9232& 4616& 2308& 1154& 577& 1732& 866& 433\\
1300& 650& 325& 976& 488& 244& 122& 61& 184& 92\\
46& 23& 70& 35& 106& 53& 160& 80& 40& 20\\
10& 5& 16& 8& 4& 2& 1& \\

1620&&&&&&&&&\\
810& 405& 1216& 608& 304& 152& 76& 38& 19& 58\\
29& 88& 44& 22& 11& 34& 17& 52& 26& 13\\
40& 20& 10& 5& 16& 8& 4& 2& 1& \\

1621&&&&&&&&&\\
4864& 2432& 1216& 608& 304& 152& 76& 38& 19& 58\\
29& 88& 44& 22& 11& 34& 17& 52& 26& 13\\
40& 20& 10& 5& 16& 8& 4& 2& 1& \\

1622&&&&&&&&&\\
811& 2434& 1217& 3652& 1826& 913& 2740& 1370& 685& 2056\\
1028& 514& 257& 772& 386& 193& 580& 290& 145& 436\\
218& 109& 328& 164& 82& 41& 124& 62& 31& 94\\
47& 142& 71& 214& 107& 322& 161& 484& 242& 121\\
364& 182& 91& 274& 137& 412& 206& 103& 310& 155\\
466& 233& 700& 350& 175& 526& 263& 790& 395& 1186\\
593& 1780& 890& 445& 1336& 668& 334& 167& 502& 251\\
754& 377& 1132& 566& 283& 850& 425& 1276& 638& 319\\
958& 479& 1438& 719& 2158& 1079& 3238& 1619& 4858& 2429\\
7288& 3644& 1822& 911& 2734& 1367& 4102& 2051& 6154& 3077\\
9232& 4616& 2308& 1154& 577& 1732& 866& 433& 1300& 650\\
325& 976& 488& 244& 122& 61& 184& 92& 46& 23\\
70& 35& 106& 53& 160& 80& 40& 20& 10& 5\\
16& 8& 4& 2& 1& \\

1623&&&&&&&&&\\
4870& 2435& 7306& 3653& 10960& 5480& 2740& 1370& 685& 2056\\
1028& 514& 257& 772& 386& 193& 580& 290& 145& 436\\
218& 109& 328& 164& 82& 41& 124& 62& 31& 94\\
47& 142& 71& 214& 107& 322& 161& 484& 242& 121\\
364& 182& 91& 274& 137& 412& 206& 103& 310& 155\\
466& 233& 700& 350& 175& 526& 263& 790& 395& 1186\\
593& 1780& 890& 445& 1336& 668& 334& 167& 502& 251\\
754& 377& 1132& 566& 283& 850& 425& 1276& 638& 319\\
958& 479& 1438& 719& 2158& 1079& 3238& 1619& 4858& 2429\\
7288& 3644& 1822& 911& 2734& 1367& 4102& 2051& 6154& 3077\\
9232& 4616& 2308& 1154& 577& 1732& 866& 433& 1300& 650\\
325& 976& 488& 244& 122& 61& 184& 92& 46& 23\\
70& 35& 106& 53& 160& 80& 40& 20& 10& 5\\
16& 8& 4& 2& 1& \\

1624&&&&&&&&&\\
812& 406& 203& 610& 305& 916& 458& 229& 688& 344\\
172& 86& 43& 130& 65& 196& 98& 49& 148& 74\\
37& 112& 56& 28& 14& 7& 22& 11& 34& 17\\
52& 26& 13& 40& 20& 10& 5& 16& 8& 4\\
2& 1& \\

1625&&&&&&&&&\\
4876& 2438& 1219& 3658& 1829& 5488& 2744& 1372& 686& 343\\
1030& 515& 1546& 773& 2320& 1160& 580& 290& 145& 436\\
218& 109& 328& 164& 82& 41& 124& 62& 31& 94\\
47& 142& 71& 214& 107& 322& 161& 484& 242& 121\\
364& 182& 91& 274& 137& 412& 206& 103& 310& 155\\
466& 233& 700& 350& 175& 526& 263& 790& 395& 1186\\
593& 1780& 890& 445& 1336& 668& 334& 167& 502& 251\\
754& 377& 1132& 566& 283& 850& 425& 1276& 638& 319\\
958& 479& 1438& 719& 2158& 1079& 3238& 1619& 4858& 2429\\
7288& 3644& 1822& 911& 2734& 1367& 4102& 2051& 6154& 3077\\
9232& 4616& 2308& 1154& 577& 1732& 866& 433& 1300& 650\\
325& 976& 488& 244& 122& 61& 184& 92& 46& 23\\
70& 35& 106& 53& 160& 80& 40& 20& 10& 5\\
16& 8& 4& 2& 1& \\

1626&&&&&&&&&\\
813& 2440& 1220& 610& 305& 916& 458& 229& 688& 344\\
172& 86& 43& 130& 65& 196& 98& 49& 148& 74\\
37& 112& 56& 28& 14& 7& 22& 11& 34& 17\\
52& 26& 13& 40& 20& 10& 5& 16& 8& 4\\
2& 1& \\

1627&&&&&&&&&\\
4882& 2441& 7324& 3662& 1831& 5494& 2747& 8242& 4121& 12364\\
6182& 3091& 9274& 4637& 13912& 6956& 3478& 1739& 5218& 2609\\
7828& 3914& 1957& 5872& 2936& 1468& 734& 367& 1102& 551\\
1654& 827& 2482& 1241& 3724& 1862& 931& 2794& 1397& 4192\\
2096& 1048& 524& 262& 131& 394& 197& 592& 296& 148\\
74& 37& 112& 56& 28& 14& 7& 22& 11& 34\\
17& 52& 26& 13& 40& 20& 10& 5& 16& 8\\
4& 2& 1& \\

1628&&&&&&&&&\\
814& 407& 1222& 611& 1834& 917& 2752& 1376& 688& 344\\
172& 86& 43& 130& 65& 196& 98& 49& 148& 74\\
37& 112& 56& 28& 14& 7& 22& 11& 34& 17\\
52& 26& 13& 40& 20& 10& 5& 16& 8& 4\\
2& 1& \\

1629&&&&&&&&&\\
4888& 2444& 1222& 611& 1834& 917& 2752& 1376& 688& 344\\
172& 86& 43& 130& 65& 196& 98& 49& 148& 74\\
37& 112& 56& 28& 14& 7& 22& 11& 34& 17\\
52& 26& 13& 40& 20& 10& 5& 16& 8& 4\\
2& 1& \\

1630&&&&&&&&&\\
815& 2446& 1223& 3670& 1835& 5506& 2753& 8260& 4130& 2065\\
6196& 3098& 1549& 4648& 2324& 1162& 581& 1744& 872& 436\\
218& 109& 328& 164& 82& 41& 124& 62& 31& 94\\
47& 142& 71& 214& 107& 322& 161& 484& 242& 121\\
364& 182& 91& 274& 137& 412& 206& 103& 310& 155\\
466& 233& 700& 350& 175& 526& 263& 790& 395& 1186\\
593& 1780& 890& 445& 1336& 668& 334& 167& 502& 251\\
754& 377& 1132& 566& 283& 850& 425& 1276& 638& 319\\
958& 479& 1438& 719& 2158& 1079& 3238& 1619& 4858& 2429\\
7288& 3644& 1822& 911& 2734& 1367& 4102& 2051& 6154& 3077\\
9232& 4616& 2308& 1154& 577& 1732& 866& 433& 1300& 650\\
325& 976& 488& 244& 122& 61& 184& 92& 46& 23\\
70& 35& 106& 53& 160& 80& 40& 20& 10& 5\\
16& 8& 4& 2& 1& \\

1631&&&&&&&&&\\
4894& 2447& 7342& 3671& 11014& 5507& 16522& 8261& 24784& 12392\\
6196& 3098& 1549& 4648& 2324& 1162& 581& 1744& 872& 436\\
218& 109& 328& 164& 82& 41& 124& 62& 31& 94\\
47& 142& 71& 214& 107& 322& 161& 484& 242& 121\\
364& 182& 91& 274& 137& 412& 206& 103& 310& 155\\
466& 233& 700& 350& 175& 526& 263& 790& 395& 1186\\
593& 1780& 890& 445& 1336& 668& 334& 167& 502& 251\\
754& 377& 1132& 566& 283& 850& 425& 1276& 638& 319\\
958& 479& 1438& 719& 2158& 1079& 3238& 1619& 4858& 2429\\
7288& 3644& 1822& 911& 2734& 1367& 4102& 2051& 6154& 3077\\
9232& 4616& 2308& 1154& 577& 1732& 866& 433& 1300& 650\\
325& 976& 488& 244& 122& 61& 184& 92& 46& 23\\
70& 35& 106& 53& 160& 80& 40& 20& 10& 5\\
16& 8& 4& 2& 1& \\

1632&&&&&&&&&\\
816& 408& 204& 102& 51& 154& 77& 232& 116& 58\\
29& 88& 44& 22& 11& 34& 17& 52& 26& 13\\
40& 20& 10& 5& 16& 8& 4& 2& 1& \\

1633&&&&&&&&&\\
4900& 2450& 1225& 3676& 1838& 919& 2758& 1379& 4138& 2069\\
6208& 3104& 1552& 776& 388& 194& 97& 292& 146& 73\\
220& 110& 55& 166& 83& 250& 125& 376& 188& 94\\
47& 142& 71& 214& 107& 322& 161& 484& 242& 121\\
364& 182& 91& 274& 137& 412& 206& 103& 310& 155\\
466& 233& 700& 350& 175& 526& 263& 790& 395& 1186\\
593& 1780& 890& 445& 1336& 668& 334& 167& 502& 251\\
754& 377& 1132& 566& 283& 850& 425& 1276& 638& 319\\
958& 479& 1438& 719& 2158& 1079& 3238& 1619& 4858& 2429\\
7288& 3644& 1822& 911& 2734& 1367& 4102& 2051& 6154& 3077\\
9232& 4616& 2308& 1154& 577& 1732& 866& 433& 1300& 650\\
325& 976& 488& 244& 122& 61& 184& 92& 46& 23\\
70& 35& 106& 53& 160& 80& 40& 20& 10& 5\\
16& 8& 4& 2& 1& \\

1634&&&&&&&&&\\
817& 2452& 1226& 613& 1840& 920& 460& 230& 115& 346\\
173& 520& 260& 130& 65& 196& 98& 49& 148& 74\\
37& 112& 56& 28& 14& 7& 22& 11& 34& 17\\
52& 26& 13& 40& 20& 10& 5& 16& 8& 4\\
2& 1& \\

1635&&&&&&&&&\\
4906& 2453& 7360& 3680& 1840& 920& 460& 230& 115& 346\\
173& 520& 260& 130& 65& 196& 98& 49& 148& 74\\
37& 112& 56& 28& 14& 7& 22& 11& 34& 17\\
52& 26& 13& 40& 20& 10& 5& 16& 8& 4\\
2& 1& \\

1636&&&&&&&&&\\
818& 409& 1228& 614& 307& 922& 461& 1384& 692& 346\\
173& 520& 260& 130& 65& 196& 98& 49& 148& 74\\
37& 112& 56& 28& 14& 7& 22& 11& 34& 17\\
52& 26& 13& 40& 20& 10& 5& 16& 8& 4\\
2& 1& \\

1637&&&&&&&&&\\
4912& 2456& 1228& 614& 307& 922& 461& 1384& 692& 346\\
173& 520& 260& 130& 65& 196& 98& 49& 148& 74\\
37& 112& 56& 28& 14& 7& 22& 11& 34& 17\\
52& 26& 13& 40& 20& 10& 5& 16& 8& 4\\
2& 1& \\

1638&&&&&&&&&\\
819& 2458& 1229& 3688& 1844& 922& 461& 1384& 692& 346\\
173& 520& 260& 130& 65& 196& 98& 49& 148& 74\\
37& 112& 56& 28& 14& 7& 22& 11& 34& 17\\
52& 26& 13& 40& 20& 10& 5& 16& 8& 4\\
2& 1& \\

1639&&&&&&&&&\\
4918& 2459& 7378& 3689& 11068& 5534& 2767& 8302& 4151& 12454\\
6227& 18682& 9341& 28024& 14012& 7006& 3503& 10510& 5255& 15766\\
7883& 23650& 11825& 35476& 17738& 8869& 26608& 13304& 6652& 3326\\
1663& 4990& 2495& 7486& 3743& 11230& 5615& 16846& 8423& 25270\\
12635& 37906& 18953& 56860& 28430& 14215& 42646& 21323& 63970& 31985\\
95956& 47978& 23989& 71968& 35984& 17992& 8996& 4498& 2249& 6748\\
3374& 1687& 5062& 2531& 7594& 3797& 11392& 5696& 2848& 1424\\
712& 356& 178& 89& 268& 134& 67& 202& 101& 304\\
152& 76& 38& 19& 58& 29& 88& 44& 22& 11\\
34& 17& 52& 26& 13& 40& 20& 10& 5& 16\\
8& 4& 2& 1& \\

1640&&&&&&&&&\\
820& 410& 205& 616& 308& 154& 77& 232& 116& 58\\
29& 88& 44& 22& 11& 34& 17& 52& 26& 13\\
40& 20& 10& 5& 16& 8& 4& 2& 1& \\

1641&&&&&&&&&\\
4924& 2462& 1231& 3694& 1847& 5542& 2771& 8314& 4157& 12472\\
6236& 3118& 1559& 4678& 2339& 7018& 3509& 10528& 5264& 2632\\
1316& 658& 329& 988& 494& 247& 742& 371& 1114& 557\\
1672& 836& 418& 209& 628& 314& 157& 472& 236& 118\\
59& 178& 89& 268& 134& 67& 202& 101& 304& 152\\
76& 38& 19& 58& 29& 88& 44& 22& 11& 34\\
17& 52& 26& 13& 40& 20& 10& 5& 16& 8\\
4& 2& 1& \\

1642&&&&&&&&&\\
821& 2464& 1232& 616& 308& 154& 77& 232& 116& 58\\
29& 88& 44& 22& 11& 34& 17& 52& 26& 13\\
40& 20& 10& 5& 16& 8& 4& 2& 1& \\

1643&&&&&&&&&\\
4930& 2465& 7396& 3698& 1849& 5548& 2774& 1387& 4162& 2081\\
6244& 3122& 1561& 4684& 2342& 1171& 3514& 1757& 5272& 2636\\
1318& 659& 1978& 989& 2968& 1484& 742& 371& 1114& 557\\
1672& 836& 418& 209& 628& 314& 157& 472& 236& 118\\
59& 178& 89& 268& 134& 67& 202& 101& 304& 152\\
76& 38& 19& 58& 29& 88& 44& 22& 11& 34\\
17& 52& 26& 13& 40& 20& 10& 5& 16& 8\\
4& 2& 1& \\

1644&&&&&&&&&\\
822& 411& 1234& 617& 1852& 926& 463& 1390& 695& 2086\\
1043& 3130& 1565& 4696& 2348& 1174& 587& 1762& 881& 2644\\
1322& 661& 1984& 992& 496& 248& 124& 62& 31& 94\\
47& 142& 71& 214& 107& 322& 161& 484& 242& 121\\
364& 182& 91& 274& 137& 412& 206& 103& 310& 155\\
466& 233& 700& 350& 175& 526& 263& 790& 395& 1186\\
593& 1780& 890& 445& 1336& 668& 334& 167& 502& 251\\
754& 377& 1132& 566& 283& 850& 425& 1276& 638& 319\\
958& 479& 1438& 719& 2158& 1079& 3238& 1619& 4858& 2429\\
7288& 3644& 1822& 911& 2734& 1367& 4102& 2051& 6154& 3077\\
9232& 4616& 2308& 1154& 577& 1732& 866& 433& 1300& 650\\
325& 976& 488& 244& 122& 61& 184& 92& 46& 23\\
70& 35& 106& 53& 160& 80& 40& 20& 10& 5\\
16& 8& 4& 2& 1& \\

1645&&&&&&&&&\\
4936& 2468& 1234& 617& 1852& 926& 463& 1390& 695& 2086\\
1043& 3130& 1565& 4696& 2348& 1174& 587& 1762& 881& 2644\\
1322& 661& 1984& 992& 496& 248& 124& 62& 31& 94\\
47& 142& 71& 214& 107& 322& 161& 484& 242& 121\\
364& 182& 91& 274& 137& 412& 206& 103& 310& 155\\
466& 233& 700& 350& 175& 526& 263& 790& 395& 1186\\
593& 1780& 890& 445& 1336& 668& 334& 167& 502& 251\\
754& 377& 1132& 566& 283& 850& 425& 1276& 638& 319\\
958& 479& 1438& 719& 2158& 1079& 3238& 1619& 4858& 2429\\
7288& 3644& 1822& 911& 2734& 1367& 4102& 2051& 6154& 3077\\
9232& 4616& 2308& 1154& 577& 1732& 866& 433& 1300& 650\\
325& 976& 488& 244& 122& 61& 184& 92& 46& 23\\
70& 35& 106& 53& 160& 80& 40& 20& 10& 5\\
16& 8& 4& 2& 1& \\

1646&&&&&&&&&\\
823& 2470& 1235& 3706& 1853& 5560& 2780& 1390& 695& 2086\\
1043& 3130& 1565& 4696& 2348& 1174& 587& 1762& 881& 2644\\
1322& 661& 1984& 992& 496& 248& 124& 62& 31& 94\\
47& 142& 71& 214& 107& 322& 161& 484& 242& 121\\
364& 182& 91& 274& 137& 412& 206& 103& 310& 155\\
466& 233& 700& 350& 175& 526& 263& 790& 395& 1186\\
593& 1780& 890& 445& 1336& 668& 334& 167& 502& 251\\
754& 377& 1132& 566& 283& 850& 425& 1276& 638& 319\\
958& 479& 1438& 719& 2158& 1079& 3238& 1619& 4858& 2429\\
7288& 3644& 1822& 911& 2734& 1367& 4102& 2051& 6154& 3077\\
9232& 4616& 2308& 1154& 577& 1732& 866& 433& 1300& 650\\
325& 976& 488& 244& 122& 61& 184& 92& 46& 23\\
70& 35& 106& 53& 160& 80& 40& 20& 10& 5\\
16& 8& 4& 2& 1& \\

1647&&&&&&&&&\\
4942& 2471& 7414& 3707& 11122& 5561& 16684& 8342& 4171& 12514\\
6257& 18772& 9386& 4693& 14080& 7040& 3520& 1760& 880& 440\\
220& 110& 55& 166& 83& 250& 125& 376& 188& 94\\
47& 142& 71& 214& 107& 322& 161& 484& 242& 121\\
364& 182& 91& 274& 137& 412& 206& 103& 310& 155\\
466& 233& 700& 350& 175& 526& 263& 790& 395& 1186\\
593& 1780& 890& 445& 1336& 668& 334& 167& 502& 251\\
754& 377& 1132& 566& 283& 850& 425& 1276& 638& 319\\
958& 479& 1438& 719& 2158& 1079& 3238& 1619& 4858& 2429\\
7288& 3644& 1822& 911& 2734& 1367& 4102& 2051& 6154& 3077\\
9232& 4616& 2308& 1154& 577& 1732& 866& 433& 1300& 650\\
325& 976& 488& 244& 122& 61& 184& 92& 46& 23\\
70& 35& 106& 53& 160& 80& 40& 20& 10& 5\\
16& 8& 4& 2& 1& \\

1648&&&&&&&&&\\
824& 412& 206& 103& 310& 155& 466& 233& 700& 350\\
175& 526& 263& 790& 395& 1186& 593& 1780& 890& 445\\
1336& 668& 334& 167& 502& 251& 754& 377& 1132& 566\\
283& 850& 425& 1276& 638& 319& 958& 479& 1438& 719\\
2158& 1079& 3238& 1619& 4858& 2429& 7288& 3644& 1822& 911\\
2734& 1367& 4102& 2051& 6154& 3077& 9232& 4616& 2308& 1154\\
577& 1732& 866& 433& 1300& 650& 325& 976& 488& 244\\
122& 61& 184& 92& 46& 23& 70& 35& 106& 53\\
160& 80& 40& 20& 10& 5& 16& 8& 4& 2\\
1& \\

1649&&&&&&&&&\\
4948& 2474& 1237& 3712& 1856& 928& 464& 232& 116& 58\\
29& 88& 44& 22& 11& 34& 17& 52& 26& 13\\
40& 20& 10& 5& 16& 8& 4& 2& 1& \\

1650&&&&&&&&&\\
825& 2476& 1238& 619& 1858& 929& 2788& 1394& 697& 2092\\
1046& 523& 1570& 785& 2356& 1178& 589& 1768& 884& 442\\
221& 664& 332& 166& 83& 250& 125& 376& 188& 94\\
47& 142& 71& 214& 107& 322& 161& 484& 242& 121\\
364& 182& 91& 274& 137& 412& 206& 103& 310& 155\\
466& 233& 700& 350& 175& 526& 263& 790& 395& 1186\\
593& 1780& 890& 445& 1336& 668& 334& 167& 502& 251\\
754& 377& 1132& 566& 283& 850& 425& 1276& 638& 319\\
958& 479& 1438& 719& 2158& 1079& 3238& 1619& 4858& 2429\\
7288& 3644& 1822& 911& 2734& 1367& 4102& 2051& 6154& 3077\\
9232& 4616& 2308& 1154& 577& 1732& 866& 433& 1300& 650\\
325& 976& 488& 244& 122& 61& 184& 92& 46& 23\\
70& 35& 106& 53& 160& 80& 40& 20& 10& 5\\
16& 8& 4& 2& 1& \\

1651&&&&&&&&&\\
4954& 2477& 7432& 3716& 1858& 929& 2788& 1394& 697& 2092\\
1046& 523& 1570& 785& 2356& 1178& 589& 1768& 884& 442\\
221& 664& 332& 166& 83& 250& 125& 376& 188& 94\\
47& 142& 71& 214& 107& 322& 161& 484& 242& 121\\
364& 182& 91& 274& 137& 412& 206& 103& 310& 155\\
466& 233& 700& 350& 175& 526& 263& 790& 395& 1186\\
593& 1780& 890& 445& 1336& 668& 334& 167& 502& 251\\
754& 377& 1132& 566& 283& 850& 425& 1276& 638& 319\\
958& 479& 1438& 719& 2158& 1079& 3238& 1619& 4858& 2429\\
7288& 3644& 1822& 911& 2734& 1367& 4102& 2051& 6154& 3077\\
9232& 4616& 2308& 1154& 577& 1732& 866& 433& 1300& 650\\
325& 976& 488& 244& 122& 61& 184& 92& 46& 23\\
70& 35& 106& 53& 160& 80& 40& 20& 10& 5\\
16& 8& 4& 2& 1& \\

1652&&&&&&&&&\\
826& 413& 1240& 620& 310& 155& 466& 233& 700& 350\\
175& 526& 263& 790& 395& 1186& 593& 1780& 890& 445\\
1336& 668& 334& 167& 502& 251& 754& 377& 1132& 566\\
283& 850& 425& 1276& 638& 319& 958& 479& 1438& 719\\
2158& 1079& 3238& 1619& 4858& 2429& 7288& 3644& 1822& 911\\
2734& 1367& 4102& 2051& 6154& 3077& 9232& 4616& 2308& 1154\\
577& 1732& 866& 433& 1300& 650& 325& 976& 488& 244\\
122& 61& 184& 92& 46& 23& 70& 35& 106& 53\\
160& 80& 40& 20& 10& 5& 16& 8& 4& 2\\
1& \\

1653&&&&&&&&&\\
4960& 2480& 1240& 620& 310& 155& 466& 233& 700& 350\\
175& 526& 263& 790& 395& 1186& 593& 1780& 890& 445\\
1336& 668& 334& 167& 502& 251& 754& 377& 1132& 566\\
283& 850& 425& 1276& 638& 319& 958& 479& 1438& 719\\
2158& 1079& 3238& 1619& 4858& 2429& 7288& 3644& 1822& 911\\
2734& 1367& 4102& 2051& 6154& 3077& 9232& 4616& 2308& 1154\\
577& 1732& 866& 433& 1300& 650& 325& 976& 488& 244\\
122& 61& 184& 92& 46& 23& 70& 35& 106& 53\\
160& 80& 40& 20& 10& 5& 16& 8& 4& 2\\
1& \\

1654&&&&&&&&&\\
827& 2482& 1241& 3724& 1862& 931& 2794& 1397& 4192& 2096\\
1048& 524& 262& 131& 394& 197& 592& 296& 148& 74\\
37& 112& 56& 28& 14& 7& 22& 11& 34& 17\\
52& 26& 13& 40& 20& 10& 5& 16& 8& 4\\
2& 1& \\

1655&&&&&&&&&\\
4966& 2483& 7450& 3725& 11176& 5588& 2794& 1397& 4192& 2096\\
1048& 524& 262& 131& 394& 197& 592& 296& 148& 74\\
37& 112& 56& 28& 14& 7& 22& 11& 34& 17\\
52& 26& 13& 40& 20& 10& 5& 16& 8& 4\\
2& 1& \\

1656&&&&&&&&&\\
828& 414& 207& 622& 311& 934& 467& 1402& 701& 2104\\
1052& 526& 263& 790& 395& 1186& 593& 1780& 890& 445\\
1336& 668& 334& 167& 502& 251& 754& 377& 1132& 566\\
283& 850& 425& 1276& 638& 319& 958& 479& 1438& 719\\
2158& 1079& 3238& 1619& 4858& 2429& 7288& 3644& 1822& 911\\
2734& 1367& 4102& 2051& 6154& 3077& 9232& 4616& 2308& 1154\\
577& 1732& 866& 433& 1300& 650& 325& 976& 488& 244\\
122& 61& 184& 92& 46& 23& 70& 35& 106& 53\\
160& 80& 40& 20& 10& 5& 16& 8& 4& 2\\
1& \\

1657&&&&&&&&&\\
4972& 2486& 1243& 3730& 1865& 5596& 2798& 1399& 4198& 2099\\
6298& 3149& 9448& 4724& 2362& 1181& 3544& 1772& 886& 443\\
1330& 665& 1996& 998& 499& 1498& 749& 2248& 1124& 562\\
281& 844& 422& 211& 634& 317& 952& 476& 238& 119\\
358& 179& 538& 269& 808& 404& 202& 101& 304& 152\\
76& 38& 19& 58& 29& 88& 44& 22& 11& 34\\
17& 52& 26& 13& 40& 20& 10& 5& 16& 8\\
4& 2& 1& \\

1658&&&&&&&&&\\
829& 2488& 1244& 622& 311& 934& 467& 1402& 701& 2104\\
1052& 526& 263& 790& 395& 1186& 593& 1780& 890& 445\\
1336& 668& 334& 167& 502& 251& 754& 377& 1132& 566\\
283& 850& 425& 1276& 638& 319& 958& 479& 1438& 719\\
2158& 1079& 3238& 1619& 4858& 2429& 7288& 3644& 1822& 911\\
2734& 1367& 4102& 2051& 6154& 3077& 9232& 4616& 2308& 1154\\
577& 1732& 866& 433& 1300& 650& 325& 976& 488& 244\\
122& 61& 184& 92& 46& 23& 70& 35& 106& 53\\
160& 80& 40& 20& 10& 5& 16& 8& 4& 2\\
1& \\

1659&&&&&&&&&\\
4978& 2489& 7468& 3734& 1867& 5602& 2801& 8404& 4202& 2101\\
6304& 3152& 1576& 788& 394& 197& 592& 296& 148& 74\\
37& 112& 56& 28& 14& 7& 22& 11& 34& 17\\
52& 26& 13& 40& 20& 10& 5& 16& 8& 4\\
2& 1& \\

1660&&&&&&&&&\\
830& 415& 1246& 623& 1870& 935& 2806& 1403& 4210& 2105\\
6316& 3158& 1579& 4738& 2369& 7108& 3554& 1777& 5332& 2666\\
1333& 4000& 2000& 1000& 500& 250& 125& 376& 188& 94\\
47& 142& 71& 214& 107& 322& 161& 484& 242& 121\\
364& 182& 91& 274& 137& 412& 206& 103& 310& 155\\
466& 233& 700& 350& 175& 526& 263& 790& 395& 1186\\
593& 1780& 890& 445& 1336& 668& 334& 167& 502& 251\\
754& 377& 1132& 566& 283& 850& 425& 1276& 638& 319\\
958& 479& 1438& 719& 2158& 1079& 3238& 1619& 4858& 2429\\
7288& 3644& 1822& 911& 2734& 1367& 4102& 2051& 6154& 3077\\
9232& 4616& 2308& 1154& 577& 1732& 866& 433& 1300& 650\\
325& 976& 488& 244& 122& 61& 184& 92& 46& 23\\
70& 35& 106& 53& 160& 80& 40& 20& 10& 5\\
16& 8& 4& 2& 1& \\

1661&&&&&&&&&\\
4984& 2492& 1246& 623& 1870& 935& 2806& 1403& 4210& 2105\\
6316& 3158& 1579& 4738& 2369& 7108& 3554& 1777& 5332& 2666\\
1333& 4000& 2000& 1000& 500& 250& 125& 376& 188& 94\\
47& 142& 71& 214& 107& 322& 161& 484& 242& 121\\
364& 182& 91& 274& 137& 412& 206& 103& 310& 155\\
466& 233& 700& 350& 175& 526& 263& 790& 395& 1186\\
593& 1780& 890& 445& 1336& 668& 334& 167& 502& 251\\
754& 377& 1132& 566& 283& 850& 425& 1276& 638& 319\\
958& 479& 1438& 719& 2158& 1079& 3238& 1619& 4858& 2429\\
7288& 3644& 1822& 911& 2734& 1367& 4102& 2051& 6154& 3077\\
9232& 4616& 2308& 1154& 577& 1732& 866& 433& 1300& 650\\
325& 976& 488& 244& 122& 61& 184& 92& 46& 23\\
70& 35& 106& 53& 160& 80& 40& 20& 10& 5\\
16& 8& 4& 2& 1& \\

1662&&&&&&&&&\\
831& 2494& 1247& 3742& 1871& 5614& 2807& 8422& 4211& 12634\\
6317& 18952& 9476& 4738& 2369& 7108& 3554& 1777& 5332& 2666\\
1333& 4000& 2000& 1000& 500& 250& 125& 376& 188& 94\\
47& 142& 71& 214& 107& 322& 161& 484& 242& 121\\
364& 182& 91& 274& 137& 412& 206& 103& 310& 155\\
466& 233& 700& 350& 175& 526& 263& 790& 395& 1186\\
593& 1780& 890& 445& 1336& 668& 334& 167& 502& 251\\
754& 377& 1132& 566& 283& 850& 425& 1276& 638& 319\\
958& 479& 1438& 719& 2158& 1079& 3238& 1619& 4858& 2429\\
7288& 3644& 1822& 911& 2734& 1367& 4102& 2051& 6154& 3077\\
9232& 4616& 2308& 1154& 577& 1732& 866& 433& 1300& 650\\
325& 976& 488& 244& 122& 61& 184& 92& 46& 23\\
70& 35& 106& 53& 160& 80& 40& 20& 10& 5\\
16& 8& 4& 2& 1& \\

1663&&&&&&&&&\\
4990& 2495& 7486& 3743& 11230& 5615& 16846& 8423& 25270& 12635\\
37906& 18953& 56860& 28430& 14215& 42646& 21323& 63970& 31985& 95956\\
47978& 23989& 71968& 35984& 17992& 8996& 4498& 2249& 6748& 3374\\
1687& 5062& 2531& 7594& 3797& 11392& 5696& 2848& 1424& 712\\
356& 178& 89& 268& 134& 67& 202& 101& 304& 152\\
76& 38& 19& 58& 29& 88& 44& 22& 11& 34\\
17& 52& 26& 13& 40& 20& 10& 5& 16& 8\\
4& 2& 1& \\

1664&&&&&&&&&\\
832& 416& 208& 104& 52& 26& 13& 40& 20& 10\\
5& 16& 8& 4& 2& 1& \\

1665&&&&&&&&&\\
4996& 2498& 1249& 3748& 1874& 937& 2812& 1406& 703& 2110\\
1055& 3166& 1583& 4750& 2375& 7126& 3563& 10690& 5345& 16036\\
8018& 4009& 12028& 6014& 3007& 9022& 4511& 13534& 6767& 20302\\
10151& 30454& 15227& 45682& 22841& 68524& 34262& 17131& 51394& 25697\\
77092& 38546& 19273& 57820& 28910& 14455& 43366& 21683& 65050& 32525\\
97576& 48788& 24394& 12197& 36592& 18296& 9148& 4574& 2287& 6862\\
3431& 10294& 5147& 15442& 7721& 23164& 11582& 5791& 17374& 8687\\
26062& 13031& 39094& 19547& 58642& 29321& 87964& 43982& 21991& 65974\\
32987& 98962& 49481& 148444& 74222& 37111& 111334& 55667& 167002& 83501\\
250504& 125252& 62626& 31313& 93940& 46970& 23485& 70456& 35228& 17614\\
8807& 26422& 13211& 39634& 19817& 59452& 29726& 14863& 44590& 22295\\
66886& 33443& 100330& 50165& 150496& 75248& 37624& 18812& 9406& 4703\\
14110& 7055& 21166& 10583& 31750& 15875& 47626& 23813& 71440& 35720\\
17860& 8930& 4465& 13396& 6698& 3349& 10048& 5024& 2512& 1256\\
628& 314& 157& 472& 236& 118& 59& 178& 89& 268\\
134& 67& 202& 101& 304& 152& 76& 38& 19& 58\\
29& 88& 44& 22& 11& 34& 17& 52& 26& 13\\
40& 20& 10& 5& 16& 8& 4& 2& 1& \\

1666&&&&&&&&&\\
833& 2500& 1250& 625& 1876& 938& 469& 1408& 704& 352\\
176& 88& 44& 22& 11& 34& 17& 52& 26& 13\\
40& 20& 10& 5& 16& 8& 4& 2& 1& \\

1667&&&&&&&&&\\
5002& 2501& 7504& 3752& 1876& 938& 469& 1408& 704& 352\\
176& 88& 44& 22& 11& 34& 17& 52& 26& 13\\
40& 20& 10& 5& 16& 8& 4& 2& 1& \\

1668&&&&&&&&&\\
834& 417& 1252& 626& 313& 940& 470& 235& 706& 353\\
1060& 530& 265& 796& 398& 199& 598& 299& 898& 449\\
1348& 674& 337& 1012& 506& 253& 760& 380& 190& 95\\
286& 143& 430& 215& 646& 323& 970& 485& 1456& 728\\
364& 182& 91& 274& 137& 412& 206& 103& 310& 155\\
466& 233& 700& 350& 175& 526& 263& 790& 395& 1186\\
593& 1780& 890& 445& 1336& 668& 334& 167& 502& 251\\
754& 377& 1132& 566& 283& 850& 425& 1276& 638& 319\\
958& 479& 1438& 719& 2158& 1079& 3238& 1619& 4858& 2429\\
7288& 3644& 1822& 911& 2734& 1367& 4102& 2051& 6154& 3077\\
9232& 4616& 2308& 1154& 577& 1732& 866& 433& 1300& 650\\
325& 976& 488& 244& 122& 61& 184& 92& 46& 23\\
70& 35& 106& 53& 160& 80& 40& 20& 10& 5\\
16& 8& 4& 2& 1& \\

1669&&&&&&&&&\\
5008& 2504& 1252& 626& 313& 940& 470& 235& 706& 353\\
1060& 530& 265& 796& 398& 199& 598& 299& 898& 449\\
1348& 674& 337& 1012& 506& 253& 760& 380& 190& 95\\
286& 143& 430& 215& 646& 323& 970& 485& 1456& 728\\
364& 182& 91& 274& 137& 412& 206& 103& 310& 155\\
466& 233& 700& 350& 175& 526& 263& 790& 395& 1186\\
593& 1780& 890& 445& 1336& 668& 334& 167& 502& 251\\
754& 377& 1132& 566& 283& 850& 425& 1276& 638& 319\\
958& 479& 1438& 719& 2158& 1079& 3238& 1619& 4858& 2429\\
7288& 3644& 1822& 911& 2734& 1367& 4102& 2051& 6154& 3077\\
9232& 4616& 2308& 1154& 577& 1732& 866& 433& 1300& 650\\
325& 976& 488& 244& 122& 61& 184& 92& 46& 23\\
70& 35& 106& 53& 160& 80& 40& 20& 10& 5\\
16& 8& 4& 2& 1& \\

1670&&&&&&&&&\\
835& 2506& 1253& 3760& 1880& 940& 470& 235& 706& 353\\
1060& 530& 265& 796& 398& 199& 598& 299& 898& 449\\
1348& 674& 337& 1012& 506& 253& 760& 380& 190& 95\\
286& 143& 430& 215& 646& 323& 970& 485& 1456& 728\\
364& 182& 91& 274& 137& 412& 206& 103& 310& 155\\
466& 233& 700& 350& 175& 526& 263& 790& 395& 1186\\
593& 1780& 890& 445& 1336& 668& 334& 167& 502& 251\\
754& 377& 1132& 566& 283& 850& 425& 1276& 638& 319\\
958& 479& 1438& 719& 2158& 1079& 3238& 1619& 4858& 2429\\
7288& 3644& 1822& 911& 2734& 1367& 4102& 2051& 6154& 3077\\
9232& 4616& 2308& 1154& 577& 1732& 866& 433& 1300& 650\\
325& 976& 488& 244& 122& 61& 184& 92& 46& 23\\
70& 35& 106& 53& 160& 80& 40& 20& 10& 5\\
16& 8& 4& 2& 1& \\

1671&&&&&&&&&\\
5014& 2507& 7522& 3761& 11284& 5642& 2821& 8464& 4232& 2116\\
1058& 529& 1588& 794& 397& 1192& 596& 298& 149& 448\\
224& 112& 56& 28& 14& 7& 22& 11& 34& 17\\
52& 26& 13& 40& 20& 10& 5& 16& 8& 4\\
2& 1& \\

1672&&&&&&&&&\\
836& 418& 209& 628& 314& 157& 472& 236& 118& 59\\
178& 89& 268& 134& 67& 202& 101& 304& 152& 76\\
38& 19& 58& 29& 88& 44& 22& 11& 34& 17\\
52& 26& 13& 40& 20& 10& 5& 16& 8& 4\\
2& 1& \\

1673&&&&&&&&&\\
5020& 2510& 1255& 3766& 1883& 5650& 2825& 8476& 4238& 2119\\
6358& 3179& 9538& 4769& 14308& 7154& 3577& 10732& 5366& 2683\\
8050& 4025& 12076& 6038& 3019& 9058& 4529& 13588& 6794& 3397\\
10192& 5096& 2548& 1274& 637& 1912& 956& 478& 239& 718\\
359& 1078& 539& 1618& 809& 2428& 1214& 607& 1822& 911\\
2734& 1367& 4102& 2051& 6154& 3077& 9232& 4616& 2308& 1154\\
577& 1732& 866& 433& 1300& 650& 325& 976& 488& 244\\
122& 61& 184& 92& 46& 23& 70& 35& 106& 53\\
160& 80& 40& 20& 10& 5& 16& 8& 4& 2\\
1& \\

1674&&&&&&&&&\\
837& 2512& 1256& 628& 314& 157& 472& 236& 118& 59\\
178& 89& 268& 134& 67& 202& 101& 304& 152& 76\\
38& 19& 58& 29& 88& 44& 22& 11& 34& 17\\
52& 26& 13& 40& 20& 10& 5& 16& 8& 4\\
2& 1& \\

1675&&&&&&&&&\\
5026& 2513& 7540& 3770& 1885& 5656& 2828& 1414& 707& 2122\\
1061& 3184& 1592& 796& 398& 199& 598& 299& 898& 449\\
1348& 674& 337& 1012& 506& 253& 760& 380& 190& 95\\
286& 143& 430& 215& 646& 323& 970& 485& 1456& 728\\
364& 182& 91& 274& 137& 412& 206& 103& 310& 155\\
466& 233& 700& 350& 175& 526& 263& 790& 395& 1186\\
593& 1780& 890& 445& 1336& 668& 334& 167& 502& 251\\
754& 377& 1132& 566& 283& 850& 425& 1276& 638& 319\\
958& 479& 1438& 719& 2158& 1079& 3238& 1619& 4858& 2429\\
7288& 3644& 1822& 911& 2734& 1367& 4102& 2051& 6154& 3077\\
9232& 4616& 2308& 1154& 577& 1732& 866& 433& 1300& 650\\
325& 976& 488& 244& 122& 61& 184& 92& 46& 23\\
70& 35& 106& 53& 160& 80& 40& 20& 10& 5\\
16& 8& 4& 2& 1& \\

1676&&&&&&&&&\\
838& 419& 1258& 629& 1888& 944& 472& 236& 118& 59\\
178& 89& 268& 134& 67& 202& 101& 304& 152& 76\\
38& 19& 58& 29& 88& 44& 22& 11& 34& 17\\
52& 26& 13& 40& 20& 10& 5& 16& 8& 4\\
2& 1& \\

1677&&&&&&&&&\\
5032& 2516& 1258& 629& 1888& 944& 472& 236& 118& 59\\
178& 89& 268& 134& 67& 202& 101& 304& 152& 76\\
38& 19& 58& 29& 88& 44& 22& 11& 34& 17\\
52& 26& 13& 40& 20& 10& 5& 16& 8& 4\\
2& 1& \\

1678&&&&&&&&&\\
839& 2518& 1259& 3778& 1889& 5668& 2834& 1417& 4252& 2126\\
1063& 3190& 1595& 4786& 2393& 7180& 3590& 1795& 5386& 2693\\
8080& 4040& 2020& 1010& 505& 1516& 758& 379& 1138& 569\\
1708& 854& 427& 1282& 641& 1924& 962& 481& 1444& 722\\
361& 1084& 542& 271& 814& 407& 1222& 611& 1834& 917\\
2752& 1376& 688& 344& 172& 86& 43& 130& 65& 196\\
98& 49& 148& 74& 37& 112& 56& 28& 14& 7\\
22& 11& 34& 17& 52& 26& 13& 40& 20& 10\\
5& 16& 8& 4& 2& 1& \\

1679&&&&&&&&&\\
5038& 2519& 7558& 3779& 11338& 5669& 17008& 8504& 4252& 2126\\
1063& 3190& 1595& 4786& 2393& 7180& 3590& 1795& 5386& 2693\\
8080& 4040& 2020& 1010& 505& 1516& 758& 379& 1138& 569\\
1708& 854& 427& 1282& 641& 1924& 962& 481& 1444& 722\\
361& 1084& 542& 271& 814& 407& 1222& 611& 1834& 917\\
2752& 1376& 688& 344& 172& 86& 43& 130& 65& 196\\
98& 49& 148& 74& 37& 112& 56& 28& 14& 7\\
22& 11& 34& 17& 52& 26& 13& 40& 20& 10\\
5& 16& 8& 4& 2& 1& \\

1680&&&&&&&&&\\
840& 420& 210& 105& 316& 158& 79& 238& 119& 358\\
179& 538& 269& 808& 404& 202& 101& 304& 152& 76\\
38& 19& 58& 29& 88& 44& 22& 11& 34& 17\\
52& 26& 13& 40& 20& 10& 5& 16& 8& 4\\
2& 1& \\

1681&&&&&&&&&\\
5044& 2522& 1261& 3784& 1892& 946& 473& 1420& 710& 355\\
1066& 533& 1600& 800& 400& 200& 100& 50& 25& 76\\
38& 19& 58& 29& 88& 44& 22& 11& 34& 17\\
52& 26& 13& 40& 20& 10& 5& 16& 8& 4\\
2& 1& \\

1682&&&&&&&&&\\
841& 2524& 1262& 631& 1894& 947& 2842& 1421& 4264& 2132\\
1066& 533& 1600& 800& 400& 200& 100& 50& 25& 76\\
38& 19& 58& 29& 88& 44& 22& 11& 34& 17\\
52& 26& 13& 40& 20& 10& 5& 16& 8& 4\\
2& 1& \\

1683&&&&&&&&&\\
5050& 2525& 7576& 3788& 1894& 947& 2842& 1421& 4264& 2132\\
1066& 533& 1600& 800& 400& 200& 100& 50& 25& 76\\
38& 19& 58& 29& 88& 44& 22& 11& 34& 17\\
52& 26& 13& 40& 20& 10& 5& 16& 8& 4\\
2& 1& \\

1684&&&&&&&&&\\
842& 421& 1264& 632& 316& 158& 79& 238& 119& 358\\
179& 538& 269& 808& 404& 202& 101& 304& 152& 76\\
38& 19& 58& 29& 88& 44& 22& 11& 34& 17\\
52& 26& 13& 40& 20& 10& 5& 16& 8& 4\\
2& 1& \\

1685&&&&&&&&&\\
5056& 2528& 1264& 632& 316& 158& 79& 238& 119& 358\\
179& 538& 269& 808& 404& 202& 101& 304& 152& 76\\
38& 19& 58& 29& 88& 44& 22& 11& 34& 17\\
52& 26& 13& 40& 20& 10& 5& 16& 8& 4\\
2& 1& \\

1686&&&&&&&&&\\
843& 2530& 1265& 3796& 1898& 949& 2848& 1424& 712& 356\\
178& 89& 268& 134& 67& 202& 101& 304& 152& 76\\
38& 19& 58& 29& 88& 44& 22& 11& 34& 17\\
52& 26& 13& 40& 20& 10& 5& 16& 8& 4\\
2& 1& \\

1687&&&&&&&&&\\
5062& 2531& 7594& 3797& 11392& 5696& 2848& 1424& 712& 356\\
178& 89& 268& 134& 67& 202& 101& 304& 152& 76\\
38& 19& 58& 29& 88& 44& 22& 11& 34& 17\\
52& 26& 13& 40& 20& 10& 5& 16& 8& 4\\
2& 1& \\

1688&&&&&&&&&\\
844& 422& 211& 634& 317& 952& 476& 238& 119& 358\\
179& 538& 269& 808& 404& 202& 101& 304& 152& 76\\
38& 19& 58& 29& 88& 44& 22& 11& 34& 17\\
52& 26& 13& 40& 20& 10& 5& 16& 8& 4\\
2& 1& \\

1689&&&&&&&&&\\
5068& 2534& 1267& 3802& 1901& 5704& 2852& 1426& 713& 2140\\
1070& 535& 1606& 803& 2410& 1205& 3616& 1808& 904& 452\\
226& 113& 340& 170& 85& 256& 128& 64& 32& 16\\
8& 4& 2& 1& \\

1690&&&&&&&&&\\
845& 2536& 1268& 634& 317& 952& 476& 238& 119& 358\\
179& 538& 269& 808& 404& 202& 101& 304& 152& 76\\
38& 19& 58& 29& 88& 44& 22& 11& 34& 17\\
52& 26& 13& 40& 20& 10& 5& 16& 8& 4\\
2& 1& \\

1691&&&&&&&&&\\
5074& 2537& 7612& 3806& 1903& 5710& 2855& 8566& 4283& 12850\\
6425& 19276& 9638& 4819& 14458& 7229& 21688& 10844& 5422& 2711\\
8134& 4067& 12202& 6101& 18304& 9152& 4576& 2288& 1144& 572\\
286& 143& 430& 215& 646& 323& 970& 485& 1456& 728\\
364& 182& 91& 274& 137& 412& 206& 103& 310& 155\\
466& 233& 700& 350& 175& 526& 263& 790& 395& 1186\\
593& 1780& 890& 445& 1336& 668& 334& 167& 502& 251\\
754& 377& 1132& 566& 283& 850& 425& 1276& 638& 319\\
958& 479& 1438& 719& 2158& 1079& 3238& 1619& 4858& 2429\\
7288& 3644& 1822& 911& 2734& 1367& 4102& 2051& 6154& 3077\\
9232& 4616& 2308& 1154& 577& 1732& 866& 433& 1300& 650\\
325& 976& 488& 244& 122& 61& 184& 92& 46& 23\\
70& 35& 106& 53& 160& 80& 40& 20& 10& 5\\
16& 8& 4& 2& 1& \\

1692&&&&&&&&&\\
846& 423& 1270& 635& 1906& 953& 2860& 1430& 715& 2146\\
1073& 3220& 1610& 805& 2416& 1208& 604& 302& 151& 454\\
227& 682& 341& 1024& 512& 256& 128& 64& 32& 16\\
8& 4& 2& 1& \\

1693&&&&&&&&&\\
5080& 2540& 1270& 635& 1906& 953& 2860& 1430& 715& 2146\\
1073& 3220& 1610& 805& 2416& 1208& 604& 302& 151& 454\\
227& 682& 341& 1024& 512& 256& 128& 64& 32& 16\\
8& 4& 2& 1& \\

1694&&&&&&&&&\\
847& 2542& 1271& 3814& 1907& 5722& 2861& 8584& 4292& 2146\\
1073& 3220& 1610& 805& 2416& 1208& 604& 302& 151& 454\\
227& 682& 341& 1024& 512& 256& 128& 64& 32& 16\\
8& 4& 2& 1& \\

1695&&&&&&&&&\\
5086& 2543& 7630& 3815& 11446& 5723& 17170& 8585& 25756& 12878\\
6439& 19318& 9659& 28978& 14489& 43468& 21734& 10867& 32602& 16301\\
48904& 24452& 12226& 6113& 18340& 9170& 4585& 13756& 6878& 3439\\
10318& 5159& 15478& 7739& 23218& 11609& 34828& 17414& 8707& 26122\\
13061& 39184& 19592& 9796& 4898& 2449& 7348& 3674& 1837& 5512\\
2756& 1378& 689& 2068& 1034& 517& 1552& 776& 388& 194\\
97& 292& 146& 73& 220& 110& 55& 166& 83& 250\\
125& 376& 188& 94& 47& 142& 71& 214& 107& 322\\
161& 484& 242& 121& 364& 182& 91& 274& 137& 412\\
206& 103& 310& 155& 466& 233& 700& 350& 175& 526\\
263& 790& 395& 1186& 593& 1780& 890& 445& 1336& 668\\
334& 167& 502& 251& 754& 377& 1132& 566& 283& 850\\
425& 1276& 638& 319& 958& 479& 1438& 719& 2158& 1079\\
3238& 1619& 4858& 2429& 7288& 3644& 1822& 911& 2734& 1367\\
4102& 2051& 6154& 3077& 9232& 4616& 2308& 1154& 577& 1732\\
866& 433& 1300& 650& 325& 976& 488& 244& 122& 61\\
184& 92& 46& 23& 70& 35& 106& 53& 160& 80\\
40& 20& 10& 5& 16& 8& 4& 2& 1& \\

1696&&&&&&&&&\\
848& 424& 212& 106& 53& 160& 80& 40& 20& 10\\
5& 16& 8& 4& 2& 1& \\

1697&&&&&&&&&\\
5092& 2546& 1273& 3820& 1910& 955& 2866& 1433& 4300& 2150\\
1075& 3226& 1613& 4840& 2420& 1210& 605& 1816& 908& 454\\
227& 682& 341& 1024& 512& 256& 128& 64& 32& 16\\
8& 4& 2& 1& \\

1698&&&&&&&&&\\
849& 2548& 1274& 637& 1912& 956& 478& 239& 718& 359\\
1078& 539& 1618& 809& 2428& 1214& 607& 1822& 911& 2734\\
1367& 4102& 2051& 6154& 3077& 9232& 4616& 2308& 1154& 577\\
1732& 866& 433& 1300& 650& 325& 976& 488& 244& 122\\
61& 184& 92& 46& 23& 70& 35& 106& 53& 160\\
80& 40& 20& 10& 5& 16& 8& 4& 2& 1\\

1699&&&&&&&&&\\
5098& 2549& 7648& 3824& 1912& 956& 478& 239& 718& 359\\
1078& 539& 1618& 809& 2428& 1214& 607& 1822& 911& 2734\\
1367& 4102& 2051& 6154& 3077& 9232& 4616& 2308& 1154& 577\\
1732& 866& 433& 1300& 650& 325& 976& 488& 244& 122\\
61& 184& 92& 46& 23& 70& 35& 106& 53& 160\\
80& 40& 20& 10& 5& 16& 8& 4& 2& 1\\

1700&&&&&&&&&\\
850& 425& 1276& 638& 319& 958& 479& 1438& 719& 2158\\
1079& 3238& 1619& 4858& 2429& 7288& 3644& 1822& 911& 2734\\
1367& 4102& 2051& 6154& 3077& 9232& 4616& 2308& 1154& 577\\
1732& 866& 433& 1300& 650& 325& 976& 488& 244& 122\\
61& 184& 92& 46& 23& 70& 35& 106& 53& 160\\
80& 40& 20& 10& 5& 16& 8& 4& 2& 1\\

1701&&&&&&&&&\\
5104& 2552& 1276& 638& 319& 958& 479& 1438& 719& 2158\\
1079& 3238& 1619& 4858& 2429& 7288& 3644& 1822& 911& 2734\\
1367& 4102& 2051& 6154& 3077& 9232& 4616& 2308& 1154& 577\\
1732& 866& 433& 1300& 650& 325& 976& 488& 244& 122\\
61& 184& 92& 46& 23& 70& 35& 106& 53& 160\\
80& 40& 20& 10& 5& 16& 8& 4& 2& 1\\

1702&&&&&&&&&\\
851& 2554& 1277& 3832& 1916& 958& 479& 1438& 719& 2158\\
1079& 3238& 1619& 4858& 2429& 7288& 3644& 1822& 911& 2734\\
1367& 4102& 2051& 6154& 3077& 9232& 4616& 2308& 1154& 577\\
1732& 866& 433& 1300& 650& 325& 976& 488& 244& 122\\
61& 184& 92& 46& 23& 70& 35& 106& 53& 160\\
80& 40& 20& 10& 5& 16& 8& 4& 2& 1\\

1703&&&&&&&&&\\
5110& 2555& 7666& 3833& 11500& 5750& 2875& 8626& 4313& 12940\\
6470& 3235& 9706& 4853& 14560& 7280& 3640& 1820& 910& 455\\
1366& 683& 2050& 1025& 3076& 1538& 769& 2308& 1154& 577\\
1732& 866& 433& 1300& 650& 325& 976& 488& 244& 122\\
61& 184& 92& 46& 23& 70& 35& 106& 53& 160\\
80& 40& 20& 10& 5& 16& 8& 4& 2& 1\\

1704&&&&&&&&&\\
852& 426& 213& 640& 320& 160& 80& 40& 20& 10\\
5& 16& 8& 4& 2& 1& \\

1705&&&&&&&&&\\
5116& 2558& 1279& 3838& 1919& 5758& 2879& 8638& 4319& 12958\\
6479& 19438& 9719& 29158& 14579& 43738& 21869& 65608& 32804& 16402\\
8201& 24604& 12302& 6151& 18454& 9227& 27682& 13841& 41524& 20762\\
10381& 31144& 15572& 7786& 3893& 11680& 5840& 2920& 1460& 730\\
365& 1096& 548& 274& 137& 412& 206& 103& 310& 155\\
466& 233& 700& 350& 175& 526& 263& 790& 395& 1186\\
593& 1780& 890& 445& 1336& 668& 334& 167& 502& 251\\
754& 377& 1132& 566& 283& 850& 425& 1276& 638& 319\\
958& 479& 1438& 719& 2158& 1079& 3238& 1619& 4858& 2429\\
7288& 3644& 1822& 911& 2734& 1367& 4102& 2051& 6154& 3077\\
9232& 4616& 2308& 1154& 577& 1732& 866& 433& 1300& 650\\
325& 976& 488& 244& 122& 61& 184& 92& 46& 23\\
70& 35& 106& 53& 160& 80& 40& 20& 10& 5\\
16& 8& 4& 2& 1& \\

1706&&&&&&&&&\\
853& 2560& 1280& 640& 320& 160& 80& 40& 20& 10\\
5& 16& 8& 4& 2& 1& \\

1707&&&&&&&&&\\
5122& 2561& 7684& 3842& 1921& 5764& 2882& 1441& 4324& 2162\\
1081& 3244& 1622& 811& 2434& 1217& 3652& 1826& 913& 2740\\
1370& 685& 2056& 1028& 514& 257& 772& 386& 193& 580\\
290& 145& 436& 218& 109& 328& 164& 82& 41& 124\\
62& 31& 94& 47& 142& 71& 214& 107& 322& 161\\
484& 242& 121& 364& 182& 91& 274& 137& 412& 206\\
103& 310& 155& 466& 233& 700& 350& 175& 526& 263\\
790& 395& 1186& 593& 1780& 890& 445& 1336& 668& 334\\
167& 502& 251& 754& 377& 1132& 566& 283& 850& 425\\
1276& 638& 319& 958& 479& 1438& 719& 2158& 1079& 3238\\
1619& 4858& 2429& 7288& 3644& 1822& 911& 2734& 1367& 4102\\
2051& 6154& 3077& 9232& 4616& 2308& 1154& 577& 1732& 866\\
433& 1300& 650& 325& 976& 488& 244& 122& 61& 184\\
92& 46& 23& 70& 35& 106& 53& 160& 80& 40\\
20& 10& 5& 16& 8& 4& 2& 1& \\

1708&&&&&&&&&\\
854& 427& 1282& 641& 1924& 962& 481& 1444& 722& 361\\
1084& 542& 271& 814& 407& 1222& 611& 1834& 917& 2752\\
1376& 688& 344& 172& 86& 43& 130& 65& 196& 98\\
49& 148& 74& 37& 112& 56& 28& 14& 7& 22\\
11& 34& 17& 52& 26& 13& 40& 20& 10& 5\\
16& 8& 4& 2& 1& \\

1709&&&&&&&&&\\
5128& 2564& 1282& 641& 1924& 962& 481& 1444& 722& 361\\
1084& 542& 271& 814& 407& 1222& 611& 1834& 917& 2752\\
1376& 688& 344& 172& 86& 43& 130& 65& 196& 98\\
49& 148& 74& 37& 112& 56& 28& 14& 7& 22\\
11& 34& 17& 52& 26& 13& 40& 20& 10& 5\\
16& 8& 4& 2& 1& \\

1710&&&&&&&&&\\
855& 2566& 1283& 3850& 1925& 5776& 2888& 1444& 722& 361\\
1084& 542& 271& 814& 407& 1222& 611& 1834& 917& 2752\\
1376& 688& 344& 172& 86& 43& 130& 65& 196& 98\\
49& 148& 74& 37& 112& 56& 28& 14& 7& 22\\
11& 34& 17& 52& 26& 13& 40& 20& 10& 5\\
16& 8& 4& 2& 1& \\

1711&&&&&&&&&\\
5134& 2567& 7702& 3851& 11554& 5777& 17332& 8666& 4333& 13000\\
6500& 3250& 1625& 4876& 2438& 1219& 3658& 1829& 5488& 2744\\
1372& 686& 343& 1030& 515& 1546& 773& 2320& 1160& 580\\
290& 145& 436& 218& 109& 328& 164& 82& 41& 124\\
62& 31& 94& 47& 142& 71& 214& 107& 322& 161\\
484& 242& 121& 364& 182& 91& 274& 137& 412& 206\\
103& 310& 155& 466& 233& 700& 350& 175& 526& 263\\
790& 395& 1186& 593& 1780& 890& 445& 1336& 668& 334\\
167& 502& 251& 754& 377& 1132& 566& 283& 850& 425\\
1276& 638& 319& 958& 479& 1438& 719& 2158& 1079& 3238\\
1619& 4858& 2429& 7288& 3644& 1822& 911& 2734& 1367& 4102\\
2051& 6154& 3077& 9232& 4616& 2308& 1154& 577& 1732& 866\\
433& 1300& 650& 325& 976& 488& 244& 122& 61& 184\\
92& 46& 23& 70& 35& 106& 53& 160& 80& 40\\
20& 10& 5& 16& 8& 4& 2& 1& \\

1712&&&&&&&&&\\
856& 428& 214& 107& 322& 161& 484& 242& 121& 364\\
182& 91& 274& 137& 412& 206& 103& 310& 155& 466\\
233& 700& 350& 175& 526& 263& 790& 395& 1186& 593\\
1780& 890& 445& 1336& 668& 334& 167& 502& 251& 754\\
377& 1132& 566& 283& 850& 425& 1276& 638& 319& 958\\
479& 1438& 719& 2158& 1079& 3238& 1619& 4858& 2429& 7288\\
3644& 1822& 911& 2734& 1367& 4102& 2051& 6154& 3077& 9232\\
4616& 2308& 1154& 577& 1732& 866& 433& 1300& 650& 325\\
976& 488& 244& 122& 61& 184& 92& 46& 23& 70\\
35& 106& 53& 160& 80& 40& 20& 10& 5& 16\\
8& 4& 2& 1& \\

1713&&&&&&&&&\\
5140& 2570& 1285& 3856& 1928& 964& 482& 241& 724& 362\\
181& 544& 272& 136& 68& 34& 17& 52& 26& 13\\
40& 20& 10& 5& 16& 8& 4& 2& 1& \\

1714&&&&&&&&&\\
857& 2572& 1286& 643& 1930& 965& 2896& 1448& 724& 362\\
181& 544& 272& 136& 68& 34& 17& 52& 26& 13\\
40& 20& 10& 5& 16& 8& 4& 2& 1& \\

1715&&&&&&&&&\\
5146& 2573& 7720& 3860& 1930& 965& 2896& 1448& 724& 362\\
181& 544& 272& 136& 68& 34& 17& 52& 26& 13\\
40& 20& 10& 5& 16& 8& 4& 2& 1& \\

1716&&&&&&&&&\\
858& 429& 1288& 644& 322& 161& 484& 242& 121& 364\\
182& 91& 274& 137& 412& 206& 103& 310& 155& 466\\
233& 700& 350& 175& 526& 263& 790& 395& 1186& 593\\
1780& 890& 445& 1336& 668& 334& 167& 502& 251& 754\\
377& 1132& 566& 283& 850& 425& 1276& 638& 319& 958\\
479& 1438& 719& 2158& 1079& 3238& 1619& 4858& 2429& 7288\\
3644& 1822& 911& 2734& 1367& 4102& 2051& 6154& 3077& 9232\\
4616& 2308& 1154& 577& 1732& 866& 433& 1300& 650& 325\\
976& 488& 244& 122& 61& 184& 92& 46& 23& 70\\
35& 106& 53& 160& 80& 40& 20& 10& 5& 16\\
8& 4& 2& 1& \\

1717&&&&&&&&&\\
5152& 2576& 1288& 644& 322& 161& 484& 242& 121& 364\\
182& 91& 274& 137& 412& 206& 103& 310& 155& 466\\
233& 700& 350& 175& 526& 263& 790& 395& 1186& 593\\
1780& 890& 445& 1336& 668& 334& 167& 502& 251& 754\\
377& 1132& 566& 283& 850& 425& 1276& 638& 319& 958\\
479& 1438& 719& 2158& 1079& 3238& 1619& 4858& 2429& 7288\\
3644& 1822& 911& 2734& 1367& 4102& 2051& 6154& 3077& 9232\\
4616& 2308& 1154& 577& 1732& 866& 433& 1300& 650& 325\\
976& 488& 244& 122& 61& 184& 92& 46& 23& 70\\
35& 106& 53& 160& 80& 40& 20& 10& 5& 16\\
8& 4& 2& 1& \\

1718&&&&&&&&&\\
859& 2578& 1289& 3868& 1934& 967& 2902& 1451& 4354& 2177\\
6532& 3266& 1633& 4900& 2450& 1225& 3676& 1838& 919& 2758\\
1379& 4138& 2069& 6208& 3104& 1552& 776& 388& 194& 97\\
292& 146& 73& 220& 110& 55& 166& 83& 250& 125\\
376& 188& 94& 47& 142& 71& 214& 107& 322& 161\\
484& 242& 121& 364& 182& 91& 274& 137& 412& 206\\
103& 310& 155& 466& 233& 700& 350& 175& 526& 263\\
790& 395& 1186& 593& 1780& 890& 445& 1336& 668& 334\\
167& 502& 251& 754& 377& 1132& 566& 283& 850& 425\\
1276& 638& 319& 958& 479& 1438& 719& 2158& 1079& 3238\\
1619& 4858& 2429& 7288& 3644& 1822& 911& 2734& 1367& 4102\\
2051& 6154& 3077& 9232& 4616& 2308& 1154& 577& 1732& 866\\
433& 1300& 650& 325& 976& 488& 244& 122& 61& 184\\
92& 46& 23& 70& 35& 106& 53& 160& 80& 40\\
20& 10& 5& 16& 8& 4& 2& 1& \\

1719&&&&&&&&&\\
5158& 2579& 7738& 3869& 11608& 5804& 2902& 1451& 4354& 2177\\
6532& 3266& 1633& 4900& 2450& 1225& 3676& 1838& 919& 2758\\
1379& 4138& 2069& 6208& 3104& 1552& 776& 388& 194& 97\\
292& 146& 73& 220& 110& 55& 166& 83& 250& 125\\
376& 188& 94& 47& 142& 71& 214& 107& 322& 161\\
484& 242& 121& 364& 182& 91& 274& 137& 412& 206\\
103& 310& 155& 466& 233& 700& 350& 175& 526& 263\\
790& 395& 1186& 593& 1780& 890& 445& 1336& 668& 334\\
167& 502& 251& 754& 377& 1132& 566& 283& 850& 425\\
1276& 638& 319& 958& 479& 1438& 719& 2158& 1079& 3238\\
1619& 4858& 2429& 7288& 3644& 1822& 911& 2734& 1367& 4102\\
2051& 6154& 3077& 9232& 4616& 2308& 1154& 577& 1732& 866\\
433& 1300& 650& 325& 976& 488& 244& 122& 61& 184\\
92& 46& 23& 70& 35& 106& 53& 160& 80& 40\\
20& 10& 5& 16& 8& 4& 2& 1& \\

1720&&&&&&&&&\\
860& 430& 215& 646& 323& 970& 485& 1456& 728& 364\\
182& 91& 274& 137& 412& 206& 103& 310& 155& 466\\
233& 700& 350& 175& 526& 263& 790& 395& 1186& 593\\
1780& 890& 445& 1336& 668& 334& 167& 502& 251& 754\\
377& 1132& 566& 283& 850& 425& 1276& 638& 319& 958\\
479& 1438& 719& 2158& 1079& 3238& 1619& 4858& 2429& 7288\\
3644& 1822& 911& 2734& 1367& 4102& 2051& 6154& 3077& 9232\\
4616& 2308& 1154& 577& 1732& 866& 433& 1300& 650& 325\\
976& 488& 244& 122& 61& 184& 92& 46& 23& 70\\
35& 106& 53& 160& 80& 40& 20& 10& 5& 16\\
8& 4& 2& 1& \\

1721&&&&&&&&&\\
5164& 2582& 1291& 3874& 1937& 5812& 2906& 1453& 4360& 2180\\
1090& 545& 1636& 818& 409& 1228& 614& 307& 922& 461\\
1384& 692& 346& 173& 520& 260& 130& 65& 196& 98\\
49& 148& 74& 37& 112& 56& 28& 14& 7& 22\\
11& 34& 17& 52& 26& 13& 40& 20& 10& 5\\
16& 8& 4& 2& 1& \\

1722&&&&&&&&&\\
861& 2584& 1292& 646& 323& 970& 485& 1456& 728& 364\\
182& 91& 274& 137& 412& 206& 103& 310& 155& 466\\
233& 700& 350& 175& 526& 263& 790& 395& 1186& 593\\
1780& 890& 445& 1336& 668& 334& 167& 502& 251& 754\\
377& 1132& 566& 283& 850& 425& 1276& 638& 319& 958\\
479& 1438& 719& 2158& 1079& 3238& 1619& 4858& 2429& 7288\\
3644& 1822& 911& 2734& 1367& 4102& 2051& 6154& 3077& 9232\\
4616& 2308& 1154& 577& 1732& 866& 433& 1300& 650& 325\\
976& 488& 244& 122& 61& 184& 92& 46& 23& 70\\
35& 106& 53& 160& 80& 40& 20& 10& 5& 16\\
8& 4& 2& 1& \\

1723&&&&&&&&&\\
5170& 2585& 7756& 3878& 1939& 5818& 2909& 8728& 4364& 2182\\
1091& 3274& 1637& 4912& 2456& 1228& 614& 307& 922& 461\\
1384& 692& 346& 173& 520& 260& 130& 65& 196& 98\\
49& 148& 74& 37& 112& 56& 28& 14& 7& 22\\
11& 34& 17& 52& 26& 13& 40& 20& 10& 5\\
16& 8& 4& 2& 1& \\

1724&&&&&&&&&\\
862& 431& 1294& 647& 1942& 971& 2914& 1457& 4372& 2186\\
1093& 3280& 1640& 820& 410& 205& 616& 308& 154& 77\\
232& 116& 58& 29& 88& 44& 22& 11& 34& 17\\
52& 26& 13& 40& 20& 10& 5& 16& 8& 4\\
2& 1& \\

1725&&&&&&&&&\\
5176& 2588& 1294& 647& 1942& 971& 2914& 1457& 4372& 2186\\
1093& 3280& 1640& 820& 410& 205& 616& 308& 154& 77\\
232& 116& 58& 29& 88& 44& 22& 11& 34& 17\\
52& 26& 13& 40& 20& 10& 5& 16& 8& 4\\
2& 1& \\

1726&&&&&&&&&\\
863& 2590& 1295& 3886& 1943& 5830& 2915& 8746& 4373& 13120\\
6560& 3280& 1640& 820& 410& 205& 616& 308& 154& 77\\
232& 116& 58& 29& 88& 44& 22& 11& 34& 17\\
52& 26& 13& 40& 20& 10& 5& 16& 8& 4\\
2& 1& \\

1727&&&&&&&&&\\
5182& 2591& 7774& 3887& 11662& 5831& 17494& 8747& 26242& 13121\\
39364& 19682& 9841& 29524& 14762& 7381& 22144& 11072& 5536& 2768\\
1384& 692& 346& 173& 520& 260& 130& 65& 196& 98\\
49& 148& 74& 37& 112& 56& 28& 14& 7& 22\\
11& 34& 17& 52& 26& 13& 40& 20& 10& 5\\
16& 8& 4& 2& 1& \\

1728&&&&&&&&&\\
864& 432& 216& 108& 54& 27& 82& 41& 124& 62\\
31& 94& 47& 142& 71& 214& 107& 322& 161& 484\\
242& 121& 364& 182& 91& 274& 137& 412& 206& 103\\
310& 155& 466& 233& 700& 350& 175& 526& 263& 790\\
395& 1186& 593& 1780& 890& 445& 1336& 668& 334& 167\\
502& 251& 754& 377& 1132& 566& 283& 850& 425& 1276\\
638& 319& 958& 479& 1438& 719& 2158& 1079& 3238& 1619\\
4858& 2429& 7288& 3644& 1822& 911& 2734& 1367& 4102& 2051\\
6154& 3077& 9232& 4616& 2308& 1154& 577& 1732& 866& 433\\
1300& 650& 325& 976& 488& 244& 122& 61& 184& 92\\
46& 23& 70& 35& 106& 53& 160& 80& 40& 20\\
10& 5& 16& 8& 4& 2& 1& \\

1729&&&&&&&&&\\
5188& 2594& 1297& 3892& 1946& 973& 2920& 1460& 730& 365\\
1096& 548& 274& 137& 412& 206& 103& 310& 155& 466\\
233& 700& 350& 175& 526& 263& 790& 395& 1186& 593\\
1780& 890& 445& 1336& 668& 334& 167& 502& 251& 754\\
377& 1132& 566& 283& 850& 425& 1276& 638& 319& 958\\
479& 1438& 719& 2158& 1079& 3238& 1619& 4858& 2429& 7288\\
3644& 1822& 911& 2734& 1367& 4102& 2051& 6154& 3077& 9232\\
4616& 2308& 1154& 577& 1732& 866& 433& 1300& 650& 325\\
976& 488& 244& 122& 61& 184& 92& 46& 23& 70\\
35& 106& 53& 160& 80& 40& 20& 10& 5& 16\\
8& 4& 2& 1& \\

1730&&&&&&&&&\\
865& 2596& 1298& 649& 1948& 974& 487& 1462& 731& 2194\\
1097& 3292& 1646& 823& 2470& 1235& 3706& 1853& 5560& 2780\\
1390& 695& 2086& 1043& 3130& 1565& 4696& 2348& 1174& 587\\
1762& 881& 2644& 1322& 661& 1984& 992& 496& 248& 124\\
62& 31& 94& 47& 142& 71& 214& 107& 322& 161\\
484& 242& 121& 364& 182& 91& 274& 137& 412& 206\\
103& 310& 155& 466& 233& 700& 350& 175& 526& 263\\
790& 395& 1186& 593& 1780& 890& 445& 1336& 668& 334\\
167& 502& 251& 754& 377& 1132& 566& 283& 850& 425\\
1276& 638& 319& 958& 479& 1438& 719& 2158& 1079& 3238\\
1619& 4858& 2429& 7288& 3644& 1822& 911& 2734& 1367& 4102\\
2051& 6154& 3077& 9232& 4616& 2308& 1154& 577& 1732& 866\\
433& 1300& 650& 325& 976& 488& 244& 122& 61& 184\\
92& 46& 23& 70& 35& 106& 53& 160& 80& 40\\
20& 10& 5& 16& 8& 4& 2& 1& \\

1731&&&&&&&&&\\
5194& 2597& 7792& 3896& 1948& 974& 487& 1462& 731& 2194\\
1097& 3292& 1646& 823& 2470& 1235& 3706& 1853& 5560& 2780\\
1390& 695& 2086& 1043& 3130& 1565& 4696& 2348& 1174& 587\\
1762& 881& 2644& 1322& 661& 1984& 992& 496& 248& 124\\
62& 31& 94& 47& 142& 71& 214& 107& 322& 161\\
484& 242& 121& 364& 182& 91& 274& 137& 412& 206\\
103& 310& 155& 466& 233& 700& 350& 175& 526& 263\\
790& 395& 1186& 593& 1780& 890& 445& 1336& 668& 334\\
167& 502& 251& 754& 377& 1132& 566& 283& 850& 425\\
1276& 638& 319& 958& 479& 1438& 719& 2158& 1079& 3238\\
1619& 4858& 2429& 7288& 3644& 1822& 911& 2734& 1367& 4102\\
2051& 6154& 3077& 9232& 4616& 2308& 1154& 577& 1732& 866\\
433& 1300& 650& 325& 976& 488& 244& 122& 61& 184\\
92& 46& 23& 70& 35& 106& 53& 160& 80& 40\\
20& 10& 5& 16& 8& 4& 2& 1& \\

1732&&&&&&&&&\\
866& 433& 1300& 650& 325& 976& 488& 244& 122& 61\\
184& 92& 46& 23& 70& 35& 106& 53& 160& 80\\
40& 20& 10& 5& 16& 8& 4& 2& 1& \\

1733&&&&&&&&&\\
5200& 2600& 1300& 650& 325& 976& 488& 244& 122& 61\\
184& 92& 46& 23& 70& 35& 106& 53& 160& 80\\
40& 20& 10& 5& 16& 8& 4& 2& 1& \\

1734&&&&&&&&&\\
867& 2602& 1301& 3904& 1952& 976& 488& 244& 122& 61\\
184& 92& 46& 23& 70& 35& 106& 53& 160& 80\\
40& 20& 10& 5& 16& 8& 4& 2& 1& \\

1735&&&&&&&&&\\
5206& 2603& 7810& 3905& 11716& 5858& 2929& 8788& 4394& 2197\\
6592& 3296& 1648& 824& 412& 206& 103& 310& 155& 466\\
233& 700& 350& 175& 526& 263& 790& 395& 1186& 593\\
1780& 890& 445& 1336& 668& 334& 167& 502& 251& 754\\
377& 1132& 566& 283& 850& 425& 1276& 638& 319& 958\\
479& 1438& 719& 2158& 1079& 3238& 1619& 4858& 2429& 7288\\
3644& 1822& 911& 2734& 1367& 4102& 2051& 6154& 3077& 9232\\
4616& 2308& 1154& 577& 1732& 866& 433& 1300& 650& 325\\
976& 488& 244& 122& 61& 184& 92& 46& 23& 70\\
35& 106& 53& 160& 80& 40& 20& 10& 5& 16\\
8& 4& 2& 1& \\

1736&&&&&&&&&\\
868& 434& 217& 652& 326& 163& 490& 245& 736& 368\\
184& 92& 46& 23& 70& 35& 106& 53& 160& 80\\
40& 20& 10& 5& 16& 8& 4& 2& 1& \\

1737&&&&&&&&&\\
5212& 2606& 1303& 3910& 1955& 5866& 2933& 8800& 4400& 2200\\
1100& 550& 275& 826& 413& 1240& 620& 310& 155& 466\\
233& 700& 350& 175& 526& 263& 790& 395& 1186& 593\\
1780& 890& 445& 1336& 668& 334& 167& 502& 251& 754\\
377& 1132& 566& 283& 850& 425& 1276& 638& 319& 958\\
479& 1438& 719& 2158& 1079& 3238& 1619& 4858& 2429& 7288\\
3644& 1822& 911& 2734& 1367& 4102& 2051& 6154& 3077& 9232\\
4616& 2308& 1154& 577& 1732& 866& 433& 1300& 650& 325\\
976& 488& 244& 122& 61& 184& 92& 46& 23& 70\\
35& 106& 53& 160& 80& 40& 20& 10& 5& 16\\
8& 4& 2& 1& \\

1738&&&&&&&&&\\
869& 2608& 1304& 652& 326& 163& 490& 245& 736& 368\\
184& 92& 46& 23& 70& 35& 106& 53& 160& 80\\
40& 20& 10& 5& 16& 8& 4& 2& 1& \\

1739&&&&&&&&&\\
5218& 2609& 7828& 3914& 1957& 5872& 2936& 1468& 734& 367\\
1102& 551& 1654& 827& 2482& 1241& 3724& 1862& 931& 2794\\
1397& 4192& 2096& 1048& 524& 262& 131& 394& 197& 592\\
296& 148& 74& 37& 112& 56& 28& 14& 7& 22\\
11& 34& 17& 52& 26& 13& 40& 20& 10& 5\\
16& 8& 4& 2& 1& \\

1740&&&&&&&&&\\
870& 435& 1306& 653& 1960& 980& 490& 245& 736& 368\\
184& 92& 46& 23& 70& 35& 106& 53& 160& 80\\
40& 20& 10& 5& 16& 8& 4& 2& 1& \\

1741&&&&&&&&&\\
5224& 2612& 1306& 653& 1960& 980& 490& 245& 736& 368\\
184& 92& 46& 23& 70& 35& 106& 53& 160& 80\\
40& 20& 10& 5& 16& 8& 4& 2& 1& \\

1742&&&&&&&&&\\
871& 2614& 1307& 3922& 1961& 5884& 2942& 1471& 4414& 2207\\
6622& 3311& 9934& 4967& 14902& 7451& 22354& 11177& 33532& 16766\\
8383& 25150& 12575& 37726& 18863& 56590& 28295& 84886& 42443& 127330\\
63665& 190996& 95498& 47749& 143248& 71624& 35812& 17906& 8953& 26860\\
13430& 6715& 20146& 10073& 30220& 15110& 7555& 22666& 11333& 34000\\
17000& 8500& 4250& 2125& 6376& 3188& 1594& 797& 2392& 1196\\
598& 299& 898& 449& 1348& 674& 337& 1012& 506& 253\\
760& 380& 190& 95& 286& 143& 430& 215& 646& 323\\
970& 485& 1456& 728& 364& 182& 91& 274& 137& 412\\
206& 103& 310& 155& 466& 233& 700& 350& 175& 526\\
263& 790& 395& 1186& 593& 1780& 890& 445& 1336& 668\\
334& 167& 502& 251& 754& 377& 1132& 566& 283& 850\\
425& 1276& 638& 319& 958& 479& 1438& 719& 2158& 1079\\
3238& 1619& 4858& 2429& 7288& 3644& 1822& 911& 2734& 1367\\
4102& 2051& 6154& 3077& 9232& 4616& 2308& 1154& 577& 1732\\
866& 433& 1300& 650& 325& 976& 488& 244& 122& 61\\
184& 92& 46& 23& 70& 35& 106& 53& 160& 80\\
40& 20& 10& 5& 16& 8& 4& 2& 1& \\

1743&&&&&&&&&\\
5230& 2615& 7846& 3923& 11770& 5885& 17656& 8828& 4414& 2207\\
6622& 3311& 9934& 4967& 14902& 7451& 22354& 11177& 33532& 16766\\
8383& 25150& 12575& 37726& 18863& 56590& 28295& 84886& 42443& 127330\\
63665& 190996& 95498& 47749& 143248& 71624& 35812& 17906& 8953& 26860\\
13430& 6715& 20146& 10073& 30220& 15110& 7555& 22666& 11333& 34000\\
17000& 8500& 4250& 2125& 6376& 3188& 1594& 797& 2392& 1196\\
598& 299& 898& 449& 1348& 674& 337& 1012& 506& 253\\
760& 380& 190& 95& 286& 143& 430& 215& 646& 323\\
970& 485& 1456& 728& 364& 182& 91& 274& 137& 412\\
206& 103& 310& 155& 466& 233& 700& 350& 175& 526\\
263& 790& 395& 1186& 593& 1780& 890& 445& 1336& 668\\
334& 167& 502& 251& 754& 377& 1132& 566& 283& 850\\
425& 1276& 638& 319& 958& 479& 1438& 719& 2158& 1079\\
3238& 1619& 4858& 2429& 7288& 3644& 1822& 911& 2734& 1367\\
4102& 2051& 6154& 3077& 9232& 4616& 2308& 1154& 577& 1732\\
866& 433& 1300& 650& 325& 976& 488& 244& 122& 61\\
184& 92& 46& 23& 70& 35& 106& 53& 160& 80\\
40& 20& 10& 5& 16& 8& 4& 2& 1& \\

1744&&&&&&&&&\\
872& 436& 218& 109& 328& 164& 82& 41& 124& 62\\
31& 94& 47& 142& 71& 214& 107& 322& 161& 484\\
242& 121& 364& 182& 91& 274& 137& 412& 206& 103\\
310& 155& 466& 233& 700& 350& 175& 526& 263& 790\\
395& 1186& 593& 1780& 890& 445& 1336& 668& 334& 167\\
502& 251& 754& 377& 1132& 566& 283& 850& 425& 1276\\
638& 319& 958& 479& 1438& 719& 2158& 1079& 3238& 1619\\
4858& 2429& 7288& 3644& 1822& 911& 2734& 1367& 4102& 2051\\
6154& 3077& 9232& 4616& 2308& 1154& 577& 1732& 866& 433\\
1300& 650& 325& 976& 488& 244& 122& 61& 184& 92\\
46& 23& 70& 35& 106& 53& 160& 80& 40& 20\\
10& 5& 16& 8& 4& 2& 1& \\

1745&&&&&&&&&\\
5236& 2618& 1309& 3928& 1964& 982& 491& 1474& 737& 2212\\
1106& 553& 1660& 830& 415& 1246& 623& 1870& 935& 2806\\
1403& 4210& 2105& 6316& 3158& 1579& 4738& 2369& 7108& 3554\\
1777& 5332& 2666& 1333& 4000& 2000& 1000& 500& 250& 125\\
376& 188& 94& 47& 142& 71& 214& 107& 322& 161\\
484& 242& 121& 364& 182& 91& 274& 137& 412& 206\\
103& 310& 155& 466& 233& 700& 350& 175& 526& 263\\
790& 395& 1186& 593& 1780& 890& 445& 1336& 668& 334\\
167& 502& 251& 754& 377& 1132& 566& 283& 850& 425\\
1276& 638& 319& 958& 479& 1438& 719& 2158& 1079& 3238\\
1619& 4858& 2429& 7288& 3644& 1822& 911& 2734& 1367& 4102\\
2051& 6154& 3077& 9232& 4616& 2308& 1154& 577& 1732& 866\\
433& 1300& 650& 325& 976& 488& 244& 122& 61& 184\\
92& 46& 23& 70& 35& 106& 53& 160& 80& 40\\
20& 10& 5& 16& 8& 4& 2& 1& \\

1746&&&&&&&&&\\
873& 2620& 1310& 655& 1966& 983& 2950& 1475& 4426& 2213\\
6640& 3320& 1660& 830& 415& 1246& 623& 1870& 935& 2806\\
1403& 4210& 2105& 6316& 3158& 1579& 4738& 2369& 7108& 3554\\
1777& 5332& 2666& 1333& 4000& 2000& 1000& 500& 250& 125\\
376& 188& 94& 47& 142& 71& 214& 107& 322& 161\\
484& 242& 121& 364& 182& 91& 274& 137& 412& 206\\
103& 310& 155& 466& 233& 700& 350& 175& 526& 263\\
790& 395& 1186& 593& 1780& 890& 445& 1336& 668& 334\\
167& 502& 251& 754& 377& 1132& 566& 283& 850& 425\\
1276& 638& 319& 958& 479& 1438& 719& 2158& 1079& 3238\\
1619& 4858& 2429& 7288& 3644& 1822& 911& 2734& 1367& 4102\\
2051& 6154& 3077& 9232& 4616& 2308& 1154& 577& 1732& 866\\
433& 1300& 650& 325& 976& 488& 244& 122& 61& 184\\
92& 46& 23& 70& 35& 106& 53& 160& 80& 40\\
20& 10& 5& 16& 8& 4& 2& 1& \\

1747&&&&&&&&&\\
5242& 2621& 7864& 3932& 1966& 983& 2950& 1475& 4426& 2213\\
6640& 3320& 1660& 830& 415& 1246& 623& 1870& 935& 2806\\
1403& 4210& 2105& 6316& 3158& 1579& 4738& 2369& 7108& 3554\\
1777& 5332& 2666& 1333& 4000& 2000& 1000& 500& 250& 125\\
376& 188& 94& 47& 142& 71& 214& 107& 322& 161\\
484& 242& 121& 364& 182& 91& 274& 137& 412& 206\\
103& 310& 155& 466& 233& 700& 350& 175& 526& 263\\
790& 395& 1186& 593& 1780& 890& 445& 1336& 668& 334\\
167& 502& 251& 754& 377& 1132& 566& 283& 850& 425\\
1276& 638& 319& 958& 479& 1438& 719& 2158& 1079& 3238\\
1619& 4858& 2429& 7288& 3644& 1822& 911& 2734& 1367& 4102\\
2051& 6154& 3077& 9232& 4616& 2308& 1154& 577& 1732& 866\\
433& 1300& 650& 325& 976& 488& 244& 122& 61& 184\\
92& 46& 23& 70& 35& 106& 53& 160& 80& 40\\
20& 10& 5& 16& 8& 4& 2& 1& \\

1748&&&&&&&&&\\
874& 437& 1312& 656& 328& 164& 82& 41& 124& 62\\
31& 94& 47& 142& 71& 214& 107& 322& 161& 484\\
242& 121& 364& 182& 91& 274& 137& 412& 206& 103\\
310& 155& 466& 233& 700& 350& 175& 526& 263& 790\\
395& 1186& 593& 1780& 890& 445& 1336& 668& 334& 167\\
502& 251& 754& 377& 1132& 566& 283& 850& 425& 1276\\
638& 319& 958& 479& 1438& 719& 2158& 1079& 3238& 1619\\
4858& 2429& 7288& 3644& 1822& 911& 2734& 1367& 4102& 2051\\
6154& 3077& 9232& 4616& 2308& 1154& 577& 1732& 866& 433\\
1300& 650& 325& 976& 488& 244& 122& 61& 184& 92\\
46& 23& 70& 35& 106& 53& 160& 80& 40& 20\\
10& 5& 16& 8& 4& 2& 1& \\

1749&&&&&&&&&\\
5248& 2624& 1312& 656& 328& 164& 82& 41& 124& 62\\
31& 94& 47& 142& 71& 214& 107& 322& 161& 484\\
242& 121& 364& 182& 91& 274& 137& 412& 206& 103\\
310& 155& 466& 233& 700& 350& 175& 526& 263& 790\\
395& 1186& 593& 1780& 890& 445& 1336& 668& 334& 167\\
502& 251& 754& 377& 1132& 566& 283& 850& 425& 1276\\
638& 319& 958& 479& 1438& 719& 2158& 1079& 3238& 1619\\
4858& 2429& 7288& 3644& 1822& 911& 2734& 1367& 4102& 2051\\
6154& 3077& 9232& 4616& 2308& 1154& 577& 1732& 866& 433\\
1300& 650& 325& 976& 488& 244& 122& 61& 184& 92\\
46& 23& 70& 35& 106& 53& 160& 80& 40& 20\\
10& 5& 16& 8& 4& 2& 1& \\

1750&&&&&&&&&\\
875& 2626& 1313& 3940& 1970& 985& 2956& 1478& 739& 2218\\
1109& 3328& 1664& 832& 416& 208& 104& 52& 26& 13\\
40& 20& 10& 5& 16& 8& 4& 2& 1& \\

1751&&&&&&&&&\\
5254& 2627& 7882& 3941& 11824& 5912& 2956& 1478& 739& 2218\\
1109& 3328& 1664& 832& 416& 208& 104& 52& 26& 13\\
40& 20& 10& 5& 16& 8& 4& 2& 1& \\

1752&&&&&&&&&\\
876& 438& 219& 658& 329& 988& 494& 247& 742& 371\\
1114& 557& 1672& 836& 418& 209& 628& 314& 157& 472\\
236& 118& 59& 178& 89& 268& 134& 67& 202& 101\\
304& 152& 76& 38& 19& 58& 29& 88& 44& 22\\
11& 34& 17& 52& 26& 13& 40& 20& 10& 5\\
16& 8& 4& 2& 1& \\

1753&&&&&&&&&\\
5260& 2630& 1315& 3946& 1973& 5920& 2960& 1480& 740& 370\\
185& 556& 278& 139& 418& 209& 628& 314& 157& 472\\
236& 118& 59& 178& 89& 268& 134& 67& 202& 101\\
304& 152& 76& 38& 19& 58& 29& 88& 44& 22\\
11& 34& 17& 52& 26& 13& 40& 20& 10& 5\\
16& 8& 4& 2& 1& \\

1754&&&&&&&&&\\
877& 2632& 1316& 658& 329& 988& 494& 247& 742& 371\\
1114& 557& 1672& 836& 418& 209& 628& 314& 157& 472\\
236& 118& 59& 178& 89& 268& 134& 67& 202& 101\\
304& 152& 76& 38& 19& 58& 29& 88& 44& 22\\
11& 34& 17& 52& 26& 13& 40& 20& 10& 5\\
16& 8& 4& 2& 1& \\

1755&&&&&&&&&\\
5266& 2633& 7900& 3950& 1975& 5926& 2963& 8890& 4445& 13336\\
6668& 3334& 1667& 5002& 2501& 7504& 3752& 1876& 938& 469\\
1408& 704& 352& 176& 88& 44& 22& 11& 34& 17\\
52& 26& 13& 40& 20& 10& 5& 16& 8& 4\\
2& 1& \\

1756&&&&&&&&&\\
878& 439& 1318& 659& 1978& 989& 2968& 1484& 742& 371\\
1114& 557& 1672& 836& 418& 209& 628& 314& 157& 472\\
236& 118& 59& 178& 89& 268& 134& 67& 202& 101\\
304& 152& 76& 38& 19& 58& 29& 88& 44& 22\\
11& 34& 17& 52& 26& 13& 40& 20& 10& 5\\
16& 8& 4& 2& 1& \\

1757&&&&&&&&&\\
5272& 2636& 1318& 659& 1978& 989& 2968& 1484& 742& 371\\
1114& 557& 1672& 836& 418& 209& 628& 314& 157& 472\\
236& 118& 59& 178& 89& 268& 134& 67& 202& 101\\
304& 152& 76& 38& 19& 58& 29& 88& 44& 22\\
11& 34& 17& 52& 26& 13& 40& 20& 10& 5\\
16& 8& 4& 2& 1& \\

1758&&&&&&&&&\\
879& 2638& 1319& 3958& 1979& 5938& 2969& 8908& 4454& 2227\\
6682& 3341& 10024& 5012& 2506& 1253& 3760& 1880& 940& 470\\
235& 706& 353& 1060& 530& 265& 796& 398& 199& 598\\
299& 898& 449& 1348& 674& 337& 1012& 506& 253& 760\\
380& 190& 95& 286& 143& 430& 215& 646& 323& 970\\
485& 1456& 728& 364& 182& 91& 274& 137& 412& 206\\
103& 310& 155& 466& 233& 700& 350& 175& 526& 263\\
790& 395& 1186& 593& 1780& 890& 445& 1336& 668& 334\\
167& 502& 251& 754& 377& 1132& 566& 283& 850& 425\\
1276& 638& 319& 958& 479& 1438& 719& 2158& 1079& 3238\\
1619& 4858& 2429& 7288& 3644& 1822& 911& 2734& 1367& 4102\\
2051& 6154& 3077& 9232& 4616& 2308& 1154& 577& 1732& 866\\
433& 1300& 650& 325& 976& 488& 244& 122& 61& 184\\
92& 46& 23& 70& 35& 106& 53& 160& 80& 40\\
20& 10& 5& 16& 8& 4& 2& 1& \\

1759&&&&&&&&&\\
5278& 2639& 7918& 3959& 11878& 5939& 17818& 8909& 26728& 13364\\
6682& 3341& 10024& 5012& 2506& 1253& 3760& 1880& 940& 470\\
235& 706& 353& 1060& 530& 265& 796& 398& 199& 598\\
299& 898& 449& 1348& 674& 337& 1012& 506& 253& 760\\
380& 190& 95& 286& 143& 430& 215& 646& 323& 970\\
485& 1456& 728& 364& 182& 91& 274& 137& 412& 206\\
103& 310& 155& 466& 233& 700& 350& 175& 526& 263\\
790& 395& 1186& 593& 1780& 890& 445& 1336& 668& 334\\
167& 502& 251& 754& 377& 1132& 566& 283& 850& 425\\
1276& 638& 319& 958& 479& 1438& 719& 2158& 1079& 3238\\
1619& 4858& 2429& 7288& 3644& 1822& 911& 2734& 1367& 4102\\
2051& 6154& 3077& 9232& 4616& 2308& 1154& 577& 1732& 866\\
433& 1300& 650& 325& 976& 488& 244& 122& 61& 184\\
92& 46& 23& 70& 35& 106& 53& 160& 80& 40\\
20& 10& 5& 16& 8& 4& 2& 1& \\

1760&&&&&&&&&\\
880& 440& 220& 110& 55& 166& 83& 250& 125& 376\\
188& 94& 47& 142& 71& 214& 107& 322& 161& 484\\
242& 121& 364& 182& 91& 274& 137& 412& 206& 103\\
310& 155& 466& 233& 700& 350& 175& 526& 263& 790\\
395& 1186& 593& 1780& 890& 445& 1336& 668& 334& 167\\
502& 251& 754& 377& 1132& 566& 283& 850& 425& 1276\\
638& 319& 958& 479& 1438& 719& 2158& 1079& 3238& 1619\\
4858& 2429& 7288& 3644& 1822& 911& 2734& 1367& 4102& 2051\\
6154& 3077& 9232& 4616& 2308& 1154& 577& 1732& 866& 433\\
1300& 650& 325& 976& 488& 244& 122& 61& 184& 92\\
46& 23& 70& 35& 106& 53& 160& 80& 40& 20\\
10& 5& 16& 8& 4& 2& 1& \\

1761&&&&&&&&&\\
5284& 2642& 1321& 3964& 1982& 991& 2974& 1487& 4462& 2231\\
6694& 3347& 10042& 5021& 15064& 7532& 3766& 1883& 5650& 2825\\
8476& 4238& 2119& 6358& 3179& 9538& 4769& 14308& 7154& 3577\\
10732& 5366& 2683& 8050& 4025& 12076& 6038& 3019& 9058& 4529\\
13588& 6794& 3397& 10192& 5096& 2548& 1274& 637& 1912& 956\\
478& 239& 718& 359& 1078& 539& 1618& 809& 2428& 1214\\
607& 1822& 911& 2734& 1367& 4102& 2051& 6154& 3077& 9232\\
4616& 2308& 1154& 577& 1732& 866& 433& 1300& 650& 325\\
976& 488& 244& 122& 61& 184& 92& 46& 23& 70\\
35& 106& 53& 160& 80& 40& 20& 10& 5& 16\\
8& 4& 2& 1& \\

1762&&&&&&&&&\\
881& 2644& 1322& 661& 1984& 992& 496& 248& 124& 62\\
31& 94& 47& 142& 71& 214& 107& 322& 161& 484\\
242& 121& 364& 182& 91& 274& 137& 412& 206& 103\\
310& 155& 466& 233& 700& 350& 175& 526& 263& 790\\
395& 1186& 593& 1780& 890& 445& 1336& 668& 334& 167\\
502& 251& 754& 377& 1132& 566& 283& 850& 425& 1276\\
638& 319& 958& 479& 1438& 719& 2158& 1079& 3238& 1619\\
4858& 2429& 7288& 3644& 1822& 911& 2734& 1367& 4102& 2051\\
6154& 3077& 9232& 4616& 2308& 1154& 577& 1732& 866& 433\\
1300& 650& 325& 976& 488& 244& 122& 61& 184& 92\\
46& 23& 70& 35& 106& 53& 160& 80& 40& 20\\
10& 5& 16& 8& 4& 2& 1& \\

1763&&&&&&&&&\\
5290& 2645& 7936& 3968& 1984& 992& 496& 248& 124& 62\\
31& 94& 47& 142& 71& 214& 107& 322& 161& 484\\
242& 121& 364& 182& 91& 274& 137& 412& 206& 103\\
310& 155& 466& 233& 700& 350& 175& 526& 263& 790\\
395& 1186& 593& 1780& 890& 445& 1336& 668& 334& 167\\
502& 251& 754& 377& 1132& 566& 283& 850& 425& 1276\\
638& 319& 958& 479& 1438& 719& 2158& 1079& 3238& 1619\\
4858& 2429& 7288& 3644& 1822& 911& 2734& 1367& 4102& 2051\\
6154& 3077& 9232& 4616& 2308& 1154& 577& 1732& 866& 433\\
1300& 650& 325& 976& 488& 244& 122& 61& 184& 92\\
46& 23& 70& 35& 106& 53& 160& 80& 40& 20\\
10& 5& 16& 8& 4& 2& 1& \\

1764&&&&&&&&&\\
882& 441& 1324& 662& 331& 994& 497& 1492& 746& 373\\
1120& 560& 280& 140& 70& 35& 106& 53& 160& 80\\
40& 20& 10& 5& 16& 8& 4& 2& 1& \\

1765&&&&&&&&&\\
5296& 2648& 1324& 662& 331& 994& 497& 1492& 746& 373\\
1120& 560& 280& 140& 70& 35& 106& 53& 160& 80\\
40& 20& 10& 5& 16& 8& 4& 2& 1& \\

1766&&&&&&&&&\\
883& 2650& 1325& 3976& 1988& 994& 497& 1492& 746& 373\\
1120& 560& 280& 140& 70& 35& 106& 53& 160& 80\\
40& 20& 10& 5& 16& 8& 4& 2& 1& \\

1767&&&&&&&&&\\
5302& 2651& 7954& 3977& 11932& 5966& 2983& 8950& 4475& 13426\\
6713& 20140& 10070& 5035& 15106& 7553& 22660& 11330& 5665& 16996\\
8498& 4249& 12748& 6374& 3187& 9562& 4781& 14344& 7172& 3586\\
1793& 5380& 2690& 1345& 4036& 2018& 1009& 3028& 1514& 757\\
2272& 1136& 568& 284& 142& 71& 214& 107& 322& 161\\
484& 242& 121& 364& 182& 91& 274& 137& 412& 206\\
103& 310& 155& 466& 233& 700& 350& 175& 526& 263\\
790& 395& 1186& 593& 1780& 890& 445& 1336& 668& 334\\
167& 502& 251& 754& 377& 1132& 566& 283& 850& 425\\
1276& 638& 319& 958& 479& 1438& 719& 2158& 1079& 3238\\
1619& 4858& 2429& 7288& 3644& 1822& 911& 2734& 1367& 4102\\
2051& 6154& 3077& 9232& 4616& 2308& 1154& 577& 1732& 866\\
433& 1300& 650& 325& 976& 488& 244& 122& 61& 184\\
92& 46& 23& 70& 35& 106& 53& 160& 80& 40\\
20& 10& 5& 16& 8& 4& 2& 1& \\

1768&&&&&&&&&\\
884& 442& 221& 664& 332& 166& 83& 250& 125& 376\\
188& 94& 47& 142& 71& 214& 107& 322& 161& 484\\
242& 121& 364& 182& 91& 274& 137& 412& 206& 103\\
310& 155& 466& 233& 700& 350& 175& 526& 263& 790\\
395& 1186& 593& 1780& 890& 445& 1336& 668& 334& 167\\
502& 251& 754& 377& 1132& 566& 283& 850& 425& 1276\\
638& 319& 958& 479& 1438& 719& 2158& 1079& 3238& 1619\\
4858& 2429& 7288& 3644& 1822& 911& 2734& 1367& 4102& 2051\\
6154& 3077& 9232& 4616& 2308& 1154& 577& 1732& 866& 433\\
1300& 650& 325& 976& 488& 244& 122& 61& 184& 92\\
46& 23& 70& 35& 106& 53& 160& 80& 40& 20\\
10& 5& 16& 8& 4& 2& 1& \\

1769&&&&&&&&&\\
5308& 2654& 1327& 3982& 1991& 5974& 2987& 8962& 4481& 13444\\
6722& 3361& 10084& 5042& 2521& 7564& 3782& 1891& 5674& 2837\\
8512& 4256& 2128& 1064& 532& 266& 133& 400& 200& 100\\
50& 25& 76& 38& 19& 58& 29& 88& 44& 22\\
11& 34& 17& 52& 26& 13& 40& 20& 10& 5\\
16& 8& 4& 2& 1& \\

1770&&&&&&&&&\\
885& 2656& 1328& 664& 332& 166& 83& 250& 125& 376\\
188& 94& 47& 142& 71& 214& 107& 322& 161& 484\\
242& 121& 364& 182& 91& 274& 137& 412& 206& 103\\
310& 155& 466& 233& 700& 350& 175& 526& 263& 790\\
395& 1186& 593& 1780& 890& 445& 1336& 668& 334& 167\\
502& 251& 754& 377& 1132& 566& 283& 850& 425& 1276\\
638& 319& 958& 479& 1438& 719& 2158& 1079& 3238& 1619\\
4858& 2429& 7288& 3644& 1822& 911& 2734& 1367& 4102& 2051\\
6154& 3077& 9232& 4616& 2308& 1154& 577& 1732& 866& 433\\
1300& 650& 325& 976& 488& 244& 122& 61& 184& 92\\
46& 23& 70& 35& 106& 53& 160& 80& 40& 20\\
10& 5& 16& 8& 4& 2& 1& \\

1771&&&&&&&&&\\
5314& 2657& 7972& 3986& 1993& 5980& 2990& 1495& 4486& 2243\\
6730& 3365& 10096& 5048& 2524& 1262& 631& 1894& 947& 2842\\
1421& 4264& 2132& 1066& 533& 1600& 800& 400& 200& 100\\
50& 25& 76& 38& 19& 58& 29& 88& 44& 22\\
11& 34& 17& 52& 26& 13& 40& 20& 10& 5\\
16& 8& 4& 2& 1& \\

1772&&&&&&&&&\\
886& 443& 1330& 665& 1996& 998& 499& 1498& 749& 2248\\
1124& 562& 281& 844& 422& 211& 634& 317& 952& 476\\
238& 119& 358& 179& 538& 269& 808& 404& 202& 101\\
304& 152& 76& 38& 19& 58& 29& 88& 44& 22\\
11& 34& 17& 52& 26& 13& 40& 20& 10& 5\\
16& 8& 4& 2& 1& \\

1773&&&&&&&&&\\
5320& 2660& 1330& 665& 1996& 998& 499& 1498& 749& 2248\\
1124& 562& 281& 844& 422& 211& 634& 317& 952& 476\\
238& 119& 358& 179& 538& 269& 808& 404& 202& 101\\
304& 152& 76& 38& 19& 58& 29& 88& 44& 22\\
11& 34& 17& 52& 26& 13& 40& 20& 10& 5\\
16& 8& 4& 2& 1& \\

1774&&&&&&&&&\\
887& 2662& 1331& 3994& 1997& 5992& 2996& 1498& 749& 2248\\
1124& 562& 281& 844& 422& 211& 634& 317& 952& 476\\
238& 119& 358& 179& 538& 269& 808& 404& 202& 101\\
304& 152& 76& 38& 19& 58& 29& 88& 44& 22\\
11& 34& 17& 52& 26& 13& 40& 20& 10& 5\\
16& 8& 4& 2& 1& \\

1775&&&&&&&&&\\
5326& 2663& 7990& 3995& 11986& 5993& 17980& 8990& 4495& 13486\\
6743& 20230& 10115& 30346& 15173& 45520& 22760& 11380& 5690& 2845\\
8536& 4268& 2134& 1067& 3202& 1601& 4804& 2402& 1201& 3604\\
1802& 901& 2704& 1352& 676& 338& 169& 508& 254& 127\\
382& 191& 574& 287& 862& 431& 1294& 647& 1942& 971\\
2914& 1457& 4372& 2186& 1093& 3280& 1640& 820& 410& 205\\
616& 308& 154& 77& 232& 116& 58& 29& 88& 44\\
22& 11& 34& 17& 52& 26& 13& 40& 20& 10\\
5& 16& 8& 4& 2& 1& \\

1776&&&&&&&&&\\
888& 444& 222& 111& 334& 167& 502& 251& 754& 377\\
1132& 566& 283& 850& 425& 1276& 638& 319& 958& 479\\
1438& 719& 2158& 1079& 3238& 1619& 4858& 2429& 7288& 3644\\
1822& 911& 2734& 1367& 4102& 2051& 6154& 3077& 9232& 4616\\
2308& 1154& 577& 1732& 866& 433& 1300& 650& 325& 976\\
488& 244& 122& 61& 184& 92& 46& 23& 70& 35\\
106& 53& 160& 80& 40& 20& 10& 5& 16& 8\\
4& 2& 1& \\

1777&&&&&&&&&\\
5332& 2666& 1333& 4000& 2000& 1000& 500& 250& 125& 376\\
188& 94& 47& 142& 71& 214& 107& 322& 161& 484\\
242& 121& 364& 182& 91& 274& 137& 412& 206& 103\\
310& 155& 466& 233& 700& 350& 175& 526& 263& 790\\
395& 1186& 593& 1780& 890& 445& 1336& 668& 334& 167\\
502& 251& 754& 377& 1132& 566& 283& 850& 425& 1276\\
638& 319& 958& 479& 1438& 719& 2158& 1079& 3238& 1619\\
4858& 2429& 7288& 3644& 1822& 911& 2734& 1367& 4102& 2051\\
6154& 3077& 9232& 4616& 2308& 1154& 577& 1732& 866& 433\\
1300& 650& 325& 976& 488& 244& 122& 61& 184& 92\\
46& 23& 70& 35& 106& 53& 160& 80& 40& 20\\
10& 5& 16& 8& 4& 2& 1& \\

1778&&&&&&&&&\\
889& 2668& 1334& 667& 2002& 1001& 3004& 1502& 751& 2254\\
1127& 3382& 1691& 5074& 2537& 7612& 3806& 1903& 5710& 2855\\
8566& 4283& 12850& 6425& 19276& 9638& 4819& 14458& 7229& 21688\\
10844& 5422& 2711& 8134& 4067& 12202& 6101& 18304& 9152& 4576\\
2288& 1144& 572& 286& 143& 430& 215& 646& 323& 970\\
485& 1456& 728& 364& 182& 91& 274& 137& 412& 206\\
103& 310& 155& 466& 233& 700& 350& 175& 526& 263\\
790& 395& 1186& 593& 1780& 890& 445& 1336& 668& 334\\
167& 502& 251& 754& 377& 1132& 566& 283& 850& 425\\
1276& 638& 319& 958& 479& 1438& 719& 2158& 1079& 3238\\
1619& 4858& 2429& 7288& 3644& 1822& 911& 2734& 1367& 4102\\
2051& 6154& 3077& 9232& 4616& 2308& 1154& 577& 1732& 866\\
433& 1300& 650& 325& 976& 488& 244& 122& 61& 184\\
92& 46& 23& 70& 35& 106& 53& 160& 80& 40\\
20& 10& 5& 16& 8& 4& 2& 1& \\

1779&&&&&&&&&\\
5338& 2669& 8008& 4004& 2002& 1001& 3004& 1502& 751& 2254\\
1127& 3382& 1691& 5074& 2537& 7612& 3806& 1903& 5710& 2855\\
8566& 4283& 12850& 6425& 19276& 9638& 4819& 14458& 7229& 21688\\
10844& 5422& 2711& 8134& 4067& 12202& 6101& 18304& 9152& 4576\\
2288& 1144& 572& 286& 143& 430& 215& 646& 323& 970\\
485& 1456& 728& 364& 182& 91& 274& 137& 412& 206\\
103& 310& 155& 466& 233& 700& 350& 175& 526& 263\\
790& 395& 1186& 593& 1780& 890& 445& 1336& 668& 334\\
167& 502& 251& 754& 377& 1132& 566& 283& 850& 425\\
1276& 638& 319& 958& 479& 1438& 719& 2158& 1079& 3238\\
1619& 4858& 2429& 7288& 3644& 1822& 911& 2734& 1367& 4102\\
2051& 6154& 3077& 9232& 4616& 2308& 1154& 577& 1732& 866\\
433& 1300& 650& 325& 976& 488& 244& 122& 61& 184\\
92& 46& 23& 70& 35& 106& 53& 160& 80& 40\\
20& 10& 5& 16& 8& 4& 2& 1& \\

1780&&&&&&&&&\\
890& 445& 1336& 668& 334& 167& 502& 251& 754& 377\\
1132& 566& 283& 850& 425& 1276& 638& 319& 958& 479\\
1438& 719& 2158& 1079& 3238& 1619& 4858& 2429& 7288& 3644\\
1822& 911& 2734& 1367& 4102& 2051& 6154& 3077& 9232& 4616\\
2308& 1154& 577& 1732& 866& 433& 1300& 650& 325& 976\\
488& 244& 122& 61& 184& 92& 46& 23& 70& 35\\
106& 53& 160& 80& 40& 20& 10& 5& 16& 8\\
4& 2& 1& \\

1781&&&&&&&&&\\
5344& 2672& 1336& 668& 334& 167& 502& 251& 754& 377\\
1132& 566& 283& 850& 425& 1276& 638& 319& 958& 479\\
1438& 719& 2158& 1079& 3238& 1619& 4858& 2429& 7288& 3644\\
1822& 911& 2734& 1367& 4102& 2051& 6154& 3077& 9232& 4616\\
2308& 1154& 577& 1732& 866& 433& 1300& 650& 325& 976\\
488& 244& 122& 61& 184& 92& 46& 23& 70& 35\\
106& 53& 160& 80& 40& 20& 10& 5& 16& 8\\
4& 2& 1& \\

1782&&&&&&&&&\\
891& 2674& 1337& 4012& 2006& 1003& 3010& 1505& 4516& 2258\\
1129& 3388& 1694& 847& 2542& 1271& 3814& 1907& 5722& 2861\\
8584& 4292& 2146& 1073& 3220& 1610& 805& 2416& 1208& 604\\
302& 151& 454& 227& 682& 341& 1024& 512& 256& 128\\
64& 32& 16& 8& 4& 2& 1& \\

1783&&&&&&&&&\\
5350& 2675& 8026& 4013& 12040& 6020& 3010& 1505& 4516& 2258\\
1129& 3388& 1694& 847& 2542& 1271& 3814& 1907& 5722& 2861\\
8584& 4292& 2146& 1073& 3220& 1610& 805& 2416& 1208& 604\\
302& 151& 454& 227& 682& 341& 1024& 512& 256& 128\\
64& 32& 16& 8& 4& 2& 1& \\

1784&&&&&&&&&\\
892& 446& 223& 670& 335& 1006& 503& 1510& 755& 2266\\
1133& 3400& 1700& 850& 425& 1276& 638& 319& 958& 479\\
1438& 719& 2158& 1079& 3238& 1619& 4858& 2429& 7288& 3644\\
1822& 911& 2734& 1367& 4102& 2051& 6154& 3077& 9232& 4616\\
2308& 1154& 577& 1732& 866& 433& 1300& 650& 325& 976\\
488& 244& 122& 61& 184& 92& 46& 23& 70& 35\\
106& 53& 160& 80& 40& 20& 10& 5& 16& 8\\
4& 2& 1& \\

1785&&&&&&&&&\\
5356& 2678& 1339& 4018& 2009& 6028& 3014& 1507& 4522& 2261\\
6784& 3392& 1696& 848& 424& 212& 106& 53& 160& 80\\
40& 20& 10& 5& 16& 8& 4& 2& 1& \\

1786&&&&&&&&&\\
893& 2680& 1340& 670& 335& 1006& 503& 1510& 755& 2266\\
1133& 3400& 1700& 850& 425& 1276& 638& 319& 958& 479\\
1438& 719& 2158& 1079& 3238& 1619& 4858& 2429& 7288& 3644\\
1822& 911& 2734& 1367& 4102& 2051& 6154& 3077& 9232& 4616\\
2308& 1154& 577& 1732& 866& 433& 1300& 650& 325& 976\\
488& 244& 122& 61& 184& 92& 46& 23& 70& 35\\
106& 53& 160& 80& 40& 20& 10& 5& 16& 8\\
4& 2& 1& \\

1787&&&&&&&&&\\
5362& 2681& 8044& 4022& 2011& 6034& 3017& 9052& 4526& 2263\\
6790& 3395& 10186& 5093& 15280& 7640& 3820& 1910& 955& 2866\\
1433& 4300& 2150& 1075& 3226& 1613& 4840& 2420& 1210& 605\\
1816& 908& 454& 227& 682& 341& 1024& 512& 256& 128\\
64& 32& 16& 8& 4& 2& 1& \\

1788&&&&&&&&&\\
894& 447& 1342& 671& 2014& 1007& 3022& 1511& 4534& 2267\\
6802& 3401& 10204& 5102& 2551& 7654& 3827& 11482& 5741& 17224\\
8612& 4306& 2153& 6460& 3230& 1615& 4846& 2423& 7270& 3635\\
10906& 5453& 16360& 8180& 4090& 2045& 6136& 3068& 1534& 767\\
2302& 1151& 3454& 1727& 5182& 2591& 7774& 3887& 11662& 5831\\
17494& 8747& 26242& 13121& 39364& 19682& 9841& 29524& 14762& 7381\\
22144& 11072& 5536& 2768& 1384& 692& 346& 173& 520& 260\\
130& 65& 196& 98& 49& 148& 74& 37& 112& 56\\
28& 14& 7& 22& 11& 34& 17& 52& 26& 13\\
40& 20& 10& 5& 16& 8& 4& 2& 1& \\

1789&&&&&&&&&\\
5368& 2684& 1342& 671& 2014& 1007& 3022& 1511& 4534& 2267\\
6802& 3401& 10204& 5102& 2551& 7654& 3827& 11482& 5741& 17224\\
8612& 4306& 2153& 6460& 3230& 1615& 4846& 2423& 7270& 3635\\
10906& 5453& 16360& 8180& 4090& 2045& 6136& 3068& 1534& 767\\
2302& 1151& 3454& 1727& 5182& 2591& 7774& 3887& 11662& 5831\\
17494& 8747& 26242& 13121& 39364& 19682& 9841& 29524& 14762& 7381\\
22144& 11072& 5536& 2768& 1384& 692& 346& 173& 520& 260\\
130& 65& 196& 98& 49& 148& 74& 37& 112& 56\\
28& 14& 7& 22& 11& 34& 17& 52& 26& 13\\
40& 20& 10& 5& 16& 8& 4& 2& 1& \\

1790&&&&&&&&&\\
895& 2686& 1343& 4030& 2015& 6046& 3023& 9070& 4535& 13606\\
6803& 20410& 10205& 30616& 15308& 7654& 3827& 11482& 5741& 17224\\
8612& 4306& 2153& 6460& 3230& 1615& 4846& 2423& 7270& 3635\\
10906& 5453& 16360& 8180& 4090& 2045& 6136& 3068& 1534& 767\\
2302& 1151& 3454& 1727& 5182& 2591& 7774& 3887& 11662& 5831\\
17494& 8747& 26242& 13121& 39364& 19682& 9841& 29524& 14762& 7381\\
22144& 11072& 5536& 2768& 1384& 692& 346& 173& 520& 260\\
130& 65& 196& 98& 49& 148& 74& 37& 112& 56\\
28& 14& 7& 22& 11& 34& 17& 52& 26& 13\\
40& 20& 10& 5& 16& 8& 4& 2& 1& \\

1791&&&&&&&&&\\
5374& 2687& 8062& 4031& 12094& 6047& 18142& 9071& 27214& 13607\\
40822& 20411& 61234& 30617& 91852& 45926& 22963& 68890& 34445& 103336\\
51668& 25834& 12917& 38752& 19376& 9688& 4844& 2422& 1211& 3634\\
1817& 5452& 2726& 1363& 4090& 2045& 6136& 3068& 1534& 767\\
2302& 1151& 3454& 1727& 5182& 2591& 7774& 3887& 11662& 5831\\
17494& 8747& 26242& 13121& 39364& 19682& 9841& 29524& 14762& 7381\\
22144& 11072& 5536& 2768& 1384& 692& 346& 173& 520& 260\\
130& 65& 196& 98& 49& 148& 74& 37& 112& 56\\
28& 14& 7& 22& 11& 34& 17& 52& 26& 13\\
40& 20& 10& 5& 16& 8& 4& 2& 1& \\

1792&&&&&&&&&\\
896& 448& 224& 112& 56& 28& 14& 7& 22& 11\\
34& 17& 52& 26& 13& 40& 20& 10& 5& 16\\
8& 4& 2& 1& \\

1793&&&&&&&&&\\
5380& 2690& 1345& 4036& 2018& 1009& 3028& 1514& 757& 2272\\
1136& 568& 284& 142& 71& 214& 107& 322& 161& 484\\
242& 121& 364& 182& 91& 274& 137& 412& 206& 103\\
310& 155& 466& 233& 700& 350& 175& 526& 263& 790\\
395& 1186& 593& 1780& 890& 445& 1336& 668& 334& 167\\
502& 251& 754& 377& 1132& 566& 283& 850& 425& 1276\\
638& 319& 958& 479& 1438& 719& 2158& 1079& 3238& 1619\\
4858& 2429& 7288& 3644& 1822& 911& 2734& 1367& 4102& 2051\\
6154& 3077& 9232& 4616& 2308& 1154& 577& 1732& 866& 433\\
1300& 650& 325& 976& 488& 244& 122& 61& 184& 92\\
46& 23& 70& 35& 106& 53& 160& 80& 40& 20\\
10& 5& 16& 8& 4& 2& 1& \\

1794&&&&&&&&&\\
897& 2692& 1346& 673& 2020& 1010& 505& 1516& 758& 379\\
1138& 569& 1708& 854& 427& 1282& 641& 1924& 962& 481\\
1444& 722& 361& 1084& 542& 271& 814& 407& 1222& 611\\
1834& 917& 2752& 1376& 688& 344& 172& 86& 43& 130\\
65& 196& 98& 49& 148& 74& 37& 112& 56& 28\\
14& 7& 22& 11& 34& 17& 52& 26& 13& 40\\
20& 10& 5& 16& 8& 4& 2& 1& \\

1795&&&&&&&&&\\
5386& 2693& 8080& 4040& 2020& 1010& 505& 1516& 758& 379\\
1138& 569& 1708& 854& 427& 1282& 641& 1924& 962& 481\\
1444& 722& 361& 1084& 542& 271& 814& 407& 1222& 611\\
1834& 917& 2752& 1376& 688& 344& 172& 86& 43& 130\\
65& 196& 98& 49& 148& 74& 37& 112& 56& 28\\
14& 7& 22& 11& 34& 17& 52& 26& 13& 40\\
20& 10& 5& 16& 8& 4& 2& 1& \\

1796&&&&&&&&&\\
898& 449& 1348& 674& 337& 1012& 506& 253& 760& 380\\
190& 95& 286& 143& 430& 215& 646& 323& 970& 485\\
1456& 728& 364& 182& 91& 274& 137& 412& 206& 103\\
310& 155& 466& 233& 700& 350& 175& 526& 263& 790\\
395& 1186& 593& 1780& 890& 445& 1336& 668& 334& 167\\
502& 251& 754& 377& 1132& 566& 283& 850& 425& 1276\\
638& 319& 958& 479& 1438& 719& 2158& 1079& 3238& 1619\\
4858& 2429& 7288& 3644& 1822& 911& 2734& 1367& 4102& 2051\\
6154& 3077& 9232& 4616& 2308& 1154& 577& 1732& 866& 433\\
1300& 650& 325& 976& 488& 244& 122& 61& 184& 92\\
46& 23& 70& 35& 106& 53& 160& 80& 40& 20\\
10& 5& 16& 8& 4& 2& 1& \\

1797&&&&&&&&&\\
5392& 2696& 1348& 674& 337& 1012& 506& 253& 760& 380\\
190& 95& 286& 143& 430& 215& 646& 323& 970& 485\\
1456& 728& 364& 182& 91& 274& 137& 412& 206& 103\\
310& 155& 466& 233& 700& 350& 175& 526& 263& 790\\
395& 1186& 593& 1780& 890& 445& 1336& 668& 334& 167\\
502& 251& 754& 377& 1132& 566& 283& 850& 425& 1276\\
638& 319& 958& 479& 1438& 719& 2158& 1079& 3238& 1619\\
4858& 2429& 7288& 3644& 1822& 911& 2734& 1367& 4102& 2051\\
6154& 3077& 9232& 4616& 2308& 1154& 577& 1732& 866& 433\\
1300& 650& 325& 976& 488& 244& 122& 61& 184& 92\\
46& 23& 70& 35& 106& 53& 160& 80& 40& 20\\
10& 5& 16& 8& 4& 2& 1& \\

1798&&&&&&&&&\\
899& 2698& 1349& 4048& 2024& 1012& 506& 253& 760& 380\\
190& 95& 286& 143& 430& 215& 646& 323& 970& 485\\
1456& 728& 364& 182& 91& 274& 137& 412& 206& 103\\
310& 155& 466& 233& 700& 350& 175& 526& 263& 790\\
395& 1186& 593& 1780& 890& 445& 1336& 668& 334& 167\\
502& 251& 754& 377& 1132& 566& 283& 850& 425& 1276\\
638& 319& 958& 479& 1438& 719& 2158& 1079& 3238& 1619\\
4858& 2429& 7288& 3644& 1822& 911& 2734& 1367& 4102& 2051\\
6154& 3077& 9232& 4616& 2308& 1154& 577& 1732& 866& 433\\
1300& 650& 325& 976& 488& 244& 122& 61& 184& 92\\
46& 23& 70& 35& 106& 53& 160& 80& 40& 20\\
10& 5& 16& 8& 4& 2& 1& \\

1799&&&&&&&&&\\
5398& 2699& 8098& 4049& 12148& 6074& 3037& 9112& 4556& 2278\\
1139& 3418& 1709& 5128& 2564& 1282& 641& 1924& 962& 481\\
1444& 722& 361& 1084& 542& 271& 814& 407& 1222& 611\\
1834& 917& 2752& 1376& 688& 344& 172& 86& 43& 130\\
65& 196& 98& 49& 148& 74& 37& 112& 56& 28\\
14& 7& 22& 11& 34& 17& 52& 26& 13& 40\\
20& 10& 5& 16& 8& 4& 2& 1& \\

1800&&&&&&&&&\\
900& 450& 225& 676& 338& 169& 508& 254& 127& 382\\
191& 574& 287& 862& 431& 1294& 647& 1942& 971& 2914\\
1457& 4372& 2186& 1093& 3280& 1640& 820& 410& 205& 616\\
308& 154& 77& 232& 116& 58& 29& 88& 44& 22\\
11& 34& 17& 52& 26& 13& 40& 20& 10& 5\\
16& 8& 4& 2& 1& \\

1801&&&&&&&&&\\
5404& 2702& 1351& 4054& 2027& 6082& 3041& 9124& 4562& 2281\\
6844& 3422& 1711& 5134& 2567& 7702& 3851& 11554& 5777& 17332\\
8666& 4333& 13000& 6500& 3250& 1625& 4876& 2438& 1219& 3658\\
1829& 5488& 2744& 1372& 686& 343& 1030& 515& 1546& 773\\
2320& 1160& 580& 290& 145& 436& 218& 109& 328& 164\\
82& 41& 124& 62& 31& 94& 47& 142& 71& 214\\
107& 322& 161& 484& 242& 121& 364& 182& 91& 274\\
137& 412& 206& 103& 310& 155& 466& 233& 700& 350\\
175& 526& 263& 790& 395& 1186& 593& 1780& 890& 445\\
1336& 668& 334& 167& 502& 251& 754& 377& 1132& 566\\
283& 850& 425& 1276& 638& 319& 958& 479& 1438& 719\\
2158& 1079& 3238& 1619& 4858& 2429& 7288& 3644& 1822& 911\\
2734& 1367& 4102& 2051& 6154& 3077& 9232& 4616& 2308& 1154\\
577& 1732& 866& 433& 1300& 650& 325& 976& 488& 244\\
122& 61& 184& 92& 46& 23& 70& 35& 106& 53\\
160& 80& 40& 20& 10& 5& 16& 8& 4& 2\\
1& \\

1802&&&&&&&&&\\
901& 2704& 1352& 676& 338& 169& 508& 254& 127& 382\\
191& 574& 287& 862& 431& 1294& 647& 1942& 971& 2914\\
1457& 4372& 2186& 1093& 3280& 1640& 820& 410& 205& 616\\
308& 154& 77& 232& 116& 58& 29& 88& 44& 22\\
11& 34& 17& 52& 26& 13& 40& 20& 10& 5\\
16& 8& 4& 2& 1& \\

1803&&&&&&&&&\\
5410& 2705& 8116& 4058& 2029& 6088& 3044& 1522& 761& 2284\\
1142& 571& 1714& 857& 2572& 1286& 643& 1930& 965& 2896\\
1448& 724& 362& 181& 544& 272& 136& 68& 34& 17\\
52& 26& 13& 40& 20& 10& 5& 16& 8& 4\\
2& 1& \\

1804&&&&&&&&&\\
902& 451& 1354& 677& 2032& 1016& 508& 254& 127& 382\\
191& 574& 287& 862& 431& 1294& 647& 1942& 971& 2914\\
1457& 4372& 2186& 1093& 3280& 1640& 820& 410& 205& 616\\
308& 154& 77& 232& 116& 58& 29& 88& 44& 22\\
11& 34& 17& 52& 26& 13& 40& 20& 10& 5\\
16& 8& 4& 2& 1& \\

1805&&&&&&&&&\\
5416& 2708& 1354& 677& 2032& 1016& 508& 254& 127& 382\\
191& 574& 287& 862& 431& 1294& 647& 1942& 971& 2914\\
1457& 4372& 2186& 1093& 3280& 1640& 820& 410& 205& 616\\
308& 154& 77& 232& 116& 58& 29& 88& 44& 22\\
11& 34& 17& 52& 26& 13& 40& 20& 10& 5\\
16& 8& 4& 2& 1& \\

1806&&&&&&&&&\\
903& 2710& 1355& 4066& 2033& 6100& 3050& 1525& 4576& 2288\\
1144& 572& 286& 143& 430& 215& 646& 323& 970& 485\\
1456& 728& 364& 182& 91& 274& 137& 412& 206& 103\\
310& 155& 466& 233& 700& 350& 175& 526& 263& 790\\
395& 1186& 593& 1780& 890& 445& 1336& 668& 334& 167\\
502& 251& 754& 377& 1132& 566& 283& 850& 425& 1276\\
638& 319& 958& 479& 1438& 719& 2158& 1079& 3238& 1619\\
4858& 2429& 7288& 3644& 1822& 911& 2734& 1367& 4102& 2051\\
6154& 3077& 9232& 4616& 2308& 1154& 577& 1732& 866& 433\\
1300& 650& 325& 976& 488& 244& 122& 61& 184& 92\\
46& 23& 70& 35& 106& 53& 160& 80& 40& 20\\
10& 5& 16& 8& 4& 2& 1& \\

1807&&&&&&&&&\\
5422& 2711& 8134& 4067& 12202& 6101& 18304& 9152& 4576& 2288\\
1144& 572& 286& 143& 430& 215& 646& 323& 970& 485\\
1456& 728& 364& 182& 91& 274& 137& 412& 206& 103\\
310& 155& 466& 233& 700& 350& 175& 526& 263& 790\\
395& 1186& 593& 1780& 890& 445& 1336& 668& 334& 167\\
502& 251& 754& 377& 1132& 566& 283& 850& 425& 1276\\
638& 319& 958& 479& 1438& 719& 2158& 1079& 3238& 1619\\
4858& 2429& 7288& 3644& 1822& 911& 2734& 1367& 4102& 2051\\
6154& 3077& 9232& 4616& 2308& 1154& 577& 1732& 866& 433\\
1300& 650& 325& 976& 488& 244& 122& 61& 184& 92\\
46& 23& 70& 35& 106& 53& 160& 80& 40& 20\\
10& 5& 16& 8& 4& 2& 1& \\

1808&&&&&&&&&\\
904& 452& 226& 113& 340& 170& 85& 256& 128& 64\\
32& 16& 8& 4& 2& 1& \\

1809&&&&&&&&&\\
5428& 2714& 1357& 4072& 2036& 1018& 509& 1528& 764& 382\\
191& 574& 287& 862& 431& 1294& 647& 1942& 971& 2914\\
1457& 4372& 2186& 1093& 3280& 1640& 820& 410& 205& 616\\
308& 154& 77& 232& 116& 58& 29& 88& 44& 22\\
11& 34& 17& 52& 26& 13& 40& 20& 10& 5\\
16& 8& 4& 2& 1& \\

1810&&&&&&&&&\\
905& 2716& 1358& 679& 2038& 1019& 3058& 1529& 4588& 2294\\
1147& 3442& 1721& 5164& 2582& 1291& 3874& 1937& 5812& 2906\\
1453& 4360& 2180& 1090& 545& 1636& 818& 409& 1228& 614\\
307& 922& 461& 1384& 692& 346& 173& 520& 260& 130\\
65& 196& 98& 49& 148& 74& 37& 112& 56& 28\\
14& 7& 22& 11& 34& 17& 52& 26& 13& 40\\
20& 10& 5& 16& 8& 4& 2& 1& \\

1811&&&&&&&&&\\
5434& 2717& 8152& 4076& 2038& 1019& 3058& 1529& 4588& 2294\\
1147& 3442& 1721& 5164& 2582& 1291& 3874& 1937& 5812& 2906\\
1453& 4360& 2180& 1090& 545& 1636& 818& 409& 1228& 614\\
307& 922& 461& 1384& 692& 346& 173& 520& 260& 130\\
65& 196& 98& 49& 148& 74& 37& 112& 56& 28\\
14& 7& 22& 11& 34& 17& 52& 26& 13& 40\\
20& 10& 5& 16& 8& 4& 2& 1& \\

1812&&&&&&&&&\\
906& 453& 1360& 680& 340& 170& 85& 256& 128& 64\\
32& 16& 8& 4& 2& 1& \\

1813&&&&&&&&&\\
5440& 2720& 1360& 680& 340& 170& 85& 256& 128& 64\\
32& 16& 8& 4& 2& 1& \\

1814&&&&&&&&&\\
907& 2722& 1361& 4084& 2042& 1021& 3064& 1532& 766& 383\\
1150& 575& 1726& 863& 2590& 1295& 3886& 1943& 5830& 2915\\
8746& 4373& 13120& 6560& 3280& 1640& 820& 410& 205& 616\\
308& 154& 77& 232& 116& 58& 29& 88& 44& 22\\
11& 34& 17& 52& 26& 13& 40& 20& 10& 5\\
16& 8& 4& 2& 1& \\

1815&&&&&&&&&\\
5446& 2723& 8170& 4085& 12256& 6128& 3064& 1532& 766& 383\\
1150& 575& 1726& 863& 2590& 1295& 3886& 1943& 5830& 2915\\
8746& 4373& 13120& 6560& 3280& 1640& 820& 410& 205& 616\\
308& 154& 77& 232& 116& 58& 29& 88& 44& 22\\
11& 34& 17& 52& 26& 13& 40& 20& 10& 5\\
16& 8& 4& 2& 1& \\

1816&&&&&&&&&\\
908& 454& 227& 682& 341& 1024& 512& 256& 128& 64\\
32& 16& 8& 4& 2& 1& \\

1817&&&&&&&&&\\
5452& 2726& 1363& 4090& 2045& 6136& 3068& 1534& 767& 2302\\
1151& 3454& 1727& 5182& 2591& 7774& 3887& 11662& 5831& 17494\\
8747& 26242& 13121& 39364& 19682& 9841& 29524& 14762& 7381& 22144\\
11072& 5536& 2768& 1384& 692& 346& 173& 520& 260& 130\\
65& 196& 98& 49& 148& 74& 37& 112& 56& 28\\
14& 7& 22& 11& 34& 17& 52& 26& 13& 40\\
20& 10& 5& 16& 8& 4& 2& 1& \\

1818&&&&&&&&&\\
909& 2728& 1364& 682& 341& 1024& 512& 256& 128& 64\\
32& 16& 8& 4& 2& 1& \\

1819&&&&&&&&&\\
5458& 2729& 8188& 4094& 2047& 6142& 3071& 9214& 4607& 13822\\
6911& 20734& 10367& 31102& 15551& 46654& 23327& 69982& 34991& 104974\\
52487& 157462& 78731& 236194& 118097& 354292& 177146& 88573& 265720& 132860\\
66430& 33215& 99646& 49823& 149470& 74735& 224206& 112103& 336310& 168155\\
504466& 252233& 756700& 378350& 189175& 567526& 283763& 851290& 425645& 1276936\\
638468& 319234& 159617& 478852& 239426& 119713& 359140& 179570& 89785& 269356\\
134678& 67339& 202018& 101009& 303028& 151514& 75757& 227272& 113636& 56818\\
28409& 85228& 42614& 21307& 63922& 31961& 95884& 47942& 23971& 71914\\
35957& 107872& 53936& 26968& 13484& 6742& 3371& 10114& 5057& 15172\\
7586& 3793& 11380& 5690& 2845& 8536& 4268& 2134& 1067& 3202\\
1601& 4804& 2402& 1201& 3604& 1802& 901& 2704& 1352& 676\\
338& 169& 508& 254& 127& 382& 191& 574& 287& 862\\
431& 1294& 647& 1942& 971& 2914& 1457& 4372& 2186& 1093\\
3280& 1640& 820& 410& 205& 616& 308& 154& 77& 232\\
116& 58& 29& 88& 44& 22& 11& 34& 17& 52\\
26& 13& 40& 20& 10& 5& 16& 8& 4& 2\\
1& \\

1820&&&&&&&&&\\
910& 455& 1366& 683& 2050& 1025& 3076& 1538& 769& 2308\\
1154& 577& 1732& 866& 433& 1300& 650& 325& 976& 488\\
244& 122& 61& 184& 92& 46& 23& 70& 35& 106\\
53& 160& 80& 40& 20& 10& 5& 16& 8& 4\\
2& 1& \\

1821&&&&&&&&&\\
5464& 2732& 1366& 683& 2050& 1025& 3076& 1538& 769& 2308\\
1154& 577& 1732& 866& 433& 1300& 650& 325& 976& 488\\
244& 122& 61& 184& 92& 46& 23& 70& 35& 106\\
53& 160& 80& 40& 20& 10& 5& 16& 8& 4\\
2& 1& \\

1822&&&&&&&&&\\
911& 2734& 1367& 4102& 2051& 6154& 3077& 9232& 4616& 2308\\
1154& 577& 1732& 866& 433& 1300& 650& 325& 976& 488\\
244& 122& 61& 184& 92& 46& 23& 70& 35& 106\\
53& 160& 80& 40& 20& 10& 5& 16& 8& 4\\
2& 1& \\

1823&&&&&&&&&\\
5470& 2735& 8206& 4103& 12310& 6155& 18466& 9233& 27700& 13850\\
6925& 20776& 10388& 5194& 2597& 7792& 3896& 1948& 974& 487\\
1462& 731& 2194& 1097& 3292& 1646& 823& 2470& 1235& 3706\\
1853& 5560& 2780& 1390& 695& 2086& 1043& 3130& 1565& 4696\\
2348& 1174& 587& 1762& 881& 2644& 1322& 661& 1984& 992\\
496& 248& 124& 62& 31& 94& 47& 142& 71& 214\\
107& 322& 161& 484& 242& 121& 364& 182& 91& 274\\
137& 412& 206& 103& 310& 155& 466& 233& 700& 350\\
175& 526& 263& 790& 395& 1186& 593& 1780& 890& 445\\
1336& 668& 334& 167& 502& 251& 754& 377& 1132& 566\\
283& 850& 425& 1276& 638& 319& 958& 479& 1438& 719\\
2158& 1079& 3238& 1619& 4858& 2429& 7288& 3644& 1822& 911\\
2734& 1367& 4102& 2051& 6154& 3077& 9232& 4616& 2308& 1154\\
577& 1732& 866& 433& 1300& 650& 325& 976& 488& 244\\
122& 61& 184& 92& 46& 23& 70& 35& 106& 53\\
160& 80& 40& 20& 10& 5& 16& 8& 4& 2\\
1& \\

1824&&&&&&&&&\\
912& 456& 228& 114& 57& 172& 86& 43& 130& 65\\
196& 98& 49& 148& 74& 37& 112& 56& 28& 14\\
7& 22& 11& 34& 17& 52& 26& 13& 40& 20\\
10& 5& 16& 8& 4& 2& 1& \\

1825&&&&&&&&&\\
5476& 2738& 1369& 4108& 2054& 1027& 3082& 1541& 4624& 2312\\
1156& 578& 289& 868& 434& 217& 652& 326& 163& 490\\
245& 736& 368& 184& 92& 46& 23& 70& 35& 106\\
53& 160& 80& 40& 20& 10& 5& 16& 8& 4\\
2& 1& \\

1826&&&&&&&&&\\
913& 2740& 1370& 685& 2056& 1028& 514& 257& 772& 386\\
193& 580& 290& 145& 436& 218& 109& 328& 164& 82\\
41& 124& 62& 31& 94& 47& 142& 71& 214& 107\\
322& 161& 484& 242& 121& 364& 182& 91& 274& 137\\
412& 206& 103& 310& 155& 466& 233& 700& 350& 175\\
526& 263& 790& 395& 1186& 593& 1780& 890& 445& 1336\\
668& 334& 167& 502& 251& 754& 377& 1132& 566& 283\\
850& 425& 1276& 638& 319& 958& 479& 1438& 719& 2158\\
1079& 3238& 1619& 4858& 2429& 7288& 3644& 1822& 911& 2734\\
1367& 4102& 2051& 6154& 3077& 9232& 4616& 2308& 1154& 577\\
1732& 866& 433& 1300& 650& 325& 976& 488& 244& 122\\
61& 184& 92& 46& 23& 70& 35& 106& 53& 160\\
80& 40& 20& 10& 5& 16& 8& 4& 2& 1\\

1827&&&&&&&&&\\
5482& 2741& 8224& 4112& 2056& 1028& 514& 257& 772& 386\\
193& 580& 290& 145& 436& 218& 109& 328& 164& 82\\
41& 124& 62& 31& 94& 47& 142& 71& 214& 107\\
322& 161& 484& 242& 121& 364& 182& 91& 274& 137\\
412& 206& 103& 310& 155& 466& 233& 700& 350& 175\\
526& 263& 790& 395& 1186& 593& 1780& 890& 445& 1336\\
668& 334& 167& 502& 251& 754& 377& 1132& 566& 283\\
850& 425& 1276& 638& 319& 958& 479& 1438& 719& 2158\\
1079& 3238& 1619& 4858& 2429& 7288& 3644& 1822& 911& 2734\\
1367& 4102& 2051& 6154& 3077& 9232& 4616& 2308& 1154& 577\\
1732& 866& 433& 1300& 650& 325& 976& 488& 244& 122\\
61& 184& 92& 46& 23& 70& 35& 106& 53& 160\\
80& 40& 20& 10& 5& 16& 8& 4& 2& 1\\

1828&&&&&&&&&\\
914& 457& 1372& 686& 343& 1030& 515& 1546& 773& 2320\\
1160& 580& 290& 145& 436& 218& 109& 328& 164& 82\\
41& 124& 62& 31& 94& 47& 142& 71& 214& 107\\
322& 161& 484& 242& 121& 364& 182& 91& 274& 137\\
412& 206& 103& 310& 155& 466& 233& 700& 350& 175\\
526& 263& 790& 395& 1186& 593& 1780& 890& 445& 1336\\
668& 334& 167& 502& 251& 754& 377& 1132& 566& 283\\
850& 425& 1276& 638& 319& 958& 479& 1438& 719& 2158\\
1079& 3238& 1619& 4858& 2429& 7288& 3644& 1822& 911& 2734\\
1367& 4102& 2051& 6154& 3077& 9232& 4616& 2308& 1154& 577\\
1732& 866& 433& 1300& 650& 325& 976& 488& 244& 122\\
61& 184& 92& 46& 23& 70& 35& 106& 53& 160\\
80& 40& 20& 10& 5& 16& 8& 4& 2& 1\\

1829&&&&&&&&&\\
5488& 2744& 1372& 686& 343& 1030& 515& 1546& 773& 2320\\
1160& 580& 290& 145& 436& 218& 109& 328& 164& 82\\
41& 124& 62& 31& 94& 47& 142& 71& 214& 107\\
322& 161& 484& 242& 121& 364& 182& 91& 274& 137\\
412& 206& 103& 310& 155& 466& 233& 700& 350& 175\\
526& 263& 790& 395& 1186& 593& 1780& 890& 445& 1336\\
668& 334& 167& 502& 251& 754& 377& 1132& 566& 283\\
850& 425& 1276& 638& 319& 958& 479& 1438& 719& 2158\\
1079& 3238& 1619& 4858& 2429& 7288& 3644& 1822& 911& 2734\\
1367& 4102& 2051& 6154& 3077& 9232& 4616& 2308& 1154& 577\\
1732& 866& 433& 1300& 650& 325& 976& 488& 244& 122\\
61& 184& 92& 46& 23& 70& 35& 106& 53& 160\\
80& 40& 20& 10& 5& 16& 8& 4& 2& 1\\

1830&&&&&&&&&\\
915& 2746& 1373& 4120& 2060& 1030& 515& 1546& 773& 2320\\
1160& 580& 290& 145& 436& 218& 109& 328& 164& 82\\
41& 124& 62& 31& 94& 47& 142& 71& 214& 107\\
322& 161& 484& 242& 121& 364& 182& 91& 274& 137\\
412& 206& 103& 310& 155& 466& 233& 700& 350& 175\\
526& 263& 790& 395& 1186& 593& 1780& 890& 445& 1336\\
668& 334& 167& 502& 251& 754& 377& 1132& 566& 283\\
850& 425& 1276& 638& 319& 958& 479& 1438& 719& 2158\\
1079& 3238& 1619& 4858& 2429& 7288& 3644& 1822& 911& 2734\\
1367& 4102& 2051& 6154& 3077& 9232& 4616& 2308& 1154& 577\\
1732& 866& 433& 1300& 650& 325& 976& 488& 244& 122\\
61& 184& 92& 46& 23& 70& 35& 106& 53& 160\\
80& 40& 20& 10& 5& 16& 8& 4& 2& 1\\

1831&&&&&&&&&\\
5494& 2747& 8242& 4121& 12364& 6182& 3091& 9274& 4637& 13912\\
6956& 3478& 1739& 5218& 2609& 7828& 3914& 1957& 5872& 2936\\
1468& 734& 367& 1102& 551& 1654& 827& 2482& 1241& 3724\\
1862& 931& 2794& 1397& 4192& 2096& 1048& 524& 262& 131\\
394& 197& 592& 296& 148& 74& 37& 112& 56& 28\\
14& 7& 22& 11& 34& 17& 52& 26& 13& 40\\
20& 10& 5& 16& 8& 4& 2& 1& \\

1832&&&&&&&&&\\
916& 458& 229& 688& 344& 172& 86& 43& 130& 65\\
196& 98& 49& 148& 74& 37& 112& 56& 28& 14\\
7& 22& 11& 34& 17& 52& 26& 13& 40& 20\\
10& 5& 16& 8& 4& 2& 1& \\

1833&&&&&&&&&\\
5500& 2750& 1375& 4126& 2063& 6190& 3095& 9286& 4643& 13930\\
6965& 20896& 10448& 5224& 2612& 1306& 653& 1960& 980& 490\\
245& 736& 368& 184& 92& 46& 23& 70& 35& 106\\
53& 160& 80& 40& 20& 10& 5& 16& 8& 4\\
2& 1& \\

1834&&&&&&&&&\\
917& 2752& 1376& 688& 344& 172& 86& 43& 130& 65\\
196& 98& 49& 148& 74& 37& 112& 56& 28& 14\\
7& 22& 11& 34& 17& 52& 26& 13& 40& 20\\
10& 5& 16& 8& 4& 2& 1& \\

1835&&&&&&&&&\\
5506& 2753& 8260& 4130& 2065& 6196& 3098& 1549& 4648& 2324\\
1162& 581& 1744& 872& 436& 218& 109& 328& 164& 82\\
41& 124& 62& 31& 94& 47& 142& 71& 214& 107\\
322& 161& 484& 242& 121& 364& 182& 91& 274& 137\\
412& 206& 103& 310& 155& 466& 233& 700& 350& 175\\
526& 263& 790& 395& 1186& 593& 1780& 890& 445& 1336\\
668& 334& 167& 502& 251& 754& 377& 1132& 566& 283\\
850& 425& 1276& 638& 319& 958& 479& 1438& 719& 2158\\
1079& 3238& 1619& 4858& 2429& 7288& 3644& 1822& 911& 2734\\
1367& 4102& 2051& 6154& 3077& 9232& 4616& 2308& 1154& 577\\
1732& 866& 433& 1300& 650& 325& 976& 488& 244& 122\\
61& 184& 92& 46& 23& 70& 35& 106& 53& 160\\
80& 40& 20& 10& 5& 16& 8& 4& 2& 1\\

1836&&&&&&&&&\\
918& 459& 1378& 689& 2068& 1034& 517& 1552& 776& 388\\
194& 97& 292& 146& 73& 220& 110& 55& 166& 83\\
250& 125& 376& 188& 94& 47& 142& 71& 214& 107\\
322& 161& 484& 242& 121& 364& 182& 91& 274& 137\\
412& 206& 103& 310& 155& 466& 233& 700& 350& 175\\
526& 263& 790& 395& 1186& 593& 1780& 890& 445& 1336\\
668& 334& 167& 502& 251& 754& 377& 1132& 566& 283\\
850& 425& 1276& 638& 319& 958& 479& 1438& 719& 2158\\
1079& 3238& 1619& 4858& 2429& 7288& 3644& 1822& 911& 2734\\
1367& 4102& 2051& 6154& 3077& 9232& 4616& 2308& 1154& 577\\
1732& 866& 433& 1300& 650& 325& 976& 488& 244& 122\\
61& 184& 92& 46& 23& 70& 35& 106& 53& 160\\
80& 40& 20& 10& 5& 16& 8& 4& 2& 1\\

1837&&&&&&&&&\\
5512& 2756& 1378& 689& 2068& 1034& 517& 1552& 776& 388\\
194& 97& 292& 146& 73& 220& 110& 55& 166& 83\\
250& 125& 376& 188& 94& 47& 142& 71& 214& 107\\
322& 161& 484& 242& 121& 364& 182& 91& 274& 137\\
412& 206& 103& 310& 155& 466& 233& 700& 350& 175\\
526& 263& 790& 395& 1186& 593& 1780& 890& 445& 1336\\
668& 334& 167& 502& 251& 754& 377& 1132& 566& 283\\
850& 425& 1276& 638& 319& 958& 479& 1438& 719& 2158\\
1079& 3238& 1619& 4858& 2429& 7288& 3644& 1822& 911& 2734\\
1367& 4102& 2051& 6154& 3077& 9232& 4616& 2308& 1154& 577\\
1732& 866& 433& 1300& 650& 325& 976& 488& 244& 122\\
61& 184& 92& 46& 23& 70& 35& 106& 53& 160\\
80& 40& 20& 10& 5& 16& 8& 4& 2& 1\\

1838&&&&&&&&&\\
919& 2758& 1379& 4138& 2069& 6208& 3104& 1552& 776& 388\\
194& 97& 292& 146& 73& 220& 110& 55& 166& 83\\
250& 125& 376& 188& 94& 47& 142& 71& 214& 107\\
322& 161& 484& 242& 121& 364& 182& 91& 274& 137\\
412& 206& 103& 310& 155& 466& 233& 700& 350& 175\\
526& 263& 790& 395& 1186& 593& 1780& 890& 445& 1336\\
668& 334& 167& 502& 251& 754& 377& 1132& 566& 283\\
850& 425& 1276& 638& 319& 958& 479& 1438& 719& 2158\\
1079& 3238& 1619& 4858& 2429& 7288& 3644& 1822& 911& 2734\\
1367& 4102& 2051& 6154& 3077& 9232& 4616& 2308& 1154& 577\\
1732& 866& 433& 1300& 650& 325& 976& 488& 244& 122\\
61& 184& 92& 46& 23& 70& 35& 106& 53& 160\\
80& 40& 20& 10& 5& 16& 8& 4& 2& 1\\

1839&&&&&&&&&\\
5518& 2759& 8278& 4139& 12418& 6209& 18628& 9314& 4657& 13972\\
6986& 3493& 10480& 5240& 2620& 1310& 655& 1966& 983& 2950\\
1475& 4426& 2213& 6640& 3320& 1660& 830& 415& 1246& 623\\
1870& 935& 2806& 1403& 4210& 2105& 6316& 3158& 1579& 4738\\
2369& 7108& 3554& 1777& 5332& 2666& 1333& 4000& 2000& 1000\\
500& 250& 125& 376& 188& 94& 47& 142& 71& 214\\
107& 322& 161& 484& 242& 121& 364& 182& 91& 274\\
137& 412& 206& 103& 310& 155& 466& 233& 700& 350\\
175& 526& 263& 790& 395& 1186& 593& 1780& 890& 445\\
1336& 668& 334& 167& 502& 251& 754& 377& 1132& 566\\
283& 850& 425& 1276& 638& 319& 958& 479& 1438& 719\\
2158& 1079& 3238& 1619& 4858& 2429& 7288& 3644& 1822& 911\\
2734& 1367& 4102& 2051& 6154& 3077& 9232& 4616& 2308& 1154\\
577& 1732& 866& 433& 1300& 650& 325& 976& 488& 244\\
122& 61& 184& 92& 46& 23& 70& 35& 106& 53\\
160& 80& 40& 20& 10& 5& 16& 8& 4& 2\\
1& \\

1840&&&&&&&&&\\
920& 460& 230& 115& 346& 173& 520& 260& 130& 65\\
196& 98& 49& 148& 74& 37& 112& 56& 28& 14\\
7& 22& 11& 34& 17& 52& 26& 13& 40& 20\\
10& 5& 16& 8& 4& 2& 1& \\

1841&&&&&&&&&\\
5524& 2762& 1381& 4144& 2072& 1036& 518& 259& 778& 389\\
1168& 584& 292& 146& 73& 220& 110& 55& 166& 83\\
250& 125& 376& 188& 94& 47& 142& 71& 214& 107\\
322& 161& 484& 242& 121& 364& 182& 91& 274& 137\\
412& 206& 103& 310& 155& 466& 233& 700& 350& 175\\
526& 263& 790& 395& 1186& 593& 1780& 890& 445& 1336\\
668& 334& 167& 502& 251& 754& 377& 1132& 566& 283\\
850& 425& 1276& 638& 319& 958& 479& 1438& 719& 2158\\
1079& 3238& 1619& 4858& 2429& 7288& 3644& 1822& 911& 2734\\
1367& 4102& 2051& 6154& 3077& 9232& 4616& 2308& 1154& 577\\
1732& 866& 433& 1300& 650& 325& 976& 488& 244& 122\\
61& 184& 92& 46& 23& 70& 35& 106& 53& 160\\
80& 40& 20& 10& 5& 16& 8& 4& 2& 1\\

1842&&&&&&&&&\\
921& 2764& 1382& 691& 2074& 1037& 3112& 1556& 778& 389\\
1168& 584& 292& 146& 73& 220& 110& 55& 166& 83\\
250& 125& 376& 188& 94& 47& 142& 71& 214& 107\\
322& 161& 484& 242& 121& 364& 182& 91& 274& 137\\
412& 206& 103& 310& 155& 466& 233& 700& 350& 175\\
526& 263& 790& 395& 1186& 593& 1780& 890& 445& 1336\\
668& 334& 167& 502& 251& 754& 377& 1132& 566& 283\\
850& 425& 1276& 638& 319& 958& 479& 1438& 719& 2158\\
1079& 3238& 1619& 4858& 2429& 7288& 3644& 1822& 911& 2734\\
1367& 4102& 2051& 6154& 3077& 9232& 4616& 2308& 1154& 577\\
1732& 866& 433& 1300& 650& 325& 976& 488& 244& 122\\
61& 184& 92& 46& 23& 70& 35& 106& 53& 160\\
80& 40& 20& 10& 5& 16& 8& 4& 2& 1\\

1843&&&&&&&&&\\
5530& 2765& 8296& 4148& 2074& 1037& 3112& 1556& 778& 389\\
1168& 584& 292& 146& 73& 220& 110& 55& 166& 83\\
250& 125& 376& 188& 94& 47& 142& 71& 214& 107\\
322& 161& 484& 242& 121& 364& 182& 91& 274& 137\\
412& 206& 103& 310& 155& 466& 233& 700& 350& 175\\
526& 263& 790& 395& 1186& 593& 1780& 890& 445& 1336\\
668& 334& 167& 502& 251& 754& 377& 1132& 566& 283\\
850& 425& 1276& 638& 319& 958& 479& 1438& 719& 2158\\
1079& 3238& 1619& 4858& 2429& 7288& 3644& 1822& 911& 2734\\
1367& 4102& 2051& 6154& 3077& 9232& 4616& 2308& 1154& 577\\
1732& 866& 433& 1300& 650& 325& 976& 488& 244& 122\\
61& 184& 92& 46& 23& 70& 35& 106& 53& 160\\
80& 40& 20& 10& 5& 16& 8& 4& 2& 1\\

1844&&&&&&&&&\\
922& 461& 1384& 692& 346& 173& 520& 260& 130& 65\\
196& 98& 49& 148& 74& 37& 112& 56& 28& 14\\
7& 22& 11& 34& 17& 52& 26& 13& 40& 20\\
10& 5& 16& 8& 4& 2& 1& \\

1845&&&&&&&&&\\
5536& 2768& 1384& 692& 346& 173& 520& 260& 130& 65\\
196& 98& 49& 148& 74& 37& 112& 56& 28& 14\\
7& 22& 11& 34& 17& 52& 26& 13& 40& 20\\
10& 5& 16& 8& 4& 2& 1& \\

1846&&&&&&&&&\\
923& 2770& 1385& 4156& 2078& 1039& 3118& 1559& 4678& 2339\\
7018& 3509& 10528& 5264& 2632& 1316& 658& 329& 988& 494\\
247& 742& 371& 1114& 557& 1672& 836& 418& 209& 628\\
314& 157& 472& 236& 118& 59& 178& 89& 268& 134\\
67& 202& 101& 304& 152& 76& 38& 19& 58& 29\\
88& 44& 22& 11& 34& 17& 52& 26& 13& 40\\
20& 10& 5& 16& 8& 4& 2& 1& \\

1847&&&&&&&&&\\
5542& 2771& 8314& 4157& 12472& 6236& 3118& 1559& 4678& 2339\\
7018& 3509& 10528& 5264& 2632& 1316& 658& 329& 988& 494\\
247& 742& 371& 1114& 557& 1672& 836& 418& 209& 628\\
314& 157& 472& 236& 118& 59& 178& 89& 268& 134\\
67& 202& 101& 304& 152& 76& 38& 19& 58& 29\\
88& 44& 22& 11& 34& 17& 52& 26& 13& 40\\
20& 10& 5& 16& 8& 4& 2& 1& \\

1848&&&&&&&&&\\
924& 462& 231& 694& 347& 1042& 521& 1564& 782& 391\\
1174& 587& 1762& 881& 2644& 1322& 661& 1984& 992& 496\\
248& 124& 62& 31& 94& 47& 142& 71& 214& 107\\
322& 161& 484& 242& 121& 364& 182& 91& 274& 137\\
412& 206& 103& 310& 155& 466& 233& 700& 350& 175\\
526& 263& 790& 395& 1186& 593& 1780& 890& 445& 1336\\
668& 334& 167& 502& 251& 754& 377& 1132& 566& 283\\
850& 425& 1276& 638& 319& 958& 479& 1438& 719& 2158\\
1079& 3238& 1619& 4858& 2429& 7288& 3644& 1822& 911& 2734\\
1367& 4102& 2051& 6154& 3077& 9232& 4616& 2308& 1154& 577\\
1732& 866& 433& 1300& 650& 325& 976& 488& 244& 122\\
61& 184& 92& 46& 23& 70& 35& 106& 53& 160\\
80& 40& 20& 10& 5& 16& 8& 4& 2& 1\\

1849&&&&&&&&&\\
5548& 2774& 1387& 4162& 2081& 6244& 3122& 1561& 4684& 2342\\
1171& 3514& 1757& 5272& 2636& 1318& 659& 1978& 989& 2968\\
1484& 742& 371& 1114& 557& 1672& 836& 418& 209& 628\\
314& 157& 472& 236& 118& 59& 178& 89& 268& 134\\
67& 202& 101& 304& 152& 76& 38& 19& 58& 29\\
88& 44& 22& 11& 34& 17& 52& 26& 13& 40\\
20& 10& 5& 16& 8& 4& 2& 1& \\

1850&&&&&&&&&\\
925& 2776& 1388& 694& 347& 1042& 521& 1564& 782& 391\\
1174& 587& 1762& 881& 2644& 1322& 661& 1984& 992& 496\\
248& 124& 62& 31& 94& 47& 142& 71& 214& 107\\
322& 161& 484& 242& 121& 364& 182& 91& 274& 137\\
412& 206& 103& 310& 155& 466& 233& 700& 350& 175\\
526& 263& 790& 395& 1186& 593& 1780& 890& 445& 1336\\
668& 334& 167& 502& 251& 754& 377& 1132& 566& 283\\
850& 425& 1276& 638& 319& 958& 479& 1438& 719& 2158\\
1079& 3238& 1619& 4858& 2429& 7288& 3644& 1822& 911& 2734\\
1367& 4102& 2051& 6154& 3077& 9232& 4616& 2308& 1154& 577\\
1732& 866& 433& 1300& 650& 325& 976& 488& 244& 122\\
61& 184& 92& 46& 23& 70& 35& 106& 53& 160\\
80& 40& 20& 10& 5& 16& 8& 4& 2& 1\\

1851&&&&&&&&&\\
5554& 2777& 8332& 4166& 2083& 6250& 3125& 9376& 4688& 2344\\
1172& 586& 293& 880& 440& 220& 110& 55& 166& 83\\
250& 125& 376& 188& 94& 47& 142& 71& 214& 107\\
322& 161& 484& 242& 121& 364& 182& 91& 274& 137\\
412& 206& 103& 310& 155& 466& 233& 700& 350& 175\\
526& 263& 790& 395& 1186& 593& 1780& 890& 445& 1336\\
668& 334& 167& 502& 251& 754& 377& 1132& 566& 283\\
850& 425& 1276& 638& 319& 958& 479& 1438& 719& 2158\\
1079& 3238& 1619& 4858& 2429& 7288& 3644& 1822& 911& 2734\\
1367& 4102& 2051& 6154& 3077& 9232& 4616& 2308& 1154& 577\\
1732& 866& 433& 1300& 650& 325& 976& 488& 244& 122\\
61& 184& 92& 46& 23& 70& 35& 106& 53& 160\\
80& 40& 20& 10& 5& 16& 8& 4& 2& 1\\

1852&&&&&&&&&\\
926& 463& 1390& 695& 2086& 1043& 3130& 1565& 4696& 2348\\
1174& 587& 1762& 881& 2644& 1322& 661& 1984& 992& 496\\
248& 124& 62& 31& 94& 47& 142& 71& 214& 107\\
322& 161& 484& 242& 121& 364& 182& 91& 274& 137\\
412& 206& 103& 310& 155& 466& 233& 700& 350& 175\\
526& 263& 790& 395& 1186& 593& 1780& 890& 445& 1336\\
668& 334& 167& 502& 251& 754& 377& 1132& 566& 283\\
850& 425& 1276& 638& 319& 958& 479& 1438& 719& 2158\\
1079& 3238& 1619& 4858& 2429& 7288& 3644& 1822& 911& 2734\\
1367& 4102& 2051& 6154& 3077& 9232& 4616& 2308& 1154& 577\\
1732& 866& 433& 1300& 650& 325& 976& 488& 244& 122\\
61& 184& 92& 46& 23& 70& 35& 106& 53& 160\\
80& 40& 20& 10& 5& 16& 8& 4& 2& 1\\

1853&&&&&&&&&\\
5560& 2780& 1390& 695& 2086& 1043& 3130& 1565& 4696& 2348\\
1174& 587& 1762& 881& 2644& 1322& 661& 1984& 992& 496\\
248& 124& 62& 31& 94& 47& 142& 71& 214& 107\\
322& 161& 484& 242& 121& 364& 182& 91& 274& 137\\
412& 206& 103& 310& 155& 466& 233& 700& 350& 175\\
526& 263& 790& 395& 1186& 593& 1780& 890& 445& 1336\\
668& 334& 167& 502& 251& 754& 377& 1132& 566& 283\\
850& 425& 1276& 638& 319& 958& 479& 1438& 719& 2158\\
1079& 3238& 1619& 4858& 2429& 7288& 3644& 1822& 911& 2734\\
1367& 4102& 2051& 6154& 3077& 9232& 4616& 2308& 1154& 577\\
1732& 866& 433& 1300& 650& 325& 976& 488& 244& 122\\
61& 184& 92& 46& 23& 70& 35& 106& 53& 160\\
80& 40& 20& 10& 5& 16& 8& 4& 2& 1\\

1854&&&&&&&&&\\
927& 2782& 1391& 4174& 2087& 6262& 3131& 9394& 4697& 14092\\
7046& 3523& 10570& 5285& 15856& 7928& 3964& 1982& 991& 2974\\
1487& 4462& 2231& 6694& 3347& 10042& 5021& 15064& 7532& 3766\\
1883& 5650& 2825& 8476& 4238& 2119& 6358& 3179& 9538& 4769\\
14308& 7154& 3577& 10732& 5366& 2683& 8050& 4025& 12076& 6038\\
3019& 9058& 4529& 13588& 6794& 3397& 10192& 5096& 2548& 1274\\
637& 1912& 956& 478& 239& 718& 359& 1078& 539& 1618\\
809& 2428& 1214& 607& 1822& 911& 2734& 1367& 4102& 2051\\
6154& 3077& 9232& 4616& 2308& 1154& 577& 1732& 866& 433\\
1300& 650& 325& 976& 488& 244& 122& 61& 184& 92\\
46& 23& 70& 35& 106& 53& 160& 80& 40& 20\\
10& 5& 16& 8& 4& 2& 1& \\

1855&&&&&&&&&\\
5566& 2783& 8350& 4175& 12526& 6263& 18790& 9395& 28186& 14093\\
42280& 21140& 10570& 5285& 15856& 7928& 3964& 1982& 991& 2974\\
1487& 4462& 2231& 6694& 3347& 10042& 5021& 15064& 7532& 3766\\
1883& 5650& 2825& 8476& 4238& 2119& 6358& 3179& 9538& 4769\\
14308& 7154& 3577& 10732& 5366& 2683& 8050& 4025& 12076& 6038\\
3019& 9058& 4529& 13588& 6794& 3397& 10192& 5096& 2548& 1274\\
637& 1912& 956& 478& 239& 718& 359& 1078& 539& 1618\\
809& 2428& 1214& 607& 1822& 911& 2734& 1367& 4102& 2051\\
6154& 3077& 9232& 4616& 2308& 1154& 577& 1732& 866& 433\\
1300& 650& 325& 976& 488& 244& 122& 61& 184& 92\\
46& 23& 70& 35& 106& 53& 160& 80& 40& 20\\
10& 5& 16& 8& 4& 2& 1& \\

1856&&&&&&&&&\\
928& 464& 232& 116& 58& 29& 88& 44& 22& 11\\
34& 17& 52& 26& 13& 40& 20& 10& 5& 16\\
8& 4& 2& 1& \\

1857&&&&&&&&&\\
5572& 2786& 1393& 4180& 2090& 1045& 3136& 1568& 784& 392\\
196& 98& 49& 148& 74& 37& 112& 56& 28& 14\\
7& 22& 11& 34& 17& 52& 26& 13& 40& 20\\
10& 5& 16& 8& 4& 2& 1& \\

1858&&&&&&&&&\\
929& 2788& 1394& 697& 2092& 1046& 523& 1570& 785& 2356\\
1178& 589& 1768& 884& 442& 221& 664& 332& 166& 83\\
250& 125& 376& 188& 94& 47& 142& 71& 214& 107\\
322& 161& 484& 242& 121& 364& 182& 91& 274& 137\\
412& 206& 103& 310& 155& 466& 233& 700& 350& 175\\
526& 263& 790& 395& 1186& 593& 1780& 890& 445& 1336\\
668& 334& 167& 502& 251& 754& 377& 1132& 566& 283\\
850& 425& 1276& 638& 319& 958& 479& 1438& 719& 2158\\
1079& 3238& 1619& 4858& 2429& 7288& 3644& 1822& 911& 2734\\
1367& 4102& 2051& 6154& 3077& 9232& 4616& 2308& 1154& 577\\
1732& 866& 433& 1300& 650& 325& 976& 488& 244& 122\\
61& 184& 92& 46& 23& 70& 35& 106& 53& 160\\
80& 40& 20& 10& 5& 16& 8& 4& 2& 1\\

1859&&&&&&&&&\\
5578& 2789& 8368& 4184& 2092& 1046& 523& 1570& 785& 2356\\
1178& 589& 1768& 884& 442& 221& 664& 332& 166& 83\\
250& 125& 376& 188& 94& 47& 142& 71& 214& 107\\
322& 161& 484& 242& 121& 364& 182& 91& 274& 137\\
412& 206& 103& 310& 155& 466& 233& 700& 350& 175\\
526& 263& 790& 395& 1186& 593& 1780& 890& 445& 1336\\
668& 334& 167& 502& 251& 754& 377& 1132& 566& 283\\
850& 425& 1276& 638& 319& 958& 479& 1438& 719& 2158\\
1079& 3238& 1619& 4858& 2429& 7288& 3644& 1822& 911& 2734\\
1367& 4102& 2051& 6154& 3077& 9232& 4616& 2308& 1154& 577\\
1732& 866& 433& 1300& 650& 325& 976& 488& 244& 122\\
61& 184& 92& 46& 23& 70& 35& 106& 53& 160\\
80& 40& 20& 10& 5& 16& 8& 4& 2& 1\\

1860&&&&&&&&&\\
930& 465& 1396& 698& 349& 1048& 524& 262& 131& 394\\
197& 592& 296& 148& 74& 37& 112& 56& 28& 14\\
7& 22& 11& 34& 17& 52& 26& 13& 40& 20\\
10& 5& 16& 8& 4& 2& 1& \\

1861&&&&&&&&&\\
5584& 2792& 1396& 698& 349& 1048& 524& 262& 131& 394\\
197& 592& 296& 148& 74& 37& 112& 56& 28& 14\\
7& 22& 11& 34& 17& 52& 26& 13& 40& 20\\
10& 5& 16& 8& 4& 2& 1& \\

1862&&&&&&&&&\\
931& 2794& 1397& 4192& 2096& 1048& 524& 262& 131& 394\\
197& 592& 296& 148& 74& 37& 112& 56& 28& 14\\
7& 22& 11& 34& 17& 52& 26& 13& 40& 20\\
10& 5& 16& 8& 4& 2& 1& \\

1863&&&&&&&&&\\
5590& 2795& 8386& 4193& 12580& 6290& 3145& 9436& 4718& 2359\\
7078& 3539& 10618& 5309& 15928& 7964& 3982& 1991& 5974& 2987\\
8962& 4481& 13444& 6722& 3361& 10084& 5042& 2521& 7564& 3782\\
1891& 5674& 2837& 8512& 4256& 2128& 1064& 532& 266& 133\\
400& 200& 100& 50& 25& 76& 38& 19& 58& 29\\
88& 44& 22& 11& 34& 17& 52& 26& 13& 40\\
20& 10& 5& 16& 8& 4& 2& 1& \\

1864&&&&&&&&&\\
932& 466& 233& 700& 350& 175& 526& 263& 790& 395\\
1186& 593& 1780& 890& 445& 1336& 668& 334& 167& 502\\
251& 754& 377& 1132& 566& 283& 850& 425& 1276& 638\\
319& 958& 479& 1438& 719& 2158& 1079& 3238& 1619& 4858\\
2429& 7288& 3644& 1822& 911& 2734& 1367& 4102& 2051& 6154\\
3077& 9232& 4616& 2308& 1154& 577& 1732& 866& 433& 1300\\
650& 325& 976& 488& 244& 122& 61& 184& 92& 46\\
23& 70& 35& 106& 53& 160& 80& 40& 20& 10\\
5& 16& 8& 4& 2& 1& \\

1865&&&&&&&&&\\
5596& 2798& 1399& 4198& 2099& 6298& 3149& 9448& 4724& 2362\\
1181& 3544& 1772& 886& 443& 1330& 665& 1996& 998& 499\\
1498& 749& 2248& 1124& 562& 281& 844& 422& 211& 634\\
317& 952& 476& 238& 119& 358& 179& 538& 269& 808\\
404& 202& 101& 304& 152& 76& 38& 19& 58& 29\\
88& 44& 22& 11& 34& 17& 52& 26& 13& 40\\
20& 10& 5& 16& 8& 4& 2& 1& \\

1866&&&&&&&&&\\
933& 2800& 1400& 700& 350& 175& 526& 263& 790& 395\\
1186& 593& 1780& 890& 445& 1336& 668& 334& 167& 502\\
251& 754& 377& 1132& 566& 283& 850& 425& 1276& 638\\
319& 958& 479& 1438& 719& 2158& 1079& 3238& 1619& 4858\\
2429& 7288& 3644& 1822& 911& 2734& 1367& 4102& 2051& 6154\\
3077& 9232& 4616& 2308& 1154& 577& 1732& 866& 433& 1300\\
650& 325& 976& 488& 244& 122& 61& 184& 92& 46\\
23& 70& 35& 106& 53& 160& 80& 40& 20& 10\\
5& 16& 8& 4& 2& 1& \\

1867&&&&&&&&&\\
5602& 2801& 8404& 4202& 2101& 6304& 3152& 1576& 788& 394\\
197& 592& 296& 148& 74& 37& 112& 56& 28& 14\\
7& 22& 11& 34& 17& 52& 26& 13& 40& 20\\
10& 5& 16& 8& 4& 2& 1& \\

1868&&&&&&&&&\\
934& 467& 1402& 701& 2104& 1052& 526& 263& 790& 395\\
1186& 593& 1780& 890& 445& 1336& 668& 334& 167& 502\\
251& 754& 377& 1132& 566& 283& 850& 425& 1276& 638\\
319& 958& 479& 1438& 719& 2158& 1079& 3238& 1619& 4858\\
2429& 7288& 3644& 1822& 911& 2734& 1367& 4102& 2051& 6154\\
3077& 9232& 4616& 2308& 1154& 577& 1732& 866& 433& 1300\\
650& 325& 976& 488& 244& 122& 61& 184& 92& 46\\
23& 70& 35& 106& 53& 160& 80& 40& 20& 10\\
5& 16& 8& 4& 2& 1& \\

1869&&&&&&&&&\\
5608& 2804& 1402& 701& 2104& 1052& 526& 263& 790& 395\\
1186& 593& 1780& 890& 445& 1336& 668& 334& 167& 502\\
251& 754& 377& 1132& 566& 283& 850& 425& 1276& 638\\
319& 958& 479& 1438& 719& 2158& 1079& 3238& 1619& 4858\\
2429& 7288& 3644& 1822& 911& 2734& 1367& 4102& 2051& 6154\\
3077& 9232& 4616& 2308& 1154& 577& 1732& 866& 433& 1300\\
650& 325& 976& 488& 244& 122& 61& 184& 92& 46\\
23& 70& 35& 106& 53& 160& 80& 40& 20& 10\\
5& 16& 8& 4& 2& 1& \\

1870&&&&&&&&&\\
935& 2806& 1403& 4210& 2105& 6316& 3158& 1579& 4738& 2369\\
7108& 3554& 1777& 5332& 2666& 1333& 4000& 2000& 1000& 500\\
250& 125& 376& 188& 94& 47& 142& 71& 214& 107\\
322& 161& 484& 242& 121& 364& 182& 91& 274& 137\\
412& 206& 103& 310& 155& 466& 233& 700& 350& 175\\
526& 263& 790& 395& 1186& 593& 1780& 890& 445& 1336\\
668& 334& 167& 502& 251& 754& 377& 1132& 566& 283\\
850& 425& 1276& 638& 319& 958& 479& 1438& 719& 2158\\
1079& 3238& 1619& 4858& 2429& 7288& 3644& 1822& 911& 2734\\
1367& 4102& 2051& 6154& 3077& 9232& 4616& 2308& 1154& 577\\
1732& 866& 433& 1300& 650& 325& 976& 488& 244& 122\\
61& 184& 92& 46& 23& 70& 35& 106& 53& 160\\
80& 40& 20& 10& 5& 16& 8& 4& 2& 1\\

1871&&&&&&&&&\\
5614& 2807& 8422& 4211& 12634& 6317& 18952& 9476& 4738& 2369\\
7108& 3554& 1777& 5332& 2666& 1333& 4000& 2000& 1000& 500\\
250& 125& 376& 188& 94& 47& 142& 71& 214& 107\\
322& 161& 484& 242& 121& 364& 182& 91& 274& 137\\
412& 206& 103& 310& 155& 466& 233& 700& 350& 175\\
526& 263& 790& 395& 1186& 593& 1780& 890& 445& 1336\\
668& 334& 167& 502& 251& 754& 377& 1132& 566& 283\\
850& 425& 1276& 638& 319& 958& 479& 1438& 719& 2158\\
1079& 3238& 1619& 4858& 2429& 7288& 3644& 1822& 911& 2734\\
1367& 4102& 2051& 6154& 3077& 9232& 4616& 2308& 1154& 577\\
1732& 866& 433& 1300& 650& 325& 976& 488& 244& 122\\
61& 184& 92& 46& 23& 70& 35& 106& 53& 160\\
80& 40& 20& 10& 5& 16& 8& 4& 2& 1\\

1872&&&&&&&&&\\
936& 468& 234& 117& 352& 176& 88& 44& 22& 11\\
34& 17& 52& 26& 13& 40& 20& 10& 5& 16\\
8& 4& 2& 1& \\

1873&&&&&&&&&\\
5620& 2810& 1405& 4216& 2108& 1054& 527& 1582& 791& 2374\\
1187& 3562& 1781& 5344& 2672& 1336& 668& 334& 167& 502\\
251& 754& 377& 1132& 566& 283& 850& 425& 1276& 638\\
319& 958& 479& 1438& 719& 2158& 1079& 3238& 1619& 4858\\
2429& 7288& 3644& 1822& 911& 2734& 1367& 4102& 2051& 6154\\
3077& 9232& 4616& 2308& 1154& 577& 1732& 866& 433& 1300\\
650& 325& 976& 488& 244& 122& 61& 184& 92& 46\\
23& 70& 35& 106& 53& 160& 80& 40& 20& 10\\
5& 16& 8& 4& 2& 1& \\

1874&&&&&&&&&\\
937& 2812& 1406& 703& 2110& 1055& 3166& 1583& 4750& 2375\\
7126& 3563& 10690& 5345& 16036& 8018& 4009& 12028& 6014& 3007\\
9022& 4511& 13534& 6767& 20302& 10151& 30454& 15227& 45682& 22841\\
68524& 34262& 17131& 51394& 25697& 77092& 38546& 19273& 57820& 28910\\
14455& 43366& 21683& 65050& 32525& 97576& 48788& 24394& 12197& 36592\\
18296& 9148& 4574& 2287& 6862& 3431& 10294& 5147& 15442& 7721\\
23164& 11582& 5791& 17374& 8687& 26062& 13031& 39094& 19547& 58642\\
29321& 87964& 43982& 21991& 65974& 32987& 98962& 49481& 148444& 74222\\
37111& 111334& 55667& 167002& 83501& 250504& 125252& 62626& 31313& 93940\\
46970& 23485& 70456& 35228& 17614& 8807& 26422& 13211& 39634& 19817\\
59452& 29726& 14863& 44590& 22295& 66886& 33443& 100330& 50165& 150496\\
75248& 37624& 18812& 9406& 4703& 14110& 7055& 21166& 10583& 31750\\
15875& 47626& 23813& 71440& 35720& 17860& 8930& 4465& 13396& 6698\\
3349& 10048& 5024& 2512& 1256& 628& 314& 157& 472& 236\\
118& 59& 178& 89& 268& 134& 67& 202& 101& 304\\
152& 76& 38& 19& 58& 29& 88& 44& 22& 11\\
34& 17& 52& 26& 13& 40& 20& 10& 5& 16\\
8& 4& 2& 1& \\

1875&&&&&&&&&\\
5626& 2813& 8440& 4220& 2110& 1055& 3166& 1583& 4750& 2375\\
7126& 3563& 10690& 5345& 16036& 8018& 4009& 12028& 6014& 3007\\
9022& 4511& 13534& 6767& 20302& 10151& 30454& 15227& 45682& 22841\\
68524& 34262& 17131& 51394& 25697& 77092& 38546& 19273& 57820& 28910\\
14455& 43366& 21683& 65050& 32525& 97576& 48788& 24394& 12197& 36592\\
18296& 9148& 4574& 2287& 6862& 3431& 10294& 5147& 15442& 7721\\
23164& 11582& 5791& 17374& 8687& 26062& 13031& 39094& 19547& 58642\\
29321& 87964& 43982& 21991& 65974& 32987& 98962& 49481& 148444& 74222\\
37111& 111334& 55667& 167002& 83501& 250504& 125252& 62626& 31313& 93940\\
46970& 23485& 70456& 35228& 17614& 8807& 26422& 13211& 39634& 19817\\
59452& 29726& 14863& 44590& 22295& 66886& 33443& 100330& 50165& 150496\\
75248& 37624& 18812& 9406& 4703& 14110& 7055& 21166& 10583& 31750\\
15875& 47626& 23813& 71440& 35720& 17860& 8930& 4465& 13396& 6698\\
3349& 10048& 5024& 2512& 1256& 628& 314& 157& 472& 236\\
118& 59& 178& 89& 268& 134& 67& 202& 101& 304\\
152& 76& 38& 19& 58& 29& 88& 44& 22& 11\\
34& 17& 52& 26& 13& 40& 20& 10& 5& 16\\
8& 4& 2& 1& \\

1876&&&&&&&&&\\
938& 469& 1408& 704& 352& 176& 88& 44& 22& 11\\
34& 17& 52& 26& 13& 40& 20& 10& 5& 16\\
8& 4& 2& 1& \\

1877&&&&&&&&&\\
5632& 2816& 1408& 704& 352& 176& 88& 44& 22& 11\\
34& 17& 52& 26& 13& 40& 20& 10& 5& 16\\
8& 4& 2& 1& \\

1878&&&&&&&&&\\
939& 2818& 1409& 4228& 2114& 1057& 3172& 1586& 793& 2380\\
1190& 595& 1786& 893& 2680& 1340& 670& 335& 1006& 503\\
1510& 755& 2266& 1133& 3400& 1700& 850& 425& 1276& 638\\
319& 958& 479& 1438& 719& 2158& 1079& 3238& 1619& 4858\\
2429& 7288& 3644& 1822& 911& 2734& 1367& 4102& 2051& 6154\\
3077& 9232& 4616& 2308& 1154& 577& 1732& 866& 433& 1300\\
650& 325& 976& 488& 244& 122& 61& 184& 92& 46\\
23& 70& 35& 106& 53& 160& 80& 40& 20& 10\\
5& 16& 8& 4& 2& 1& \\

1879&&&&&&&&&\\
5638& 2819& 8458& 4229& 12688& 6344& 3172& 1586& 793& 2380\\
1190& 595& 1786& 893& 2680& 1340& 670& 335& 1006& 503\\
1510& 755& 2266& 1133& 3400& 1700& 850& 425& 1276& 638\\
319& 958& 479& 1438& 719& 2158& 1079& 3238& 1619& 4858\\
2429& 7288& 3644& 1822& 911& 2734& 1367& 4102& 2051& 6154\\
3077& 9232& 4616& 2308& 1154& 577& 1732& 866& 433& 1300\\
650& 325& 976& 488& 244& 122& 61& 184& 92& 46\\
23& 70& 35& 106& 53& 160& 80& 40& 20& 10\\
5& 16& 8& 4& 2& 1& \\

1880&&&&&&&&&\\
940& 470& 235& 706& 353& 1060& 530& 265& 796& 398\\
199& 598& 299& 898& 449& 1348& 674& 337& 1012& 506\\
253& 760& 380& 190& 95& 286& 143& 430& 215& 646\\
323& 970& 485& 1456& 728& 364& 182& 91& 274& 137\\
412& 206& 103& 310& 155& 466& 233& 700& 350& 175\\
526& 263& 790& 395& 1186& 593& 1780& 890& 445& 1336\\
668& 334& 167& 502& 251& 754& 377& 1132& 566& 283\\
850& 425& 1276& 638& 319& 958& 479& 1438& 719& 2158\\
1079& 3238& 1619& 4858& 2429& 7288& 3644& 1822& 911& 2734\\
1367& 4102& 2051& 6154& 3077& 9232& 4616& 2308& 1154& 577\\
1732& 866& 433& 1300& 650& 325& 976& 488& 244& 122\\
61& 184& 92& 46& 23& 70& 35& 106& 53& 160\\
80& 40& 20& 10& 5& 16& 8& 4& 2& 1\\

1881&&&&&&&&&\\
5644& 2822& 1411& 4234& 2117& 6352& 3176& 1588& 794& 397\\
1192& 596& 298& 149& 448& 224& 112& 56& 28& 14\\
7& 22& 11& 34& 17& 52& 26& 13& 40& 20\\
10& 5& 16& 8& 4& 2& 1& \\

1882&&&&&&&&&\\
941& 2824& 1412& 706& 353& 1060& 530& 265& 796& 398\\
199& 598& 299& 898& 449& 1348& 674& 337& 1012& 506\\
253& 760& 380& 190& 95& 286& 143& 430& 215& 646\\
323& 970& 485& 1456& 728& 364& 182& 91& 274& 137\\
412& 206& 103& 310& 155& 466& 233& 700& 350& 175\\
526& 263& 790& 395& 1186& 593& 1780& 890& 445& 1336\\
668& 334& 167& 502& 251& 754& 377& 1132& 566& 283\\
850& 425& 1276& 638& 319& 958& 479& 1438& 719& 2158\\
1079& 3238& 1619& 4858& 2429& 7288& 3644& 1822& 911& 2734\\
1367& 4102& 2051& 6154& 3077& 9232& 4616& 2308& 1154& 577\\
1732& 866& 433& 1300& 650& 325& 976& 488& 244& 122\\
61& 184& 92& 46& 23& 70& 35& 106& 53& 160\\
80& 40& 20& 10& 5& 16& 8& 4& 2& 1\\

1883&&&&&&&&&\\
5650& 2825& 8476& 4238& 2119& 6358& 3179& 9538& 4769& 14308\\
7154& 3577& 10732& 5366& 2683& 8050& 4025& 12076& 6038& 3019\\
9058& 4529& 13588& 6794& 3397& 10192& 5096& 2548& 1274& 637\\
1912& 956& 478& 239& 718& 359& 1078& 539& 1618& 809\\
2428& 1214& 607& 1822& 911& 2734& 1367& 4102& 2051& 6154\\
3077& 9232& 4616& 2308& 1154& 577& 1732& 866& 433& 1300\\
650& 325& 976& 488& 244& 122& 61& 184& 92& 46\\
23& 70& 35& 106& 53& 160& 80& 40& 20& 10\\
5& 16& 8& 4& 2& 1& \\

1884&&&&&&&&&\\
942& 471& 1414& 707& 2122& 1061& 3184& 1592& 796& 398\\
199& 598& 299& 898& 449& 1348& 674& 337& 1012& 506\\
253& 760& 380& 190& 95& 286& 143& 430& 215& 646\\
323& 970& 485& 1456& 728& 364& 182& 91& 274& 137\\
412& 206& 103& 310& 155& 466& 233& 700& 350& 175\\
526& 263& 790& 395& 1186& 593& 1780& 890& 445& 1336\\
668& 334& 167& 502& 251& 754& 377& 1132& 566& 283\\
850& 425& 1276& 638& 319& 958& 479& 1438& 719& 2158\\
1079& 3238& 1619& 4858& 2429& 7288& 3644& 1822& 911& 2734\\
1367& 4102& 2051& 6154& 3077& 9232& 4616& 2308& 1154& 577\\
1732& 866& 433& 1300& 650& 325& 976& 488& 244& 122\\
61& 184& 92& 46& 23& 70& 35& 106& 53& 160\\
80& 40& 20& 10& 5& 16& 8& 4& 2& 1\\

1885&&&&&&&&&\\
5656& 2828& 1414& 707& 2122& 1061& 3184& 1592& 796& 398\\
199& 598& 299& 898& 449& 1348& 674& 337& 1012& 506\\
253& 760& 380& 190& 95& 286& 143& 430& 215& 646\\
323& 970& 485& 1456& 728& 364& 182& 91& 274& 137\\
412& 206& 103& 310& 155& 466& 233& 700& 350& 175\\
526& 263& 790& 395& 1186& 593& 1780& 890& 445& 1336\\
668& 334& 167& 502& 251& 754& 377& 1132& 566& 283\\
850& 425& 1276& 638& 319& 958& 479& 1438& 719& 2158\\
1079& 3238& 1619& 4858& 2429& 7288& 3644& 1822& 911& 2734\\
1367& 4102& 2051& 6154& 3077& 9232& 4616& 2308& 1154& 577\\
1732& 866& 433& 1300& 650& 325& 976& 488& 244& 122\\
61& 184& 92& 46& 23& 70& 35& 106& 53& 160\\
80& 40& 20& 10& 5& 16& 8& 4& 2& 1\\

1886&&&&&&&&&\\
943& 2830& 1415& 4246& 2123& 6370& 3185& 9556& 4778& 2389\\
7168& 3584& 1792& 896& 448& 224& 112& 56& 28& 14\\
7& 22& 11& 34& 17& 52& 26& 13& 40& 20\\
10& 5& 16& 8& 4& 2& 1& \\

1887&&&&&&&&&\\
5662& 2831& 8494& 4247& 12742& 6371& 19114& 9557& 28672& 14336\\
7168& 3584& 1792& 896& 448& 224& 112& 56& 28& 14\\
7& 22& 11& 34& 17& 52& 26& 13& 40& 20\\
10& 5& 16& 8& 4& 2& 1& \\

1888&&&&&&&&&\\
944& 472& 236& 118& 59& 178& 89& 268& 134& 67\\
202& 101& 304& 152& 76& 38& 19& 58& 29& 88\\
44& 22& 11& 34& 17& 52& 26& 13& 40& 20\\
10& 5& 16& 8& 4& 2& 1& \\

1889&&&&&&&&&\\
5668& 2834& 1417& 4252& 2126& 1063& 3190& 1595& 4786& 2393\\
7180& 3590& 1795& 5386& 2693& 8080& 4040& 2020& 1010& 505\\
1516& 758& 379& 1138& 569& 1708& 854& 427& 1282& 641\\
1924& 962& 481& 1444& 722& 361& 1084& 542& 271& 814\\
407& 1222& 611& 1834& 917& 2752& 1376& 688& 344& 172\\
86& 43& 130& 65& 196& 98& 49& 148& 74& 37\\
112& 56& 28& 14& 7& 22& 11& 34& 17& 52\\
26& 13& 40& 20& 10& 5& 16& 8& 4& 2\\
1& \\

1890&&&&&&&&&\\
945& 2836& 1418& 709& 2128& 1064& 532& 266& 133& 400\\
200& 100& 50& 25& 76& 38& 19& 58& 29& 88\\
44& 22& 11& 34& 17& 52& 26& 13& 40& 20\\
10& 5& 16& 8& 4& 2& 1& \\

1891&&&&&&&&&\\
5674& 2837& 8512& 4256& 2128& 1064& 532& 266& 133& 400\\
200& 100& 50& 25& 76& 38& 19& 58& 29& 88\\
44& 22& 11& 34& 17& 52& 26& 13& 40& 20\\
10& 5& 16& 8& 4& 2& 1& \\

1892&&&&&&&&&\\
946& 473& 1420& 710& 355& 1066& 533& 1600& 800& 400\\
200& 100& 50& 25& 76& 38& 19& 58& 29& 88\\
44& 22& 11& 34& 17& 52& 26& 13& 40& 20\\
10& 5& 16& 8& 4& 2& 1& \\

1893&&&&&&&&&\\
5680& 2840& 1420& 710& 355& 1066& 533& 1600& 800& 400\\
200& 100& 50& 25& 76& 38& 19& 58& 29& 88\\
44& 22& 11& 34& 17& 52& 26& 13& 40& 20\\
10& 5& 16& 8& 4& 2& 1& \\

1894&&&&&&&&&\\
947& 2842& 1421& 4264& 2132& 1066& 533& 1600& 800& 400\\
200& 100& 50& 25& 76& 38& 19& 58& 29& 88\\
44& 22& 11& 34& 17& 52& 26& 13& 40& 20\\
10& 5& 16& 8& 4& 2& 1& \\

1895&&&&&&&&&\\
5686& 2843& 8530& 4265& 12796& 6398& 3199& 9598& 4799& 14398\\
7199& 21598& 10799& 32398& 16199& 48598& 24299& 72898& 36449& 109348\\
54674& 27337& 82012& 41006& 20503& 61510& 30755& 92266& 46133& 138400\\
69200& 34600& 17300& 8650& 4325& 12976& 6488& 3244& 1622& 811\\
2434& 1217& 3652& 1826& 913& 2740& 1370& 685& 2056& 1028\\
514& 257& 772& 386& 193& 580& 290& 145& 436& 218\\
109& 328& 164& 82& 41& 124& 62& 31& 94& 47\\
142& 71& 214& 107& 322& 161& 484& 242& 121& 364\\
182& 91& 274& 137& 412& 206& 103& 310& 155& 466\\
233& 700& 350& 175& 526& 263& 790& 395& 1186& 593\\
1780& 890& 445& 1336& 668& 334& 167& 502& 251& 754\\
377& 1132& 566& 283& 850& 425& 1276& 638& 319& 958\\
479& 1438& 719& 2158& 1079& 3238& 1619& 4858& 2429& 7288\\
3644& 1822& 911& 2734& 1367& 4102& 2051& 6154& 3077& 9232\\
4616& 2308& 1154& 577& 1732& 866& 433& 1300& 650& 325\\
976& 488& 244& 122& 61& 184& 92& 46& 23& 70\\
35& 106& 53& 160& 80& 40& 20& 10& 5& 16\\
8& 4& 2& 1& \\

1896&&&&&&&&&\\
948& 474& 237& 712& 356& 178& 89& 268& 134& 67\\
202& 101& 304& 152& 76& 38& 19& 58& 29& 88\\
44& 22& 11& 34& 17& 52& 26& 13& 40& 20\\
10& 5& 16& 8& 4& 2& 1& \\

1897&&&&&&&&&\\
5692& 2846& 1423& 4270& 2135& 6406& 3203& 9610& 4805& 14416\\
7208& 3604& 1802& 901& 2704& 1352& 676& 338& 169& 508\\
254& 127& 382& 191& 574& 287& 862& 431& 1294& 647\\
1942& 971& 2914& 1457& 4372& 2186& 1093& 3280& 1640& 820\\
410& 205& 616& 308& 154& 77& 232& 116& 58& 29\\
88& 44& 22& 11& 34& 17& 52& 26& 13& 40\\
20& 10& 5& 16& 8& 4& 2& 1& \\

1898&&&&&&&&&\\
949& 2848& 1424& 712& 356& 178& 89& 268& 134& 67\\
202& 101& 304& 152& 76& 38& 19& 58& 29& 88\\
44& 22& 11& 34& 17& 52& 26& 13& 40& 20\\
10& 5& 16& 8& 4& 2& 1& \\

1899&&&&&&&&&\\
5698& 2849& 8548& 4274& 2137& 6412& 3206& 1603& 4810& 2405\\
7216& 3608& 1804& 902& 451& 1354& 677& 2032& 1016& 508\\
254& 127& 382& 191& 574& 287& 862& 431& 1294& 647\\
1942& 971& 2914& 1457& 4372& 2186& 1093& 3280& 1640& 820\\
410& 205& 616& 308& 154& 77& 232& 116& 58& 29\\
88& 44& 22& 11& 34& 17& 52& 26& 13& 40\\
20& 10& 5& 16& 8& 4& 2& 1& \\

1900&&&&&&&&&\\
950& 475& 1426& 713& 2140& 1070& 535& 1606& 803& 2410\\
1205& 3616& 1808& 904& 452& 226& 113& 340& 170& 85\\
256& 128& 64& 32& 16& 8& 4& 2& 1& \\

1901&&&&&&&&&\\
5704& 2852& 1426& 713& 2140& 1070& 535& 1606& 803& 2410\\
1205& 3616& 1808& 904& 452& 226& 113& 340& 170& 85\\
256& 128& 64& 32& 16& 8& 4& 2& 1& \\

1902&&&&&&&&&\\
951& 2854& 1427& 4282& 2141& 6424& 3212& 1606& 803& 2410\\
1205& 3616& 1808& 904& 452& 226& 113& 340& 170& 85\\
256& 128& 64& 32& 16& 8& 4& 2& 1& \\

1903&&&&&&&&&\\
5710& 2855& 8566& 4283& 12850& 6425& 19276& 9638& 4819& 14458\\
7229& 21688& 10844& 5422& 2711& 8134& 4067& 12202& 6101& 18304\\
9152& 4576& 2288& 1144& 572& 286& 143& 430& 215& 646\\
323& 970& 485& 1456& 728& 364& 182& 91& 274& 137\\
412& 206& 103& 310& 155& 466& 233& 700& 350& 175\\
526& 263& 790& 395& 1186& 593& 1780& 890& 445& 1336\\
668& 334& 167& 502& 251& 754& 377& 1132& 566& 283\\
850& 425& 1276& 638& 319& 958& 479& 1438& 719& 2158\\
1079& 3238& 1619& 4858& 2429& 7288& 3644& 1822& 911& 2734\\
1367& 4102& 2051& 6154& 3077& 9232& 4616& 2308& 1154& 577\\
1732& 866& 433& 1300& 650& 325& 976& 488& 244& 122\\
61& 184& 92& 46& 23& 70& 35& 106& 53& 160\\
80& 40& 20& 10& 5& 16& 8& 4& 2& 1\\

1904&&&&&&&&&\\
952& 476& 238& 119& 358& 179& 538& 269& 808& 404\\
202& 101& 304& 152& 76& 38& 19& 58& 29& 88\\
44& 22& 11& 34& 17& 52& 26& 13& 40& 20\\
10& 5& 16& 8& 4& 2& 1& \\

1905&&&&&&&&&\\
5716& 2858& 1429& 4288& 2144& 1072& 536& 268& 134& 67\\
202& 101& 304& 152& 76& 38& 19& 58& 29& 88\\
44& 22& 11& 34& 17& 52& 26& 13& 40& 20\\
10& 5& 16& 8& 4& 2& 1& \\

1906&&&&&&&&&\\
953& 2860& 1430& 715& 2146& 1073& 3220& 1610& 805& 2416\\
1208& 604& 302& 151& 454& 227& 682& 341& 1024& 512\\
256& 128& 64& 32& 16& 8& 4& 2& 1& \\

1907&&&&&&&&&\\
5722& 2861& 8584& 4292& 2146& 1073& 3220& 1610& 805& 2416\\
1208& 604& 302& 151& 454& 227& 682& 341& 1024& 512\\
256& 128& 64& 32& 16& 8& 4& 2& 1& \\

1908&&&&&&&&&\\
954& 477& 1432& 716& 358& 179& 538& 269& 808& 404\\
202& 101& 304& 152& 76& 38& 19& 58& 29& 88\\
44& 22& 11& 34& 17& 52& 26& 13& 40& 20\\
10& 5& 16& 8& 4& 2& 1& \\

1909&&&&&&&&&\\
5728& 2864& 1432& 716& 358& 179& 538& 269& 808& 404\\
202& 101& 304& 152& 76& 38& 19& 58& 29& 88\\
44& 22& 11& 34& 17& 52& 26& 13& 40& 20\\
10& 5& 16& 8& 4& 2& 1& \\

1910&&&&&&&&&\\
955& 2866& 1433& 4300& 2150& 1075& 3226& 1613& 4840& 2420\\
1210& 605& 1816& 908& 454& 227& 682& 341& 1024& 512\\
256& 128& 64& 32& 16& 8& 4& 2& 1& \\

1911&&&&&&&&&\\
5734& 2867& 8602& 4301& 12904& 6452& 3226& 1613& 4840& 2420\\
1210& 605& 1816& 908& 454& 227& 682& 341& 1024& 512\\
256& 128& 64& 32& 16& 8& 4& 2& 1& \\

1912&&&&&&&&&\\
956& 478& 239& 718& 359& 1078& 539& 1618& 809& 2428\\
1214& 607& 1822& 911& 2734& 1367& 4102& 2051& 6154& 3077\\
9232& 4616& 2308& 1154& 577& 1732& 866& 433& 1300& 650\\
325& 976& 488& 244& 122& 61& 184& 92& 46& 23\\
70& 35& 106& 53& 160& 80& 40& 20& 10& 5\\
16& 8& 4& 2& 1& \\

1913&&&&&&&&&\\
5740& 2870& 1435& 4306& 2153& 6460& 3230& 1615& 4846& 2423\\
7270& 3635& 10906& 5453& 16360& 8180& 4090& 2045& 6136& 3068\\
1534& 767& 2302& 1151& 3454& 1727& 5182& 2591& 7774& 3887\\
11662& 5831& 17494& 8747& 26242& 13121& 39364& 19682& 9841& 29524\\
14762& 7381& 22144& 11072& 5536& 2768& 1384& 692& 346& 173\\
520& 260& 130& 65& 196& 98& 49& 148& 74& 37\\
112& 56& 28& 14& 7& 22& 11& 34& 17& 52\\
26& 13& 40& 20& 10& 5& 16& 8& 4& 2\\
1& \\

1914&&&&&&&&&\\
957& 2872& 1436& 718& 359& 1078& 539& 1618& 809& 2428\\
1214& 607& 1822& 911& 2734& 1367& 4102& 2051& 6154& 3077\\
9232& 4616& 2308& 1154& 577& 1732& 866& 433& 1300& 650\\
325& 976& 488& 244& 122& 61& 184& 92& 46& 23\\
70& 35& 106& 53& 160& 80& 40& 20& 10& 5\\
16& 8& 4& 2& 1& \\

1915&&&&&&&&&\\
5746& 2873& 8620& 4310& 2155& 6466& 3233& 9700& 4850& 2425\\
7276& 3638& 1819& 5458& 2729& 8188& 4094& 2047& 6142& 3071\\
9214& 4607& 13822& 6911& 20734& 10367& 31102& 15551& 46654& 23327\\
69982& 34991& 104974& 52487& 157462& 78731& 236194& 118097& 354292& 177146\\
88573& 265720& 132860& 66430& 33215& 99646& 49823& 149470& 74735& 224206\\
112103& 336310& 168155& 504466& 252233& 756700& 378350& 189175& 567526& 283763\\
851290& 425645& 1276936& 638468& 319234& 159617& 478852& 239426& 119713& 359140\\
179570& 89785& 269356& 134678& 67339& 202018& 101009& 303028& 151514& 75757\\
227272& 113636& 56818& 28409& 85228& 42614& 21307& 63922& 31961& 95884\\
47942& 23971& 71914& 35957& 107872& 53936& 26968& 13484& 6742& 3371\\
10114& 5057& 15172& 7586& 3793& 11380& 5690& 2845& 8536& 4268\\
2134& 1067& 3202& 1601& 4804& 2402& 1201& 3604& 1802& 901\\
2704& 1352& 676& 338& 169& 508& 254& 127& 382& 191\\
574& 287& 862& 431& 1294& 647& 1942& 971& 2914& 1457\\
4372& 2186& 1093& 3280& 1640& 820& 410& 205& 616& 308\\
154& 77& 232& 116& 58& 29& 88& 44& 22& 11\\
34& 17& 52& 26& 13& 40& 20& 10& 5& 16\\
8& 4& 2& 1& \\

1916&&&&&&&&&\\
958& 479& 1438& 719& 2158& 1079& 3238& 1619& 4858& 2429\\
7288& 3644& 1822& 911& 2734& 1367& 4102& 2051& 6154& 3077\\
9232& 4616& 2308& 1154& 577& 1732& 866& 433& 1300& 650\\
325& 976& 488& 244& 122& 61& 184& 92& 46& 23\\
70& 35& 106& 53& 160& 80& 40& 20& 10& 5\\
16& 8& 4& 2& 1& \\

1917&&&&&&&&&\\
5752& 2876& 1438& 719& 2158& 1079& 3238& 1619& 4858& 2429\\
7288& 3644& 1822& 911& 2734& 1367& 4102& 2051& 6154& 3077\\
9232& 4616& 2308& 1154& 577& 1732& 866& 433& 1300& 650\\
325& 976& 488& 244& 122& 61& 184& 92& 46& 23\\
70& 35& 106& 53& 160& 80& 40& 20& 10& 5\\
16& 8& 4& 2& 1& \\

1918&&&&&&&&&\\
959& 2878& 1439& 4318& 2159& 6478& 3239& 9718& 4859& 14578\\
7289& 21868& 10934& 5467& 16402& 8201& 24604& 12302& 6151& 18454\\
9227& 27682& 13841& 41524& 20762& 10381& 31144& 15572& 7786& 3893\\
11680& 5840& 2920& 1460& 730& 365& 1096& 548& 274& 137\\
412& 206& 103& 310& 155& 466& 233& 700& 350& 175\\
526& 263& 790& 395& 1186& 593& 1780& 890& 445& 1336\\
668& 334& 167& 502& 251& 754& 377& 1132& 566& 283\\
850& 425& 1276& 638& 319& 958& 479& 1438& 719& 2158\\
1079& 3238& 1619& 4858& 2429& 7288& 3644& 1822& 911& 2734\\
1367& 4102& 2051& 6154& 3077& 9232& 4616& 2308& 1154& 577\\
1732& 866& 433& 1300& 650& 325& 976& 488& 244& 122\\
61& 184& 92& 46& 23& 70& 35& 106& 53& 160\\
80& 40& 20& 10& 5& 16& 8& 4& 2& 1\\

1919&&&&&&&&&\\
5758& 2879& 8638& 4319& 12958& 6479& 19438& 9719& 29158& 14579\\
43738& 21869& 65608& 32804& 16402& 8201& 24604& 12302& 6151& 18454\\
9227& 27682& 13841& 41524& 20762& 10381& 31144& 15572& 7786& 3893\\
11680& 5840& 2920& 1460& 730& 365& 1096& 548& 274& 137\\
412& 206& 103& 310& 155& 466& 233& 700& 350& 175\\
526& 263& 790& 395& 1186& 593& 1780& 890& 445& 1336\\
668& 334& 167& 502& 251& 754& 377& 1132& 566& 283\\
850& 425& 1276& 638& 319& 958& 479& 1438& 719& 2158\\
1079& 3238& 1619& 4858& 2429& 7288& 3644& 1822& 911& 2734\\
1367& 4102& 2051& 6154& 3077& 9232& 4616& 2308& 1154& 577\\
1732& 866& 433& 1300& 650& 325& 976& 488& 244& 122\\
61& 184& 92& 46& 23& 70& 35& 106& 53& 160\\
80& 40& 20& 10& 5& 16& 8& 4& 2& 1\\

1920&&&&&&&&&\\
960& 480& 240& 120& 60& 30& 15& 46& 23& 70\\
35& 106& 53& 160& 80& 40& 20& 10& 5& 16\\
8& 4& 2& 1& \\

1921&&&&&&&&&\\
5764& 2882& 1441& 4324& 2162& 1081& 3244& 1622& 811& 2434\\
1217& 3652& 1826& 913& 2740& 1370& 685& 2056& 1028& 514\\
257& 772& 386& 193& 580& 290& 145& 436& 218& 109\\
328& 164& 82& 41& 124& 62& 31& 94& 47& 142\\
71& 214& 107& 322& 161& 484& 242& 121& 364& 182\\
91& 274& 137& 412& 206& 103& 310& 155& 466& 233\\
700& 350& 175& 526& 263& 790& 395& 1186& 593& 1780\\
890& 445& 1336& 668& 334& 167& 502& 251& 754& 377\\
1132& 566& 283& 850& 425& 1276& 638& 319& 958& 479\\
1438& 719& 2158& 1079& 3238& 1619& 4858& 2429& 7288& 3644\\
1822& 911& 2734& 1367& 4102& 2051& 6154& 3077& 9232& 4616\\
2308& 1154& 577& 1732& 866& 433& 1300& 650& 325& 976\\
488& 244& 122& 61& 184& 92& 46& 23& 70& 35\\
106& 53& 160& 80& 40& 20& 10& 5& 16& 8\\
4& 2& 1& \\

1922&&&&&&&&&\\
961& 2884& 1442& 721& 2164& 1082& 541& 1624& 812& 406\\
203& 610& 305& 916& 458& 229& 688& 344& 172& 86\\
43& 130& 65& 196& 98& 49& 148& 74& 37& 112\\
56& 28& 14& 7& 22& 11& 34& 17& 52& 26\\
13& 40& 20& 10& 5& 16& 8& 4& 2& 1\\

1923&&&&&&&&&\\
5770& 2885& 8656& 4328& 2164& 1082& 541& 1624& 812& 406\\
203& 610& 305& 916& 458& 229& 688& 344& 172& 86\\
43& 130& 65& 196& 98& 49& 148& 74& 37& 112\\
56& 28& 14& 7& 22& 11& 34& 17& 52& 26\\
13& 40& 20& 10& 5& 16& 8& 4& 2& 1\\

1924&&&&&&&&&\\
962& 481& 1444& 722& 361& 1084& 542& 271& 814& 407\\
1222& 611& 1834& 917& 2752& 1376& 688& 344& 172& 86\\
43& 130& 65& 196& 98& 49& 148& 74& 37& 112\\
56& 28& 14& 7& 22& 11& 34& 17& 52& 26\\
13& 40& 20& 10& 5& 16& 8& 4& 2& 1\\

1925&&&&&&&&&\\
5776& 2888& 1444& 722& 361& 1084& 542& 271& 814& 407\\
1222& 611& 1834& 917& 2752& 1376& 688& 344& 172& 86\\
43& 130& 65& 196& 98& 49& 148& 74& 37& 112\\
56& 28& 14& 7& 22& 11& 34& 17& 52& 26\\
13& 40& 20& 10& 5& 16& 8& 4& 2& 1\\

1926&&&&&&&&&\\
963& 2890& 1445& 4336& 2168& 1084& 542& 271& 814& 407\\
1222& 611& 1834& 917& 2752& 1376& 688& 344& 172& 86\\
43& 130& 65& 196& 98& 49& 148& 74& 37& 112\\
56& 28& 14& 7& 22& 11& 34& 17& 52& 26\\
13& 40& 20& 10& 5& 16& 8& 4& 2& 1\\

1927&&&&&&&&&\\
5782& 2891& 8674& 4337& 13012& 6506& 3253& 9760& 4880& 2440\\
1220& 610& 305& 916& 458& 229& 688& 344& 172& 86\\
43& 130& 65& 196& 98& 49& 148& 74& 37& 112\\
56& 28& 14& 7& 22& 11& 34& 17& 52& 26\\
13& 40& 20& 10& 5& 16& 8& 4& 2& 1\\

1928&&&&&&&&&\\
964& 482& 241& 724& 362& 181& 544& 272& 136& 68\\
34& 17& 52& 26& 13& 40& 20& 10& 5& 16\\
8& 4& 2& 1& \\

1929&&&&&&&&&\\
5788& 2894& 1447& 4342& 2171& 6514& 3257& 9772& 4886& 2443\\
7330& 3665& 10996& 5498& 2749& 8248& 4124& 2062& 1031& 3094\\
1547& 4642& 2321& 6964& 3482& 1741& 5224& 2612& 1306& 653\\
1960& 980& 490& 245& 736& 368& 184& 92& 46& 23\\
70& 35& 106& 53& 160& 80& 40& 20& 10& 5\\
16& 8& 4& 2& 1& \\

1930&&&&&&&&&\\
965& 2896& 1448& 724& 362& 181& 544& 272& 136& 68\\
34& 17& 52& 26& 13& 40& 20& 10& 5& 16\\
8& 4& 2& 1& \\

1931&&&&&&&&&\\
5794& 2897& 8692& 4346& 2173& 6520& 3260& 1630& 815& 2446\\
1223& 3670& 1835& 5506& 2753& 8260& 4130& 2065& 6196& 3098\\
1549& 4648& 2324& 1162& 581& 1744& 872& 436& 218& 109\\
328& 164& 82& 41& 124& 62& 31& 94& 47& 142\\
71& 214& 107& 322& 161& 484& 242& 121& 364& 182\\
91& 274& 137& 412& 206& 103& 310& 155& 466& 233\\
700& 350& 175& 526& 263& 790& 395& 1186& 593& 1780\\
890& 445& 1336& 668& 334& 167& 502& 251& 754& 377\\
1132& 566& 283& 850& 425& 1276& 638& 319& 958& 479\\
1438& 719& 2158& 1079& 3238& 1619& 4858& 2429& 7288& 3644\\
1822& 911& 2734& 1367& 4102& 2051& 6154& 3077& 9232& 4616\\
2308& 1154& 577& 1732& 866& 433& 1300& 650& 325& 976\\
488& 244& 122& 61& 184& 92& 46& 23& 70& 35\\
106& 53& 160& 80& 40& 20& 10& 5& 16& 8\\
4& 2& 1& \\

1932&&&&&&&&&\\
966& 483& 1450& 725& 2176& 1088& 544& 272& 136& 68\\
34& 17& 52& 26& 13& 40& 20& 10& 5& 16\\
8& 4& 2& 1& \\

1933&&&&&&&&&\\
5800& 2900& 1450& 725& 2176& 1088& 544& 272& 136& 68\\
34& 17& 52& 26& 13& 40& 20& 10& 5& 16\\
8& 4& 2& 1& \\

1934&&&&&&&&&\\
967& 2902& 1451& 4354& 2177& 6532& 3266& 1633& 4900& 2450\\
1225& 3676& 1838& 919& 2758& 1379& 4138& 2069& 6208& 3104\\
1552& 776& 388& 194& 97& 292& 146& 73& 220& 110\\
55& 166& 83& 250& 125& 376& 188& 94& 47& 142\\
71& 214& 107& 322& 161& 484& 242& 121& 364& 182\\
91& 274& 137& 412& 206& 103& 310& 155& 466& 233\\
700& 350& 175& 526& 263& 790& 395& 1186& 593& 1780\\
890& 445& 1336& 668& 334& 167& 502& 251& 754& 377\\
1132& 566& 283& 850& 425& 1276& 638& 319& 958& 479\\
1438& 719& 2158& 1079& 3238& 1619& 4858& 2429& 7288& 3644\\
1822& 911& 2734& 1367& 4102& 2051& 6154& 3077& 9232& 4616\\
2308& 1154& 577& 1732& 866& 433& 1300& 650& 325& 976\\
488& 244& 122& 61& 184& 92& 46& 23& 70& 35\\
106& 53& 160& 80& 40& 20& 10& 5& 16& 8\\
4& 2& 1& \\

1935&&&&&&&&&\\
5806& 2903& 8710& 4355& 13066& 6533& 19600& 9800& 4900& 2450\\
1225& 3676& 1838& 919& 2758& 1379& 4138& 2069& 6208& 3104\\
1552& 776& 388& 194& 97& 292& 146& 73& 220& 110\\
55& 166& 83& 250& 125& 376& 188& 94& 47& 142\\
71& 214& 107& 322& 161& 484& 242& 121& 364& 182\\
91& 274& 137& 412& 206& 103& 310& 155& 466& 233\\
700& 350& 175& 526& 263& 790& 395& 1186& 593& 1780\\
890& 445& 1336& 668& 334& 167& 502& 251& 754& 377\\
1132& 566& 283& 850& 425& 1276& 638& 319& 958& 479\\
1438& 719& 2158& 1079& 3238& 1619& 4858& 2429& 7288& 3644\\
1822& 911& 2734& 1367& 4102& 2051& 6154& 3077& 9232& 4616\\
2308& 1154& 577& 1732& 866& 433& 1300& 650& 325& 976\\
488& 244& 122& 61& 184& 92& 46& 23& 70& 35\\
106& 53& 160& 80& 40& 20& 10& 5& 16& 8\\
4& 2& 1& \\

1936&&&&&&&&&\\
968& 484& 242& 121& 364& 182& 91& 274& 137& 412\\
206& 103& 310& 155& 466& 233& 700& 350& 175& 526\\
263& 790& 395& 1186& 593& 1780& 890& 445& 1336& 668\\
334& 167& 502& 251& 754& 377& 1132& 566& 283& 850\\
425& 1276& 638& 319& 958& 479& 1438& 719& 2158& 1079\\
3238& 1619& 4858& 2429& 7288& 3644& 1822& 911& 2734& 1367\\
4102& 2051& 6154& 3077& 9232& 4616& 2308& 1154& 577& 1732\\
866& 433& 1300& 650& 325& 976& 488& 244& 122& 61\\
184& 92& 46& 23& 70& 35& 106& 53& 160& 80\\
40& 20& 10& 5& 16& 8& 4& 2& 1& \\

1937&&&&&&&&&\\
5812& 2906& 1453& 4360& 2180& 1090& 545& 1636& 818& 409\\
1228& 614& 307& 922& 461& 1384& 692& 346& 173& 520\\
260& 130& 65& 196& 98& 49& 148& 74& 37& 112\\
56& 28& 14& 7& 22& 11& 34& 17& 52& 26\\
13& 40& 20& 10& 5& 16& 8& 4& 2& 1\\

1938&&&&&&&&&\\
969& 2908& 1454& 727& 2182& 1091& 3274& 1637& 4912& 2456\\
1228& 614& 307& 922& 461& 1384& 692& 346& 173& 520\\
260& 130& 65& 196& 98& 49& 148& 74& 37& 112\\
56& 28& 14& 7& 22& 11& 34& 17& 52& 26\\
13& 40& 20& 10& 5& 16& 8& 4& 2& 1\\

1939&&&&&&&&&\\
5818& 2909& 8728& 4364& 2182& 1091& 3274& 1637& 4912& 2456\\
1228& 614& 307& 922& 461& 1384& 692& 346& 173& 520\\
260& 130& 65& 196& 98& 49& 148& 74& 37& 112\\
56& 28& 14& 7& 22& 11& 34& 17& 52& 26\\
13& 40& 20& 10& 5& 16& 8& 4& 2& 1\\

1940&&&&&&&&&\\
970& 485& 1456& 728& 364& 182& 91& 274& 137& 412\\
206& 103& 310& 155& 466& 233& 700& 350& 175& 526\\
263& 790& 395& 1186& 593& 1780& 890& 445& 1336& 668\\
334& 167& 502& 251& 754& 377& 1132& 566& 283& 850\\
425& 1276& 638& 319& 958& 479& 1438& 719& 2158& 1079\\
3238& 1619& 4858& 2429& 7288& 3644& 1822& 911& 2734& 1367\\
4102& 2051& 6154& 3077& 9232& 4616& 2308& 1154& 577& 1732\\
866& 433& 1300& 650& 325& 976& 488& 244& 122& 61\\
184& 92& 46& 23& 70& 35& 106& 53& 160& 80\\
40& 20& 10& 5& 16& 8& 4& 2& 1& \\

1941&&&&&&&&&\\
5824& 2912& 1456& 728& 364& 182& 91& 274& 137& 412\\
206& 103& 310& 155& 466& 233& 700& 350& 175& 526\\
263& 790& 395& 1186& 593& 1780& 890& 445& 1336& 668\\
334& 167& 502& 251& 754& 377& 1132& 566& 283& 850\\
425& 1276& 638& 319& 958& 479& 1438& 719& 2158& 1079\\
3238& 1619& 4858& 2429& 7288& 3644& 1822& 911& 2734& 1367\\
4102& 2051& 6154& 3077& 9232& 4616& 2308& 1154& 577& 1732\\
866& 433& 1300& 650& 325& 976& 488& 244& 122& 61\\
184& 92& 46& 23& 70& 35& 106& 53& 160& 80\\
40& 20& 10& 5& 16& 8& 4& 2& 1& \\

1942&&&&&&&&&\\
971& 2914& 1457& 4372& 2186& 1093& 3280& 1640& 820& 410\\
205& 616& 308& 154& 77& 232& 116& 58& 29& 88\\
44& 22& 11& 34& 17& 52& 26& 13& 40& 20\\
10& 5& 16& 8& 4& 2& 1& \\

1943&&&&&&&&&\\
5830& 2915& 8746& 4373& 13120& 6560& 3280& 1640& 820& 410\\
205& 616& 308& 154& 77& 232& 116& 58& 29& 88\\
44& 22& 11& 34& 17& 52& 26& 13& 40& 20\\
10& 5& 16& 8& 4& 2& 1& \\

1944&&&&&&&&&\\
972& 486& 243& 730& 365& 1096& 548& 274& 137& 412\\
206& 103& 310& 155& 466& 233& 700& 350& 175& 526\\
263& 790& 395& 1186& 593& 1780& 890& 445& 1336& 668\\
334& 167& 502& 251& 754& 377& 1132& 566& 283& 850\\
425& 1276& 638& 319& 958& 479& 1438& 719& 2158& 1079\\
3238& 1619& 4858& 2429& 7288& 3644& 1822& 911& 2734& 1367\\
4102& 2051& 6154& 3077& 9232& 4616& 2308& 1154& 577& 1732\\
866& 433& 1300& 650& 325& 976& 488& 244& 122& 61\\
184& 92& 46& 23& 70& 35& 106& 53& 160& 80\\
40& 20& 10& 5& 16& 8& 4& 2& 1& \\

1945&&&&&&&&&\\
5836& 2918& 1459& 4378& 2189& 6568& 3284& 1642& 821& 2464\\
1232& 616& 308& 154& 77& 232& 116& 58& 29& 88\\
44& 22& 11& 34& 17& 52& 26& 13& 40& 20\\
10& 5& 16& 8& 4& 2& 1& \\

1946&&&&&&&&&\\
973& 2920& 1460& 730& 365& 1096& 548& 274& 137& 412\\
206& 103& 310& 155& 466& 233& 700& 350& 175& 526\\
263& 790& 395& 1186& 593& 1780& 890& 445& 1336& 668\\
334& 167& 502& 251& 754& 377& 1132& 566& 283& 850\\
425& 1276& 638& 319& 958& 479& 1438& 719& 2158& 1079\\
3238& 1619& 4858& 2429& 7288& 3644& 1822& 911& 2734& 1367\\
4102& 2051& 6154& 3077& 9232& 4616& 2308& 1154& 577& 1732\\
866& 433& 1300& 650& 325& 976& 488& 244& 122& 61\\
184& 92& 46& 23& 70& 35& 106& 53& 160& 80\\
40& 20& 10& 5& 16& 8& 4& 2& 1& \\

1947&&&&&&&&&\\
5842& 2921& 8764& 4382& 2191& 6574& 3287& 9862& 4931& 14794\\
7397& 22192& 11096& 5548& 2774& 1387& 4162& 2081& 6244& 3122\\
1561& 4684& 2342& 1171& 3514& 1757& 5272& 2636& 1318& 659\\
1978& 989& 2968& 1484& 742& 371& 1114& 557& 1672& 836\\
418& 209& 628& 314& 157& 472& 236& 118& 59& 178\\
89& 268& 134& 67& 202& 101& 304& 152& 76& 38\\
19& 58& 29& 88& 44& 22& 11& 34& 17& 52\\
26& 13& 40& 20& 10& 5& 16& 8& 4& 2\\
1& \\

1948&&&&&&&&&\\
974& 487& 1462& 731& 2194& 1097& 3292& 1646& 823& 2470\\
1235& 3706& 1853& 5560& 2780& 1390& 695& 2086& 1043& 3130\\
1565& 4696& 2348& 1174& 587& 1762& 881& 2644& 1322& 661\\
1984& 992& 496& 248& 124& 62& 31& 94& 47& 142\\
71& 214& 107& 322& 161& 484& 242& 121& 364& 182\\
91& 274& 137& 412& 206& 103& 310& 155& 466& 233\\
700& 350& 175& 526& 263& 790& 395& 1186& 593& 1780\\
890& 445& 1336& 668& 334& 167& 502& 251& 754& 377\\
1132& 566& 283& 850& 425& 1276& 638& 319& 958& 479\\
1438& 719& 2158& 1079& 3238& 1619& 4858& 2429& 7288& 3644\\
1822& 911& 2734& 1367& 4102& 2051& 6154& 3077& 9232& 4616\\
2308& 1154& 577& 1732& 866& 433& 1300& 650& 325& 976\\
488& 244& 122& 61& 184& 92& 46& 23& 70& 35\\
106& 53& 160& 80& 40& 20& 10& 5& 16& 8\\
4& 2& 1& \\

1949&&&&&&&&&\\
5848& 2924& 1462& 731& 2194& 1097& 3292& 1646& 823& 2470\\
1235& 3706& 1853& 5560& 2780& 1390& 695& 2086& 1043& 3130\\
1565& 4696& 2348& 1174& 587& 1762& 881& 2644& 1322& 661\\
1984& 992& 496& 248& 124& 62& 31& 94& 47& 142\\
71& 214& 107& 322& 161& 484& 242& 121& 364& 182\\
91& 274& 137& 412& 206& 103& 310& 155& 466& 233\\
700& 350& 175& 526& 263& 790& 395& 1186& 593& 1780\\
890& 445& 1336& 668& 334& 167& 502& 251& 754& 377\\
1132& 566& 283& 850& 425& 1276& 638& 319& 958& 479\\
1438& 719& 2158& 1079& 3238& 1619& 4858& 2429& 7288& 3644\\
1822& 911& 2734& 1367& 4102& 2051& 6154& 3077& 9232& 4616\\
2308& 1154& 577& 1732& 866& 433& 1300& 650& 325& 976\\
488& 244& 122& 61& 184& 92& 46& 23& 70& 35\\
106& 53& 160& 80& 40& 20& 10& 5& 16& 8\\
4& 2& 1& \\

1950&&&&&&&&&\\
975& 2926& 1463& 4390& 2195& 6586& 3293& 9880& 4940& 2470\\
1235& 3706& 1853& 5560& 2780& 1390& 695& 2086& 1043& 3130\\
1565& 4696& 2348& 1174& 587& 1762& 881& 2644& 1322& 661\\
1984& 992& 496& 248& 124& 62& 31& 94& 47& 142\\
71& 214& 107& 322& 161& 484& 242& 121& 364& 182\\
91& 274& 137& 412& 206& 103& 310& 155& 466& 233\\
700& 350& 175& 526& 263& 790& 395& 1186& 593& 1780\\
890& 445& 1336& 668& 334& 167& 502& 251& 754& 377\\
1132& 566& 283& 850& 425& 1276& 638& 319& 958& 479\\
1438& 719& 2158& 1079& 3238& 1619& 4858& 2429& 7288& 3644\\
1822& 911& 2734& 1367& 4102& 2051& 6154& 3077& 9232& 4616\\
2308& 1154& 577& 1732& 866& 433& 1300& 650& 325& 976\\
488& 244& 122& 61& 184& 92& 46& 23& 70& 35\\
106& 53& 160& 80& 40& 20& 10& 5& 16& 8\\
4& 2& 1& \\

1951&&&&&&&&&\\
5854& 2927& 8782& 4391& 13174& 6587& 19762& 9881& 29644& 14822\\
7411& 22234& 11117& 33352& 16676& 8338& 4169& 12508& 6254& 3127\\
9382& 4691& 14074& 7037& 21112& 10556& 5278& 2639& 7918& 3959\\
11878& 5939& 17818& 8909& 26728& 13364& 6682& 3341& 10024& 5012\\
2506& 1253& 3760& 1880& 940& 470& 235& 706& 353& 1060\\
530& 265& 796& 398& 199& 598& 299& 898& 449& 1348\\
674& 337& 1012& 506& 253& 760& 380& 190& 95& 286\\
143& 430& 215& 646& 323& 970& 485& 1456& 728& 364\\
182& 91& 274& 137& 412& 206& 103& 310& 155& 466\\
233& 700& 350& 175& 526& 263& 790& 395& 1186& 593\\
1780& 890& 445& 1336& 668& 334& 167& 502& 251& 754\\
377& 1132& 566& 283& 850& 425& 1276& 638& 319& 958\\
479& 1438& 719& 2158& 1079& 3238& 1619& 4858& 2429& 7288\\
3644& 1822& 911& 2734& 1367& 4102& 2051& 6154& 3077& 9232\\
4616& 2308& 1154& 577& 1732& 866& 433& 1300& 650& 325\\
976& 488& 244& 122& 61& 184& 92& 46& 23& 70\\
35& 106& 53& 160& 80& 40& 20& 10& 5& 16\\
8& 4& 2& 1& \\

1952&&&&&&&&&\\
976& 488& 244& 122& 61& 184& 92& 46& 23& 70\\
35& 106& 53& 160& 80& 40& 20& 10& 5& 16\\
8& 4& 2& 1& \\

1953&&&&&&&&&\\
5860& 2930& 1465& 4396& 2198& 1099& 3298& 1649& 4948& 2474\\
1237& 3712& 1856& 928& 464& 232& 116& 58& 29& 88\\
44& 22& 11& 34& 17& 52& 26& 13& 40& 20\\
10& 5& 16& 8& 4& 2& 1& \\

1954&&&&&&&&&\\
977& 2932& 1466& 733& 2200& 1100& 550& 275& 826& 413\\
1240& 620& 310& 155& 466& 233& 700& 350& 175& 526\\
263& 790& 395& 1186& 593& 1780& 890& 445& 1336& 668\\
334& 167& 502& 251& 754& 377& 1132& 566& 283& 850\\
425& 1276& 638& 319& 958& 479& 1438& 719& 2158& 1079\\
3238& 1619& 4858& 2429& 7288& 3644& 1822& 911& 2734& 1367\\
4102& 2051& 6154& 3077& 9232& 4616& 2308& 1154& 577& 1732\\
866& 433& 1300& 650& 325& 976& 488& 244& 122& 61\\
184& 92& 46& 23& 70& 35& 106& 53& 160& 80\\
40& 20& 10& 5& 16& 8& 4& 2& 1& \\

1955&&&&&&&&&\\
5866& 2933& 8800& 4400& 2200& 1100& 550& 275& 826& 413\\
1240& 620& 310& 155& 466& 233& 700& 350& 175& 526\\
263& 790& 395& 1186& 593& 1780& 890& 445& 1336& 668\\
334& 167& 502& 251& 754& 377& 1132& 566& 283& 850\\
425& 1276& 638& 319& 958& 479& 1438& 719& 2158& 1079\\
3238& 1619& 4858& 2429& 7288& 3644& 1822& 911& 2734& 1367\\
4102& 2051& 6154& 3077& 9232& 4616& 2308& 1154& 577& 1732\\
866& 433& 1300& 650& 325& 976& 488& 244& 122& 61\\
184& 92& 46& 23& 70& 35& 106& 53& 160& 80\\
40& 20& 10& 5& 16& 8& 4& 2& 1& \\

1956&&&&&&&&&\\
978& 489& 1468& 734& 367& 1102& 551& 1654& 827& 2482\\
1241& 3724& 1862& 931& 2794& 1397& 4192& 2096& 1048& 524\\
262& 131& 394& 197& 592& 296& 148& 74& 37& 112\\
56& 28& 14& 7& 22& 11& 34& 17& 52& 26\\
13& 40& 20& 10& 5& 16& 8& 4& 2& 1\\

1957&&&&&&&&&\\
5872& 2936& 1468& 734& 367& 1102& 551& 1654& 827& 2482\\
1241& 3724& 1862& 931& 2794& 1397& 4192& 2096& 1048& 524\\
262& 131& 394& 197& 592& 296& 148& 74& 37& 112\\
56& 28& 14& 7& 22& 11& 34& 17& 52& 26\\
13& 40& 20& 10& 5& 16& 8& 4& 2& 1\\

1958&&&&&&&&&\\
979& 2938& 1469& 4408& 2204& 1102& 551& 1654& 827& 2482\\
1241& 3724& 1862& 931& 2794& 1397& 4192& 2096& 1048& 524\\
262& 131& 394& 197& 592& 296& 148& 74& 37& 112\\
56& 28& 14& 7& 22& 11& 34& 17& 52& 26\\
13& 40& 20& 10& 5& 16& 8& 4& 2& 1\\

1959&&&&&&&&&\\
5878& 2939& 8818& 4409& 13228& 6614& 3307& 9922& 4961& 14884\\
7442& 3721& 11164& 5582& 2791& 8374& 4187& 12562& 6281& 18844\\
9422& 4711& 14134& 7067& 21202& 10601& 31804& 15902& 7951& 23854\\
11927& 35782& 17891& 53674& 26837& 80512& 40256& 20128& 10064& 5032\\
2516& 1258& 629& 1888& 944& 472& 236& 118& 59& 178\\
89& 268& 134& 67& 202& 101& 304& 152& 76& 38\\
19& 58& 29& 88& 44& 22& 11& 34& 17& 52\\
26& 13& 40& 20& 10& 5& 16& 8& 4& 2\\
1& \\

1960&&&&&&&&&\\
980& 490& 245& 736& 368& 184& 92& 46& 23& 70\\
35& 106& 53& 160& 80& 40& 20& 10& 5& 16\\
8& 4& 2& 1& \\

1961&&&&&&&&&\\
5884& 2942& 1471& 4414& 2207& 6622& 3311& 9934& 4967& 14902\\
7451& 22354& 11177& 33532& 16766& 8383& 25150& 12575& 37726& 18863\\
56590& 28295& 84886& 42443& 127330& 63665& 190996& 95498& 47749& 143248\\
71624& 35812& 17906& 8953& 26860& 13430& 6715& 20146& 10073& 30220\\
15110& 7555& 22666& 11333& 34000& 17000& 8500& 4250& 2125& 6376\\
3188& 1594& 797& 2392& 1196& 598& 299& 898& 449& 1348\\
674& 337& 1012& 506& 253& 760& 380& 190& 95& 286\\
143& 430& 215& 646& 323& 970& 485& 1456& 728& 364\\
182& 91& 274& 137& 412& 206& 103& 310& 155& 466\\
233& 700& 350& 175& 526& 263& 790& 395& 1186& 593\\
1780& 890& 445& 1336& 668& 334& 167& 502& 251& 754\\
377& 1132& 566& 283& 850& 425& 1276& 638& 319& 958\\
479& 1438& 719& 2158& 1079& 3238& 1619& 4858& 2429& 7288\\
3644& 1822& 911& 2734& 1367& 4102& 2051& 6154& 3077& 9232\\
4616& 2308& 1154& 577& 1732& 866& 433& 1300& 650& 325\\
976& 488& 244& 122& 61& 184& 92& 46& 23& 70\\
35& 106& 53& 160& 80& 40& 20& 10& 5& 16\\
8& 4& 2& 1& \\

1962&&&&&&&&&\\
981& 2944& 1472& 736& 368& 184& 92& 46& 23& 70\\
35& 106& 53& 160& 80& 40& 20& 10& 5& 16\\
8& 4& 2& 1& \\

1963&&&&&&&&&\\
5890& 2945& 8836& 4418& 2209& 6628& 3314& 1657& 4972& 2486\\
1243& 3730& 1865& 5596& 2798& 1399& 4198& 2099& 6298& 3149\\
9448& 4724& 2362& 1181& 3544& 1772& 886& 443& 1330& 665\\
1996& 998& 499& 1498& 749& 2248& 1124& 562& 281& 844\\
422& 211& 634& 317& 952& 476& 238& 119& 358& 179\\
538& 269& 808& 404& 202& 101& 304& 152& 76& 38\\
19& 58& 29& 88& 44& 22& 11& 34& 17& 52\\
26& 13& 40& 20& 10& 5& 16& 8& 4& 2\\
1& \\

1964&&&&&&&&&\\
982& 491& 1474& 737& 2212& 1106& 553& 1660& 830& 415\\
1246& 623& 1870& 935& 2806& 1403& 4210& 2105& 6316& 3158\\
1579& 4738& 2369& 7108& 3554& 1777& 5332& 2666& 1333& 4000\\
2000& 1000& 500& 250& 125& 376& 188& 94& 47& 142\\
71& 214& 107& 322& 161& 484& 242& 121& 364& 182\\
91& 274& 137& 412& 206& 103& 310& 155& 466& 233\\
700& 350& 175& 526& 263& 790& 395& 1186& 593& 1780\\
890& 445& 1336& 668& 334& 167& 502& 251& 754& 377\\
1132& 566& 283& 850& 425& 1276& 638& 319& 958& 479\\
1438& 719& 2158& 1079& 3238& 1619& 4858& 2429& 7288& 3644\\
1822& 911& 2734& 1367& 4102& 2051& 6154& 3077& 9232& 4616\\
2308& 1154& 577& 1732& 866& 433& 1300& 650& 325& 976\\
488& 244& 122& 61& 184& 92& 46& 23& 70& 35\\
106& 53& 160& 80& 40& 20& 10& 5& 16& 8\\
4& 2& 1& \\

1965&&&&&&&&&\\
5896& 2948& 1474& 737& 2212& 1106& 553& 1660& 830& 415\\
1246& 623& 1870& 935& 2806& 1403& 4210& 2105& 6316& 3158\\
1579& 4738& 2369& 7108& 3554& 1777& 5332& 2666& 1333& 4000\\
2000& 1000& 500& 250& 125& 376& 188& 94& 47& 142\\
71& 214& 107& 322& 161& 484& 242& 121& 364& 182\\
91& 274& 137& 412& 206& 103& 310& 155& 466& 233\\
700& 350& 175& 526& 263& 790& 395& 1186& 593& 1780\\
890& 445& 1336& 668& 334& 167& 502& 251& 754& 377\\
1132& 566& 283& 850& 425& 1276& 638& 319& 958& 479\\
1438& 719& 2158& 1079& 3238& 1619& 4858& 2429& 7288& 3644\\
1822& 911& 2734& 1367& 4102& 2051& 6154& 3077& 9232& 4616\\
2308& 1154& 577& 1732& 866& 433& 1300& 650& 325& 976\\
488& 244& 122& 61& 184& 92& 46& 23& 70& 35\\
106& 53& 160& 80& 40& 20& 10& 5& 16& 8\\
4& 2& 1& \\

1966&&&&&&&&&\\
983& 2950& 1475& 4426& 2213& 6640& 3320& 1660& 830& 415\\
1246& 623& 1870& 935& 2806& 1403& 4210& 2105& 6316& 3158\\
1579& 4738& 2369& 7108& 3554& 1777& 5332& 2666& 1333& 4000\\
2000& 1000& 500& 250& 125& 376& 188& 94& 47& 142\\
71& 214& 107& 322& 161& 484& 242& 121& 364& 182\\
91& 274& 137& 412& 206& 103& 310& 155& 466& 233\\
700& 350& 175& 526& 263& 790& 395& 1186& 593& 1780\\
890& 445& 1336& 668& 334& 167& 502& 251& 754& 377\\
1132& 566& 283& 850& 425& 1276& 638& 319& 958& 479\\
1438& 719& 2158& 1079& 3238& 1619& 4858& 2429& 7288& 3644\\
1822& 911& 2734& 1367& 4102& 2051& 6154& 3077& 9232& 4616\\
2308& 1154& 577& 1732& 866& 433& 1300& 650& 325& 976\\
488& 244& 122& 61& 184& 92& 46& 23& 70& 35\\
106& 53& 160& 80& 40& 20& 10& 5& 16& 8\\
4& 2& 1& \\

1967&&&&&&&&&\\
5902& 2951& 8854& 4427& 13282& 6641& 19924& 9962& 4981& 14944\\
7472& 3736& 1868& 934& 467& 1402& 701& 2104& 1052& 526\\
263& 790& 395& 1186& 593& 1780& 890& 445& 1336& 668\\
334& 167& 502& 251& 754& 377& 1132& 566& 283& 850\\
425& 1276& 638& 319& 958& 479& 1438& 719& 2158& 1079\\
3238& 1619& 4858& 2429& 7288& 3644& 1822& 911& 2734& 1367\\
4102& 2051& 6154& 3077& 9232& 4616& 2308& 1154& 577& 1732\\
866& 433& 1300& 650& 325& 976& 488& 244& 122& 61\\
184& 92& 46& 23& 70& 35& 106& 53& 160& 80\\
40& 20& 10& 5& 16& 8& 4& 2& 1& \\

1968&&&&&&&&&\\
984& 492& 246& 123& 370& 185& 556& 278& 139& 418\\
209& 628& 314& 157& 472& 236& 118& 59& 178& 89\\
268& 134& 67& 202& 101& 304& 152& 76& 38& 19\\
58& 29& 88& 44& 22& 11& 34& 17& 52& 26\\
13& 40& 20& 10& 5& 16& 8& 4& 2& 1\\

1969&&&&&&&&&\\
5908& 2954& 1477& 4432& 2216& 1108& 554& 277& 832& 416\\
208& 104& 52& 26& 13& 40& 20& 10& 5& 16\\
8& 4& 2& 1& \\

1970&&&&&&&&&\\
985& 2956& 1478& 739& 2218& 1109& 3328& 1664& 832& 416\\
208& 104& 52& 26& 13& 40& 20& 10& 5& 16\\
8& 4& 2& 1& \\

1971&&&&&&&&&\\
5914& 2957& 8872& 4436& 2218& 1109& 3328& 1664& 832& 416\\
208& 104& 52& 26& 13& 40& 20& 10& 5& 16\\
8& 4& 2& 1& \\

1972&&&&&&&&&\\
986& 493& 1480& 740& 370& 185& 556& 278& 139& 418\\
209& 628& 314& 157& 472& 236& 118& 59& 178& 89\\
268& 134& 67& 202& 101& 304& 152& 76& 38& 19\\
58& 29& 88& 44& 22& 11& 34& 17& 52& 26\\
13& 40& 20& 10& 5& 16& 8& 4& 2& 1\\

1973&&&&&&&&&\\
5920& 2960& 1480& 740& 370& 185& 556& 278& 139& 418\\
209& 628& 314& 157& 472& 236& 118& 59& 178& 89\\
268& 134& 67& 202& 101& 304& 152& 76& 38& 19\\
58& 29& 88& 44& 22& 11& 34& 17& 52& 26\\
13& 40& 20& 10& 5& 16& 8& 4& 2& 1\\

1974&&&&&&&&&\\
987& 2962& 1481& 4444& 2222& 1111& 3334& 1667& 5002& 2501\\
7504& 3752& 1876& 938& 469& 1408& 704& 352& 176& 88\\
44& 22& 11& 34& 17& 52& 26& 13& 40& 20\\
10& 5& 16& 8& 4& 2& 1& \\

1975&&&&&&&&&\\
5926& 2963& 8890& 4445& 13336& 6668& 3334& 1667& 5002& 2501\\
7504& 3752& 1876& 938& 469& 1408& 704& 352& 176& 88\\
44& 22& 11& 34& 17& 52& 26& 13& 40& 20\\
10& 5& 16& 8& 4& 2& 1& \\

1976&&&&&&&&&\\
988& 494& 247& 742& 371& 1114& 557& 1672& 836& 418\\
209& 628& 314& 157& 472& 236& 118& 59& 178& 89\\
268& 134& 67& 202& 101& 304& 152& 76& 38& 19\\
58& 29& 88& 44& 22& 11& 34& 17& 52& 26\\
13& 40& 20& 10& 5& 16& 8& 4& 2& 1\\

1977&&&&&&&&&\\
5932& 2966& 1483& 4450& 2225& 6676& 3338& 1669& 5008& 2504\\
1252& 626& 313& 940& 470& 235& 706& 353& 1060& 530\\
265& 796& 398& 199& 598& 299& 898& 449& 1348& 674\\
337& 1012& 506& 253& 760& 380& 190& 95& 286& 143\\
430& 215& 646& 323& 970& 485& 1456& 728& 364& 182\\
91& 274& 137& 412& 206& 103& 310& 155& 466& 233\\
700& 350& 175& 526& 263& 790& 395& 1186& 593& 1780\\
890& 445& 1336& 668& 334& 167& 502& 251& 754& 377\\
1132& 566& 283& 850& 425& 1276& 638& 319& 958& 479\\
1438& 719& 2158& 1079& 3238& 1619& 4858& 2429& 7288& 3644\\
1822& 911& 2734& 1367& 4102& 2051& 6154& 3077& 9232& 4616\\
2308& 1154& 577& 1732& 866& 433& 1300& 650& 325& 976\\
488& 244& 122& 61& 184& 92& 46& 23& 70& 35\\
106& 53& 160& 80& 40& 20& 10& 5& 16& 8\\
4& 2& 1& \\

1978&&&&&&&&&\\
989& 2968& 1484& 742& 371& 1114& 557& 1672& 836& 418\\
209& 628& 314& 157& 472& 236& 118& 59& 178& 89\\
268& 134& 67& 202& 101& 304& 152& 76& 38& 19\\
58& 29& 88& 44& 22& 11& 34& 17& 52& 26\\
13& 40& 20& 10& 5& 16& 8& 4& 2& 1\\

1979&&&&&&&&&\\
5938& 2969& 8908& 4454& 2227& 6682& 3341& 10024& 5012& 2506\\
1253& 3760& 1880& 940& 470& 235& 706& 353& 1060& 530\\
265& 796& 398& 199& 598& 299& 898& 449& 1348& 674\\
337& 1012& 506& 253& 760& 380& 190& 95& 286& 143\\
430& 215& 646& 323& 970& 485& 1456& 728& 364& 182\\
91& 274& 137& 412& 206& 103& 310& 155& 466& 233\\
700& 350& 175& 526& 263& 790& 395& 1186& 593& 1780\\
890& 445& 1336& 668& 334& 167& 502& 251& 754& 377\\
1132& 566& 283& 850& 425& 1276& 638& 319& 958& 479\\
1438& 719& 2158& 1079& 3238& 1619& 4858& 2429& 7288& 3644\\
1822& 911& 2734& 1367& 4102& 2051& 6154& 3077& 9232& 4616\\
2308& 1154& 577& 1732& 866& 433& 1300& 650& 325& 976\\
488& 244& 122& 61& 184& 92& 46& 23& 70& 35\\
106& 53& 160& 80& 40& 20& 10& 5& 16& 8\\
4& 2& 1& \\

1980&&&&&&&&&\\
990& 495& 1486& 743& 2230& 1115& 3346& 1673& 5020& 2510\\
1255& 3766& 1883& 5650& 2825& 8476& 4238& 2119& 6358& 3179\\
9538& 4769& 14308& 7154& 3577& 10732& 5366& 2683& 8050& 4025\\
12076& 6038& 3019& 9058& 4529& 13588& 6794& 3397& 10192& 5096\\
2548& 1274& 637& 1912& 956& 478& 239& 718& 359& 1078\\
539& 1618& 809& 2428& 1214& 607& 1822& 911& 2734& 1367\\
4102& 2051& 6154& 3077& 9232& 4616& 2308& 1154& 577& 1732\\
866& 433& 1300& 650& 325& 976& 488& 244& 122& 61\\
184& 92& 46& 23& 70& 35& 106& 53& 160& 80\\
40& 20& 10& 5& 16& 8& 4& 2& 1& \\

1981&&&&&&&&&\\
5944& 2972& 1486& 743& 2230& 1115& 3346& 1673& 5020& 2510\\
1255& 3766& 1883& 5650& 2825& 8476& 4238& 2119& 6358& 3179\\
9538& 4769& 14308& 7154& 3577& 10732& 5366& 2683& 8050& 4025\\
12076& 6038& 3019& 9058& 4529& 13588& 6794& 3397& 10192& 5096\\
2548& 1274& 637& 1912& 956& 478& 239& 718& 359& 1078\\
539& 1618& 809& 2428& 1214& 607& 1822& 911& 2734& 1367\\
4102& 2051& 6154& 3077& 9232& 4616& 2308& 1154& 577& 1732\\
866& 433& 1300& 650& 325& 976& 488& 244& 122& 61\\
184& 92& 46& 23& 70& 35& 106& 53& 160& 80\\
40& 20& 10& 5& 16& 8& 4& 2& 1& \\

1982&&&&&&&&&\\
991& 2974& 1487& 4462& 2231& 6694& 3347& 10042& 5021& 15064\\
7532& 3766& 1883& 5650& 2825& 8476& 4238& 2119& 6358& 3179\\
9538& 4769& 14308& 7154& 3577& 10732& 5366& 2683& 8050& 4025\\
12076& 6038& 3019& 9058& 4529& 13588& 6794& 3397& 10192& 5096\\
2548& 1274& 637& 1912& 956& 478& 239& 718& 359& 1078\\
539& 1618& 809& 2428& 1214& 607& 1822& 911& 2734& 1367\\
4102& 2051& 6154& 3077& 9232& 4616& 2308& 1154& 577& 1732\\
866& 433& 1300& 650& 325& 976& 488& 244& 122& 61\\
184& 92& 46& 23& 70& 35& 106& 53& 160& 80\\
40& 20& 10& 5& 16& 8& 4& 2& 1& \\

1983&&&&&&&&&\\
5950& 2975& 8926& 4463& 13390& 6695& 20086& 10043& 30130& 15065\\
45196& 22598& 11299& 33898& 16949& 50848& 25424& 12712& 6356& 3178\\
1589& 4768& 2384& 1192& 596& 298& 149& 448& 224& 112\\
56& 28& 14& 7& 22& 11& 34& 17& 52& 26\\
13& 40& 20& 10& 5& 16& 8& 4& 2& 1\\

1984&&&&&&&&&\\
992& 496& 248& 124& 62& 31& 94& 47& 142& 71\\
214& 107& 322& 161& 484& 242& 121& 364& 182& 91\\
274& 137& 412& 206& 103& 310& 155& 466& 233& 700\\
350& 175& 526& 263& 790& 395& 1186& 593& 1780& 890\\
445& 1336& 668& 334& 167& 502& 251& 754& 377& 1132\\
566& 283& 850& 425& 1276& 638& 319& 958& 479& 1438\\
719& 2158& 1079& 3238& 1619& 4858& 2429& 7288& 3644& 1822\\
911& 2734& 1367& 4102& 2051& 6154& 3077& 9232& 4616& 2308\\
1154& 577& 1732& 866& 433& 1300& 650& 325& 976& 488\\
244& 122& 61& 184& 92& 46& 23& 70& 35& 106\\
53& 160& 80& 40& 20& 10& 5& 16& 8& 4\\
2& 1& \\

1985&&&&&&&&&\\
5956& 2978& 1489& 4468& 2234& 1117& 3352& 1676& 838& 419\\
1258& 629& 1888& 944& 472& 236& 118& 59& 178& 89\\
268& 134& 67& 202& 101& 304& 152& 76& 38& 19\\
58& 29& 88& 44& 22& 11& 34& 17& 52& 26\\
13& 40& 20& 10& 5& 16& 8& 4& 2& 1\\

1986&&&&&&&&&\\
993& 2980& 1490& 745& 2236& 1118& 559& 1678& 839& 2518\\
1259& 3778& 1889& 5668& 2834& 1417& 4252& 2126& 1063& 3190\\
1595& 4786& 2393& 7180& 3590& 1795& 5386& 2693& 8080& 4040\\
2020& 1010& 505& 1516& 758& 379& 1138& 569& 1708& 854\\
427& 1282& 641& 1924& 962& 481& 1444& 722& 361& 1084\\
542& 271& 814& 407& 1222& 611& 1834& 917& 2752& 1376\\
688& 344& 172& 86& 43& 130& 65& 196& 98& 49\\
148& 74& 37& 112& 56& 28& 14& 7& 22& 11\\
34& 17& 52& 26& 13& 40& 20& 10& 5& 16\\
8& 4& 2& 1& \\

1987&&&&&&&&&\\
5962& 2981& 8944& 4472& 2236& 1118& 559& 1678& 839& 2518\\
1259& 3778& 1889& 5668& 2834& 1417& 4252& 2126& 1063& 3190\\
1595& 4786& 2393& 7180& 3590& 1795& 5386& 2693& 8080& 4040\\
2020& 1010& 505& 1516& 758& 379& 1138& 569& 1708& 854\\
427& 1282& 641& 1924& 962& 481& 1444& 722& 361& 1084\\
542& 271& 814& 407& 1222& 611& 1834& 917& 2752& 1376\\
688& 344& 172& 86& 43& 130& 65& 196& 98& 49\\
148& 74& 37& 112& 56& 28& 14& 7& 22& 11\\
34& 17& 52& 26& 13& 40& 20& 10& 5& 16\\
8& 4& 2& 1& \\

1988&&&&&&&&&\\
994& 497& 1492& 746& 373& 1120& 560& 280& 140& 70\\
35& 106& 53& 160& 80& 40& 20& 10& 5& 16\\
8& 4& 2& 1& \\

1989&&&&&&&&&\\
5968& 2984& 1492& 746& 373& 1120& 560& 280& 140& 70\\
35& 106& 53& 160& 80& 40& 20& 10& 5& 16\\
8& 4& 2& 1& \\

1990&&&&&&&&&\\
995& 2986& 1493& 4480& 2240& 1120& 560& 280& 140& 70\\
35& 106& 53& 160& 80& 40& 20& 10& 5& 16\\
8& 4& 2& 1& \\

1991&&&&&&&&&\\
5974& 2987& 8962& 4481& 13444& 6722& 3361& 10084& 5042& 2521\\
7564& 3782& 1891& 5674& 2837& 8512& 4256& 2128& 1064& 532\\
266& 133& 400& 200& 100& 50& 25& 76& 38& 19\\
58& 29& 88& 44& 22& 11& 34& 17& 52& 26\\
13& 40& 20& 10& 5& 16& 8& 4& 2& 1\\

1992&&&&&&&&&\\
996& 498& 249& 748& 374& 187& 562& 281& 844& 422\\
211& 634& 317& 952& 476& 238& 119& 358& 179& 538\\
269& 808& 404& 202& 101& 304& 152& 76& 38& 19\\
58& 29& 88& 44& 22& 11& 34& 17& 52& 26\\
13& 40& 20& 10& 5& 16& 8& 4& 2& 1\\

1993&&&&&&&&&\\
5980& 2990& 1495& 4486& 2243& 6730& 3365& 10096& 5048& 2524\\
1262& 631& 1894& 947& 2842& 1421& 4264& 2132& 1066& 533\\
1600& 800& 400& 200& 100& 50& 25& 76& 38& 19\\
58& 29& 88& 44& 22& 11& 34& 17& 52& 26\\
13& 40& 20& 10& 5& 16& 8& 4& 2& 1\\

1994&&&&&&&&&\\
997& 2992& 1496& 748& 374& 187& 562& 281& 844& 422\\
211& 634& 317& 952& 476& 238& 119& 358& 179& 538\\
269& 808& 404& 202& 101& 304& 152& 76& 38& 19\\
58& 29& 88& 44& 22& 11& 34& 17& 52& 26\\
13& 40& 20& 10& 5& 16& 8& 4& 2& 1\\

1995&&&&&&&&&\\
5986& 2993& 8980& 4490& 2245& 6736& 3368& 1684& 842& 421\\
1264& 632& 316& 158& 79& 238& 119& 358& 179& 538\\
269& 808& 404& 202& 101& 304& 152& 76& 38& 19\\
58& 29& 88& 44& 22& 11& 34& 17& 52& 26\\
13& 40& 20& 10& 5& 16& 8& 4& 2& 1\\

1996&&&&&&&&&\\
998& 499& 1498& 749& 2248& 1124& 562& 281& 844& 422\\
211& 634& 317& 952& 476& 238& 119& 358& 179& 538\\
269& 808& 404& 202& 101& 304& 152& 76& 38& 19\\
58& 29& 88& 44& 22& 11& 34& 17& 52& 26\\
13& 40& 20& 10& 5& 16& 8& 4& 2& 1\\

1997&&&&&&&&&\\
5992& 2996& 1498& 749& 2248& 1124& 562& 281& 844& 422\\
211& 634& 317& 952& 476& 238& 119& 358& 179& 538\\
269& 808& 404& 202& 101& 304& 152& 76& 38& 19\\
58& 29& 88& 44& 22& 11& 34& 17& 52& 26\\
13& 40& 20& 10& 5& 16& 8& 4& 2& 1\\

1998&&&&&&&&&\\
999& 2998& 1499& 4498& 2249& 6748& 3374& 1687& 5062& 2531\\
7594& 3797& 11392& 5696& 2848& 1424& 712& 356& 178& 89\\
268& 134& 67& 202& 101& 304& 152& 76& 38& 19\\
58& 29& 88& 44& 22& 11& 34& 17& 52& 26\\
13& 40& 20& 10& 5& 16& 8& 4& 2& 1\\

1999&&&&&&&&&\\
5998& 2999& 8998& 4499& 13498& 6749& 20248& 10124& 5062& 2531\\
7594& 3797& 11392& 5696& 2848& 1424& 712& 356& 178& 89\\
268& 134& 67& 202& 101& 304& 152& 76& 38& 19\\
58& 29& 88& 44& 22& 11& 34& 17& 52& 26\\
13& 40& 20& 10& 5& 16& 8& 4& 2& 1\\

2000&&&&&&&&&\\
1000& 500& 250& 125& 376& 188& 94& 47& 142& 71\\
214& 107& 322& 161& 484& 242& 121& 364& 182& 91\\
274& 137& 412& 206& 103& 310& 155& 466& 233& 700\\
350& 175& 526& 263& 790& 395& 1186& 593& 1780& 890\\
445& 1336& 668& 334& 167& 502& 251& 754& 377& 1132\\
566& 283& 850& 425& 1276& 638& 319& 958& 479& 1438\\
719& 2158& 1079& 3238& 1619& 4858& 2429& 7288& 3644& 1822\\
911& 2734& 1367& 4102& 2051& 6154& 3077& 9232& 4616& 2308\\
1154& 577& 1732& 866& 433& 1300& 650& 325& 976& 488\\
244& 122& 61& 184& 92& 46& 23& 70& 35& 106\\
53& 160& 80& 40& 20& 10& 5& 16& 8& 4\\
2& 1& \\

2001&&&&&&&&&\\
6004& 3002& 1501& 4504& 2252& 1126& 563& 1690& 845& 2536\\
1268& 634& 317& 952& 476& 238& 119& 358& 179& 538\\
269& 808& 404& 202& 101& 304& 152& 76& 38& 19\\
58& 29& 88& 44& 22& 11& 34& 17& 52& 26\\
13& 40& 20& 10& 5& 16& 8& 4& 2& 1\\

2002&&&&&&&&&\\
1001& 3004& 1502& 751& 2254& 1127& 3382& 1691& 5074& 2537\\
7612& 3806& 1903& 5710& 2855& 8566& 4283& 12850& 6425& 19276\\
9638& 4819& 14458& 7229& 21688& 10844& 5422& 2711& 8134& 4067\\
12202& 6101& 18304& 9152& 4576& 2288& 1144& 572& 286& 143\\
430& 215& 646& 323& 970& 485& 1456& 728& 364& 182\\
91& 274& 137& 412& 206& 103& 310& 155& 466& 233\\
700& 350& 175& 526& 263& 790& 395& 1186& 593& 1780\\
890& 445& 1336& 668& 334& 167& 502& 251& 754& 377\\
1132& 566& 283& 850& 425& 1276& 638& 319& 958& 479\\
1438& 719& 2158& 1079& 3238& 1619& 4858& 2429& 7288& 3644\\
1822& 911& 2734& 1367& 4102& 2051& 6154& 3077& 9232& 4616\\
2308& 1154& 577& 1732& 866& 433& 1300& 650& 325& 976\\
488& 244& 122& 61& 184& 92& 46& 23& 70& 35\\
106& 53& 160& 80& 40& 20& 10& 5& 16& 8\\
4& 2& 1& \\

2003&&&&&&&&&\\
6010& 3005& 9016& 4508& 2254& 1127& 3382& 1691& 5074& 2537\\
7612& 3806& 1903& 5710& 2855& 8566& 4283& 12850& 6425& 19276\\
9638& 4819& 14458& 7229& 21688& 10844& 5422& 2711& 8134& 4067\\
12202& 6101& 18304& 9152& 4576& 2288& 1144& 572& 286& 143\\
430& 215& 646& 323& 970& 485& 1456& 728& 364& 182\\
91& 274& 137& 412& 206& 103& 310& 155& 466& 233\\
700& 350& 175& 526& 263& 790& 395& 1186& 593& 1780\\
890& 445& 1336& 668& 334& 167& 502& 251& 754& 377\\
1132& 566& 283& 850& 425& 1276& 638& 319& 958& 479\\
1438& 719& 2158& 1079& 3238& 1619& 4858& 2429& 7288& 3644\\
1822& 911& 2734& 1367& 4102& 2051& 6154& 3077& 9232& 4616\\
2308& 1154& 577& 1732& 866& 433& 1300& 650& 325& 976\\
488& 244& 122& 61& 184& 92& 46& 23& 70& 35\\
106& 53& 160& 80& 40& 20& 10& 5& 16& 8\\
4& 2& 1& \\

2004&&&&&&&&&\\
1002& 501& 1504& 752& 376& 188& 94& 47& 142& 71\\
214& 107& 322& 161& 484& 242& 121& 364& 182& 91\\
274& 137& 412& 206& 103& 310& 155& 466& 233& 700\\
350& 175& 526& 263& 790& 395& 1186& 593& 1780& 890\\
445& 1336& 668& 334& 167& 502& 251& 754& 377& 1132\\
566& 283& 850& 425& 1276& 638& 319& 958& 479& 1438\\
719& 2158& 1079& 3238& 1619& 4858& 2429& 7288& 3644& 1822\\
911& 2734& 1367& 4102& 2051& 6154& 3077& 9232& 4616& 2308\\
1154& 577& 1732& 866& 433& 1300& 650& 325& 976& 488\\
244& 122& 61& 184& 92& 46& 23& 70& 35& 106\\
53& 160& 80& 40& 20& 10& 5& 16& 8& 4\\
2& 1& \\

2005&&&&&&&&&\\
6016& 3008& 1504& 752& 376& 188& 94& 47& 142& 71\\
214& 107& 322& 161& 484& 242& 121& 364& 182& 91\\
274& 137& 412& 206& 103& 310& 155& 466& 233& 700\\
350& 175& 526& 263& 790& 395& 1186& 593& 1780& 890\\
445& 1336& 668& 334& 167& 502& 251& 754& 377& 1132\\
566& 283& 850& 425& 1276& 638& 319& 958& 479& 1438\\
719& 2158& 1079& 3238& 1619& 4858& 2429& 7288& 3644& 1822\\
911& 2734& 1367& 4102& 2051& 6154& 3077& 9232& 4616& 2308\\
1154& 577& 1732& 866& 433& 1300& 650& 325& 976& 488\\
244& 122& 61& 184& 92& 46& 23& 70& 35& 106\\
53& 160& 80& 40& 20& 10& 5& 16& 8& 4\\
2& 1& \\

2006&&&&&&&&&\\
1003& 3010& 1505& 4516& 2258& 1129& 3388& 1694& 847& 2542\\
1271& 3814& 1907& 5722& 2861& 8584& 4292& 2146& 1073& 3220\\
1610& 805& 2416& 1208& 604& 302& 151& 454& 227& 682\\
341& 1024& 512& 256& 128& 64& 32& 16& 8& 4\\
2& 1& \\

2007&&&&&&&&&\\
6022& 3011& 9034& 4517& 13552& 6776& 3388& 1694& 847& 2542\\
1271& 3814& 1907& 5722& 2861& 8584& 4292& 2146& 1073& 3220\\
1610& 805& 2416& 1208& 604& 302& 151& 454& 227& 682\\
341& 1024& 512& 256& 128& 64& 32& 16& 8& 4\\
2& 1& \\

2008&&&&&&&&&\\
1004& 502& 251& 754& 377& 1132& 566& 283& 850& 425\\
1276& 638& 319& 958& 479& 1438& 719& 2158& 1079& 3238\\
1619& 4858& 2429& 7288& 3644& 1822& 911& 2734& 1367& 4102\\
2051& 6154& 3077& 9232& 4616& 2308& 1154& 577& 1732& 866\\
433& 1300& 650& 325& 976& 488& 244& 122& 61& 184\\
92& 46& 23& 70& 35& 106& 53& 160& 80& 40\\
20& 10& 5& 16& 8& 4& 2& 1& \\

2009&&&&&&&&&\\
6028& 3014& 1507& 4522& 2261& 6784& 3392& 1696& 848& 424\\
212& 106& 53& 160& 80& 40& 20& 10& 5& 16\\
8& 4& 2& 1& \\

2010&&&&&&&&&\\
1005& 3016& 1508& 754& 377& 1132& 566& 283& 850& 425\\
1276& 638& 319& 958& 479& 1438& 719& 2158& 1079& 3238\\
1619& 4858& 2429& 7288& 3644& 1822& 911& 2734& 1367& 4102\\
2051& 6154& 3077& 9232& 4616& 2308& 1154& 577& 1732& 866\\
433& 1300& 650& 325& 976& 488& 244& 122& 61& 184\\
92& 46& 23& 70& 35& 106& 53& 160& 80& 40\\
20& 10& 5& 16& 8& 4& 2& 1& \\

2011&&&&&&&&&\\
6034& 3017& 9052& 4526& 2263& 6790& 3395& 10186& 5093& 15280\\
7640& 3820& 1910& 955& 2866& 1433& 4300& 2150& 1075& 3226\\
1613& 4840& 2420& 1210& 605& 1816& 908& 454& 227& 682\\
341& 1024& 512& 256& 128& 64& 32& 16& 8& 4\\
2& 1& \\

2012&&&&&&&&&\\
1006& 503& 1510& 755& 2266& 1133& 3400& 1700& 850& 425\\
1276& 638& 319& 958& 479& 1438& 719& 2158& 1079& 3238\\
1619& 4858& 2429& 7288& 3644& 1822& 911& 2734& 1367& 4102\\
2051& 6154& 3077& 9232& 4616& 2308& 1154& 577& 1732& 866\\
433& 1300& 650& 325& 976& 488& 244& 122& 61& 184\\
92& 46& 23& 70& 35& 106& 53& 160& 80& 40\\
20& 10& 5& 16& 8& 4& 2& 1& \\

2013&&&&&&&&&\\
6040& 3020& 1510& 755& 2266& 1133& 3400& 1700& 850& 425\\
1276& 638& 319& 958& 479& 1438& 719& 2158& 1079& 3238\\
1619& 4858& 2429& 7288& 3644& 1822& 911& 2734& 1367& 4102\\
2051& 6154& 3077& 9232& 4616& 2308& 1154& 577& 1732& 866\\
433& 1300& 650& 325& 976& 488& 244& 122& 61& 184\\
92& 46& 23& 70& 35& 106& 53& 160& 80& 40\\
20& 10& 5& 16& 8& 4& 2& 1& \\

2014&&&&&&&&&\\
1007& 3022& 1511& 4534& 2267& 6802& 3401& 10204& 5102& 2551\\
7654& 3827& 11482& 5741& 17224& 8612& 4306& 2153& 6460& 3230\\
1615& 4846& 2423& 7270& 3635& 10906& 5453& 16360& 8180& 4090\\
2045& 6136& 3068& 1534& 767& 2302& 1151& 3454& 1727& 5182\\
2591& 7774& 3887& 11662& 5831& 17494& 8747& 26242& 13121& 39364\\
19682& 9841& 29524& 14762& 7381& 22144& 11072& 5536& 2768& 1384\\
692& 346& 173& 520& 260& 130& 65& 196& 98& 49\\
148& 74& 37& 112& 56& 28& 14& 7& 22& 11\\
34& 17& 52& 26& 13& 40& 20& 10& 5& 16\\
8& 4& 2& 1& \\

2015&&&&&&&&&\\
6046& 3023& 9070& 4535& 13606& 6803& 20410& 10205& 30616& 15308\\
7654& 3827& 11482& 5741& 17224& 8612& 4306& 2153& 6460& 3230\\
1615& 4846& 2423& 7270& 3635& 10906& 5453& 16360& 8180& 4090\\
2045& 6136& 3068& 1534& 767& 2302& 1151& 3454& 1727& 5182\\
2591& 7774& 3887& 11662& 5831& 17494& 8747& 26242& 13121& 39364\\
19682& 9841& 29524& 14762& 7381& 22144& 11072& 5536& 2768& 1384\\
692& 346& 173& 520& 260& 130& 65& 196& 98& 49\\
148& 74& 37& 112& 56& 28& 14& 7& 22& 11\\
34& 17& 52& 26& 13& 40& 20& 10& 5& 16\\
8& 4& 2& 1& \\

2016&&&&&&&&&\\
1008& 504& 252& 126& 63& 190& 95& 286& 143& 430\\
215& 646& 323& 970& 485& 1456& 728& 364& 182& 91\\
274& 137& 412& 206& 103& 310& 155& 466& 233& 700\\
350& 175& 526& 263& 790& 395& 1186& 593& 1780& 890\\
445& 1336& 668& 334& 167& 502& 251& 754& 377& 1132\\
566& 283& 850& 425& 1276& 638& 319& 958& 479& 1438\\
719& 2158& 1079& 3238& 1619& 4858& 2429& 7288& 3644& 1822\\
911& 2734& 1367& 4102& 2051& 6154& 3077& 9232& 4616& 2308\\
1154& 577& 1732& 866& 433& 1300& 650& 325& 976& 488\\
244& 122& 61& 184& 92& 46& 23& 70& 35& 106\\
53& 160& 80& 40& 20& 10& 5& 16& 8& 4\\
2& 1& \\

2017&&&&&&&&&\\
6052& 3026& 1513& 4540& 2270& 1135& 3406& 1703& 5110& 2555\\
7666& 3833& 11500& 5750& 2875& 8626& 4313& 12940& 6470& 3235\\
9706& 4853& 14560& 7280& 3640& 1820& 910& 455& 1366& 683\\
2050& 1025& 3076& 1538& 769& 2308& 1154& 577& 1732& 866\\
433& 1300& 650& 325& 976& 488& 244& 122& 61& 184\\
92& 46& 23& 70& 35& 106& 53& 160& 80& 40\\
20& 10& 5& 16& 8& 4& 2& 1& \\

2018&&&&&&&&&\\
1009& 3028& 1514& 757& 2272& 1136& 568& 284& 142& 71\\
214& 107& 322& 161& 484& 242& 121& 364& 182& 91\\
274& 137& 412& 206& 103& 310& 155& 466& 233& 700\\
350& 175& 526& 263& 790& 395& 1186& 593& 1780& 890\\
445& 1336& 668& 334& 167& 502& 251& 754& 377& 1132\\
566& 283& 850& 425& 1276& 638& 319& 958& 479& 1438\\
719& 2158& 1079& 3238& 1619& 4858& 2429& 7288& 3644& 1822\\
911& 2734& 1367& 4102& 2051& 6154& 3077& 9232& 4616& 2308\\
1154& 577& 1732& 866& 433& 1300& 650& 325& 976& 488\\
244& 122& 61& 184& 92& 46& 23& 70& 35& 106\\
53& 160& 80& 40& 20& 10& 5& 16& 8& 4\\
2& 1& \\

2019&&&&&&&&&\\
6058& 3029& 9088& 4544& 2272& 1136& 568& 284& 142& 71\\
214& 107& 322& 161& 484& 242& 121& 364& 182& 91\\
274& 137& 412& 206& 103& 310& 155& 466& 233& 700\\
350& 175& 526& 263& 790& 395& 1186& 593& 1780& 890\\
445& 1336& 668& 334& 167& 502& 251& 754& 377& 1132\\
566& 283& 850& 425& 1276& 638& 319& 958& 479& 1438\\
719& 2158& 1079& 3238& 1619& 4858& 2429& 7288& 3644& 1822\\
911& 2734& 1367& 4102& 2051& 6154& 3077& 9232& 4616& 2308\\
1154& 577& 1732& 866& 433& 1300& 650& 325& 976& 488\\
244& 122& 61& 184& 92& 46& 23& 70& 35& 106\\
53& 160& 80& 40& 20& 10& 5& 16& 8& 4\\
2& 1& \\

2020&&&&&&&&&\\
1010& 505& 1516& 758& 379& 1138& 569& 1708& 854& 427\\
1282& 641& 1924& 962& 481& 1444& 722& 361& 1084& 542\\
271& 814& 407& 1222& 611& 1834& 917& 2752& 1376& 688\\
344& 172& 86& 43& 130& 65& 196& 98& 49& 148\\
74& 37& 112& 56& 28& 14& 7& 22& 11& 34\\
17& 52& 26& 13& 40& 20& 10& 5& 16& 8\\
4& 2& 1& \\

2021&&&&&&&&&\\
6064& 3032& 1516& 758& 379& 1138& 569& 1708& 854& 427\\
1282& 641& 1924& 962& 481& 1444& 722& 361& 1084& 542\\
271& 814& 407& 1222& 611& 1834& 917& 2752& 1376& 688\\
344& 172& 86& 43& 130& 65& 196& 98& 49& 148\\
74& 37& 112& 56& 28& 14& 7& 22& 11& 34\\
17& 52& 26& 13& 40& 20& 10& 5& 16& 8\\
4& 2& 1& \\

2022&&&&&&&&&\\
1011& 3034& 1517& 4552& 2276& 1138& 569& 1708& 854& 427\\
1282& 641& 1924& 962& 481& 1444& 722& 361& 1084& 542\\
271& 814& 407& 1222& 611& 1834& 917& 2752& 1376& 688\\
344& 172& 86& 43& 130& 65& 196& 98& 49& 148\\
74& 37& 112& 56& 28& 14& 7& 22& 11& 34\\
17& 52& 26& 13& 40& 20& 10& 5& 16& 8\\
4& 2& 1& \\

2023&&&&&&&&&\\
6070& 3035& 9106& 4553& 13660& 6830& 3415& 10246& 5123& 15370\\
7685& 23056& 11528& 5764& 2882& 1441& 4324& 2162& 1081& 3244\\
1622& 811& 2434& 1217& 3652& 1826& 913& 2740& 1370& 685\\
2056& 1028& 514& 257& 772& 386& 193& 580& 290& 145\\
436& 218& 109& 328& 164& 82& 41& 124& 62& 31\\
94& 47& 142& 71& 214& 107& 322& 161& 484& 242\\
121& 364& 182& 91& 274& 137& 412& 206& 103& 310\\
155& 466& 233& 700& 350& 175& 526& 263& 790& 395\\
1186& 593& 1780& 890& 445& 1336& 668& 334& 167& 502\\
251& 754& 377& 1132& 566& 283& 850& 425& 1276& 638\\
319& 958& 479& 1438& 719& 2158& 1079& 3238& 1619& 4858\\
2429& 7288& 3644& 1822& 911& 2734& 1367& 4102& 2051& 6154\\
3077& 9232& 4616& 2308& 1154& 577& 1732& 866& 433& 1300\\
650& 325& 976& 488& 244& 122& 61& 184& 92& 46\\
23& 70& 35& 106& 53& 160& 80& 40& 20& 10\\
5& 16& 8& 4& 2& 1& \\

2024&&&&&&&&&\\
1012& 506& 253& 760& 380& 190& 95& 286& 143& 430\\
215& 646& 323& 970& 485& 1456& 728& 364& 182& 91\\
274& 137& 412& 206& 103& 310& 155& 466& 233& 700\\
350& 175& 526& 263& 790& 395& 1186& 593& 1780& 890\\
445& 1336& 668& 334& 167& 502& 251& 754& 377& 1132\\
566& 283& 850& 425& 1276& 638& 319& 958& 479& 1438\\
719& 2158& 1079& 3238& 1619& 4858& 2429& 7288& 3644& 1822\\
911& 2734& 1367& 4102& 2051& 6154& 3077& 9232& 4616& 2308\\
1154& 577& 1732& 866& 433& 1300& 650& 325& 976& 488\\
244& 122& 61& 184& 92& 46& 23& 70& 35& 106\\
53& 160& 80& 40& 20& 10& 5& 16& 8& 4\\
2& 1& \\

2025&&&&&&&&&\\
6076& 3038& 1519& 4558& 2279& 6838& 3419& 10258& 5129& 15388\\
7694& 3847& 11542& 5771& 17314& 8657& 25972& 12986& 6493& 19480\\
9740& 4870& 2435& 7306& 3653& 10960& 5480& 2740& 1370& 685\\
2056& 1028& 514& 257& 772& 386& 193& 580& 290& 145\\
436& 218& 109& 328& 164& 82& 41& 124& 62& 31\\
94& 47& 142& 71& 214& 107& 322& 161& 484& 242\\
121& 364& 182& 91& 274& 137& 412& 206& 103& 310\\
155& 466& 233& 700& 350& 175& 526& 263& 790& 395\\
1186& 593& 1780& 890& 445& 1336& 668& 334& 167& 502\\
251& 754& 377& 1132& 566& 283& 850& 425& 1276& 638\\
319& 958& 479& 1438& 719& 2158& 1079& 3238& 1619& 4858\\
2429& 7288& 3644& 1822& 911& 2734& 1367& 4102& 2051& 6154\\
3077& 9232& 4616& 2308& 1154& 577& 1732& 866& 433& 1300\\
650& 325& 976& 488& 244& 122& 61& 184& 92& 46\\
23& 70& 35& 106& 53& 160& 80& 40& 20& 10\\
5& 16& 8& 4& 2& 1& \\

2026&&&&&&&&&\\
1013& 3040& 1520& 760& 380& 190& 95& 286& 143& 430\\
215& 646& 323& 970& 485& 1456& 728& 364& 182& 91\\
274& 137& 412& 206& 103& 310& 155& 466& 233& 700\\
350& 175& 526& 263& 790& 395& 1186& 593& 1780& 890\\
445& 1336& 668& 334& 167& 502& 251& 754& 377& 1132\\
566& 283& 850& 425& 1276& 638& 319& 958& 479& 1438\\
719& 2158& 1079& 3238& 1619& 4858& 2429& 7288& 3644& 1822\\
911& 2734& 1367& 4102& 2051& 6154& 3077& 9232& 4616& 2308\\
1154& 577& 1732& 866& 433& 1300& 650& 325& 976& 488\\
244& 122& 61& 184& 92& 46& 23& 70& 35& 106\\
53& 160& 80& 40& 20& 10& 5& 16& 8& 4\\
2& 1& \\

2027&&&&&&&&&\\
6082& 3041& 9124& 4562& 2281& 6844& 3422& 1711& 5134& 2567\\
7702& 3851& 11554& 5777& 17332& 8666& 4333& 13000& 6500& 3250\\
1625& 4876& 2438& 1219& 3658& 1829& 5488& 2744& 1372& 686\\
343& 1030& 515& 1546& 773& 2320& 1160& 580& 290& 145\\
436& 218& 109& 328& 164& 82& 41& 124& 62& 31\\
94& 47& 142& 71& 214& 107& 322& 161& 484& 242\\
121& 364& 182& 91& 274& 137& 412& 206& 103& 310\\
155& 466& 233& 700& 350& 175& 526& 263& 790& 395\\
1186& 593& 1780& 890& 445& 1336& 668& 334& 167& 502\\
251& 754& 377& 1132& 566& 283& 850& 425& 1276& 638\\
319& 958& 479& 1438& 719& 2158& 1079& 3238& 1619& 4858\\
2429& 7288& 3644& 1822& 911& 2734& 1367& 4102& 2051& 6154\\
3077& 9232& 4616& 2308& 1154& 577& 1732& 866& 433& 1300\\
650& 325& 976& 488& 244& 122& 61& 184& 92& 46\\
23& 70& 35& 106& 53& 160& 80& 40& 20& 10\\
5& 16& 8& 4& 2& 1& \\

2028&&&&&&&&&\\
1014& 507& 1522& 761& 2284& 1142& 571& 1714& 857& 2572\\
1286& 643& 1930& 965& 2896& 1448& 724& 362& 181& 544\\
272& 136& 68& 34& 17& 52& 26& 13& 40& 20\\
10& 5& 16& 8& 4& 2& 1& \\

2029&&&&&&&&&\\
6088& 3044& 1522& 761& 2284& 1142& 571& 1714& 857& 2572\\
1286& 643& 1930& 965& 2896& 1448& 724& 362& 181& 544\\
272& 136& 68& 34& 17& 52& 26& 13& 40& 20\\
10& 5& 16& 8& 4& 2& 1& \\

2030&&&&&&&&&\\
1015& 3046& 1523& 4570& 2285& 6856& 3428& 1714& 857& 2572\\
1286& 643& 1930& 965& 2896& 1448& 724& 362& 181& 544\\
272& 136& 68& 34& 17& 52& 26& 13& 40& 20\\
10& 5& 16& 8& 4& 2& 1& \\

2031&&&&&&&&&\\
6094& 3047& 9142& 4571& 13714& 6857& 20572& 10286& 5143& 15430\\
7715& 23146& 11573& 34720& 17360& 8680& 4340& 2170& 1085& 3256\\
1628& 814& 407& 1222& 611& 1834& 917& 2752& 1376& 688\\
344& 172& 86& 43& 130& 65& 196& 98& 49& 148\\
74& 37& 112& 56& 28& 14& 7& 22& 11& 34\\
17& 52& 26& 13& 40& 20& 10& 5& 16& 8\\
4& 2& 1& \\

2032&&&&&&&&&\\
1016& 508& 254& 127& 382& 191& 574& 287& 862& 431\\
1294& 647& 1942& 971& 2914& 1457& 4372& 2186& 1093& 3280\\
1640& 820& 410& 205& 616& 308& 154& 77& 232& 116\\
58& 29& 88& 44& 22& 11& 34& 17& 52& 26\\
13& 40& 20& 10& 5& 16& 8& 4& 2& 1\\

2033&&&&&&&&&\\
6100& 3050& 1525& 4576& 2288& 1144& 572& 286& 143& 430\\
215& 646& 323& 970& 485& 1456& 728& 364& 182& 91\\
274& 137& 412& 206& 103& 310& 155& 466& 233& 700\\
350& 175& 526& 263& 790& 395& 1186& 593& 1780& 890\\
445& 1336& 668& 334& 167& 502& 251& 754& 377& 1132\\
566& 283& 850& 425& 1276& 638& 319& 958& 479& 1438\\
719& 2158& 1079& 3238& 1619& 4858& 2429& 7288& 3644& 1822\\
911& 2734& 1367& 4102& 2051& 6154& 3077& 9232& 4616& 2308\\
1154& 577& 1732& 866& 433& 1300& 650& 325& 976& 488\\
244& 122& 61& 184& 92& 46& 23& 70& 35& 106\\
53& 160& 80& 40& 20& 10& 5& 16& 8& 4\\
2& 1& \\

2034&&&&&&&&&\\
1017& 3052& 1526& 763& 2290& 1145& 3436& 1718& 859& 2578\\
1289& 3868& 1934& 967& 2902& 1451& 4354& 2177& 6532& 3266\\
1633& 4900& 2450& 1225& 3676& 1838& 919& 2758& 1379& 4138\\
2069& 6208& 3104& 1552& 776& 388& 194& 97& 292& 146\\
73& 220& 110& 55& 166& 83& 250& 125& 376& 188\\
94& 47& 142& 71& 214& 107& 322& 161& 484& 242\\
121& 364& 182& 91& 274& 137& 412& 206& 103& 310\\
155& 466& 233& 700& 350& 175& 526& 263& 790& 395\\
1186& 593& 1780& 890& 445& 1336& 668& 334& 167& 502\\
251& 754& 377& 1132& 566& 283& 850& 425& 1276& 638\\
319& 958& 479& 1438& 719& 2158& 1079& 3238& 1619& 4858\\
2429& 7288& 3644& 1822& 911& 2734& 1367& 4102& 2051& 6154\\
3077& 9232& 4616& 2308& 1154& 577& 1732& 866& 433& 1300\\
650& 325& 976& 488& 244& 122& 61& 184& 92& 46\\
23& 70& 35& 106& 53& 160& 80& 40& 20& 10\\
5& 16& 8& 4& 2& 1& \\

2035&&&&&&&&&\\
6106& 3053& 9160& 4580& 2290& 1145& 3436& 1718& 859& 2578\\
1289& 3868& 1934& 967& 2902& 1451& 4354& 2177& 6532& 3266\\
1633& 4900& 2450& 1225& 3676& 1838& 919& 2758& 1379& 4138\\
2069& 6208& 3104& 1552& 776& 388& 194& 97& 292& 146\\
73& 220& 110& 55& 166& 83& 250& 125& 376& 188\\
94& 47& 142& 71& 214& 107& 322& 161& 484& 242\\
121& 364& 182& 91& 274& 137& 412& 206& 103& 310\\
155& 466& 233& 700& 350& 175& 526& 263& 790& 395\\
1186& 593& 1780& 890& 445& 1336& 668& 334& 167& 502\\
251& 754& 377& 1132& 566& 283& 850& 425& 1276& 638\\
319& 958& 479& 1438& 719& 2158& 1079& 3238& 1619& 4858\\
2429& 7288& 3644& 1822& 911& 2734& 1367& 4102& 2051& 6154\\
3077& 9232& 4616& 2308& 1154& 577& 1732& 866& 433& 1300\\
650& 325& 976& 488& 244& 122& 61& 184& 92& 46\\
23& 70& 35& 106& 53& 160& 80& 40& 20& 10\\
5& 16& 8& 4& 2& 1& \\

2036&&&&&&&&&\\
1018& 509& 1528& 764& 382& 191& 574& 287& 862& 431\\
1294& 647& 1942& 971& 2914& 1457& 4372& 2186& 1093& 3280\\
1640& 820& 410& 205& 616& 308& 154& 77& 232& 116\\
58& 29& 88& 44& 22& 11& 34& 17& 52& 26\\
13& 40& 20& 10& 5& 16& 8& 4& 2& 1\\

2037&&&&&&&&&\\
6112& 3056& 1528& 764& 382& 191& 574& 287& 862& 431\\
1294& 647& 1942& 971& 2914& 1457& 4372& 2186& 1093& 3280\\
1640& 820& 410& 205& 616& 308& 154& 77& 232& 116\\
58& 29& 88& 44& 22& 11& 34& 17& 52& 26\\
13& 40& 20& 10& 5& 16& 8& 4& 2& 1\\

2038&&&&&&&&&\\
1019& 3058& 1529& 4588& 2294& 1147& 3442& 1721& 5164& 2582\\
1291& 3874& 1937& 5812& 2906& 1453& 4360& 2180& 1090& 545\\
1636& 818& 409& 1228& 614& 307& 922& 461& 1384& 692\\
346& 173& 520& 260& 130& 65& 196& 98& 49& 148\\
74& 37& 112& 56& 28& 14& 7& 22& 11& 34\\
17& 52& 26& 13& 40& 20& 10& 5& 16& 8\\
4& 2& 1& \\

2039&&&&&&&&&\\
6118& 3059& 9178& 4589& 13768& 6884& 3442& 1721& 5164& 2582\\
1291& 3874& 1937& 5812& 2906& 1453& 4360& 2180& 1090& 545\\
1636& 818& 409& 1228& 614& 307& 922& 461& 1384& 692\\
346& 173& 520& 260& 130& 65& 196& 98& 49& 148\\
74& 37& 112& 56& 28& 14& 7& 22& 11& 34\\
17& 52& 26& 13& 40& 20& 10& 5& 16& 8\\
4& 2& 1& \\

2040&&&&&&&&&\\
1020& 510& 255& 766& 383& 1150& 575& 1726& 863& 2590\\
1295& 3886& 1943& 5830& 2915& 8746& 4373& 13120& 6560& 3280\\
1640& 820& 410& 205& 616& 308& 154& 77& 232& 116\\
58& 29& 88& 44& 22& 11& 34& 17& 52& 26\\
13& 40& 20& 10& 5& 16& 8& 4& 2& 1\\

2041&&&&&&&&&\\
6124& 3062& 1531& 4594& 2297& 6892& 3446& 1723& 5170& 2585\\
7756& 3878& 1939& 5818& 2909& 8728& 4364& 2182& 1091& 3274\\
1637& 4912& 2456& 1228& 614& 307& 922& 461& 1384& 692\\
346& 173& 520& 260& 130& 65& 196& 98& 49& 148\\
74& 37& 112& 56& 28& 14& 7& 22& 11& 34\\
17& 52& 26& 13& 40& 20& 10& 5& 16& 8\\
4& 2& 1& \\

2042&&&&&&&&&\\
1021& 3064& 1532& 766& 383& 1150& 575& 1726& 863& 2590\\
1295& 3886& 1943& 5830& 2915& 8746& 4373& 13120& 6560& 3280\\
1640& 820& 410& 205& 616& 308& 154& 77& 232& 116\\
58& 29& 88& 44& 22& 11& 34& 17& 52& 26\\
13& 40& 20& 10& 5& 16& 8& 4& 2& 1\\

2043&&&&&&&&&\\
6130& 3065& 9196& 4598& 2299& 6898& 3449& 10348& 5174& 2587\\
7762& 3881& 11644& 5822& 2911& 8734& 4367& 13102& 6551& 19654\\
9827& 29482& 14741& 44224& 22112& 11056& 5528& 2764& 1382& 691\\
2074& 1037& 3112& 1556& 778& 389& 1168& 584& 292& 146\\
73& 220& 110& 55& 166& 83& 250& 125& 376& 188\\
94& 47& 142& 71& 214& 107& 322& 161& 484& 242\\
121& 364& 182& 91& 274& 137& 412& 206& 103& 310\\
155& 466& 233& 700& 350& 175& 526& 263& 790& 395\\
1186& 593& 1780& 890& 445& 1336& 668& 334& 167& 502\\
251& 754& 377& 1132& 566& 283& 850& 425& 1276& 638\\
319& 958& 479& 1438& 719& 2158& 1079& 3238& 1619& 4858\\
2429& 7288& 3644& 1822& 911& 2734& 1367& 4102& 2051& 6154\\
3077& 9232& 4616& 2308& 1154& 577& 1732& 866& 433& 1300\\
650& 325& 976& 488& 244& 122& 61& 184& 92& 46\\
23& 70& 35& 106& 53& 160& 80& 40& 20& 10\\
5& 16& 8& 4& 2& 1& \\

2044&&&&&&&&&\\
1022& 511& 1534& 767& 2302& 1151& 3454& 1727& 5182& 2591\\
7774& 3887& 11662& 5831& 17494& 8747& 26242& 13121& 39364& 19682\\
9841& 29524& 14762& 7381& 22144& 11072& 5536& 2768& 1384& 692\\
346& 173& 520& 260& 130& 65& 196& 98& 49& 148\\
74& 37& 112& 56& 28& 14& 7& 22& 11& 34\\
17& 52& 26& 13& 40& 20& 10& 5& 16& 8\\
4& 2& 1& \\

2045&&&&&&&&&\\
6136& 3068& 1534& 767& 2302& 1151& 3454& 1727& 5182& 2591\\
7774& 3887& 11662& 5831& 17494& 8747& 26242& 13121& 39364& 19682\\
9841& 29524& 14762& 7381& 22144& 11072& 5536& 2768& 1384& 692\\
346& 173& 520& 260& 130& 65& 196& 98& 49& 148\\
74& 37& 112& 56& 28& 14& 7& 22& 11& 34\\
17& 52& 26& 13& 40& 20& 10& 5& 16& 8\\
4& 2& 1& \\

2046&&&&&&&&&\\
1023& 3070& 1535& 4606& 2303& 6910& 3455& 10366& 5183& 15550\\
7775& 23326& 11663& 34990& 17495& 52486& 26243& 78730& 39365& 118096\\
59048& 29524& 14762& 7381& 22144& 11072& 5536& 2768& 1384& 692\\
346& 173& 520& 260& 130& 65& 196& 98& 49& 148\\
74& 37& 112& 56& 28& 14& 7& 22& 11& 34\\
17& 52& 26& 13& 40& 20& 10& 5& 16& 8\\
4& 2& 1& \\

2047&&&&&&&&&\\
6142& 3071& 9214& 4607& 13822& 6911& 20734& 10367& 31102& 15551\\
46654& 23327& 69982& 34991& 104974& 52487& 157462& 78731& 236194& 118097\\
354292& 177146& 88573& 265720& 132860& 66430& 33215& 99646& 49823& 149470\\
74735& 224206& 112103& 336310& 168155& 504466& 252233& 756700& 378350& 189175\\
567526& 283763& 851290& 425645& 1276936& 638468& 319234& 159617& 478852& 239426\\
119713& 359140& 179570& 89785& 269356& 134678& 67339& 202018& 101009& 303028\\
151514& 75757& 227272& 113636& 56818& 28409& 85228& 42614& 21307& 63922\\
31961& 95884& 47942& 23971& 71914& 35957& 107872& 53936& 26968& 13484\\
6742& 3371& 10114& 5057& 15172& 7586& 3793& 11380& 5690& 2845\\
8536& 4268& 2134& 1067& 3202& 1601& 4804& 2402& 1201& 3604\\
1802& 901& 2704& 1352& 676& 338& 169& 508& 254& 127\\
382& 191& 574& 287& 862& 431& 1294& 647& 1942& 971\\
2914& 1457& 4372& 2186& 1093& 3280& 1640& 820& 410& 205\\
616& 308& 154& 77& 232& 116& 58& 29& 88& 44\\
22& 11& 34& 17& 52& 26& 13& 40& 20& 10\\
5& 16& 8& 4& 2& 1& \\

2048&&&&&&&&&\\
1024& 512& 256& 128& 64& 32& 16& 8& 4& 2\\
1& \\

2049&&&&&&&&&\\
6148& 3074& 1537& 4612& 2306& 1153& 3460& 1730& 865& 2596\\
1298& 649& 1948& 974& 487& 1462& 731& 2194& 1097& 3292\\
1646& 823& 2470& 1235& 3706& 1853& 5560& 2780& 1390& 695\\
2086& 1043& 3130& 1565& 4696& 2348& 1174& 587& 1762& 881\\
2644& 1322& 661& 1984& 992& 496& 248& 124& 62& 31\\
94& 47& 142& 71& 214& 107& 322& 161& 484& 242\\
121& 364& 182& 91& 274& 137& 412& 206& 103& 310\\
155& 466& 233& 700& 350& 175& 526& 263& 790& 395\\
1186& 593& 1780& 890& 445& 1336& 668& 334& 167& 502\\
251& 754& 377& 1132& 566& 283& 850& 425& 1276& 638\\
319& 958& 479& 1438& 719& 2158& 1079& 3238& 1619& 4858\\
2429& 7288& 3644& 1822& 911& 2734& 1367& 4102& 2051& 6154\\
3077& 9232& 4616& 2308& 1154& 577& 1732& 866& 433& 1300\\
650& 325& 976& 488& 244& 122& 61& 184& 92& 46\\
23& 70& 35& 106& 53& 160& 80& 40& 20& 10\\
5& 16& 8& 4& 2& 1& \\

2050&&&&&&&&&\\
1025& 3076& 1538& 769& 2308& 1154& 577& 1732& 866& 433\\
1300& 650& 325& 976& 488& 244& 122& 61& 184& 92\\
46& 23& 70& 35& 106& 53& 160& 80& 40& 20\\
10& 5& 16& 8& 4& 2& 1& \\

2051&&&&&&&&&\\
6154& 3077& 9232& 4616& 2308& 1154& 577& 1732& 866& 433\\
1300& 650& 325& 976& 488& 244& 122& 61& 184& 92\\
46& 23& 70& 35& 106& 53& 160& 80& 40& 20\\
10& 5& 16& 8& 4& 2& 1& \\

2052&&&&&&&&&\\
1026& 513& 1540& 770& 385& 1156& 578& 289& 868& 434\\
217& 652& 326& 163& 490& 245& 736& 368& 184& 92\\
46& 23& 70& 35& 106& 53& 160& 80& 40& 20\\
10& 5& 16& 8& 4& 2& 1& \\

2053&&&&&&&&&\\
6160& 3080& 1540& 770& 385& 1156& 578& 289& 868& 434\\
217& 652& 326& 163& 490& 245& 736& 368& 184& 92\\
46& 23& 70& 35& 106& 53& 160& 80& 40& 20\\
10& 5& 16& 8& 4& 2& 1& \\

2054&&&&&&&&&\\
1027& 3082& 1541& 4624& 2312& 1156& 578& 289& 868& 434\\
217& 652& 326& 163& 490& 245& 736& 368& 184& 92\\
46& 23& 70& 35& 106& 53& 160& 80& 40& 20\\
10& 5& 16& 8& 4& 2& 1& \\

2055&&&&&&&&&\\
6166& 3083& 9250& 4625& 13876& 6938& 3469& 10408& 5204& 2602\\
1301& 3904& 1952& 976& 488& 244& 122& 61& 184& 92\\
46& 23& 70& 35& 106& 53& 160& 80& 40& 20\\
10& 5& 16& 8& 4& 2& 1& \\

2056&&&&&&&&&\\
1028& 514& 257& 772& 386& 193& 580& 290& 145& 436\\
218& 109& 328& 164& 82& 41& 124& 62& 31& 94\\
47& 142& 71& 214& 107& 322& 161& 484& 242& 121\\
364& 182& 91& 274& 137& 412& 206& 103& 310& 155\\
466& 233& 700& 350& 175& 526& 263& 790& 395& 1186\\
593& 1780& 890& 445& 1336& 668& 334& 167& 502& 251\\
754& 377& 1132& 566& 283& 850& 425& 1276& 638& 319\\
958& 479& 1438& 719& 2158& 1079& 3238& 1619& 4858& 2429\\
7288& 3644& 1822& 911& 2734& 1367& 4102& 2051& 6154& 3077\\
9232& 4616& 2308& 1154& 577& 1732& 866& 433& 1300& 650\\
325& 976& 488& 244& 122& 61& 184& 92& 46& 23\\
70& 35& 106& 53& 160& 80& 40& 20& 10& 5\\
16& 8& 4& 2& 1& \\

2057&&&&&&&&&\\
6172& 3086& 1543& 4630& 2315& 6946& 3473& 10420& 5210& 2605\\
7816& 3908& 1954& 977& 2932& 1466& 733& 2200& 1100& 550\\
275& 826& 413& 1240& 620& 310& 155& 466& 233& 700\\
350& 175& 526& 263& 790& 395& 1186& 593& 1780& 890\\
445& 1336& 668& 334& 167& 502& 251& 754& 377& 1132\\
566& 283& 850& 425& 1276& 638& 319& 958& 479& 1438\\
719& 2158& 1079& 3238& 1619& 4858& 2429& 7288& 3644& 1822\\
911& 2734& 1367& 4102& 2051& 6154& 3077& 9232& 4616& 2308\\
1154& 577& 1732& 866& 433& 1300& 650& 325& 976& 488\\
244& 122& 61& 184& 92& 46& 23& 70& 35& 106\\
53& 160& 80& 40& 20& 10& 5& 16& 8& 4\\
2& 1& \\

2058&&&&&&&&&\\
1029& 3088& 1544& 772& 386& 193& 580& 290& 145& 436\\
218& 109& 328& 164& 82& 41& 124& 62& 31& 94\\
47& 142& 71& 214& 107& 322& 161& 484& 242& 121\\
364& 182& 91& 274& 137& 412& 206& 103& 310& 155\\
466& 233& 700& 350& 175& 526& 263& 790& 395& 1186\\
593& 1780& 890& 445& 1336& 668& 334& 167& 502& 251\\
754& 377& 1132& 566& 283& 850& 425& 1276& 638& 319\\
958& 479& 1438& 719& 2158& 1079& 3238& 1619& 4858& 2429\\
7288& 3644& 1822& 911& 2734& 1367& 4102& 2051& 6154& 3077\\
9232& 4616& 2308& 1154& 577& 1732& 866& 433& 1300& 650\\
325& 976& 488& 244& 122& 61& 184& 92& 46& 23\\
70& 35& 106& 53& 160& 80& 40& 20& 10& 5\\
16& 8& 4& 2& 1& \\

2059&&&&&&&&&\\
6178& 3089& 9268& 4634& 2317& 6952& 3476& 1738& 869& 2608\\
1304& 652& 326& 163& 490& 245& 736& 368& 184& 92\\
46& 23& 70& 35& 106& 53& 160& 80& 40& 20\\
10& 5& 16& 8& 4& 2& 1& \\

2060&&&&&&&&&\\
1030& 515& 1546& 773& 2320& 1160& 580& 290& 145& 436\\
218& 109& 328& 164& 82& 41& 124& 62& 31& 94\\
47& 142& 71& 214& 107& 322& 161& 484& 242& 121\\
364& 182& 91& 274& 137& 412& 206& 103& 310& 155\\
466& 233& 700& 350& 175& 526& 263& 790& 395& 1186\\
593& 1780& 890& 445& 1336& 668& 334& 167& 502& 251\\
754& 377& 1132& 566& 283& 850& 425& 1276& 638& 319\\
958& 479& 1438& 719& 2158& 1079& 3238& 1619& 4858& 2429\\
7288& 3644& 1822& 911& 2734& 1367& 4102& 2051& 6154& 3077\\
9232& 4616& 2308& 1154& 577& 1732& 866& 433& 1300& 650\\
325& 976& 488& 244& 122& 61& 184& 92& 46& 23\\
70& 35& 106& 53& 160& 80& 40& 20& 10& 5\\
16& 8& 4& 2& 1& \\

2061&&&&&&&&&\\
6184& 3092& 1546& 773& 2320& 1160& 580& 290& 145& 436\\
218& 109& 328& 164& 82& 41& 124& 62& 31& 94\\
47& 142& 71& 214& 107& 322& 161& 484& 242& 121\\
364& 182& 91& 274& 137& 412& 206& 103& 310& 155\\
466& 233& 700& 350& 175& 526& 263& 790& 395& 1186\\
593& 1780& 890& 445& 1336& 668& 334& 167& 502& 251\\
754& 377& 1132& 566& 283& 850& 425& 1276& 638& 319\\
958& 479& 1438& 719& 2158& 1079& 3238& 1619& 4858& 2429\\
7288& 3644& 1822& 911& 2734& 1367& 4102& 2051& 6154& 3077\\
9232& 4616& 2308& 1154& 577& 1732& 866& 433& 1300& 650\\
325& 976& 488& 244& 122& 61& 184& 92& 46& 23\\
70& 35& 106& 53& 160& 80& 40& 20& 10& 5\\
16& 8& 4& 2& 1& \\

2062&&&&&&&&&\\
1031& 3094& 1547& 4642& 2321& 6964& 3482& 1741& 5224& 2612\\
1306& 653& 1960& 980& 490& 245& 736& 368& 184& 92\\
46& 23& 70& 35& 106& 53& 160& 80& 40& 20\\
10& 5& 16& 8& 4& 2& 1& \\

2063&&&&&&&&&\\
6190& 3095& 9286& 4643& 13930& 6965& 20896& 10448& 5224& 2612\\
1306& 653& 1960& 980& 490& 245& 736& 368& 184& 92\\
46& 23& 70& 35& 106& 53& 160& 80& 40& 20\\
10& 5& 16& 8& 4& 2& 1& \\

2064&&&&&&&&&\\
1032& 516& 258& 129& 388& 194& 97& 292& 146& 73\\
220& 110& 55& 166& 83& 250& 125& 376& 188& 94\\
47& 142& 71& 214& 107& 322& 161& 484& 242& 121\\
364& 182& 91& 274& 137& 412& 206& 103& 310& 155\\
466& 233& 700& 350& 175& 526& 263& 790& 395& 1186\\
593& 1780& 890& 445& 1336& 668& 334& 167& 502& 251\\
754& 377& 1132& 566& 283& 850& 425& 1276& 638& 319\\
958& 479& 1438& 719& 2158& 1079& 3238& 1619& 4858& 2429\\
7288& 3644& 1822& 911& 2734& 1367& 4102& 2051& 6154& 3077\\
9232& 4616& 2308& 1154& 577& 1732& 866& 433& 1300& 650\\
325& 976& 488& 244& 122& 61& 184& 92& 46& 23\\
70& 35& 106& 53& 160& 80& 40& 20& 10& 5\\
16& 8& 4& 2& 1& \\

2065&&&&&&&&&\\
6196& 3098& 1549& 4648& 2324& 1162& 581& 1744& 872& 436\\
218& 109& 328& 164& 82& 41& 124& 62& 31& 94\\
47& 142& 71& 214& 107& 322& 161& 484& 242& 121\\
364& 182& 91& 274& 137& 412& 206& 103& 310& 155\\
466& 233& 700& 350& 175& 526& 263& 790& 395& 1186\\
593& 1780& 890& 445& 1336& 668& 334& 167& 502& 251\\
754& 377& 1132& 566& 283& 850& 425& 1276& 638& 319\\
958& 479& 1438& 719& 2158& 1079& 3238& 1619& 4858& 2429\\
7288& 3644& 1822& 911& 2734& 1367& 4102& 2051& 6154& 3077\\
9232& 4616& 2308& 1154& 577& 1732& 866& 433& 1300& 650\\
325& 976& 488& 244& 122& 61& 184& 92& 46& 23\\
70& 35& 106& 53& 160& 80& 40& 20& 10& 5\\
16& 8& 4& 2& 1& \\

2066&&&&&&&&&\\
1033& 3100& 1550& 775& 2326& 1163& 3490& 1745& 5236& 2618\\
1309& 3928& 1964& 982& 491& 1474& 737& 2212& 1106& 553\\
1660& 830& 415& 1246& 623& 1870& 935& 2806& 1403& 4210\\
2105& 6316& 3158& 1579& 4738& 2369& 7108& 3554& 1777& 5332\\
2666& 1333& 4000& 2000& 1000& 500& 250& 125& 376& 188\\
94& 47& 142& 71& 214& 107& 322& 161& 484& 242\\
121& 364& 182& 91& 274& 137& 412& 206& 103& 310\\
155& 466& 233& 700& 350& 175& 526& 263& 790& 395\\
1186& 593& 1780& 890& 445& 1336& 668& 334& 167& 502\\
251& 754& 377& 1132& 566& 283& 850& 425& 1276& 638\\
319& 958& 479& 1438& 719& 2158& 1079& 3238& 1619& 4858\\
2429& 7288& 3644& 1822& 911& 2734& 1367& 4102& 2051& 6154\\
3077& 9232& 4616& 2308& 1154& 577& 1732& 866& 433& 1300\\
650& 325& 976& 488& 244& 122& 61& 184& 92& 46\\
23& 70& 35& 106& 53& 160& 80& 40& 20& 10\\
5& 16& 8& 4& 2& 1& \\

2067&&&&&&&&&\\
6202& 3101& 9304& 4652& 2326& 1163& 3490& 1745& 5236& 2618\\
1309& 3928& 1964& 982& 491& 1474& 737& 2212& 1106& 553\\
1660& 830& 415& 1246& 623& 1870& 935& 2806& 1403& 4210\\
2105& 6316& 3158& 1579& 4738& 2369& 7108& 3554& 1777& 5332\\
2666& 1333& 4000& 2000& 1000& 500& 250& 125& 376& 188\\
94& 47& 142& 71& 214& 107& 322& 161& 484& 242\\
121& 364& 182& 91& 274& 137& 412& 206& 103& 310\\
155& 466& 233& 700& 350& 175& 526& 263& 790& 395\\
1186& 593& 1780& 890& 445& 1336& 668& 334& 167& 502\\
251& 754& 377& 1132& 566& 283& 850& 425& 1276& 638\\
319& 958& 479& 1438& 719& 2158& 1079& 3238& 1619& 4858\\
2429& 7288& 3644& 1822& 911& 2734& 1367& 4102& 2051& 6154\\
3077& 9232& 4616& 2308& 1154& 577& 1732& 866& 433& 1300\\
650& 325& 976& 488& 244& 122& 61& 184& 92& 46\\
23& 70& 35& 106& 53& 160& 80& 40& 20& 10\\
5& 16& 8& 4& 2& 1& \\

2068&&&&&&&&&\\
1034& 517& 1552& 776& 388& 194& 97& 292& 146& 73\\
220& 110& 55& 166& 83& 250& 125& 376& 188& 94\\
47& 142& 71& 214& 107& 322& 161& 484& 242& 121\\
364& 182& 91& 274& 137& 412& 206& 103& 310& 155\\
466& 233& 700& 350& 175& 526& 263& 790& 395& 1186\\
593& 1780& 890& 445& 1336& 668& 334& 167& 502& 251\\
754& 377& 1132& 566& 283& 850& 425& 1276& 638& 319\\
958& 479& 1438& 719& 2158& 1079& 3238& 1619& 4858& 2429\\
7288& 3644& 1822& 911& 2734& 1367& 4102& 2051& 6154& 3077\\
9232& 4616& 2308& 1154& 577& 1732& 866& 433& 1300& 650\\
325& 976& 488& 244& 122& 61& 184& 92& 46& 23\\
70& 35& 106& 53& 160& 80& 40& 20& 10& 5\\
16& 8& 4& 2& 1& \\

2069&&&&&&&&&\\
6208& 3104& 1552& 776& 388& 194& 97& 292& 146& 73\\
220& 110& 55& 166& 83& 250& 125& 376& 188& 94\\
47& 142& 71& 214& 107& 322& 161& 484& 242& 121\\
364& 182& 91& 274& 137& 412& 206& 103& 310& 155\\
466& 233& 700& 350& 175& 526& 263& 790& 395& 1186\\
593& 1780& 890& 445& 1336& 668& 334& 167& 502& 251\\
754& 377& 1132& 566& 283& 850& 425& 1276& 638& 319\\
958& 479& 1438& 719& 2158& 1079& 3238& 1619& 4858& 2429\\
7288& 3644& 1822& 911& 2734& 1367& 4102& 2051& 6154& 3077\\
9232& 4616& 2308& 1154& 577& 1732& 866& 433& 1300& 650\\
325& 976& 488& 244& 122& 61& 184& 92& 46& 23\\
70& 35& 106& 53& 160& 80& 40& 20& 10& 5\\
16& 8& 4& 2& 1& \\

2070&&&&&&&&&\\
1035& 3106& 1553& 4660& 2330& 1165& 3496& 1748& 874& 437\\
1312& 656& 328& 164& 82& 41& 124& 62& 31& 94\\
47& 142& 71& 214& 107& 322& 161& 484& 242& 121\\
364& 182& 91& 274& 137& 412& 206& 103& 310& 155\\
466& 233& 700& 350& 175& 526& 263& 790& 395& 1186\\
593& 1780& 890& 445& 1336& 668& 334& 167& 502& 251\\
754& 377& 1132& 566& 283& 850& 425& 1276& 638& 319\\
958& 479& 1438& 719& 2158& 1079& 3238& 1619& 4858& 2429\\
7288& 3644& 1822& 911& 2734& 1367& 4102& 2051& 6154& 3077\\
9232& 4616& 2308& 1154& 577& 1732& 866& 433& 1300& 650\\
325& 976& 488& 244& 122& 61& 184& 92& 46& 23\\
70& 35& 106& 53& 160& 80& 40& 20& 10& 5\\
16& 8& 4& 2& 1& \\

2071&&&&&&&&&\\
6214& 3107& 9322& 4661& 13984& 6992& 3496& 1748& 874& 437\\
1312& 656& 328& 164& 82& 41& 124& 62& 31& 94\\
47& 142& 71& 214& 107& 322& 161& 484& 242& 121\\
364& 182& 91& 274& 137& 412& 206& 103& 310& 155\\
466& 233& 700& 350& 175& 526& 263& 790& 395& 1186\\
593& 1780& 890& 445& 1336& 668& 334& 167& 502& 251\\
754& 377& 1132& 566& 283& 850& 425& 1276& 638& 319\\
958& 479& 1438& 719& 2158& 1079& 3238& 1619& 4858& 2429\\
7288& 3644& 1822& 911& 2734& 1367& 4102& 2051& 6154& 3077\\
9232& 4616& 2308& 1154& 577& 1732& 866& 433& 1300& 650\\
325& 976& 488& 244& 122& 61& 184& 92& 46& 23\\
70& 35& 106& 53& 160& 80& 40& 20& 10& 5\\
16& 8& 4& 2& 1& \\

2072&&&&&&&&&\\
1036& 518& 259& 778& 389& 1168& 584& 292& 146& 73\\
220& 110& 55& 166& 83& 250& 125& 376& 188& 94\\
47& 142& 71& 214& 107& 322& 161& 484& 242& 121\\
364& 182& 91& 274& 137& 412& 206& 103& 310& 155\\
466& 233& 700& 350& 175& 526& 263& 790& 395& 1186\\
593& 1780& 890& 445& 1336& 668& 334& 167& 502& 251\\
754& 377& 1132& 566& 283& 850& 425& 1276& 638& 319\\
958& 479& 1438& 719& 2158& 1079& 3238& 1619& 4858& 2429\\
7288& 3644& 1822& 911& 2734& 1367& 4102& 2051& 6154& 3077\\
9232& 4616& 2308& 1154& 577& 1732& 866& 433& 1300& 650\\
325& 976& 488& 244& 122& 61& 184& 92& 46& 23\\
70& 35& 106& 53& 160& 80& 40& 20& 10& 5\\
16& 8& 4& 2& 1& \\

2073&&&&&&&&&\\
6220& 3110& 1555& 4666& 2333& 7000& 3500& 1750& 875& 2626\\
1313& 3940& 1970& 985& 2956& 1478& 739& 2218& 1109& 3328\\
1664& 832& 416& 208& 104& 52& 26& 13& 40& 20\\
10& 5& 16& 8& 4& 2& 1& \\

2074&&&&&&&&&\\
1037& 3112& 1556& 778& 389& 1168& 584& 292& 146& 73\\
220& 110& 55& 166& 83& 250& 125& 376& 188& 94\\
47& 142& 71& 214& 107& 322& 161& 484& 242& 121\\
364& 182& 91& 274& 137& 412& 206& 103& 310& 155\\
466& 233& 700& 350& 175& 526& 263& 790& 395& 1186\\
593& 1780& 890& 445& 1336& 668& 334& 167& 502& 251\\
754& 377& 1132& 566& 283& 850& 425& 1276& 638& 319\\
958& 479& 1438& 719& 2158& 1079& 3238& 1619& 4858& 2429\\
7288& 3644& 1822& 911& 2734& 1367& 4102& 2051& 6154& 3077\\
9232& 4616& 2308& 1154& 577& 1732& 866& 433& 1300& 650\\
325& 976& 488& 244& 122& 61& 184& 92& 46& 23\\
70& 35& 106& 53& 160& 80& 40& 20& 10& 5\\
16& 8& 4& 2& 1& \\

2075&&&&&&&&&\\
6226& 3113& 9340& 4670& 2335& 7006& 3503& 10510& 5255& 15766\\
7883& 23650& 11825& 35476& 17738& 8869& 26608& 13304& 6652& 3326\\
1663& 4990& 2495& 7486& 3743& 11230& 5615& 16846& 8423& 25270\\
12635& 37906& 18953& 56860& 28430& 14215& 42646& 21323& 63970& 31985\\
95956& 47978& 23989& 71968& 35984& 17992& 8996& 4498& 2249& 6748\\
3374& 1687& 5062& 2531& 7594& 3797& 11392& 5696& 2848& 1424\\
712& 356& 178& 89& 268& 134& 67& 202& 101& 304\\
152& 76& 38& 19& 58& 29& 88& 44& 22& 11\\
34& 17& 52& 26& 13& 40& 20& 10& 5& 16\\
8& 4& 2& 1& \\

2076&&&&&&&&&\\
1038& 519& 1558& 779& 2338& 1169& 3508& 1754& 877& 2632\\
1316& 658& 329& 988& 494& 247& 742& 371& 1114& 557\\
1672& 836& 418& 209& 628& 314& 157& 472& 236& 118\\
59& 178& 89& 268& 134& 67& 202& 101& 304& 152\\
76& 38& 19& 58& 29& 88& 44& 22& 11& 34\\
17& 52& 26& 13& 40& 20& 10& 5& 16& 8\\
4& 2& 1& \\

2077&&&&&&&&&\\
6232& 3116& 1558& 779& 2338& 1169& 3508& 1754& 877& 2632\\
1316& 658& 329& 988& 494& 247& 742& 371& 1114& 557\\
1672& 836& 418& 209& 628& 314& 157& 472& 236& 118\\
59& 178& 89& 268& 134& 67& 202& 101& 304& 152\\
76& 38& 19& 58& 29& 88& 44& 22& 11& 34\\
17& 52& 26& 13& 40& 20& 10& 5& 16& 8\\
4& 2& 1& \\

2078&&&&&&&&&\\
1039& 3118& 1559& 4678& 2339& 7018& 3509& 10528& 5264& 2632\\
1316& 658& 329& 988& 494& 247& 742& 371& 1114& 557\\
1672& 836& 418& 209& 628& 314& 157& 472& 236& 118\\
59& 178& 89& 268& 134& 67& 202& 101& 304& 152\\
76& 38& 19& 58& 29& 88& 44& 22& 11& 34\\
17& 52& 26& 13& 40& 20& 10& 5& 16& 8\\
4& 2& 1& \\

2079&&&&&&&&&\\
6238& 3119& 9358& 4679& 14038& 7019& 21058& 10529& 31588& 15794\\
7897& 23692& 11846& 5923& 17770& 8885& 26656& 13328& 6664& 3332\\
1666& 833& 2500& 1250& 625& 1876& 938& 469& 1408& 704\\
352& 176& 88& 44& 22& 11& 34& 17& 52& 26\\
13& 40& 20& 10& 5& 16& 8& 4& 2& 1\\

2080&&&&&&&&&\\
1040& 520& 260& 130& 65& 196& 98& 49& 148& 74\\
37& 112& 56& 28& 14& 7& 22& 11& 34& 17\\
52& 26& 13& 40& 20& 10& 5& 16& 8& 4\\
2& 1& \\

2081&&&&&&&&&\\
6244& 3122& 1561& 4684& 2342& 1171& 3514& 1757& 5272& 2636\\
1318& 659& 1978& 989& 2968& 1484& 742& 371& 1114& 557\\
1672& 836& 418& 209& 628& 314& 157& 472& 236& 118\\
59& 178& 89& 268& 134& 67& 202& 101& 304& 152\\
76& 38& 19& 58& 29& 88& 44& 22& 11& 34\\
17& 52& 26& 13& 40& 20& 10& 5& 16& 8\\
4& 2& 1& \\

2082&&&&&&&&&\\
1041& 3124& 1562& 781& 2344& 1172& 586& 293& 880& 440\\
220& 110& 55& 166& 83& 250& 125& 376& 188& 94\\
47& 142& 71& 214& 107& 322& 161& 484& 242& 121\\
364& 182& 91& 274& 137& 412& 206& 103& 310& 155\\
466& 233& 700& 350& 175& 526& 263& 790& 395& 1186\\
593& 1780& 890& 445& 1336& 668& 334& 167& 502& 251\\
754& 377& 1132& 566& 283& 850& 425& 1276& 638& 319\\
958& 479& 1438& 719& 2158& 1079& 3238& 1619& 4858& 2429\\
7288& 3644& 1822& 911& 2734& 1367& 4102& 2051& 6154& 3077\\
9232& 4616& 2308& 1154& 577& 1732& 866& 433& 1300& 650\\
325& 976& 488& 244& 122& 61& 184& 92& 46& 23\\
70& 35& 106& 53& 160& 80& 40& 20& 10& 5\\
16& 8& 4& 2& 1& \\

2083&&&&&&&&&\\
6250& 3125& 9376& 4688& 2344& 1172& 586& 293& 880& 440\\
220& 110& 55& 166& 83& 250& 125& 376& 188& 94\\
47& 142& 71& 214& 107& 322& 161& 484& 242& 121\\
364& 182& 91& 274& 137& 412& 206& 103& 310& 155\\
466& 233& 700& 350& 175& 526& 263& 790& 395& 1186\\
593& 1780& 890& 445& 1336& 668& 334& 167& 502& 251\\
754& 377& 1132& 566& 283& 850& 425& 1276& 638& 319\\
958& 479& 1438& 719& 2158& 1079& 3238& 1619& 4858& 2429\\
7288& 3644& 1822& 911& 2734& 1367& 4102& 2051& 6154& 3077\\
9232& 4616& 2308& 1154& 577& 1732& 866& 433& 1300& 650\\
325& 976& 488& 244& 122& 61& 184& 92& 46& 23\\
70& 35& 106& 53& 160& 80& 40& 20& 10& 5\\
16& 8& 4& 2& 1& \\

2084&&&&&&&&&\\
1042& 521& 1564& 782& 391& 1174& 587& 1762& 881& 2644\\
1322& 661& 1984& 992& 496& 248& 124& 62& 31& 94\\
47& 142& 71& 214& 107& 322& 161& 484& 242& 121\\
364& 182& 91& 274& 137& 412& 206& 103& 310& 155\\
466& 233& 700& 350& 175& 526& 263& 790& 395& 1186\\
593& 1780& 890& 445& 1336& 668& 334& 167& 502& 251\\
754& 377& 1132& 566& 283& 850& 425& 1276& 638& 319\\
958& 479& 1438& 719& 2158& 1079& 3238& 1619& 4858& 2429\\
7288& 3644& 1822& 911& 2734& 1367& 4102& 2051& 6154& 3077\\
9232& 4616& 2308& 1154& 577& 1732& 866& 433& 1300& 650\\
325& 976& 488& 244& 122& 61& 184& 92& 46& 23\\
70& 35& 106& 53& 160& 80& 40& 20& 10& 5\\
16& 8& 4& 2& 1& \\

2085&&&&&&&&&\\
6256& 3128& 1564& 782& 391& 1174& 587& 1762& 881& 2644\\
1322& 661& 1984& 992& 496& 248& 124& 62& 31& 94\\
47& 142& 71& 214& 107& 322& 161& 484& 242& 121\\
364& 182& 91& 274& 137& 412& 206& 103& 310& 155\\
466& 233& 700& 350& 175& 526& 263& 790& 395& 1186\\
593& 1780& 890& 445& 1336& 668& 334& 167& 502& 251\\
754& 377& 1132& 566& 283& 850& 425& 1276& 638& 319\\
958& 479& 1438& 719& 2158& 1079& 3238& 1619& 4858& 2429\\
7288& 3644& 1822& 911& 2734& 1367& 4102& 2051& 6154& 3077\\
9232& 4616& 2308& 1154& 577& 1732& 866& 433& 1300& 650\\
325& 976& 488& 244& 122& 61& 184& 92& 46& 23\\
70& 35& 106& 53& 160& 80& 40& 20& 10& 5\\
16& 8& 4& 2& 1& \\

2086&&&&&&&&&\\
1043& 3130& 1565& 4696& 2348& 1174& 587& 1762& 881& 2644\\
1322& 661& 1984& 992& 496& 248& 124& 62& 31& 94\\
47& 142& 71& 214& 107& 322& 161& 484& 242& 121\\
364& 182& 91& 274& 137& 412& 206& 103& 310& 155\\
466& 233& 700& 350& 175& 526& 263& 790& 395& 1186\\
593& 1780& 890& 445& 1336& 668& 334& 167& 502& 251\\
754& 377& 1132& 566& 283& 850& 425& 1276& 638& 319\\
958& 479& 1438& 719& 2158& 1079& 3238& 1619& 4858& 2429\\
7288& 3644& 1822& 911& 2734& 1367& 4102& 2051& 6154& 3077\\
9232& 4616& 2308& 1154& 577& 1732& 866& 433& 1300& 650\\
325& 976& 488& 244& 122& 61& 184& 92& 46& 23\\
70& 35& 106& 53& 160& 80& 40& 20& 10& 5\\
16& 8& 4& 2& 1& \\

2087&&&&&&&&&\\
6262& 3131& 9394& 4697& 14092& 7046& 3523& 10570& 5285& 15856\\
7928& 3964& 1982& 991& 2974& 1487& 4462& 2231& 6694& 3347\\
10042& 5021& 15064& 7532& 3766& 1883& 5650& 2825& 8476& 4238\\
2119& 6358& 3179& 9538& 4769& 14308& 7154& 3577& 10732& 5366\\
2683& 8050& 4025& 12076& 6038& 3019& 9058& 4529& 13588& 6794\\
3397& 10192& 5096& 2548& 1274& 637& 1912& 956& 478& 239\\
718& 359& 1078& 539& 1618& 809& 2428& 1214& 607& 1822\\
911& 2734& 1367& 4102& 2051& 6154& 3077& 9232& 4616& 2308\\
1154& 577& 1732& 866& 433& 1300& 650& 325& 976& 488\\
244& 122& 61& 184& 92& 46& 23& 70& 35& 106\\
53& 160& 80& 40& 20& 10& 5& 16& 8& 4\\
2& 1& \\

2088&&&&&&&&&\\
1044& 522& 261& 784& 392& 196& 98& 49& 148& 74\\
37& 112& 56& 28& 14& 7& 22& 11& 34& 17\\
52& 26& 13& 40& 20& 10& 5& 16& 8& 4\\
2& 1& \\

2089&&&&&&&&&\\
6268& 3134& 1567& 4702& 2351& 7054& 3527& 10582& 5291& 15874\\
7937& 23812& 11906& 5953& 17860& 8930& 4465& 13396& 6698& 3349\\
10048& 5024& 2512& 1256& 628& 314& 157& 472& 236& 118\\
59& 178& 89& 268& 134& 67& 202& 101& 304& 152\\
76& 38& 19& 58& 29& 88& 44& 22& 11& 34\\
17& 52& 26& 13& 40& 20& 10& 5& 16& 8\\
4& 2& 1& \\

2090&&&&&&&&&\\
1045& 3136& 1568& 784& 392& 196& 98& 49& 148& 74\\
37& 112& 56& 28& 14& 7& 22& 11& 34& 17\\
52& 26& 13& 40& 20& 10& 5& 16& 8& 4\\
2& 1& \\

2091&&&&&&&&&\\
6274& 3137& 9412& 4706& 2353& 7060& 3530& 1765& 5296& 2648\\
1324& 662& 331& 994& 497& 1492& 746& 373& 1120& 560\\
280& 140& 70& 35& 106& 53& 160& 80& 40& 20\\
10& 5& 16& 8& 4& 2& 1& \\

2092&&&&&&&&&\\
1046& 523& 1570& 785& 2356& 1178& 589& 1768& 884& 442\\
221& 664& 332& 166& 83& 250& 125& 376& 188& 94\\
47& 142& 71& 214& 107& 322& 161& 484& 242& 121\\
364& 182& 91& 274& 137& 412& 206& 103& 310& 155\\
466& 233& 700& 350& 175& 526& 263& 790& 395& 1186\\
593& 1780& 890& 445& 1336& 668& 334& 167& 502& 251\\
754& 377& 1132& 566& 283& 850& 425& 1276& 638& 319\\
958& 479& 1438& 719& 2158& 1079& 3238& 1619& 4858& 2429\\
7288& 3644& 1822& 911& 2734& 1367& 4102& 2051& 6154& 3077\\
9232& 4616& 2308& 1154& 577& 1732& 866& 433& 1300& 650\\
325& 976& 488& 244& 122& 61& 184& 92& 46& 23\\
70& 35& 106& 53& 160& 80& 40& 20& 10& 5\\
16& 8& 4& 2& 1& \\

2093&&&&&&&&&\\
6280& 3140& 1570& 785& 2356& 1178& 589& 1768& 884& 442\\
221& 664& 332& 166& 83& 250& 125& 376& 188& 94\\
47& 142& 71& 214& 107& 322& 161& 484& 242& 121\\
364& 182& 91& 274& 137& 412& 206& 103& 310& 155\\
466& 233& 700& 350& 175& 526& 263& 790& 395& 1186\\
593& 1780& 890& 445& 1336& 668& 334& 167& 502& 251\\
754& 377& 1132& 566& 283& 850& 425& 1276& 638& 319\\
958& 479& 1438& 719& 2158& 1079& 3238& 1619& 4858& 2429\\
7288& 3644& 1822& 911& 2734& 1367& 4102& 2051& 6154& 3077\\
9232& 4616& 2308& 1154& 577& 1732& 866& 433& 1300& 650\\
325& 976& 488& 244& 122& 61& 184& 92& 46& 23\\
70& 35& 106& 53& 160& 80& 40& 20& 10& 5\\
16& 8& 4& 2& 1& \\

2094&&&&&&&&&\\
1047& 3142& 1571& 4714& 2357& 7072& 3536& 1768& 884& 442\\
221& 664& 332& 166& 83& 250& 125& 376& 188& 94\\
47& 142& 71& 214& 107& 322& 161& 484& 242& 121\\
364& 182& 91& 274& 137& 412& 206& 103& 310& 155\\
466& 233& 700& 350& 175& 526& 263& 790& 395& 1186\\
593& 1780& 890& 445& 1336& 668& 334& 167& 502& 251\\
754& 377& 1132& 566& 283& 850& 425& 1276& 638& 319\\
958& 479& 1438& 719& 2158& 1079& 3238& 1619& 4858& 2429\\
7288& 3644& 1822& 911& 2734& 1367& 4102& 2051& 6154& 3077\\
9232& 4616& 2308& 1154& 577& 1732& 866& 433& 1300& 650\\
325& 976& 488& 244& 122& 61& 184& 92& 46& 23\\
70& 35& 106& 53& 160& 80& 40& 20& 10& 5\\
16& 8& 4& 2& 1& \\

2095&&&&&&&&&\\
6286& 3143& 9430& 4715& 14146& 7073& 21220& 10610& 5305& 15916\\
7958& 3979& 11938& 5969& 17908& 8954& 4477& 13432& 6716& 3358\\
1679& 5038& 2519& 7558& 3779& 11338& 5669& 17008& 8504& 4252\\
2126& 1063& 3190& 1595& 4786& 2393& 7180& 3590& 1795& 5386\\
2693& 8080& 4040& 2020& 1010& 505& 1516& 758& 379& 1138\\
569& 1708& 854& 427& 1282& 641& 1924& 962& 481& 1444\\
722& 361& 1084& 542& 271& 814& 407& 1222& 611& 1834\\
917& 2752& 1376& 688& 344& 172& 86& 43& 130& 65\\
196& 98& 49& 148& 74& 37& 112& 56& 28& 14\\
7& 22& 11& 34& 17& 52& 26& 13& 40& 20\\
10& 5& 16& 8& 4& 2& 1& \\

2096&&&&&&&&&\\
1048& 524& 262& 131& 394& 197& 592& 296& 148& 74\\
37& 112& 56& 28& 14& 7& 22& 11& 34& 17\\
52& 26& 13& 40& 20& 10& 5& 16& 8& 4\\
2& 1& \\

2097&&&&&&&&&\\
6292& 3146& 1573& 4720& 2360& 1180& 590& 295& 886& 443\\
1330& 665& 1996& 998& 499& 1498& 749& 2248& 1124& 562\\
281& 844& 422& 211& 634& 317& 952& 476& 238& 119\\
358& 179& 538& 269& 808& 404& 202& 101& 304& 152\\
76& 38& 19& 58& 29& 88& 44& 22& 11& 34\\
17& 52& 26& 13& 40& 20& 10& 5& 16& 8\\
4& 2& 1& \\

2098&&&&&&&&&\\
1049& 3148& 1574& 787& 2362& 1181& 3544& 1772& 886& 443\\
1330& 665& 1996& 998& 499& 1498& 749& 2248& 1124& 562\\
281& 844& 422& 211& 634& 317& 952& 476& 238& 119\\
358& 179& 538& 269& 808& 404& 202& 101& 304& 152\\
76& 38& 19& 58& 29& 88& 44& 22& 11& 34\\
17& 52& 26& 13& 40& 20& 10& 5& 16& 8\\
4& 2& 1& \\

2099&&&&&&&&&\\
6298& 3149& 9448& 4724& 2362& 1181& 3544& 1772& 886& 443\\
1330& 665& 1996& 998& 499& 1498& 749& 2248& 1124& 562\\
281& 844& 422& 211& 634& 317& 952& 476& 238& 119\\
358& 179& 538& 269& 808& 404& 202& 101& 304& 152\\
76& 38& 19& 58& 29& 88& 44& 22& 11& 34\\
17& 52& 26& 13& 40& 20& 10& 5& 16& 8\\
4& 2& 1& \\

2100&&&&&&&&&\\
1050& 525& 1576& 788& 394& 197& 592& 296& 148& 74\\
37& 112& 56& 28& 14& 7& 22& 11& 34& 17\\
52& 26& 13& 40& 20& 10& 5& 16& 8& 4\\
2& 1& \\

2101&&&&&&&&&\\
6304& 3152& 1576& 788& 394& 197& 592& 296& 148& 74\\
37& 112& 56& 28& 14& 7& 22& 11& 34& 17\\
52& 26& 13& 40& 20& 10& 5& 16& 8& 4\\
2& 1& \\

2102&&&&&&&&&\\
1051& 3154& 1577& 4732& 2366& 1183& 3550& 1775& 5326& 2663\\
7990& 3995& 11986& 5993& 17980& 8990& 4495& 13486& 6743& 20230\\
10115& 30346& 15173& 45520& 22760& 11380& 5690& 2845& 8536& 4268\\
2134& 1067& 3202& 1601& 4804& 2402& 1201& 3604& 1802& 901\\
2704& 1352& 676& 338& 169& 508& 254& 127& 382& 191\\
574& 287& 862& 431& 1294& 647& 1942& 971& 2914& 1457\\
4372& 2186& 1093& 3280& 1640& 820& 410& 205& 616& 308\\
154& 77& 232& 116& 58& 29& 88& 44& 22& 11\\
34& 17& 52& 26& 13& 40& 20& 10& 5& 16\\
8& 4& 2& 1& \\

2103&&&&&&&&&\\
6310& 3155& 9466& 4733& 14200& 7100& 3550& 1775& 5326& 2663\\
7990& 3995& 11986& 5993& 17980& 8990& 4495& 13486& 6743& 20230\\
10115& 30346& 15173& 45520& 22760& 11380& 5690& 2845& 8536& 4268\\
2134& 1067& 3202& 1601& 4804& 2402& 1201& 3604& 1802& 901\\
2704& 1352& 676& 338& 169& 508& 254& 127& 382& 191\\
574& 287& 862& 431& 1294& 647& 1942& 971& 2914& 1457\\
4372& 2186& 1093& 3280& 1640& 820& 410& 205& 616& 308\\
154& 77& 232& 116& 58& 29& 88& 44& 22& 11\\
34& 17& 52& 26& 13& 40& 20& 10& 5& 16\\
8& 4& 2& 1& \\

2104&&&&&&&&&\\
1052& 526& 263& 790& 395& 1186& 593& 1780& 890& 445\\
1336& 668& 334& 167& 502& 251& 754& 377& 1132& 566\\
283& 850& 425& 1276& 638& 319& 958& 479& 1438& 719\\
2158& 1079& 3238& 1619& 4858& 2429& 7288& 3644& 1822& 911\\
2734& 1367& 4102& 2051& 6154& 3077& 9232& 4616& 2308& 1154\\
577& 1732& 866& 433& 1300& 650& 325& 976& 488& 244\\
122& 61& 184& 92& 46& 23& 70& 35& 106& 53\\
160& 80& 40& 20& 10& 5& 16& 8& 4& 2\\
1& \\

2105&&&&&&&&&\\
6316& 3158& 1579& 4738& 2369& 7108& 3554& 1777& 5332& 2666\\
1333& 4000& 2000& 1000& 500& 250& 125& 376& 188& 94\\
47& 142& 71& 214& 107& 322& 161& 484& 242& 121\\
364& 182& 91& 274& 137& 412& 206& 103& 310& 155\\
466& 233& 700& 350& 175& 526& 263& 790& 395& 1186\\
593& 1780& 890& 445& 1336& 668& 334& 167& 502& 251\\
754& 377& 1132& 566& 283& 850& 425& 1276& 638& 319\\
958& 479& 1438& 719& 2158& 1079& 3238& 1619& 4858& 2429\\
7288& 3644& 1822& 911& 2734& 1367& 4102& 2051& 6154& 3077\\
9232& 4616& 2308& 1154& 577& 1732& 866& 433& 1300& 650\\
325& 976& 488& 244& 122& 61& 184& 92& 46& 23\\
70& 35& 106& 53& 160& 80& 40& 20& 10& 5\\
16& 8& 4& 2& 1& \\

2106&&&&&&&&&\\
1053& 3160& 1580& 790& 395& 1186& 593& 1780& 890& 445\\
1336& 668& 334& 167& 502& 251& 754& 377& 1132& 566\\
283& 850& 425& 1276& 638& 319& 958& 479& 1438& 719\\
2158& 1079& 3238& 1619& 4858& 2429& 7288& 3644& 1822& 911\\
2734& 1367& 4102& 2051& 6154& 3077& 9232& 4616& 2308& 1154\\
577& 1732& 866& 433& 1300& 650& 325& 976& 488& 244\\
122& 61& 184& 92& 46& 23& 70& 35& 106& 53\\
160& 80& 40& 20& 10& 5& 16& 8& 4& 2\\
1& \\

2107&&&&&&&&&\\
6322& 3161& 9484& 4742& 2371& 7114& 3557& 10672& 5336& 2668\\
1334& 667& 2002& 1001& 3004& 1502& 751& 2254& 1127& 3382\\
1691& 5074& 2537& 7612& 3806& 1903& 5710& 2855& 8566& 4283\\
12850& 6425& 19276& 9638& 4819& 14458& 7229& 21688& 10844& 5422\\
2711& 8134& 4067& 12202& 6101& 18304& 9152& 4576& 2288& 1144\\
572& 286& 143& 430& 215& 646& 323& 970& 485& 1456\\
728& 364& 182& 91& 274& 137& 412& 206& 103& 310\\
155& 466& 233& 700& 350& 175& 526& 263& 790& 395\\
1186& 593& 1780& 890& 445& 1336& 668& 334& 167& 502\\
251& 754& 377& 1132& 566& 283& 850& 425& 1276& 638\\
319& 958& 479& 1438& 719& 2158& 1079& 3238& 1619& 4858\\
2429& 7288& 3644& 1822& 911& 2734& 1367& 4102& 2051& 6154\\
3077& 9232& 4616& 2308& 1154& 577& 1732& 866& 433& 1300\\
650& 325& 976& 488& 244& 122& 61& 184& 92& 46\\
23& 70& 35& 106& 53& 160& 80& 40& 20& 10\\
5& 16& 8& 4& 2& 1& \\

2108&&&&&&&&&\\
1054& 527& 1582& 791& 2374& 1187& 3562& 1781& 5344& 2672\\
1336& 668& 334& 167& 502& 251& 754& 377& 1132& 566\\
283& 850& 425& 1276& 638& 319& 958& 479& 1438& 719\\
2158& 1079& 3238& 1619& 4858& 2429& 7288& 3644& 1822& 911\\
2734& 1367& 4102& 2051& 6154& 3077& 9232& 4616& 2308& 1154\\
577& 1732& 866& 433& 1300& 650& 325& 976& 488& 244\\
122& 61& 184& 92& 46& 23& 70& 35& 106& 53\\
160& 80& 40& 20& 10& 5& 16& 8& 4& 2\\
1& \\

2109&&&&&&&&&\\
6328& 3164& 1582& 791& 2374& 1187& 3562& 1781& 5344& 2672\\
1336& 668& 334& 167& 502& 251& 754& 377& 1132& 566\\
283& 850& 425& 1276& 638& 319& 958& 479& 1438& 719\\
2158& 1079& 3238& 1619& 4858& 2429& 7288& 3644& 1822& 911\\
2734& 1367& 4102& 2051& 6154& 3077& 9232& 4616& 2308& 1154\\
577& 1732& 866& 433& 1300& 650& 325& 976& 488& 244\\
122& 61& 184& 92& 46& 23& 70& 35& 106& 53\\
160& 80& 40& 20& 10& 5& 16& 8& 4& 2\\
1& \\

2110&&&&&&&&&\\
1055& 3166& 1583& 4750& 2375& 7126& 3563& 10690& 5345& 16036\\
8018& 4009& 12028& 6014& 3007& 9022& 4511& 13534& 6767& 20302\\
10151& 30454& 15227& 45682& 22841& 68524& 34262& 17131& 51394& 25697\\
77092& 38546& 19273& 57820& 28910& 14455& 43366& 21683& 65050& 32525\\
97576& 48788& 24394& 12197& 36592& 18296& 9148& 4574& 2287& 6862\\
3431& 10294& 5147& 15442& 7721& 23164& 11582& 5791& 17374& 8687\\
26062& 13031& 39094& 19547& 58642& 29321& 87964& 43982& 21991& 65974\\
32987& 98962& 49481& 148444& 74222& 37111& 111334& 55667& 167002& 83501\\
250504& 125252& 62626& 31313& 93940& 46970& 23485& 70456& 35228& 17614\\
8807& 26422& 13211& 39634& 19817& 59452& 29726& 14863& 44590& 22295\\
66886& 33443& 100330& 50165& 150496& 75248& 37624& 18812& 9406& 4703\\
14110& 7055& 21166& 10583& 31750& 15875& 47626& 23813& 71440& 35720\\
17860& 8930& 4465& 13396& 6698& 3349& 10048& 5024& 2512& 1256\\
628& 314& 157& 472& 236& 118& 59& 178& 89& 268\\
134& 67& 202& 101& 304& 152& 76& 38& 19& 58\\
29& 88& 44& 22& 11& 34& 17& 52& 26& 13\\
40& 20& 10& 5& 16& 8& 4& 2& 1& \\

2111&&&&&&&&&\\
6334& 3167& 9502& 4751& 14254& 7127& 21382& 10691& 32074& 16037\\
48112& 24056& 12028& 6014& 3007& 9022& 4511& 13534& 6767& 20302\\
10151& 30454& 15227& 45682& 22841& 68524& 34262& 17131& 51394& 25697\\
77092& 38546& 19273& 57820& 28910& 14455& 43366& 21683& 65050& 32525\\
97576& 48788& 24394& 12197& 36592& 18296& 9148& 4574& 2287& 6862\\
3431& 10294& 5147& 15442& 7721& 23164& 11582& 5791& 17374& 8687\\
26062& 13031& 39094& 19547& 58642& 29321& 87964& 43982& 21991& 65974\\
32987& 98962& 49481& 148444& 74222& 37111& 111334& 55667& 167002& 83501\\
250504& 125252& 62626& 31313& 93940& 46970& 23485& 70456& 35228& 17614\\
8807& 26422& 13211& 39634& 19817& 59452& 29726& 14863& 44590& 22295\\
66886& 33443& 100330& 50165& 150496& 75248& 37624& 18812& 9406& 4703\\
14110& 7055& 21166& 10583& 31750& 15875& 47626& 23813& 71440& 35720\\
17860& 8930& 4465& 13396& 6698& 3349& 10048& 5024& 2512& 1256\\
628& 314& 157& 472& 236& 118& 59& 178& 89& 268\\
134& 67& 202& 101& 304& 152& 76& 38& 19& 58\\
29& 88& 44& 22& 11& 34& 17& 52& 26& 13\\
40& 20& 10& 5& 16& 8& 4& 2& 1& \\

2112&&&&&&&&&\\
1056& 528& 264& 132& 66& 33& 100& 50& 25& 76\\
38& 19& 58& 29& 88& 44& 22& 11& 34& 17\\
52& 26& 13& 40& 20& 10& 5& 16& 8& 4\\
2& 1& \\

2113&&&&&&&&&\\
6340& 3170& 1585& 4756& 2378& 1189& 3568& 1784& 892& 446\\
223& 670& 335& 1006& 503& 1510& 755& 2266& 1133& 3400\\
1700& 850& 425& 1276& 638& 319& 958& 479& 1438& 719\\
2158& 1079& 3238& 1619& 4858& 2429& 7288& 3644& 1822& 911\\
2734& 1367& 4102& 2051& 6154& 3077& 9232& 4616& 2308& 1154\\
577& 1732& 866& 433& 1300& 650& 325& 976& 488& 244\\
122& 61& 184& 92& 46& 23& 70& 35& 106& 53\\
160& 80& 40& 20& 10& 5& 16& 8& 4& 2\\
1& \\

2114&&&&&&&&&\\
1057& 3172& 1586& 793& 2380& 1190& 595& 1786& 893& 2680\\
1340& 670& 335& 1006& 503& 1510& 755& 2266& 1133& 3400\\
1700& 850& 425& 1276& 638& 319& 958& 479& 1438& 719\\
2158& 1079& 3238& 1619& 4858& 2429& 7288& 3644& 1822& 911\\
2734& 1367& 4102& 2051& 6154& 3077& 9232& 4616& 2308& 1154\\
577& 1732& 866& 433& 1300& 650& 325& 976& 488& 244\\
122& 61& 184& 92& 46& 23& 70& 35& 106& 53\\
160& 80& 40& 20& 10& 5& 16& 8& 4& 2\\
1& \\

2115&&&&&&&&&\\
6346& 3173& 9520& 4760& 2380& 1190& 595& 1786& 893& 2680\\
1340& 670& 335& 1006& 503& 1510& 755& 2266& 1133& 3400\\
1700& 850& 425& 1276& 638& 319& 958& 479& 1438& 719\\
2158& 1079& 3238& 1619& 4858& 2429& 7288& 3644& 1822& 911\\
2734& 1367& 4102& 2051& 6154& 3077& 9232& 4616& 2308& 1154\\
577& 1732& 866& 433& 1300& 650& 325& 976& 488& 244\\
122& 61& 184& 92& 46& 23& 70& 35& 106& 53\\
160& 80& 40& 20& 10& 5& 16& 8& 4& 2\\
1& \\

2116&&&&&&&&&\\
1058& 529& 1588& 794& 397& 1192& 596& 298& 149& 448\\
224& 112& 56& 28& 14& 7& 22& 11& 34& 17\\
52& 26& 13& 40& 20& 10& 5& 16& 8& 4\\
2& 1& \\

2117&&&&&&&&&\\
6352& 3176& 1588& 794& 397& 1192& 596& 298& 149& 448\\
224& 112& 56& 28& 14& 7& 22& 11& 34& 17\\
52& 26& 13& 40& 20& 10& 5& 16& 8& 4\\
2& 1& \\

2118&&&&&&&&&\\
1059& 3178& 1589& 4768& 2384& 1192& 596& 298& 149& 448\\
224& 112& 56& 28& 14& 7& 22& 11& 34& 17\\
52& 26& 13& 40& 20& 10& 5& 16& 8& 4\\
2& 1& \\

2119&&&&&&&&&\\
6358& 3179& 9538& 4769& 14308& 7154& 3577& 10732& 5366& 2683\\
8050& 4025& 12076& 6038& 3019& 9058& 4529& 13588& 6794& 3397\\
10192& 5096& 2548& 1274& 637& 1912& 956& 478& 239& 718\\
359& 1078& 539& 1618& 809& 2428& 1214& 607& 1822& 911\\
2734& 1367& 4102& 2051& 6154& 3077& 9232& 4616& 2308& 1154\\
577& 1732& 866& 433& 1300& 650& 325& 976& 488& 244\\
122& 61& 184& 92& 46& 23& 70& 35& 106& 53\\
160& 80& 40& 20& 10& 5& 16& 8& 4& 2\\
1& \\

2120&&&&&&&&&\\
1060& 530& 265& 796& 398& 199& 598& 299& 898& 449\\
1348& 674& 337& 1012& 506& 253& 760& 380& 190& 95\\
286& 143& 430& 215& 646& 323& 970& 485& 1456& 728\\
364& 182& 91& 274& 137& 412& 206& 103& 310& 155\\
466& 233& 700& 350& 175& 526& 263& 790& 395& 1186\\
593& 1780& 890& 445& 1336& 668& 334& 167& 502& 251\\
754& 377& 1132& 566& 283& 850& 425& 1276& 638& 319\\
958& 479& 1438& 719& 2158& 1079& 3238& 1619& 4858& 2429\\
7288& 3644& 1822& 911& 2734& 1367& 4102& 2051& 6154& 3077\\
9232& 4616& 2308& 1154& 577& 1732& 866& 433& 1300& 650\\
325& 976& 488& 244& 122& 61& 184& 92& 46& 23\\
70& 35& 106& 53& 160& 80& 40& 20& 10& 5\\
16& 8& 4& 2& 1& \\

2121&&&&&&&&&\\
6364& 3182& 1591& 4774& 2387& 7162& 3581& 10744& 5372& 2686\\
1343& 4030& 2015& 6046& 3023& 9070& 4535& 13606& 6803& 20410\\
10205& 30616& 15308& 7654& 3827& 11482& 5741& 17224& 8612& 4306\\
2153& 6460& 3230& 1615& 4846& 2423& 7270& 3635& 10906& 5453\\
16360& 8180& 4090& 2045& 6136& 3068& 1534& 767& 2302& 1151\\
3454& 1727& 5182& 2591& 7774& 3887& 11662& 5831& 17494& 8747\\
26242& 13121& 39364& 19682& 9841& 29524& 14762& 7381& 22144& 11072\\
5536& 2768& 1384& 692& 346& 173& 520& 260& 130& 65\\
196& 98& 49& 148& 74& 37& 112& 56& 28& 14\\
7& 22& 11& 34& 17& 52& 26& 13& 40& 20\\
10& 5& 16& 8& 4& 2& 1& \\

2122&&&&&&&&&\\
1061& 3184& 1592& 796& 398& 199& 598& 299& 898& 449\\
1348& 674& 337& 1012& 506& 253& 760& 380& 190& 95\\
286& 143& 430& 215& 646& 323& 970& 485& 1456& 728\\
364& 182& 91& 274& 137& 412& 206& 103& 310& 155\\
466& 233& 700& 350& 175& 526& 263& 790& 395& 1186\\
593& 1780& 890& 445& 1336& 668& 334& 167& 502& 251\\
754& 377& 1132& 566& 283& 850& 425& 1276& 638& 319\\
958& 479& 1438& 719& 2158& 1079& 3238& 1619& 4858& 2429\\
7288& 3644& 1822& 911& 2734& 1367& 4102& 2051& 6154& 3077\\
9232& 4616& 2308& 1154& 577& 1732& 866& 433& 1300& 650\\
325& 976& 488& 244& 122& 61& 184& 92& 46& 23\\
70& 35& 106& 53& 160& 80& 40& 20& 10& 5\\
16& 8& 4& 2& 1& \\

2123&&&&&&&&&\\
6370& 3185& 9556& 4778& 2389& 7168& 3584& 1792& 896& 448\\
224& 112& 56& 28& 14& 7& 22& 11& 34& 17\\
52& 26& 13& 40& 20& 10& 5& 16& 8& 4\\
2& 1& \\

2124&&&&&&&&&\\
1062& 531& 1594& 797& 2392& 1196& 598& 299& 898& 449\\
1348& 674& 337& 1012& 506& 253& 760& 380& 190& 95\\
286& 143& 430& 215& 646& 323& 970& 485& 1456& 728\\
364& 182& 91& 274& 137& 412& 206& 103& 310& 155\\
466& 233& 700& 350& 175& 526& 263& 790& 395& 1186\\
593& 1780& 890& 445& 1336& 668& 334& 167& 502& 251\\
754& 377& 1132& 566& 283& 850& 425& 1276& 638& 319\\
958& 479& 1438& 719& 2158& 1079& 3238& 1619& 4858& 2429\\
7288& 3644& 1822& 911& 2734& 1367& 4102& 2051& 6154& 3077\\
9232& 4616& 2308& 1154& 577& 1732& 866& 433& 1300& 650\\
325& 976& 488& 244& 122& 61& 184& 92& 46& 23\\
70& 35& 106& 53& 160& 80& 40& 20& 10& 5\\
16& 8& 4& 2& 1& \\

2125&&&&&&&&&\\
6376& 3188& 1594& 797& 2392& 1196& 598& 299& 898& 449\\
1348& 674& 337& 1012& 506& 253& 760& 380& 190& 95\\
286& 143& 430& 215& 646& 323& 970& 485& 1456& 728\\
364& 182& 91& 274& 137& 412& 206& 103& 310& 155\\
466& 233& 700& 350& 175& 526& 263& 790& 395& 1186\\
593& 1780& 890& 445& 1336& 668& 334& 167& 502& 251\\
754& 377& 1132& 566& 283& 850& 425& 1276& 638& 319\\
958& 479& 1438& 719& 2158& 1079& 3238& 1619& 4858& 2429\\
7288& 3644& 1822& 911& 2734& 1367& 4102& 2051& 6154& 3077\\
9232& 4616& 2308& 1154& 577& 1732& 866& 433& 1300& 650\\
325& 976& 488& 244& 122& 61& 184& 92& 46& 23\\
70& 35& 106& 53& 160& 80& 40& 20& 10& 5\\
16& 8& 4& 2& 1& \\

2126&&&&&&&&&\\
1063& 3190& 1595& 4786& 2393& 7180& 3590& 1795& 5386& 2693\\
8080& 4040& 2020& 1010& 505& 1516& 758& 379& 1138& 569\\
1708& 854& 427& 1282& 641& 1924& 962& 481& 1444& 722\\
361& 1084& 542& 271& 814& 407& 1222& 611& 1834& 917\\
2752& 1376& 688& 344& 172& 86& 43& 130& 65& 196\\
98& 49& 148& 74& 37& 112& 56& 28& 14& 7\\
22& 11& 34& 17& 52& 26& 13& 40& 20& 10\\
5& 16& 8& 4& 2& 1& \\

2127&&&&&&&&&\\
6382& 3191& 9574& 4787& 14362& 7181& 21544& 10772& 5386& 2693\\
8080& 4040& 2020& 1010& 505& 1516& 758& 379& 1138& 569\\
1708& 854& 427& 1282& 641& 1924& 962& 481& 1444& 722\\
361& 1084& 542& 271& 814& 407& 1222& 611& 1834& 917\\
2752& 1376& 688& 344& 172& 86& 43& 130& 65& 196\\
98& 49& 148& 74& 37& 112& 56& 28& 14& 7\\
22& 11& 34& 17& 52& 26& 13& 40& 20& 10\\
5& 16& 8& 4& 2& 1& \\

2128&&&&&&&&&\\
1064& 532& 266& 133& 400& 200& 100& 50& 25& 76\\
38& 19& 58& 29& 88& 44& 22& 11& 34& 17\\
52& 26& 13& 40& 20& 10& 5& 16& 8& 4\\
2& 1& \\

2129&&&&&&&&&\\
6388& 3194& 1597& 4792& 2396& 1198& 599& 1798& 899& 2698\\
1349& 4048& 2024& 1012& 506& 253& 760& 380& 190& 95\\
286& 143& 430& 215& 646& 323& 970& 485& 1456& 728\\
364& 182& 91& 274& 137& 412& 206& 103& 310& 155\\
466& 233& 700& 350& 175& 526& 263& 790& 395& 1186\\
593& 1780& 890& 445& 1336& 668& 334& 167& 502& 251\\
754& 377& 1132& 566& 283& 850& 425& 1276& 638& 319\\
958& 479& 1438& 719& 2158& 1079& 3238& 1619& 4858& 2429\\
7288& 3644& 1822& 911& 2734& 1367& 4102& 2051& 6154& 3077\\
9232& 4616& 2308& 1154& 577& 1732& 866& 433& 1300& 650\\
325& 976& 488& 244& 122& 61& 184& 92& 46& 23\\
70& 35& 106& 53& 160& 80& 40& 20& 10& 5\\
16& 8& 4& 2& 1& \\

2130&&&&&&&&&\\
1065& 3196& 1598& 799& 2398& 1199& 3598& 1799& 5398& 2699\\
8098& 4049& 12148& 6074& 3037& 9112& 4556& 2278& 1139& 3418\\
1709& 5128& 2564& 1282& 641& 1924& 962& 481& 1444& 722\\
361& 1084& 542& 271& 814& 407& 1222& 611& 1834& 917\\
2752& 1376& 688& 344& 172& 86& 43& 130& 65& 196\\
98& 49& 148& 74& 37& 112& 56& 28& 14& 7\\
22& 11& 34& 17& 52& 26& 13& 40& 20& 10\\
5& 16& 8& 4& 2& 1& \\

2131&&&&&&&&&\\
6394& 3197& 9592& 4796& 2398& 1199& 3598& 1799& 5398& 2699\\
8098& 4049& 12148& 6074& 3037& 9112& 4556& 2278& 1139& 3418\\
1709& 5128& 2564& 1282& 641& 1924& 962& 481& 1444& 722\\
361& 1084& 542& 271& 814& 407& 1222& 611& 1834& 917\\
2752& 1376& 688& 344& 172& 86& 43& 130& 65& 196\\
98& 49& 148& 74& 37& 112& 56& 28& 14& 7\\
22& 11& 34& 17& 52& 26& 13& 40& 20& 10\\
5& 16& 8& 4& 2& 1& \\

2132&&&&&&&&&\\
1066& 533& 1600& 800& 400& 200& 100& 50& 25& 76\\
38& 19& 58& 29& 88& 44& 22& 11& 34& 17\\
52& 26& 13& 40& 20& 10& 5& 16& 8& 4\\
2& 1& \\

2133&&&&&&&&&\\
6400& 3200& 1600& 800& 400& 200& 100& 50& 25& 76\\
38& 19& 58& 29& 88& 44& 22& 11& 34& 17\\
52& 26& 13& 40& 20& 10& 5& 16& 8& 4\\
2& 1& \\

2134&&&&&&&&&\\
1067& 3202& 1601& 4804& 2402& 1201& 3604& 1802& 901& 2704\\
1352& 676& 338& 169& 508& 254& 127& 382& 191& 574\\
287& 862& 431& 1294& 647& 1942& 971& 2914& 1457& 4372\\
2186& 1093& 3280& 1640& 820& 410& 205& 616& 308& 154\\
77& 232& 116& 58& 29& 88& 44& 22& 11& 34\\
17& 52& 26& 13& 40& 20& 10& 5& 16& 8\\
4& 2& 1& \\

2135&&&&&&&&&\\
6406& 3203& 9610& 4805& 14416& 7208& 3604& 1802& 901& 2704\\
1352& 676& 338& 169& 508& 254& 127& 382& 191& 574\\
287& 862& 431& 1294& 647& 1942& 971& 2914& 1457& 4372\\
2186& 1093& 3280& 1640& 820& 410& 205& 616& 308& 154\\
77& 232& 116& 58& 29& 88& 44& 22& 11& 34\\
17& 52& 26& 13& 40& 20& 10& 5& 16& 8\\
4& 2& 1& \\

2136&&&&&&&&&\\
1068& 534& 267& 802& 401& 1204& 602& 301& 904& 452\\
226& 113& 340& 170& 85& 256& 128& 64& 32& 16\\
8& 4& 2& 1& \\

2137&&&&&&&&&\\
6412& 3206& 1603& 4810& 2405& 7216& 3608& 1804& 902& 451\\
1354& 677& 2032& 1016& 508& 254& 127& 382& 191& 574\\
287& 862& 431& 1294& 647& 1942& 971& 2914& 1457& 4372\\
2186& 1093& 3280& 1640& 820& 410& 205& 616& 308& 154\\
77& 232& 116& 58& 29& 88& 44& 22& 11& 34\\
17& 52& 26& 13& 40& 20& 10& 5& 16& 8\\
4& 2& 1& \\

2138&&&&&&&&&\\
1069& 3208& 1604& 802& 401& 1204& 602& 301& 904& 452\\
226& 113& 340& 170& 85& 256& 128& 64& 32& 16\\
8& 4& 2& 1& \\

2139&&&&&&&&&\\
6418& 3209& 9628& 4814& 2407& 7222& 3611& 10834& 5417& 16252\\
8126& 4063& 12190& 6095& 18286& 9143& 27430& 13715& 41146& 20573\\
61720& 30860& 15430& 7715& 23146& 11573& 34720& 17360& 8680& 4340\\
2170& 1085& 3256& 1628& 814& 407& 1222& 611& 1834& 917\\
2752& 1376& 688& 344& 172& 86& 43& 130& 65& 196\\
98& 49& 148& 74& 37& 112& 56& 28& 14& 7\\
22& 11& 34& 17& 52& 26& 13& 40& 20& 10\\
5& 16& 8& 4& 2& 1& \\

2140&&&&&&&&&\\
1070& 535& 1606& 803& 2410& 1205& 3616& 1808& 904& 452\\
226& 113& 340& 170& 85& 256& 128& 64& 32& 16\\
8& 4& 2& 1& \\

2141&&&&&&&&&\\
6424& 3212& 1606& 803& 2410& 1205& 3616& 1808& 904& 452\\
226& 113& 340& 170& 85& 256& 128& 64& 32& 16\\
8& 4& 2& 1& \\

2142&&&&&&&&&\\
1071& 3214& 1607& 4822& 2411& 7234& 3617& 10852& 5426& 2713\\
8140& 4070& 2035& 6106& 3053& 9160& 4580& 2290& 1145& 3436\\
1718& 859& 2578& 1289& 3868& 1934& 967& 2902& 1451& 4354\\
2177& 6532& 3266& 1633& 4900& 2450& 1225& 3676& 1838& 919\\
2758& 1379& 4138& 2069& 6208& 3104& 1552& 776& 388& 194\\
97& 292& 146& 73& 220& 110& 55& 166& 83& 250\\
125& 376& 188& 94& 47& 142& 71& 214& 107& 322\\
161& 484& 242& 121& 364& 182& 91& 274& 137& 412\\
206& 103& 310& 155& 466& 233& 700& 350& 175& 526\\
263& 790& 395& 1186& 593& 1780& 890& 445& 1336& 668\\
334& 167& 502& 251& 754& 377& 1132& 566& 283& 850\\
425& 1276& 638& 319& 958& 479& 1438& 719& 2158& 1079\\
3238& 1619& 4858& 2429& 7288& 3644& 1822& 911& 2734& 1367\\
4102& 2051& 6154& 3077& 9232& 4616& 2308& 1154& 577& 1732\\
866& 433& 1300& 650& 325& 976& 488& 244& 122& 61\\
184& 92& 46& 23& 70& 35& 106& 53& 160& 80\\
40& 20& 10& 5& 16& 8& 4& 2& 1& \\

2143&&&&&&&&&\\
6430& 3215& 9646& 4823& 14470& 7235& 21706& 10853& 32560& 16280\\
8140& 4070& 2035& 6106& 3053& 9160& 4580& 2290& 1145& 3436\\
1718& 859& 2578& 1289& 3868& 1934& 967& 2902& 1451& 4354\\
2177& 6532& 3266& 1633& 4900& 2450& 1225& 3676& 1838& 919\\
2758& 1379& 4138& 2069& 6208& 3104& 1552& 776& 388& 194\\
97& 292& 146& 73& 220& 110& 55& 166& 83& 250\\
125& 376& 188& 94& 47& 142& 71& 214& 107& 322\\
161& 484& 242& 121& 364& 182& 91& 274& 137& 412\\
206& 103& 310& 155& 466& 233& 700& 350& 175& 526\\
263& 790& 395& 1186& 593& 1780& 890& 445& 1336& 668\\
334& 167& 502& 251& 754& 377& 1132& 566& 283& 850\\
425& 1276& 638& 319& 958& 479& 1438& 719& 2158& 1079\\
3238& 1619& 4858& 2429& 7288& 3644& 1822& 911& 2734& 1367\\
4102& 2051& 6154& 3077& 9232& 4616& 2308& 1154& 577& 1732\\
866& 433& 1300& 650& 325& 976& 488& 244& 122& 61\\
184& 92& 46& 23& 70& 35& 106& 53& 160& 80\\
40& 20& 10& 5& 16& 8& 4& 2& 1& \\

2144&&&&&&&&&\\
1072& 536& 268& 134& 67& 202& 101& 304& 152& 76\\
38& 19& 58& 29& 88& 44& 22& 11& 34& 17\\
52& 26& 13& 40& 20& 10& 5& 16& 8& 4\\
2& 1& \\

2145&&&&&&&&&\\
6436& 3218& 1609& 4828& 2414& 1207& 3622& 1811& 5434& 2717\\
8152& 4076& 2038& 1019& 3058& 1529& 4588& 2294& 1147& 3442\\
1721& 5164& 2582& 1291& 3874& 1937& 5812& 2906& 1453& 4360\\
2180& 1090& 545& 1636& 818& 409& 1228& 614& 307& 922\\
461& 1384& 692& 346& 173& 520& 260& 130& 65& 196\\
98& 49& 148& 74& 37& 112& 56& 28& 14& 7\\
22& 11& 34& 17& 52& 26& 13& 40& 20& 10\\
5& 16& 8& 4& 2& 1& \\

2146&&&&&&&&&\\
1073& 3220& 1610& 805& 2416& 1208& 604& 302& 151& 454\\
227& 682& 341& 1024& 512& 256& 128& 64& 32& 16\\
8& 4& 2& 1& \\

2147&&&&&&&&&\\
6442& 3221& 9664& 4832& 2416& 1208& 604& 302& 151& 454\\
227& 682& 341& 1024& 512& 256& 128& 64& 32& 16\\
8& 4& 2& 1& \\

2148&&&&&&&&&\\
1074& 537& 1612& 806& 403& 1210& 605& 1816& 908& 454\\
227& 682& 341& 1024& 512& 256& 128& 64& 32& 16\\
8& 4& 2& 1& \\

2149&&&&&&&&&\\
6448& 3224& 1612& 806& 403& 1210& 605& 1816& 908& 454\\
227& 682& 341& 1024& 512& 256& 128& 64& 32& 16\\
8& 4& 2& 1& \\

2150&&&&&&&&&\\
1075& 3226& 1613& 4840& 2420& 1210& 605& 1816& 908& 454\\
227& 682& 341& 1024& 512& 256& 128& 64& 32& 16\\
8& 4& 2& 1& \\

2151&&&&&&&&&\\
6454& 3227& 9682& 4841& 14524& 7262& 3631& 10894& 5447& 16342\\
8171& 24514& 12257& 36772& 18386& 9193& 27580& 13790& 6895& 20686\\
10343& 31030& 15515& 46546& 23273& 69820& 34910& 17455& 52366& 26183\\
78550& 39275& 117826& 58913& 176740& 88370& 44185& 132556& 66278& 33139\\
99418& 49709& 149128& 74564& 37282& 18641& 55924& 27962& 13981& 41944\\
20972& 10486& 5243& 15730& 7865& 23596& 11798& 5899& 17698& 8849\\
26548& 13274& 6637& 19912& 9956& 4978& 2489& 7468& 3734& 1867\\
5602& 2801& 8404& 4202& 2101& 6304& 3152& 1576& 788& 394\\
197& 592& 296& 148& 74& 37& 112& 56& 28& 14\\
7& 22& 11& 34& 17& 52& 26& 13& 40& 20\\
10& 5& 16& 8& 4& 2& 1& \\

2152&&&&&&&&&\\
1076& 538& 269& 808& 404& 202& 101& 304& 152& 76\\
38& 19& 58& 29& 88& 44& 22& 11& 34& 17\\
52& 26& 13& 40& 20& 10& 5& 16& 8& 4\\
2& 1& \\

2153&&&&&&&&&\\
6460& 3230& 1615& 4846& 2423& 7270& 3635& 10906& 5453& 16360\\
8180& 4090& 2045& 6136& 3068& 1534& 767& 2302& 1151& 3454\\
1727& 5182& 2591& 7774& 3887& 11662& 5831& 17494& 8747& 26242\\
13121& 39364& 19682& 9841& 29524& 14762& 7381& 22144& 11072& 5536\\
2768& 1384& 692& 346& 173& 520& 260& 130& 65& 196\\
98& 49& 148& 74& 37& 112& 56& 28& 14& 7\\
22& 11& 34& 17& 52& 26& 13& 40& 20& 10\\
5& 16& 8& 4& 2& 1& \\

2154&&&&&&&&&\\
1077& 3232& 1616& 808& 404& 202& 101& 304& 152& 76\\
38& 19& 58& 29& 88& 44& 22& 11& 34& 17\\
52& 26& 13& 40& 20& 10& 5& 16& 8& 4\\
2& 1& \\

2155&&&&&&&&&\\
6466& 3233& 9700& 4850& 2425& 7276& 3638& 1819& 5458& 2729\\
8188& 4094& 2047& 6142& 3071& 9214& 4607& 13822& 6911& 20734\\
10367& 31102& 15551& 46654& 23327& 69982& 34991& 104974& 52487& 157462\\
78731& 236194& 118097& 354292& 177146& 88573& 265720& 132860& 66430& 33215\\
99646& 49823& 149470& 74735& 224206& 112103& 336310& 168155& 504466& 252233\\
756700& 378350& 189175& 567526& 283763& 851290& 425645& 1276936& 638468& 319234\\
159617& 478852& 239426& 119713& 359140& 179570& 89785& 269356& 134678& 67339\\
202018& 101009& 303028& 151514& 75757& 227272& 113636& 56818& 28409& 85228\\
42614& 21307& 63922& 31961& 95884& 47942& 23971& 71914& 35957& 107872\\
53936& 26968& 13484& 6742& 3371& 10114& 5057& 15172& 7586& 3793\\
11380& 5690& 2845& 8536& 4268& 2134& 1067& 3202& 1601& 4804\\
2402& 1201& 3604& 1802& 901& 2704& 1352& 676& 338& 169\\
508& 254& 127& 382& 191& 574& 287& 862& 431& 1294\\
647& 1942& 971& 2914& 1457& 4372& 2186& 1093& 3280& 1640\\
820& 410& 205& 616& 308& 154& 77& 232& 116& 58\\
29& 88& 44& 22& 11& 34& 17& 52& 26& 13\\
40& 20& 10& 5& 16& 8& 4& 2& 1& \\

2156&&&&&&&&&\\
1078& 539& 1618& 809& 2428& 1214& 607& 1822& 911& 2734\\
1367& 4102& 2051& 6154& 3077& 9232& 4616& 2308& 1154& 577\\
1732& 866& 433& 1300& 650& 325& 976& 488& 244& 122\\
61& 184& 92& 46& 23& 70& 35& 106& 53& 160\\
80& 40& 20& 10& 5& 16& 8& 4& 2& 1\\

2157&&&&&&&&&\\
6472& 3236& 1618& 809& 2428& 1214& 607& 1822& 911& 2734\\
1367& 4102& 2051& 6154& 3077& 9232& 4616& 2308& 1154& 577\\
1732& 866& 433& 1300& 650& 325& 976& 488& 244& 122\\
61& 184& 92& 46& 23& 70& 35& 106& 53& 160\\
80& 40& 20& 10& 5& 16& 8& 4& 2& 1\\

2158&&&&&&&&&\\
1079& 3238& 1619& 4858& 2429& 7288& 3644& 1822& 911& 2734\\
1367& 4102& 2051& 6154& 3077& 9232& 4616& 2308& 1154& 577\\
1732& 866& 433& 1300& 650& 325& 976& 488& 244& 122\\
61& 184& 92& 46& 23& 70& 35& 106& 53& 160\\
80& 40& 20& 10& 5& 16& 8& 4& 2& 1\\

2159&&&&&&&&&\\
6478& 3239& 9718& 4859& 14578& 7289& 21868& 10934& 5467& 16402\\
8201& 24604& 12302& 6151& 18454& 9227& 27682& 13841& 41524& 20762\\
10381& 31144& 15572& 7786& 3893& 11680& 5840& 2920& 1460& 730\\
365& 1096& 548& 274& 137& 412& 206& 103& 310& 155\\
466& 233& 700& 350& 175& 526& 263& 790& 395& 1186\\
593& 1780& 890& 445& 1336& 668& 334& 167& 502& 251\\
754& 377& 1132& 566& 283& 850& 425& 1276& 638& 319\\
958& 479& 1438& 719& 2158& 1079& 3238& 1619& 4858& 2429\\
7288& 3644& 1822& 911& 2734& 1367& 4102& 2051& 6154& 3077\\
9232& 4616& 2308& 1154& 577& 1732& 866& 433& 1300& 650\\
325& 976& 488& 244& 122& 61& 184& 92& 46& 23\\
70& 35& 106& 53& 160& 80& 40& 20& 10& 5\\
16& 8& 4& 2& 1& \\

2160&&&&&&&&&\\
1080& 540& 270& 135& 406& 203& 610& 305& 916& 458\\
229& 688& 344& 172& 86& 43& 130& 65& 196& 98\\
49& 148& 74& 37& 112& 56& 28& 14& 7& 22\\
11& 34& 17& 52& 26& 13& 40& 20& 10& 5\\
16& 8& 4& 2& 1& \\

2161&&&&&&&&&\\
6484& 3242& 1621& 4864& 2432& 1216& 608& 304& 152& 76\\
38& 19& 58& 29& 88& 44& 22& 11& 34& 17\\
52& 26& 13& 40& 20& 10& 5& 16& 8& 4\\
2& 1& \\

2162&&&&&&&&&\\
1081& 3244& 1622& 811& 2434& 1217& 3652& 1826& 913& 2740\\
1370& 685& 2056& 1028& 514& 257& 772& 386& 193& 580\\
290& 145& 436& 218& 109& 328& 164& 82& 41& 124\\
62& 31& 94& 47& 142& 71& 214& 107& 322& 161\\
484& 242& 121& 364& 182& 91& 274& 137& 412& 206\\
103& 310& 155& 466& 233& 700& 350& 175& 526& 263\\
790& 395& 1186& 593& 1780& 890& 445& 1336& 668& 334\\
167& 502& 251& 754& 377& 1132& 566& 283& 850& 425\\
1276& 638& 319& 958& 479& 1438& 719& 2158& 1079& 3238\\
1619& 4858& 2429& 7288& 3644& 1822& 911& 2734& 1367& 4102\\
2051& 6154& 3077& 9232& 4616& 2308& 1154& 577& 1732& 866\\
433& 1300& 650& 325& 976& 488& 244& 122& 61& 184\\
92& 46& 23& 70& 35& 106& 53& 160& 80& 40\\
20& 10& 5& 16& 8& 4& 2& 1& \\

2163&&&&&&&&&\\
6490& 3245& 9736& 4868& 2434& 1217& 3652& 1826& 913& 2740\\
1370& 685& 2056& 1028& 514& 257& 772& 386& 193& 580\\
290& 145& 436& 218& 109& 328& 164& 82& 41& 124\\
62& 31& 94& 47& 142& 71& 214& 107& 322& 161\\
484& 242& 121& 364& 182& 91& 274& 137& 412& 206\\
103& 310& 155& 466& 233& 700& 350& 175& 526& 263\\
790& 395& 1186& 593& 1780& 890& 445& 1336& 668& 334\\
167& 502& 251& 754& 377& 1132& 566& 283& 850& 425\\
1276& 638& 319& 958& 479& 1438& 719& 2158& 1079& 3238\\
1619& 4858& 2429& 7288& 3644& 1822& 911& 2734& 1367& 4102\\
2051& 6154& 3077& 9232& 4616& 2308& 1154& 577& 1732& 866\\
433& 1300& 650& 325& 976& 488& 244& 122& 61& 184\\
92& 46& 23& 70& 35& 106& 53& 160& 80& 40\\
20& 10& 5& 16& 8& 4& 2& 1& \\

2164&&&&&&&&&\\
1082& 541& 1624& 812& 406& 203& 610& 305& 916& 458\\
229& 688& 344& 172& 86& 43& 130& 65& 196& 98\\
49& 148& 74& 37& 112& 56& 28& 14& 7& 22\\
11& 34& 17& 52& 26& 13& 40& 20& 10& 5\\
16& 8& 4& 2& 1& \\

2165&&&&&&&&&\\
6496& 3248& 1624& 812& 406& 203& 610& 305& 916& 458\\
229& 688& 344& 172& 86& 43& 130& 65& 196& 98\\
49& 148& 74& 37& 112& 56& 28& 14& 7& 22\\
11& 34& 17& 52& 26& 13& 40& 20& 10& 5\\
16& 8& 4& 2& 1& \\

2166&&&&&&&&&\\
1083& 3250& 1625& 4876& 2438& 1219& 3658& 1829& 5488& 2744\\
1372& 686& 343& 1030& 515& 1546& 773& 2320& 1160& 580\\
290& 145& 436& 218& 109& 328& 164& 82& 41& 124\\
62& 31& 94& 47& 142& 71& 214& 107& 322& 161\\
484& 242& 121& 364& 182& 91& 274& 137& 412& 206\\
103& 310& 155& 466& 233& 700& 350& 175& 526& 263\\
790& 395& 1186& 593& 1780& 890& 445& 1336& 668& 334\\
167& 502& 251& 754& 377& 1132& 566& 283& 850& 425\\
1276& 638& 319& 958& 479& 1438& 719& 2158& 1079& 3238\\
1619& 4858& 2429& 7288& 3644& 1822& 911& 2734& 1367& 4102\\
2051& 6154& 3077& 9232& 4616& 2308& 1154& 577& 1732& 866\\
433& 1300& 650& 325& 976& 488& 244& 122& 61& 184\\
92& 46& 23& 70& 35& 106& 53& 160& 80& 40\\
20& 10& 5& 16& 8& 4& 2& 1& \\

2167&&&&&&&&&\\
6502& 3251& 9754& 4877& 14632& 7316& 3658& 1829& 5488& 2744\\
1372& 686& 343& 1030& 515& 1546& 773& 2320& 1160& 580\\
290& 145& 436& 218& 109& 328& 164& 82& 41& 124\\
62& 31& 94& 47& 142& 71& 214& 107& 322& 161\\
484& 242& 121& 364& 182& 91& 274& 137& 412& 206\\
103& 310& 155& 466& 233& 700& 350& 175& 526& 263\\
790& 395& 1186& 593& 1780& 890& 445& 1336& 668& 334\\
167& 502& 251& 754& 377& 1132& 566& 283& 850& 425\\
1276& 638& 319& 958& 479& 1438& 719& 2158& 1079& 3238\\
1619& 4858& 2429& 7288& 3644& 1822& 911& 2734& 1367& 4102\\
2051& 6154& 3077& 9232& 4616& 2308& 1154& 577& 1732& 866\\
433& 1300& 650& 325& 976& 488& 244& 122& 61& 184\\
92& 46& 23& 70& 35& 106& 53& 160& 80& 40\\
20& 10& 5& 16& 8& 4& 2& 1& \\

2168&&&&&&&&&\\
1084& 542& 271& 814& 407& 1222& 611& 1834& 917& 2752\\
1376& 688& 344& 172& 86& 43& 130& 65& 196& 98\\
49& 148& 74& 37& 112& 56& 28& 14& 7& 22\\
11& 34& 17& 52& 26& 13& 40& 20& 10& 5\\
16& 8& 4& 2& 1& \\

2169&&&&&&&&&\\
6508& 3254& 1627& 4882& 2441& 7324& 3662& 1831& 5494& 2747\\
8242& 4121& 12364& 6182& 3091& 9274& 4637& 13912& 6956& 3478\\
1739& 5218& 2609& 7828& 3914& 1957& 5872& 2936& 1468& 734\\
367& 1102& 551& 1654& 827& 2482& 1241& 3724& 1862& 931\\
2794& 1397& 4192& 2096& 1048& 524& 262& 131& 394& 197\\
592& 296& 148& 74& 37& 112& 56& 28& 14& 7\\
22& 11& 34& 17& 52& 26& 13& 40& 20& 10\\
5& 16& 8& 4& 2& 1& \\

2170&&&&&&&&&\\
1085& 3256& 1628& 814& 407& 1222& 611& 1834& 917& 2752\\
1376& 688& 344& 172& 86& 43& 130& 65& 196& 98\\
49& 148& 74& 37& 112& 56& 28& 14& 7& 22\\
11& 34& 17& 52& 26& 13& 40& 20& 10& 5\\
16& 8& 4& 2& 1& \\

2171&&&&&&&&&\\
6514& 3257& 9772& 4886& 2443& 7330& 3665& 10996& 5498& 2749\\
8248& 4124& 2062& 1031& 3094& 1547& 4642& 2321& 6964& 3482\\
1741& 5224& 2612& 1306& 653& 1960& 980& 490& 245& 736\\
368& 184& 92& 46& 23& 70& 35& 106& 53& 160\\
80& 40& 20& 10& 5& 16& 8& 4& 2& 1\\

2172&&&&&&&&&\\
1086& 543& 1630& 815& 2446& 1223& 3670& 1835& 5506& 2753\\
8260& 4130& 2065& 6196& 3098& 1549& 4648& 2324& 1162& 581\\
1744& 872& 436& 218& 109& 328& 164& 82& 41& 124\\
62& 31& 94& 47& 142& 71& 214& 107& 322& 161\\
484& 242& 121& 364& 182& 91& 274& 137& 412& 206\\
103& 310& 155& 466& 233& 700& 350& 175& 526& 263\\
790& 395& 1186& 593& 1780& 890& 445& 1336& 668& 334\\
167& 502& 251& 754& 377& 1132& 566& 283& 850& 425\\
1276& 638& 319& 958& 479& 1438& 719& 2158& 1079& 3238\\
1619& 4858& 2429& 7288& 3644& 1822& 911& 2734& 1367& 4102\\
2051& 6154& 3077& 9232& 4616& 2308& 1154& 577& 1732& 866\\
433& 1300& 650& 325& 976& 488& 244& 122& 61& 184\\
92& 46& 23& 70& 35& 106& 53& 160& 80& 40\\
20& 10& 5& 16& 8& 4& 2& 1& \\

2173&&&&&&&&&\\
6520& 3260& 1630& 815& 2446& 1223& 3670& 1835& 5506& 2753\\
8260& 4130& 2065& 6196& 3098& 1549& 4648& 2324& 1162& 581\\
1744& 872& 436& 218& 109& 328& 164& 82& 41& 124\\
62& 31& 94& 47& 142& 71& 214& 107& 322& 161\\
484& 242& 121& 364& 182& 91& 274& 137& 412& 206\\
103& 310& 155& 466& 233& 700& 350& 175& 526& 263\\
790& 395& 1186& 593& 1780& 890& 445& 1336& 668& 334\\
167& 502& 251& 754& 377& 1132& 566& 283& 850& 425\\
1276& 638& 319& 958& 479& 1438& 719& 2158& 1079& 3238\\
1619& 4858& 2429& 7288& 3644& 1822& 911& 2734& 1367& 4102\\
2051& 6154& 3077& 9232& 4616& 2308& 1154& 577& 1732& 866\\
433& 1300& 650& 325& 976& 488& 244& 122& 61& 184\\
92& 46& 23& 70& 35& 106& 53& 160& 80& 40\\
20& 10& 5& 16& 8& 4& 2& 1& \\

2174&&&&&&&&&\\
1087& 3262& 1631& 4894& 2447& 7342& 3671& 11014& 5507& 16522\\
8261& 24784& 12392& 6196& 3098& 1549& 4648& 2324& 1162& 581\\
1744& 872& 436& 218& 109& 328& 164& 82& 41& 124\\
62& 31& 94& 47& 142& 71& 214& 107& 322& 161\\
484& 242& 121& 364& 182& 91& 274& 137& 412& 206\\
103& 310& 155& 466& 233& 700& 350& 175& 526& 263\\
790& 395& 1186& 593& 1780& 890& 445& 1336& 668& 334\\
167& 502& 251& 754& 377& 1132& 566& 283& 850& 425\\
1276& 638& 319& 958& 479& 1438& 719& 2158& 1079& 3238\\
1619& 4858& 2429& 7288& 3644& 1822& 911& 2734& 1367& 4102\\
2051& 6154& 3077& 9232& 4616& 2308& 1154& 577& 1732& 866\\
433& 1300& 650& 325& 976& 488& 244& 122& 61& 184\\
92& 46& 23& 70& 35& 106& 53& 160& 80& 40\\
20& 10& 5& 16& 8& 4& 2& 1& \\

2175&&&&&&&&&\\
6526& 3263& 9790& 4895& 14686& 7343& 22030& 11015& 33046& 16523\\
49570& 24785& 74356& 37178& 18589& 55768& 27884& 13942& 6971& 20914\\
10457& 31372& 15686& 7843& 23530& 11765& 35296& 17648& 8824& 4412\\
2206& 1103& 3310& 1655& 4966& 2483& 7450& 3725& 11176& 5588\\
2794& 1397& 4192& 2096& 1048& 524& 262& 131& 394& 197\\
592& 296& 148& 74& 37& 112& 56& 28& 14& 7\\
22& 11& 34& 17& 52& 26& 13& 40& 20& 10\\
5& 16& 8& 4& 2& 1& \\

2176&&&&&&&&&\\
1088& 544& 272& 136& 68& 34& 17& 52& 26& 13\\
40& 20& 10& 5& 16& 8& 4& 2& 1& \\

2177&&&&&&&&&\\
6532& 3266& 1633& 4900& 2450& 1225& 3676& 1838& 919& 2758\\
1379& 4138& 2069& 6208& 3104& 1552& 776& 388& 194& 97\\
292& 146& 73& 220& 110& 55& 166& 83& 250& 125\\
376& 188& 94& 47& 142& 71& 214& 107& 322& 161\\
484& 242& 121& 364& 182& 91& 274& 137& 412& 206\\
103& 310& 155& 466& 233& 700& 350& 175& 526& 263\\
790& 395& 1186& 593& 1780& 890& 445& 1336& 668& 334\\
167& 502& 251& 754& 377& 1132& 566& 283& 850& 425\\
1276& 638& 319& 958& 479& 1438& 719& 2158& 1079& 3238\\
1619& 4858& 2429& 7288& 3644& 1822& 911& 2734& 1367& 4102\\
2051& 6154& 3077& 9232& 4616& 2308& 1154& 577& 1732& 866\\
433& 1300& 650& 325& 976& 488& 244& 122& 61& 184\\
92& 46& 23& 70& 35& 106& 53& 160& 80& 40\\
20& 10& 5& 16& 8& 4& 2& 1& \\

2178&&&&&&&&&\\
1089& 3268& 1634& 817& 2452& 1226& 613& 1840& 920& 460\\
230& 115& 346& 173& 520& 260& 130& 65& 196& 98\\
49& 148& 74& 37& 112& 56& 28& 14& 7& 22\\
11& 34& 17& 52& 26& 13& 40& 20& 10& 5\\
16& 8& 4& 2& 1& \\

2179&&&&&&&&&\\
6538& 3269& 9808& 4904& 2452& 1226& 613& 1840& 920& 460\\
230& 115& 346& 173& 520& 260& 130& 65& 196& 98\\
49& 148& 74& 37& 112& 56& 28& 14& 7& 22\\
11& 34& 17& 52& 26& 13& 40& 20& 10& 5\\
16& 8& 4& 2& 1& \\

2180&&&&&&&&&\\
1090& 545& 1636& 818& 409& 1228& 614& 307& 922& 461\\
1384& 692& 346& 173& 520& 260& 130& 65& 196& 98\\
49& 148& 74& 37& 112& 56& 28& 14& 7& 22\\
11& 34& 17& 52& 26& 13& 40& 20& 10& 5\\
16& 8& 4& 2& 1& \\

2181&&&&&&&&&\\
6544& 3272& 1636& 818& 409& 1228& 614& 307& 922& 461\\
1384& 692& 346& 173& 520& 260& 130& 65& 196& 98\\
49& 148& 74& 37& 112& 56& 28& 14& 7& 22\\
11& 34& 17& 52& 26& 13& 40& 20& 10& 5\\
16& 8& 4& 2& 1& \\

2182&&&&&&&&&\\
1091& 3274& 1637& 4912& 2456& 1228& 614& 307& 922& 461\\
1384& 692& 346& 173& 520& 260& 130& 65& 196& 98\\
49& 148& 74& 37& 112& 56& 28& 14& 7& 22\\
11& 34& 17& 52& 26& 13& 40& 20& 10& 5\\
16& 8& 4& 2& 1& \\

2183&&&&&&&&&\\
6550& 3275& 9826& 4913& 14740& 7370& 3685& 11056& 5528& 2764\\
1382& 691& 2074& 1037& 3112& 1556& 778& 389& 1168& 584\\
292& 146& 73& 220& 110& 55& 166& 83& 250& 125\\
376& 188& 94& 47& 142& 71& 214& 107& 322& 161\\
484& 242& 121& 364& 182& 91& 274& 137& 412& 206\\
103& 310& 155& 466& 233& 700& 350& 175& 526& 263\\
790& 395& 1186& 593& 1780& 890& 445& 1336& 668& 334\\
167& 502& 251& 754& 377& 1132& 566& 283& 850& 425\\
1276& 638& 319& 958& 479& 1438& 719& 2158& 1079& 3238\\
1619& 4858& 2429& 7288& 3644& 1822& 911& 2734& 1367& 4102\\
2051& 6154& 3077& 9232& 4616& 2308& 1154& 577& 1732& 866\\
433& 1300& 650& 325& 976& 488& 244& 122& 61& 184\\
92& 46& 23& 70& 35& 106& 53& 160& 80& 40\\
20& 10& 5& 16& 8& 4& 2& 1& \\

2184&&&&&&&&&\\
1092& 546& 273& 820& 410& 205& 616& 308& 154& 77\\
232& 116& 58& 29& 88& 44& 22& 11& 34& 17\\
52& 26& 13& 40& 20& 10& 5& 16& 8& 4\\
2& 1& \\

2185&&&&&&&&&\\
6556& 3278& 1639& 4918& 2459& 7378& 3689& 11068& 5534& 2767\\
8302& 4151& 12454& 6227& 18682& 9341& 28024& 14012& 7006& 3503\\
10510& 5255& 15766& 7883& 23650& 11825& 35476& 17738& 8869& 26608\\
13304& 6652& 3326& 1663& 4990& 2495& 7486& 3743& 11230& 5615\\
16846& 8423& 25270& 12635& 37906& 18953& 56860& 28430& 14215& 42646\\
21323& 63970& 31985& 95956& 47978& 23989& 71968& 35984& 17992& 8996\\
4498& 2249& 6748& 3374& 1687& 5062& 2531& 7594& 3797& 11392\\
5696& 2848& 1424& 712& 356& 178& 89& 268& 134& 67\\
202& 101& 304& 152& 76& 38& 19& 58& 29& 88\\
44& 22& 11& 34& 17& 52& 26& 13& 40& 20\\
10& 5& 16& 8& 4& 2& 1& \\

2186&&&&&&&&&\\
1093& 3280& 1640& 820& 410& 205& 616& 308& 154& 77\\
232& 116& 58& 29& 88& 44& 22& 11& 34& 17\\
52& 26& 13& 40& 20& 10& 5& 16& 8& 4\\
2& 1& \\

2187&&&&&&&&&\\
6562& 3281& 9844& 4922& 2461& 7384& 3692& 1846& 923& 2770\\
1385& 4156& 2078& 1039& 3118& 1559& 4678& 2339& 7018& 3509\\
10528& 5264& 2632& 1316& 658& 329& 988& 494& 247& 742\\
371& 1114& 557& 1672& 836& 418& 209& 628& 314& 157\\
472& 236& 118& 59& 178& 89& 268& 134& 67& 202\\
101& 304& 152& 76& 38& 19& 58& 29& 88& 44\\
22& 11& 34& 17& 52& 26& 13& 40& 20& 10\\
5& 16& 8& 4& 2& 1& \\

2188&&&&&&&&&\\
1094& 547& 1642& 821& 2464& 1232& 616& 308& 154& 77\\
232& 116& 58& 29& 88& 44& 22& 11& 34& 17\\
52& 26& 13& 40& 20& 10& 5& 16& 8& 4\\
2& 1& \\

2189&&&&&&&&&\\
6568& 3284& 1642& 821& 2464& 1232& 616& 308& 154& 77\\
232& 116& 58& 29& 88& 44& 22& 11& 34& 17\\
52& 26& 13& 40& 20& 10& 5& 16& 8& 4\\
2& 1& \\

2190&&&&&&&&&\\
1095& 3286& 1643& 4930& 2465& 7396& 3698& 1849& 5548& 2774\\
1387& 4162& 2081& 6244& 3122& 1561& 4684& 2342& 1171& 3514\\
1757& 5272& 2636& 1318& 659& 1978& 989& 2968& 1484& 742\\
371& 1114& 557& 1672& 836& 418& 209& 628& 314& 157\\
472& 236& 118& 59& 178& 89& 268& 134& 67& 202\\
101& 304& 152& 76& 38& 19& 58& 29& 88& 44\\
22& 11& 34& 17& 52& 26& 13& 40& 20& 10\\
5& 16& 8& 4& 2& 1& \\

2191&&&&&&&&&\\
6574& 3287& 9862& 4931& 14794& 7397& 22192& 11096& 5548& 2774\\
1387& 4162& 2081& 6244& 3122& 1561& 4684& 2342& 1171& 3514\\
1757& 5272& 2636& 1318& 659& 1978& 989& 2968& 1484& 742\\
371& 1114& 557& 1672& 836& 418& 209& 628& 314& 157\\
472& 236& 118& 59& 178& 89& 268& 134& 67& 202\\
101& 304& 152& 76& 38& 19& 58& 29& 88& 44\\
22& 11& 34& 17& 52& 26& 13& 40& 20& 10\\
5& 16& 8& 4& 2& 1& \\

2192&&&&&&&&&\\
1096& 548& 274& 137& 412& 206& 103& 310& 155& 466\\
233& 700& 350& 175& 526& 263& 790& 395& 1186& 593\\
1780& 890& 445& 1336& 668& 334& 167& 502& 251& 754\\
377& 1132& 566& 283& 850& 425& 1276& 638& 319& 958\\
479& 1438& 719& 2158& 1079& 3238& 1619& 4858& 2429& 7288\\
3644& 1822& 911& 2734& 1367& 4102& 2051& 6154& 3077& 9232\\
4616& 2308& 1154& 577& 1732& 866& 433& 1300& 650& 325\\
976& 488& 244& 122& 61& 184& 92& 46& 23& 70\\
35& 106& 53& 160& 80& 40& 20& 10& 5& 16\\
8& 4& 2& 1& \\

2193&&&&&&&&&\\
6580& 3290& 1645& 4936& 2468& 1234& 617& 1852& 926& 463\\
1390& 695& 2086& 1043& 3130& 1565& 4696& 2348& 1174& 587\\
1762& 881& 2644& 1322& 661& 1984& 992& 496& 248& 124\\
62& 31& 94& 47& 142& 71& 214& 107& 322& 161\\
484& 242& 121& 364& 182& 91& 274& 137& 412& 206\\
103& 310& 155& 466& 233& 700& 350& 175& 526& 263\\
790& 395& 1186& 593& 1780& 890& 445& 1336& 668& 334\\
167& 502& 251& 754& 377& 1132& 566& 283& 850& 425\\
1276& 638& 319& 958& 479& 1438& 719& 2158& 1079& 3238\\
1619& 4858& 2429& 7288& 3644& 1822& 911& 2734& 1367& 4102\\
2051& 6154& 3077& 9232& 4616& 2308& 1154& 577& 1732& 866\\
433& 1300& 650& 325& 976& 488& 244& 122& 61& 184\\
92& 46& 23& 70& 35& 106& 53& 160& 80& 40\\
20& 10& 5& 16& 8& 4& 2& 1& \\

2194&&&&&&&&&\\
1097& 3292& 1646& 823& 2470& 1235& 3706& 1853& 5560& 2780\\
1390& 695& 2086& 1043& 3130& 1565& 4696& 2348& 1174& 587\\
1762& 881& 2644& 1322& 661& 1984& 992& 496& 248& 124\\
62& 31& 94& 47& 142& 71& 214& 107& 322& 161\\
484& 242& 121& 364& 182& 91& 274& 137& 412& 206\\
103& 310& 155& 466& 233& 700& 350& 175& 526& 263\\
790& 395& 1186& 593& 1780& 890& 445& 1336& 668& 334\\
167& 502& 251& 754& 377& 1132& 566& 283& 850& 425\\
1276& 638& 319& 958& 479& 1438& 719& 2158& 1079& 3238\\
1619& 4858& 2429& 7288& 3644& 1822& 911& 2734& 1367& 4102\\
2051& 6154& 3077& 9232& 4616& 2308& 1154& 577& 1732& 866\\
433& 1300& 650& 325& 976& 488& 244& 122& 61& 184\\
92& 46& 23& 70& 35& 106& 53& 160& 80& 40\\
20& 10& 5& 16& 8& 4& 2& 1& \\

2195&&&&&&&&&\\
6586& 3293& 9880& 4940& 2470& 1235& 3706& 1853& 5560& 2780\\
1390& 695& 2086& 1043& 3130& 1565& 4696& 2348& 1174& 587\\
1762& 881& 2644& 1322& 661& 1984& 992& 496& 248& 124\\
62& 31& 94& 47& 142& 71& 214& 107& 322& 161\\
484& 242& 121& 364& 182& 91& 274& 137& 412& 206\\
103& 310& 155& 466& 233& 700& 350& 175& 526& 263\\
790& 395& 1186& 593& 1780& 890& 445& 1336& 668& 334\\
167& 502& 251& 754& 377& 1132& 566& 283& 850& 425\\
1276& 638& 319& 958& 479& 1438& 719& 2158& 1079& 3238\\
1619& 4858& 2429& 7288& 3644& 1822& 911& 2734& 1367& 4102\\
2051& 6154& 3077& 9232& 4616& 2308& 1154& 577& 1732& 866\\
433& 1300& 650& 325& 976& 488& 244& 122& 61& 184\\
92& 46& 23& 70& 35& 106& 53& 160& 80& 40\\
20& 10& 5& 16& 8& 4& 2& 1& \\

2196&&&&&&&&&\\
1098& 549& 1648& 824& 412& 206& 103& 310& 155& 466\\
233& 700& 350& 175& 526& 263& 790& 395& 1186& 593\\
1780& 890& 445& 1336& 668& 334& 167& 502& 251& 754\\
377& 1132& 566& 283& 850& 425& 1276& 638& 319& 958\\
479& 1438& 719& 2158& 1079& 3238& 1619& 4858& 2429& 7288\\
3644& 1822& 911& 2734& 1367& 4102& 2051& 6154& 3077& 9232\\
4616& 2308& 1154& 577& 1732& 866& 433& 1300& 650& 325\\
976& 488& 244& 122& 61& 184& 92& 46& 23& 70\\
35& 106& 53& 160& 80& 40& 20& 10& 5& 16\\
8& 4& 2& 1& \\

2197&&&&&&&&&\\
6592& 3296& 1648& 824& 412& 206& 103& 310& 155& 466\\
233& 700& 350& 175& 526& 263& 790& 395& 1186& 593\\
1780& 890& 445& 1336& 668& 334& 167& 502& 251& 754\\
377& 1132& 566& 283& 850& 425& 1276& 638& 319& 958\\
479& 1438& 719& 2158& 1079& 3238& 1619& 4858& 2429& 7288\\
3644& 1822& 911& 2734& 1367& 4102& 2051& 6154& 3077& 9232\\
4616& 2308& 1154& 577& 1732& 866& 433& 1300& 650& 325\\
976& 488& 244& 122& 61& 184& 92& 46& 23& 70\\
35& 106& 53& 160& 80& 40& 20& 10& 5& 16\\
8& 4& 2& 1& \\

2198&&&&&&&&&\\
1099& 3298& 1649& 4948& 2474& 1237& 3712& 1856& 928& 464\\
232& 116& 58& 29& 88& 44& 22& 11& 34& 17\\
52& 26& 13& 40& 20& 10& 5& 16& 8& 4\\
2& 1& \\

2199&&&&&&&&&\\
6598& 3299& 9898& 4949& 14848& 7424& 3712& 1856& 928& 464\\
232& 116& 58& 29& 88& 44& 22& 11& 34& 17\\
52& 26& 13& 40& 20& 10& 5& 16& 8& 4\\
2& 1& \\

2200&&&&&&&&&\\
1100& 550& 275& 826& 413& 1240& 620& 310& 155& 466\\
233& 700& 350& 175& 526& 263& 790& 395& 1186& 593\\
1780& 890& 445& 1336& 668& 334& 167& 502& 251& 754\\
377& 1132& 566& 283& 850& 425& 1276& 638& 319& 958\\
479& 1438& 719& 2158& 1079& 3238& 1619& 4858& 2429& 7288\\
3644& 1822& 911& 2734& 1367& 4102& 2051& 6154& 3077& 9232\\
4616& 2308& 1154& 577& 1732& 866& 433& 1300& 650& 325\\
976& 488& 244& 122& 61& 184& 92& 46& 23& 70\\
35& 106& 53& 160& 80& 40& 20& 10& 5& 16\\
8& 4& 2& 1& \\

2201&&&&&&&&&\\
6604& 3302& 1651& 4954& 2477& 7432& 3716& 1858& 929& 2788\\
1394& 697& 2092& 1046& 523& 1570& 785& 2356& 1178& 589\\
1768& 884& 442& 221& 664& 332& 166& 83& 250& 125\\
376& 188& 94& 47& 142& 71& 214& 107& 322& 161\\
484& 242& 121& 364& 182& 91& 274& 137& 412& 206\\
103& 310& 155& 466& 233& 700& 350& 175& 526& 263\\
790& 395& 1186& 593& 1780& 890& 445& 1336& 668& 334\\
167& 502& 251& 754& 377& 1132& 566& 283& 850& 425\\
1276& 638& 319& 958& 479& 1438& 719& 2158& 1079& 3238\\
1619& 4858& 2429& 7288& 3644& 1822& 911& 2734& 1367& 4102\\
2051& 6154& 3077& 9232& 4616& 2308& 1154& 577& 1732& 866\\
433& 1300& 650& 325& 976& 488& 244& 122& 61& 184\\
92& 46& 23& 70& 35& 106& 53& 160& 80& 40\\
20& 10& 5& 16& 8& 4& 2& 1& \\

2202&&&&&&&&&\\
1101& 3304& 1652& 826& 413& 1240& 620& 310& 155& 466\\
233& 700& 350& 175& 526& 263& 790& 395& 1186& 593\\
1780& 890& 445& 1336& 668& 334& 167& 502& 251& 754\\
377& 1132& 566& 283& 850& 425& 1276& 638& 319& 958\\
479& 1438& 719& 2158& 1079& 3238& 1619& 4858& 2429& 7288\\
3644& 1822& 911& 2734& 1367& 4102& 2051& 6154& 3077& 9232\\
4616& 2308& 1154& 577& 1732& 866& 433& 1300& 650& 325\\
976& 488& 244& 122& 61& 184& 92& 46& 23& 70\\
35& 106& 53& 160& 80& 40& 20& 10& 5& 16\\
8& 4& 2& 1& \\

2203&&&&&&&&&\\
6610& 3305& 9916& 4958& 2479& 7438& 3719& 11158& 5579& 16738\\
8369& 25108& 12554& 6277& 18832& 9416& 4708& 2354& 1177& 3532\\
1766& 883& 2650& 1325& 3976& 1988& 994& 497& 1492& 746\\
373& 1120& 560& 280& 140& 70& 35& 106& 53& 160\\
80& 40& 20& 10& 5& 16& 8& 4& 2& 1\\

2204&&&&&&&&&\\
1102& 551& 1654& 827& 2482& 1241& 3724& 1862& 931& 2794\\
1397& 4192& 2096& 1048& 524& 262& 131& 394& 197& 592\\
296& 148& 74& 37& 112& 56& 28& 14& 7& 22\\
11& 34& 17& 52& 26& 13& 40& 20& 10& 5\\
16& 8& 4& 2& 1& \\

2205&&&&&&&&&\\
6616& 3308& 1654& 827& 2482& 1241& 3724& 1862& 931& 2794\\
1397& 4192& 2096& 1048& 524& 262& 131& 394& 197& 592\\
296& 148& 74& 37& 112& 56& 28& 14& 7& 22\\
11& 34& 17& 52& 26& 13& 40& 20& 10& 5\\
16& 8& 4& 2& 1& \\

2206&&&&&&&&&\\
1103& 3310& 1655& 4966& 2483& 7450& 3725& 11176& 5588& 2794\\
1397& 4192& 2096& 1048& 524& 262& 131& 394& 197& 592\\
296& 148& 74& 37& 112& 56& 28& 14& 7& 22\\
11& 34& 17& 52& 26& 13& 40& 20& 10& 5\\
16& 8& 4& 2& 1& \\

2207&&&&&&&&&\\
6622& 3311& 9934& 4967& 14902& 7451& 22354& 11177& 33532& 16766\\
8383& 25150& 12575& 37726& 18863& 56590& 28295& 84886& 42443& 127330\\
63665& 190996& 95498& 47749& 143248& 71624& 35812& 17906& 8953& 26860\\
13430& 6715& 20146& 10073& 30220& 15110& 7555& 22666& 11333& 34000\\
17000& 8500& 4250& 2125& 6376& 3188& 1594& 797& 2392& 1196\\
598& 299& 898& 449& 1348& 674& 337& 1012& 506& 253\\
760& 380& 190& 95& 286& 143& 430& 215& 646& 323\\
970& 485& 1456& 728& 364& 182& 91& 274& 137& 412\\
206& 103& 310& 155& 466& 233& 700& 350& 175& 526\\
263& 790& 395& 1186& 593& 1780& 890& 445& 1336& 668\\
334& 167& 502& 251& 754& 377& 1132& 566& 283& 850\\
425& 1276& 638& 319& 958& 479& 1438& 719& 2158& 1079\\
3238& 1619& 4858& 2429& 7288& 3644& 1822& 911& 2734& 1367\\
4102& 2051& 6154& 3077& 9232& 4616& 2308& 1154& 577& 1732\\
866& 433& 1300& 650& 325& 976& 488& 244& 122& 61\\
184& 92& 46& 23& 70& 35& 106& 53& 160& 80\\
40& 20& 10& 5& 16& 8& 4& 2& 1& \\

2208&&&&&&&&&\\
1104& 552& 276& 138& 69& 208& 104& 52& 26& 13\\
40& 20& 10& 5& 16& 8& 4& 2& 1& \\

2209&&&&&&&&&\\
6628& 3314& 1657& 4972& 2486& 1243& 3730& 1865& 5596& 2798\\
1399& 4198& 2099& 6298& 3149& 9448& 4724& 2362& 1181& 3544\\
1772& 886& 443& 1330& 665& 1996& 998& 499& 1498& 749\\
2248& 1124& 562& 281& 844& 422& 211& 634& 317& 952\\
476& 238& 119& 358& 179& 538& 269& 808& 404& 202\\
101& 304& 152& 76& 38& 19& 58& 29& 88& 44\\
22& 11& 34& 17& 52& 26& 13& 40& 20& 10\\
5& 16& 8& 4& 2& 1& \\

2210&&&&&&&&&\\
1105& 3316& 1658& 829& 2488& 1244& 622& 311& 934& 467\\
1402& 701& 2104& 1052& 526& 263& 790& 395& 1186& 593\\
1780& 890& 445& 1336& 668& 334& 167& 502& 251& 754\\
377& 1132& 566& 283& 850& 425& 1276& 638& 319& 958\\
479& 1438& 719& 2158& 1079& 3238& 1619& 4858& 2429& 7288\\
3644& 1822& 911& 2734& 1367& 4102& 2051& 6154& 3077& 9232\\
4616& 2308& 1154& 577& 1732& 866& 433& 1300& 650& 325\\
976& 488& 244& 122& 61& 184& 92& 46& 23& 70\\
35& 106& 53& 160& 80& 40& 20& 10& 5& 16\\
8& 4& 2& 1& \\

2211&&&&&&&&&\\
6634& 3317& 9952& 4976& 2488& 1244& 622& 311& 934& 467\\
1402& 701& 2104& 1052& 526& 263& 790& 395& 1186& 593\\
1780& 890& 445& 1336& 668& 334& 167& 502& 251& 754\\
377& 1132& 566& 283& 850& 425& 1276& 638& 319& 958\\
479& 1438& 719& 2158& 1079& 3238& 1619& 4858& 2429& 7288\\
3644& 1822& 911& 2734& 1367& 4102& 2051& 6154& 3077& 9232\\
4616& 2308& 1154& 577& 1732& 866& 433& 1300& 650& 325\\
976& 488& 244& 122& 61& 184& 92& 46& 23& 70\\
35& 106& 53& 160& 80& 40& 20& 10& 5& 16\\
8& 4& 2& 1& \\

2212&&&&&&&&&\\
1106& 553& 1660& 830& 415& 1246& 623& 1870& 935& 2806\\
1403& 4210& 2105& 6316& 3158& 1579& 4738& 2369& 7108& 3554\\
1777& 5332& 2666& 1333& 4000& 2000& 1000& 500& 250& 125\\
376& 188& 94& 47& 142& 71& 214& 107& 322& 161\\
484& 242& 121& 364& 182& 91& 274& 137& 412& 206\\
103& 310& 155& 466& 233& 700& 350& 175& 526& 263\\
790& 395& 1186& 593& 1780& 890& 445& 1336& 668& 334\\
167& 502& 251& 754& 377& 1132& 566& 283& 850& 425\\
1276& 638& 319& 958& 479& 1438& 719& 2158& 1079& 3238\\
1619& 4858& 2429& 7288& 3644& 1822& 911& 2734& 1367& 4102\\
2051& 6154& 3077& 9232& 4616& 2308& 1154& 577& 1732& 866\\
433& 1300& 650& 325& 976& 488& 244& 122& 61& 184\\
92& 46& 23& 70& 35& 106& 53& 160& 80& 40\\
20& 10& 5& 16& 8& 4& 2& 1& \\

2213&&&&&&&&&\\
6640& 3320& 1660& 830& 415& 1246& 623& 1870& 935& 2806\\
1403& 4210& 2105& 6316& 3158& 1579& 4738& 2369& 7108& 3554\\
1777& 5332& 2666& 1333& 4000& 2000& 1000& 500& 250& 125\\
376& 188& 94& 47& 142& 71& 214& 107& 322& 161\\
484& 242& 121& 364& 182& 91& 274& 137& 412& 206\\
103& 310& 155& 466& 233& 700& 350& 175& 526& 263\\
790& 395& 1186& 593& 1780& 890& 445& 1336& 668& 334\\
167& 502& 251& 754& 377& 1132& 566& 283& 850& 425\\
1276& 638& 319& 958& 479& 1438& 719& 2158& 1079& 3238\\
1619& 4858& 2429& 7288& 3644& 1822& 911& 2734& 1367& 4102\\
2051& 6154& 3077& 9232& 4616& 2308& 1154& 577& 1732& 866\\
433& 1300& 650& 325& 976& 488& 244& 122& 61& 184\\
92& 46& 23& 70& 35& 106& 53& 160& 80& 40\\
20& 10& 5& 16& 8& 4& 2& 1& \\

2214&&&&&&&&&\\
1107& 3322& 1661& 4984& 2492& 1246& 623& 1870& 935& 2806\\
1403& 4210& 2105& 6316& 3158& 1579& 4738& 2369& 7108& 3554\\
1777& 5332& 2666& 1333& 4000& 2000& 1000& 500& 250& 125\\
376& 188& 94& 47& 142& 71& 214& 107& 322& 161\\
484& 242& 121& 364& 182& 91& 274& 137& 412& 206\\
103& 310& 155& 466& 233& 700& 350& 175& 526& 263\\
790& 395& 1186& 593& 1780& 890& 445& 1336& 668& 334\\
167& 502& 251& 754& 377& 1132& 566& 283& 850& 425\\
1276& 638& 319& 958& 479& 1438& 719& 2158& 1079& 3238\\
1619& 4858& 2429& 7288& 3644& 1822& 911& 2734& 1367& 4102\\
2051& 6154& 3077& 9232& 4616& 2308& 1154& 577& 1732& 866\\
433& 1300& 650& 325& 976& 488& 244& 122& 61& 184\\
92& 46& 23& 70& 35& 106& 53& 160& 80& 40\\
20& 10& 5& 16& 8& 4& 2& 1& \\

2215&&&&&&&&&\\
6646& 3323& 9970& 4985& 14956& 7478& 3739& 11218& 5609& 16828\\
8414& 4207& 12622& 6311& 18934& 9467& 28402& 14201& 42604& 21302\\
10651& 31954& 15977& 47932& 23966& 11983& 35950& 17975& 53926& 26963\\
80890& 40445& 121336& 60668& 30334& 15167& 45502& 22751& 68254& 34127\\
102382& 51191& 153574& 76787& 230362& 115181& 345544& 172772& 86386& 43193\\
129580& 64790& 32395& 97186& 48593& 145780& 72890& 36445& 109336& 54668\\
27334& 13667& 41002& 20501& 61504& 30752& 15376& 7688& 3844& 1922\\
961& 2884& 1442& 721& 2164& 1082& 541& 1624& 812& 406\\
203& 610& 305& 916& 458& 229& 688& 344& 172& 86\\
43& 130& 65& 196& 98& 49& 148& 74& 37& 112\\
56& 28& 14& 7& 22& 11& 34& 17& 52& 26\\
13& 40& 20& 10& 5& 16& 8& 4& 2& 1\\

2216&&&&&&&&&\\
1108& 554& 277& 832& 416& 208& 104& 52& 26& 13\\
40& 20& 10& 5& 16& 8& 4& 2& 1& \\

2217&&&&&&&&&\\
6652& 3326& 1663& 4990& 2495& 7486& 3743& 11230& 5615& 16846\\
8423& 25270& 12635& 37906& 18953& 56860& 28430& 14215& 42646& 21323\\
63970& 31985& 95956& 47978& 23989& 71968& 35984& 17992& 8996& 4498\\
2249& 6748& 3374& 1687& 5062& 2531& 7594& 3797& 11392& 5696\\
2848& 1424& 712& 356& 178& 89& 268& 134& 67& 202\\
101& 304& 152& 76& 38& 19& 58& 29& 88& 44\\
22& 11& 34& 17& 52& 26& 13& 40& 20& 10\\
5& 16& 8& 4& 2& 1& \\

2218&&&&&&&&&\\
1109& 3328& 1664& 832& 416& 208& 104& 52& 26& 13\\
40& 20& 10& 5& 16& 8& 4& 2& 1& \\

2219&&&&&&&&&\\
6658& 3329& 9988& 4994& 2497& 7492& 3746& 1873& 5620& 2810\\
1405& 4216& 2108& 1054& 527& 1582& 791& 2374& 1187& 3562\\
1781& 5344& 2672& 1336& 668& 334& 167& 502& 251& 754\\
377& 1132& 566& 283& 850& 425& 1276& 638& 319& 958\\
479& 1438& 719& 2158& 1079& 3238& 1619& 4858& 2429& 7288\\
3644& 1822& 911& 2734& 1367& 4102& 2051& 6154& 3077& 9232\\
4616& 2308& 1154& 577& 1732& 866& 433& 1300& 650& 325\\
976& 488& 244& 122& 61& 184& 92& 46& 23& 70\\
35& 106& 53& 160& 80& 40& 20& 10& 5& 16\\
8& 4& 2& 1& \\

2220&&&&&&&&&\\
1110& 555& 1666& 833& 2500& 1250& 625& 1876& 938& 469\\
1408& 704& 352& 176& 88& 44& 22& 11& 34& 17\\
52& 26& 13& 40& 20& 10& 5& 16& 8& 4\\
2& 1& \\

2221&&&&&&&&&\\
6664& 3332& 1666& 833& 2500& 1250& 625& 1876& 938& 469\\
1408& 704& 352& 176& 88& 44& 22& 11& 34& 17\\
52& 26& 13& 40& 20& 10& 5& 16& 8& 4\\
2& 1& \\

2222&&&&&&&&&\\
1111& 3334& 1667& 5002& 2501& 7504& 3752& 1876& 938& 469\\
1408& 704& 352& 176& 88& 44& 22& 11& 34& 17\\
52& 26& 13& 40& 20& 10& 5& 16& 8& 4\\
2& 1& \\

2223&&&&&&&&&\\
6670& 3335& 10006& 5003& 15010& 7505& 22516& 11258& 5629& 16888\\
8444& 4222& 2111& 6334& 3167& 9502& 4751& 14254& 7127& 21382\\
10691& 32074& 16037& 48112& 24056& 12028& 6014& 3007& 9022& 4511\\
13534& 6767& 20302& 10151& 30454& 15227& 45682& 22841& 68524& 34262\\
17131& 51394& 25697& 77092& 38546& 19273& 57820& 28910& 14455& 43366\\
21683& 65050& 32525& 97576& 48788& 24394& 12197& 36592& 18296& 9148\\
4574& 2287& 6862& 3431& 10294& 5147& 15442& 7721& 23164& 11582\\
5791& 17374& 8687& 26062& 13031& 39094& 19547& 58642& 29321& 87964\\
43982& 21991& 65974& 32987& 98962& 49481& 148444& 74222& 37111& 111334\\
55667& 167002& 83501& 250504& 125252& 62626& 31313& 93940& 46970& 23485\\
70456& 35228& 17614& 8807& 26422& 13211& 39634& 19817& 59452& 29726\\
14863& 44590& 22295& 66886& 33443& 100330& 50165& 150496& 75248& 37624\\
18812& 9406& 4703& 14110& 7055& 21166& 10583& 31750& 15875& 47626\\
23813& 71440& 35720& 17860& 8930& 4465& 13396& 6698& 3349& 10048\\
5024& 2512& 1256& 628& 314& 157& 472& 236& 118& 59\\
178& 89& 268& 134& 67& 202& 101& 304& 152& 76\\
38& 19& 58& 29& 88& 44& 22& 11& 34& 17\\
52& 26& 13& 40& 20& 10& 5& 16& 8& 4\\
2& 1& \\

2224&&&&&&&&&\\
1112& 556& 278& 139& 418& 209& 628& 314& 157& 472\\
236& 118& 59& 178& 89& 268& 134& 67& 202& 101\\
304& 152& 76& 38& 19& 58& 29& 88& 44& 22\\
11& 34& 17& 52& 26& 13& 40& 20& 10& 5\\
16& 8& 4& 2& 1& \\

2225&&&&&&&&&\\
6676& 3338& 1669& 5008& 2504& 1252& 626& 313& 940& 470\\
235& 706& 353& 1060& 530& 265& 796& 398& 199& 598\\
299& 898& 449& 1348& 674& 337& 1012& 506& 253& 760\\
380& 190& 95& 286& 143& 430& 215& 646& 323& 970\\
485& 1456& 728& 364& 182& 91& 274& 137& 412& 206\\
103& 310& 155& 466& 233& 700& 350& 175& 526& 263\\
790& 395& 1186& 593& 1780& 890& 445& 1336& 668& 334\\
167& 502& 251& 754& 377& 1132& 566& 283& 850& 425\\
1276& 638& 319& 958& 479& 1438& 719& 2158& 1079& 3238\\
1619& 4858& 2429& 7288& 3644& 1822& 911& 2734& 1367& 4102\\
2051& 6154& 3077& 9232& 4616& 2308& 1154& 577& 1732& 866\\
433& 1300& 650& 325& 976& 488& 244& 122& 61& 184\\
92& 46& 23& 70& 35& 106& 53& 160& 80& 40\\
20& 10& 5& 16& 8& 4& 2& 1& \\

2226&&&&&&&&&\\
1113& 3340& 1670& 835& 2506& 1253& 3760& 1880& 940& 470\\
235& 706& 353& 1060& 530& 265& 796& 398& 199& 598\\
299& 898& 449& 1348& 674& 337& 1012& 506& 253& 760\\
380& 190& 95& 286& 143& 430& 215& 646& 323& 970\\
485& 1456& 728& 364& 182& 91& 274& 137& 412& 206\\
103& 310& 155& 466& 233& 700& 350& 175& 526& 263\\
790& 395& 1186& 593& 1780& 890& 445& 1336& 668& 334\\
167& 502& 251& 754& 377& 1132& 566& 283& 850& 425\\
1276& 638& 319& 958& 479& 1438& 719& 2158& 1079& 3238\\
1619& 4858& 2429& 7288& 3644& 1822& 911& 2734& 1367& 4102\\
2051& 6154& 3077& 9232& 4616& 2308& 1154& 577& 1732& 866\\
433& 1300& 650& 325& 976& 488& 244& 122& 61& 184\\
92& 46& 23& 70& 35& 106& 53& 160& 80& 40\\
20& 10& 5& 16& 8& 4& 2& 1& \\

2227&&&&&&&&&\\
6682& 3341& 10024& 5012& 2506& 1253& 3760& 1880& 940& 470\\
235& 706& 353& 1060& 530& 265& 796& 398& 199& 598\\
299& 898& 449& 1348& 674& 337& 1012& 506& 253& 760\\
380& 190& 95& 286& 143& 430& 215& 646& 323& 970\\
485& 1456& 728& 364& 182& 91& 274& 137& 412& 206\\
103& 310& 155& 466& 233& 700& 350& 175& 526& 263\\
790& 395& 1186& 593& 1780& 890& 445& 1336& 668& 334\\
167& 502& 251& 754& 377& 1132& 566& 283& 850& 425\\
1276& 638& 319& 958& 479& 1438& 719& 2158& 1079& 3238\\
1619& 4858& 2429& 7288& 3644& 1822& 911& 2734& 1367& 4102\\
2051& 6154& 3077& 9232& 4616& 2308& 1154& 577& 1732& 866\\
433& 1300& 650& 325& 976& 488& 244& 122& 61& 184\\
92& 46& 23& 70& 35& 106& 53& 160& 80& 40\\
20& 10& 5& 16& 8& 4& 2& 1& \\

2228&&&&&&&&&\\
1114& 557& 1672& 836& 418& 209& 628& 314& 157& 472\\
236& 118& 59& 178& 89& 268& 134& 67& 202& 101\\
304& 152& 76& 38& 19& 58& 29& 88& 44& 22\\
11& 34& 17& 52& 26& 13& 40& 20& 10& 5\\
16& 8& 4& 2& 1& \\

2229&&&&&&&&&\\
6688& 3344& 1672& 836& 418& 209& 628& 314& 157& 472\\
236& 118& 59& 178& 89& 268& 134& 67& 202& 101\\
304& 152& 76& 38& 19& 58& 29& 88& 44& 22\\
11& 34& 17& 52& 26& 13& 40& 20& 10& 5\\
16& 8& 4& 2& 1& \\

2230&&&&&&&&&\\
1115& 3346& 1673& 5020& 2510& 1255& 3766& 1883& 5650& 2825\\
8476& 4238& 2119& 6358& 3179& 9538& 4769& 14308& 7154& 3577\\
10732& 5366& 2683& 8050& 4025& 12076& 6038& 3019& 9058& 4529\\
13588& 6794& 3397& 10192& 5096& 2548& 1274& 637& 1912& 956\\
478& 239& 718& 359& 1078& 539& 1618& 809& 2428& 1214\\
607& 1822& 911& 2734& 1367& 4102& 2051& 6154& 3077& 9232\\
4616& 2308& 1154& 577& 1732& 866& 433& 1300& 650& 325\\
976& 488& 244& 122& 61& 184& 92& 46& 23& 70\\
35& 106& 53& 160& 80& 40& 20& 10& 5& 16\\
8& 4& 2& 1& \\

2231&&&&&&&&&\\
6694& 3347& 10042& 5021& 15064& 7532& 3766& 1883& 5650& 2825\\
8476& 4238& 2119& 6358& 3179& 9538& 4769& 14308& 7154& 3577\\
10732& 5366& 2683& 8050& 4025& 12076& 6038& 3019& 9058& 4529\\
13588& 6794& 3397& 10192& 5096& 2548& 1274& 637& 1912& 956\\
478& 239& 718& 359& 1078& 539& 1618& 809& 2428& 1214\\
607& 1822& 911& 2734& 1367& 4102& 2051& 6154& 3077& 9232\\
4616& 2308& 1154& 577& 1732& 866& 433& 1300& 650& 325\\
976& 488& 244& 122& 61& 184& 92& 46& 23& 70\\
35& 106& 53& 160& 80& 40& 20& 10& 5& 16\\
8& 4& 2& 1& \\

2232&&&&&&&&&\\
1116& 558& 279& 838& 419& 1258& 629& 1888& 944& 472\\
236& 118& 59& 178& 89& 268& 134& 67& 202& 101\\
304& 152& 76& 38& 19& 58& 29& 88& 44& 22\\
11& 34& 17& 52& 26& 13& 40& 20& 10& 5\\
16& 8& 4& 2& 1& \\

2233&&&&&&&&&\\
6700& 3350& 1675& 5026& 2513& 7540& 3770& 1885& 5656& 2828\\
1414& 707& 2122& 1061& 3184& 1592& 796& 398& 199& 598\\
299& 898& 449& 1348& 674& 337& 1012& 506& 253& 760\\
380& 190& 95& 286& 143& 430& 215& 646& 323& 970\\
485& 1456& 728& 364& 182& 91& 274& 137& 412& 206\\
103& 310& 155& 466& 233& 700& 350& 175& 526& 263\\
790& 395& 1186& 593& 1780& 890& 445& 1336& 668& 334\\
167& 502& 251& 754& 377& 1132& 566& 283& 850& 425\\
1276& 638& 319& 958& 479& 1438& 719& 2158& 1079& 3238\\
1619& 4858& 2429& 7288& 3644& 1822& 911& 2734& 1367& 4102\\
2051& 6154& 3077& 9232& 4616& 2308& 1154& 577& 1732& 866\\
433& 1300& 650& 325& 976& 488& 244& 122& 61& 184\\
92& 46& 23& 70& 35& 106& 53& 160& 80& 40\\
20& 10& 5& 16& 8& 4& 2& 1& \\

2234&&&&&&&&&\\
1117& 3352& 1676& 838& 419& 1258& 629& 1888& 944& 472\\
236& 118& 59& 178& 89& 268& 134& 67& 202& 101\\
304& 152& 76& 38& 19& 58& 29& 88& 44& 22\\
11& 34& 17& 52& 26& 13& 40& 20& 10& 5\\
16& 8& 4& 2& 1& \\

2235&&&&&&&&&\\
6706& 3353& 10060& 5030& 2515& 7546& 3773& 11320& 5660& 2830\\
1415& 4246& 2123& 6370& 3185& 9556& 4778& 2389& 7168& 3584\\
1792& 896& 448& 224& 112& 56& 28& 14& 7& 22\\
11& 34& 17& 52& 26& 13& 40& 20& 10& 5\\
16& 8& 4& 2& 1& \\

2236&&&&&&&&&\\
1118& 559& 1678& 839& 2518& 1259& 3778& 1889& 5668& 2834\\
1417& 4252& 2126& 1063& 3190& 1595& 4786& 2393& 7180& 3590\\
1795& 5386& 2693& 8080& 4040& 2020& 1010& 505& 1516& 758\\
379& 1138& 569& 1708& 854& 427& 1282& 641& 1924& 962\\
481& 1444& 722& 361& 1084& 542& 271& 814& 407& 1222\\
611& 1834& 917& 2752& 1376& 688& 344& 172& 86& 43\\
130& 65& 196& 98& 49& 148& 74& 37& 112& 56\\
28& 14& 7& 22& 11& 34& 17& 52& 26& 13\\
40& 20& 10& 5& 16& 8& 4& 2& 1& \\

2237&&&&&&&&&\\
6712& 3356& 1678& 839& 2518& 1259& 3778& 1889& 5668& 2834\\
1417& 4252& 2126& 1063& 3190& 1595& 4786& 2393& 7180& 3590\\
1795& 5386& 2693& 8080& 4040& 2020& 1010& 505& 1516& 758\\
379& 1138& 569& 1708& 854& 427& 1282& 641& 1924& 962\\
481& 1444& 722& 361& 1084& 542& 271& 814& 407& 1222\\
611& 1834& 917& 2752& 1376& 688& 344& 172& 86& 43\\
130& 65& 196& 98& 49& 148& 74& 37& 112& 56\\
28& 14& 7& 22& 11& 34& 17& 52& 26& 13\\
40& 20& 10& 5& 16& 8& 4& 2& 1& \\

2238&&&&&&&&&\\
1119& 3358& 1679& 5038& 2519& 7558& 3779& 11338& 5669& 17008\\
8504& 4252& 2126& 1063& 3190& 1595& 4786& 2393& 7180& 3590\\
1795& 5386& 2693& 8080& 4040& 2020& 1010& 505& 1516& 758\\
379& 1138& 569& 1708& 854& 427& 1282& 641& 1924& 962\\
481& 1444& 722& 361& 1084& 542& 271& 814& 407& 1222\\
611& 1834& 917& 2752& 1376& 688& 344& 172& 86& 43\\
130& 65& 196& 98& 49& 148& 74& 37& 112& 56\\
28& 14& 7& 22& 11& 34& 17& 52& 26& 13\\
40& 20& 10& 5& 16& 8& 4& 2& 1& \\

2239&&&&&&&&&\\
6718& 3359& 10078& 5039& 15118& 7559& 22678& 11339& 34018& 17009\\
51028& 25514& 12757& 38272& 19136& 9568& 4784& 2392& 1196& 598\\
299& 898& 449& 1348& 674& 337& 1012& 506& 253& 760\\
380& 190& 95& 286& 143& 430& 215& 646& 323& 970\\
485& 1456& 728& 364& 182& 91& 274& 137& 412& 206\\
103& 310& 155& 466& 233& 700& 350& 175& 526& 263\\
790& 395& 1186& 593& 1780& 890& 445& 1336& 668& 334\\
167& 502& 251& 754& 377& 1132& 566& 283& 850& 425\\
1276& 638& 319& 958& 479& 1438& 719& 2158& 1079& 3238\\
1619& 4858& 2429& 7288& 3644& 1822& 911& 2734& 1367& 4102\\
2051& 6154& 3077& 9232& 4616& 2308& 1154& 577& 1732& 866\\
433& 1300& 650& 325& 976& 488& 244& 122& 61& 184\\
92& 46& 23& 70& 35& 106& 53& 160& 80& 40\\
20& 10& 5& 16& 8& 4& 2& 1& \\

2240&&&&&&&&&\\
1120& 560& 280& 140& 70& 35& 106& 53& 160& 80\\
40& 20& 10& 5& 16& 8& 4& 2& 1& \\

2241&&&&&&&&&\\
6724& 3362& 1681& 5044& 2522& 1261& 3784& 1892& 946& 473\\
1420& 710& 355& 1066& 533& 1600& 800& 400& 200& 100\\
50& 25& 76& 38& 19& 58& 29& 88& 44& 22\\
11& 34& 17& 52& 26& 13& 40& 20& 10& 5\\
16& 8& 4& 2& 1& \\

2242&&&&&&&&&\\
1121& 3364& 1682& 841& 2524& 1262& 631& 1894& 947& 2842\\
1421& 4264& 2132& 1066& 533& 1600& 800& 400& 200& 100\\
50& 25& 76& 38& 19& 58& 29& 88& 44& 22\\
11& 34& 17& 52& 26& 13& 40& 20& 10& 5\\
16& 8& 4& 2& 1& \\

2243&&&&&&&&&\\
6730& 3365& 10096& 5048& 2524& 1262& 631& 1894& 947& 2842\\
1421& 4264& 2132& 1066& 533& 1600& 800& 400& 200& 100\\
50& 25& 76& 38& 19& 58& 29& 88& 44& 22\\
11& 34& 17& 52& 26& 13& 40& 20& 10& 5\\
16& 8& 4& 2& 1& \\

2244&&&&&&&&&\\
1122& 561& 1684& 842& 421& 1264& 632& 316& 158& 79\\
238& 119& 358& 179& 538& 269& 808& 404& 202& 101\\
304& 152& 76& 38& 19& 58& 29& 88& 44& 22\\
11& 34& 17& 52& 26& 13& 40& 20& 10& 5\\
16& 8& 4& 2& 1& \\

2245&&&&&&&&&\\
6736& 3368& 1684& 842& 421& 1264& 632& 316& 158& 79\\
238& 119& 358& 179& 538& 269& 808& 404& 202& 101\\
304& 152& 76& 38& 19& 58& 29& 88& 44& 22\\
11& 34& 17& 52& 26& 13& 40& 20& 10& 5\\
16& 8& 4& 2& 1& \\

2246&&&&&&&&&\\
1123& 3370& 1685& 5056& 2528& 1264& 632& 316& 158& 79\\
238& 119& 358& 179& 538& 269& 808& 404& 202& 101\\
304& 152& 76& 38& 19& 58& 29& 88& 44& 22\\
11& 34& 17& 52& 26& 13& 40& 20& 10& 5\\
16& 8& 4& 2& 1& \\

2247&&&&&&&&&\\
6742& 3371& 10114& 5057& 15172& 7586& 3793& 11380& 5690& 2845\\
8536& 4268& 2134& 1067& 3202& 1601& 4804& 2402& 1201& 3604\\
1802& 901& 2704& 1352& 676& 338& 169& 508& 254& 127\\
382& 191& 574& 287& 862& 431& 1294& 647& 1942& 971\\
2914& 1457& 4372& 2186& 1093& 3280& 1640& 820& 410& 205\\
616& 308& 154& 77& 232& 116& 58& 29& 88& 44\\
22& 11& 34& 17& 52& 26& 13& 40& 20& 10\\
5& 16& 8& 4& 2& 1& \\

2248&&&&&&&&&\\
1124& 562& 281& 844& 422& 211& 634& 317& 952& 476\\
238& 119& 358& 179& 538& 269& 808& 404& 202& 101\\
304& 152& 76& 38& 19& 58& 29& 88& 44& 22\\
11& 34& 17& 52& 26& 13& 40& 20& 10& 5\\
16& 8& 4& 2& 1& \\

2249&&&&&&&&&\\
6748& 3374& 1687& 5062& 2531& 7594& 3797& 11392& 5696& 2848\\
1424& 712& 356& 178& 89& 268& 134& 67& 202& 101\\
304& 152& 76& 38& 19& 58& 29& 88& 44& 22\\
11& 34& 17& 52& 26& 13& 40& 20& 10& 5\\
16& 8& 4& 2& 1& \\

2250&&&&&&&&&\\
1125& 3376& 1688& 844& 422& 211& 634& 317& 952& 476\\
238& 119& 358& 179& 538& 269& 808& 404& 202& 101\\
304& 152& 76& 38& 19& 58& 29& 88& 44& 22\\
11& 34& 17& 52& 26& 13& 40& 20& 10& 5\\
16& 8& 4& 2& 1& \\

2251&&&&&&&&&\\
6754& 3377& 10132& 5066& 2533& 7600& 3800& 1900& 950& 475\\
1426& 713& 2140& 1070& 535& 1606& 803& 2410& 1205& 3616\\
1808& 904& 452& 226& 113& 340& 170& 85& 256& 128\\
64& 32& 16& 8& 4& 2& 1& \\

2252&&&&&&&&&\\
1126& 563& 1690& 845& 2536& 1268& 634& 317& 952& 476\\
238& 119& 358& 179& 538& 269& 808& 404& 202& 101\\
304& 152& 76& 38& 19& 58& 29& 88& 44& 22\\
11& 34& 17& 52& 26& 13& 40& 20& 10& 5\\
16& 8& 4& 2& 1& \\

2253&&&&&&&&&\\
6760& 3380& 1690& 845& 2536& 1268& 634& 317& 952& 476\\
238& 119& 358& 179& 538& 269& 808& 404& 202& 101\\
304& 152& 76& 38& 19& 58& 29& 88& 44& 22\\
11& 34& 17& 52& 26& 13& 40& 20& 10& 5\\
16& 8& 4& 2& 1& \\

2254&&&&&&&&&\\
1127& 3382& 1691& 5074& 2537& 7612& 3806& 1903& 5710& 2855\\
8566& 4283& 12850& 6425& 19276& 9638& 4819& 14458& 7229& 21688\\
10844& 5422& 2711& 8134& 4067& 12202& 6101& 18304& 9152& 4576\\
2288& 1144& 572& 286& 143& 430& 215& 646& 323& 970\\
485& 1456& 728& 364& 182& 91& 274& 137& 412& 206\\
103& 310& 155& 466& 233& 700& 350& 175& 526& 263\\
790& 395& 1186& 593& 1780& 890& 445& 1336& 668& 334\\
167& 502& 251& 754& 377& 1132& 566& 283& 850& 425\\
1276& 638& 319& 958& 479& 1438& 719& 2158& 1079& 3238\\
1619& 4858& 2429& 7288& 3644& 1822& 911& 2734& 1367& 4102\\
2051& 6154& 3077& 9232& 4616& 2308& 1154& 577& 1732& 866\\
433& 1300& 650& 325& 976& 488& 244& 122& 61& 184\\
92& 46& 23& 70& 35& 106& 53& 160& 80& 40\\
20& 10& 5& 16& 8& 4& 2& 1& \\

2255&&&&&&&&&\\
6766& 3383& 10150& 5075& 15226& 7613& 22840& 11420& 5710& 2855\\
8566& 4283& 12850& 6425& 19276& 9638& 4819& 14458& 7229& 21688\\
10844& 5422& 2711& 8134& 4067& 12202& 6101& 18304& 9152& 4576\\
2288& 1144& 572& 286& 143& 430& 215& 646& 323& 970\\
485& 1456& 728& 364& 182& 91& 274& 137& 412& 206\\
103& 310& 155& 466& 233& 700& 350& 175& 526& 263\\
790& 395& 1186& 593& 1780& 890& 445& 1336& 668& 334\\
167& 502& 251& 754& 377& 1132& 566& 283& 850& 425\\
1276& 638& 319& 958& 479& 1438& 719& 2158& 1079& 3238\\
1619& 4858& 2429& 7288& 3644& 1822& 911& 2734& 1367& 4102\\
2051& 6154& 3077& 9232& 4616& 2308& 1154& 577& 1732& 866\\
433& 1300& 650& 325& 976& 488& 244& 122& 61& 184\\
92& 46& 23& 70& 35& 106& 53& 160& 80& 40\\
20& 10& 5& 16& 8& 4& 2& 1& \\

2256&&&&&&&&&\\
1128& 564& 282& 141& 424& 212& 106& 53& 160& 80\\
40& 20& 10& 5& 16& 8& 4& 2& 1& \\

2257&&&&&&&&&\\
6772& 3386& 1693& 5080& 2540& 1270& 635& 1906& 953& 2860\\
1430& 715& 2146& 1073& 3220& 1610& 805& 2416& 1208& 604\\
302& 151& 454& 227& 682& 341& 1024& 512& 256& 128\\
64& 32& 16& 8& 4& 2& 1& \\

2258&&&&&&&&&\\
1129& 3388& 1694& 847& 2542& 1271& 3814& 1907& 5722& 2861\\
8584& 4292& 2146& 1073& 3220& 1610& 805& 2416& 1208& 604\\
302& 151& 454& 227& 682& 341& 1024& 512& 256& 128\\
64& 32& 16& 8& 4& 2& 1& \\

2259&&&&&&&&&\\
6778& 3389& 10168& 5084& 2542& 1271& 3814& 1907& 5722& 2861\\
8584& 4292& 2146& 1073& 3220& 1610& 805& 2416& 1208& 604\\
302& 151& 454& 227& 682& 341& 1024& 512& 256& 128\\
64& 32& 16& 8& 4& 2& 1& \\

2260&&&&&&&&&\\
1130& 565& 1696& 848& 424& 212& 106& 53& 160& 80\\
40& 20& 10& 5& 16& 8& 4& 2& 1& \\

2261&&&&&&&&&\\
6784& 3392& 1696& 848& 424& 212& 106& 53& 160& 80\\
40& 20& 10& 5& 16& 8& 4& 2& 1& \\

2262&&&&&&&&&\\
1131& 3394& 1697& 5092& 2546& 1273& 3820& 1910& 955& 2866\\
1433& 4300& 2150& 1075& 3226& 1613& 4840& 2420& 1210& 605\\
1816& 908& 454& 227& 682& 341& 1024& 512& 256& 128\\
64& 32& 16& 8& 4& 2& 1& \\

2263&&&&&&&&&\\
6790& 3395& 10186& 5093& 15280& 7640& 3820& 1910& 955& 2866\\
1433& 4300& 2150& 1075& 3226& 1613& 4840& 2420& 1210& 605\\
1816& 908& 454& 227& 682& 341& 1024& 512& 256& 128\\
64& 32& 16& 8& 4& 2& 1& \\

2264&&&&&&&&&\\
1132& 566& 283& 850& 425& 1276& 638& 319& 958& 479\\
1438& 719& 2158& 1079& 3238& 1619& 4858& 2429& 7288& 3644\\
1822& 911& 2734& 1367& 4102& 2051& 6154& 3077& 9232& 4616\\
2308& 1154& 577& 1732& 866& 433& 1300& 650& 325& 976\\
488& 244& 122& 61& 184& 92& 46& 23& 70& 35\\
106& 53& 160& 80& 40& 20& 10& 5& 16& 8\\
4& 2& 1& \\

2265&&&&&&&&&\\
6796& 3398& 1699& 5098& 2549& 7648& 3824& 1912& 956& 478\\
239& 718& 359& 1078& 539& 1618& 809& 2428& 1214& 607\\
1822& 911& 2734& 1367& 4102& 2051& 6154& 3077& 9232& 4616\\
2308& 1154& 577& 1732& 866& 433& 1300& 650& 325& 976\\
488& 244& 122& 61& 184& 92& 46& 23& 70& 35\\
106& 53& 160& 80& 40& 20& 10& 5& 16& 8\\
4& 2& 1& \\

2266&&&&&&&&&\\
1133& 3400& 1700& 850& 425& 1276& 638& 319& 958& 479\\
1438& 719& 2158& 1079& 3238& 1619& 4858& 2429& 7288& 3644\\
1822& 911& 2734& 1367& 4102& 2051& 6154& 3077& 9232& 4616\\
2308& 1154& 577& 1732& 866& 433& 1300& 650& 325& 976\\
488& 244& 122& 61& 184& 92& 46& 23& 70& 35\\
106& 53& 160& 80& 40& 20& 10& 5& 16& 8\\
4& 2& 1& \\

2267&&&&&&&&&\\
6802& 3401& 10204& 5102& 2551& 7654& 3827& 11482& 5741& 17224\\
8612& 4306& 2153& 6460& 3230& 1615& 4846& 2423& 7270& 3635\\
10906& 5453& 16360& 8180& 4090& 2045& 6136& 3068& 1534& 767\\
2302& 1151& 3454& 1727& 5182& 2591& 7774& 3887& 11662& 5831\\
17494& 8747& 26242& 13121& 39364& 19682& 9841& 29524& 14762& 7381\\
22144& 11072& 5536& 2768& 1384& 692& 346& 173& 520& 260\\
130& 65& 196& 98& 49& 148& 74& 37& 112& 56\\
28& 14& 7& 22& 11& 34& 17& 52& 26& 13\\
40& 20& 10& 5& 16& 8& 4& 2& 1& \\

2268&&&&&&&&&\\
1134& 567& 1702& 851& 2554& 1277& 3832& 1916& 958& 479\\
1438& 719& 2158& 1079& 3238& 1619& 4858& 2429& 7288& 3644\\
1822& 911& 2734& 1367& 4102& 2051& 6154& 3077& 9232& 4616\\
2308& 1154& 577& 1732& 866& 433& 1300& 650& 325& 976\\
488& 244& 122& 61& 184& 92& 46& 23& 70& 35\\
106& 53& 160& 80& 40& 20& 10& 5& 16& 8\\
4& 2& 1& \\

2269&&&&&&&&&\\
6808& 3404& 1702& 851& 2554& 1277& 3832& 1916& 958& 479\\
1438& 719& 2158& 1079& 3238& 1619& 4858& 2429& 7288& 3644\\
1822& 911& 2734& 1367& 4102& 2051& 6154& 3077& 9232& 4616\\
2308& 1154& 577& 1732& 866& 433& 1300& 650& 325& 976\\
488& 244& 122& 61& 184& 92& 46& 23& 70& 35\\
106& 53& 160& 80& 40& 20& 10& 5& 16& 8\\
4& 2& 1& \\

2270&&&&&&&&&\\
1135& 3406& 1703& 5110& 2555& 7666& 3833& 11500& 5750& 2875\\
8626& 4313& 12940& 6470& 3235& 9706& 4853& 14560& 7280& 3640\\
1820& 910& 455& 1366& 683& 2050& 1025& 3076& 1538& 769\\
2308& 1154& 577& 1732& 866& 433& 1300& 650& 325& 976\\
488& 244& 122& 61& 184& 92& 46& 23& 70& 35\\
106& 53& 160& 80& 40& 20& 10& 5& 16& 8\\
4& 2& 1& \\

2271&&&&&&&&&\\
6814& 3407& 10222& 5111& 15334& 7667& 23002& 11501& 34504& 17252\\
8626& 4313& 12940& 6470& 3235& 9706& 4853& 14560& 7280& 3640\\
1820& 910& 455& 1366& 683& 2050& 1025& 3076& 1538& 769\\
2308& 1154& 577& 1732& 866& 433& 1300& 650& 325& 976\\
488& 244& 122& 61& 184& 92& 46& 23& 70& 35\\
106& 53& 160& 80& 40& 20& 10& 5& 16& 8\\
4& 2& 1& \\

2272&&&&&&&&&\\
1136& 568& 284& 142& 71& 214& 107& 322& 161& 484\\
242& 121& 364& 182& 91& 274& 137& 412& 206& 103\\
310& 155& 466& 233& 700& 350& 175& 526& 263& 790\\
395& 1186& 593& 1780& 890& 445& 1336& 668& 334& 167\\
502& 251& 754& 377& 1132& 566& 283& 850& 425& 1276\\
638& 319& 958& 479& 1438& 719& 2158& 1079& 3238& 1619\\
4858& 2429& 7288& 3644& 1822& 911& 2734& 1367& 4102& 2051\\
6154& 3077& 9232& 4616& 2308& 1154& 577& 1732& 866& 433\\
1300& 650& 325& 976& 488& 244& 122& 61& 184& 92\\
46& 23& 70& 35& 106& 53& 160& 80& 40& 20\\
10& 5& 16& 8& 4& 2& 1& \\

2273&&&&&&&&&\\
6820& 3410& 1705& 5116& 2558& 1279& 3838& 1919& 5758& 2879\\
8638& 4319& 12958& 6479& 19438& 9719& 29158& 14579& 43738& 21869\\
65608& 32804& 16402& 8201& 24604& 12302& 6151& 18454& 9227& 27682\\
13841& 41524& 20762& 10381& 31144& 15572& 7786& 3893& 11680& 5840\\
2920& 1460& 730& 365& 1096& 548& 274& 137& 412& 206\\
103& 310& 155& 466& 233& 700& 350& 175& 526& 263\\
790& 395& 1186& 593& 1780& 890& 445& 1336& 668& 334\\
167& 502& 251& 754& 377& 1132& 566& 283& 850& 425\\
1276& 638& 319& 958& 479& 1438& 719& 2158& 1079& 3238\\
1619& 4858& 2429& 7288& 3644& 1822& 911& 2734& 1367& 4102\\
2051& 6154& 3077& 9232& 4616& 2308& 1154& 577& 1732& 866\\
433& 1300& 650& 325& 976& 488& 244& 122& 61& 184\\
92& 46& 23& 70& 35& 106& 53& 160& 80& 40\\
20& 10& 5& 16& 8& 4& 2& 1& \\

2274&&&&&&&&&\\
1137& 3412& 1706& 853& 2560& 1280& 640& 320& 160& 80\\
40& 20& 10& 5& 16& 8& 4& 2& 1& \\

2275&&&&&&&&&\\
6826& 3413& 10240& 5120& 2560& 1280& 640& 320& 160& 80\\
40& 20& 10& 5& 16& 8& 4& 2& 1& \\

2276&&&&&&&&&\\
1138& 569& 1708& 854& 427& 1282& 641& 1924& 962& 481\\
1444& 722& 361& 1084& 542& 271& 814& 407& 1222& 611\\
1834& 917& 2752& 1376& 688& 344& 172& 86& 43& 130\\
65& 196& 98& 49& 148& 74& 37& 112& 56& 28\\
14& 7& 22& 11& 34& 17& 52& 26& 13& 40\\
20& 10& 5& 16& 8& 4& 2& 1& \\

2277&&&&&&&&&\\
6832& 3416& 1708& 854& 427& 1282& 641& 1924& 962& 481\\
1444& 722& 361& 1084& 542& 271& 814& 407& 1222& 611\\
1834& 917& 2752& 1376& 688& 344& 172& 86& 43& 130\\
65& 196& 98& 49& 148& 74& 37& 112& 56& 28\\
14& 7& 22& 11& 34& 17& 52& 26& 13& 40\\
20& 10& 5& 16& 8& 4& 2& 1& \\

2278&&&&&&&&&\\
1139& 3418& 1709& 5128& 2564& 1282& 641& 1924& 962& 481\\
1444& 722& 361& 1084& 542& 271& 814& 407& 1222& 611\\
1834& 917& 2752& 1376& 688& 344& 172& 86& 43& 130\\
65& 196& 98& 49& 148& 74& 37& 112& 56& 28\\
14& 7& 22& 11& 34& 17& 52& 26& 13& 40\\
20& 10& 5& 16& 8& 4& 2& 1& \\

2279&&&&&&&&&\\
6838& 3419& 10258& 5129& 15388& 7694& 3847& 11542& 5771& 17314\\
8657& 25972& 12986& 6493& 19480& 9740& 4870& 2435& 7306& 3653\\
10960& 5480& 2740& 1370& 685& 2056& 1028& 514& 257& 772\\
386& 193& 580& 290& 145& 436& 218& 109& 328& 164\\
82& 41& 124& 62& 31& 94& 47& 142& 71& 214\\
107& 322& 161& 484& 242& 121& 364& 182& 91& 274\\
137& 412& 206& 103& 310& 155& 466& 233& 700& 350\\
175& 526& 263& 790& 395& 1186& 593& 1780& 890& 445\\
1336& 668& 334& 167& 502& 251& 754& 377& 1132& 566\\
283& 850& 425& 1276& 638& 319& 958& 479& 1438& 719\\
2158& 1079& 3238& 1619& 4858& 2429& 7288& 3644& 1822& 911\\
2734& 1367& 4102& 2051& 6154& 3077& 9232& 4616& 2308& 1154\\
577& 1732& 866& 433& 1300& 650& 325& 976& 488& 244\\
122& 61& 184& 92& 46& 23& 70& 35& 106& 53\\
160& 80& 40& 20& 10& 5& 16& 8& 4& 2\\
1& \\

2280&&&&&&&&&\\
1140& 570& 285& 856& 428& 214& 107& 322& 161& 484\\
242& 121& 364& 182& 91& 274& 137& 412& 206& 103\\
310& 155& 466& 233& 700& 350& 175& 526& 263& 790\\
395& 1186& 593& 1780& 890& 445& 1336& 668& 334& 167\\
502& 251& 754& 377& 1132& 566& 283& 850& 425& 1276\\
638& 319& 958& 479& 1438& 719& 2158& 1079& 3238& 1619\\
4858& 2429& 7288& 3644& 1822& 911& 2734& 1367& 4102& 2051\\
6154& 3077& 9232& 4616& 2308& 1154& 577& 1732& 866& 433\\
1300& 650& 325& 976& 488& 244& 122& 61& 184& 92\\
46& 23& 70& 35& 106& 53& 160& 80& 40& 20\\
10& 5& 16& 8& 4& 2& 1& \\

2281&&&&&&&&&\\
6844& 3422& 1711& 5134& 2567& 7702& 3851& 11554& 5777& 17332\\
8666& 4333& 13000& 6500& 3250& 1625& 4876& 2438& 1219& 3658\\
1829& 5488& 2744& 1372& 686& 343& 1030& 515& 1546& 773\\
2320& 1160& 580& 290& 145& 436& 218& 109& 328& 164\\
82& 41& 124& 62& 31& 94& 47& 142& 71& 214\\
107& 322& 161& 484& 242& 121& 364& 182& 91& 274\\
137& 412& 206& 103& 310& 155& 466& 233& 700& 350\\
175& 526& 263& 790& 395& 1186& 593& 1780& 890& 445\\
1336& 668& 334& 167& 502& 251& 754& 377& 1132& 566\\
283& 850& 425& 1276& 638& 319& 958& 479& 1438& 719\\
2158& 1079& 3238& 1619& 4858& 2429& 7288& 3644& 1822& 911\\
2734& 1367& 4102& 2051& 6154& 3077& 9232& 4616& 2308& 1154\\
577& 1732& 866& 433& 1300& 650& 325& 976& 488& 244\\
122& 61& 184& 92& 46& 23& 70& 35& 106& 53\\
160& 80& 40& 20& 10& 5& 16& 8& 4& 2\\
1& \\

2282&&&&&&&&&\\
1141& 3424& 1712& 856& 428& 214& 107& 322& 161& 484\\
242& 121& 364& 182& 91& 274& 137& 412& 206& 103\\
310& 155& 466& 233& 700& 350& 175& 526& 263& 790\\
395& 1186& 593& 1780& 890& 445& 1336& 668& 334& 167\\
502& 251& 754& 377& 1132& 566& 283& 850& 425& 1276\\
638& 319& 958& 479& 1438& 719& 2158& 1079& 3238& 1619\\
4858& 2429& 7288& 3644& 1822& 911& 2734& 1367& 4102& 2051\\
6154& 3077& 9232& 4616& 2308& 1154& 577& 1732& 866& 433\\
1300& 650& 325& 976& 488& 244& 122& 61& 184& 92\\
46& 23& 70& 35& 106& 53& 160& 80& 40& 20\\
10& 5& 16& 8& 4& 2& 1& \\

2283&&&&&&&&&\\
6850& 3425& 10276& 5138& 2569& 7708& 3854& 1927& 5782& 2891\\
8674& 4337& 13012& 6506& 3253& 9760& 4880& 2440& 1220& 610\\
305& 916& 458& 229& 688& 344& 172& 86& 43& 130\\
65& 196& 98& 49& 148& 74& 37& 112& 56& 28\\
14& 7& 22& 11& 34& 17& 52& 26& 13& 40\\
20& 10& 5& 16& 8& 4& 2& 1& \\

2284&&&&&&&&&\\
1142& 571& 1714& 857& 2572& 1286& 643& 1930& 965& 2896\\
1448& 724& 362& 181& 544& 272& 136& 68& 34& 17\\
52& 26& 13& 40& 20& 10& 5& 16& 8& 4\\
2& 1& \\

2285&&&&&&&&&\\
6856& 3428& 1714& 857& 2572& 1286& 643& 1930& 965& 2896\\
1448& 724& 362& 181& 544& 272& 136& 68& 34& 17\\
52& 26& 13& 40& 20& 10& 5& 16& 8& 4\\
2& 1& \\

2286&&&&&&&&&\\
1143& 3430& 1715& 5146& 2573& 7720& 3860& 1930& 965& 2896\\
1448& 724& 362& 181& 544& 272& 136& 68& 34& 17\\
52& 26& 13& 40& 20& 10& 5& 16& 8& 4\\
2& 1& \\

2287&&&&&&&&&\\
6862& 3431& 10294& 5147& 15442& 7721& 23164& 11582& 5791& 17374\\
8687& 26062& 13031& 39094& 19547& 58642& 29321& 87964& 43982& 21991\\
65974& 32987& 98962& 49481& 148444& 74222& 37111& 111334& 55667& 167002\\
83501& 250504& 125252& 62626& 31313& 93940& 46970& 23485& 70456& 35228\\
17614& 8807& 26422& 13211& 39634& 19817& 59452& 29726& 14863& 44590\\
22295& 66886& 33443& 100330& 50165& 150496& 75248& 37624& 18812& 9406\\
4703& 14110& 7055& 21166& 10583& 31750& 15875& 47626& 23813& 71440\\
35720& 17860& 8930& 4465& 13396& 6698& 3349& 10048& 5024& 2512\\
1256& 628& 314& 157& 472& 236& 118& 59& 178& 89\\
268& 134& 67& 202& 101& 304& 152& 76& 38& 19\\
58& 29& 88& 44& 22& 11& 34& 17& 52& 26\\
13& 40& 20& 10& 5& 16& 8& 4& 2& 1\\

2288&&&&&&&&&\\
1144& 572& 286& 143& 430& 215& 646& 323& 970& 485\\
1456& 728& 364& 182& 91& 274& 137& 412& 206& 103\\
310& 155& 466& 233& 700& 350& 175& 526& 263& 790\\
395& 1186& 593& 1780& 890& 445& 1336& 668& 334& 167\\
502& 251& 754& 377& 1132& 566& 283& 850& 425& 1276\\
638& 319& 958& 479& 1438& 719& 2158& 1079& 3238& 1619\\
4858& 2429& 7288& 3644& 1822& 911& 2734& 1367& 4102& 2051\\
6154& 3077& 9232& 4616& 2308& 1154& 577& 1732& 866& 433\\
1300& 650& 325& 976& 488& 244& 122& 61& 184& 92\\
46& 23& 70& 35& 106& 53& 160& 80& 40& 20\\
10& 5& 16& 8& 4& 2& 1& \\

2289&&&&&&&&&\\
6868& 3434& 1717& 5152& 2576& 1288& 644& 322& 161& 484\\
242& 121& 364& 182& 91& 274& 137& 412& 206& 103\\
310& 155& 466& 233& 700& 350& 175& 526& 263& 790\\
395& 1186& 593& 1780& 890& 445& 1336& 668& 334& 167\\
502& 251& 754& 377& 1132& 566& 283& 850& 425& 1276\\
638& 319& 958& 479& 1438& 719& 2158& 1079& 3238& 1619\\
4858& 2429& 7288& 3644& 1822& 911& 2734& 1367& 4102& 2051\\
6154& 3077& 9232& 4616& 2308& 1154& 577& 1732& 866& 433\\
1300& 650& 325& 976& 488& 244& 122& 61& 184& 92\\
46& 23& 70& 35& 106& 53& 160& 80& 40& 20\\
10& 5& 16& 8& 4& 2& 1& \\

2290&&&&&&&&&\\
1145& 3436& 1718& 859& 2578& 1289& 3868& 1934& 967& 2902\\
1451& 4354& 2177& 6532& 3266& 1633& 4900& 2450& 1225& 3676\\
1838& 919& 2758& 1379& 4138& 2069& 6208& 3104& 1552& 776\\
388& 194& 97& 292& 146& 73& 220& 110& 55& 166\\
83& 250& 125& 376& 188& 94& 47& 142& 71& 214\\
107& 322& 161& 484& 242& 121& 364& 182& 91& 274\\
137& 412& 206& 103& 310& 155& 466& 233& 700& 350\\
175& 526& 263& 790& 395& 1186& 593& 1780& 890& 445\\
1336& 668& 334& 167& 502& 251& 754& 377& 1132& 566\\
283& 850& 425& 1276& 638& 319& 958& 479& 1438& 719\\
2158& 1079& 3238& 1619& 4858& 2429& 7288& 3644& 1822& 911\\
2734& 1367& 4102& 2051& 6154& 3077& 9232& 4616& 2308& 1154\\
577& 1732& 866& 433& 1300& 650& 325& 976& 488& 244\\
122& 61& 184& 92& 46& 23& 70& 35& 106& 53\\
160& 80& 40& 20& 10& 5& 16& 8& 4& 2\\
1& \\

2291&&&&&&&&&\\
6874& 3437& 10312& 5156& 2578& 1289& 3868& 1934& 967& 2902\\
1451& 4354& 2177& 6532& 3266& 1633& 4900& 2450& 1225& 3676\\
1838& 919& 2758& 1379& 4138& 2069& 6208& 3104& 1552& 776\\
388& 194& 97& 292& 146& 73& 220& 110& 55& 166\\
83& 250& 125& 376& 188& 94& 47& 142& 71& 214\\
107& 322& 161& 484& 242& 121& 364& 182& 91& 274\\
137& 412& 206& 103& 310& 155& 466& 233& 700& 350\\
175& 526& 263& 790& 395& 1186& 593& 1780& 890& 445\\
1336& 668& 334& 167& 502& 251& 754& 377& 1132& 566\\
283& 850& 425& 1276& 638& 319& 958& 479& 1438& 719\\
2158& 1079& 3238& 1619& 4858& 2429& 7288& 3644& 1822& 911\\
2734& 1367& 4102& 2051& 6154& 3077& 9232& 4616& 2308& 1154\\
577& 1732& 866& 433& 1300& 650& 325& 976& 488& 244\\
122& 61& 184& 92& 46& 23& 70& 35& 106& 53\\
160& 80& 40& 20& 10& 5& 16& 8& 4& 2\\
1& \\

2292&&&&&&&&&\\
1146& 573& 1720& 860& 430& 215& 646& 323& 970& 485\\
1456& 728& 364& 182& 91& 274& 137& 412& 206& 103\\
310& 155& 466& 233& 700& 350& 175& 526& 263& 790\\
395& 1186& 593& 1780& 890& 445& 1336& 668& 334& 167\\
502& 251& 754& 377& 1132& 566& 283& 850& 425& 1276\\
638& 319& 958& 479& 1438& 719& 2158& 1079& 3238& 1619\\
4858& 2429& 7288& 3644& 1822& 911& 2734& 1367& 4102& 2051\\
6154& 3077& 9232& 4616& 2308& 1154& 577& 1732& 866& 433\\
1300& 650& 325& 976& 488& 244& 122& 61& 184& 92\\
46& 23& 70& 35& 106& 53& 160& 80& 40& 20\\
10& 5& 16& 8& 4& 2& 1& \\

2293&&&&&&&&&\\
6880& 3440& 1720& 860& 430& 215& 646& 323& 970& 485\\
1456& 728& 364& 182& 91& 274& 137& 412& 206& 103\\
310& 155& 466& 233& 700& 350& 175& 526& 263& 790\\
395& 1186& 593& 1780& 890& 445& 1336& 668& 334& 167\\
502& 251& 754& 377& 1132& 566& 283& 850& 425& 1276\\
638& 319& 958& 479& 1438& 719& 2158& 1079& 3238& 1619\\
4858& 2429& 7288& 3644& 1822& 911& 2734& 1367& 4102& 2051\\
6154& 3077& 9232& 4616& 2308& 1154& 577& 1732& 866& 433\\
1300& 650& 325& 976& 488& 244& 122& 61& 184& 92\\
46& 23& 70& 35& 106& 53& 160& 80& 40& 20\\
10& 5& 16& 8& 4& 2& 1& \\

2294&&&&&&&&&\\
1147& 3442& 1721& 5164& 2582& 1291& 3874& 1937& 5812& 2906\\
1453& 4360& 2180& 1090& 545& 1636& 818& 409& 1228& 614\\
307& 922& 461& 1384& 692& 346& 173& 520& 260& 130\\
65& 196& 98& 49& 148& 74& 37& 112& 56& 28\\
14& 7& 22& 11& 34& 17& 52& 26& 13& 40\\
20& 10& 5& 16& 8& 4& 2& 1& \\

2295&&&&&&&&&\\
6886& 3443& 10330& 5165& 15496& 7748& 3874& 1937& 5812& 2906\\
1453& 4360& 2180& 1090& 545& 1636& 818& 409& 1228& 614\\
307& 922& 461& 1384& 692& 346& 173& 520& 260& 130\\
65& 196& 98& 49& 148& 74& 37& 112& 56& 28\\
14& 7& 22& 11& 34& 17& 52& 26& 13& 40\\
20& 10& 5& 16& 8& 4& 2& 1& \\

2296&&&&&&&&&\\
1148& 574& 287& 862& 431& 1294& 647& 1942& 971& 2914\\
1457& 4372& 2186& 1093& 3280& 1640& 820& 410& 205& 616\\
308& 154& 77& 232& 116& 58& 29& 88& 44& 22\\
11& 34& 17& 52& 26& 13& 40& 20& 10& 5\\
16& 8& 4& 2& 1& \\

2297&&&&&&&&&\\
6892& 3446& 1723& 5170& 2585& 7756& 3878& 1939& 5818& 2909\\
8728& 4364& 2182& 1091& 3274& 1637& 4912& 2456& 1228& 614\\
307& 922& 461& 1384& 692& 346& 173& 520& 260& 130\\
65& 196& 98& 49& 148& 74& 37& 112& 56& 28\\
14& 7& 22& 11& 34& 17& 52& 26& 13& 40\\
20& 10& 5& 16& 8& 4& 2& 1& \\

2298&&&&&&&&&\\
1149& 3448& 1724& 862& 431& 1294& 647& 1942& 971& 2914\\
1457& 4372& 2186& 1093& 3280& 1640& 820& 410& 205& 616\\
308& 154& 77& 232& 116& 58& 29& 88& 44& 22\\
11& 34& 17& 52& 26& 13& 40& 20& 10& 5\\
16& 8& 4& 2& 1& \\

2299&&&&&&&&&\\
6898& 3449& 10348& 5174& 2587& 7762& 3881& 11644& 5822& 2911\\
8734& 4367& 13102& 6551& 19654& 9827& 29482& 14741& 44224& 22112\\
11056& 5528& 2764& 1382& 691& 2074& 1037& 3112& 1556& 778\\
389& 1168& 584& 292& 146& 73& 220& 110& 55& 166\\
83& 250& 125& 376& 188& 94& 47& 142& 71& 214\\
107& 322& 161& 484& 242& 121& 364& 182& 91& 274\\
137& 412& 206& 103& 310& 155& 466& 233& 700& 350\\
175& 526& 263& 790& 395& 1186& 593& 1780& 890& 445\\
1336& 668& 334& 167& 502& 251& 754& 377& 1132& 566\\
283& 850& 425& 1276& 638& 319& 958& 479& 1438& 719\\
2158& 1079& 3238& 1619& 4858& 2429& 7288& 3644& 1822& 911\\
2734& 1367& 4102& 2051& 6154& 3077& 9232& 4616& 2308& 1154\\
577& 1732& 866& 433& 1300& 650& 325& 976& 488& 244\\
122& 61& 184& 92& 46& 23& 70& 35& 106& 53\\
160& 80& 40& 20& 10& 5& 16& 8& 4& 2\\
1& \\

2300&&&&&&&&&\\
1150& 575& 1726& 863& 2590& 1295& 3886& 1943& 5830& 2915\\
8746& 4373& 13120& 6560& 3280& 1640& 820& 410& 205& 616\\
308& 154& 77& 232& 116& 58& 29& 88& 44& 22\\
11& 34& 17& 52& 26& 13& 40& 20& 10& 5\\
16& 8& 4& 2& 1& \\

2301&&&&&&&&&\\
6904& 3452& 1726& 863& 2590& 1295& 3886& 1943& 5830& 2915\\
8746& 4373& 13120& 6560& 3280& 1640& 820& 410& 205& 616\\
308& 154& 77& 232& 116& 58& 29& 88& 44& 22\\
11& 34& 17& 52& 26& 13& 40& 20& 10& 5\\
16& 8& 4& 2& 1& \\

2302&&&&&&&&&\\
1151& 3454& 1727& 5182& 2591& 7774& 3887& 11662& 5831& 17494\\
8747& 26242& 13121& 39364& 19682& 9841& 29524& 14762& 7381& 22144\\
11072& 5536& 2768& 1384& 692& 346& 173& 520& 260& 130\\
65& 196& 98& 49& 148& 74& 37& 112& 56& 28\\
14& 7& 22& 11& 34& 17& 52& 26& 13& 40\\
20& 10& 5& 16& 8& 4& 2& 1& \\

2303&&&&&&&&&\\
6910& 3455& 10366& 5183& 15550& 7775& 23326& 11663& 34990& 17495\\
52486& 26243& 78730& 39365& 118096& 59048& 29524& 14762& 7381& 22144\\
11072& 5536& 2768& 1384& 692& 346& 173& 520& 260& 130\\
65& 196& 98& 49& 148& 74& 37& 112& 56& 28\\
14& 7& 22& 11& 34& 17& 52& 26& 13& 40\\
20& 10& 5& 16& 8& 4& 2& 1& \\

2304&&&&&&&&&\\
1152& 576& 288& 144& 72& 36& 18& 9& 28& 14\\
7& 22& 11& 34& 17& 52& 26& 13& 40& 20\\
10& 5& 16& 8& 4& 2& 1& \\

2305&&&&&&&&&\\
6916& 3458& 1729& 5188& 2594& 1297& 3892& 1946& 973& 2920\\
1460& 730& 365& 1096& 548& 274& 137& 412& 206& 103\\
310& 155& 466& 233& 700& 350& 175& 526& 263& 790\\
395& 1186& 593& 1780& 890& 445& 1336& 668& 334& 167\\
502& 251& 754& 377& 1132& 566& 283& 850& 425& 1276\\
638& 319& 958& 479& 1438& 719& 2158& 1079& 3238& 1619\\
4858& 2429& 7288& 3644& 1822& 911& 2734& 1367& 4102& 2051\\
6154& 3077& 9232& 4616& 2308& 1154& 577& 1732& 866& 433\\
1300& 650& 325& 976& 488& 244& 122& 61& 184& 92\\
46& 23& 70& 35& 106& 53& 160& 80& 40& 20\\
10& 5& 16& 8& 4& 2& 1& \\

2306&&&&&&&&&\\
1153& 3460& 1730& 865& 2596& 1298& 649& 1948& 974& 487\\
1462& 731& 2194& 1097& 3292& 1646& 823& 2470& 1235& 3706\\
1853& 5560& 2780& 1390& 695& 2086& 1043& 3130& 1565& 4696\\
2348& 1174& 587& 1762& 881& 2644& 1322& 661& 1984& 992\\
496& 248& 124& 62& 31& 94& 47& 142& 71& 214\\
107& 322& 161& 484& 242& 121& 364& 182& 91& 274\\
137& 412& 206& 103& 310& 155& 466& 233& 700& 350\\
175& 526& 263& 790& 395& 1186& 593& 1780& 890& 445\\
1336& 668& 334& 167& 502& 251& 754& 377& 1132& 566\\
283& 850& 425& 1276& 638& 319& 958& 479& 1438& 719\\
2158& 1079& 3238& 1619& 4858& 2429& 7288& 3644& 1822& 911\\
2734& 1367& 4102& 2051& 6154& 3077& 9232& 4616& 2308& 1154\\
577& 1732& 866& 433& 1300& 650& 325& 976& 488& 244\\
122& 61& 184& 92& 46& 23& 70& 35& 106& 53\\
160& 80& 40& 20& 10& 5& 16& 8& 4& 2\\
1& \\

2307&&&&&&&&&\\
6922& 3461& 10384& 5192& 2596& 1298& 649& 1948& 974& 487\\
1462& 731& 2194& 1097& 3292& 1646& 823& 2470& 1235& 3706\\
1853& 5560& 2780& 1390& 695& 2086& 1043& 3130& 1565& 4696\\
2348& 1174& 587& 1762& 881& 2644& 1322& 661& 1984& 992\\
496& 248& 124& 62& 31& 94& 47& 142& 71& 214\\
107& 322& 161& 484& 242& 121& 364& 182& 91& 274\\
137& 412& 206& 103& 310& 155& 466& 233& 700& 350\\
175& 526& 263& 790& 395& 1186& 593& 1780& 890& 445\\
1336& 668& 334& 167& 502& 251& 754& 377& 1132& 566\\
283& 850& 425& 1276& 638& 319& 958& 479& 1438& 719\\
2158& 1079& 3238& 1619& 4858& 2429& 7288& 3644& 1822& 911\\
2734& 1367& 4102& 2051& 6154& 3077& 9232& 4616& 2308& 1154\\
577& 1732& 866& 433& 1300& 650& 325& 976& 488& 244\\
122& 61& 184& 92& 46& 23& 70& 35& 106& 53\\
160& 80& 40& 20& 10& 5& 16& 8& 4& 2\\
1& \\

2308&&&&&&&&&\\
1154& 577& 1732& 866& 433& 1300& 650& 325& 976& 488\\
244& 122& 61& 184& 92& 46& 23& 70& 35& 106\\
53& 160& 80& 40& 20& 10& 5& 16& 8& 4\\
2& 1& \\

2309&&&&&&&&&\\
6928& 3464& 1732& 866& 433& 1300& 650& 325& 976& 488\\
244& 122& 61& 184& 92& 46& 23& 70& 35& 106\\
53& 160& 80& 40& 20& 10& 5& 16& 8& 4\\
2& 1& \\

2310&&&&&&&&&\\
1155& 3466& 1733& 5200& 2600& 1300& 650& 325& 976& 488\\
244& 122& 61& 184& 92& 46& 23& 70& 35& 106\\
53& 160& 80& 40& 20& 10& 5& 16& 8& 4\\
2& 1& \\

2311&&&&&&&&&\\
6934& 3467& 10402& 5201& 15604& 7802& 3901& 11704& 5852& 2926\\
1463& 4390& 2195& 6586& 3293& 9880& 4940& 2470& 1235& 3706\\
1853& 5560& 2780& 1390& 695& 2086& 1043& 3130& 1565& 4696\\
2348& 1174& 587& 1762& 881& 2644& 1322& 661& 1984& 992\\
496& 248& 124& 62& 31& 94& 47& 142& 71& 214\\
107& 322& 161& 484& 242& 121& 364& 182& 91& 274\\
137& 412& 206& 103& 310& 155& 466& 233& 700& 350\\
175& 526& 263& 790& 395& 1186& 593& 1780& 890& 445\\
1336& 668& 334& 167& 502& 251& 754& 377& 1132& 566\\
283& 850& 425& 1276& 638& 319& 958& 479& 1438& 719\\
2158& 1079& 3238& 1619& 4858& 2429& 7288& 3644& 1822& 911\\
2734& 1367& 4102& 2051& 6154& 3077& 9232& 4616& 2308& 1154\\
577& 1732& 866& 433& 1300& 650& 325& 976& 488& 244\\
122& 61& 184& 92& 46& 23& 70& 35& 106& 53\\
160& 80& 40& 20& 10& 5& 16& 8& 4& 2\\
1& \\

2312&&&&&&&&&\\
1156& 578& 289& 868& 434& 217& 652& 326& 163& 490\\
245& 736& 368& 184& 92& 46& 23& 70& 35& 106\\
53& 160& 80& 40& 20& 10& 5& 16& 8& 4\\
2& 1& \\

2313&&&&&&&&&\\
6940& 3470& 1735& 5206& 2603& 7810& 3905& 11716& 5858& 2929\\
8788& 4394& 2197& 6592& 3296& 1648& 824& 412& 206& 103\\
310& 155& 466& 233& 700& 350& 175& 526& 263& 790\\
395& 1186& 593& 1780& 890& 445& 1336& 668& 334& 167\\
502& 251& 754& 377& 1132& 566& 283& 850& 425& 1276\\
638& 319& 958& 479& 1438& 719& 2158& 1079& 3238& 1619\\
4858& 2429& 7288& 3644& 1822& 911& 2734& 1367& 4102& 2051\\
6154& 3077& 9232& 4616& 2308& 1154& 577& 1732& 866& 433\\
1300& 650& 325& 976& 488& 244& 122& 61& 184& 92\\
46& 23& 70& 35& 106& 53& 160& 80& 40& 20\\
10& 5& 16& 8& 4& 2& 1& \\

2314&&&&&&&&&\\
1157& 3472& 1736& 868& 434& 217& 652& 326& 163& 490\\
245& 736& 368& 184& 92& 46& 23& 70& 35& 106\\
53& 160& 80& 40& 20& 10& 5& 16& 8& 4\\
2& 1& \\

2315&&&&&&&&&\\
6946& 3473& 10420& 5210& 2605& 7816& 3908& 1954& 977& 2932\\
1466& 733& 2200& 1100& 550& 275& 826& 413& 1240& 620\\
310& 155& 466& 233& 700& 350& 175& 526& 263& 790\\
395& 1186& 593& 1780& 890& 445& 1336& 668& 334& 167\\
502& 251& 754& 377& 1132& 566& 283& 850& 425& 1276\\
638& 319& 958& 479& 1438& 719& 2158& 1079& 3238& 1619\\
4858& 2429& 7288& 3644& 1822& 911& 2734& 1367& 4102& 2051\\
6154& 3077& 9232& 4616& 2308& 1154& 577& 1732& 866& 433\\
1300& 650& 325& 976& 488& 244& 122& 61& 184& 92\\
46& 23& 70& 35& 106& 53& 160& 80& 40& 20\\
10& 5& 16& 8& 4& 2& 1& \\

2316&&&&&&&&&\\
1158& 579& 1738& 869& 2608& 1304& 652& 326& 163& 490\\
245& 736& 368& 184& 92& 46& 23& 70& 35& 106\\
53& 160& 80& 40& 20& 10& 5& 16& 8& 4\\
2& 1& \\

2317&&&&&&&&&\\
6952& 3476& 1738& 869& 2608& 1304& 652& 326& 163& 490\\
245& 736& 368& 184& 92& 46& 23& 70& 35& 106\\
53& 160& 80& 40& 20& 10& 5& 16& 8& 4\\
2& 1& \\

2318&&&&&&&&&\\
1159& 3478& 1739& 5218& 2609& 7828& 3914& 1957& 5872& 2936\\
1468& 734& 367& 1102& 551& 1654& 827& 2482& 1241& 3724\\
1862& 931& 2794& 1397& 4192& 2096& 1048& 524& 262& 131\\
394& 197& 592& 296& 148& 74& 37& 112& 56& 28\\
14& 7& 22& 11& 34& 17& 52& 26& 13& 40\\
20& 10& 5& 16& 8& 4& 2& 1& \\

2319&&&&&&&&&\\
6958& 3479& 10438& 5219& 15658& 7829& 23488& 11744& 5872& 2936\\
1468& 734& 367& 1102& 551& 1654& 827& 2482& 1241& 3724\\
1862& 931& 2794& 1397& 4192& 2096& 1048& 524& 262& 131\\
394& 197& 592& 296& 148& 74& 37& 112& 56& 28\\
14& 7& 22& 11& 34& 17& 52& 26& 13& 40\\
20& 10& 5& 16& 8& 4& 2& 1& \\

2320&&&&&&&&&\\
1160& 580& 290& 145& 436& 218& 109& 328& 164& 82\\
41& 124& 62& 31& 94& 47& 142& 71& 214& 107\\
322& 161& 484& 242& 121& 364& 182& 91& 274& 137\\
412& 206& 103& 310& 155& 466& 233& 700& 350& 175\\
526& 263& 790& 395& 1186& 593& 1780& 890& 445& 1336\\
668& 334& 167& 502& 251& 754& 377& 1132& 566& 283\\
850& 425& 1276& 638& 319& 958& 479& 1438& 719& 2158\\
1079& 3238& 1619& 4858& 2429& 7288& 3644& 1822& 911& 2734\\
1367& 4102& 2051& 6154& 3077& 9232& 4616& 2308& 1154& 577\\
1732& 866& 433& 1300& 650& 325& 976& 488& 244& 122\\
61& 184& 92& 46& 23& 70& 35& 106& 53& 160\\
80& 40& 20& 10& 5& 16& 8& 4& 2& 1\\

2321&&&&&&&&&\\
6964& 3482& 1741& 5224& 2612& 1306& 653& 1960& 980& 490\\
245& 736& 368& 184& 92& 46& 23& 70& 35& 106\\
53& 160& 80& 40& 20& 10& 5& 16& 8& 4\\
2& 1& \\

2322&&&&&&&&&\\
1161& 3484& 1742& 871& 2614& 1307& 3922& 1961& 5884& 2942\\
1471& 4414& 2207& 6622& 3311& 9934& 4967& 14902& 7451& 22354\\
11177& 33532& 16766& 8383& 25150& 12575& 37726& 18863& 56590& 28295\\
84886& 42443& 127330& 63665& 190996& 95498& 47749& 143248& 71624& 35812\\
17906& 8953& 26860& 13430& 6715& 20146& 10073& 30220& 15110& 7555\\
22666& 11333& 34000& 17000& 8500& 4250& 2125& 6376& 3188& 1594\\
797& 2392& 1196& 598& 299& 898& 449& 1348& 674& 337\\
1012& 506& 253& 760& 380& 190& 95& 286& 143& 430\\
215& 646& 323& 970& 485& 1456& 728& 364& 182& 91\\
274& 137& 412& 206& 103& 310& 155& 466& 233& 700\\
350& 175& 526& 263& 790& 395& 1186& 593& 1780& 890\\
445& 1336& 668& 334& 167& 502& 251& 754& 377& 1132\\
566& 283& 850& 425& 1276& 638& 319& 958& 479& 1438\\
719& 2158& 1079& 3238& 1619& 4858& 2429& 7288& 3644& 1822\\
911& 2734& 1367& 4102& 2051& 6154& 3077& 9232& 4616& 2308\\
1154& 577& 1732& 866& 433& 1300& 650& 325& 976& 488\\
244& 122& 61& 184& 92& 46& 23& 70& 35& 106\\
53& 160& 80& 40& 20& 10& 5& 16& 8& 4\\
2& 1& \\

2323&&&&&&&&&\\
6970& 3485& 10456& 5228& 2614& 1307& 3922& 1961& 5884& 2942\\
1471& 4414& 2207& 6622& 3311& 9934& 4967& 14902& 7451& 22354\\
11177& 33532& 16766& 8383& 25150& 12575& 37726& 18863& 56590& 28295\\
84886& 42443& 127330& 63665& 190996& 95498& 47749& 143248& 71624& 35812\\
17906& 8953& 26860& 13430& 6715& 20146& 10073& 30220& 15110& 7555\\
22666& 11333& 34000& 17000& 8500& 4250& 2125& 6376& 3188& 1594\\
797& 2392& 1196& 598& 299& 898& 449& 1348& 674& 337\\
1012& 506& 253& 760& 380& 190& 95& 286& 143& 430\\
215& 646& 323& 970& 485& 1456& 728& 364& 182& 91\\
274& 137& 412& 206& 103& 310& 155& 466& 233& 700\\
350& 175& 526& 263& 790& 395& 1186& 593& 1780& 890\\
445& 1336& 668& 334& 167& 502& 251& 754& 377& 1132\\
566& 283& 850& 425& 1276& 638& 319& 958& 479& 1438\\
719& 2158& 1079& 3238& 1619& 4858& 2429& 7288& 3644& 1822\\
911& 2734& 1367& 4102& 2051& 6154& 3077& 9232& 4616& 2308\\
1154& 577& 1732& 866& 433& 1300& 650& 325& 976& 488\\
244& 122& 61& 184& 92& 46& 23& 70& 35& 106\\
53& 160& 80& 40& 20& 10& 5& 16& 8& 4\\
2& 1& \\

2324&&&&&&&&&\\
1162& 581& 1744& 872& 436& 218& 109& 328& 164& 82\\
41& 124& 62& 31& 94& 47& 142& 71& 214& 107\\
322& 161& 484& 242& 121& 364& 182& 91& 274& 137\\
412& 206& 103& 310& 155& 466& 233& 700& 350& 175\\
526& 263& 790& 395& 1186& 593& 1780& 890& 445& 1336\\
668& 334& 167& 502& 251& 754& 377& 1132& 566& 283\\
850& 425& 1276& 638& 319& 958& 479& 1438& 719& 2158\\
1079& 3238& 1619& 4858& 2429& 7288& 3644& 1822& 911& 2734\\
1367& 4102& 2051& 6154& 3077& 9232& 4616& 2308& 1154& 577\\
1732& 866& 433& 1300& 650& 325& 976& 488& 244& 122\\
61& 184& 92& 46& 23& 70& 35& 106& 53& 160\\
80& 40& 20& 10& 5& 16& 8& 4& 2& 1\\

2325&&&&&&&&&\\
6976& 3488& 1744& 872& 436& 218& 109& 328& 164& 82\\
41& 124& 62& 31& 94& 47& 142& 71& 214& 107\\
322& 161& 484& 242& 121& 364& 182& 91& 274& 137\\
412& 206& 103& 310& 155& 466& 233& 700& 350& 175\\
526& 263& 790& 395& 1186& 593& 1780& 890& 445& 1336\\
668& 334& 167& 502& 251& 754& 377& 1132& 566& 283\\
850& 425& 1276& 638& 319& 958& 479& 1438& 719& 2158\\
1079& 3238& 1619& 4858& 2429& 7288& 3644& 1822& 911& 2734\\
1367& 4102& 2051& 6154& 3077& 9232& 4616& 2308& 1154& 577\\
1732& 866& 433& 1300& 650& 325& 976& 488& 244& 122\\
61& 184& 92& 46& 23& 70& 35& 106& 53& 160\\
80& 40& 20& 10& 5& 16& 8& 4& 2& 1\\

2326&&&&&&&&&\\
1163& 3490& 1745& 5236& 2618& 1309& 3928& 1964& 982& 491\\
1474& 737& 2212& 1106& 553& 1660& 830& 415& 1246& 623\\
1870& 935& 2806& 1403& 4210& 2105& 6316& 3158& 1579& 4738\\
2369& 7108& 3554& 1777& 5332& 2666& 1333& 4000& 2000& 1000\\
500& 250& 125& 376& 188& 94& 47& 142& 71& 214\\
107& 322& 161& 484& 242& 121& 364& 182& 91& 274\\
137& 412& 206& 103& 310& 155& 466& 233& 700& 350\\
175& 526& 263& 790& 395& 1186& 593& 1780& 890& 445\\
1336& 668& 334& 167& 502& 251& 754& 377& 1132& 566\\
283& 850& 425& 1276& 638& 319& 958& 479& 1438& 719\\
2158& 1079& 3238& 1619& 4858& 2429& 7288& 3644& 1822& 911\\
2734& 1367& 4102& 2051& 6154& 3077& 9232& 4616& 2308& 1154\\
577& 1732& 866& 433& 1300& 650& 325& 976& 488& 244\\
122& 61& 184& 92& 46& 23& 70& 35& 106& 53\\
160& 80& 40& 20& 10& 5& 16& 8& 4& 2\\
1& \\

2327&&&&&&&&&\\
6982& 3491& 10474& 5237& 15712& 7856& 3928& 1964& 982& 491\\
1474& 737& 2212& 1106& 553& 1660& 830& 415& 1246& 623\\
1870& 935& 2806& 1403& 4210& 2105& 6316& 3158& 1579& 4738\\
2369& 7108& 3554& 1777& 5332& 2666& 1333& 4000& 2000& 1000\\
500& 250& 125& 376& 188& 94& 47& 142& 71& 214\\
107& 322& 161& 484& 242& 121& 364& 182& 91& 274\\
137& 412& 206& 103& 310& 155& 466& 233& 700& 350\\
175& 526& 263& 790& 395& 1186& 593& 1780& 890& 445\\
1336& 668& 334& 167& 502& 251& 754& 377& 1132& 566\\
283& 850& 425& 1276& 638& 319& 958& 479& 1438& 719\\
2158& 1079& 3238& 1619& 4858& 2429& 7288& 3644& 1822& 911\\
2734& 1367& 4102& 2051& 6154& 3077& 9232& 4616& 2308& 1154\\
577& 1732& 866& 433& 1300& 650& 325& 976& 488& 244\\
122& 61& 184& 92& 46& 23& 70& 35& 106& 53\\
160& 80& 40& 20& 10& 5& 16& 8& 4& 2\\
1& \\

2328&&&&&&&&&\\
1164& 582& 291& 874& 437& 1312& 656& 328& 164& 82\\
41& 124& 62& 31& 94& 47& 142& 71& 214& 107\\
322& 161& 484& 242& 121& 364& 182& 91& 274& 137\\
412& 206& 103& 310& 155& 466& 233& 700& 350& 175\\
526& 263& 790& 395& 1186& 593& 1780& 890& 445& 1336\\
668& 334& 167& 502& 251& 754& 377& 1132& 566& 283\\
850& 425& 1276& 638& 319& 958& 479& 1438& 719& 2158\\
1079& 3238& 1619& 4858& 2429& 7288& 3644& 1822& 911& 2734\\
1367& 4102& 2051& 6154& 3077& 9232& 4616& 2308& 1154& 577\\
1732& 866& 433& 1300& 650& 325& 976& 488& 244& 122\\
61& 184& 92& 46& 23& 70& 35& 106& 53& 160\\
80& 40& 20& 10& 5& 16& 8& 4& 2& 1\\

2329&&&&&&&&&\\
6988& 3494& 1747& 5242& 2621& 7864& 3932& 1966& 983& 2950\\
1475& 4426& 2213& 6640& 3320& 1660& 830& 415& 1246& 623\\
1870& 935& 2806& 1403& 4210& 2105& 6316& 3158& 1579& 4738\\
2369& 7108& 3554& 1777& 5332& 2666& 1333& 4000& 2000& 1000\\
500& 250& 125& 376& 188& 94& 47& 142& 71& 214\\
107& 322& 161& 484& 242& 121& 364& 182& 91& 274\\
137& 412& 206& 103& 310& 155& 466& 233& 700& 350\\
175& 526& 263& 790& 395& 1186& 593& 1780& 890& 445\\
1336& 668& 334& 167& 502& 251& 754& 377& 1132& 566\\
283& 850& 425& 1276& 638& 319& 958& 479& 1438& 719\\
2158& 1079& 3238& 1619& 4858& 2429& 7288& 3644& 1822& 911\\
2734& 1367& 4102& 2051& 6154& 3077& 9232& 4616& 2308& 1154\\
577& 1732& 866& 433& 1300& 650& 325& 976& 488& 244\\
122& 61& 184& 92& 46& 23& 70& 35& 106& 53\\
160& 80& 40& 20& 10& 5& 16& 8& 4& 2\\
1& \\

2330&&&&&&&&&\\
1165& 3496& 1748& 874& 437& 1312& 656& 328& 164& 82\\
41& 124& 62& 31& 94& 47& 142& 71& 214& 107\\
322& 161& 484& 242& 121& 364& 182& 91& 274& 137\\
412& 206& 103& 310& 155& 466& 233& 700& 350& 175\\
526& 263& 790& 395& 1186& 593& 1780& 890& 445& 1336\\
668& 334& 167& 502& 251& 754& 377& 1132& 566& 283\\
850& 425& 1276& 638& 319& 958& 479& 1438& 719& 2158\\
1079& 3238& 1619& 4858& 2429& 7288& 3644& 1822& 911& 2734\\
1367& 4102& 2051& 6154& 3077& 9232& 4616& 2308& 1154& 577\\
1732& 866& 433& 1300& 650& 325& 976& 488& 244& 122\\
61& 184& 92& 46& 23& 70& 35& 106& 53& 160\\
80& 40& 20& 10& 5& 16& 8& 4& 2& 1\\

2331&&&&&&&&&\\
6994& 3497& 10492& 5246& 2623& 7870& 3935& 11806& 5903& 17710\\
8855& 26566& 13283& 39850& 19925& 59776& 29888& 14944& 7472& 3736\\
1868& 934& 467& 1402& 701& 2104& 1052& 526& 263& 790\\
395& 1186& 593& 1780& 890& 445& 1336& 668& 334& 167\\
502& 251& 754& 377& 1132& 566& 283& 850& 425& 1276\\
638& 319& 958& 479& 1438& 719& 2158& 1079& 3238& 1619\\
4858& 2429& 7288& 3644& 1822& 911& 2734& 1367& 4102& 2051\\
6154& 3077& 9232& 4616& 2308& 1154& 577& 1732& 866& 433\\
1300& 650& 325& 976& 488& 244& 122& 61& 184& 92\\
46& 23& 70& 35& 106& 53& 160& 80& 40& 20\\
10& 5& 16& 8& 4& 2& 1& \\

2332&&&&&&&&&\\
1166& 583& 1750& 875& 2626& 1313& 3940& 1970& 985& 2956\\
1478& 739& 2218& 1109& 3328& 1664& 832& 416& 208& 104\\
52& 26& 13& 40& 20& 10& 5& 16& 8& 4\\
2& 1& \\

2333&&&&&&&&&\\
7000& 3500& 1750& 875& 2626& 1313& 3940& 1970& 985& 2956\\
1478& 739& 2218& 1109& 3328& 1664& 832& 416& 208& 104\\
52& 26& 13& 40& 20& 10& 5& 16& 8& 4\\
2& 1& \\

2334&&&&&&&&&\\
1167& 3502& 1751& 5254& 2627& 7882& 3941& 11824& 5912& 2956\\
1478& 739& 2218& 1109& 3328& 1664& 832& 416& 208& 104\\
52& 26& 13& 40& 20& 10& 5& 16& 8& 4\\
2& 1& \\

2335&&&&&&&&&\\
7006& 3503& 10510& 5255& 15766& 7883& 23650& 11825& 35476& 17738\\
8869& 26608& 13304& 6652& 3326& 1663& 4990& 2495& 7486& 3743\\
11230& 5615& 16846& 8423& 25270& 12635& 37906& 18953& 56860& 28430\\
14215& 42646& 21323& 63970& 31985& 95956& 47978& 23989& 71968& 35984\\
17992& 8996& 4498& 2249& 6748& 3374& 1687& 5062& 2531& 7594\\
3797& 11392& 5696& 2848& 1424& 712& 356& 178& 89& 268\\
134& 67& 202& 101& 304& 152& 76& 38& 19& 58\\
29& 88& 44& 22& 11& 34& 17& 52& 26& 13\\
40& 20& 10& 5& 16& 8& 4& 2& 1& \\

2336&&&&&&&&&\\
1168& 584& 292& 146& 73& 220& 110& 55& 166& 83\\
250& 125& 376& 188& 94& 47& 142& 71& 214& 107\\
322& 161& 484& 242& 121& 364& 182& 91& 274& 137\\
412& 206& 103& 310& 155& 466& 233& 700& 350& 175\\
526& 263& 790& 395& 1186& 593& 1780& 890& 445& 1336\\
668& 334& 167& 502& 251& 754& 377& 1132& 566& 283\\
850& 425& 1276& 638& 319& 958& 479& 1438& 719& 2158\\
1079& 3238& 1619& 4858& 2429& 7288& 3644& 1822& 911& 2734\\
1367& 4102& 2051& 6154& 3077& 9232& 4616& 2308& 1154& 577\\
1732& 866& 433& 1300& 650& 325& 976& 488& 244& 122\\
61& 184& 92& 46& 23& 70& 35& 106& 53& 160\\
80& 40& 20& 10& 5& 16& 8& 4& 2& 1\\

2337&&&&&&&&&\\
7012& 3506& 1753& 5260& 2630& 1315& 3946& 1973& 5920& 2960\\
1480& 740& 370& 185& 556& 278& 139& 418& 209& 628\\
314& 157& 472& 236& 118& 59& 178& 89& 268& 134\\
67& 202& 101& 304& 152& 76& 38& 19& 58& 29\\
88& 44& 22& 11& 34& 17& 52& 26& 13& 40\\
20& 10& 5& 16& 8& 4& 2& 1& \\

2338&&&&&&&&&\\
1169& 3508& 1754& 877& 2632& 1316& 658& 329& 988& 494\\
247& 742& 371& 1114& 557& 1672& 836& 418& 209& 628\\
314& 157& 472& 236& 118& 59& 178& 89& 268& 134\\
67& 202& 101& 304& 152& 76& 38& 19& 58& 29\\
88& 44& 22& 11& 34& 17& 52& 26& 13& 40\\
20& 10& 5& 16& 8& 4& 2& 1& \\

2339&&&&&&&&&\\
7018& 3509& 10528& 5264& 2632& 1316& 658& 329& 988& 494\\
247& 742& 371& 1114& 557& 1672& 836& 418& 209& 628\\
314& 157& 472& 236& 118& 59& 178& 89& 268& 134\\
67& 202& 101& 304& 152& 76& 38& 19& 58& 29\\
88& 44& 22& 11& 34& 17& 52& 26& 13& 40\\
20& 10& 5& 16& 8& 4& 2& 1& \\

2340&&&&&&&&&\\
1170& 585& 1756& 878& 439& 1318& 659& 1978& 989& 2968\\
1484& 742& 371& 1114& 557& 1672& 836& 418& 209& 628\\
314& 157& 472& 236& 118& 59& 178& 89& 268& 134\\
67& 202& 101& 304& 152& 76& 38& 19& 58& 29\\
88& 44& 22& 11& 34& 17& 52& 26& 13& 40\\
20& 10& 5& 16& 8& 4& 2& 1& \\

2341&&&&&&&&&\\
7024& 3512& 1756& 878& 439& 1318& 659& 1978& 989& 2968\\
1484& 742& 371& 1114& 557& 1672& 836& 418& 209& 628\\
314& 157& 472& 236& 118& 59& 178& 89& 268& 134\\
67& 202& 101& 304& 152& 76& 38& 19& 58& 29\\
88& 44& 22& 11& 34& 17& 52& 26& 13& 40\\
20& 10& 5& 16& 8& 4& 2& 1& \\

2342&&&&&&&&&\\
1171& 3514& 1757& 5272& 2636& 1318& 659& 1978& 989& 2968\\
1484& 742& 371& 1114& 557& 1672& 836& 418& 209& 628\\
314& 157& 472& 236& 118& 59& 178& 89& 268& 134\\
67& 202& 101& 304& 152& 76& 38& 19& 58& 29\\
88& 44& 22& 11& 34& 17& 52& 26& 13& 40\\
20& 10& 5& 16& 8& 4& 2& 1& \\

2343&&&&&&&&&\\
7030& 3515& 10546& 5273& 15820& 7910& 3955& 11866& 5933& 17800\\
8900& 4450& 2225& 6676& 3338& 1669& 5008& 2504& 1252& 626\\
313& 940& 470& 235& 706& 353& 1060& 530& 265& 796\\
398& 199& 598& 299& 898& 449& 1348& 674& 337& 1012\\
506& 253& 760& 380& 190& 95& 286& 143& 430& 215\\
646& 323& 970& 485& 1456& 728& 364& 182& 91& 274\\
137& 412& 206& 103& 310& 155& 466& 233& 700& 350\\
175& 526& 263& 790& 395& 1186& 593& 1780& 890& 445\\
1336& 668& 334& 167& 502& 251& 754& 377& 1132& 566\\
283& 850& 425& 1276& 638& 319& 958& 479& 1438& 719\\
2158& 1079& 3238& 1619& 4858& 2429& 7288& 3644& 1822& 911\\
2734& 1367& 4102& 2051& 6154& 3077& 9232& 4616& 2308& 1154\\
577& 1732& 866& 433& 1300& 650& 325& 976& 488& 244\\
122& 61& 184& 92& 46& 23& 70& 35& 106& 53\\
160& 80& 40& 20& 10& 5& 16& 8& 4& 2\\
1& \\

2344&&&&&&&&&\\
1172& 586& 293& 880& 440& 220& 110& 55& 166& 83\\
250& 125& 376& 188& 94& 47& 142& 71& 214& 107\\
322& 161& 484& 242& 121& 364& 182& 91& 274& 137\\
412& 206& 103& 310& 155& 466& 233& 700& 350& 175\\
526& 263& 790& 395& 1186& 593& 1780& 890& 445& 1336\\
668& 334& 167& 502& 251& 754& 377& 1132& 566& 283\\
850& 425& 1276& 638& 319& 958& 479& 1438& 719& 2158\\
1079& 3238& 1619& 4858& 2429& 7288& 3644& 1822& 911& 2734\\
1367& 4102& 2051& 6154& 3077& 9232& 4616& 2308& 1154& 577\\
1732& 866& 433& 1300& 650& 325& 976& 488& 244& 122\\
61& 184& 92& 46& 23& 70& 35& 106& 53& 160\\
80& 40& 20& 10& 5& 16& 8& 4& 2& 1\\

2345&&&&&&&&&\\
7036& 3518& 1759& 5278& 2639& 7918& 3959& 11878& 5939& 17818\\
8909& 26728& 13364& 6682& 3341& 10024& 5012& 2506& 1253& 3760\\
1880& 940& 470& 235& 706& 353& 1060& 530& 265& 796\\
398& 199& 598& 299& 898& 449& 1348& 674& 337& 1012\\
506& 253& 760& 380& 190& 95& 286& 143& 430& 215\\
646& 323& 970& 485& 1456& 728& 364& 182& 91& 274\\
137& 412& 206& 103& 310& 155& 466& 233& 700& 350\\
175& 526& 263& 790& 395& 1186& 593& 1780& 890& 445\\
1336& 668& 334& 167& 502& 251& 754& 377& 1132& 566\\
283& 850& 425& 1276& 638& 319& 958& 479& 1438& 719\\
2158& 1079& 3238& 1619& 4858& 2429& 7288& 3644& 1822& 911\\
2734& 1367& 4102& 2051& 6154& 3077& 9232& 4616& 2308& 1154\\
577& 1732& 866& 433& 1300& 650& 325& 976& 488& 244\\
122& 61& 184& 92& 46& 23& 70& 35& 106& 53\\
160& 80& 40& 20& 10& 5& 16& 8& 4& 2\\
1& \\

2346&&&&&&&&&\\
1173& 3520& 1760& 880& 440& 220& 110& 55& 166& 83\\
250& 125& 376& 188& 94& 47& 142& 71& 214& 107\\
322& 161& 484& 242& 121& 364& 182& 91& 274& 137\\
412& 206& 103& 310& 155& 466& 233& 700& 350& 175\\
526& 263& 790& 395& 1186& 593& 1780& 890& 445& 1336\\
668& 334& 167& 502& 251& 754& 377& 1132& 566& 283\\
850& 425& 1276& 638& 319& 958& 479& 1438& 719& 2158\\
1079& 3238& 1619& 4858& 2429& 7288& 3644& 1822& 911& 2734\\
1367& 4102& 2051& 6154& 3077& 9232& 4616& 2308& 1154& 577\\
1732& 866& 433& 1300& 650& 325& 976& 488& 244& 122\\
61& 184& 92& 46& 23& 70& 35& 106& 53& 160\\
80& 40& 20& 10& 5& 16& 8& 4& 2& 1\\

2347&&&&&&&&&\\
7042& 3521& 10564& 5282& 2641& 7924& 3962& 1981& 5944& 2972\\
1486& 743& 2230& 1115& 3346& 1673& 5020& 2510& 1255& 3766\\
1883& 5650& 2825& 8476& 4238& 2119& 6358& 3179& 9538& 4769\\
14308& 7154& 3577& 10732& 5366& 2683& 8050& 4025& 12076& 6038\\
3019& 9058& 4529& 13588& 6794& 3397& 10192& 5096& 2548& 1274\\
637& 1912& 956& 478& 239& 718& 359& 1078& 539& 1618\\
809& 2428& 1214& 607& 1822& 911& 2734& 1367& 4102& 2051\\
6154& 3077& 9232& 4616& 2308& 1154& 577& 1732& 866& 433\\
1300& 650& 325& 976& 488& 244& 122& 61& 184& 92\\
46& 23& 70& 35& 106& 53& 160& 80& 40& 20\\
10& 5& 16& 8& 4& 2& 1& \\

2348&&&&&&&&&\\
1174& 587& 1762& 881& 2644& 1322& 661& 1984& 992& 496\\
248& 124& 62& 31& 94& 47& 142& 71& 214& 107\\
322& 161& 484& 242& 121& 364& 182& 91& 274& 137\\
412& 206& 103& 310& 155& 466& 233& 700& 350& 175\\
526& 263& 790& 395& 1186& 593& 1780& 890& 445& 1336\\
668& 334& 167& 502& 251& 754& 377& 1132& 566& 283\\
850& 425& 1276& 638& 319& 958& 479& 1438& 719& 2158\\
1079& 3238& 1619& 4858& 2429& 7288& 3644& 1822& 911& 2734\\
1367& 4102& 2051& 6154& 3077& 9232& 4616& 2308& 1154& 577\\
1732& 866& 433& 1300& 650& 325& 976& 488& 244& 122\\
61& 184& 92& 46& 23& 70& 35& 106& 53& 160\\
80& 40& 20& 10& 5& 16& 8& 4& 2& 1\\

2349&&&&&&&&&\\
7048& 3524& 1762& 881& 2644& 1322& 661& 1984& 992& 496\\
248& 124& 62& 31& 94& 47& 142& 71& 214& 107\\
322& 161& 484& 242& 121& 364& 182& 91& 274& 137\\
412& 206& 103& 310& 155& 466& 233& 700& 350& 175\\
526& 263& 790& 395& 1186& 593& 1780& 890& 445& 1336\\
668& 334& 167& 502& 251& 754& 377& 1132& 566& 283\\
850& 425& 1276& 638& 319& 958& 479& 1438& 719& 2158\\
1079& 3238& 1619& 4858& 2429& 7288& 3644& 1822& 911& 2734\\
1367& 4102& 2051& 6154& 3077& 9232& 4616& 2308& 1154& 577\\
1732& 866& 433& 1300& 650& 325& 976& 488& 244& 122\\
61& 184& 92& 46& 23& 70& 35& 106& 53& 160\\
80& 40& 20& 10& 5& 16& 8& 4& 2& 1\\

2350&&&&&&&&&\\
1175& 3526& 1763& 5290& 2645& 7936& 3968& 1984& 992& 496\\
248& 124& 62& 31& 94& 47& 142& 71& 214& 107\\
322& 161& 484& 242& 121& 364& 182& 91& 274& 137\\
412& 206& 103& 310& 155& 466& 233& 700& 350& 175\\
526& 263& 790& 395& 1186& 593& 1780& 890& 445& 1336\\
668& 334& 167& 502& 251& 754& 377& 1132& 566& 283\\
850& 425& 1276& 638& 319& 958& 479& 1438& 719& 2158\\
1079& 3238& 1619& 4858& 2429& 7288& 3644& 1822& 911& 2734\\
1367& 4102& 2051& 6154& 3077& 9232& 4616& 2308& 1154& 577\\
1732& 866& 433& 1300& 650& 325& 976& 488& 244& 122\\
61& 184& 92& 46& 23& 70& 35& 106& 53& 160\\
80& 40& 20& 10& 5& 16& 8& 4& 2& 1\\

2351&&&&&&&&&\\
7054& 3527& 10582& 5291& 15874& 7937& 23812& 11906& 5953& 17860\\
8930& 4465& 13396& 6698& 3349& 10048& 5024& 2512& 1256& 628\\
314& 157& 472& 236& 118& 59& 178& 89& 268& 134\\
67& 202& 101& 304& 152& 76& 38& 19& 58& 29\\
88& 44& 22& 11& 34& 17& 52& 26& 13& 40\\
20& 10& 5& 16& 8& 4& 2& 1& \\

2352&&&&&&&&&\\
1176& 588& 294& 147& 442& 221& 664& 332& 166& 83\\
250& 125& 376& 188& 94& 47& 142& 71& 214& 107\\
322& 161& 484& 242& 121& 364& 182& 91& 274& 137\\
412& 206& 103& 310& 155& 466& 233& 700& 350& 175\\
526& 263& 790& 395& 1186& 593& 1780& 890& 445& 1336\\
668& 334& 167& 502& 251& 754& 377& 1132& 566& 283\\
850& 425& 1276& 638& 319& 958& 479& 1438& 719& 2158\\
1079& 3238& 1619& 4858& 2429& 7288& 3644& 1822& 911& 2734\\
1367& 4102& 2051& 6154& 3077& 9232& 4616& 2308& 1154& 577\\
1732& 866& 433& 1300& 650& 325& 976& 488& 244& 122\\
61& 184& 92& 46& 23& 70& 35& 106& 53& 160\\
80& 40& 20& 10& 5& 16& 8& 4& 2& 1\\

2353&&&&&&&&&\\
7060& 3530& 1765& 5296& 2648& 1324& 662& 331& 994& 497\\
1492& 746& 373& 1120& 560& 280& 140& 70& 35& 106\\
53& 160& 80& 40& 20& 10& 5& 16& 8& 4\\
2& 1& \\

2354&&&&&&&&&\\
1177& 3532& 1766& 883& 2650& 1325& 3976& 1988& 994& 497\\
1492& 746& 373& 1120& 560& 280& 140& 70& 35& 106\\
53& 160& 80& 40& 20& 10& 5& 16& 8& 4\\
2& 1& \\

2355&&&&&&&&&\\
7066& 3533& 10600& 5300& 2650& 1325& 3976& 1988& 994& 497\\
1492& 746& 373& 1120& 560& 280& 140& 70& 35& 106\\
53& 160& 80& 40& 20& 10& 5& 16& 8& 4\\
2& 1& \\

2356&&&&&&&&&\\
1178& 589& 1768& 884& 442& 221& 664& 332& 166& 83\\
250& 125& 376& 188& 94& 47& 142& 71& 214& 107\\
322& 161& 484& 242& 121& 364& 182& 91& 274& 137\\
412& 206& 103& 310& 155& 466& 233& 700& 350& 175\\
526& 263& 790& 395& 1186& 593& 1780& 890& 445& 1336\\
668& 334& 167& 502& 251& 754& 377& 1132& 566& 283\\
850& 425& 1276& 638& 319& 958& 479& 1438& 719& 2158\\
1079& 3238& 1619& 4858& 2429& 7288& 3644& 1822& 911& 2734\\
1367& 4102& 2051& 6154& 3077& 9232& 4616& 2308& 1154& 577\\
1732& 866& 433& 1300& 650& 325& 976& 488& 244& 122\\
61& 184& 92& 46& 23& 70& 35& 106& 53& 160\\
80& 40& 20& 10& 5& 16& 8& 4& 2& 1\\

2357&&&&&&&&&\\
7072& 3536& 1768& 884& 442& 221& 664& 332& 166& 83\\
250& 125& 376& 188& 94& 47& 142& 71& 214& 107\\
322& 161& 484& 242& 121& 364& 182& 91& 274& 137\\
412& 206& 103& 310& 155& 466& 233& 700& 350& 175\\
526& 263& 790& 395& 1186& 593& 1780& 890& 445& 1336\\
668& 334& 167& 502& 251& 754& 377& 1132& 566& 283\\
850& 425& 1276& 638& 319& 958& 479& 1438& 719& 2158\\
1079& 3238& 1619& 4858& 2429& 7288& 3644& 1822& 911& 2734\\
1367& 4102& 2051& 6154& 3077& 9232& 4616& 2308& 1154& 577\\
1732& 866& 433& 1300& 650& 325& 976& 488& 244& 122\\
61& 184& 92& 46& 23& 70& 35& 106& 53& 160\\
80& 40& 20& 10& 5& 16& 8& 4& 2& 1\\

2358&&&&&&&&&\\
1179& 3538& 1769& 5308& 2654& 1327& 3982& 1991& 5974& 2987\\
8962& 4481& 13444& 6722& 3361& 10084& 5042& 2521& 7564& 3782\\
1891& 5674& 2837& 8512& 4256& 2128& 1064& 532& 266& 133\\
400& 200& 100& 50& 25& 76& 38& 19& 58& 29\\
88& 44& 22& 11& 34& 17& 52& 26& 13& 40\\
20& 10& 5& 16& 8& 4& 2& 1& \\

2359&&&&&&&&&\\
7078& 3539& 10618& 5309& 15928& 7964& 3982& 1991& 5974& 2987\\
8962& 4481& 13444& 6722& 3361& 10084& 5042& 2521& 7564& 3782\\
1891& 5674& 2837& 8512& 4256& 2128& 1064& 532& 266& 133\\
400& 200& 100& 50& 25& 76& 38& 19& 58& 29\\
88& 44& 22& 11& 34& 17& 52& 26& 13& 40\\
20& 10& 5& 16& 8& 4& 2& 1& \\

2360&&&&&&&&&\\
1180& 590& 295& 886& 443& 1330& 665& 1996& 998& 499\\
1498& 749& 2248& 1124& 562& 281& 844& 422& 211& 634\\
317& 952& 476& 238& 119& 358& 179& 538& 269& 808\\
404& 202& 101& 304& 152& 76& 38& 19& 58& 29\\
88& 44& 22& 11& 34& 17& 52& 26& 13& 40\\
20& 10& 5& 16& 8& 4& 2& 1& \\

2361&&&&&&&&&\\
7084& 3542& 1771& 5314& 2657& 7972& 3986& 1993& 5980& 2990\\
1495& 4486& 2243& 6730& 3365& 10096& 5048& 2524& 1262& 631\\
1894& 947& 2842& 1421& 4264& 2132& 1066& 533& 1600& 800\\
400& 200& 100& 50& 25& 76& 38& 19& 58& 29\\
88& 44& 22& 11& 34& 17& 52& 26& 13& 40\\
20& 10& 5& 16& 8& 4& 2& 1& \\

2362&&&&&&&&&\\
1181& 3544& 1772& 886& 443& 1330& 665& 1996& 998& 499\\
1498& 749& 2248& 1124& 562& 281& 844& 422& 211& 634\\
317& 952& 476& 238& 119& 358& 179& 538& 269& 808\\
404& 202& 101& 304& 152& 76& 38& 19& 58& 29\\
88& 44& 22& 11& 34& 17& 52& 26& 13& 40\\
20& 10& 5& 16& 8& 4& 2& 1& \\

2363&&&&&&&&&\\
7090& 3545& 10636& 5318& 2659& 7978& 3989& 11968& 5984& 2992\\
1496& 748& 374& 187& 562& 281& 844& 422& 211& 634\\
317& 952& 476& 238& 119& 358& 179& 538& 269& 808\\
404& 202& 101& 304& 152& 76& 38& 19& 58& 29\\
88& 44& 22& 11& 34& 17& 52& 26& 13& 40\\
20& 10& 5& 16& 8& 4& 2& 1& \\

2364&&&&&&&&&\\
1182& 591& 1774& 887& 2662& 1331& 3994& 1997& 5992& 2996\\
1498& 749& 2248& 1124& 562& 281& 844& 422& 211& 634\\
317& 952& 476& 238& 119& 358& 179& 538& 269& 808\\
404& 202& 101& 304& 152& 76& 38& 19& 58& 29\\
88& 44& 22& 11& 34& 17& 52& 26& 13& 40\\
20& 10& 5& 16& 8& 4& 2& 1& \\

2365&&&&&&&&&\\
7096& 3548& 1774& 887& 2662& 1331& 3994& 1997& 5992& 2996\\
1498& 749& 2248& 1124& 562& 281& 844& 422& 211& 634\\
317& 952& 476& 238& 119& 358& 179& 538& 269& 808\\
404& 202& 101& 304& 152& 76& 38& 19& 58& 29\\
88& 44& 22& 11& 34& 17& 52& 26& 13& 40\\
20& 10& 5& 16& 8& 4& 2& 1& \\

2366&&&&&&&&&\\
1183& 3550& 1775& 5326& 2663& 7990& 3995& 11986& 5993& 17980\\
8990& 4495& 13486& 6743& 20230& 10115& 30346& 15173& 45520& 22760\\
11380& 5690& 2845& 8536& 4268& 2134& 1067& 3202& 1601& 4804\\
2402& 1201& 3604& 1802& 901& 2704& 1352& 676& 338& 169\\
508& 254& 127& 382& 191& 574& 287& 862& 431& 1294\\
647& 1942& 971& 2914& 1457& 4372& 2186& 1093& 3280& 1640\\
820& 410& 205& 616& 308& 154& 77& 232& 116& 58\\
29& 88& 44& 22& 11& 34& 17& 52& 26& 13\\
40& 20& 10& 5& 16& 8& 4& 2& 1& \\

2367&&&&&&&&&\\
7102& 3551& 10654& 5327& 15982& 7991& 23974& 11987& 35962& 17981\\
53944& 26972& 13486& 6743& 20230& 10115& 30346& 15173& 45520& 22760\\
11380& 5690& 2845& 8536& 4268& 2134& 1067& 3202& 1601& 4804\\
2402& 1201& 3604& 1802& 901& 2704& 1352& 676& 338& 169\\
508& 254& 127& 382& 191& 574& 287& 862& 431& 1294\\
647& 1942& 971& 2914& 1457& 4372& 2186& 1093& 3280& 1640\\
820& 410& 205& 616& 308& 154& 77& 232& 116& 58\\
29& 88& 44& 22& 11& 34& 17& 52& 26& 13\\
40& 20& 10& 5& 16& 8& 4& 2& 1& \\

2368&&&&&&&&&\\
1184& 592& 296& 148& 74& 37& 112& 56& 28& 14\\
7& 22& 11& 34& 17& 52& 26& 13& 40& 20\\
10& 5& 16& 8& 4& 2& 1& \\

2369&&&&&&&&&\\
7108& 3554& 1777& 5332& 2666& 1333& 4000& 2000& 1000& 500\\
250& 125& 376& 188& 94& 47& 142& 71& 214& 107\\
322& 161& 484& 242& 121& 364& 182& 91& 274& 137\\
412& 206& 103& 310& 155& 466& 233& 700& 350& 175\\
526& 263& 790& 395& 1186& 593& 1780& 890& 445& 1336\\
668& 334& 167& 502& 251& 754& 377& 1132& 566& 283\\
850& 425& 1276& 638& 319& 958& 479& 1438& 719& 2158\\
1079& 3238& 1619& 4858& 2429& 7288& 3644& 1822& 911& 2734\\
1367& 4102& 2051& 6154& 3077& 9232& 4616& 2308& 1154& 577\\
1732& 866& 433& 1300& 650& 325& 976& 488& 244& 122\\
61& 184& 92& 46& 23& 70& 35& 106& 53& 160\\
80& 40& 20& 10& 5& 16& 8& 4& 2& 1\\

2370&&&&&&&&&\\
1185& 3556& 1778& 889& 2668& 1334& 667& 2002& 1001& 3004\\
1502& 751& 2254& 1127& 3382& 1691& 5074& 2537& 7612& 3806\\
1903& 5710& 2855& 8566& 4283& 12850& 6425& 19276& 9638& 4819\\
14458& 7229& 21688& 10844& 5422& 2711& 8134& 4067& 12202& 6101\\
18304& 9152& 4576& 2288& 1144& 572& 286& 143& 430& 215\\
646& 323& 970& 485& 1456& 728& 364& 182& 91& 274\\
137& 412& 206& 103& 310& 155& 466& 233& 700& 350\\
175& 526& 263& 790& 395& 1186& 593& 1780& 890& 445\\
1336& 668& 334& 167& 502& 251& 754& 377& 1132& 566\\
283& 850& 425& 1276& 638& 319& 958& 479& 1438& 719\\
2158& 1079& 3238& 1619& 4858& 2429& 7288& 3644& 1822& 911\\
2734& 1367& 4102& 2051& 6154& 3077& 9232& 4616& 2308& 1154\\
577& 1732& 866& 433& 1300& 650& 325& 976& 488& 244\\
122& 61& 184& 92& 46& 23& 70& 35& 106& 53\\
160& 80& 40& 20& 10& 5& 16& 8& 4& 2\\
1& \\

2371&&&&&&&&&\\
7114& 3557& 10672& 5336& 2668& 1334& 667& 2002& 1001& 3004\\
1502& 751& 2254& 1127& 3382& 1691& 5074& 2537& 7612& 3806\\
1903& 5710& 2855& 8566& 4283& 12850& 6425& 19276& 9638& 4819\\
14458& 7229& 21688& 10844& 5422& 2711& 8134& 4067& 12202& 6101\\
18304& 9152& 4576& 2288& 1144& 572& 286& 143& 430& 215\\
646& 323& 970& 485& 1456& 728& 364& 182& 91& 274\\
137& 412& 206& 103& 310& 155& 466& 233& 700& 350\\
175& 526& 263& 790& 395& 1186& 593& 1780& 890& 445\\
1336& 668& 334& 167& 502& 251& 754& 377& 1132& 566\\
283& 850& 425& 1276& 638& 319& 958& 479& 1438& 719\\
2158& 1079& 3238& 1619& 4858& 2429& 7288& 3644& 1822& 911\\
2734& 1367& 4102& 2051& 6154& 3077& 9232& 4616& 2308& 1154\\
577& 1732& 866& 433& 1300& 650& 325& 976& 488& 244\\
122& 61& 184& 92& 46& 23& 70& 35& 106& 53\\
160& 80& 40& 20& 10& 5& 16& 8& 4& 2\\
1& \\

2372&&&&&&&&&\\
1186& 593& 1780& 890& 445& 1336& 668& 334& 167& 502\\
251& 754& 377& 1132& 566& 283& 850& 425& 1276& 638\\
319& 958& 479& 1438& 719& 2158& 1079& 3238& 1619& 4858\\
2429& 7288& 3644& 1822& 911& 2734& 1367& 4102& 2051& 6154\\
3077& 9232& 4616& 2308& 1154& 577& 1732& 866& 433& 1300\\
650& 325& 976& 488& 244& 122& 61& 184& 92& 46\\
23& 70& 35& 106& 53& 160& 80& 40& 20& 10\\
5& 16& 8& 4& 2& 1& \\

2373&&&&&&&&&\\
7120& 3560& 1780& 890& 445& 1336& 668& 334& 167& 502\\
251& 754& 377& 1132& 566& 283& 850& 425& 1276& 638\\
319& 958& 479& 1438& 719& 2158& 1079& 3238& 1619& 4858\\
2429& 7288& 3644& 1822& 911& 2734& 1367& 4102& 2051& 6154\\
3077& 9232& 4616& 2308& 1154& 577& 1732& 866& 433& 1300\\
650& 325& 976& 488& 244& 122& 61& 184& 92& 46\\
23& 70& 35& 106& 53& 160& 80& 40& 20& 10\\
5& 16& 8& 4& 2& 1& \\

2374&&&&&&&&&\\
1187& 3562& 1781& 5344& 2672& 1336& 668& 334& 167& 502\\
251& 754& 377& 1132& 566& 283& 850& 425& 1276& 638\\
319& 958& 479& 1438& 719& 2158& 1079& 3238& 1619& 4858\\
2429& 7288& 3644& 1822& 911& 2734& 1367& 4102& 2051& 6154\\
3077& 9232& 4616& 2308& 1154& 577& 1732& 866& 433& 1300\\
650& 325& 976& 488& 244& 122& 61& 184& 92& 46\\
23& 70& 35& 106& 53& 160& 80& 40& 20& 10\\
5& 16& 8& 4& 2& 1& \\

2375&&&&&&&&&\\
7126& 3563& 10690& 5345& 16036& 8018& 4009& 12028& 6014& 3007\\
9022& 4511& 13534& 6767& 20302& 10151& 30454& 15227& 45682& 22841\\
68524& 34262& 17131& 51394& 25697& 77092& 38546& 19273& 57820& 28910\\
14455& 43366& 21683& 65050& 32525& 97576& 48788& 24394& 12197& 36592\\
18296& 9148& 4574& 2287& 6862& 3431& 10294& 5147& 15442& 7721\\
23164& 11582& 5791& 17374& 8687& 26062& 13031& 39094& 19547& 58642\\
29321& 87964& 43982& 21991& 65974& 32987& 98962& 49481& 148444& 74222\\
37111& 111334& 55667& 167002& 83501& 250504& 125252& 62626& 31313& 93940\\
46970& 23485& 70456& 35228& 17614& 8807& 26422& 13211& 39634& 19817\\
59452& 29726& 14863& 44590& 22295& 66886& 33443& 100330& 50165& 150496\\
75248& 37624& 18812& 9406& 4703& 14110& 7055& 21166& 10583& 31750\\
15875& 47626& 23813& 71440& 35720& 17860& 8930& 4465& 13396& 6698\\
3349& 10048& 5024& 2512& 1256& 628& 314& 157& 472& 236\\
118& 59& 178& 89& 268& 134& 67& 202& 101& 304\\
152& 76& 38& 19& 58& 29& 88& 44& 22& 11\\
34& 17& 52& 26& 13& 40& 20& 10& 5& 16\\
8& 4& 2& 1& \\

2376&&&&&&&&&\\
1188& 594& 297& 892& 446& 223& 670& 335& 1006& 503\\
1510& 755& 2266& 1133& 3400& 1700& 850& 425& 1276& 638\\
319& 958& 479& 1438& 719& 2158& 1079& 3238& 1619& 4858\\
2429& 7288& 3644& 1822& 911& 2734& 1367& 4102& 2051& 6154\\
3077& 9232& 4616& 2308& 1154& 577& 1732& 866& 433& 1300\\
650& 325& 976& 488& 244& 122& 61& 184& 92& 46\\
23& 70& 35& 106& 53& 160& 80& 40& 20& 10\\
5& 16& 8& 4& 2& 1& \\

2377&&&&&&&&&\\
7132& 3566& 1783& 5350& 2675& 8026& 4013& 12040& 6020& 3010\\
1505& 4516& 2258& 1129& 3388& 1694& 847& 2542& 1271& 3814\\
1907& 5722& 2861& 8584& 4292& 2146& 1073& 3220& 1610& 805\\
2416& 1208& 604& 302& 151& 454& 227& 682& 341& 1024\\
512& 256& 128& 64& 32& 16& 8& 4& 2& 1\\

2378&&&&&&&&&\\
1189& 3568& 1784& 892& 446& 223& 670& 335& 1006& 503\\
1510& 755& 2266& 1133& 3400& 1700& 850& 425& 1276& 638\\
319& 958& 479& 1438& 719& 2158& 1079& 3238& 1619& 4858\\
2429& 7288& 3644& 1822& 911& 2734& 1367& 4102& 2051& 6154\\
3077& 9232& 4616& 2308& 1154& 577& 1732& 866& 433& 1300\\
650& 325& 976& 488& 244& 122& 61& 184& 92& 46\\
23& 70& 35& 106& 53& 160& 80& 40& 20& 10\\
5& 16& 8& 4& 2& 1& \\

2379&&&&&&&&&\\
7138& 3569& 10708& 5354& 2677& 8032& 4016& 2008& 1004& 502\\
251& 754& 377& 1132& 566& 283& 850& 425& 1276& 638\\
319& 958& 479& 1438& 719& 2158& 1079& 3238& 1619& 4858\\
2429& 7288& 3644& 1822& 911& 2734& 1367& 4102& 2051& 6154\\
3077& 9232& 4616& 2308& 1154& 577& 1732& 866& 433& 1300\\
650& 325& 976& 488& 244& 122& 61& 184& 92& 46\\
23& 70& 35& 106& 53& 160& 80& 40& 20& 10\\
5& 16& 8& 4& 2& 1& \\

2380&&&&&&&&&\\
1190& 595& 1786& 893& 2680& 1340& 670& 335& 1006& 503\\
1510& 755& 2266& 1133& 3400& 1700& 850& 425& 1276& 638\\
319& 958& 479& 1438& 719& 2158& 1079& 3238& 1619& 4858\\
2429& 7288& 3644& 1822& 911& 2734& 1367& 4102& 2051& 6154\\
3077& 9232& 4616& 2308& 1154& 577& 1732& 866& 433& 1300\\
650& 325& 976& 488& 244& 122& 61& 184& 92& 46\\
23& 70& 35& 106& 53& 160& 80& 40& 20& 10\\
5& 16& 8& 4& 2& 1& \\

2381&&&&&&&&&\\
7144& 3572& 1786& 893& 2680& 1340& 670& 335& 1006& 503\\
1510& 755& 2266& 1133& 3400& 1700& 850& 425& 1276& 638\\
319& 958& 479& 1438& 719& 2158& 1079& 3238& 1619& 4858\\
2429& 7288& 3644& 1822& 911& 2734& 1367& 4102& 2051& 6154\\
3077& 9232& 4616& 2308& 1154& 577& 1732& 866& 433& 1300\\
650& 325& 976& 488& 244& 122& 61& 184& 92& 46\\
23& 70& 35& 106& 53& 160& 80& 40& 20& 10\\
5& 16& 8& 4& 2& 1& \\

2382&&&&&&&&&\\
1191& 3574& 1787& 5362& 2681& 8044& 4022& 2011& 6034& 3017\\
9052& 4526& 2263& 6790& 3395& 10186& 5093& 15280& 7640& 3820\\
1910& 955& 2866& 1433& 4300& 2150& 1075& 3226& 1613& 4840\\
2420& 1210& 605& 1816& 908& 454& 227& 682& 341& 1024\\
512& 256& 128& 64& 32& 16& 8& 4& 2& 1\\

2383&&&&&&&&&\\
7150& 3575& 10726& 5363& 16090& 8045& 24136& 12068& 6034& 3017\\
9052& 4526& 2263& 6790& 3395& 10186& 5093& 15280& 7640& 3820\\
1910& 955& 2866& 1433& 4300& 2150& 1075& 3226& 1613& 4840\\
2420& 1210& 605& 1816& 908& 454& 227& 682& 341& 1024\\
512& 256& 128& 64& 32& 16& 8& 4& 2& 1\\

2384&&&&&&&&&\\
1192& 596& 298& 149& 448& 224& 112& 56& 28& 14\\
7& 22& 11& 34& 17& 52& 26& 13& 40& 20\\
10& 5& 16& 8& 4& 2& 1& \\

2385&&&&&&&&&\\
7156& 3578& 1789& 5368& 2684& 1342& 671& 2014& 1007& 3022\\
1511& 4534& 2267& 6802& 3401& 10204& 5102& 2551& 7654& 3827\\
11482& 5741& 17224& 8612& 4306& 2153& 6460& 3230& 1615& 4846\\
2423& 7270& 3635& 10906& 5453& 16360& 8180& 4090& 2045& 6136\\
3068& 1534& 767& 2302& 1151& 3454& 1727& 5182& 2591& 7774\\
3887& 11662& 5831& 17494& 8747& 26242& 13121& 39364& 19682& 9841\\
29524& 14762& 7381& 22144& 11072& 5536& 2768& 1384& 692& 346\\
173& 520& 260& 130& 65& 196& 98& 49& 148& 74\\
37& 112& 56& 28& 14& 7& 22& 11& 34& 17\\
52& 26& 13& 40& 20& 10& 5& 16& 8& 4\\
2& 1& \\

2386&&&&&&&&&\\
1193& 3580& 1790& 895& 2686& 1343& 4030& 2015& 6046& 3023\\
9070& 4535& 13606& 6803& 20410& 10205& 30616& 15308& 7654& 3827\\
11482& 5741& 17224& 8612& 4306& 2153& 6460& 3230& 1615& 4846\\
2423& 7270& 3635& 10906& 5453& 16360& 8180& 4090& 2045& 6136\\
3068& 1534& 767& 2302& 1151& 3454& 1727& 5182& 2591& 7774\\
3887& 11662& 5831& 17494& 8747& 26242& 13121& 39364& 19682& 9841\\
29524& 14762& 7381& 22144& 11072& 5536& 2768& 1384& 692& 346\\
173& 520& 260& 130& 65& 196& 98& 49& 148& 74\\
37& 112& 56& 28& 14& 7& 22& 11& 34& 17\\
52& 26& 13& 40& 20& 10& 5& 16& 8& 4\\
2& 1& \\

2387&&&&&&&&&\\
7162& 3581& 10744& 5372& 2686& 1343& 4030& 2015& 6046& 3023\\
9070& 4535& 13606& 6803& 20410& 10205& 30616& 15308& 7654& 3827\\
11482& 5741& 17224& 8612& 4306& 2153& 6460& 3230& 1615& 4846\\
2423& 7270& 3635& 10906& 5453& 16360& 8180& 4090& 2045& 6136\\
3068& 1534& 767& 2302& 1151& 3454& 1727& 5182& 2591& 7774\\
3887& 11662& 5831& 17494& 8747& 26242& 13121& 39364& 19682& 9841\\
29524& 14762& 7381& 22144& 11072& 5536& 2768& 1384& 692& 346\\
173& 520& 260& 130& 65& 196& 98& 49& 148& 74\\
37& 112& 56& 28& 14& 7& 22& 11& 34& 17\\
52& 26& 13& 40& 20& 10& 5& 16& 8& 4\\
2& 1& \\

2388&&&&&&&&&\\
1194& 597& 1792& 896& 448& 224& 112& 56& 28& 14\\
7& 22& 11& 34& 17& 52& 26& 13& 40& 20\\
10& 5& 16& 8& 4& 2& 1& \\

2389&&&&&&&&&\\
7168& 3584& 1792& 896& 448& 224& 112& 56& 28& 14\\
7& 22& 11& 34& 17& 52& 26& 13& 40& 20\\
10& 5& 16& 8& 4& 2& 1& \\

2390&&&&&&&&&\\
1195& 3586& 1793& 5380& 2690& 1345& 4036& 2018& 1009& 3028\\
1514& 757& 2272& 1136& 568& 284& 142& 71& 214& 107\\
322& 161& 484& 242& 121& 364& 182& 91& 274& 137\\
412& 206& 103& 310& 155& 466& 233& 700& 350& 175\\
526& 263& 790& 395& 1186& 593& 1780& 890& 445& 1336\\
668& 334& 167& 502& 251& 754& 377& 1132& 566& 283\\
850& 425& 1276& 638& 319& 958& 479& 1438& 719& 2158\\
1079& 3238& 1619& 4858& 2429& 7288& 3644& 1822& 911& 2734\\
1367& 4102& 2051& 6154& 3077& 9232& 4616& 2308& 1154& 577\\
1732& 866& 433& 1300& 650& 325& 976& 488& 244& 122\\
61& 184& 92& 46& 23& 70& 35& 106& 53& 160\\
80& 40& 20& 10& 5& 16& 8& 4& 2& 1\\

2391&&&&&&&&&\\
7174& 3587& 10762& 5381& 16144& 8072& 4036& 2018& 1009& 3028\\
1514& 757& 2272& 1136& 568& 284& 142& 71& 214& 107\\
322& 161& 484& 242& 121& 364& 182& 91& 274& 137\\
412& 206& 103& 310& 155& 466& 233& 700& 350& 175\\
526& 263& 790& 395& 1186& 593& 1780& 890& 445& 1336\\
668& 334& 167& 502& 251& 754& 377& 1132& 566& 283\\
850& 425& 1276& 638& 319& 958& 479& 1438& 719& 2158\\
1079& 3238& 1619& 4858& 2429& 7288& 3644& 1822& 911& 2734\\
1367& 4102& 2051& 6154& 3077& 9232& 4616& 2308& 1154& 577\\
1732& 866& 433& 1300& 650& 325& 976& 488& 244& 122\\
61& 184& 92& 46& 23& 70& 35& 106& 53& 160\\
80& 40& 20& 10& 5& 16& 8& 4& 2& 1\\

2392&&&&&&&&&\\
1196& 598& 299& 898& 449& 1348& 674& 337& 1012& 506\\
253& 760& 380& 190& 95& 286& 143& 430& 215& 646\\
323& 970& 485& 1456& 728& 364& 182& 91& 274& 137\\
412& 206& 103& 310& 155& 466& 233& 700& 350& 175\\
526& 263& 790& 395& 1186& 593& 1780& 890& 445& 1336\\
668& 334& 167& 502& 251& 754& 377& 1132& 566& 283\\
850& 425& 1276& 638& 319& 958& 479& 1438& 719& 2158\\
1079& 3238& 1619& 4858& 2429& 7288& 3644& 1822& 911& 2734\\
1367& 4102& 2051& 6154& 3077& 9232& 4616& 2308& 1154& 577\\
1732& 866& 433& 1300& 650& 325& 976& 488& 244& 122\\
61& 184& 92& 46& 23& 70& 35& 106& 53& 160\\
80& 40& 20& 10& 5& 16& 8& 4& 2& 1\\

2393&&&&&&&&&\\
7180& 3590& 1795& 5386& 2693& 8080& 4040& 2020& 1010& 505\\
1516& 758& 379& 1138& 569& 1708& 854& 427& 1282& 641\\
1924& 962& 481& 1444& 722& 361& 1084& 542& 271& 814\\
407& 1222& 611& 1834& 917& 2752& 1376& 688& 344& 172\\
86& 43& 130& 65& 196& 98& 49& 148& 74& 37\\
112& 56& 28& 14& 7& 22& 11& 34& 17& 52\\
26& 13& 40& 20& 10& 5& 16& 8& 4& 2\\
1& \\

2394&&&&&&&&&\\
1197& 3592& 1796& 898& 449& 1348& 674& 337& 1012& 506\\
253& 760& 380& 190& 95& 286& 143& 430& 215& 646\\
323& 970& 485& 1456& 728& 364& 182& 91& 274& 137\\
412& 206& 103& 310& 155& 466& 233& 700& 350& 175\\
526& 263& 790& 395& 1186& 593& 1780& 890& 445& 1336\\
668& 334& 167& 502& 251& 754& 377& 1132& 566& 283\\
850& 425& 1276& 638& 319& 958& 479& 1438& 719& 2158\\
1079& 3238& 1619& 4858& 2429& 7288& 3644& 1822& 911& 2734\\
1367& 4102& 2051& 6154& 3077& 9232& 4616& 2308& 1154& 577\\
1732& 866& 433& 1300& 650& 325& 976& 488& 244& 122\\
61& 184& 92& 46& 23& 70& 35& 106& 53& 160\\
80& 40& 20& 10& 5& 16& 8& 4& 2& 1\\

2395&&&&&&&&&\\
7186& 3593& 10780& 5390& 2695& 8086& 4043& 12130& 6065& 18196\\
9098& 4549& 13648& 6824& 3412& 1706& 853& 2560& 1280& 640\\
320& 160& 80& 40& 20& 10& 5& 16& 8& 4\\
2& 1& \\

2396&&&&&&&&&\\
1198& 599& 1798& 899& 2698& 1349& 4048& 2024& 1012& 506\\
253& 760& 380& 190& 95& 286& 143& 430& 215& 646\\
323& 970& 485& 1456& 728& 364& 182& 91& 274& 137\\
412& 206& 103& 310& 155& 466& 233& 700& 350& 175\\
526& 263& 790& 395& 1186& 593& 1780& 890& 445& 1336\\
668& 334& 167& 502& 251& 754& 377& 1132& 566& 283\\
850& 425& 1276& 638& 319& 958& 479& 1438& 719& 2158\\
1079& 3238& 1619& 4858& 2429& 7288& 3644& 1822& 911& 2734\\
1367& 4102& 2051& 6154& 3077& 9232& 4616& 2308& 1154& 577\\
1732& 866& 433& 1300& 650& 325& 976& 488& 244& 122\\
61& 184& 92& 46& 23& 70& 35& 106& 53& 160\\
80& 40& 20& 10& 5& 16& 8& 4& 2& 1\\

2397&&&&&&&&&\\
7192& 3596& 1798& 899& 2698& 1349& 4048& 2024& 1012& 506\\
253& 760& 380& 190& 95& 286& 143& 430& 215& 646\\
323& 970& 485& 1456& 728& 364& 182& 91& 274& 137\\
412& 206& 103& 310& 155& 466& 233& 700& 350& 175\\
526& 263& 790& 395& 1186& 593& 1780& 890& 445& 1336\\
668& 334& 167& 502& 251& 754& 377& 1132& 566& 283\\
850& 425& 1276& 638& 319& 958& 479& 1438& 719& 2158\\
1079& 3238& 1619& 4858& 2429& 7288& 3644& 1822& 911& 2734\\
1367& 4102& 2051& 6154& 3077& 9232& 4616& 2308& 1154& 577\\
1732& 866& 433& 1300& 650& 325& 976& 488& 244& 122\\
61& 184& 92& 46& 23& 70& 35& 106& 53& 160\\
80& 40& 20& 10& 5& 16& 8& 4& 2& 1\\

2398&&&&&&&&&\\
1199& 3598& 1799& 5398& 2699& 8098& 4049& 12148& 6074& 3037\\
9112& 4556& 2278& 1139& 3418& 1709& 5128& 2564& 1282& 641\\
1924& 962& 481& 1444& 722& 361& 1084& 542& 271& 814\\
407& 1222& 611& 1834& 917& 2752& 1376& 688& 344& 172\\
86& 43& 130& 65& 196& 98& 49& 148& 74& 37\\
112& 56& 28& 14& 7& 22& 11& 34& 17& 52\\
26& 13& 40& 20& 10& 5& 16& 8& 4& 2\\
1& \\

2399&&&&&&&&&\\
7198& 3599& 10798& 5399& 16198& 8099& 24298& 12149& 36448& 18224\\
9112& 4556& 2278& 1139& 3418& 1709& 5128& 2564& 1282& 641\\
1924& 962& 481& 1444& 722& 361& 1084& 542& 271& 814\\
407& 1222& 611& 1834& 917& 2752& 1376& 688& 344& 172\\
86& 43& 130& 65& 196& 98& 49& 148& 74& 37\\
112& 56& 28& 14& 7& 22& 11& 34& 17& 52\\
26& 13& 40& 20& 10& 5& 16& 8& 4& 2\\
1& \\

2400&&&&&&&&&\\
1200& 600& 300& 150& 75& 226& 113& 340& 170& 85\\
256& 128& 64& 32& 16& 8& 4& 2& 1& \\
\bottomrule\end{longtable}
\section{Lunghezza ciclo}
\begin{longtable}{*{16}{l}}\toprule
\caption{Lunghezza ciclo}\\
\midrule
\endfirsthead
\multicolumn{16}{c} {\tablename\ \thetable\ -- \textit{Continua dalla pagina precedente}} \\
\toprule
\endhead
\bottomrule
\multicolumn{16}{r} {\textit{Continua nella pagina successiva}} \\
\endfoot
\endlastfoot
3& 1& 7& 2& 5& 8& 16& 3& 19& 6& 14& 9& 9& 17& 17& 4\\
12& 20& 20& 7& 7& 15& 15& 10& 23& 10& 111& 18& 18& 18& 106& 5\\
26& 13& 13& 21& 21& 21& 34& 8& 109& 8& 29& 16& 16& 16& 104& 11\\
24& 24& 24& 11& 11& 112& 112& 19& 32& 19& 32& 19& 19& 107& 107& 6\\
27& 27& 27& 14& 14& 14& 102& 22& 115& 22& 14& 22& 22& 35& 35& 9\\
22& 110& 110& 9& 9& 30& 30& 17& 30& 17& 92& 17& 17& 105& 105& 12\\
118& 25& 25& 25& 25& 25& 87& 12& 38& 12& 100& 113& 113& 113& 69& 20\\
12& 33& 33& 20& 20& 33& 33& 20& 95& 20& 46& 108& 108& 108& 46& 7\\
121& 28& 28& 28& 28& 28& 41& 15& 90& 15& 41& 15& 15& 103& 103& 23\\
116& 116& 116& 23& 23& 15& 15& 23& 36& 23& 85& 36& 36& 36& 54& 10\\
98& 23& 23& 111& 111& 111& 67& 10& 49& 10& 124& 31& 31& 31& 80& 18\\
31& 31& 31& 18& 18& 93& 93& 18& 44& 18& 44& 106& 106& 106& 44& 13\\
119& 119& 119& 26& 26& 26& 119& 26& 18& 26& 39& 26& 26& 88& 88& 13\\
39& 39& 39& 13& 13& 101& 101& 114& 26& 114& 52& 114& 114& 70& 70& 21\\
52& 13& 13& 34& 34& 34& 127& 21& 83& 21& 127& 34& 34& 34& 52& 21\\
21& 96& 96& 21& 21& 47& 47& 109& 47& 109& 65& 109& 109& 47& 47& 8\\
122& 122& 122& 29& 29& 29& 78& 29& 122& 29& 21& 29& 29& 42& 42& 16\\
29& 91& 91& 16& 16& 42& 42& 16& 42& 16& 60& 104& 104& 104& 42& 24\\
29& 117& 117& 117& 117& 117& 55& 24& 73& 24& 117& 16& 16& 16& 42& 24\\
37& 37& 37& 24& 24& 86& 86& 37& 130& 37& 37& 37& 37& 55& 55& 11\\
24& 99& 99& 24& 24& 24& 143& 112& 50& 112& 24& 112& 112& 68& 68& 11\\
112& 50& 50& 11& 11& 125& 125& 32& 125& 32& 125& 32& 32& 81& 81& 19\\
125& 32& 32& 32& 32& 32& 50& 19& 45& 19& 45& 94& 94& 94& 45& 19\\
19& 45& 45& 19& 19& 45& 45& 107& 63& 107& 58& 107& 107& 45& 45& 14\\
32& 120& 120& 120& 120& 120& 120& 27& 58& 27& 76& 27& 27& 120& 120& 27\\
19& 19& 19& 27& 27& 40& 40& 27& 40& 27& 133& 89& 89& 89& 133& 14\\
133& 40& 40& 40& 40& 40& 32& 14& 58& 14& 53& 102& 102& 102& 40& 115\\
27& 27& 27& 115& 115& 53& 53& 115& 27& 115& 53& 71& 71& 71& 97& 22\\
115& 53& 53& 14& 14& 14& 40& 35& 128& 35& 128& 35& 35& 128& 128& 22\\
35& 84& 84& 22& 22& 128& 128& 35& 35& 35& 27& 35& 35& 53& 53& 22\\
48& 22& 22& 97& 97& 97& 141& 22& 48& 22& 141& 48& 48& 48& 97& 110\\
22& 48& 48& 110& 110& 66& 66& 110& 61& 110& 35& 48& 48& 48& 61& 9\\
35& 123& 123& 123& 123& 123& 61& 30& 123& 30& 123& 30& 30& 79& 79& 30\\
30& 123& 123& 30& 30& 22& 22& 30& 22& 30& 48& 43& 43& 43& 136& 17\\
43& 30& 30& 92& 92& 92& 43& 17& 136& 17& 30& 43& 43& 43& 87& 17\\
43& 43& 43& 17& 17& 61& 61& 105& 56& 105& 30& 105& 105& 43& 43& 25\\
30& 30& 30& 118& 118& 118& 30& 118& 56& 118& 118& 118& 118& 56& 56& 25\\
74& 74& 74& 25& 25& 118& 118& 17& 56& 17& 69& 17& 17& 43& 43& 25\\
131& 38& 38& 38& 38& 38& 69& 25& 131& 25& 131& 87& 87& 87& 131& 38\\
25& 131& 131& 38& 38& 38& 38& 38& 30& 38& 30& 56& 56& 56& 131& 12\\
51& 25& 25& 100& 100& 100& 38& 25& 144& 25& 100& 25& 25& 144& 144& 113\\
51& 51& 51& 113& 113& 25& 25& 113& 51& 113& 144& 69& 69& 69& 95& 12\\
64& 113& 113& 51& 51& 51& 64& 12& 64& 12& 38& 126& 126& 126& 38& 33\\
126& 126& 126& 33& 33& 126& 126& 33& 126& 33& 64& 82& 82& 82& 170& 20\\
33& 126& 126& 33& 33& 33& 64& 33& 25& 33& 25& 33& 33& 51& 51& 20\\
46& 46& 46& 20& 20& 46& 46& 95& 33& 95& 139& 95& 95& 46& 46& 20\\
139& 20& 20& 46& 46& 46& 95& 20& 90& 20& 46& 46& 46& 46& 139& 108\\
20& 64& 64& 108& 108& 59& 59& 108& 33& 108& 152& 46& 46& 46& 59& 15\\
33& 33& 33& 121& 121& 121& 152& 121& 33& 121& 59& 121& 121& 121& 121& 28\\
121& 59& 59& 28& 28& 77& 77& 28& 77& 28& 103& 121& 121& 121& 72& 28\\
59& 20& 20& 20& 20& 20& 72& 28& 46& 28& 134& 41& 41& 41& 134& 28\\
41& 41& 41& 28& 28& 134& 134& 90& 134& 90& 41& 90& 90& 134& 134& 15\\
28& 134& 134& 41& 41& 41& 85& 41& 41& 41& 41& 41& 41& 33& 33& 15\\
59& 59& 59& 15& 15& 54& 54& 103& 28& 103& 147& 103& 103& 41& 41& 116\\
147& 28& 28& 28& 28& 28& 178& 116& 147& 116& 28& 54& 54& 54& 147& 116\\
116& 28& 28& 116& 116& 54& 54& 72& 147& 72& 46& 72& 72& 98& 98& 23\\
67& 116& 116& 54& 54& 54& 116& 15& 67& 15& 54& 15& 15& 41& 41& 36\\
129& 129& 129& 36& 36& 129& 129& 36& 129& 36& 67& 129& 129& 129& 116& 23\\
129& 36& 36& 85& 85& 85& 129& 23& 173& 23& 85& 129& 129& 129& 36& 36\\
36& 36& 36& 36& 36& 28& 28& 36& 28& 36& 28& 54& 54& 54& 129& 23\\
49& 49& 49& 23& 23& 23& 142& 98& 49& 98& 36& 98& 98& 142& 142& 23\\
98& 49& 49& 23& 23& 142& 142& 49& 23& 49& 36& 49& 49& 98& 98& 111\\
93& 23& 23& 49& 49& 49& 49& 111& 142& 111& 41& 67& 67& 67& 93& 111\\
111& 62& 62& 111& 111& 36& 36& 49& 155& 49& 62& 49& 49& 62& 62& 10\\
36& 36& 36& 124& 124& 124& 36& 124& 155& 124& 124& 124& 124& 62& 62& 31\\
124& 124& 124& 31& 31& 124& 124& 31& 62& 31& 93& 80& 80& 80& 168& 31\\
80& 31& 31& 124& 124& 124& 75& 31& 75& 31& 62& 23& 23& 23& 168& 31\\
23& 23& 23& 31& 31& 49& 49& 44& 137& 44& 137& 44& 44& 137& 137& 18\\
44& 44& 44& 31& 31& 31& 75& 93& 137& 93& 31& 93& 93& 44& 44& 18\\
93& 137& 137& 18& 18& 31& 31& 44& 137& 44& 93& 44& 44& 88& 88& 18\\
44& 44& 44& 44& 44& 44& 137& 18& 36& 18& 36& 62& 62& 62& 62& 106\\
18& 57& 57& 106& 106& 31& 31& 106& 150& 106& 57& 44& 44& 44& 57& 26\\
150& 31& 31& 31& 31& 31& 57& 119& 181& 119& 150& 119& 119& 31& 31& 119\\
57& 57& 57& 119& 119& 119& 119& 119& 31& 119& 57& 57& 57& 57& 88& 26\\
150& 75& 75& 75& 75& 75& 49& 26& 101& 26& 119& 119& 119& 119& 70& 18\\
57& 57& 57& 18& 18& 70& 70& 18& 57& 18& 70& 44& 44& 44& 163& 26\\
132& 132& 132& 39& 39& 39& 132& 39& 132& 39& 132& 39& 39& 70& 70& 26\\
132& 132& 132& 26& 26& 132& 132& 88& 39& 88& 70& 88& 88& 132& 132& 39\\
176& 26& 26& 132& 132& 132& 88& 39& 39& 39& 83& 39& 39& 39& 176& 39\\
39& 31& 31& 39& 39& 31& 31& 57& 31& 57& 83& 57& 57& 132& 132& 13\\
52& 52& 52& 26& 26& 26& 145& 101& 145& 101& 52& 101& 101& 39& 39& 26\\
101& 145& 145& 26& 26& 101& 101& 26& 52& 26& 176& 145& 145& 145& 101& 114\\
26& 52& 52& 52& 52& 52& 145& 114& 101& 114& 52& 26& 26& 26& 52& 114\\
52& 52& 52& 114& 114& 145& 145& 70& 44& 70& 26& 70& 70& 96& 96& 13\\
114& 65& 65& 114& 114& 114& 158& 52& 39& 52& 114& 52& 52& 65& 65& 13\\
52& 65& 65& 13& 13& 39& 39& 127& 39& 127& 114& 127& 127& 39& 39& 34\\
158& 127& 127& 127& 127& 127& 96& 34& 65& 34& 65& 127& 127& 127& 114& 34\\
34& 127& 127& 34& 34& 65& 65& 83& 96& 83& 127& 83& 83& 171& 171& 21\\
83& 34& 34& 127& 127& 127& 34& 34& 78& 34& 127& 34& 34& 65& 65& 34\\
26& 26& 26& 34& 34& 26& 26& 34& 26& 34& 78& 52& 52& 52& 127& 21\\
140& 47& 47& 47& 47& 47& 52& 21& 140& 21& 140& 47& 47& 47& 140& 96\\
34& 34& 34& 96& 96& 140& 140& 96& 34& 96& 140& 47& 47& 47& 171& 21\\
96& 140& 140& 21& 21& 21& 96& 47& 34& 47& 140& 47& 47& 96& 96& 21\\
47& 91& 91& 21& 21& 47& 47& 47& 47& 47& 47& 47& 47& 140& 140& 109\\
39& 21& 21& 65& 65& 65& 91& 109& 65& 109& 140& 60& 60& 60& 153& 109\\
109& 34& 34& 109& 109& 153& 153& 47& 60& 47& 60& 47& 47& 60& 60& 16\\
153& 34& 34& 34& 34& 34& 109& 122& 60& 122& 34& 122& 122& 153& 153& 122\\
122& 34& 34& 122& 122& 60& 60& 122& 60& 122& 153& 122& 122& 122& 60& 29\\
34& 122& 122& 60& 60& 60& 60& 29& 91& 29& 122& 78& 78& 78& 166& 29\\
78& 78& 78& 29& 29& 104& 104& 122& 122& 122& 73& 122& 122& 73& 73& 29\\
60& 60& 60& 21& 21& 21& 166& 21& 73& 21& 21& 21& 21& 73& 73& 29\\
47& 47& 47& 29& 29& 135& 135& 42& 135& 42& 73& 42& 42& 135& 135& 29\\
135& 42& 42& 42& 42& 42& 104& 29& 73& 29& 73& 135& 135& 135& 135& 91\\
29& 135& 135& 91& 91& 42& 42& 91& 73& 91& 42& 135& 135& 135& 73& 16\\
179& 29& 29& 135& 135& 135& 42& 42& 91& 42& 135& 42& 42& 86& 86& 42\\
42& 42& 42& 42& 42& 42& 42& 42& 34& 42& 135& 34& 34& 34& 179& 16\\
34& 60& 60& 60& 60& 60& 60& 16& 135& 16& 148& 55& 55& 55& 148& 104\\
29& 29& 29& 104& 104& 148& 148& 104& 55& 104& 55& 42& 42& 42& 55& 117\\
104& 148& 148& 29& 29& 29& 104& 29& 104& 29& 55& 29& 29& 179& 179& 117\\
148& 148& 148& 117& 117& 29& 29& 55& 55& 55& 42& 55& 55& 148& 148& 117\\
104& 117& 117& 29& 29& 29& 148& 117& 55& 117& 55& 55& 55& 55& 86& 73\\
117& 148& 148& 73& 73& 47& 47& 73& 29& 73& 47& 99& 99& 99& 99& 24\\
117& 68& 68& 117& 117& 117& 68& 55& 161& 55& 42& 55& 55& 117& 117& 16\\
55& 68& 68& 16& 16& 55& 55& 16& 68& 16& 161& 42& 42& 42& 161& 37\\
42& 130& 130& 130& 130& 130& 68& 37& 42& 37& 130& 130& 130& 130& 161& 37\\
130& 130& 130& 37& 37& 68& 68& 130& 68& 130& 130& 130& 130& 117& 117& 24\\
37& 130& 130& 37& 37& 37& 68& 86& 68& 86& 37& 86& 86& 130& 130& 24\\
86& 174& 174& 24& 24& 86& 86& 130& 37& 130& 86& 130& 130& 37& 37& 37\\
81& 37& 37& 37& 37& 37& 174& 37& 68& 37& 68& 29& 29& 29& 130& 37\\
37& 29& 29& 37& 37& 29& 29& 55& 81& 55& 174& 55& 55& 130& 130& 24\\
143& 50& 50& 50& 50& 50& 50& 24& 55& 24& 143& 24& 24& 143& 143& 99\\
50& 50& 50& 99& 99& 37& 37& 99& 37& 99& 81& 143& 143& 143& 174& 24\\
37& 99& 99& 50& 50& 50& 81& 24& 174& 24& 81& 143& 143& 143& 99& 50\\
24& 24& 24& 50& 50& 37& 37& 50& 143& 50& 143& 99& 99& 99& 50& 112\\
50& 94& 94& 24& 24& 24& 50& 50& 50& 50& 50& 50& 50& 50& 50& 112\\
50& 143& 143& 112& 112& 42& 42& 68& 24& 68& 42& 68& 68& 94& 94& 112\\
68& 112& 112& 63& 63& 63& 156& 112& 156& 112& 156& 37& 37& 37& 63& 50\\
112& 156& 156& 50& 50& 63& 63& 50& 63& 50& 156& 63& 63& 63& 156& 11\\
156& 37& 37& 37& 37& 37& 37& 125& 112& 125& 37& 125& 125& 37& 37& 125\\
125& 156& 156& 125& 125& 125& 125& 125& 37& 125& 94& 63& 63& 63& 50& 32\\
63& 125& 125& 125& 125& 125& 112& 32& 63& 32& 37& 125& 125& 125& 107& 32\\
63& 63& 63& 32& 32& 94& 94& 81& 125& 81& 156& 81& 81& 169& 169& 32\\
81& 81& 81& 32& 32& 32& 81& 125& 107& 125& 32& 125& 125& 76& 76& 32\\
125& 76& 76& 32& 32& 63& 63& 24& 63& 24& 76& 24& 24& 169& 169& 32\\
76& 24& 24& 24& 24& 24& 107& 32& 76& 32& 169& 50& 50& 50& 125& 45\\
32& 138& 138& 45& 45& 138& 138& 45& 76& 45& 50& 138& 138& 138& 76& 19\\
138& 45& 45& 45& 45& 45& 138& 32& 107& 32& 76& 32& 32& 76& 76& 94\\
138& 138& 138& 94& 94& 32& 32& 94& 138& 94& 50& 45& 45& 45& 169& 19\\
76& 94& 94& 138& 138& 138& 120& 19& 76& 19& 94& 32& 32& 32& 182& 45\\
138& 138& 138& 45& 45& 94& 94& 45& 138& 45& 45& 89& 89& 89& 138& 19\\
45& 45& 45& 45& 45& 45& 76& 45& 45& 45& 37& 45& 45& 138& 138& 19\\
37& 37& 37& 19& 19& 37& 37& 63& 63& 63& 89& 63& 63& 63& 63& 107\\
138& 19& 19& 58& 58& 58& 151& 107& 151& 107& 58& 32& 32& 32& 120& 107\\
107& 151& 151& 107& 107& 58& 58& 45& 58& 45& 151& 45& 45& 58& 58& 27\\
107& 151& 151& 32& 32& 32& 151& 32& 107& 32& 107& 32& 32& 58& 58& 120\\
32& 182& 182& 120& 120& 151& 151& 120& 151& 120& 107& 32& 32& 32& 89& 120\\
58& 58& 58& 58& 58& 58& 151& 120& 151& 120& 107& 120& 120& 120& 58& 120\\
32& 32& 32& 120& 120& 58& 58& 58& 58& 58& 58& 58& 58& 89& 89& 27\\
120& 151& 151& 76& 76& 76& 164& 76& 50& 76& 76& 76& 76& 50& 50& 27\\
102& 102& 102& 27& 27& 120& 120& 120& 71& 120& 32& 120& 120& 71& 71& 19\\
\bottomrule\end{longtable}
\section{Frequenza cicli}
\begin{longtable}{llllllllllll}\toprule
\caption{Frequenza cicli}\\
\midrule
\textbf{c} & \textbf{f} & \textbf{c} & \textbf{f}& \textbf{c} & \textbf{f} & \textbf{c} & \textbf{f}& \textbf{c} & \textbf{f} & \textbf{c} & \textbf{f}\\
\toprule
\endfirsthead
\multicolumn{12}{c} {\tablename\ \thetable\ -- \textit{Continua dalla pagina precedente}} \\
\textbf{c} & \textbf{f} & \textbf{c} & \textbf{f}& \textbf{c} & \textbf{f} & \textbf{c} & \textbf{f}& \textbf{c} & \textbf{f} & \textbf{c} & \textbf{f}\\
\toprule
\endhead
\bottomrule
\multicolumn{12}{r} {\textit{Continua nella pagina successiva}} \\
\endfoot
\endlastfoot
1 & 1&2 &1&3& 2&4 &1&5 &2&6& 2\\
7 & 4&8 &4&9& 6&10 &6&11 &8&12& 9\\
13 & 13&14 &12&15& 17&16 &22&17 &16&18& 22\\
19 & 26&20 &22&21& 32&22 &18&23 &28&24& 43\\
25 & 23&26 &38&27& 22&28 &30&29 &49&30& 23\\
31 & 40&32 &58&33& 25&34 &47&35 &14&36& 30\\
37 & 60&38 &16&39& 32&40 &10&41 &23&42& 47\\
43 & 15&44 &29&45& 43&46 &21&47 &39&48& 11\\
49 & 22&50 &47&51& 10&52 &27&53 &8&54& 15\\
55 & 32&56 &8&57& 20&58 &29&59 &10&60& 26\\
61 & 5&62 &13&63& 29&64 &7&65 &17&66& 2\\
67 & 7&68 &21&69& 6&70 &13&71 &6&72& 6\\
73 & 17&74 &3&75& 8&76 &24&77 &3&78& 9\\
79 & 2&80 &5&81& 15&82 &3&83 &8&84& 2\\
85 & 6&86 &13&87& 5&88 &10&89 &10&90& 6\\
91 & 12&92 &4&93& 11&94 &20&95 &7&96& 15\\
97 & 5&98 &10&99& 17&100 &5&101 &12&102& 7\\
103 & 7&104 &16&105& 6&106 &9&107 &22&108& 8\\
109 & 13&110 &7&111& 11&112 &19&113 &10&114& 17\\
115 & 7&116 &14&117& 24&118 &11&119 &19&120& 31\\
121 & 14&122 &25&123& 9&124 &17&125 &32&126& 11\\
127 & 22&128 &6&129& 15&130 &29&131 &7&132& 18\\
133 & 3&134 &9&135& 21&136 &2&137 &9&138& 24\\
139 & 3&140 &13&141& 2&142 &6&143 &15&144& 4\\
145 & 10&146 &0&147& 5&148 &14&149 &0&150& 4\\
151 & 15&152 &2&153& 7&154 &0&155 &2&156& 11\\
157 & 0&158 &2&159& 0&160 &0&161 &4&162& 0\\
163 & 1&164 &1&165& 0&166 &2&167 &0&168& 2\\
169 & 6&170 &1&171& 3&172 &0&173 &1&174& 6\\
175 & 0&176 &3&177& 0&178 &1&179 &4&180& 0\\
181 & 1&182 &3&183& 0&184 &0&185 &0&186& 0\\
\bottomrule\end{longtable}
\section{Valori massini}
\begin{longtable}{llllllllllll}\toprule
\caption{Valori massimi}\\
\midrule
\textbf{N} & \textbf{R} & \textbf{N} & \textbf{R}& \textbf{N} & \textbf{R} & \textbf{N} & \textbf{R}& \textbf{N} & \textbf{R} & \textbf{N} & \textbf{R}\\
\midrule
\endfirsthead
\multicolumn{12}{c} {\tablename\ \thetable\ -- \textit{Continua dalla pagina precedente}} \\
\textbf{N} & \textbf{R} & \textbf{N} & \textbf{R}& \textbf{N} & \textbf{R} & \textbf{N} & \textbf{R}& \textbf{N} & \textbf{R} & \textbf{N} & \textbf{R}\\
\toprule
\endhead
\bottomrule
\multicolumn{12}{r} {\textit{Continua nella pagina successiva}} \\
\endfoot
\endlastfoot
1 & 4&2 &1&3& 16&4 &2&5 &16&6& 16\\
7 & 52&8 &4&9& 52&10 &16&11 &52&12& 16\\
13 & 40&14 &52&15& 160&16 &8&17 &52&18& 52\\
19 & 88&20 &16&21& 64&22 &52&23 &160&24& 16\\
25 & 88&26 &40&27& 9232&28 &52&29 &88&30& 160\\
31 & 9232&32 &16&33& 100&34 &52&35 &160&36& 52\\
37 & 112&38 &88&39& 304&40 &20&41 &9232&42& 64\\
43 & 196&44 &52&45& 136&46 &160&47 &9232&48& 24\\
49 & 148&50 &88&51& 232&52 &40&53 &160&54& 9232\\
55 & 9232&56 &52&57& 196&58 &88&59 &304&60& 160\\
61 & 184&62 &9232&63& 9232&64 &32&65 &196&66& 100\\
67 & 304&68 &52&69& 208&70 &160&71 &9232&72& 52\\
73 & 9232&74 &112&75& 340&76 &88&77 &232&78& 304\\
79 & 808&80 &40&81& 244&82 &9232&83 &9232&84& 64\\
85 & 256&86 &196&87& 592&88 &52&89 &304&90& 136\\
91 & 9232&92 &160&93& 280&94 &9232&95 &9232&96& 48\\
97 & 9232&98 &148&99& 448&100 &88&101 &304&102& 232\\
103 & 9232&104 &52&105& 808&106 &160&107 &9232&108& 9232\\
109 & 9232&110 &9232&111& 9232&112 &56&113 &340&114& 196\\
115 & 520&116 &88&117& 352&118 &304&119 &808&120& 160\\
121 & 9232&122 &184&123& 628&124 &9232&125 &9232&126& 9232\\
127 & 4372&128 &64&129& 9232&130 &196&131 &592&132& 100\\
133 & 400&134 &304&135& 916&136 &68&137 &9232&138& 208\\
139 & 628&140 &160&141& 424&142 &9232&143 &9232&144& 72\\
145 & 9232&146 &9232&147& 9232&148 &112&149 &448&150& 340\\
151 & 1024&152 &88&153& 520&154 &232&155 &9232&156& 304\\
157 & 472&158 &808&159& 9232&160 &80&161 &9232&162& 244\\
163 & 736&164 &9232&165& 9232&166 &9232&167 &9232&168& 84\\
169 & 4372&170 &256&171& 9232&172 &196&173 &520&174& 592\\
175 & 9232&176 &88&177& 532&178 &304&179 &808&180& 136\\
181 & 544&182 &9232&183& 9232&184 &160&185 &628&186& 280\\
187 & 952&188 &9232&189& 9232&190 &9232&191 &4372&192& 96\\
193 & 9232&194 &9232&195& 9232&196 &148&197 &592&198& 448\\
199 & 9232&200 &100&201& 1024&202 &304&203 &916&204& 232\\
205 & 616&206 &9232&207& 9232&208 &104&209 &628&210& 808\\
211 & 952&212 &160&213& 640&214 &9232&215 &9232&216& 9232\\
217 & 736&218 &9232&219& 1672&220 &9232&221 &9232&222& 9232\\
223 & 9232&224 &112&225& 4372&226 &340&227 &1024&228& 196\\
229 & 688&230 &520&231& 9232&232 &116&233 &9232&234& 352\\
235 & 9232&236 &304&237& 712&238 &808&239 &9232&240& 160\\
241 & 724&242 &9232&243& 9232&244 &184&245 &736&246& 628\\
247 & 1672&248 &9232&249& 952&250 &9232&251 &9232&252& 9232\\
253 & 9232&254 &4372&255& 13120&256 &128&257 &9232&258& 9232\\
259 & 9232&260 &196&261& 784&262 &592&263 &9232&264& 132\\
265 & 9232&266 &400&267& 1204&268 &304&269 &808&270& 916\\
271 & 2752&272 &136&273& 820&274 &9232&275 &9232&276& 208\\
277 & 832&278 &628&279& 1888&280 &160&281 &952&282& 424\\
283 & 9232&284 &9232&285& 9232&286 &9232&287 &4372&288& 144\\
289 & 868&290 &9232&291& 9232&292 &9232&293 &9232&294& 9232\\
295 & 2248&296 &148&297& 9232&298 &448&299 &9232&300& 340\\
301 & 904&302 &1024&303& 3076&304 &152&305 &916&306& 520\\
307 & 1384&308 &232&309& 928&310 &9232&311 &9232&312& 304\\
313 & 9232&314 &472&315& 1600&316 &808&317 &952&318& 9232\\
319 & 9232&320 &160&321& 964&322 &9232&323 &9232&324& 244\\
325 & 976&326 &736&327& 9232&328 &9232&329 &1672&330& 9232\\
331 & 1492&332 &9232&333& 9232&334 &9232&335 &9232&336& 168\\
337 & 9232&338 &4372&339& 4372&340 &256&341 &1024&342& 9232\\
343 & 9232&344 &196&345& 9232&346 &520&347 &9232&348& 592\\
349 & 1048&350 &9232&351& 9232&352 &176&353 &9232&354& 532\\
355 & 1600&356 &304&357& 1072&358 &808&359 &9232&360& 180\\
361 & 2752&362 &544&363& 1636&364 &9232&365 &9232&366& 9232\\
367 & 4192&368 &184&369& 1108&370 &628&371 &1672&372& 280\\
373 & 1120&374 &952&375& 2536&376 &9232&377 &9232&378& 9232\\
379 & 2752&380 &9232&381& 9232&382 &4372&383 &13120&384& 192\\
385 & 1156&386 &9232&387& 9232&388 &9232&389 &9232&390& 9232\\
391 & 9232&392 &196&393& 2248&394 &592&395 &9232&396& 448\\
397 & 1192&398 &9232&399& 9232&400 &200&401 &1204&402& 1024\\
403 & 1816&404 &304&405& 1216&406 &916&407 &2752&408& 232\\
409 & 1384&410 &616&411& 9232&412 &9232&413 &9232&414& 9232\\
415 & 9232&416 &208&417& 9232&418 &628&419 &1888&420& 808\\
421 & 1264&422 &952&423& 3220&424 &212&425 &9232&426& 640\\
427 & 2752&428 &9232&429& 9232&430 &9232&431 &4372&432& 9232\\
433 & 1300&434 &736&435& 1960&436 &9232&437 &9232&438& 1672\\
439 & 2968&440 &9232&441& 1492&442 &9232&443 &2248&444& 9232\\
445 & 9232&446 &9232&447& 39364&448 &224&449 &9232&450& 4372\\
451 & 4372&452 &340&453& 1360&454 &1024&455 &3076&456& 228\\
457 & 9232&458 &688&459& 9232&460 &520&461 &1384&462& 9232\\
463 & 9232&464 &232&465& 1396&466 &9232&467 &9232&468& 352\\
469 & 1408&470 &9232&471& 9232&472 &304&473 &1600&474& 712\\
475 & 3616&476 &808&477& 1432&478 &9232&479 &9232&480& 240\\
481 & 2752&482 &724&483& 2176&484 &9232&485 &9232&486& 9232\\
487 & 9232&488 &244&489& 4192&490 &736&491 &9232&492& 628\\
493 & 1480&494 &1672&495& 14308&496 &9232&497 &1492&498& 952\\
499 & 2248&500 &9232&501& 9232&502 &9232&503 &9232&504& 9232\\
505 & 2752&506 &9232&507& 2896&508 &4372&509 &4372&510& 13120\\
511 & 39364&512 &256&513& 1540&514 &9232&515 &9232&516& 9232\\
517 & 9232&518 &9232&519& 3508&520 &260&521 &9232&522& 784\\
523 & 9232&524 &592&525& 1576&526 &9232&527 &9232&528& 264\\
529 & 1588&530 &9232&531& 9232&532 &400&533 &1600&534& 1204\\
535 & 3616&536 &304&537& 1816&538 &808&539 &9232&540& 916\\
541 & 1624&542 &2752&543& 9232&544 &272&545 &1636&546& 820\\
547 & 2464&548 &9232&549& 9232&550 &9232&551 &4192&552& 276\\
553 & 9232&554 &832&555& 2500&556 &628&557 &1672&558& 1888\\
559 & 8080&560 &280&561& 1684&562 &952&563 &2536&564& 424\\
565 & 1696&566 &9232&567& 9232&568 &9232&569 &2752&570& 9232\\
571 & 2896&572 &9232&573& 9232&574 &4372&575 &13120&576& 288\\
577 & 1732&578 &868&579& 2608&580 &9232&581 &9232&582& 9232\\
583 & 3940&584 &9232&585& 2968&586 &9232&587 &9232&588& 9232\\
589 & 9232&590 &2248&591& 5992&592 &296&593 &9232&594& 9232\\
595 & 9232&596 &448&597& 1792&598 &9232&599 &9232&600& 340\\
601 & 4372&602 &904&603& 5812&604 &1024&605 &1816&606& 3076\\
607 & 9232&608 &304&609& 9232&610 &916&611 &2752&612& 520\\
613 & 1840&614 &1384&615& 10528&616 &308&617 &9232&618& 928\\
619 & 9232&620 &9232&621& 9232&622 &9232&623 &9232&624& 312\\
625 & 1876&626 &9232&627& 9232&628 &472&629 &1888&630& 1600\\
631 & 4264&632 &808&633& 3616&634 &952&635 &3220&636& 9232\\
637 & 9232&638 &9232&639& 41524&640 &320&641 &2752&642& 964\\
643 & 2896&644 &9232&645& 9232&646 &9232&647 &4372&648& 324\\
649 & 9232&650 &976&651& 9232&652 &736&653 &1960&654& 9232\\
655 & 9232&656 &9232&657& 1972&658 &1672&659 &2968&660& 9232\\
661 & 9232&662 &1492&663& 4480&664 &9232&665 &2248&666& 9232\\
667 & 21688&668 &9232&669& 9232&670 &9232&671 &39364&672& 336\\
673 & 2752&674 &9232&675& 9232&676 &4372&677 &4372&678& 4372\\
679 & 5812&680 &340&681& 39364&682 &1024&683 &3076&684& 9232\\
685 & 9232&686 &9232&687& 6964&688 &344&689 &9232&690& 9232\\
691 & 9232&692 &520&693& 2080&694 &9232&695 &9232&696& 592\\
697 & 9232&698 &1048&699& 3544&700 &9232&701 &9232&702& 9232\\
703 & 250504&704 &352&705& 2116&706 &9232&707 &9232&708& 532\\
709 & 2128&710 &1600&711& 4804&712 &356&713 &3616&714& 1072\\
715 & 3220&716 &808&717& 2152&718 &9232&719 &9232&720& 360\\
721 & 2164&722 &2752&723& 3256&724 &544&725 &2176&726& 1636\\
727 & 4912&728 &9232&729& 2464&730 &9232&731 &9232&732& 9232\\
733 & 9232&734 &4192&735& 11176&736 &368&737 &9232&738& 1108\\
739 & 3328&740 &628&741& 2224&742 &1672&743 &14308&744& 372\\
745 & 8080&746 &1120&747& 4264&748 &952&749 &2248&750& 2536\\
751 & 21688&752 &9232&753& 2260&754 &9232&755 &9232&756& 9232\\
757 & 9232&758 &2752&759& 5128&760 &9232&761 &2896&762& 9232\\
763 & 9232&764 &4372&765& 4372&766 &13120&767 &39364&768& 384\\
769 & 2308&770 &1156&771& 3472&772 &9232&773 &9232&774& 9232\\
775 & 9232&776 &9232&777& 3940&778 &9232&779 &3508&780& 9232\\
781 & 9232&782 &9232&783& 9232&784 &392&785 &9232&786& 2248\\
787 & 3544&788 &592&789& 2368&790 &9232&791 &9232&792& 448\\
793 & 9232&794 &1192&795& 39364&796 &9232&797 &9232&798& 9232\\
799 & 12148&800 &400&801& 4372&802 &1204&803 &3616&804& 1024\\
805 & 2416&806 &1816&807& 39364&808 &404&809 &9232&810& 1216\\
811 & 9232&812 &916&813& 2440&814 &2752&815 &9232&816& 408\\
817 & 2452&818 &1384&819& 3688&820 &616&821 &2464&822& 9232\\
823 & 9232&824 &9232&825& 9232&826 &9232&827 &4192&828& 9232\\
829 & 9232&830 &9232&831& 18952&832 &416&833 &2500&834& 9232\\
835 & 9232&836 &628&837& 2512&838 &1888&839 &8080&840& 808\\
841 & 4264&842 &1264&843& 3796&844 &952&845 &2536&846& 3220\\
847 & 8584&848 &424&849& 9232&850 &9232&851 &9232&852& 640\\
853 & 2560&854 &2752&855& 5776&856 &9232&857 &2896&858& 9232\\
859 & 9232&860 &9232&861& 9232&862 &4372&863 &13120&864& 9232\\
865 & 9232&866 &1300&867& 3904&868 &736&869 &2608&870& 1960\\
871 & 190996&872 &9232&873& 9232&874 &9232&875 &3940&876& 1672\\
877 & 2632&878 &2968&879& 10024&880 &9232&881 &9232&882& 1492\\
883 & 3976&884 &9232&885& 9232&886 &2248&887 &5992&888& 9232\\
889 & 21688&890 &9232&891& 8584&892 &9232&893 &9232&894& 39364\\
895 & 39364&896 &448&897& 2752&898 &9232&899 &9232&900& 4372\\
901 & 4372&902 &4372&903& 9232&904 &452&905 &5812&906& 1360\\
907 & 13120&908 &1024&909& 2728&910 &3076&911 &9232&912& 456\\
913 & 9232&914 &9232&915& 9232&916 &688&917 &2752&918& 9232\\
919 & 9232&920 &520&921& 9232&922 &1384&923 &10528&924& 9232\\
925 & 9232&926 &9232&927& 15856&928 &464&929 &9232&930& 1396\\
931 & 4192&932 &9232&933& 9232&934 &9232&935 &9232&936& 468\\
937 & 250504&938 &1408&939& 9232&940 &9232&941 &9232&942& 9232\\
943 & 9556&944 &472&945& 2836&946 &1600&947 &4264&948& 712\\
949 & 2848&950 &3616&951& 6424&952 &808&953 &3220&954& 1432\\
955 & 4840&956 &9232&957& 9232&958 &9232&959 &41524&960& 480\\
961 & 2884&962 &2752&963& 4336&964 &724&965 &2896&966& 2176\\
967 & 9232&968 &9232&969& 4912&970 &9232&971 &4372&972& 9232\\
973 & 9232&974 &9232&975& 9880&976 &488&977 &9232&978& 4192\\
979 & 4408&980 &736&981& 2944&982 &9232&983 &9232&984& 628\\
985 & 3328&986 &1480&987& 7504&988 &1672&989 &2968&990& 14308\\
991 & 15064&992 &9232&993& 8080&994 &1492&995 &4480&996& 952\\
997 & 2992&998 &2248&999& 11392&1000 &9232&1001 &21688&1002& 9232\\
1003 & 8584&1004 &9232&1005& 9232&1006 &9232&1007 &39364&1008& 9232\\
1009 & 9232&1010 &2752&1011& 4552&1012 &9232&1013 &9232&1014& 2896\\
1015 & 6856&1016 &4372&1017& 9232&1018 &4372&1019 &5812&1020& 13120\\
1021 & 13120&1022 &39364&1023& 118096&1024 &512&1025 &3076&1026& 1540\\
1027 & 4624&1028 &9232&1029& 9232&1030 &9232&1031 &6964&1032& 9232\\
1033 & 9232&1034 &9232&1035& 9232&1036 &9232&1037 &9232&1038& 3508\\
1039 & 10528&1040 &520&1041& 9232&1042 &9232&1043 &9232&1044& 784\\
1045 & 3136&1046 &9232&1047& 9232&1048 &592&1049 &3544&1050& 1576\\
1051 & 45520&1052 &9232&1053& 9232&1054 &9232&1055 &250504&1056& 528\\
1057 & 9232&1058 &1588&1059& 4768&1060 &9232&1061 &9232&1062& 9232\\
1063 & 8080&1064 &532&1065& 12148&1066 &1600&1067 &4804&1068& 1204\\
1069 & 3208&1070 &3616&1071& 10852&1072 &536&1073 &3220&1074& 1816\\
1075 & 4840&1076 &808&1077& 3232&1078 &9232&1079 &9232&1080& 916\\
1081 & 9232&1082 &1624&1083& 9232&1084 &2752&1085 &3256&1086& 9232\\
1087 & 24784&1088 &544&1089& 3268&1090 &1636&1091 &4912&1092& 820\\
1093 & 3280&1094 &2464&1095& 7396&1096 &9232&1097 &9232&1098& 9232\\
1099 & 4948&1100 &9232&1101& 9232&1102 &4192&1103 &11176&1104& 552\\
1105 & 9232&1106 &9232&1107& 9232&1108 &832&1109 &3328&1110& 2500\\
1111 & 7504&1112 &628&1113& 9232&1114 &1672&1115 &14308&1116& 1888\\
1117 & 3352&1118 &8080&1119& 17008&1120 &560&1121 &4264&1122& 1684\\
1123 & 5056&1124 &952&1125& 3376&1126 &2536&1127 &21688&1128& 564\\
1129 & 8584&1130 &1696&1131& 5092&1132 &9232&1133 &9232&1134& 9232\\
1135 & 14560&1136 &9232&1137& 3412&1138 &2752&1139 &5128&1140& 9232\\
1141 & 9232&1142 &2896&1143& 7720&1144 &9232&1145 &9232&1146& 9232\\
1147 & 5812&1148 &4372&1149& 4372&1150 &13120&1151 &39364&1152& 576\\
1153 & 9232&1154 &1732&1155& 5200&1156 &868&1157 &3472&1158& 2608\\
1159 & 7828&1160 &9232&1161& 190996&1162 &9232&1163 &9232&1164& 9232\\
1165 & 9232&1166 &3940&1167& 11824&1168 &9232&1169 &3508&1170& 2968\\
1171 & 5272&1172 &9232&1173& 9232&1174 &9232&1175 &9232&1176& 9232\\
1177 & 3976&1178 &9232&1179& 13444&1180 &2248&1181 &3544&1182& 5992\\
1183 & 45520&1184 &592&1185& 21688&1186 &9232&1187 &9232&1188& 9232\\
1189 & 9232&1190 &9232&1191& 15280&1192 &596&1193 &39364&1194& 1792\\
1195 & 9232&1196 &9232&1197& 9232&1198 &9232&1199 &12148&1200& 600\\
1201 & 4372&1202 &4372&1203& 5416&1204 &904&1205 &3616&1206& 5812\\
1207 & 8152&1208 &1024&1209& 13120&1210 &1816&1211 &39364&1212& 3076\\
1213 & 3640&1214 &9232&1215& 27700&1216 &608&1217 &9232&1218& 9232\\
1219 & 9232&1220 &916&1221& 3664&1222 &2752&1223 &9232&1224& 612\\
1225 & 9232&1226 &1840&1227& 9232&1228 &1384&1229 &3688&1230& 10528\\
1231 & 12472&1232 &616&1233& 9232&1234 &9232&1235 &9232&1236& 928\\
1237 & 3712&1238 &9232&1239& 9232&1240 &9232&1241 &4192&1242& 9232\\
1243 & 9448&1244 &9232&1245& 9232&1246 &9232&1247 &18952&1248& 624\\
1249 & 250504&1250 &1876&1251& 5632&1252 &9232&1253 &9232&1254& 9232\\
1255 & 14308&1256 &628&1257& 9556&1258 &1888&1259 &8080&1260& 1600\\
1261 & 3784&1262 &4264&1263& 138400&1264 &808&1265 &3796&1266& 3616\\
1267 & 5704&1268 &952&1269& 3808&1270 &3220&1271 &8584&1272& 9232\\
1273 & 4840&1274 &9232&1275& 39364&1276 &9232&1277 &9232&1278& 41524\\
1279 & 65608&1280 &640&1281& 3844&1282 &2752&1283 &5776&1284& 964\\
1285 & 3856&1286 &2896&1287& 9232&1288 &9232&1289 &9232&1290& 9232\\
1291 & 5812&1292 &9232&1293& 9232&1294 &4372&1295 &13120&1296& 648\\
1297 & 9232&1298 &9232&1299& 9232&1300 &976&1301 &3904&1302& 9232\\
1303 & 9232&1304 &736&1305& 4408&1306 &1960&1307 &190996&1308& 9232\\
1309 & 9232&1310 &9232&1311& 19924&1312 &9232&1313 &3940&1314& 1972\\
1315 & 5920&1316 &1672&1317& 3952&1318 &2968&1319 &10024&1320& 9232\\
1321 & 15064&1322 &9232&1323& 5956&1324 &1492&1325 &3976&1326& 4480\\
1327 & 13444&1328 &9232&1329& 3988&1330 &2248&1331 &5992&1332& 9232\\
1333 & 9232&1334 &21688&1335& 21688&1336 &9232&1337 &8584&1338& 9232\\
1339 & 6784&1340 &9232&1341& 9232&1342 &39364&1343 &39364&1344& 672\\
1345 & 9232&1346 &2752&1347& 6064&1348 &9232&1349 &9232&1350& 9232\\
1351 & 17332&1352 &4372&1353& 6856&1354 &4372&1355 &9232&1356& 4372\\
1357 & 4372&1358 &5812&1359& 13768&1360 &680&1361 &13120&1362& 39364\\
1363 & 39364&1364 &1024&1365& 4096&1366 &3076&1367 &9232&1368& 9232\\
1369 & 4624&1370 &9232&1371& 10420&1372 &9232&1373 &9232&1374& 6964\\
1375 & 20896&1376 &688&1377& 9232&1378 &9232&1379 &9232&1380& 9232\\
1381 & 9232&1382 &9232&1383& 95956&1384 &692&1385 &10528&1386& 2080\\
1387 & 6244&1388 &9232&1389& 9232&1390 &9232&1391 &15856&1392& 696\\
1393 & 4180&1394 &9232&1395& 9232&1396 &1048&1397 &4192&1398& 3544\\
1399 & 9448&1400 &9232&1401& 45520&1402 &9232&1403 &9232&1404& 9232\\
1405 & 9232&1406 &250504&1407& 250504&1408 &704&1409 &9232&1410& 2116\\
1411 & 6352&1412 &9232&1413& 9232&1414 &9232&1415 &9556&1416& 708\\
1417 & 8080&1418 &2128&1419& 9232&1420 &1600&1421 &4264&1422& 4804\\
1423 & 14416&1424 &712&1425& 4276&1426 &3616&1427 &6424&1428& 1072\\
1429 & 4288&1430 &3220&1431& 9664&1432 &808&1433 &4840&1434& 2152\\
1435 & 39364&1436 &9232&1437& 9232&1438 &9232&1439 &41524&1440& 720\\
1441 & 9232&1442 &2164&1443& 6496&1444 &2752&1445 &4336&1446& 3256\\
1447 & 10996&1448 &724&1449& 24784&1450 &2176&1451 &9232&1452& 1636\\
1453 & 4360&1454 &4912&1455& 14740&1456 &9232&1457 &4372&1458& 2464\\
1459 & 6568&1460 &9232&1461& 9232&1462 &9232&1463 &9880&1464& 9232\\
1465 & 4948&1466 &9232&1467& 9232&1468 &4192&1469 &4408&1470& 11176\\
1471 & 190996&1472 &736&1473& 9232&1474 &9232&1475 &9232&1476& 1108\\
1477 & 4432&1478 &3328&1479& 9988&1480 &740&1481 &7504&1482& 2224\\
1483 & 9232&1484 &1672&1485& 4456&1486 &14308&1487 &15064&1488& 744\\
1489 & 4468&1490 &8080&1491& 8080&1492 &1120&1493 &4480&1494& 4264\\
1495 & 10096&1496 &952&1497& 5056&1498 &2248&1499 &11392&1500& 2536\\
1501 & 4504&1502 &21688&1503& 22840&1504 &9232&1505 &8584&1506& 2260\\
1507 & 6784&1508 &9232&1509& 9232&1510 &9232&1511 &39364&1512& 9232\\
1513 & 14560&1514 &9232&1515& 65608&1516 &2752&1517 &4552&1518& 5128\\
1519 & 25972&1520 &9232&1521& 9232&1522 &2896&1523 &6856&1524& 9232\\
1525 & 9232&1526 &9232&1527& 10312&1528 &4372&1529 &5812&1530& 4372\\
1531 & 8728&1532 &13120&1533& 13120&1534 &39364&1535 &118096&1536& 768\\
1537 & 9232&1538 &2308&1539& 6928&1540 &1156&1541 &4624&1542& 3472\\
1543 & 10420&1544 &9232&1545& 7828&1546 &9232&1547 &6964&1548& 9232\\
1549 & 9232&1550 &9232&1551& 15712&1552 &9232&1553 &9232&1554& 3940\\
1555 & 7000&1556 &9232&1557& 9232&1558 &3508&1559 &10528&1560& 9232\\
1561 & 5272&1562 &9232&1563& 26728&1564 &9232&1565 &9232&1566& 9232\\
1567 & 23812&1568 &784&1569& 4708&1570 &9232&1571 &9232&1572& 2248\\
1573 & 4720&1574 &3544&1575& 11968&1576 &788&1577 &45520&1578& 2368\\
1579 & 9232&1580 &9232&1581& 9232&1582 &9232&1583 &250504&1584& 792\\
1585 & 9232&1586 &9232&1587& 9232&1588 &1192&1589 &4768&1590& 39364\\
1591 & 39364&1592 &9232&1593& 9232&1594 &9232&1595 &8080&1596& 9232\\
1597 & 9232&1598 &12148&1599& 36448&1600 &800&1601 &4804&1602& 4372\\
1603 & 7216&1604 &1204&1605& 4816&1606 &3616&1607 &10852&1608& 1024\\
1609 & 8152&1610 &2416&1611& 7252&1612 &1816&1613 &4840&1614& 39364\\
1615 & 39364&1616 &808&1617& 4852&1618 &9232&1619 &9232&1620& 1216\\
1621 & 4864&1622 &9232&1623& 10960&1624 &916&1625 &9232&1626& 2440\\
1627 & 13912&1628 &2752&1629& 4888&1630 &9232&1631 &24784&1632& 816\\
1633 & 9232&1634 &2452&1635& 7360&1636 &1384&1637 &4912&1638& 3688\\
1639 & 95956&1640 &820&1641& 12472&1642 &2464&1643 &7396&1644& 9232\\
1645 & 9232&1646 &9232&1647& 18772&1648 &9232&1649 &4948&1650& 9232\\
1651 & 9232&1652 &9232&1653& 9232&1654 &4192&1655 &11176&1656& 9232\\
1657 & 9448&1658 &9232&1659& 8404&1660 &9232&1661 &9232&1662& 18952\\
1663 & 95956&1664 &832&1665& 250504&1666 &2500&1667 &7504&1668& 9232\\
1669 & 9232&1670 &9232&1671& 11284&1672 &836&1673 &14308&1674& 2512\\
1675 & 9232&1676 &1888&1677& 5032&1678 &8080&1679 &17008&1680& 840\\
1681 & 5044&1682 &4264&1683& 7576&1684 &1264&1685 &5056&1686& 3796\\
1687 & 11392&1688 &952&1689& 5704&1690 &2536&1691 &21688&1692& 3220\\
1693 & 5080&1694 &8584&1695& 48904&1696 &848&1697 &5092&1698& 9232\\
1699 & 9232&1700 &9232&1701& 9232&1702 &9232&1703 &14560&1704& 852\\
1705 & 65608&1706 &2560&1707& 9232&1708 &2752&1709 &5128&1710& 5776\\
1711 & 17332&1712 &9232&1713& 5140&1714 &2896&1715 &7720&1716& 9232\\
1717 & 9232&1718 &9232&1719& 11608&1720 &9232&1721 &5812&1722& 9232\\
1723 & 8728&1724 &4372&1725& 5176&1726 &13120&1727 &39364&1728& 9232\\
1729 & 9232&1730 &9232&1731& 9232&1732 &1300&1733 &5200&1734& 3904\\
1735 & 11716&1736 &868&1737& 9232&1738 &2608&1739 &7828&1740& 1960\\
1741 & 5224&1742 &190996&1743& 190996&1744 &9232&1745 &9232&1746& 9232\\
1747 & 9232&1748 &9232&1749& 9232&1750 &3940&1751 &11824&1752& 1672\\
1753 & 5920&1754 &2632&1755& 13336&1756 &2968&1757 &5272&1758& 10024\\
1759 & 26728&1760 &9232&1761& 15064&1762 &9232&1763 &9232&1764& 1492\\
1765 & 5296&1766 &3976&1767& 22660&1768 &9232&1769 &13444&1770& 9232\\
1771 & 10096&1772 &2248&1773& 5320&1774 &5992&1775 &45520&1776& 9232\\
1777 & 9232&1778 &21688&1779& 21688&1780 &9232&1781 &9232&1782& 8584\\
1783 & 12040&1784 &9232&1785& 6784&1786 &9232&1787 &15280&1788& 39364\\
1789 & 39364&1790 &39364&1791& 103336&1792 &896&1793 &9232&1794& 2752\\
1795 & 8080&1796 &9232&1797& 9232&1798 &9232&1799 &12148&1800& 4372\\
1801 & 17332&1802 &4372&1803& 8116&1804 &4372&1805 &5416&1806& 9232\\
1807 & 18304&1808 &904&1809& 5428&1810 &5812&1811 &8152&1812& 1360\\
1813 & 5440&1814 &13120&1815& 13120&1816 &1024&1817 &39364&1818& 2728\\
1819 & 1276936&1820 &3076&1821& 5464&1822 &9232&1823 &27700&1824& 912\\
1825 & 5476&1826 &9232&1827& 9232&1828 &9232&1829 &9232&1830& 9232\\
1831 & 13912&1832 &916&1833& 20896&1834 &2752&1835 &9232&1836& 9232\\
1837 & 9232&1838 &9232&1839& 18628&1840 &920&1841 &9232&1842& 9232\\
1843 & 9232&1844 &1384&1845& 5536&1846 &10528&1847 &12472&1848& 9232\\
1849 & 6244&1850 &9232&1851& 9376&1852 &9232&1853 &9232&1854& 15856\\
1855 & 42280&1856 &928&1857& 5572&1858 &9232&1859 &9232&1860& 1396\\
1861 & 5584&1862 &4192&1863& 15928&1864 &9232&1865 &9448&1866& 9232\\
1867 & 8404&1868 &9232&1869& 9232&1870 &9232&1871 &18952&1872& 936\\
1873 & 9232&1874 &250504&1875& 250504&1876 &1408&1877 &5632&1878& 9232\\
1879 & 12688&1880 &9232&1881& 6352&1882 &9232&1883 &14308&1884& 9232\\
1885 & 9232&1886 &9556&1887& 28672&1888 &944&1889 &8080&1890& 2836\\
1891 & 8512&1892 &1600&1893& 5680&1894 &4264&1895 &138400&1896& 948\\
1897 & 14416&1898 &2848&1899& 8548&1900 &3616&1901 &5704&1902& 6424\\
1903 & 21688&1904 &952&1905& 5716&1906 &3220&1907 &8584&1908& 1432\\
1909 & 5728&1910 &4840&1911& 12904&1912 &9232&1913 &39364&1914& 9232\\
1915 & 1276936&1916 &9232&1917& 9232&1918 &41524&1919 &65608&1920& 960\\
1921 & 9232&1922 &2884&1923& 8656&1924 &2752&1925 &5776&1926& 4336\\
1927 & 13012&1928 &964&1929& 10996&1930 &2896&1931 &9232&1932& 2176\\
1933 & 5800&1934 &9232&1935& 19600&1936 &9232&1937 &5812&1938& 4912\\
1939 & 8728&1940 &9232&1941& 9232&1942 &4372&1943 &13120&1944& 9232\\
1945 & 6568&1946 &9232&1947& 22192&1948 &9232&1949 &9232&1950& 9880\\
1951 & 33352&1952 &976&1953& 5860&1954 &9232&1955 &9232&1956& 4192\\
1957 & 5872&1958 &4408&1959& 80512&1960 &980&1961 &190996&1962& 2944\\
1963 & 9448&1964 &9232&1965& 9232&1966 &9232&1967 &19924&1968& 984\\
1969 & 5908&1970 &3328&1971& 8872&1972 &1480&1973 &5920&1974& 7504\\
1975 & 13336&1976 &1672&1977& 9232&1978 &2968&1979 &10024&1980& 14308\\
1981 & 14308&1982 &15064&1983& 50848&1984 &9232&1985 &5956&1986& 8080\\
1987 & 8944&1988 &1492&1989& 5968&1990 &4480&1991 &13444&1992& 996\\
1993 & 10096&1994 &2992&1995& 8980&1996 &2248&1997 &5992&1998& 11392\\
1999 & 20248&2000 &9232&2001& 6004&2002 &21688&2003 &21688&2004& 9232\\
2005 & 9232&2006 &8584&2007& 13552&2008 &9232&2009 &6784&2010& 9232\\
2011 & 15280&2012 &9232&2013& 9232&2014 &39364&2015 &39364&2016& 9232\\
2017 & 14560&2018 &9232&2019& 9232&2020 &2752&2021 &6064&2022& 4552\\
2023 & 23056&2024 &9232&2025& 25972&2026 &9232&2027 &17332&2028& 2896\\
2029 & 6088&2030 &6856&2031& 34720&2032 &4372&2033 &9232&2034& 9232\\
2035 & 9232&2036 &4372&2037& 6112&2038 &5812&2039 &13768&2040& 13120\\
2041 & 8728&2042 &13120&2043& 44224&2044 &39364&2045 &39364&2046& 118096\\
2047 & 1276936&2048 &1024&2049& 9232&2050 &3076&2051 &9232&2052& 1540\\
2053 & 6160&2054 &4624&2055& 13876&2056 &9232&2057 &10420&2058& 9232\\
2059 & 9268&2060 &9232&2061& 9232&2062 &6964&2063 &20896&2064& 9232\\
2065 & 9232&2066 &9232&2067& 9304&2068 &9232&2069 &9232&2070& 9232\\
2071 & 13984&2072 &9232&2073& 7000&2074 &9232&2075 &95956&2076& 3508\\
2077 & 6232&2078 &10528&2079& 31588&2080 &1040&2081 &6244&2082& 9232\\
2083 & 9376&2084 &9232&2085& 9232&2086 &9232&2087 &15856&2088& 1044\\
2089 & 23812&2090 &3136&2091& 9412&2092 &9232&2093 &9232&2094& 9232\\
2095 & 21220&2096 &1048&2097& 6292&2098 &3544&2099 &9448&2100& 1576\\
2101 & 6304&2102 &45520&2103& 45520&2104 &9232&2105 &9232&2106& 9232\\
2107 & 21688&2108 &9232&2109& 9232&2110 &250504&2111 &250504&2112& 1056\\
2113 & 9232&2114 &9232&2115& 9520&2116 &1588&2117 &6352&2118& 4768\\
2119 & 14308&2120 &9232&2121& 39364&2122 &9232&2123 &9556&2124& 9232\\
2125 & 9232&2126 &8080&2127& 21544&2128 &1064&2129 &9232&2130& 12148\\
2131 & 12148&2132 &1600&2133& 6400&2134 &4804&2135 &14416&2136& 1204\\
2137 & 7216&2138 &3208&2139& 61720&2140 &3616&2141 &6424&2142& 10852\\
2143 & 32560&2144 &1072&2145& 8152&2146 &3220&2147 &9664&2148& 1816\\
2149 & 6448&2150 &4840&2151& 176740&2152 &1076&2153 &39364&2154& 3232\\
2155 & 1276936&2156 &9232&2157& 9232&2158 &9232&2159 &41524&2160& 1080\\
2161 & 6484&2162 &9232&2163& 9736&2164 &1624&2165 &6496&2166& 9232\\
2167 & 14632&2168 &2752&2169& 13912&2170 &3256&2171 &10996&2172& 9232\\
2173 & 9232&2174 &24784&2175& 74356&2176 &1088&2177 &9232&2178& 3268\\
2179 & 9808&2180 &1636&2181& 6544&2182 &4912&2183 &14740&2184& 1092\\
2185 & 95956&2186 &3280&2187& 10528&2188 &2464&2189 &6568&2190& 7396\\
2191 & 22192&2192 &9232&2193& 9232&2194 &9232&2195 &9880&2196& 9232\\
2197 & 9232&2198 &4948&2199& 14848&2200 &9232&2201 &9232&2202& 9232\\
2203 & 25108&2204 &4192&2205& 6616&2206 &11176&2207 &190996&2208& 1104\\
2209 & 9448&2210 &9232&2211& 9952&2212 &9232&2213 &9232&2214& 9232\\
2215 & 345544&2216 &1108&2217& 95956&2218 &3328&2219 &9988&2220& 2500\\
2221 & 6664&2222 &7504&2223& 250504&2224 &1112&2225 &9232&2226& 9232\\
2227 & 10024&2228 &1672&2229& 6688&2230 &14308&2231 &15064&2232& 1888\\
2233 & 9232&2234 &3352&2235& 11320&2236 &8080&2237 &8080&2238& 17008\\
2239 & 51028&2240 &1120&2241& 6724&2242 &4264&2243 &10096&2244& 1684\\
2245 & 6736&2246 &5056&2247& 15172&2248 &1124&2249 &11392&2250& 3376\\
2251 & 10132&2252 &2536&2253& 6760&2254 &21688&2255 &22840&2256& 1128\\
2257 & 6772&2258 &8584&2259& 10168&2260 &1696&2261 &6784&2262& 5092\\
2263 & 15280&2264 &9232&2265& 9232&2266 &9232&2267 &39364&2268& 9232\\
2269 & 9232&2270 &14560&2271& 34504&2272 &9232&2273 &65608&2274& 3412\\
2275 & 10240&2276 &2752&2277& 6832&2278 &5128&2279 &25972&2280& 9232\\
2281 & 17332&2282 &9232&2283& 13012&2284 &2896&2285 &6856&2286& 7720\\
2287 & 250504&2288 &9232&2289& 9232&2290 &9232&2291 &10312&2292& 9232\\
2293 & 9232&2294 &5812&2295& 15496&2296 &4372&2297 &8728&2298& 4372\\
2299 & 44224&2300 &13120&2301& 13120&2302 &39364&2303 &118096&2304& 1152\\
2305 & 9232&2306 &9232&2307& 10384&2308 &1732&2309 &6928&2310& 5200\\
2311 & 15604&2312 &1156&2313& 11716&2314 &3472&2315 &10420&2316& 2608\\
2317 & 6952&2318 &7828&2319& 23488&2320 &9232&2321 &6964&2322& 190996\\
2323 & 190996&2324 &9232&2325& 9232&2326 &9232&2327 &15712&2328& 9232\\
2329 & 9232&2330 &9232&2331& 59776&2332 &3940&2333 &7000&2334& 11824\\
2335 & 95956&2336 &9232&2337& 7012&2338 &3508&2339 &10528&2340& 2968\\
2341 & 7024&2342 &5272&2343& 17800&2344 &9232&2345 &26728&2346& 9232\\
2347 & 14308&2348 &9232&2349& 9232&2350 &9232&2351 &23812&2352& 9232\\
2353 & 7060&2354 &3976&2355& 10600&2356 &9232&2357 &9232&2358& 13444\\
2359 & 15928&2360 &2248&2361& 10096&2362 &3544&2363 &11968&2364& 5992\\
2365 & 7096&2366 &45520&2367& 53944&2368 &1184&2369 &9232&2370& 21688\\
2371 & 21688&2372 &9232&2373& 9232&2374 &9232&2375 &250504&2376& 9232\\
2377 & 12040&2378 &9232&2379& 10708&2380 &9232&2381 &9232&2382& 15280\\
2383 & 24136&2384 &1192&2385& 39364&2386 &39364&2387 &39364&2388& 1792\\
2389 & 7168&2390 &9232&2391& 16144&2392 &9232&2393 &8080&2394& 9232\\
2395 & 18196&2396 &9232&2397& 9232&2398 &12148&2399 &36448&2400& 1200\\
\bottomrule\end{longtable}
\section{Cicli massini}
\begin{longtable}{*{16}{l}}\toprule
\caption{Cicli massini}\\
\midrule
\endfirsthead
\multicolumn{16}{c} {\tablename\ \thetable\ -- \textit{Continua dalla pagina precedente}} \\
\toprule
\endhead
\bottomrule
\multicolumn{16}{r} {\textit{Continua nella pagina successiva}} \\
\endfoot
\endlastfoot
1& 2& 4& 8& 16& 20& 24& 32& 40& 48& 52& 56& 64& 68& 72& 80\\
84& 88& 96& 100& 104& 112& 116& 128& 132& 136& 144& 148& 152& 160& 168& 176\\
180& 184& 192& 196& 200& 208& 212& 224& 228& 232& 240& 244& 256& 260& 264& 272\\
276& 280& 288& 296& 304& 308& 312& 320& 324& 336& 340& 344& 352& 356& 360& 368\\
372& 384& 392& 400& 404& 408& 416& 424& 448& 452& 456& 464& 468& 472& 480& 488\\
512& 520& 528& 532& 536& 544& 552& 560& 564& 576& 592& 596& 600& 608& 612& 616\\
624& 628& 640& 648& 672& 680& 688& 692& 696& 704& 708& 712& 720& 724& 736& 740\\
744& 768& 784& 788& 792& 800& 808& 816& 820& 832& 836& 840& 848& 852& 868& 896\\
904& 912& 916& 920& 928& 936& 944& 948& 952& 960& 964& 976& 980& 984& 996& 1024\\
1040& 1044& 1048& 1056& 1064& 1072& 1076& 1080& 1088& 1092& 1104& 1108& 1112& 1120& 1124& 1128\\
1152& 1156& 1184& 1192& 1200& 1204& 1216& 1264& 1300& 1360& 1384& 1396& 1408& 1432& 1480& 1492\\
1540& 1576& 1588& 1600& 1624& 1636& 1672& 1684& 1696& 1732& 1792& 1816& 1840& 1876& 1888& 1960\\
1972& 2080& 2116& 2128& 2152& 2164& 2176& 2224& 2248& 2260& 2308& 2368& 2416& 2440& 2452& 2464\\
2500& 2512& 2536& 2560& 2608& 2632& 2728& 2752& 2836& 2848& 2884& 2896& 2944& 2968& 2992& 3076\\
3136& 3208& 3220& 3232& 3256& 3268& 3280& 3328& 3352& 3376& 3412& 3472& 3508& 3544& 3616& 3640\\
3664& 3688& 3712& 3784& 3796& 3808& 3844& 3856& 3904& 3940& 3952& 3976& 3988& 4096& 4180& 4192\\
4264& 4276& 4288& 4336& 4360& 4372& 4408& 4432& 4456& 4468& 4480& 4504& 4552& 4624& 4708& 4720\\
4768& 4804& 4816& 4840& 4852& 4864& 4888& 4912& 4948& 5032& 5044& 5056& 5080& 5092& 5128& 5140\\
5176& 5200& 5224& 5272& 5296& 5320& 5416& 5428& 5440& 5464& 5476& 5536& 5572& 5584& 5632& 5680\\
5704& 5716& 5728& 5776& 5800& 5812& 5860& 5872& 5908& 5920& 5956& 5968& 5992& 6004& 6064& 6088\\
6112& 6160& 6232& 6244& 6292& 6304& 6352& 6400& 6424& 6448& 6484& 6496& 6544& 6568& 6616& 6664\\
6688& 6724& 6736& 6760& 6772& 6784& 6832& 6856& 6928& 6952& 6964& 7000& 7012& 7024& 7060& 7096\\
7168& 7216& 7252& 7360& 7396& 7504& 7576& 7720& 7828& 8080& 8116& 8152& 8404& 8512& 8548& 8584\\
8656& 8728& 8872& 8944& 8980& 9232& 9268& 9304& 9376& 9412& 9448& 9520& 9556& 9664& 9736& 9808\\
9880& 9952& 9988& 10024& 10096& 10132& 10168& 10240& 10312& 10384& 10420& 10528& 10600& 10708& 10852& 10960\\
10996& 11176& 11284& 11320& 11392& 11608& 11716& 11824& 11968& 12040& 12148& 12472& 12688& 12904& 13012& 13120\\
13336& 13444& 13552& 13768& 13876& 13912& 13984& 14308& 14416& 14560& 14632& 14740& 14848& 15064& 15172& 15280\\
15496& 15604& 15712& 15856& 15928& 16144& 17008& 17332& 17800& 18196& 18304& 18628& 18772& 18952& 19600& 19924\\
20248& 20896& 21220& 21544& 21688& 22192& 22660& 22840& 23056& 23488& 23812& 24136& 24784& 25108& 25972& 26728\\
27700& 28672& 31588& 32560& 33352& 34504& 34720& 36448& 39364& 41524& 42280& 44224& 45520& 48904& 50848& 51028\\
53944& 59776& 61720& 65608& 74356& 80512& 95956& 103336& 118096& 138400& 176740& 190996& 250504& 345544& 1276936& \\
\bottomrule\end{longtable}
\section{Cicli massimi comuni}
\begin{longtable}{*{24}{l}}\toprule
\caption{Cicli massimi comuni}\\
\midrule
\endfirsthead
\multicolumn{24}{c} {\tablename\ \thetable\ -- \textit{Continua dalla pagina precedente}} \\
\toprule
\endhead
\bottomrule
\multicolumn{24}{r} {\textit{Continua nella pagina successiva}} \\
\endfoot
\endlastfoot
1&&&&&&&&&\\
2& \\
2&&&&&&&&&\\
4& \\
4&&&&&&&&&\\
1& 8& \\
8&&&&&&&&&\\
16& \\
16&&&&&&&&&\\
3& 5& 6& 10& 12\\
20& 24& 32& \\
20&&&&&&&&&\\
40& \\
24&&&&&&&&&\\
48& \\
32&&&&&&&&&\\
64& \\
40&&&&&&&&&\\
13& 26& 52& 80\\
48&&&&&&&&&\\
96& \\
52&&&&&&&&&\\
7& 9& 11& 14& 17& 18& 22& 28& 34\\
36& 44& 56& 68& 72& 88& 104& \\
56&&&&&&&&&\\
112& \\
64&&&&&&&&&\\
21& 42\\
84& 128& \\
68&&&&&&&&&\\
136& \\
72&&&&&&&&&\\
144& \\
80&&&&&&&&&\\
160& \\
84&&&&&&&&&\\
168& \\
88&&&&&&&&&\\
19& 25& 29& 38\\
50& 58& 76& 100& 116& 152& 176& \\
96&&&&&&&&&\\
192& \\
100&&&&&&&&&\\
33& 66\\
132& 200& \\
104&&&&&&&&&\\
208& \\
112&&&&&&&&&\\
37& 74& 148& 224& \\
116&&&&&&&&&\\
232& \\
128&&&&&&&&&\\
256& \\
132&&&&&&&&&\\
264\\
136&&&&&&&&&\\
45& 90& 180& 272& \\
144&&&&&&&&&\\
288& \\
148&&&&&&&&&\\
49& 98& 196& 296& \\
152&&&&&&&&&\\
304\\
160&&&&&&&&&\\
15& 23& 30& 35& 46& 53& 60& 70& 92& 106\\
120& 140& 184& 212& 240& 280& 320& \\
168&&&&&&&&&\\
336& \\
176&&&&&&&&&\\
352& \\
180&&&&&&&&&\\
360\\
184&&&&&&&&&\\
61& 122& 244& 368& \\
192&&&&&&&&&\\
384& \\
196&&&&&&&&&\\
43& 57& 65& 86& 114\\
130& 172& 228& 260& 344& 392& \\
200&&&&&&&&&\\
400& \\
208&&&&&&&&&\\
69& 138& 276\\
416& \\
212&&&&&&&&&\\
424& \\
224&&&&&&&&&\\
448& \\
228&&&&&&&&&\\
456& \\
232&&&&&&&&&\\
51& 77& 102& 154& 204& 308\\
408& 464& \\
240&&&&&&&&&\\
480& \\
244&&&&&&&&&\\
81& 162& 324& 488& \\
256&&&&&&&&&\\
85& 170& 340\\
512& \\
260&&&&&&&&&\\
520& \\
264&&&&&&&&&\\
528& \\
272&&&&&&&&&\\
544& \\
276&&&&&&&&&\\
552& \\
280&&&&&&&&&\\
93& 186& 372& 560& \\
288&&&&&&&&&\\
576\\
296&&&&&&&&&\\
592& \\
304&&&&&&&&&\\
39& 59& 67& 78& 89& 101& 118& 134& 156\\
178& 202& 236& 268& 312& 356& 404& 472& 536& 608\\
308&&&&&&&&&\\
616& \\
312&&&&&&&&&\\
624& \\
320&&&&&&&&&\\
640& \\
324&&&&&&&&&\\
648& \\
336&&&&&&&&&\\
672& \\
340&&&&&&&&&\\
75& 113& 150& 226& 300\\
452& 600& 680& \\
344&&&&&&&&&\\
688& \\
352&&&&&&&&&\\
117& 234& 468& 704& \\
356&&&&&&&&&\\
712& \\
360&&&&&&&&&\\
720\\
368&&&&&&&&&\\
736& \\
372&&&&&&&&&\\
744& \\
384&&&&&&&&&\\
768& \\
392&&&&&&&&&\\
784& \\
400&&&&&&&&&\\
133& 266& 532& 800& \\
404&&&&&&&&&\\
808& \\
408&&&&&&&&&\\
816\\
416&&&&&&&&&\\
832& \\
424&&&&&&&&&\\
141& 282& 564& 848& \\
448&&&&&&&&&\\
99& 149& 198& 298& 396\\
596& 792& 896& \\
452&&&&&&&&&\\
904& \\
456&&&&&&&&&\\
912& \\
464&&&&&&&&&\\
928& \\
468&&&&&&&&&\\
936& \\
472&&&&&&&&&\\
157& 314& 628\\
944& \\
480&&&&&&&&&\\
960& \\
488&&&&&&&&&\\
976& \\
512&&&&&&&&&\\
1024& \\
520&&&&&&&&&\\
115& 153& 173& 230& 306& 346\\
460& 612& 692& 920& 1040& \\
528&&&&&&&&&\\
1056& \\
532&&&&&&&&&\\
177& 354& 708& 1064\\
536&&&&&&&&&\\
1072& \\
544&&&&&&&&&\\
181& 362& 724& 1088& \\
552&&&&&&&&&\\
1104& \\
560&&&&&&&&&\\
1120& \\
564&&&&&&&&&\\
1128& \\
576&&&&&&&&&\\
1152& \\
592&&&&&&&&&\\
87\\
131& 174& 197& 262& 348& 394& 524& 696& 788& 1048\\
1184& \\
596&&&&&&&&&\\
1192& \\
600&&&&&&&&&\\
1200& \\
608&&&&&&&&&\\
1216& \\
612&&&&&&&&&\\
1224& \\
616&&&&&&&&&\\
205& 410& 820& 1232& \\
624&&&&&&&&&\\
1248\\
628&&&&&&&&&\\
123& 139& 185& 209& 246& 278& 370& 418& 492& 556\\
740& 836& 984& 1112& 1256& \\
640&&&&&&&&&\\
213& 426& 852& 1280& \\
648&&&&&&&&&\\
1296\\
672&&&&&&&&&\\
1344& \\
680&&&&&&&&&\\
1360& \\
688&&&&&&&&&\\
229& 458& 916& 1376& \\
692&&&&&&&&&\\
1384& \\
696&&&&&&&&&\\
1392& \\
704&&&&&&&&&\\
1408& \\
708&&&&&&&&&\\
1416\\
712&&&&&&&&&\\
237& 474& 948& 1424& \\
720&&&&&&&&&\\
1440& \\
724&&&&&&&&&\\
241& 482& 964& 1448& \\
736&&&&&&&&&\\
163\\
217& 245& 326& 434& 490& 652& 868& 980& 1304& 1472\\
740&&&&&&&&&\\
1480& \\
744&&&&&&&&&\\
1488& \\
768&&&&&&&&&\\
1536& \\
784&&&&&&&&&\\
261& 522& 1044& 1568& \\
788&&&&&&&&&\\
1576& \\
792&&&&&&&&&\\
1584& \\
800&&&&&&&&&\\
1600\\
808&&&&&&&&&\\
79& 105& 119& 158& 179& 210& 238& 269& 316& 358\\
420& 476& 538& 632& 716& 840& 952& 1076& 1264& 1432\\
1616& \\
816&&&&&&&&&\\
1632& \\
820&&&&&&&&&\\
273& 546& 1092& 1640& \\
832&&&&&&&&&\\
277& 554& 1108& 1664\\
836&&&&&&&&&\\
1672& \\
840&&&&&&&&&\\
1680& \\
848&&&&&&&&&\\
1696& \\
852&&&&&&&&&\\
1704& \\
868&&&&&&&&&\\
289& 578& 1156& 1736& \\
896&&&&&&&&&\\
1792& \\
904&&&&&&&&&\\
301\\
602& 1204& 1808& \\
912&&&&&&&&&\\
1824& \\
916&&&&&&&&&\\
135& 203& 270& 305& 406& 540\\
610& 812& 1080& 1220& 1624& 1832& \\
920&&&&&&&&&\\
1840& \\
928&&&&&&&&&\\
309& 618& 1236\\
1856& \\
936&&&&&&&&&\\
1872& \\
944&&&&&&&&&\\
1888& \\
948&&&&&&&&&\\
1896& \\
952&&&&&&&&&\\
187& 211& 249& 281& 317& 374\\
422& 498& 562& 634& 748& 844& 996& 1124& 1268& 1496\\
1688& 1904& \\
960&&&&&&&&&\\
1920& \\
964&&&&&&&&&\\
321& 642& 1284& 1928& \\
976&&&&&&&&&\\
325& 650& 1300\\
1952& \\
980&&&&&&&&&\\
1960& \\
984&&&&&&&&&\\
1968& \\
996&&&&&&&&&\\
1992& \\
1024&&&&&&&&&\\
151& 201& 227& 302& 341& 402\\
454& 604& 682& 804& 908& 1208& 1364& 1608& 1816& 2048\\
1040&&&&&&&&&\\
2080& \\
1044&&&&&&&&&\\
2088& \\
1048&&&&&&&&&\\
349& 698& 1396& 2096& \\
1056&&&&&&&&&\\
2112& \\
1064&&&&&&&&&\\
2128& \\
1072&&&&&&&&&\\
357& 714\\
1428& 2144& \\
1076&&&&&&&&&\\
2152& \\
1080&&&&&&&&&\\
2160& \\
1088&&&&&&&&&\\
2176& \\
1092&&&&&&&&&\\
2184& \\
1104&&&&&&&&&\\
2208& \\
1108&&&&&&&&&\\
369& 738& 1476\\
2216& \\
1112&&&&&&&&&\\
2224& \\
1120&&&&&&&&&\\
373& 746& 1492& 2240& \\
1124&&&&&&&&&\\
2248& \\
1128&&&&&&&&&\\
2256& \\
1152&&&&&&&&&\\
2304& \\
1156&&&&&&&&&\\
385\\
770& 1540& 2312& \\
1184&&&&&&&&&\\
2368& \\
1192&&&&&&&&&\\
397& 794& 1588& 2384& \\
1200&&&&&&&&&\\
2400& \\
1204&&&&&&&&&\\
267\\
401& 534& 802& 1068& 1604& 2136& \\
1216&&&&&&&&&\\
405& 810& 1620& \\
1264&&&&&&&&&\\
421\\
842& 1684& \\
1300&&&&&&&&&\\
433& 866& 1732& \\
1360&&&&&&&&&\\
453& 906& 1812& \\
1384&&&&&&&&&\\
307& 409\\
461& 614& 818& 922& 1228& 1636& 1844& \\
1396&&&&&&&&&\\
465& 930& 1860\\
1408&&&&&&&&&\\
469& 938& 1876& \\
1432&&&&&&&&&\\
477& 954& 1908& \\
1480&&&&&&&&&\\
493& 986& 1972& \\
1492&&&&&&&&&\\
331\\
441& 497& 662& 882& 994& 1324& 1764& 1988& \\
1540&&&&&&&&&\\
513& 1026\\
2052& \\
1576&&&&&&&&&\\
525& 1050& 2100& \\
1588&&&&&&&&&\\
529& 1058& 2116& \\
1600&&&&&&&&&\\
315& 355& 473\\
533& 630& 710& 946& 1066& 1260& 1420& 1892& 2132& \\
1624&&&&&&&&&\\
541\\
1082& 2164& \\
1636&&&&&&&&&\\
363& 545& 726& 1090& 1452& 2180& \\
1672&&&&&&&&&\\
219& 247\\
329& 371& 438& 494& 557& 658& 742& 876& 988& 1114\\
1316& 1484& 1752& 1976& 2228& \\
1684&&&&&&&&&\\
561& 1122& 2244& \\
1696&&&&&&&&&\\
565& 1130\\
2260& \\
1732&&&&&&&&&\\
577& 1154& 2308& \\
1792&&&&&&&&&\\
597& 1194& 2388& \\
1816&&&&&&&&&\\
403& 537& 605\\
806& 1074& 1210& 1612& 2148& \\
1840&&&&&&&&&\\
613& 1226& \\
1876&&&&&&&&&\\
625& 1250& \\
1888&&&&&&&&&\\
279\\
419& 558& 629& 838& 1116& 1258& 1676& 2232& \\
1960&&&&&&&&&\\
435& 653\\
870& 1306& 1740& \\
1972&&&&&&&&&\\
657& 1314& \\
2080&&&&&&&&&\\
693& 1386& \\
2116&&&&&&&&&\\
705& 1410& \\
2128&&&&&&&&&\\
709\\
1418& \\
2152&&&&&&&&&\\
717& 1434& \\
2164&&&&&&&&&\\
721& 1442& \\
2176&&&&&&&&&\\
483& 725& 966& 1450& 1932\\
2224&&&&&&&&&\\
741& 1482& \\
2248&&&&&&&&&\\
295& 393& 443& 499& 590& 665& 749& 786\\
886& 998& 1180& 1330& 1498& 1572& 1772& 1996& 2360& \\
2260&&&&&&&&&\\
753\\
1506& \\
2308&&&&&&&&&\\
769& 1538& \\
2368&&&&&&&&&\\
789& 1578& \\
2416&&&&&&&&&\\
805& 1610& \\
2440&&&&&&&&&\\
813& 1626& \\
2452&&&&&&&&&\\
817\\
1634& \\
2464&&&&&&&&&\\
547& 729& 821& 1094& 1458& 1642& 2188& \\
2500&&&&&&&&&\\
555& 833\\
1110& 1666& 2220& \\
2512&&&&&&&&&\\
837& 1674& \\
2536&&&&&&&&&\\
375& 563& 750& 845& 1126\\
1500& 1690& 2252& \\
2560&&&&&&&&&\\
853& 1706& \\
2608&&&&&&&&&\\
579& 869& 1158& 1738& 2316\\
2632&&&&&&&&&\\
877& 1754& \\
2728&&&&&&&&&\\
909& 1818& \\
2752&&&&&&&&&\\
271& 361& 379& 407& 427& 481\\
505& 542& 569& 611& 641& 673& 722& 758& 814& 854\\
897& 917& 962& 1010& 1084& 1138& 1222& 1282& 1346& 1444\\
1516& 1628& 1708& 1794& 1834& 1924& 2020& 2168& 2276& \\
2836&&&&&&&&&\\
945\\
1890& \\
2848&&&&&&&&&\\
949& 1898& \\
2884&&&&&&&&&\\
961& 1922& \\
2896&&&&&&&&&\\
507& 571& 643& 761& 857\\
965& 1014& 1142& 1286& 1522& 1714& 1930& 2028& 2284& \\
2944&&&&&&&&&\\
981\\
1962& \\
2968&&&&&&&&&\\
439& 585& 659& 878& 989& 1170& 1318& 1756& 1978\\
2340& \\
2992&&&&&&&&&\\
997& 1994& \\
3076&&&&&&&&&\\
303& 455& 606& 683& 910& 1025& 1212\\
1366& 1820& 2050& \\
3136&&&&&&&&&\\
1045& 2090& \\
3208&&&&&&&&&\\
1069& 2138& \\
3220&&&&&&&&&\\
423& 635& 715\\
846& 953& 1073& 1270& 1430& 1692& 1906& 2146& \\
3232&&&&&&&&&\\
1077& 2154\\
3256&&&&&&&&&\\
723& 1085& 1446& 2170& \\
3268&&&&&&&&&\\
1089& 2178& \\
3280&&&&&&&&&\\
1093& 2186& \\
3328&&&&&&&&&\\
739& 985\\
1109& 1478& 1970& 2218& \\
3352&&&&&&&&&\\
1117& 2234& \\
3376&&&&&&&&&\\
1125& 2250& \\
3412&&&&&&&&&\\
1137& 2274\\
3472&&&&&&&&&\\
771& 1157& 1542& 2314& \\
3508&&&&&&&&&\\
519& 779& 1038& 1169& 1558& 2076\\
2338& \\
3544&&&&&&&&&\\
699& 787& 1049& 1181& 1398& 1574& 2098& 2362& \\
3616&&&&&&&&&\\
475\\
535& 633& 713& 803& 950& 1070& 1205& 1266& 1426& 1606\\
1900& 2140& \\
3640&&&&&&&&&\\
1213& \\
3664&&&&&&&&&\\
1221& \\
3688&&&&&&&&&\\
819& 1229& 1638& \\
3712&&&&&&&&&\\
1237& \\
3784&&&&&&&&&\\
1261& \\
3796&&&&&&&&&\\
843\\
1265& 1686& \\
3808&&&&&&&&&\\
1269& \\
3844&&&&&&&&&\\
1281& \\
3856&&&&&&&&&\\
1285& \\
3904&&&&&&&&&\\
867& 1301& 1734& \\
3940&&&&&&&&&\\
583& 777\\
875& 1166& 1313& 1554& 1750& 2332& \\
3952&&&&&&&&&\\
1317& \\
3976&&&&&&&&&\\
883& 1177& 1325\\
1766& 2354& \\
3988&&&&&&&&&\\
1329& \\
4096&&&&&&&&&\\
1365& \\
4180&&&&&&&&&\\
1393& \\
4192&&&&&&&&&\\
367& 489& 551& 734& 827\\
931& 978& 1102& 1241& 1397& 1468& 1654& 1862& 1956& 2204\\
4264&&&&&&&&&\\
631& 747& 841& 947& 1121& 1262& 1421& 1494& 1682& 1894\\
2242& \\
4276&&&&&&&&&\\
1425& \\
4288&&&&&&&&&\\
1429& \\
4336&&&&&&&&&\\
963& 1445& 1926& \\
4360&&&&&&&&&\\
1453& \\
4372&&&&&&&&&\\
127& 169& 191\\
225& 254& 287& 338& 339& 382& 431& 450& 451& 508\\
509& 574& 601& 647& 676& 677& 678& 764& 765& 801\\
862& 900& 901& 902& 971& 1016& 1018& 1148& 1149& 1201\\
1202& 1294& 1352& 1354& 1356& 1357& 1457& 1528& 1530& 1602\\
1724& 1800& 1802& 1804& 1942& 2032& 2036& 2296& 2298& \\
4408&&&&&&&&&\\
979\\
1305& 1469& 1958& \\
4432&&&&&&&&&\\
1477& \\
4456&&&&&&&&&\\
1485& \\
4468&&&&&&&&&\\
1489& \\
4480&&&&&&&&&\\
663& 995& 1326& 1493\\
1990& \\
4504&&&&&&&&&\\
1501& \\
4552&&&&&&&&&\\
1011& 1517& 2022& \\
4624&&&&&&&&&\\
1027& 1369& 1541& 2054& \\
4708&&&&&&&&&\\
1569\\
4720&&&&&&&&&\\
1573& \\
4768&&&&&&&&&\\
1059& 1589& 2118& \\
4804&&&&&&&&&\\
711& 1067& 1422& 1601& 2134& \\
4816&&&&&&&&&\\
1605\\
4840&&&&&&&&&\\
955& 1075& 1273& 1433& 1613& 1910& 2150& \\
4852&&&&&&&&&\\
1617& \\
4864&&&&&&&&&\\
1621& \\
4888&&&&&&&&&\\
1629\\
4912&&&&&&&&&\\
727& 969& 1091& 1454& 1637& 1938& 2182& \\
4948&&&&&&&&&\\
1099& 1465& 1649\\
2198& \\
5032&&&&&&&&&\\
1677& \\
5044&&&&&&&&&\\
1681& \\
5056&&&&&&&&&\\
1123& 1497& 1685& 2246& \\
5080&&&&&&&&&\\
1693& \\
5092&&&&&&&&&\\
1131& 1697\\
2262& \\
5128&&&&&&&&&\\
759& 1139& 1518& 1709& 2278& \\
5140&&&&&&&&&\\
1713& \\
5176&&&&&&&&&\\
1725& \\
5200&&&&&&&&&\\
1155& 1733\\
2310& \\
5224&&&&&&&&&\\
1741& \\
5272&&&&&&&&&\\
1171& 1561& 1757& 2342& \\
5296&&&&&&&&&\\
1765& \\
5320&&&&&&&&&\\
1773& \\
5416&&&&&&&&&\\
1203& 1805\\
5428&&&&&&&&&\\
1809& \\
5440&&&&&&&&&\\
1813& \\
5464&&&&&&&&&\\
1821& \\
5476&&&&&&&&&\\
1825& \\
5536&&&&&&&&&\\
1845& \\
5572&&&&&&&&&\\
1857& \\
5584&&&&&&&&&\\
1861& \\
5632&&&&&&&&&\\
1251& 1877& \\
5680&&&&&&&&&\\
1893\\
5704&&&&&&&&&\\
1267& 1689& 1901& \\
5716&&&&&&&&&\\
1905& \\
5728&&&&&&&&&\\
1909& \\
5776&&&&&&&&&\\
855& 1283& 1710& 1925& \\
5800&&&&&&&&&\\
1933\\
5812&&&&&&&&&\\
603& 679& 905& 1019& 1147& 1206& 1291& 1358& 1529& 1721\\
1810& 1937& 2038& 2294& \\
5860&&&&&&&&&\\
1953& \\
5872&&&&&&&&&\\
1957& \\
5908&&&&&&&&&\\
1969& \\
5920&&&&&&&&&\\
1315& 1753& 1973\\
5956&&&&&&&&&\\
1323& 1985& \\
5968&&&&&&&&&\\
1989& \\
5992&&&&&&&&&\\
591& 887& 1182& 1331& 1774& 1997& 2364\\
6004&&&&&&&&&\\
2001& \\
6064&&&&&&&&&\\
1347& 2021& \\
6088&&&&&&&&&\\
2029& \\
6112&&&&&&&&&\\
2037& \\
6160&&&&&&&&&\\
2053& \\
6232&&&&&&&&&\\
2077& \\
6244&&&&&&&&&\\
1387& 1849& 2081\\
6292&&&&&&&&&\\
2097& \\
6304&&&&&&&&&\\
2101& \\
6352&&&&&&&&&\\
1411& 1881& 2117& \\
6400&&&&&&&&&\\
2133& \\
6424&&&&&&&&&\\
951& 1427& 1902& 2141\\
6448&&&&&&&&&\\
2149& \\
6484&&&&&&&&&\\
2161& \\
6496&&&&&&&&&\\
1443& 2165& \\
6544&&&&&&&&&\\
2181& \\
6568&&&&&&&&&\\
1459& 1945& 2189& \\
6616&&&&&&&&&\\
2205& \\
6664&&&&&&&&&\\
2221\\
6688&&&&&&&&&\\
2229& \\
6724&&&&&&&&&\\
2241& \\
6736&&&&&&&&&\\
2245& \\
6760&&&&&&&&&\\
2253& \\
6772&&&&&&&&&\\
2257& \\
6784&&&&&&&&&\\
1339& 1507& 1785& 2009& 2261\\
6832&&&&&&&&&\\
2277& \\
6856&&&&&&&&&\\
1015& 1353& 1523& 2030& 2285& \\
6928&&&&&&&&&\\
1539& 2309& \\
6952&&&&&&&&&\\
2317& \\
6964&&&&&&&&&\\
687\\
1031& 1374& 1547& 2062& 2321& \\
7000&&&&&&&&&\\
1555& 2073& 2333& \\
7012&&&&&&&&&\\
2337& \\
7024&&&&&&&&&\\
2341\\
7060&&&&&&&&&\\
2353& \\
7096&&&&&&&&&\\
2365& \\
7168&&&&&&&&&\\
2389& \\
7216&&&&&&&&&\\
1603& 2137& \\
7252&&&&&&&&&\\
1611& \\
7360&&&&&&&&&\\
1635& \\
7396&&&&&&&&&\\
1095& 1643& 2190\\
7504&&&&&&&&&\\
987& 1111& 1481& 1667& 1974& 2222& \\
7576&&&&&&&&&\\
1683& \\
7720&&&&&&&&&\\
1143& 1715& 2286\\
7828&&&&&&&&&\\
1159& 1545& 1739& 2318& \\
8080&&&&&&&&&\\
559& 745& 839& 993& 1063& 1118\\
1259& 1417& 1490& 1491& 1595& 1678& 1795& 1889& 1986& 2126\\
2236& 2237& 2393& \\
8116&&&&&&&&&\\
1803& \\
8152&&&&&&&&&\\
1207& 1609& 1811& 2145& \\
8404&&&&&&&&&\\
1659& 1867\\
8512&&&&&&&&&\\
1891& \\
8548&&&&&&&&&\\
1899& \\
8584&&&&&&&&&\\
847& 891& 1003& 1129& 1271& 1337& 1505& 1694\\
1782& 1907& 2006& 2258& \\
8656&&&&&&&&&\\
1923& \\
8728&&&&&&&&&\\
1531& 1723& 1939& 2041& 2297\\
8872&&&&&&&&&\\
1971& \\
8944&&&&&&&&&\\
1987& \\
8980&&&&&&&&&\\
1995& \\
9232&&&&&&&&&\\
27& 31& 41& 47& 54& 55& 62\\
63& 71& 73& 82& 83& 91& 94& 95& 97& 103\\
107& 108& 109& 110& 111& 121& 124& 125& 126& 129\\
137& 142& 143& 145& 146& 147& 155& 159& 161& 164\\
165& 166& 167& 171& 175& 182& 183& 188& 189& 190\\
193& 194& 195& 199& 206& 207& 214& 215& 216& 218\\
220& 221& 222& 223& 231& 233& 235& 239& 242& 243\\
248& 250& 251& 252& 253& 257& 258& 259& 263& 265\\
274& 275& 283& 284& 285& 286& 290& 291& 292& 293\\
294& 297& 299& 310& 311& 313& 318& 319& 322& 323\\
327& 328& 330& 332& 333& 334& 335& 337& 342& 343\\
345& 347& 350& 351& 353& 359& 364& 365& 366& 376\\
377& 378& 380& 381& 386& 387& 388& 389& 390& 391\\
395& 398& 399& 411& 412& 413& 414& 415& 417& 425\\
428& 429& 430& 432& 436& 437& 440& 442& 444& 445\\
446& 449& 457& 459& 462& 463& 466& 467& 470& 471\\
478& 479& 484& 485& 486& 487& 491& 496& 500& 501\\
502& 503& 504& 506& 514& 515& 516& 517& 518& 521\\
523& 526& 527& 530& 531& 539& 543& 548& 549& 550\\
553& 566& 567& 568& 570& 572& 573& 580& 581& 582\\
584& 586& 587& 588& 589& 593& 594& 595& 598& 599\\
607& 609& 617& 619& 620& 621& 622& 623& 626& 627\\
636& 637& 638& 644& 645& 646& 649& 651& 654& 655\\
656& 660& 661& 664& 666& 668& 669& 670& 674& 675\\
684& 685& 686& 689& 690& 691& 694& 695& 697& 700\\
701& 702& 706& 707& 718& 719& 728& 730& 731& 732\\
733& 737& 752& 754& 755& 756& 757& 760& 762& 763\\
772& 773& 774& 775& 776& 778& 780& 781& 782& 783\\
785& 790& 791& 793& 796& 797& 798& 809& 811& 815\\
822& 823& 824& 825& 826& 828& 829& 830& 834& 835\\
849& 850& 851& 856& 858& 859& 860& 861& 864& 865\\
872& 873& 874& 880& 881& 884& 885& 888& 890& 892\\
893& 898& 899& 903& 911& 913& 914& 915& 918& 919\\
921& 924& 925& 926& 929& 932& 933& 934& 935& 939\\
940& 941& 942& 956& 957& 958& 967& 968& 970& 972\\
973& 974& 977& 982& 983& 992& 1000& 1002& 1004& 1005\\
1006& 1008& 1009& 1012& 1013& 1017& 1028& 1029& 1030& 1032\\
1033& 1034& 1035& 1036& 1037& 1041& 1042& 1043& 1046& 1047\\
1052& 1053& 1054& 1057& 1060& 1061& 1062& 1078& 1079& 1081\\
1083& 1086& 1096& 1097& 1098& 1100& 1101& 1105& 1106& 1107\\
1113& 1132& 1133& 1134& 1136& 1140& 1141& 1144& 1145& 1146\\
1153& 1160& 1162& 1163& 1164& 1165& 1168& 1172& 1173& 1174\\
1175& 1176& 1178& 1186& 1187& 1188& 1189& 1190& 1195& 1196\\
1197& 1198& 1214& 1217& 1218& 1219& 1223& 1225& 1227& 1233\\
1234& 1235& 1238& 1239& 1240& 1242& 1244& 1245& 1246& 1252\\
1253& 1254& 1272& 1274& 1276& 1277& 1287& 1288& 1289& 1290\\
1292& 1293& 1297& 1298& 1299& 1302& 1303& 1308& 1309& 1310\\
1312& 1320& 1322& 1328& 1332& 1333& 1336& 1338& 1340& 1341\\
1345& 1348& 1349& 1350& 1355& 1367& 1368& 1370& 1372& 1373\\
1377& 1378& 1379& 1380& 1381& 1382& 1388& 1389& 1390& 1394\\
1395& 1400& 1402& 1403& 1404& 1405& 1409& 1412& 1413& 1414\\
1419& 1436& 1437& 1438& 1441& 1451& 1456& 1460& 1461& 1462\\
1464& 1466& 1467& 1473& 1474& 1475& 1483& 1504& 1508& 1509\\
1510& 1512& 1514& 1520& 1521& 1524& 1525& 1526& 1537& 1544\\
1546& 1548& 1549& 1550& 1552& 1553& 1556& 1557& 1560& 1562\\
1564& 1565& 1566& 1570& 1571& 1579& 1580& 1581& 1582& 1585\\
1586& 1587& 1592& 1593& 1594& 1596& 1597& 1618& 1619& 1622\\
1625& 1630& 1633& 1644& 1645& 1646& 1648& 1650& 1651& 1652\\
1653& 1656& 1658& 1660& 1661& 1668& 1669& 1670& 1675& 1698\\
1699& 1700& 1701& 1702& 1707& 1712& 1716& 1717& 1718& 1720\\
1722& 1728& 1729& 1730& 1731& 1737& 1744& 1745& 1746& 1747\\
1748& 1749& 1760& 1762& 1763& 1768& 1770& 1776& 1777& 1780\\
1781& 1784& 1786& 1793& 1796& 1797& 1798& 1806& 1822& 1826\\
1827& 1828& 1829& 1830& 1835& 1836& 1837& 1838& 1841& 1842\\
1843& 1848& 1850& 1852& 1853& 1858& 1859& 1864& 1866& 1868\\
1869& 1870& 1873& 1878& 1880& 1882& 1884& 1885& 1912& 1914\\
1916& 1917& 1921& 1931& 1934& 1936& 1940& 1941& 1944& 1946\\
1948& 1949& 1954& 1955& 1964& 1965& 1966& 1977& 1984& 2000\\
2004& 2005& 2008& 2010& 2012& 2013& 2016& 2018& 2019& 2024\\
2026& 2033& 2034& 2035& 2049& 2051& 2056& 2058& 2060& 2061\\
2064& 2065& 2066& 2068& 2069& 2070& 2072& 2074& 2082& 2084\\
2085& 2086& 2092& 2093& 2094& 2104& 2105& 2106& 2108& 2109\\
2113& 2114& 2120& 2122& 2124& 2125& 2129& 2156& 2157& 2158\\
2162& 2166& 2172& 2173& 2177& 2192& 2193& 2194& 2196& 2197\\
2200& 2201& 2202& 2210& 2212& 2213& 2214& 2225& 2226& 2233\\
2264& 2265& 2266& 2268& 2269& 2272& 2280& 2282& 2288& 2289\\
2290& 2292& 2293& 2305& 2306& 2320& 2324& 2325& 2326& 2328\\
2329& 2330& 2336& 2344& 2346& 2348& 2349& 2350& 2352& 2356\\
2357& 2369& 2372& 2373& 2374& 2376& 2378& 2380& 2381& 2390\\
2392& 2394& 2396& 2397& \\
9268&&&&&&&&&\\
2059& \\
9304&&&&&&&&&\\
2067& \\
9376&&&&&&&&&\\
1851& 2083& \\
9412&&&&&&&&&\\
2091& \\
9448&&&&&&&&&\\
1243\\
1399& 1657& 1865& 1963& 2099& 2209& \\
9520&&&&&&&&&\\
2115& \\
9556&&&&&&&&&\\
943& 1257& 1415\\
1886& 2123& \\
9664&&&&&&&&&\\
1431& 2147& \\
9736&&&&&&&&&\\
2163& \\
9808&&&&&&&&&\\
2179& \\
9880&&&&&&&&&\\
975& 1463& 1950& 2195\\
9952&&&&&&&&&\\
2211& \\
9988&&&&&&&&&\\
1479& 2219& \\
10024&&&&&&&&&\\
879& 1319& 1758& 1979& 2227& \\
10096&&&&&&&&&\\
1495& 1771\\
1993& 2243& 2361& \\
10132&&&&&&&&&\\
2251& \\
10168&&&&&&&&&\\
2259& \\
10240&&&&&&&&&\\
2275& \\
10312&&&&&&&&&\\
1527& 2291& \\
10384&&&&&&&&&\\
2307& \\
10420&&&&&&&&&\\
1371\\
1543& 2057& 2315& \\
10528&&&&&&&&&\\
615& 923& 1039& 1230& 1385& 1559& 1846\\
2078& 2187& 2339& \\
10600&&&&&&&&&\\
2355& \\
10708&&&&&&&&&\\
2379& \\
10852&&&&&&&&&\\
1071& 1607& 2142& \\
10960&&&&&&&&&\\
1623& \\
10996&&&&&&&&&\\
1447\\
1929& 2171& \\
11176&&&&&&&&&\\
735& 1103& 1470& 1655& 2206& \\
11284&&&&&&&&&\\
1671& \\
11320&&&&&&&&&\\
2235& \\
11392&&&&&&&&&\\
999\\
1499& 1687& 1998& 2249& \\
11608&&&&&&&&&\\
1719& \\
11716&&&&&&&&&\\
1735& 2313& \\
11824&&&&&&&&&\\
1167& 1751& 2334\\
11968&&&&&&&&&\\
1575& 2363& \\
12040&&&&&&&&&\\
1783& 2377& \\
12148&&&&&&&&&\\
799& 1065& 1199& 1598& 1799& 2130\\
2131& 2398& \\
12472&&&&&&&&&\\
1231& 1641& 1847& \\
12688&&&&&&&&&\\
1879& \\
12904&&&&&&&&&\\
1911& \\
13012&&&&&&&&&\\
1927& 2283& \\
13120&&&&&&&&&\\
255\\
383& 510& 575& 766& 863& 907& 1020& 1021& 1150& 1209\\
1295& 1361& 1532& 1533& 1726& 1814& 1815& 1943& 2040& 2042\\
2300& 2301& \\
13336&&&&&&&&&\\
1755& 1975& \\
13444&&&&&&&&&\\
1179& 1327& 1769& 1991& 2358& \\
13552&&&&&&&&&\\
2007\\
13768&&&&&&&&&\\
1359& 2039& \\
13876&&&&&&&&&\\
2055& \\
13912&&&&&&&&&\\
1627& 1831& 2169& \\
13984&&&&&&&&&\\
2071& \\
14308&&&&&&&&&\\
495& 743& 990\\
1115& 1255& 1486& 1673& 1883& 1980& 1981& 2119& 2230& 2347\\
14416&&&&&&&&&\\
1423& 1897& 2135& \\
14560&&&&&&&&&\\
1135& 1513& 1703& 2017& 2270& \\
14632&&&&&&&&&\\
2167& \\
14740&&&&&&&&&\\
1455\\
2183& \\
14848&&&&&&&&&\\
2199& \\
15064&&&&&&&&&\\
991& 1321& 1487& 1761& 1982& 2231& \\
15172&&&&&&&&&\\
2247& \\
15280&&&&&&&&&\\
1191\\
1787& 2011& 2263& 2382& \\
15496&&&&&&&&&\\
2295& \\
15604&&&&&&&&&\\
2311& \\
15712&&&&&&&&&\\
1551& 2327& \\
15856&&&&&&&&&\\
927& 1391\\
1854& 2087& \\
15928&&&&&&&&&\\
1863& 2359& \\
16144&&&&&&&&&\\
2391& \\
17008&&&&&&&&&\\
1119& 1679& 2238& \\
17332&&&&&&&&&\\
1351& 1711\\
1801& 2027& 2281& \\
17800&&&&&&&&&\\
2343& \\
18196&&&&&&&&&\\
2395& \\
18304&&&&&&&&&\\
1807& \\
18628&&&&&&&&&\\
1839& \\
18772&&&&&&&&&\\
1647& \\
18952&&&&&&&&&\\
831& 1247\\
1662& 1871& \\
19600&&&&&&&&&\\
1935& \\
19924&&&&&&&&&\\
1311& 1967& \\
20248&&&&&&&&&\\
1999& \\
20896&&&&&&&&&\\
1375& 1833& 2063& \\
21220&&&&&&&&&\\
2095\\
21544&&&&&&&&&\\
2127& \\
21688&&&&&&&&&\\
667& 751& 889& 1001& 1127& 1185& 1334& 1335& 1502\\
1691& 1778& 1779& 1903& 2002& 2003& 2107& 2254& 2370& 2371\\
22192&&&&&&&&&\\
1947& 2191& \\
22660&&&&&&&&&\\
1767& \\
22840&&&&&&&&&\\
1503& 2255& \\
23056&&&&&&&&&\\
2023& \\
23488&&&&&&&&&\\
2319& \\
23812&&&&&&&&&\\
1567& 2089& 2351\\
24136&&&&&&&&&\\
2383& \\
24784&&&&&&&&&\\
1087& 1449& 1631& 2174& \\
25108&&&&&&&&&\\
2203& \\
25972&&&&&&&&&\\
1519& 2025& 2279& \\
26728&&&&&&&&&\\
1563\\
1759& 2345& \\
27700&&&&&&&&&\\
1215& 1823& \\
28672&&&&&&&&&\\
1887& \\
31588&&&&&&&&&\\
2079& \\
32560&&&&&&&&&\\
2143& \\
33352&&&&&&&&&\\
1951& \\
34504&&&&&&&&&\\
2271& \\
34720&&&&&&&&&\\
2031\\
36448&&&&&&&&&\\
1599& 2399& \\
39364&&&&&&&&&\\
447& 511& 671& 681& 767& 795& 807& 894\\
895& 1007& 1022& 1151& 1193& 1211& 1275& 1342& 1343& 1362\\
1363& 1435& 1511& 1534& 1590& 1591& 1614& 1615& 1727& 1788\\
1789& 1790& 1817& 1913& 2014& 2015& 2044& 2045& 2121& 2153\\
2267& 2302& 2385& 2386& 2387& \\
41524&&&&&&&&&\\
639& 959& 1278& 1439& 1918\\
2159& \\
42280&&&&&&&&&\\
1855& \\
44224&&&&&&&&&\\
2043& 2299& \\
45520&&&&&&&&&\\
1051& 1183& 1401& 1577& 1775& 2102\\
2103& 2366& \\
48904&&&&&&&&&\\
1695& \\
50848&&&&&&&&&\\
1983& \\
51028&&&&&&&&&\\
2239& \\
53944&&&&&&&&&\\
2367& \\
59776&&&&&&&&&\\
2331& \\
61720&&&&&&&&&\\
2139& \\
65608&&&&&&&&&\\
1279& 1515\\
1705& 1919& 2273& \\
74356&&&&&&&&&\\
2175& \\
80512&&&&&&&&&\\
1959& \\
95956&&&&&&&&&\\
1383& 1639& 1663& 2075& 2185\\
2217& 2335& \\
103336&&&&&&&&&\\
1791& \\
118096&&&&&&&&&\\
1023& 1535& 2046& 2303& \\
138400&&&&&&&&&\\
1263& 1895& \\
176740&&&&&&&&&\\
2151\\
190996&&&&&&&&&\\
871& 1161& 1307& 1471& 1742& 1743& 1961& 2207& 2322& 2323\\
250504&&&&&&&&&\\
703& 937& 1055& 1249& 1406& 1407& 1583& 1665& 1874& 1875\\
2110& 2111& 2223& 2287& 2375& \\
345544&&&&&&&&&\\
2215& \\
1276936&&&&&&&&&\\
1819& 1915& 2047& 2155\\
\bottomrule\end{longtable}

%\begin{longtable}{llllll}\toprule
\caption{Frequenza cicli}\\
\toprule
\textbf{n l} & \textbf{n l} & \textbf{n l} & \textbf{n l}& \textbf{n l} & \textbf{n l}\\
\midrule
\endfirsthead
\multicolumn{6}{c} {\tablename\ \thetable\ -- \textit{Continua dalla pagina precedente}} \\
\textbf{n l} & \textbf{n l} & \textbf{n l} & \textbf{n l}& \textbf{n l} & \textbf{n l}\\
\midrule
\endhead
\midrule
\multicolumn{6}{r} {\textit{Continua nella pagina successiva}} \\
\endfoot
\bottomrule
\endlastfoot
1  1&2 1&3 2&4 1&5 2&6 2\\
7  4&8 4&9 6&10 6&11 8&12 9\\
13  13&14 12&15 17&16 22&17 16&18 22\\
19  26&20 22&21 32&22 18&23 28&24 43\\
25  23&26 38&27 22&28 30&29 49&30 23\\
31  40&32 58&33 25&34 47&35 14&36 30\\
37  60&38 16&39 32&40 10&41 23&42 47\\
43  15&44 29&45 43&46 21&47 39&48 11\\
49  22&50 47&51 10&52 27&53 8&54 15\\
55  32&56 8&57 20&58 29&59 10&60 26\\
61  5&62 13&63 29&64 7&65 17&66 2\\
67  7&68 21&69 6&70 13&71 6&72 6\\
73  17&74 3&75 8&76 24&77 3&78 9\\
79  2&80 5&81 15&82 3&83 8&84 2\\
85  6&86 13&87 5&88 10&89 10&90 6\\
91  12&92 4&93 11&94 20&95 7&96 15\\
97  5&98 10&99 17&100 5&101 12&102 7\\
103  7&104 16&105 6&106 9&107 22&108 8\\
109  13&110 7&111 11&112 19&113 10&114 17\\
115  7&116 14&117 24&118 11&119 19&120 31\\
121  14&122 25&123 9&124 17&125 32&126 11\\
127  22&128 6&129 15&130 29&131 7&132 18\\
133  3&134 9&135 21&136 2&137 9&138 24\\
139  3&140 13&141 2&142 6&143 15&144 4\\
145  10&146 0&147 5&148 14&149 0&150 4\\
151  15&152 2&153 7&154 0&155 2&156 11\\
157  0&158 2&159 0&160 0&161 4&162 0\\
163  1&164 1&165 0&166 2&167 0&168 2\\
169  6&170 1&171 3&172 0&173 1&174 6\\
175  0&176 3&177 0&178 1&179 4&180 0\\
181  1&182 3&183 0&184 0&185 0&186 0\\
\bottomrule\end{longtable}
	 
\cleardoublepage	
\begin{appendices}
\input{formule}
\cleardoublepage
\nocite{*}
\addcontentsline{toc}{chapter}{\bibname}
\printbibliography
\cleardoublepage
\addcontentsline{toc}{chapter}{\indexname}
\printindex
\chapter{Mezzi usati}
\CDMezziUsati
\end{appendices}
\end{document}
