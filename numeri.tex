 % !TeX root = numeri.tex
% !BIB TS-program = biber
% !TeX encoding = UTF-8
% !TeX spellcheck = it_IT

\documentclass[a4paper,oneside]{book}%
\usepackage{cmap}
\usepackage[big]{layaureo}
\usepackage{copyright}
\frenchspacing%
\usepackage{amsmath}

\usepackage{amssymb}
\usepackage[italian]{babel}
\usepackage[thmmarks,hyperref]{ntheorem}
\usepackage{miamatematica}

\usepackage{lmodern} % load vector font
\usepackage[T1]{fontenc} % font encoding
\usepackage[utf8]{inputenc} % input encoding
%\usepackage{noto}
\usepackage[babel=true]{microtype}
%\usepackage{geometry}
\usepackage{textcomp}

%\geometry{top=1.5cm,bottom=1.5cm}
\usepackage{grafica}

%Teorema
\theoremstyle{marginbreak}
\theoremheaderfont{\normalfont\bfseries}
\theorembodyfont{\slshape}
\theoremsymbol{\ensuremath{\diamondsuit}}
\theoremseparator{:} %
\newtheorem{thm}{Teorema}[section]
%Proprietà
\theoremstyle{marginbreak}
\theoremheaderfont{\normalfont\bfseries}
\theorembodyfont{\slshape}
\theoremsymbol{\ensuremath{\diamondsuit}}
\theoremseparator{:}
\newtheorem{prop}{Proprietà}[section]
%lemma
\theoremstyle{changebreak}
\theoremsymbol{\ensuremath{\heartsuit}}
\theoremindent0.5cm
\theoremnumbering{greek}
\newtheorem{lem}[thm]{Lemma}
%corollario
\theoremindent0cm
\theoremsymbol{\ensuremath{\spadesuit}}
\theoremnumbering{arabic}
\newtheorem{cor}[thm]{Corollario}
%esempio
\theoremstyle{change}
\theorembodyfont{\upshape}
\theoremsymbol{\ensuremath{\ast}}
\theoremseparator{}
\newtheorem{exmp}{Esempio}[section]
%controesempio
\theoremstyle{change}
\theorembodyfont{\upshape}
\theoremsymbol{\ensuremath{\odot}}
\theoremseparator{}
\newtheorem{cexmp}{Contro esempio}[section]
%definizione
\theoremstyle{plain}
\theoremsymbol{\ensuremath{\clubsuit}}
\theoremseparator{.}
\theoremstyle{marginbreak}
%\theoremprework{\hrule\bigskip}
%\theorempostwork{\hrule\bigskip}
\newtheorem{defn}{Definizione}[section]
%commento
\theoremstyle{plain}
\theorembodyfont{\upshape}
\theoremsymbol{\ensuremath{\blacklozenge}}
\theoremseparator{:}
\newtheorem{commento}{Commento}
%dimostrazione

\theoremstyle{plain}
\theoremheaderfont{\sc}
\theorembodyfont{\bfseries}
\theoremstyle{nonumberplain}
%^{}\theoremseparator{.}

\theoremsymbol{\ensuremath{\blacksquare}}
\theoremheaderfont{\bfseries}
%\theoremstyle{nonumberplain}
%\theoremstyle{marginbreak}
\theorembodyfont{\normalfont}
\newtheorem{proof}{Dimostrazione}
%\input{../Mod_base/tabelle}

\usepackage{tabelle}

%\usepackage{adjustbox}
%\input{../Mod_base/stand_class}
\usepackage{pagina}

\setlength{\headheight}{13pt}
\usepackage{indice}
\usepackage{date}
\usepackage{unita_misura}

\usepackage{imakeidx}
\makeindex[options=-s ../Mod_base/oldclaudio.sti]
\newcolumntype{C}{>{\sffamily $}c<{$}}
%\newcolumntype{F}{>{$\displaystyle }c<{$}}
%\newcolumntype{L}{>{\sffamily $}l<{$}}
%\newcolumntype{M}[1]{>{\centering}p{#1}}
\newcolumntype{A}{r@{\hspace*{1mm}}r@{\hspace*{1mm}}r}
\newcolumntype{Q}{r@{\hspace*{1mm}}r}
%\newcolumntype{O}{>{\centering\arraybackslash}p{1.5em}} 
\newcolumntype{R}{>{\sffamily $}r<{$}}
%\newcolumntype{T}{>{\centering\arraybackslash}p{1em}}
%\newcolumntype{W}{>{\sffamily\Large $}c<{$}}
%\newcommand{\Co}{\numberset{C}}
\newcommand{\HRule}{\rule{\linewidth}{0.5mm}}
%\newlength{\gnat}
%\newlength{\gnam}
%\newcommand{\pilH}{\rule{0pt}{2.5ex}}
%\newcommand{\pilD}{\rule[-1ex]{0pt}{0pt}}
% 10/09/2017 :: 10:18:49 :: \usepackage{draftwatermark}
% 10/09/2017 :: 10:19:00 :: \SetWatermarkText{BOZZA}
\usepackage{altrebasitabelle}

\makeatletter
\renewcommand\frontmatter{%
	\cleardoublepage
	\@mainmatterfalse
	%\pagenumbering{roman}
}
\renewcommand\mainmatter{%
	\cleardoublepage
	\@mainmattertrue
	%\pagenumbering{arabic}
}
\makeatother
\usepackage{listings}
\lstdefinestyle{pascalstyle}{
	language=Pascal,
	commentstyle=\color{red},
	sensitive=false,
	%morecomment=[l][\color{red}]{//},
	morecomment=[l]{//},
}
%\usepackage{tkz-berge}
\usepackage[toc,page]{appendix}
\usepackage{stand_class}
\renewcommand{\appendixtocname}{Appendici}
\renewcommand{\appendixpagename}{Appendici}
\usepackage[autostyle,italian=guillemets]{csquotes}
\usepackage[%
style=philosophy-modern,
% 9/01/2018 :: 9:24:04 :: annotation=true,
hyperref,
backend=biber,
backref]{biblatex}
\addbibresource{numeri.bib}
\newcommand{\citaoeis}[1]{La successione è la successione \citetitle{#1}~\citeurl{#1} del OEIS}
\usepackage[grumpy,mark,markifdirty,raisemark=0.95\paperheight]{gitinfo2}
\usepackage{hyperxmp}
\usepackage{hyperref}
\title{Numeri}
\author{Claudio Duchi}
\date{\datetime}
\hypersetup{%
pdfencoding=auto,
urlcolor={blue},
pdftitle={Tavole},
pdfsubject={Tavole numeriche},
pdfstartview={FitH},
pdfpagemode={UseOutlines},
pdflicenseurl={http://creativecommons.org/licenses/by-nc-nd/3.0/},
pdflang={it},
pdfmetalang={it},
pdfkeywords={Numeri},
pdfcopyright={Copyright (C) 2021, Claudio Duchi},
pdfcontacturl={http://breviariomatematico.altervista.org},
pdfcontactpostcode={},
pdfcontactphone={},
pdfcontactemail={claduc},
pdfcontactcountry={Italy},
pdfcontactcity={Perugia},
pdfcontactaddress={},
pdfcaptionwriter={Claudio Duchi},
pdfauthortitle={},%
pdfauthor={Claudio Duchi},
linkcolor={blue},
colorlinks=true,
citecolor={red},
breaklinks,
bookmarksopen,
verbose,
baseurl={http://breviariomatematico.altervista.org}
}
\usepackage[italian]{varioref}
\usepackage[italian]{cleveref}
\includeonly{%
tabprimi,
tabprimiamici,	
tabella,
elencofattori,
tabPitagoriche,
ternepitagoriche2,
coprimi,
fibonacci,
fattoriali,
tartaglia,
numerifigurati,
quadratimagici,
CriteriDivisibilita,
formule
}
\usepackage{CDloghi}
\listfiles
 \begin{document}
		\frontmatter
		\hypersetup{pageanchor=false}
		\begin{titlepage}
			\begin{center}
			\Lgrandedue\\[1cm]
				\textsc{\LARGE Claudio Duchi}\\[1.5cm]
				\HRule \\[0.4cm]
				{ \huge \bfseries Numeri}\\[0.4cm]
				\HRule \\[1.5cm]
	\includestandalone[scale=0.9]{logonumeri}
		\end{center}
		\end{titlepage}
	\hypersetup{pageanchor=true}
	L'immagine di copertina è di \cite{PatrickT2014}\par
	\CDcopyright
		\tableofcontents
		\mainmatter


% !TeX encoding = UTF-8
% !TeX spellcheck = it_IT
\chapter{Tavole numeriche}
\section{Numeri primi}
\citaoeis{A000040}
\begin{longtable}{llllllllll} 
	\toprule
1	&	2	&	3	&	5	&	7	&	11	&	13	&	17	&	19	&	23	\\
29	&	31	&	37	&	41	&	43	&	47	&	53	&	59	&	61	&	67	\\
71	&	73	&	79	&	83	&	89	&	97	&	101	&	103	&	107	&	109	\\
113	&	127	&	131	&	137	&	139	&	149	&	151	&	157	&	163	&	167	\\
173	&	179	&	181	&	191	&	193	&	197	&	199	&	211	&	223	&	227	\\
229	&	233	&	239	&	241	&	251	&	257	&	263	&	269	&	271	&	277	\\
281	&	283	&	293	&	307	&	311	&	313	&	317	&	331	&	337	&	347	\\
349	&	353	&	359	&	367	&	373	&	379	&	383	&	389	&	397	&	401	\\
409	&	419	&	421	&	431	&	433	&	439	&	443	&	449	&	457	&	461	\\
463	&	467	&	479	&	487	&	491	&	499	&	503	&	509	&	521	&	523	\\
541	&	547	&	557	&	563	&	569	&	571	&	577	&	587	&	593	&	599	\\
601	&	607	&	613	&	617	&	619	&	631	&	641	&	643	&	647	&	653	\\
659	&	661	&	673	&	677	&	683	&	691	&	701	&	709	&	719	&	727	\\
733	&	739	&	743	&	751	&	757	&	761	&	769	&	773	&	787	&	797	\\
809	&	811	&	821	&	823	&	827	&	829	&	839	&	853	&	857	&	859	\\
863	&	877	&	881	&	883	&	887	&	907	&	911	&	919	&	929	&	937	\\
941	&	947	&	953	&	967	&	971	&	977	&	983	&	991	&	997	&	1009	\\
1013	&	1019	&	1021	&	1031	&	1033	&	1039	&	1049	&	1051	&	1061	&	1063	\\
1069	&	1087	&	1091	&	1093	&	1097	&	1103	&	1109	&	1117	&	1123	&	1129	\\
1151	&	1153	&	1163	&	1171	&	1181	&	1187	&	1193	&	1201	&	1213	&	1217	\\
1223	&	1229	&	1231	&	1237	&	1249	&	1259	&	1277	&	1279	&	1283	&	1289	\\
1291	&	1297	&	1301	&	1303	&	1307	&	1319	&	1321	&	1327	&	1361	&	1367	\\
1373	&	1381	&	1399	&	1409	&	1423	&	1427	&	1429	&	1433	&	1439	&	1447	\\
1451	&	1453	&	1459	&	1471	&	1481	&	1483	&	1487	&	1489	&	1493	&	1499	\\
1511	&	1523	&	1531	&	1543	&	1549	&	1553	&	1559	&	1567	&	1571	&	1579	\\
1583	&	1597	&	1601	&	1607	&	1609	&	1613	&	1619	&	1621	&	1627	&	1637	\\
1657	&	1663	&	1667	&	1669	&	1693	&	1697	&	1699	&	1709	&	1721	&	1723	\\
1733	&	1741	&	1747	&	1753	&	1759	&	1777	&	1783	&	1787	&	1789	&	1801	\\
1811	&	1823	&	1831	&	1847	&	1861	&	1867	&	1871	&	1873	&	1877	&	1879	\\
1889	&	1901	&	1907	&	1913	&	1931	&	1933	&	1949	&	1951	&	1973	&	1979	\\
1987	&	1993	&	1997	&	1999	&	2003	&	2011	&	2017	&	2027	&	2029	&	2039	\\
2053	&	2063	&	2069	&	2081	&	2083	&	2087	&	2089	&	2099	&	2111	&	2113	\\
2129	&	2131	&	2137	&	2141	&	2143	&	2153	&	2161	&	2179	&	2203	&	2207	\\
2213	&	2221	&	2237	&	2239	&	2243	&	2251	&	2267	&	2269	&	2273	&	2281	\\
2287	&	2293	&	2297	&	2309	&	2311	&	2333	&	2339	&	2341	&	2347	&	2351	\\
2357	&	2371	&	2377	&	2381	&	2383	&	2389	&	2393	&	2399	&	2411	&	2417	\\
2423	&	2437	&	2441	&	2447	&	2459	&	2467	&	2473	&	2477	&	2503	&	2521	\\
2531	&	2539	&	2543	&	2549	&	2551	&	2557	&	2579	&	2591	&	2593	&	2609	\\
2617	&	2621	&	2633	&	2647	&	2657	&	2659	&	2663	&	2671	&	2677	&	2683	\\
2687	&	2689	&	2693	&	2699	&	2707	&	2711	&	2713	&	2719	&	2729	&	2731	\\
2741	&	2749	&	2753	&	2767	&	2777	&	2789	&	2791	&	2797	&	2801	&	2803	\\
2819	&	2833	&	2837	&	2843	&	2851	&	2857	&	2861	&	2879	&	2887	&	2897	\\
2903	&	2909	&	2917	&	2927	&	2939	&	2953	&	2957	&	2963	&	2969	&	2971	\\
2999	&	3001	&	3011	&	3019	&	3023	&	3037	&	3041	&	3049	&	3061	&	3067	\\
3079	&	3083	&	3089	&	3109	&	3119	&	3121	&	3137	&	3163	&	3167	&	3169	\\
3181	&	3187	&	3191	&	3203	&	3209	&	3217	&	3221	&	3229	&	3251	&	3253	\\
3257	&	3259	&	3271	&	3299	&	3301	&	3307	&	3313	&	3319	&	3323	&	3329	\\
3331	&	3343	&	3347	&	3359	&	3361	&	3371	&	3373	&	3389	&	3391	&	3407	\\
3413	&	3433	&	3449	&	3457	&	3461	&	3463	&	3467	&	3469	&	3491	&	3499	\\
3511	&	3517	&	3527	&	3529	&	3533	&	3539	&	3541	&	3547	&	3557	&	3559	\\
3571	&	3581	&	3583	&	3593	&	3607	&	3613	&	3617	&	3623	&	3631	&	3637	\\
3643	&	3659	&	3671	&	3673	&	3677	&	3691	&	3697	&	3701	&	3709	&	3719	\\
3727	&	3733	&	3739	&	3761	&	3767	&	3769	&	3779	&	3793	&	3797	&	3803	\\
3821	&	3823	&	3833	&	3847	&	3851	&	3853	&	3863	&	3877	&	3881	&	3889	\\
3907	&	3911	&	3917	&	3919	&	3923	&	3929	&	3931	&	3943	&	3947	&	3967	\\
3989	&	4001	&	4003	&	4007	&	4013	&	4019	&	4021	&	4027	&	4049	&	4051	\\
4057	&	4073	&	4079	&	4091	&	4093	&	4099	&	4111	&	4127	&	4129	&	4133	\\
4139	&	4153	&	4157	&	4159	&	4177	&	4201	&	4211	&	4217	&	4219	&	4229	\\
4231	&	4241	&	4243	&	4253	&	4259	&	4261	&	4271	&	4273	&	4283	&	4289	\\
4297	&	4327	&	4337	&	4339	&	4349	&	4357	&	4363	&	4373	&	4391	&	4397	\\
4409	&	4421	&	4423	&	4441	&	4447	&	4451	&	4457	&	4463	&	4481	&	4483	\\
4493	&	4507	&	4513	&	4517	&	4519	&	4523	&	4547	&	4549	&	4561	&	4567	\\
4583	&	4591	&	4597	&	4603	&	4621	&	4637	&	4639	&	4643	&	4649	&	4651	\\
4657	&	4663	&	4673	&	4679	&	4691	&	4703	&	4721	&	4723	&	4729	&	4733	\\
4751	&	4759	&	4783	&	4787	&	4789	&	4793	&	4799	&	4801	&	4813	&	4817	\\
4831	&	4861	&	4871	&	4877	&	4889	&	4903	&	4909	&	4919	&	4931	&	4933	\\
4937	&	4943	&	4951	&	4957	&	4967	&	4969	&	4973	&	4987	&	4993	&	4999	\\
5003	&	5009	&	5011	&	5021	&	5023	&	5039	&	5051	&	5059	&	5077	&	5081	\\
5087	&	5099	&	5101	&	5107	&	5113	&	5119	&	5147	&	5153	&	5167	&	5171	\\
5179	&	5189	&	5197	&	5209	&	5227	&	5231	&	5233	&	5237	&	5261	&	5273	\\
5279	&	5281	&	5297	&	5303	&	5309	&	5323	&	5333	&	5347	&	5351	&	5381	\\
5387	&	5393	&	5399	&	5407	&	5413	&	5417	&	5419	&	5431	&	5437	&	5441	\\
5443	&	5449	&	5471	&	5477	&	5479	&	5483	&	5501	&	5503	&	5507	&	5519	\\
5521	&	5527	&	5531	&	5557	&	5563	&	5569	&	5573	&	5581	&	5591	&	5623	\\
5639	&	5641	&	5647	&	5651	&	5653	&	5657	&	5659	&	5669	&	5683	&	5689	\\
5693	&	5701	&	5711	&	5717	&	5737	&	5741	&	5743	&	5749	&	5779	&	5783	\\
5791	&	5801	&	5807	&	5813	&	5821	&	5827	&	5839	&	5843	&	5849	&	5851	\\
5857	&	5861	&	5867	&	5869	&	5879	&	5881	&	5897	&	5903	&	5923	&	5927	\\
5939	&	5953	&	5981	&	5987	&	6007	&	6011	&	6029	&	6037	&	6043	&	6047	\\
6053	&	6067	&	6073	&	6079	&	6089	&	6091	&	6101	&	6113	&	6121	&	6131	\\
6133	&	6143	&	6151	&	6163	&	6173	&	6197	&	6199	&	6203	&	6211	&	6217	\\
6221	&	6229	&	6247	&	6257	&	6263	&	6269	&	6271	&	6277	&	6287	&	6299	\\
6301	&	6311	&	6317	&	6323	&	6329	&	6337	&	6343	&	6353	&	6359	&	6361	\\
6367	&	6373	&	6379	&	6389	&	6397	&	6421	&	6427	&	6449	&	6451	&	6469	\\
6473	&	6481	&	6491	&	6521	&	6529	&	6547	&	6551	&	6553	&	6563	&	6569	\\
6571	&	6577	&	6581	&	6599	&	6607	&	6619	&	6637	&	6653	&	6659	&	6661	\\
6673	&	6679	&	6689	&	6691	&	6701	&	6703	&	6709	&	6719	&	6733	&	6737	\\
6761	&	6763	&	6779	&	6781	&	6791	&	6793	&	6803	&	6823	&	6827	&	6829	\\
6833	&	6841	&	6857	&	6863	&	6869	&	6871	&	6883	&	6899	&	6907	&	6911	\\
6917	&	6947	&	6949	&	6959	&	6961	&	6967	&	6971	&	6977	&	6983	&	6991	\\
6997	&	7001	&	7013	&	7019	&	7027	&	7039	&	7043	&	7057	&	7069	&	7079	\\
7103	&	7109	&	7121	&	7127	&	7129	&	7151	&	7159	&	7177	&	7187	&	7193	\\
7207	&	7211	&	7213	&	7219	&	7229	&	7237	&	7243	&	7247	&	7253	&	7283	\\
7297	&	7307	&	7309	&	7321	&	7331	&	7333	&	7349	&	7351	&	7369	&	7393	\\
7411	&	7417	&	7433	&	7451	&	7457	&	7459	&	7477	&	7481	&	7487	&	7489	\\
7499	&	7507	&	7517	&	7523	&	7529	&	7537	&	7541	&	7547	&	7549	&	7559	\\
7561	&	7573	&	7577	&	7583	&	7589	&	7591	&	7603	&	7607	&	7621	&	7639	\\
7643	&	7649	&	7669	&	7673	&	7681	&	7687	&	7691	&	7699	&	7703	&	7717	\\
7723	&	7727	&	7741	&	7753	&	7757	&	7759	&	7789	&	7793	&	7817	&	7823	\\
7829	&	7841	&	7853	&	7867	&	7873	&	7877	&	7879	&	7883	&	7901	&	7907	\\
7919	&	7927	&	7933	&	7937	&	7949	&	7951	&	7963	&	7993	&	8009	&	8011	\\
8017	&	8039	&	8053	&	8059	&	8069	&	8081	&	8087	&	8089	&	8093	&	8101	\\
8111	&	8117	&	8123	&	8147	&	8161	&	8167	&	8171	&	8179	&	8191	&	8209	\\
8219	&	8221	&	8231	&	8233	&	8237	&	8243	&	8263	&	8269	&	8273	&	8287	\\
8291	&	8293	&	8297	&	8311	&	8317	&	8329	&	8353	&	8363	&	8369	&	8377	\\
8387	&	8389	&	8419	&	8423	&	8429	&	8431	&	8443	&	8447	&	8461	&	8467	\\
8501	&	8513	&	8521	&	8527	&	8537	&	8539	&	8543	&	8563	&	8573	&	8581	\\
8597	&	8599	&	8609	&	8623	&	8627	&	8629	&	8641	&	8647	&	8663	&	8669	\\
8677	&	8681	&	8689	&	8693	&	8699	&	8707	&	8713	&	8719	&	8731	&	8737	\\
8741	&	8747	&	8753	&	8761	&	8779	&	8783	&	8803	&	8807	&	8819	&	8821	\\
8831	&	8837	&	8839	&	8849	&	8861	&	8863	&	8867	&	8887	&	8893	&	8923	\\
8929	&	8933	&	8941	&	8951	&	8963	&	8969	&	8971	&	8999	&	9001	&	9007	\\
9011	&	9013	&	9029	&	9041	&	9043	&	9049	&	9059	&	9067	&	9091	&	9103	\\
9109	&	9127	&	9133	&	9137	&	9151	&	9157	&	9161	&	9173	&	9181	&	9187	\\
9199	&	9203	&	9209	&	9221	&	9227	&	9239	&	9241	&	9257	&	9277	&	9281	\\
9283	&	9293	&	9311	&	9319	&	9323	&	9337	&	9341	&	9343	&	9349	&	9371	\\
9377	&	9391	&	9397	&	9403	&	9413	&	9419	&	9421	&	9431	&	9433	&	9437	\\
9439	&	9461	&	9463	&	9467	&	9473	&	9479	&	9491	&	9497	&	9511	&	9521	\\
9533	&	9539	&	9547	&	9551	&	9587	&	9601	&	9613	&	9619	&	9623	&	9629	\\
9631	&	9643	&	9649	&	9661	&	9677	&	9679	&	9689	&	9697	&	9719	&	9721	\\
9733	&	9739	&	9743	&	9749	&	9767	&	9769	&	9781	&	9787	&	9791	&	9803	\\
9811	&	9817	&	9829	&	9833	&	9839	&	9851	&	9857	&	9859	&	9871	&	9883	\\
9887	&	9901	&	9907	&	9923	&	9929	&	9931	&	9941	&	9949	&	9967	&	9973	\\
10007	&	10009	&	10037	&	10039	&	10061	&	10067	&	10069	&	10079	&	10091	&	10093	\\
10099	&	10103	&	10111	&	10133	&	10139	&	10141	&	10151	&	10159	&	10163	&	10169	\\
10177	&	10181	&	10193	&	10211	&	10223	&	10243	&	10247	&	10253	&	10259	&	10267	\\
10271	&	10273	&	10289	&	10301	&	10303	&	10313	&	10321	&	10331	&	10333	&	10337	\\
10343	&	10357	&	10369	&	10391	&	10399	&	10427	&	10429	&	10433	&	10453	&	10457	\\
10459	&	10463	&	10477	&	10487	&	10499	&	10501	&	10513	&	10529	&	10531	&	10559	\\
10567	&	10589	&	10597	&	10601	&	10607	&	10613	&	10627	&	10631	&	10639	&	10651	\\
10657	&	10663	&	10667	&	10687	&	10691	&	10709	&	10711	&	10723	&	10729	&	10733	\\
10739	&	10753	&	10771	&	10781	&	10789	&	10799	&	10831	&	10837	&	10847	&	10853	\\
10859	&	10861	&	10867	&	10883	&	10889	&	10891	&	10903	&	10909	&	10937	&	10939	\\
10949	&	10957	&	10973	&	10979	&	10987	&	10993	&	11003	&	11027	&	11047	&	11057	\\
11059	&	11069	&	11071	&	11083	&	11087	&	11093	&	11113	&	11117	&	11119	&	11131	\\
11149	&	11159	&	11161	&	11171	&	11173	&	11177	&	11197	&	11213	&	11239	&	11243	\\
11251	&	11257	&	11261	&	11273	&	11279	&	11287	&	11299	&	11311	&	11317	&	11321	\\
11329	&	11351	&	11353	&	11369	&	11383	&	11393	&	11399	&	11411	&	11423	&	11437	\\
11443	&	11447	&	11467	&	11471	&	11483	&	11489	&	11491	&	11497	&	11503	&	11519	\\
11527	&	11549	&	11551	&	11579	&	11587	&	11593	&	11597	&	11617	&	11621	&	11633	\\
11657	&	11677	&	11681	&	11689	&	11699	&	11701	&	11717	&	11719	&	11731	&	11743	\\
11777	&	11779	&	11783	&	11789	&	11801	&	11807	&	11813	&	11821	&	11827	&	11831	\\
11833	&	11839	&	11863	&	11867	&	11887	&	11897	&	11903	&	11909	&	11923	&	11927	\\
11933	&	11939	&	11941	&	11953	&	11959	&	11969	&	11971	&	11981	&	11987	&	12007	\\
12011	&	12037	&	12041	&	12043	&	12049	&	12071	&	12073	&	12097	&	12101	&	12107	\\
12109	&	12113	&	12119	&	12143	&	12149	&	12157	&	12161	&	12163	&	12197	&	12203	\\
12211	&	12227	&	12239	&	12241	&	12251	&	12253	&	12263	&	12269	&	12277	&	12281	\\
12289	&	12301	&	12323	&	12329	&	12343	&	12347	&	12373	&	12377	&	12379	&	12391	\\
12401	&	12409	&	12413	&	12421	&	12433	&	12437	&	12451	&	12457	&	12473	&	12479	\\
12487	&	12491	&	12497	&	12503	&	12511	&	12517	&	12527	&	12539	&	12541	&	12547	\\
12553	&	12569	&	12577	&	12583	&	12589	&	12601	&	12611	&	12613	&	12619	&	12637	\\
12641	&	12647	&	12653	&	12659	&	12671	&	12689	&	12697	&	12703	&	12713	&	12721	\\
12739	&	12743	&	12757	&	12763	&	12781	&	12791	&	12799	&	12809	&	12821	&	12823	\\
12829	&	12841	&	12853	&	12889	&	12893	&	12899	&	12907	&	12911	&	12917	&	12919	\\
12923	&	12941	&	12953	&	12959	&	12967	&	12973	&	12979	&	12983	&	13001	&	13003	\\
13007	&	13009	&	13033	&	13037	&	13043	&	13049	&	13063	&	13093	&	13099	&	13103	\\
13109	&	13121	&	13127	&	13147	&	13151	&	13159	&	13163	&	13171	&	13177	&	13183	\\
13187	&	13217	&	13219	&	13229	&	13241	&	13249	&	13259	&	13267	&	13291	&	13297	\\
13309	&	13313	&	13327	&	13331	&	13337	&	13339	&	13367	&	13381	&	13397	&	13399	\\
13411	&	13417	&	13421	&	13441	&	13451	&	13457	&	13463	&	13469	&	13477	&	13487	\\
13499	&	13513	&	13523	&	13537	&	13553	&	13567	&	13577	&	13591	&	13597	&	13613	\\
13619	&	13627	&	13633	&	13649	&	13669	&	13679	&	13681	&	13687	&	13691	&	13693	\\
13697	&	13709	&	13711	&	13721	&	13723	&	13729	&	13751	&	13757	&	13759	&	13763	\\
13781	&	13789	&	13799	&	13807	&	13829	&	13831	&	13841	&	13859	&	13873	&	13877	\\
13879	&	13883	&	13901	&	13903	&	13907	&	13913	&	13921	&	13931	&	13933	&	13963	\\
13967	&	13997	&	13999	&	14009	&	14011	&	14029	&	14033	&	14051	&	14057	&	14071	\\
14081	&	14083	&	14087	&	14107	&	14143	&	14149	&	14153	&	14159	&	14173	&	14177	\\
14197	&	14207	&	14221	&	14243	&	14249	&	14251	&	14281	&	14293	&	14303	&	14321	\\
14323	&	14327	&	14341	&	14347	&	14369	&	14387	&	14389	&	14401	&	14407	&	14411	\\
14419	&	14423	&	14431	&	14437	&	14447	&	14449	&	14461	&	14479	&	14489	&	14503	\\
14519	&	14533	&	14537	&	14543	&	14549	&	14551	&	14557	&	14561	&	14563	&	14591	\\
14593	&	14621	&	14627	&	14629	&	14633	&	14639	&	14653	&	14657	&	14669	&	14683	\\
14699	&	14713	&	14717	&	14723	&	14731	&	14737	&	14741	&	14747	&	14753	&	14759	\\
14767	&	14771	&	14779	&	14783	&	14797	&	14813	&	14821	&	14827	&	14831	&	14843	\\
14851	&	14867	&	14869	&	14879	&	14887	&	14891	&	14897	&	14923	&	14929	&	14939	\\
14947	&	14951	&	14957	&	14969	&	14983	&	15013									\\
	\bottomrule 
\end{longtable}	
% 14/01/2018 :: 14:29:47 :: \section{Numeri primi gemelli}
\label{sec:TabellaNumeriPrimigemelli}
\citaoeis{A077800}
\begin{center}
\footnotesize
\begin{tabular}{*9{Q@{\hspace*{3mm}}Q}} %{*{20}{r}}
\toprule 
3&5&5&7&11&13&17&19&29&31&41&43&59&61&71&73&101&103&107&109\\
137&139&149&151&179&181&191&193&197&199&227&229&239&241&269&271&281&283&311&313\\
347&349&419&421&431&433&461&463&521&523&569&571&599&601&617&619&641&643&659&661\\
809&811&821&823&827&829&857&859&881&883&1019&1021&1031&1033&1049&1051&1061&1063&1091&1093\\
1151&1153&1229&1231&1277&1279&1289&1291&1301&1303&1319&1321&1427&1429&1451&1453&1481&1483&1487&1489\\
1607&1609&1619&1621&1667&1669&1697&1699&1721&1723&1787&1789&1871&1873&1877&1879&1931&1933&1949&1951\\
1997&1999&2027&2029&2081&2083&2549&2551&2591&2593&2657&2659&2687&2689&2711&2713&2729&2731&2789&2791\\
\bottomrule
\end{tabular}\captionof{table}{Numeri primi gemelli}\index{Numero!primo!gemello}
\end{center}
\section{Numeri primi cugini}
\citaoeis{A094343}
\begin{center}
\footnotesize
\begin{tabular}{*9{Q@{\hspace*{3mm}}Q}} %{*{20}{r}}
\toprule 
3&7&7&11&13&17&19&23&37&41&43&47&67&71&79&83&97&101&103&107\\
109&113&127&131&163&167&193&197&223&227&229&233&277&281&307&311&313&317&349&353\\
379&383&397&401&439&443&457&461&463&467&487&491&499&503&613&617&643&647&673&677\\
739&743&757&761&769&773&823&827&853&857&859&863&877&881&883&887&907&911&937&941\\
967&971&1009&1013&1087&1091&1093&1097&1213&1217&1279&1283&1297&1301&1303&1307&1423&1427&1429&1433\\
1447&1451&1483&1487&1489&1493&1549&1553&1567&1571&1579&1583&1597&1601&1609&1613&1663&1667&1693&1697\\
1783&1787&1867&1871&1873&1877&2377&2381&2389&2393&2437&2441&2473&2477&2539&2543&2617&2621&2659&2663\\
2683&2687&2689&2693&2707&2711&2749&2753&2797&2801&2833&2837&2857&2861\\
\bottomrule
\end{tabular}\captionof{table}{Numeri primi cugini}\index{Numero!primo!cugino}
\end{center}
\section{Numeri primi sexy}
\citaoeis{A023201}
\begin{center}
\footnotesize
\begin{tabular}{*9{Q@{\hspace*{3mm}}Q}} %{*{20}{r}}
\toprule 
1&7&5&11&7&13&11&17&13&19&17&23&23&29&31&37&37&43&41&47\\
47&53&53&59&61&67&67&73&73&79&83&89&97&103&101&107&103&109&107&113\\
131&137&151&157&157&163&167&173&173&179&191&197&193&199&223&229&227&233&233&239\\
251&257&257&263&263&269&271&277&277&283&307&313&311&317&331&337&347&353&353&359\\
367&373&373&379&383&389&433&439&443&449&457&463&461&467&503&509&541&547&557&563\\
563&569&571&577&587&593&593&599&601&607&607&613&613&619&641&647&647&653&653&659\\
677&683&727&733&733&739&751&757&821&827&823&829&853&859&857&863&877&883&881&887\\
941&947&947&953&971&977&977&983&991&997&1013&1019&1033&1039&1063&1069&1087&1093&1091&1097\\
1097&1103&1103&1109&1117&1123&1123&1129&1181&1187&1187&1193&1217&1223&1223&1229&1231&1237&1277&1283\\
1283&1289&1291&1297&1297&1303&1301&1307&1321&1327&1361&1367&1367&1373&1423&1429&1427&1433&1433&1439\\
1447&1453&1453&1459&1481&1487&1483&1489&1487&1493&1493&1499&1543&1549&1553&1559&1601&1607&1607&1613\\
1613&1619&1621&1627&1657&1663&1663&1669&1693&1699&1741&1747&1747&1753&1753&1759&1777&1783&1783&1789\\
1861&1867&1867&1873&1871&1877&1873&1879&1901&1907&1907&1913&1973&1979&1987&1993&1993&1999&1997&2003\\
2011&2017&2063&2069&2081&2087&2083&2089&2131&2137&2137&2143&2207&2213&2237&2243&2267&2273&2281&2287\\
2287&2293&2333&2339&2341&2347&2351&2357&2371&2377&2377&2383&2383&2389&2393&2399&2411&2417&2417&2423\\
2441&2447&2467&2473&2543&2549&2551&2557&2657&2663&2671&2677&2677&2683&2683&2689&2687&2693&2693&2699\\
2707&2713&2713&2719&2791&2797&2797&2803&2837&2843&2851&2857&\\ 
\bottomrule
\end{tabular}\captionof{table}{Numeri primi sexy}\index{Numero!primo!sexy}
\end{center}
\section{Terzine di primi sexy}
\begin{center}
\footnotesize
\begin{tabular}{*3{A@{\hspace*{5mm}}A}} % 14/01/2018 :: 9:39:57 :: {*{5}{r}}
\toprule 
1&7&13&5&11&17&7&13&19&11&17&23&17&23&29\\
31&37&43&41&47&53&47&53&59&61&67&73&67&73&79\\
97&103&109&101&107&113&151&157&163&167&173&179&227&233&239\\
251&257&263&257&263&269&271&277&283&347&353&359&367&373&379\\
557&563&569&587&593&599&601&607&613&607&613&619&641&647&653\\
647&653&659&727&733&739&941&947&953&971&977&983&1091&1097&1103\\
1097&1103&1109&1117&1123&1129&1181&1187&1193&1217&1223&1229&1277&1283&1289\\
1291&1297&1303&1361&1367&1373&1427&1433&1439&1447&1453&1459&1481&1487&1493\\
1487&1493&1499&1601&1607&1613&1607&1613&1619&1657&1663&1669&1741&1747&1753\\
1747&1753&1759&1777&1783&1789&1861&1867&1873&1867&1873&1879&1901&1907&1913\\
1987&1993&1999&2131&2137&2143&2281&2287&2293&2371&2377&2383&2377&2383&2389\\
2411&2417&2423&2671&2677&2683&2677&2683&2689&2687&2693&2699&2707&2713&2719\\
2791&2797&2803\\ 
\bottomrule
\end{tabular}\captionof{table}{Terzine di primi sexy}\index{Numero!primo!terzine sexy}
\end{center}
\section{Terzine di primi}
\begin{center}
\footnotesize
\begin{tabular}{*4{A@{\hspace*{5mm}}A}} 
\toprule
5&7&11&7&11&13&11&13&17&13&17&19&17&19&23\\
37&41&43&41&43&47&67&71&73&97&101&103&101&103&107\\
103&107&109&107&109&113&191&193&197&193&197&199&223&227&229\\
227&229&233&277&281&283&307&311&313&311&313&317&347&349&353\\
457&461&463&461&463&467&613&617&619&641&643&647&821&823&827\\
823&827&829&853&857&859&857&859&863&877&881&883&881&883&887\\
1087&1091&1093&1091&1093&1097&1277&1279&1283&1297&1301&1303&1301&1303&1307\\
1423&1427&1429&1427&1429&1433&1447&1451&1453&1481&1483&1487&1483&1487&1489\\
1487&1489&1493&1607&1609&1613&1663&1667&1669&1693&1697&1699&1783&1787&1789\\
1867&1871&1873&1871&1873&1877&1873&1877&1879&1993&1997&1999&1997&1999&2003\\
2081&2083&2087&2083&2087&2089&2137&2141&2143&2237&2239&2243&2267&2269&2273\\
2377&2381&2383&2657&2659&2663&2683&2687&2689&2687&2689&2693&2707&2711&2713\\
2797&2801&2803\\ 
\bottomrule
\end{tabular}\captionof{table}{Terzine di primi}\index{Numero!primo!terzine}
\end{center}
\section{Quartine di primi sexy}
\begin{center}
	\footnotesize
	\begin{tabular}{*{5}{r}}
		\toprule 
		1 7 13 19&5 11 17 23&11 17 23 29&41 47 53 59&61 67 73 79\\
		251 257 263 269&601 607 613 619&641 647 653 659&1091 1097 1103 1109&1481 1487 1493 1499\\
		1601 1607 1613 1619&1741 1747 1753 1759&1861 1867 1873 1879&2371 2377 2383 2389&2671 2677 2683 2689\\ 
		\bottomrule
	\end{tabular}\captionof{table}{Quartine di primi sexy}\index{Numero!primo!quartine sexy}
\end{center}
\section{Quartine di primi}
\begin{center}
\footnotesize
\begin{tabular}{*{5}{r}}
\toprule 
 5 7 11 13&11 13 17 19&101 103 107 109&191 193 197 199&821 823 827 829\\
 1481 1483 1487 1489&1871 1873 1877 1879&2081 2083 2087 2089\\
\bottomrule
\end{tabular}\captionof{table}{Quartine di primi}\index{Numero!primo!quartine}
\end{center}	
% !TeX encoding = UTF-8
% !TeX spellcheck = it_IT

\section{Quadrati cubi radici}
\label{sec:Tabellaquadraticubiradici}
\citaoeis{A000290}. \citaoeis{A000578}
\begin{longtable}{rrrrrrrrrrr} 
	\toprule
	\bfseries $n$ &  $n^2$ & $n^3$&$\sqrt{n}$&$\sqrt[3]{n}$& &$n$ &  $n^2$ & $n^3$&$\sqrt{n}$&$\sqrt[3]{n}$  \\
	\midrule \endhead
	\bottomrule \endfoot\index{Tabella!quadrati}\index{Tabella!cubi}\index{Tabella!radici}
1&1&1&1,0000&1,0000&&51&2601&132651&7,1414&3,7084\\
2&4&8&1,4142&1,2599&&52&2704&140608&7,2111&3,7325\\
3&9&27&1,7321&1,4422&&53&2809&148877&7,2801&3,7563\\
4&16&64&2,0000&1,5874&&54&2916&157464&7,3485&3,7798\\
5&25&125&2,2361&1,7100&&55&3025&166375&7,4162&3,8030\\
6&36&216&2,4495&1,8171&&56&3136&175616&7,4833&3,8259\\
7&49&343&2,6458&1,9129&&57&3249&185193&7,5498&3,8485\\
8&64&512&2,8284&2,0000&&58&3364&195112&7,6158&3,8709\\
9&81&729&3,0000&2,0801&&59&3481&205379&7,6811&3,8930\\
10&100&1000&3,1623&2,1544&&60&3600&216000&7,7460&3,9149\\
11&121&1331&3,3166&2,2240&&61&3721&226981&7,8102&3,9365\\
12&144&1728&3,4641&2,2894&&62&3844&238328&7,8740&3,9579\\
13&169&2197&3,6056&2,3513&&63&3969&250047&7,9373&3,9791\\
14&196&2744&3,7417&2,4101&&64&4096&262144&8,0000&4,0000\\
15&225&3375&3,8730&2,4662&&65&4225&274625&8,0623&4,0207\\
16&256&4096&4,0000&2,5198&&66&4356&287496&8,1240&4,0412\\
17&289&4913&4,1231&2,5713&&67&4489&300763&8,1854&4,0615\\
18&324&5832&4,2426&2,6207&&68&4624&314432&8,2462&4,0817\\
19&361&6859&4,3589&2,6684&&69&4761&328509&8,3066&4,1016\\
20&400&8000&4,4721&2,7144&&70&4900&343000&8,3666&4,1213\\
21&441&9261&4,5826&2,7589&&71&5041&357911&8,4261&4,1408\\
22&484&10648&4,6904&2,8020&&72&5184&373248&8,4853&4,1602\\
23&529&12167&4,7958&2,8439&&73&5329&389017&8,5440&4,1793\\
24&576&13824&4,8990&2,8845&&74&5476&405224&8,6023&4,1983\\
25&625&15625&5,0000&2,9240&&75&5625&421875&8,6603&4,2172\\
26&676&17576&5,0990&2,9625&&76&5776&438976&8,7178&4,2358\\
27&729&19683&5,1962&3,0000&&77&5929&456533&8,7750&4,2543\\
28&784&21952&5,2915&3,0366&&78&6084&474552&8,8318&4,2727\\
29&841&24389&5,3852&3,0723&&79&6241&493039&8,8882&4,2908\\
30&900&27000&5,4772&3,1072&&80&6400&512000&8,9443&4,3089\\
31&961&29791&5,5678&3,1414&&81&6561&531441&9,0000&4,3267\\
32&1024&32768&5,6569&3,1748&&82&6724&551368&9,0554&4,3445\\
33&1089&35937&5,7446&3,2075&&83&6889&571787&9,1104&4,3621\\
34&1156&39304&5,8310&3,2396&&84&7056&592704&9,1652&4,3795\\
35&1225&42875&5,9161&3,2711&&85&7225&614125&9,2195&4,3968\\
36&1296&46656&6,0000&3,3019&&86&7396&636056&9,2736&4,4140\\
37&1369&50653&6,0828&3,3322&&87&7569&658503&9,3274&4,4310\\
38&1444&54872&6,1644&3,3620&&88&7744&681472&9,3808&4,4480\\
39&1521&59319&6,2450&3,3912&&89&7921&704969&9,4340&4,4647\\
40&1600&64000&6,3246&3,4200&&90&8100&729000&9,4868&4,4814\\
41&1681&68921&6,4031&3,4482&&91&8281&753571&9,5394&4,4979\\
42&1764&74088&6,4807&3,4760&&92&8464&778688&9,5917&4,5144\\
43&1849&79507&6,5574&3,5034&&93&8649&804357&9,6437&4,5307\\
44&1936&85184&6,6332&3,5303&&94&8836&830584&9,6954&4,5468\\
45&2025&91125&6,7082&3,5569&&95&9025&857375&9,7468&4,5629\\
46&2116&97336&6,7823&3,5830&&96&9216&884736&9,7980&4,5789\\
47&2209&103823&6,8557&3,6088&&97&9409&912673&9,8489&4,5947\\
48&2304&110592&6,9282&3,6342&&98&9604&941192&9,8995&4,6104\\
49&2401&117649&7,0000&3,6593&&99&9801&970299&9,9499&4,6261\\
50&2500&125000&7,0711&3,6840&&100&10000&1000000&10,0000&4,6416\\
\newpage
101&10201&1030301&10,0499&4,6570&&151&22801&3442951&12,2882&5,3251\\
102&10404&1061208&10,0995&4,6723&&152&23104&3511808&12,3288&5,3368\\
103&10609&1092727&10,1489&4,6875&&153&23409&3581577&12,3693&5,3485\\
104&10816&1124864&10,1980&4,7027&&154&23716&3652264&12,4097&5,3601\\
105&11025&1157625&10,2470&4,7177&&155&24025&3723875&12,4499&5,3717\\
106&11236&1191016&10,2956&4,7326&&156&24336&3796416&12,4900&5,3832\\
107&11449&1225043&10,3441&4,7475&&157&24649&3869893&12,5300&5,3947\\
108&11664&1259712&10,3923&4,7622&&158&24964&3944312&12,5698&5,4061\\
109&11881&1295029&10,4403&4,7769&&159&25281&4019679&12,6095&5,4175\\
110&12100&1331000&10,4881&4,7914&&160&25600&4096000&12,6491&5,4288\\
111&12321&1367631&10,5357&4,8059&&161&25921&4173281&12,6886&5,4401\\
112&12544&1404928&10,5830&4,8203&&162&26244&4251528&12,7279&5,4514\\
113&12769&1442897&10,6301&4,8346&&163&26569&4330747&12,7671&5,4626\\
114&12996&1481544&10,6771&4,8488&&164&26896&4410944&12,8062&5,4737\\
115&13225&1520875&10,7238&4,8629&&165&27225&4492125&12,8452&5,4848\\
116&13456&1560896&10,7703&4,8770&&166&27556&4574296&12,8841&5,4959\\
117&13689&1601613&10,8167&4,8910&&167&27889&4657463&12,9228&5,5069\\
118&13924&1643032&10,8628&4,9049&&168&28224&4741632&12,9615&5,5178\\
119&14161&1685159&10,9087&4,9187&&169&28561&4826809&13,0000&5,5288\\
120&14400&1728000&10,9545&4,9324&&170&28900&4913000&13,0384&5,5397\\
121&14641&1771561&11,0000&4,9461&&171&29241&5000211&13,0767&5,5505\\
122&14884&1815848&11,0454&4,9597&&172&29584&5088448&13,1149&5,5613\\
123&15129&1860867&11,0905&4,9732&&173&29929&5177717&13,1529&5,5721\\
124&15376&1906624&11,1355&4,9866&&174&30276&5268024&13,1909&5,5828\\
125&15625&1953125&11,1803&5,0000&&175&30625&5359375&13,2288&5,5934\\
126&15876&2000376&11,2250&5,0133&&176&30976&5451776&13,2665&5,6041\\
127&16129&2048383&11,2694&5,0265&&177&31329&5545233&13,3041&5,6147\\
128&16384&2097152&11,3137&5,0397&&178&31684&5639752&13,3417&5,6252\\
129&16641&2146689&11,3578&5,0528&&179&32041&5735339&13,3791&5,6357\\
130&16900&2197000&11,4018&5,0658&&180&32400&5832000&13,4164&5,6462\\
131&17161&2248091&11,4455&5,0788&&181&32761&5929741&13,4536&5,6567\\
132&17424&2299968&11,4891&5,0916&&182&33124&6028568&13,4907&5,6671\\
133&17689&2352637&11,5326&5,1045&&183&33489&6128487&13,5277&5,6774\\
134&17956&2406104&11,5758&5,1172&&184&33856&6229504&13,5647&5,6877\\
135&18225&2460375&11,6190&5,1299&&185&34225&6331625&13,6015&5,6980\\
136&18496&2515456&11,6619&5,1426&&186&34596&6434856&13,6382&5,7083\\
137&18769&2571353&11,7047&5,1551&&187&34969&6539203&13,6748&5,7185\\
138&19044&2628072&11,7473&5,1676&&188&35344&6644672&13,7113&5,7287\\
139&19321&2685619&11,7898&5,1801&&189&35721&6751269&13,7477&5,7388\\
140&19600&2744000&11,8322&5,1925&&190&36100&6859000&13,7840&5,7489\\
141&19881&2803221&11,8743&5,2048&&191&36481&6967871&13,8203&5,7590\\
142&20164&2863288&11,9164&5,2171&&192&36864&7077888&13,8564&5,7690\\
143&20449&2924207&11,9583&5,2293&&193&37249&7189057&13,8924&5,7790\\
144&20736&2985984&12,0000&5,2415&&194&37636&7301384&13,9284&5,7890\\
145&21025&3048625&12,0416&5,2536&&195&38025&7414875&13,9642&5,7989\\
146&21316&3112136&12,0830&5,2656&&196&38416&7529536&14,0000&5,8088\\
147&21609&3176523&12,1244&5,2776&&197&38809&7645373&14,0357&5,8186\\
148&21904&3241792&12,1655&5,2896&&198&39204&7762392&14,0712&5,8285\\
149&22201&3307949&12,2066&5,3015&&199&39601&7880599&14,1067&5,8383\\
150&22500&3375000&12,2474&5,3133&&200&40000&8000000&14,1421&5,8480\\
\newpage
201&40401&8120601&14,1774&5,8578&&251&63001&15813251&15,8430&6,3080\\
202&40804&8242408&14,2127&5,8675&&252&63504&16003008&15,8745&6,3164\\
203&41209&8365427&14,2478&5,8771&&253&64009&16194277&15,9060&6,3247\\
204&41616&8489664&14,2829&5,8868&&254&64516&16387064&15,9374&6,3330\\
205&42025&8615125&14,3178&5,8964&&255&65025&16581375&15,9687&6,3413\\
206&42436&8741816&14,3527&5,9059&&256&65536&16777216&16,0000&6,3496\\
207&42849&8869743&14,3875&5,9155&&257&66049&16974593&16,0312&6,3579\\
208&43264&8998912&14,4222&5,9250&&258&66564&17173512&16,0624&6,3661\\
209&43681&9129329&14,4568&5,9345&&259&67081&17373979&16,0935&6,3743\\
210&44100&9261000&14,4914&5,9439&&260&67600&17576000&16,1245&6,3825\\
211&44521&9393931&14,5258&5,9533&&261&68121&17779581&16,1555&6,3907\\
212&44944&9528128&14,5602&5,9627&&262&68644&17984728&16,1864&6,3988\\
213&45369&9663597&14,5945&5,9721&&263&69169&18191447&16,2173&6,4070\\
214&45796&9800344&14,6287&5,9814&&264&69696&18399744&16,2481&6,4151\\
215&46225&9938375&14,6629&5,9907&&265&70225&18609625&16,2788&6,4232\\
216&46656&10077696&14,6969&6,0000&&266&70756&18821096&16,3095&6,4312\\
217&47089&10218313&14,7309&6,0092&&267&71289&19034163&16,3401&6,4393\\
218&47524&10360232&14,7648&6,0185&&268&71824&19248832&16,3707&6,4473\\
219&47961&10503459&14,7986&6,0277&&269&72361&19465109&16,4012&6,4553\\
220&48400&10648000&14,8324&6,0368&&270&72900&19683000&16,4317&6,4633\\
221&48841&10793861&14,8661&6,0459&&271&73441&19902511&16,4621&6,4713\\
222&49284&10941048&14,8997&6,0550&&272&73984&20123648&16,4924&6,4792\\
223&49729&11089567&14,9332&6,0641&&273&74529&20346417&16,5227&6,4872\\
224&50176&11239424&14,9666&6,0732&&274&75076&20570824&16,5529&6,4951\\
225&50625&11390625&15,0000&6,0822&&275&75625&20796875&16,5831&6,5030\\
226&51076&11543176&15,0333&6,0912&&276&76176&21024576&16,6132&6,5108\\
227&51529&11697083&15,0665&6,1002&&277&76729&21253933&16,6433&6,5187\\
228&51984&11852352&15,0997&6,1091&&278&77284&21484952&16,6733&6,5265\\
229&52441&12008989&15,1327&6,1180&&279&77841&21717639&16,7033&6,5343\\
230&52900&12167000&15,1658&6,1269&&280&78400&21952000&16,7332&6,5421\\
231&53361&12326391&15,1987&6,1358&&281&78961&22188041&16,7631&6,5499\\
232&53824&12487168&15,2315&6,1446&&282&79524&22425768&16,7929&6,5577\\
233&54289&12649337&15,2643&6,1534&&283&80089&22665187&16,8226&6,5654\\
234&54756&12812904&15,2971&6,1622&&284&80656&22906304&16,8523&6,5731\\
235&55225&12977875&15,3297&6,1710&&285&81225&23149125&16,8819&6,5808\\
236&55696&13144256&15,3623&6,1797&&286&81796&23393656&16,9115&6,5885\\
237&56169&13312053&15,3948&6,1885&&287&82369&23639903&16,9411&6,5962\\
238&56644&13481272&15,4272&6,1972&&288&82944&23887872&16,9706&6,6039\\
239&57121&13651919&15,4596&6,2058&&289&83521&24137569&17,0000&6,6115\\
240&57600&13824000&15,4919&6,2145&&290&84100&24389000&17,0294&6,6191\\
241&58081&13997521&15,5242&6,2231&&291&84681&24642171&17,0587&6,6267\\
242&58564&14172488&15,5563&6,2317&&292&85264&24897088&17,0880&6,6343\\
243&59049&14348907&15,5885&6,2403&&293&85849&25153757&17,1172&6,6419\\
244&59536&14526784&15,6205&6,2488&&294&86436&25412184&17,1464&6,6494\\
245&60025&14706125&15,6525&6,2573&&295&87025&25672375&17,1756&6,6569\\
246&60516&14886936&15,6844&6,2658&&296&87616&25934336&17,2047&6,6644\\
247&61009&15069223&15,7162&6,2743&&297&88209&26198073&17,2337&6,6719\\
248&61504&15252992&15,7480&6,2828&&298&88804&26463592&17,2627&6,6794\\
249&62001&15438249&15,7797&6,2912&&299&89401&26730899&17,2916&6,6869\\
250&62500&15625000&15,8114&6,2996&&300&90000&27000000&17,3205&6,6943\\
\newpage
301&90601&27270901&17,3494&6,7018&&351&123201&43243551&18,7350&7,0540\\
302&91204&27543608&17,3781&6,7092&&352&123904&43614208&18,7617&7,0607\\
303&91809&27818127&17,4069&6,7166&&353&124609&43986977&18,7883&7,0674\\
304&92416&28094464&17,4356&6,7240&&354&125316&44361864&18,8149&7,0740\\
305&93025&28372625&17,4642&6,7313&&355&126025&44738875&18,8414&7,0807\\
306&93636&28652616&17,4929&6,7387&&356&126736&45118016&18,8680&7,0873\\
307&94249&28934443&17,5214&6,7460&&357&127449&45499293&18,8944&7,0940\\
308&94864&29218112&17,5499&6,7533&&358&128164&45882712&18,9209&7,1006\\
309&95481&29503629&17,5784&6,7606&&359&128881&46268279&18,9473&7,1072\\
310&96100&29791000&17,6068&6,7679&&360&129600&46656000&18,9737&7,1138\\
311&96721&30080231&17,6352&6,7752&&361&130321&47045881&19,0000&7,1204\\
312&97344&30371328&17,6635&6,7824&&362&131044&47437928&19,0263&7,1269\\
313&97969&30664297&17,6918&6,7897&&363&131769&47832147&19,0526&7,1335\\
314&98596&30959144&17,7200&6,7969&&364&132496&48228544&19,0788&7,1400\\
315&99225&31255875&17,7482&6,8041&&365&133225&48627125&19,1050&7,1466\\
316&99856&31554496&17,7764&6,8113&&366&133956&49027896&19,1311&7,1531\\
317&100489&31855013&17,8045&6,8185&&367&134689&49430863&19,1572&7,1596\\
318&101124&32157432&17,8326&6,8256&&368&135424&49836032&19,1833&7,1661\\
319&101761&32461759&17,8606&6,8328&&369&136161&50243409&19,2094&7,1726\\
320&102400&32768000&17,8885&6,8399&&370&136900&50653000&19,2354&7,1791\\
321&103041&33076161&17,9165&6,8470&&371&137641&51064811&19,2614&7,1855\\
322&103684&33386248&17,9444&6,8541&&372&138384&51478848&19,2873&7,1920\\
323&104329&33698267&17,9722&6,8612&&373&139129&51895117&19,3132&7,1984\\
324&104976&34012224&18,0000&6,8683&&374&139876&52313624&19,3391&7,2048\\
325&105625&34328125&18,0278&6,8753&&375&140625&52734375&19,3649&7,2112\\
326&106276&34645976&18,0555&6,8824&&376&141376&53157376&19,3907&7,2177\\
327&106929&34965783&18,0831&6,8894&&377&142129&53582633&19,4165&7,2240\\
328&107584&35287552&18,1108&6,8964&&378&142884&54010152&19,4422&7,2304\\
329&108241&35611289&18,1384&6,9034&&379&143641&54439939&19,4679&7,2368\\
330&108900&35937000&18,1659&6,9104&&380&144400&54872000&19,4936&7,2432\\
331&109561&36264691&18,1934&6,9174&&381&145161&55306341&19,5192&7,2495\\
332&110224&36594368&18,2209&6,9244&&382&145924&55742968&19,5448&7,2558\\
333&110889&36926037&18,2483&6,9313&&383&146689&56181887&19,5704&7,2622\\
334&111556&37259704&18,2757&6,9382&&384&147456&56623104&19,5959&7,2685\\
335&112225&37595375&18,3030&6,9451&&385&148225&57066625&19,6214&7,2748\\
336&112896&37933056&18,3303&6,9521&&386&148996&57512456&19,6469&7,2811\\
337&113569&38272753&18,3576&6,9589&&387&149769&57960603&19,6723&7,2874\\
338&114244&38614472&18,3848&6,9658&&388&150544&58411072&19,6977&7,2936\\
339&114921&38958219&18,4120&6,9727&&389&151321&58863869&19,7231&7,2999\\
340&115600&39304000&18,4391&6,9795&&390&152100&59319000&19,7484&7,3061\\
341&116281&39651821&18,4662&6,9864&&391&152881&59776471&19,7737&7,3124\\
342&116964&40001688&18,4932&6,9932&&392&153664&60236288&19,7990&7,3186\\
343&117649&40353607&18,5203&7,0000&&393&154449&60698457&19,8242&7,3248\\
344&118336&40707584&18,5472&7,0068&&394&155236&61162984&19,8494&7,3310\\
345&119025&41063625&18,5742&7,0136&&395&156025&61629875&19,8746&7,3372\\
346&119716&41421736&18,6011&7,0203&&396&156816&62099136&19,8997&7,3434\\
347&120409&41781923&18,6279&7,0271&&397&157609&62570773&19,9249&7,3496\\
348&121104&42144192&18,6548&7,0338&&398&158404&63044792&19,9499&7,3558\\
349&121801&42508549&18,6815&7,0406&&399&159201&63521199&19,9750&7,3619\\
350&122500&42875000&18,7083&7,0473&&400&160000&64000000&20,0000&7,3681\\
\newpage
401&160801&64481201&20,0250&7,3742&&451&203401&91733851&21,2368&7,6688\\
402&161604&64964808&20,0499&7,3803&&452&204304&92345408&21,2603&7,6744\\
403&162409&65450827&20,0749&7,3864&&453&205209&92959677&21,2838&7,6801\\
404&163216&65939264&20,0998&7,3925&&454&206116&93576664&21,3073&7,6857\\
405&164025&66430125&20,1246&7,3986&&455&207025&94196375&21,3307&7,6914\\
406&164836&66923416&20,1494&7,4047&&456&207936&94818816&21,3542&7,6970\\
407&165649&67419143&20,1742&7,4108&&457&208849&95443993&21,3776&7,7026\\
408&166464&67917312&20,1990&7,4169&&458&209764&96071912&21,4009&7,7082\\
409&167281&68417929&20,2237&7,4229&&459&210681&96702579&21,4243&7,7138\\
410&168100&68921000&20,2485&7,4290&&460&211600&97336000&21,4476&7,7194\\
411&168921&69426531&20,2731&7,4350&&461&212521&97972181&21,4709&7,7250\\
412&169744&69934528&20,2978&7,4410&&462&213444&98611128&21,4942&7,7306\\
413&170569&70444997&20,3224&7,4470&&463&214369&99252847&21,5174&7,7362\\
414&171396&70957944&20,3470&7,4530&&464&215296&99897344&21,5407&7,7418\\
415&172225&71473375&20,3715&7,4590&&465&216225&100544625&21,5639&7,7473\\
416&173056&71991296&20,3961&7,4650&&466&217156&101194696&21,5870&7,7529\\
417&173889&72511713&20,4206&7,4710&&467&218089&101847563&21,6102&7,7584\\
418&174724&73034632&20,4450&7,4770&&468&219024&102503232&21,6333&7,7639\\
419&175561&73560059&20,4695&7,4829&&469&219961&103161709&21,6564&7,7695\\
420&176400&74088000&20,4939&7,4889&&470&220900&103823000&21,6795&7,7750\\
421&177241&74618461&20,5183&7,4948&&471&221841&104487111&21,7025&7,7805\\
422&178084&75151448&20,5426&7,5007&&472&222784&105154048&21,7256&7,7860\\
423&178929&75686967&20,5670&7,5067&&473&223729&105823817&21,7486&7,7915\\
424&179776&76225024&20,5913&7,5126&&474&224676&106496424&21,7715&7,7970\\
425&180625&76765625&20,6155&7,5185&&475&225625&107171875&21,7945&7,8025\\
426&181476&77308776&20,6398&7,5244&&476&226576&107850176&21,8174&7,8079\\
427&182329&77854483&20,6640&7,5302&&477&227529&108531333&21,8403&7,8134\\
428&183184&78402752&20,6882&7,5361&&478&228484&109215352&21,8632&7,8188\\
429&184041&78953589&20,7123&7,5420&&479&229441&109902239&21,8861&7,8243\\
430&184900&79507000&20,7364&7,5478&&480&230400&110592000&21,9089&7,8297\\
431&185761&80062991&20,7605&7,5537&&481&231361&111284641&21,9317&7,8352\\
432&186624&80621568&20,7846&7,5595&&482&232324&111980168&21,9545&7,8406\\
433&187489&81182737&20,8087&7,5654&&483&233289&112678587&21,9773&7,8460\\
434&188356&81746504&20,8327&7,5712&&484&234256&113379904&22,0000&7,8514\\
435&189225&82312875&20,8567&7,5770&&485&235225&114084125&22,0227&7,8568\\
436&190096&82881856&20,8806&7,5828&&486&236196&114791256&22,0454&7,8622\\
437&190969&83453453&20,9045&7,5886&&487&237169&115501303&22,0681&7,8676\\
438&191844&84027672&20,9284&7,5944&&488&238144&116214272&22,0907&7,8730\\
439&192721&84604519&20,9523&7,6001&&489&239121&116930169&22,1133&7,8784\\
440&193600&85184000&20,9762&7,6059&&490&240100&117649000&22,1359&7,8837\\
441&194481&85766121&21,0000&7,6117&&491&241081&118370771&22,1585&7,8891\\
442&195364&86350888&21,0238&7,6174&&492&242064&119095488&22,1811&7,8944\\
443&196249&86938307&21,0476&7,6232&&493&243049&119823157&22,2036&7,8998\\
444&197136&87528384&21,0713&7,6289&&494&244036&120553784&22,2261&7,9051\\
445&198025&88121125&21,0950&7,6346&&495&245025&121287375&22,2486&7,9105\\
446&198916&88716536&21,1187&7,6403&&496&246016&122023936&22,2711&7,9158\\
447&199809&89314623&21,1424&7,6460&&497&247009&122763473&22,2935&7,9211\\
448&200704&89915392&21,1660&7,6517&&498&248004&123505992&22,3159&7,9264\\
449&201601&90518849&21,1896&7,6574&&499&249001&124251499&22,3383&7,9317\\
450&202500&91125000&21,2132&7,6631&&500&250000&125000000&22,3607&7,9370\\
\newpage
501&251001&125751501&22,3830&7,9423&&551&303601&167284151&23,4734&8,1982\\
502&252004&126506008&22,4054&7,9476&&552&304704&168196608&23,4947&8,2031\\
503&253009&127263527&22,4277&7,9528&&553&305809&169112377&23,5160&8,2081\\
504&254016&128024064&22,4499&7,9581&&554&306916&170031464&23,5372&8,2130\\
505&255025&128787625&22,4722&7,9634&&555&308025&170953875&23,5584&8,2180\\
506&256036&129554216&22,4944&7,9686&&556&309136&171879616&23,5797&8,2229\\
507&257049&130323843&22,5167&7,9739&&557&310249&172808693&23,6008&8,2278\\
508&258064&131096512&22,5389&7,9791&&558&311364&173741112&23,6220&8,2327\\
509&259081&131872229&22,5610&7,9843&&559&312481&174676879&23,6432&8,2377\\
510&260100&132651000&22,5832&7,9896&&560&313600&175616000&23,6643&8,2426\\
511&261121&133432831&22,6053&7,9948&&561&314721&176558481&23,6854&8,2475\\
512&262144&134217728&22,6274&8,0000&&562&315844&177504328&23,7065&8,2524\\
513&263169&135005697&22,6495&8,0052&&563&316969&178453547&23,7276&8,2573\\
514&264196&135796744&22,6716&8,0104&&564&318096&179406144&23,7487&8,2621\\
515&265225&136590875&22,6936&8,0156&&565&319225&180362125&23,7697&8,2670\\
516&266256&137388096&22,7156&8,0208&&566&320356&181321496&23,7908&8,2719\\
517&267289&138188413&22,7376&8,0260&&567&321489&182284263&23,8118&8,2768\\
518&268324&138991832&22,7596&8,0311&&568&322624&183250432&23,8328&8,2816\\
519&269361&139798359&22,7816&8,0363&&569&323761&184220009&23,8537&8,2865\\
520&270400&140608000&22,8035&8,0415&&570&324900&185193000&23,8747&8,2913\\
521&271441&141420761&22,8254&8,0466&&571&326041&186169411&23,8956&8,2962\\
522&272484&142236648&22,8473&8,0517&&572&327184&187149248&23,9165&8,3010\\
523&273529&143055667&22,8692&8,0569&&573&328329&188132517&23,9374&8,3059\\
524&274576&143877824&22,8910&8,0620&&574&329476&189119224&23,9583&8,3107\\
525&275625&144703125&22,9129&8,0671&&575&330625&190109375&23,9792&8,3155\\
526&276676&145531576&22,9347&8,0723&&576&331776&191102976&24,0000&8,3203\\
527&277729&146363183&22,9565&8,0774&&577&332929&192100033&24,0208&8,3251\\
528&278784&147197952&22,9783&8,0825&&578&334084&193100552&24,0416&8,3300\\
529&279841&148035889&23,0000&8,0876&&579&335241&194104539&24,0624&8,3348\\
530&280900&148877000&23,0217&8,0927&&580&336400&195112000&24,0832&8,3396\\
531&281961&149721291&23,0434&8,0978&&581&337561&196122941&24,1039&8,3443\\
532&283024&150568768&23,0651&8,1028&&582&338724&197137368&24,1247&8,3491\\
533&284089&151419437&23,0868&8,1079&&583&339889&198155287&24,1454&8,3539\\
534&285156&152273304&23,1084&8,1130&&584&341056&199176704&24,1661&8,3587\\
535&286225&153130375&23,1301&8,1180&&585&342225&200201625&24,1868&8,3634\\
536&287296&153990656&23,1517&8,1231&&586&343396&201230056&24,2074&8,3682\\
537&288369&154854153&23,1733&8,1281&&587&344569&202262003&24,2281&8,3730\\
538&289444&155720872&23,1948&8,1332&&588&345744&203297472&24,2487&8,3777\\
539&290521&156590819&23,2164&8,1382&&589&346921&204336469&24,2693&8,3825\\
540&291600&157464000&23,2379&8,1433&&590&348100&205379000&24,2899&8,3872\\
541&292681&158340421&23,2594&8,1483&&591&349281&206425071&24,3105&8,3919\\
542&293764&159220088&23,2809&8,1533&&592&350464&207474688&24,3311&8,3967\\
543&294849&160103007&23,3024&8,1583&&593&351649&208527857&24,3516&8,4014\\
544&295936&160989184&23,3238&8,1633&&594&352836&209584584&24,3721&8,4061\\
545&297025&161878625&23,3452&8,1683&&595&354025&210644875&24,3926&8,4108\\
546&298116&162771336&23,3666&8,1733&&596&355216&211708736&24,4131&8,4155\\
547&299209&163667323&23,3880&8,1783&&597&356409&212776173&24,4336&8,4202\\
548&300304&164566592&23,4094&8,1833&&598&357604&213847192&24,4540&8,4249\\
549&301401&165469149&23,4307&8,1882&&599&358801&214921799&24,4745&8,4296\\
550&302500&166375000&23,4521&8,1932&&600&360000&216000000&24,4949&8,4343\\
\newpage
601&361201&217081801&24,5153&8,4390&&651&423801&275894451&25,5147&8,6668\\
602&362404&218167208&24,5357&8,4437&&652&425104&277167808&25,5343&8,6713\\
603&363609&219256227&24,5561&8,4484&&653&426409&278445077&25,5539&8,6757\\
604&364816&220348864&24,5764&8,4530&&654&427716&279726264&25,5734&8,6801\\
605&366025&221445125&24,5967&8,4577&&655&429025&281011375&25,5930&8,6845\\
606&367236&222545016&24,6171&8,4623&&656&430336&282300416&25,6125&8,6890\\
607&368449&223648543&24,6374&8,4670&&657&431649&283593393&25,6320&8,6934\\
608&369664&224755712&24,6577&8,4716&&658&432964&284890312&25,6515&8,6978\\
609&370881&225866529&24,6779&8,4763&&659&434281&286191179&25,6710&8,7022\\
610&372100&226981000&24,6982&8,4809&&660&435600&287496000&25,6905&8,7066\\
611&373321&228099131&24,7184&8,4856&&661&436921&288804781&25,7099&8,7110\\
612&374544&229220928&24,7386&8,4902&&662&438244&290117528&25,7294&8,7154\\
613&375769&230346397&24,7588&8,4948&&663&439569&291434247&25,7488&8,7198\\
614&376996&231475544&24,7790&8,4994&&664&440896&292754944&25,7682&8,7241\\
615&378225&232608375&24,7992&8,5040&&665&442225&294079625&25,7876&8,7285\\
616&379456&233744896&24,8193&8,5086&&666&443556&295408296&25,8070&8,7329\\
617&380689&234885113&24,8395&8,5132&&667&444889&296740963&25,8263&8,7373\\
618&381924&236029032&24,8596&8,5178&&668&446224&298077632&25,8457&8,7416\\
619&383161&237176659&24,8797&8,5224&&669&447561&299418309&25,8650&8,7460\\
620&384400&238328000&24,8998&8,5270&&670&448900&300763000&25,8844&8,7503\\
621&385641&239483061&24,9199&8,5316&&671&450241&302111711&25,9037&8,7547\\
622&386884&240641848&24,9399&8,5362&&672&451584&303464448&25,9230&8,7590\\
623&388129&241804367&24,9600&8,5408&&673&452929&304821217&25,9422&8,7634\\
624&389376&242970624&24,9800&8,5453&&674&454276&306182024&25,9615&8,7677\\
625&390625&244140625&25,0000&8,5499&&675&455625&307546875&25,9808&8,7721\\
626&391876&245314376&25,0200&8,5544&&676&456976&308915776&26,0000&8,7764\\
627&393129&246491883&25,0400&8,5590&&677&458329&310288733&26,0192&8,7807\\
628&394384&247673152&25,0599&8,5635&&678&459684&311665752&26,0384&8,7850\\
629&395641&248858189&25,0799&8,5681&&679&461041&313046839&26,0576&8,7893\\
630&396900&250047000&25,0998&8,5726&&680&462400&314432000&26,0768&8,7937\\
631&398161&251239591&25,1197&8,5772&&681&463761&315821241&26,0960&8,7980\\
632&399424&252435968&25,1396&8,5817&&682&465124&317214568&26,1151&8,8023\\
633&400689&253636137&25,1595&8,5862&&683&466489&318611987&26,1343&8,8066\\
634&401956&254840104&25,1794&8,5907&&684&467856&320013504&26,1534&8,8109\\
635&403225&256047875&25,1992&8,5952&&685&469225&321419125&26,1725&8,8152\\
636&404496&257259456&25,2190&8,5997&&686&470596&322828856&26,1916&8,8194\\
637&405769&258474853&25,2389&8,6043&&687&471969&324242703&26,2107&8,8237\\
638&407044&259694072&25,2587&8,6088&&688&473344&325660672&26,2298&8,8280\\
639&408321&260917119&25,2784&8,6132&&689&474721&327082769&26,2488&8,8323\\
640&409600&262144000&25,2982&8,6177&&690&476100&328509000&26,2679&8,8366\\
641&410881&263374721&25,3180&8,6222&&691&477481&329939371&26,2869&8,8408\\
642&412164&264609288&25,3377&8,6267&&692&478864&331373888&26,3059&8,8451\\
643&413449&265847707&25,3574&8,6312&&693&480249&332812557&26,3249&8,8493\\
644&414736&267089984&25,3772&8,6357&&694&481636&334255384&26,3439&8,8536\\
645&416025&268336125&25,3969&8,6401&&695&483025&335702375&26,3629&8,8578\\
646&417316&269586136&25,4165&8,6446&&696&484416&337153536&26,3818&8,8621\\
647&418609&270840023&25,4362&8,6490&&697&485809&338608873&26,4008&8,8663\\
648&419904&272097792&25,4558&8,6535&&698&487204&340068392&26,4197&8,8706\\
649&421201&273359449&25,4755&8,6579&&699&488601&341532099&26,4386&8,8748\\
650&422500&274625000&25,4951&8,6624&&700&490000&343000000&26,4575&8,8790\\
\newpage
701&491401&344472101&26,4764&8,8833&&751&564001&423564751&27,4044&9,0896\\
702&492804&345948408&26,4953&8,8875&&752&565504&425259008&27,4226&9,0937\\
703&494209&347428927&26,5141&8,8917&&753&567009&426957777&27,4408&9,0977\\
704&495616&348913664&26,5330&8,8959&&754&568516&428661064&27,4591&9,1017\\
705&497025&350402625&26,5518&8,9001&&755&570025&430368875&27,4773&9,1057\\
706&498436&351895816&26,5707&8,9043&&756&571536&432081216&27,4955&9,1098\\
707&499849&353393243&26,5895&8,9085&&757&573049&433798093&27,5136&9,1138\\
708&501264&354894912&26,6083&8,9127&&758&574564&435519512&27,5318&9,1178\\
709&502681&356400829&26,6271&8,9169&&759&576081&437245479&27,5500&9,1218\\
710&504100&357911000&26,6458&8,9211&&760&577600&438976000&27,5681&9,1258\\
711&505521&359425431&26,6646&8,9253&&761&579121&440711081&27,5862&9,1298\\
712&506944&360944128&26,6833&8,9295&&762&580644&442450728&27,6043&9,1338\\
713&508369&362467097&26,7021&8,9337&&763&582169&444194947&27,6225&9,1378\\
714&509796&363994344&26,7208&8,9378&&764&583696&445943744&27,6405&9,1418\\
715&511225&365525875&26,7395&8,9420&&765&585225&447697125&27,6586&9,1458\\
716&512656&367061696&26,7582&8,9462&&766&586756&449455096&27,6767&9,1498\\
717&514089&368601813&26,7769&8,9503&&767&588289&451217663&27,6948&9,1537\\
718&515524&370146232&26,7955&8,9545&&768&589824&452984832&27,7128&9,1577\\
719&516961&371694959&26,8142&8,9587&&769&591361&454756609&27,7308&9,1617\\
720&518400&373248000&26,8328&8,9628&&770&592900&456533000&27,7489&9,1657\\
721&519841&374805361&26,8514&8,9670&&771&594441&458314011&27,7669&9,1696\\
722&521284&376367048&26,8701&8,9711&&772&595984&460099648&27,7849&9,1736\\
723&522729&377933067&26,8887&8,9752&&773&597529&461889917&27,8029&9,1775\\
724&524176&379503424&26,9072&8,9794&&774&599076&463684824&27,8209&9,1815\\
725&525625&381078125&26,9258&8,9835&&775&600625&465484375&27,8388&9,1855\\
726&527076&382657176&26,9444&8,9876&&776&602176&467288576&27,8568&9,1894\\
727&528529&384240583&26,9629&8,9918&&777&603729&469097433&27,8747&9,1933\\
728&529984&385828352&26,9815&8,9959&&778&605284&470910952&27,8927&9,1973\\
729&531441&387420489&27,0000&9,0000&&779&606841&472729139&27,9106&9,2012\\
730&532900&389017000&27,0185&9,0041&&780&608400&474552000&27,9285&9,2052\\
731&534361&390617891&27,0370&9,0082&&781&609961&476379541&27,9464&9,2091\\
732&535824&392223168&27,0555&9,0123&&782&611524&478211768&27,9643&9,2130\\
733&537289&393832837&27,0740&9,0164&&783&613089&480048687&27,9821&9,2170\\
734&538756&395446904&27,0924&9,0205&&784&614656&481890304&28,0000&9,2209\\
735&540225&397065375&27,1109&9,0246&&785&616225&483736625&28,0179&9,2248\\
736&541696&398688256&27,1293&9,0287&&786&617796&485587656&28,0357&9,2287\\
737&543169&400315553&27,1477&9,0328&&787&619369&487443403&28,0535&9,2326\\
738&544644&401947272&27,1662&9,0369&&788&620944&489303872&28,0713&9,2365\\
739&546121&403583419&27,1846&9,0410&&789&622521&491169069&28,0891&9,2404\\
740&547600&405224000&27,2029&9,0450&&790&624100&493039000&28,1069&9,2443\\
741&549081&406869021&27,2213&9,0491&&791&625681&494913671&28,1247&9,2482\\
742&550564&408518488&27,2397&9,0532&&792&627264&496793088&28,1425&9,2521\\
743&552049&410172407&27,2580&9,0572&&793&628849&498677257&28,1603&9,2560\\
744&553536&411830784&27,2764&9,0613&&794&630436&500566184&28,1780&9,2599\\
745&555025&413493625&27,2947&9,0654&&795&632025&502459875&28,1957&9,2638\\
746&556516&415160936&27,3130&9,0694&&796&633616&504358336&28,2135&9,2677\\
747&558009&416832723&27,3313&9,0735&&797&635209&506261573&28,2312&9,2716\\
748&559504&418508992&27,3496&9,0775&&798&636804&508169592&28,2489&9,2754\\
749&561001&420189749&27,3679&9,0816&&799&638401&510082399&28,2666&9,2793\\
750&562500&421875000&27,3861&9,0856&&800&640000&512000000&28,2843&9,2832\\
\newpage
801&641601&513922401&28,3019&9,2870&&851&724201&616295051&29,1719&9,4764\\
802&643204&515849608&28,3196&9,2909&&852&725904&618470208&29,1890&9,4801\\
803&644809&517781627&28,3373&9,2948&&853&727609&620650477&29,2062&9,4838\\
804&646416&519718464&28,3549&9,2986&&854&729316&622835864&29,2233&9,4875\\
805&648025&521660125&28,3725&9,3025&&855&731025&625026375&29,2404&9,4912\\
806&649636&523606616&28,3901&9,3063&&856&732736&627222016&29,2575&9,4949\\
807&651249&525557943&28,4077&9,3102&&857&734449&629422793&29,2746&9,4986\\
808&652864&527514112&28,4253&9,3140&&858&736164&631628712&29,2916&9,5023\\
809&654481&529475129&28,4429&9,3179&&859&737881&633839779&29,3087&9,5060\\
810&656100&531441000&28,4605&9,3217&&860&739600&636056000&29,3258&9,5097\\
811&657721&533411731&28,4781&9,3255&&861&741321&638277381&29,3428&9,5134\\
812&659344&535387328&28,4956&9,3294&&862&743044&640503928&29,3598&9,5171\\
813&660969&537367797&28,5132&9,3332&&863&744769&642735647&29,3769&9,5207\\
814&662596&539353144&28,5307&9,3370&&864&746496&644972544&29,3939&9,5244\\
815&664225&541343375&28,5482&9,3408&&865&748225&647214625&29,4109&9,5281\\
816&665856&543338496&28,5657&9,3447&&866&749956&649461896&29,4279&9,5317\\
817&667489&545338513&28,5832&9,3485&&867&751689&651714363&29,4449&9,5354\\
818&669124&547343432&28,6007&9,3523&&868&753424&653972032&29,4618&9,5391\\
819&670761&549353259&28,6182&9,3561&&869&755161&656234909&29,4788&9,5427\\
820&672400&551368000&28,6356&9,3599&&870&756900&658503000&29,4958&9,5464\\
821&674041&553387661&28,6531&9,3637&&871&758641&660776311&29,5127&9,5501\\
822&675684&555412248&28,6705&9,3675&&872&760384&663054848&29,5296&9,5537\\
823&677329&557441767&28,6880&9,3713&&873&762129&665338617&29,5466&9,5574\\
824&678976&559476224&28,7054&9,3751&&874&763876&667627624&29,5635&9,5610\\
825&680625&561515625&28,7228&9,3789&&875&765625&669921875&29,5804&9,5647\\
826&682276&563559976&28,7402&9,3827&&876&767376&672221376&29,5973&9,5683\\
827&683929&565609283&28,7576&9,3865&&877&769129&674526133&29,6142&9,5719\\
828&685584&567663552&28,7750&9,3902&&878&770884&676836152&29,6311&9,5756\\
829&687241&569722789&28,7924&9,3940&&879&772641&679151439&29,6479&9,5792\\
830&688900&571787000&28,8097&9,3978&&880&774400&681472000&29,6648&9,5828\\
831&690561&573856191&28,8271&9,4016&&881&776161&683797841&29,6816&9,5865\\
832&692224&575930368&28,8444&9,4053&&882&777924&686128968&29,6985&9,5901\\
833&693889&578009537&28,8617&9,4091&&883&779689&688465387&29,7153&9,5937\\
834&695556&580093704&28,8791&9,4129&&884&781456&690807104&29,7321&9,5973\\
835&697225&582182875&28,8964&9,4166&&885&783225&693154125&29,7489&9,6010\\
836&698896&584277056&28,9137&9,4204&&886&784996&695506456&29,7658&9,6046\\
837&700569&586376253&28,9310&9,4241&&887&786769&697864103&29,7825&9,6082\\
838&702244&588480472&28,9482&9,4279&&888&788544&700227072&29,7993&9,6118\\
839&703921&590589719&28,9655&9,4316&&889&790321&702595369&29,8161&9,6154\\
840&705600&592704000&28,9828&9,4354&&890&792100&704969000&29,8329&9,6190\\
841&707281&594823321&29,0000&9,4391&&891&793881&707347971&29,8496&9,6226\\
842&708964&596947688&29,0172&9,4429&&892&795664&709732288&29,8664&9,6262\\
843&710649&599077107&29,0345&9,4466&&893&797449&712121957&29,8831&9,6298\\
844&712336&601211584&29,0517&9,4503&&894&799236&714516984&29,8998&9,6334\\
845&714025&603351125&29,0689&9,4541&&895&801025&716917375&29,9166&9,6370\\
846&715716&605495736&29,0861&9,4578&&896&802816&719323136&29,9333&9,6406\\
847&717409&607645423&29,1033&9,4615&&897&804609&721734273&29,9500&9,6442\\
848&719104&609800192&29,1204&9,4652&&898&806404&724150792&29,9666&9,6477\\
849&720801&611960049&29,1376&9,4690&&899&808201&726572699&29,9833&9,6513\\
850&722500&614125000&29,1548&9,4727&&900&810000&729000000&30,0000&9,6549\\
\newpage
901&811801&731432701&30,0167&9,6585&&951&904401&860085351&30,8383&9,8339\\
902&813604&733870808&30,0333&9,6620&&952&906304&862801408&30,8545&9,8374\\
903&815409&736314327&30,0500&9,6656&&953&908209&865523177&30,8707&9,8408\\
904&817216&738763264&30,0666&9,6692&&954&910116&868250664&30,8869&9,8443\\
905&819025&741217625&30,0832&9,6727&&955&912025&870983875&30,9031&9,8477\\
906&820836&743677416&30,0998&9,6763&&956&913936&873722816&30,9192&9,8511\\
907&822649&746142643&30,1164&9,6799&&957&915849&876467493&30,9354&9,8546\\
908&824464&748613312&30,1330&9,6834&&958&917764&879217912&30,9516&9,8580\\
909&826281&751089429&30,1496&9,6870&&959&919681&881974079&30,9677&9,8614\\
910&828100&753571000&30,1662&9,6905&&960&921600&884736000&30,9839&9,8648\\
911&829921&756058031&30,1828&9,6941&&961&923521&887503681&31,0000&9,8683\\
912&831744&758550528&30,1993&9,6976&&962&925444&890277128&31,0161&9,8717\\
913&833569&761048497&30,2159&9,7012&&963&927369&893056347&31,0322&9,8751\\
914&835396&763551944&30,2324&9,7047&&964&929296&895841344&31,0483&9,8785\\
915&837225&766060875&30,2490&9,7082&&965&931225&898632125&31,0644&9,8819\\
916&839056&768575296&30,2655&9,7118&&966&933156&901428696&31,0805&9,8854\\
917&840889&771095213&30,2820&9,7153&&967&935089&904231063&31,0966&9,8888\\
918&842724&773620632&30,2985&9,7188&&968&937024&907039232&31,1127&9,8922\\
919&844561&776151559&30,3150&9,7224&&969&938961&909853209&31,1288&9,8956\\
920&846400&778688000&30,3315&9,7259&&970&940900&912673000&31,1448&9,8990\\
921&848241&781229961&30,3480&9,7294&&971&942841&915498611&31,1609&9,9024\\
922&850084&783777448&30,3645&9,7329&&972&944784&918330048&31,1769&9,9058\\
923&851929&786330467&30,3809&9,7364&&973&946729&921167317&31,1929&9,9092\\
924&853776&788889024&30,3974&9,7400&&974&948676&924010424&31,2090&9,9126\\
925&855625&791453125&30,4138&9,7435&&975&950625&926859375&31,2250&9,9160\\
926&857476&794022776&30,4302&9,7470&&976&952576&929714176&31,2410&9,9194\\
927&859329&796597983&30,4467&9,7505&&977&954529&932574833&31,2570&9,9227\\
928&861184&799178752&30,4631&9,7540&&978&956484&935441352&31,2730&9,9261\\
929&863041&801765089&30,4795&9,7575&&979&958441&938313739&31,2890&9,9295\\
930&864900&804357000&30,4959&9,7610&&980&960400&941192000&31,3050&9,9329\\
931&866761&806954491&30,5123&9,7645&&981&962361&944076141&31,3209&9,9363\\
932&868624&809557568&30,5287&9,7680&&982&964324&946966168&31,3369&9,9396\\
933&870489&812166237&30,5450&9,7715&&983&966289&949862087&31,3528&9,9430\\
934&872356&814780504&30,5614&9,7750&&984&968256&952763904&31,3688&9,9464\\
935&874225&817400375&30,5778&9,7785&&985&970225&955671625&31,3847&9,9497\\
936&876096&820025856&30,5941&9,7819&&986&972196&958585256&31,4006&9,9531\\
937&877969&822656953&30,6105&9,7854&&987&974169&961504803&31,4166&9,9565\\
938&879844&825293672&30,6268&9,7889&&988&976144&964430272&31,4325&9,9598\\
939&881721&827936019&30,6431&9,7924&&989&978121&967361669&31,4484&9,9632\\
940&883600&830584000&30,6594&9,7959&&990&980100&970299000&31,4643&9,9666\\
941&885481&833237621&30,6757&9,7993&&991&982081&973242271&31,4802&9,9699\\
942&887364&835896888&30,6920&9,8028&&992&984064&976191488&31,4960&9,9733\\
943&889249&838561807&30,7083&9,8063&&993&986049&979146657&31,5119&9,9766\\
944&891136&841232384&30,7246&9,8097&&994&988036&982107784&31,5278&9,9800\\
945&893025&843908625&30,7409&9,8132&&995&990025&985074875&31,5436&9,9833\\
946&894916&846590536&30,7571&9,8167&&996&992016&988047936&31,5595&9,9866\\
947&896809&849278123&30,7734&9,8201&&997&994009&991026973&31,5753&9,9900\\
948&898704&851971392&30,7896&9,8236&&998&996004&994011992&31,5911&9,9933\\
949&900601&854670349&30,8058&9,8270&&999&998001&997002999&31,6070&9,9967\\
950&902500&857375000&30,8221&9,8305&&1000&1000000&1000000000&31,6228&10,0000\\
\end{longtable}

\section{Scomposizione in fattori}
\label{chap:ScomposizioneInFattori}
{\small \begin{longtable}[c]{*{5}{l}}
	\toprule\endhead
	\bottomrule \endfoot
$4=2^{2}$&$68=2^{2}17^{1}$&$126=2^{1}3^{2}7^{1}$&$185=5^{1}37^{1}$&$243=3^{5}$\\
$6=2^{1}3^{1}$&$69=3^{1}23^{1}$&$128=2^{7}$&$186=2^{1}3^{1}31^{1}$&$244=2^{2}61^{1}$\\
$8=2^{3}$&$70=2^{1}5^{1}7^{1}$&$129=3^{1}43^{1}$&$187=11^{1}17^{1}$&$245=5^{1}7^{2}$\\
$9=3^{2}$&$72=2^{3}3^{2}$&$130=2^{1}5^{1}13^{1}$&$188=2^{2}47^{1}$&$246=2^{1}3^{1}41^{1}$\\
$10=2^{1}5^{1}$&$74=2^{1}37^{1}$&$132=2^{2}3^{1}11^{1}$&$189=3^{3}7^{1}$&$247=13^{1}19^{1}$\\
$12=2^{2}3^{1}$&$75=3^{1}5^{2}$&$133=7^{1}19^{1}$&$190=2^{1}5^{1}19^{1}$&$248=2^{3}31^{1}$\\
$14=2^{1}7^{1}$&$76=2^{2}19^{1}$&$134=2^{1}67^{1}$&$192=2^{6}3^{1}$&$249=3^{1}83^{1}$\\
$15=3^{1}5^{1}$&$77=7^{1}11^{1}$&$135=3^{3}5^{1}$&$194=2^{1}97^{1}$&$250=2^{1}5^{3}$\\
$16=2^{4}$&$78=2^{1}3^{1}13^{1}$&$136=2^{3}17^{1}$&$195=3^{1}5^{1}13^{1}$&$252=2^{2}3^{2}7^{1}$\\
$18=2^{1}3^{2}$&$80=2^{4}5^{1}$&$138=2^{1}3^{1}23^{1}$&$196=2^{2}7^{2}$&$253=11^{1}23^{1}$\\
$20=2^{2}5^{1}$&$81=3^{4}$&$140=2^{2}5^{1}7^{1}$&$198=2^{1}3^{2}11^{1}$&$254=2^{1}127^{1}$\\
$21=3^{1}7^{1}$&$82=2^{1}41^{1}$&$141=3^{1}47^{1}$&$200=2^{3}5^{2}$&$255=3^{1}5^{1}17^{1}$\\
$22=2^{1}11^{1}$&$84=2^{2}3^{1}7^{1}$&$142=2^{1}71^{1}$&$201=3^{1}67^{1}$&$256=2^{8}$\\
$24=2^{3}3^{1}$&$85=5^{1}17^{1}$&$143=11^{1}13^{1}$&$202=2^{1}101^{1}$&$258=2^{1}3^{1}43^{1}$\\
$25=5^{2}$&$86=2^{1}43^{1}$&$144=2^{4}3^{2}$&$203=7^{1}29^{1}$&$259=7^{1}37^{1}$\\
$26=2^{1}13^{1}$&$87=3^{1}29^{1}$&$145=5^{1}29^{1}$&$204=2^{2}3^{1}17^{1}$&$260=2^{2}5^{1}13^{1}$\\
$27=3^{3}$&$88=2^{3}11^{1}$&$146=2^{1}73^{1}$&$205=5^{1}41^{1}$&$261=3^{2}29^{1}$\\
$28=2^{2}7^{1}$&$90=2^{1}3^{2}5^{1}$&$147=3^{1}7^{2}$&$206=2^{1}103^{1}$&$262=2^{1}131^{1}$\\
$30=2^{1}3^{1}5^{1}$&$91=7^{1}13^{1}$&$148=2^{2}37^{1}$&$207=3^{2}23^{1}$&$264=2^{3}3^{1}11^{1}$\\
$32=2^{5}$&$92=2^{2}23^{1}$&$150=2^{1}3^{1}5^{2}$&$208=2^{4}13^{1}$&$265=5^{1}53^{1}$\\
$33=3^{1}11^{1}$&$93=3^{1}31^{1}$&$152=2^{3}19^{1}$&$209=11^{1}19^{1}$&$266=2^{1}7^{1}19^{1}$\\
$34=2^{1}17^{1}$&$94=2^{1}47^{1}$&$153=3^{2}17^{1}$&$210=2^{1}3^{1}5^{1}7^{1}$&$267=3^{1}89^{1}$\\
$35=5^{1}7^{1}$&$95=5^{1}19^{1}$&$154=2^{1}7^{1}11^{1}$&$212=2^{2}53^{1}$&$268=2^{2}67^{1}$\\
$36=2^{2}3^{2}$&$96=2^{5}3^{1}$&$155=5^{1}31^{1}$&$213=3^{1}71^{1}$&$270=2^{1}3^{3}5^{1}$\\
$38=2^{1}19^{1}$&$98=2^{1}7^{2}$&$156=2^{2}3^{1}13^{1}$&$214=2^{1}107^{1}$&$272=2^{4}17^{1}$\\
$39=3^{1}13^{1}$&$99=3^{2}11^{1}$&$158=2^{1}79^{1}$&$215=5^{1}43^{1}$&$273=3^{1}7^{1}13^{1}$\\
$40=2^{3}5^{1}$&$100=2^{2}5^{2}$&$159=3^{1}53^{1}$&$216=2^{3}3^{3}$&$274=2^{1}137^{1}$\\
$42=2^{1}3^{1}7^{1}$&$102=2^{1}3^{1}17^{1}$&$160=2^{5}5^{1}$&$217=7^{1}31^{1}$&$275=5^{2}11^{1}$\\
$44=2^{2}11^{1}$&$104=2^{3}13^{1}$&$161=7^{1}23^{1}$&$218=2^{1}109^{1}$&$276=2^{2}3^{1}23^{1}$\\
$45=3^{2}5^{1}$&$105=3^{1}5^{1}7^{1}$&$162=2^{1}3^{4}$&$219=3^{1}73^{1}$&$278=2^{1}139^{1}$\\
$46=2^{1}23^{1}$&$106=2^{1}53^{1}$&$164=2^{2}41^{1}$&$220=2^{2}5^{1}11^{1}$&$279=3^{2}31^{1}$\\
$48=2^{4}3^{1}$&$108=2^{2}3^{3}$&$165=3^{1}5^{1}11^{1}$&$221=13^{1}17^{1}$&$280=2^{3}5^{1}7^{1}$\\
$49=7^{2}$&$110=2^{1}5^{1}11^{1}$&$166=2^{1}83^{1}$&$222=2^{1}3^{1}37^{1}$&$282=2^{1}3^{1}47^{1}$\\
$50=2^{1}5^{2}$&$111=3^{1}37^{1}$&$168=2^{3}3^{1}7^{1}$&$224=2^{5}7^{1}$&$284=2^{2}71^{1}$\\
$51=3^{1}17^{1}$&$112=2^{4}7^{1}$&$169=13^{2}$&$225=3^{2}5^{2}$&$285=3^{1}5^{1}19^{1}$\\
$52=2^{2}13^{1}$&$114=2^{1}3^{1}19^{1}$&$170=2^{1}5^{1}17^{1}$&$226=2^{1}113^{1}$&$286=2^{1}11^{1}13^{1}$\\
$54=2^{1}3^{3}$&$115=5^{1}23^{1}$&$171=3^{2}19^{1}$&$228=2^{2}3^{1}19^{1}$&$287=7^{1}41^{1}$\\
$55=5^{1}11^{1}$&$116=2^{2}29^{1}$&$172=2^{2}43^{1}$&$230=2^{1}5^{1}23^{1}$&$288=2^{5}3^{2}$\\
$56=2^{3}7^{1}$&$117=3^{2}13^{1}$&$174=2^{1}3^{1}29^{1}$&$231=3^{1}7^{1}11^{1}$&$289=17^{2}$\\
$57=3^{1}19^{1}$&$118=2^{1}59^{1}$&$175=5^{2}7^{1}$&$232=2^{3}29^{1}$&$290=2^{1}5^{1}29^{1}$\\
$58=2^{1}29^{1}$&$119=7^{1}17^{1}$&$176=2^{4}11^{1}$&$234=2^{1}3^{2}13^{1}$&$291=3^{1}97^{1}$\\
$60=2^{2}3^{1}5^{1}$&$120=2^{3}3^{1}5^{1}$&$177=3^{1}59^{1}$&$235=5^{1}47^{1}$&$292=2^{2}73^{1}$\\
$62=2^{1}31^{1}$&$121=11^{2}$&$178=2^{1}89^{1}$&$236=2^{2}59^{1}$&$294=2^{1}3^{1}7^{2}$\\
$63=3^{2}7^{1}$&$122=2^{1}61^{1}$&$180=2^{2}3^{2}5^{1}$&$237=3^{1}79^{1}$&$295=5^{1}59^{1}$\\
$64=2^{6}$&$123=3^{1}41^{1}$&$182=2^{1}7^{1}13^{1}$&$238=2^{1}7^{1}17^{1}$&$296=2^{3}37^{1}$\\
$65=5^{1}13^{1}$&$124=2^{2}31^{1}$&$183=3^{1}61^{1}$&$240=2^{4}3^{1}5^{1}$&$297=3^{3}11^{1}$\\
$66=2^{1}3^{1}11^{1}$&$125=5^{3}$&$184=2^{3}23^{1}$&$242=2^{1}11^{2}$&$298=2^{1}149^{1}$\\
\newpage
$299=13^{1}23^{1}$&$355=5^{1}71^{1}$&$411=3^{1}137^{1}$&$469=7^{1}67^{1}$&$524=2^{2}131^{1}$\\
$300=2^{2}3^{1}5^{2}$&$356=2^{2}89^{1}$&$412=2^{2}103^{1}$&$470=2^{1}5^{1}47^{1}$&$525=3^{1}5^{2}7^{1}$\\
$301=7^{1}43^{1}$&$357=3^{1}7^{1}17^{1}$&$413=7^{1}59^{1}$&$471=3^{1}157^{1}$&$526=2^{1}263^{1}$\\
$302=2^{1}151^{1}$&$358=2^{1}179^{1}$&$414=2^{1}3^{2}23^{1}$&$472=2^{3}59^{1}$&$527=17^{1}31^{1}$\\
$303=3^{1}101^{1}$&$360=2^{3}3^{2}5^{1}$&$415=5^{1}83^{1}$&$473=11^{1}43^{1}$&$528=2^{4}3^{1}11^{1}$\\
$304=2^{4}19^{1}$&$361=19^{2}$&$416=2^{5}13^{1}$&$474=2^{1}3^{1}79^{1}$&$529=23^{2}$\\
$305=5^{1}61^{1}$&$362=2^{1}181^{1}$&$417=3^{1}139^{1}$&$475=5^{2}19^{1}$&$530=2^{1}5^{1}53^{1}$\\
$306=2^{1}3^{2}17^{1}$&$363=3^{1}11^{2}$&$418=2^{1}11^{1}19^{1}$&$476=2^{2}7^{1}17^{1}$&$531=3^{2}59^{1}$\\
$308=2^{2}7^{1}11^{1}$&$364=2^{2}7^{1}13^{1}$&$420=2^{2}3^{1}5^{1}7^{1}$&$477=3^{2}53^{1}$&$532=2^{2}7^{1}19^{1}$\\
$309=3^{1}103^{1}$&$365=5^{1}73^{1}$&$422=2^{1}211^{1}$&$478=2^{1}239^{1}$&$533=13^{1}41^{1}$\\
$310=2^{1}5^{1}31^{1}$&$366=2^{1}3^{1}61^{1}$&$423=3^{2}47^{1}$&$480=2^{5}3^{1}5^{1}$&$534=2^{1}3^{1}89^{1}$\\
$312=2^{3}3^{1}13^{1}$&$368=2^{4}23^{1}$&$424=2^{3}53^{1}$&$481=13^{1}37^{1}$&$535=5^{1}107^{1}$\\
$314=2^{1}157^{1}$&$369=3^{2}41^{1}$&$425=5^{2}17^{1}$&$482=2^{1}241^{1}$&$536=2^{3}67^{1}$\\
$315=3^{2}5^{1}7^{1}$&$370=2^{1}5^{1}37^{1}$&$426=2^{1}3^{1}71^{1}$&$483=3^{1}7^{1}23^{1}$&$537=3^{1}179^{1}$\\
$316=2^{2}79^{1}$&$371=7^{1}53^{1}$&$427=7^{1}61^{1}$&$484=2^{2}11^{2}$&$538=2^{1}269^{1}$\\
$318=2^{1}3^{1}53^{1}$&$372=2^{2}3^{1}31^{1}$&$428=2^{2}107^{1}$&$485=5^{1}97^{1}$&$539=7^{2}11^{1}$\\
$319=11^{1}29^{1}$&$374=2^{1}11^{1}17^{1}$&$429=3^{1}11^{1}13^{1}$&$486=2^{1}3^{5}$&$540=2^{2}3^{3}5^{1}$\\
$320=2^{6}5^{1}$&$375=3^{1}5^{3}$&$430=2^{1}5^{1}43^{1}$&$488=2^{3}61^{1}$&$542=2^{1}271^{1}$\\
$321=3^{1}107^{1}$&$376=2^{3}47^{1}$&$432=2^{4}3^{3}$&$489=3^{1}163^{1}$&$543=3^{1}181^{1}$\\
$322=2^{1}7^{1}23^{1}$&$377=13^{1}29^{1}$&$434=2^{1}7^{1}31^{1}$&$490=2^{1}5^{1}7^{2}$&$544=2^{5}17^{1}$\\
$323=17^{1}19^{1}$&$378=2^{1}3^{3}7^{1}$&$435=3^{1}5^{1}29^{1}$&$492=2^{2}3^{1}41^{1}$&$545=5^{1}109^{1}$\\
$324=2^{2}3^{4}$&$380=2^{2}5^{1}19^{1}$&$436=2^{2}109^{1}$&$493=17^{1}29^{1}$&$546=2^{1}3^{1}7^{1}13^{1}$\\
$325=5^{2}13^{1}$&$381=3^{1}127^{1}$&$437=19^{1}23^{1}$&$494=2^{1}13^{1}19^{1}$&$548=2^{2}137^{1}$\\
$326=2^{1}163^{1}$&$382=2^{1}191^{1}$&$438=2^{1}3^{1}73^{1}$&$495=3^{2}5^{1}11^{1}$&$549=3^{2}61^{1}$\\
$327=3^{1}109^{1}$&$384=2^{7}3^{1}$&$440=2^{3}5^{1}11^{1}$&$496=2^{4}31^{1}$&$550=2^{1}5^{2}11^{1}$\\
$328=2^{3}41^{1}$&$385=5^{1}7^{1}11^{1}$&$441=3^{2}7^{2}$&$497=7^{1}71^{1}$&$551=19^{1}29^{1}$\\
$329=7^{1}47^{1}$&$386=2^{1}193^{1}$&$442=2^{1}13^{1}17^{1}$&$498=2^{1}3^{1}83^{1}$&$552=2^{3}3^{1}23^{1}$\\
$330=2^{1}3^{1}5^{1}11^{1}$&$387=3^{2}43^{1}$&$444=2^{2}3^{1}37^{1}$&$500=2^{2}5^{3}$&$553=7^{1}79^{1}$\\
$332=2^{2}83^{1}$&$388=2^{2}97^{1}$&$445=5^{1}89^{1}$&$501=3^{1}167^{1}$&$554=2^{1}277^{1}$\\
$333=3^{2}37^{1}$&$390=2^{1}3^{1}5^{1}13^{1}$&$446=2^{1}223^{1}$&$502=2^{1}251^{1}$&$555=3^{1}5^{1}37^{1}$\\
$334=2^{1}167^{1}$&$391=17^{1}23^{1}$&$447=3^{1}149^{1}$&$504=2^{3}3^{2}7^{1}$&$556=2^{2}139^{1}$\\
$335=5^{1}67^{1}$&$392=2^{3}7^{2}$&$448=2^{6}7^{1}$&$505=5^{1}101^{1}$&$558=2^{1}3^{2}31^{1}$\\
$336=2^{4}3^{1}7^{1}$&$393=3^{1}131^{1}$&$450=2^{1}3^{2}5^{2}$&$506=2^{1}11^{1}23^{1}$&$559=13^{1}43^{1}$\\
$338=2^{1}13^{2}$&$394=2^{1}197^{1}$&$451=11^{1}41^{1}$&$507=3^{1}13^{2}$&$560=2^{4}5^{1}7^{1}$\\
$339=3^{1}113^{1}$&$395=5^{1}79^{1}$&$452=2^{2}113^{1}$&$508=2^{2}127^{1}$&$561=3^{1}11^{1}17^{1}$\\
$340=2^{2}5^{1}17^{1}$&$396=2^{2}3^{2}11^{1}$&$453=3^{1}151^{1}$&$510=2^{1}3^{1}5^{1}17^{1}$&$562=2^{1}281^{1}$\\
$341=11^{1}31^{1}$&$398=2^{1}199^{1}$&$454=2^{1}227^{1}$&$511=7^{1}73^{1}$&$564=2^{2}3^{1}47^{1}$\\
$342=2^{1}3^{2}19^{1}$&$399=3^{1}7^{1}19^{1}$&$455=5^{1}7^{1}13^{1}$&$512=2^{9}$&$565=5^{1}113^{1}$\\
$343=7^{3}$&$400=2^{4}5^{2}$&$456=2^{3}3^{1}19^{1}$&$513=3^{3}19^{1}$&$566=2^{1}283^{1}$\\
$344=2^{3}43^{1}$&$402=2^{1}3^{1}67^{1}$&$458=2^{1}229^{1}$&$514=2^{1}257^{1}$&$567=3^{4}7^{1}$\\
$345=3^{1}5^{1}23^{1}$&$403=13^{1}31^{1}$&$459=3^{3}17^{1}$&$515=5^{1}103^{1}$&$568=2^{3}71^{1}$\\
$346=2^{1}173^{1}$&$404=2^{2}101^{1}$&$460=2^{2}5^{1}23^{1}$&$516=2^{2}3^{1}43^{1}$&$570=2^{1}3^{1}5^{1}19^{1}$\\
$348=2^{2}3^{1}29^{1}$&$405=3^{4}5^{1}$&$462=2^{1}3^{1}7^{1}11^{1}$&$517=11^{1}47^{1}$&$572=2^{2}11^{1}13^{1}$\\
$350=2^{1}5^{2}7^{1}$&$406=2^{1}7^{1}29^{1}$&$464=2^{4}29^{1}$&$518=2^{1}7^{1}37^{1}$&$573=3^{1}191^{1}$\\
$351=3^{3}13^{1}$&$407=11^{1}37^{1}$&$465=3^{1}5^{1}31^{1}$&$519=3^{1}173^{1}$&$574=2^{1}7^{1}41^{1}$\\
$352=2^{5}11^{1}$&$408=2^{3}3^{1}17^{1}$&$466=2^{1}233^{1}$&$520=2^{3}5^{1}13^{1}$&$575=5^{2}23^{1}$\\
$354=2^{1}3^{1}59^{1}$&$410=2^{1}5^{1}41^{1}$&$468=2^{2}3^{2}13^{1}$&$522=2^{1}3^{2}29^{1}$&$576=2^{6}3^{2}$\\
\pagebreak
$578=2^{1}17^{2}$&$634=2^{1}317^{1}$&$690=2^{1}3^{1}5^{1}23^{1}$&$745=5^{1}149^{1}$&$799=17^{1}47^{1}$\\
$579=3^{1}193^{1}$&$635=5^{1}127^{1}$&$692=2^{2}173^{1}$&$746=2^{1}373^{1}$&$800=2^{5}5^{2}$\\
$580=2^{2}5^{1}29^{1}$&$636=2^{2}3^{1}53^{1}$&$693=3^{2}7^{1}11^{1}$&$747=3^{2}83^{1}$&$801=3^{2}89^{1}$\\
$581=7^{1}83^{1}$&$637=7^{2}13^{1}$&$694=2^{1}347^{1}$&$748=2^{2}11^{1}17^{1}$&$802=2^{1}401^{1}$\\
$582=2^{1}3^{1}97^{1}$&$638=2^{1}11^{1}29^{1}$&$695=5^{1}139^{1}$&$749=7^{1}107^{1}$&$803=11^{1}73^{1}$\\
$583=11^{1}53^{1}$&$639=3^{2}71^{1}$&$696=2^{3}3^{1}29^{1}$&$750=2^{1}3^{1}5^{3}$&$804=2^{2}3^{1}67^{1}$\\
$584=2^{3}73^{1}$&$640=2^{7}5^{1}$&$697=17^{1}41^{1}$&$752=2^{4}47^{1}$&$805=5^{1}7^{1}23^{1}$\\
$585=3^{2}5^{1}13^{1}$&$642=2^{1}3^{1}107^{1}$&$698=2^{1}349^{1}$&$753=3^{1}251^{1}$&$806=2^{1}13^{1}31^{1}$\\
$586=2^{1}293^{1}$&$644=2^{2}7^{1}23^{1}$&$699=3^{1}233^{1}$&$754=2^{1}13^{1}29^{1}$&$807=3^{1}269^{1}$\\
$588=2^{2}3^{1}7^{2}$&$645=3^{1}5^{1}43^{1}$&$700=2^{2}5^{2}7^{1}$&$755=5^{1}151^{1}$&$808=2^{3}101^{1}$\\
$589=19^{1}31^{1}$&$646=2^{1}17^{1}19^{1}$&$702=2^{1}3^{3}13^{1}$&$756=2^{2}3^{3}7^{1}$&$810=2^{1}3^{4}5^{1}$\\
$590=2^{1}5^{1}59^{1}$&$648=2^{3}3^{4}$&$703=19^{1}37^{1}$&$758=2^{1}379^{1}$&$812=2^{2}7^{1}29^{1}$\\
$591=3^{1}197^{1}$&$649=11^{1}59^{1}$&$704=2^{6}11^{1}$&$759=3^{1}11^{1}23^{1}$&$813=3^{1}271^{1}$\\
$592=2^{4}37^{1}$&$650=2^{1}5^{2}13^{1}$&$705=3^{1}5^{1}47^{1}$&$760=2^{3}5^{1}19^{1}$&$814=2^{1}11^{1}37^{1}$\\
$594=2^{1}3^{3}11^{1}$&$651=3^{1}7^{1}31^{1}$&$706=2^{1}353^{1}$&$762=2^{1}3^{1}127^{1}$&$815=5^{1}163^{1}$\\
$595=5^{1}7^{1}17^{1}$&$652=2^{2}163^{1}$&$707=7^{1}101^{1}$&$763=7^{1}109^{1}$&$816=2^{4}3^{1}17^{1}$\\
$596=2^{2}149^{1}$&$654=2^{1}3^{1}109^{1}$&$708=2^{2}3^{1}59^{1}$&$764=2^{2}191^{1}$&$817=19^{1}43^{1}$\\
$597=3^{1}199^{1}$&$655=5^{1}131^{1}$&$710=2^{1}5^{1}71^{1}$&$765=3^{2}5^{1}17^{1}$&$818=2^{1}409^{1}$\\
$598=2^{1}13^{1}23^{1}$&$656=2^{4}41^{1}$&$711=3^{2}79^{1}$&$766=2^{1}383^{1}$&$819=3^{2}7^{1}13^{1}$\\
$600=2^{3}3^{1}5^{2}$&$657=3^{2}73^{1}$&$712=2^{3}89^{1}$&$767=13^{1}59^{1}$&$820=2^{2}5^{1}41^{1}$\\
$602=2^{1}7^{1}43^{1}$&$658=2^{1}7^{1}47^{1}$&$713=23^{1}31^{1}$&$768=2^{8}3^{1}$&$822=2^{1}3^{1}137^{1}$\\
$603=3^{2}67^{1}$&$660=2^{2}3^{1}5^{1}11^{1}$&$714=2^{1}3^{1}7^{1}17^{1}$&$770=2^{1}5^{1}7^{1}11^{1}$&$824=2^{3}103^{1}$\\
$604=2^{2}151^{1}$&$662=2^{1}331^{1}$&$715=5^{1}11^{1}13^{1}$&$771=3^{1}257^{1}$&$825=3^{1}5^{2}11^{1}$\\
$605=5^{1}11^{2}$&$663=3^{1}13^{1}17^{1}$&$716=2^{2}179^{1}$&$772=2^{2}193^{1}$&$826=2^{1}7^{1}59^{1}$\\
$606=2^{1}3^{1}101^{1}$&$664=2^{3}83^{1}$&$717=3^{1}239^{1}$&$774=2^{1}3^{2}43^{1}$&$828=2^{2}3^{2}23^{1}$\\
$608=2^{5}19^{1}$&$665=5^{1}7^{1}19^{1}$&$718=2^{1}359^{1}$&$775=5^{2}31^{1}$&$830=2^{1}5^{1}83^{1}$\\
$609=3^{1}7^{1}29^{1}$&$666=2^{1}3^{2}37^{1}$&$720=2^{4}3^{2}5^{1}$&$776=2^{3}97^{1}$&$831=3^{1}277^{1}$\\
$610=2^{1}5^{1}61^{1}$&$667=23^{1}29^{1}$&$721=7^{1}103^{1}$&$777=3^{1}7^{1}37^{1}$&$832=2^{6}13^{1}$\\
$611=13^{1}47^{1}$&$668=2^{2}167^{1}$&$722=2^{1}19^{2}$&$778=2^{1}389^{1}$&$833=7^{2}17^{1}$\\
$612=2^{2}3^{2}17^{1}$&$669=3^{1}223^{1}$&$723=3^{1}241^{1}$&$779=19^{1}41^{1}$&$834=2^{1}3^{1}139^{1}$\\
$614=2^{1}307^{1}$&$670=2^{1}5^{1}67^{1}$&$724=2^{2}181^{1}$&$780=2^{2}3^{1}5^{1}13^{1}$&$835=5^{1}167^{1}$\\
$615=3^{1}5^{1}41^{1}$&$671=11^{1}61^{1}$&$725=5^{2}29^{1}$&$781=11^{1}71^{1}$&$836=2^{2}11^{1}19^{1}$\\
$616=2^{3}7^{1}11^{1}$&$672=2^{5}3^{1}7^{1}$&$726=2^{1}3^{1}11^{2}$&$782=2^{1}17^{1}23^{1}$&$837=3^{3}31^{1}$\\
$618=2^{1}3^{1}103^{1}$&$674=2^{1}337^{1}$&$728=2^{3}7^{1}13^{1}$&$783=3^{3}29^{1}$&$838=2^{1}419^{1}$\\
$620=2^{2}5^{1}31^{1}$&$675=3^{3}5^{2}$&$729=3^{6}$&$784=2^{4}7^{2}$&$840=2^{3}3^{1}5^{1}7^{1}$\\
$621=3^{3}23^{1}$&$676=2^{2}13^{2}$&$730=2^{1}5^{1}73^{1}$&$785=5^{1}157^{1}$&$841=29^{2}$\\
$622=2^{1}311^{1}$&$678=2^{1}3^{1}113^{1}$&$731=17^{1}43^{1}$&$786=2^{1}3^{1}131^{1}$&$842=2^{1}421^{1}$\\
$623=7^{1}89^{1}$&$679=7^{1}97^{1}$&$732=2^{2}3^{1}61^{1}$&$788=2^{2}197^{1}$&$843=3^{1}281^{1}$\\
$624=2^{4}3^{1}13^{1}$&$680=2^{3}5^{1}17^{1}$&$734=2^{1}367^{1}$&$789=3^{1}263^{1}$&$844=2^{2}211^{1}$\\
$625=5^{4}$&$681=3^{1}227^{1}$&$735=3^{1}5^{1}7^{2}$&$790=2^{1}5^{1}79^{1}$&$845=5^{1}13^{2}$\\
$626=2^{1}313^{1}$&$682=2^{1}11^{1}31^{1}$&$736=2^{5}23^{1}$&$791=7^{1}113^{1}$&$846=2^{1}3^{2}47^{1}$\\
$627=3^{1}11^{1}19^{1}$&$684=2^{2}3^{2}19^{1}$&$737=11^{1}67^{1}$&$792=2^{3}3^{2}11^{1}$&$847=7^{1}11^{2}$\\
$628=2^{2}157^{1}$&$685=5^{1}137^{1}$&$738=2^{1}3^{2}41^{1}$&$793=13^{1}61^{1}$&$848=2^{4}53^{1}$\\
$629=17^{1}37^{1}$&$686=2^{1}7^{3}$&$740=2^{2}5^{1}37^{1}$&$794=2^{1}397^{1}$&$849=3^{1}283^{1}$\\
$630=2^{1}3^{2}5^{1}7^{1}$&$687=3^{1}229^{1}$&$741=3^{1}13^{1}19^{1}$&$795=3^{1}5^{1}53^{1}$&$850=2^{1}5^{2}17^{1}$\\
$632=2^{3}79^{1}$&$688=2^{4}43^{1}$&$742=2^{1}7^{1}53^{1}$&$796=2^{2}199^{1}$&$851=23^{1}37^{1}$\\
$633=3^{1}211^{1}$&$689=13^{1}53^{1}$&$744=2^{3}3^{1}31^{1}$&$798=2^{1}3^{1}7^{1}19^{1}$&$852=2^{2}3^{1}71^{1}$\\
\pagebreak
$854=2^{1}7^{1}61^{1}$&$909=3^{2}101^{1}$&$963=3^{2}107^{1}$&$1018=2^{1}509^{1}$&$1075=5^{2}43^{1}$\\
$855=3^{2}5^{1}19^{1}$&$910=2^{1}5^{1}7^{1}13^{1}$&$964=2^{2}241^{1}$&$1020=2^{2}3^{1}5^{1}17^{1}$&$1076=2^{2}269^{1}$\\
$856=2^{3}107^{1}$&$912=2^{4}3^{1}19^{1}$&$965=5^{1}193^{1}$&$1022=2^{1}7^{1}73^{1}$&$1077=3^{1}359^{1}$\\
$858=2^{1}3^{1}11^{1}13^{1}$&$913=11^{1}83^{1}$&$966=2^{1}3^{1}7^{1}23^{1}$&$1023=3^{1}11^{1}31^{1}$&$1078=2^{1}7^{2}11^{1}$\\
$860=2^{2}5^{1}43^{1}$&$914=2^{1}457^{1}$&$968=2^{3}11^{2}$&$1024=2^{10}$&$1079=13^{1}83^{1}$\\
$861=3^{1}7^{1}41^{1}$&$915=3^{1}5^{1}61^{1}$&$969=3^{1}17^{1}19^{1}$&$1025=5^{2}41^{1}$&$1080=2^{3}3^{3}5^{1}$\\
$862=2^{1}431^{1}$&$916=2^{2}229^{1}$&$970=2^{1}5^{1}97^{1}$&$1026=2^{1}3^{3}19^{1}$&$1081=23^{1}47^{1}$\\
$864=2^{5}3^{3}$&$917=7^{1}131^{1}$&$972=2^{2}3^{5}$&$1027=13^{1}79^{1}$&$1082=2^{1}541^{1}$\\
$865=5^{1}173^{1}$&$918=2^{1}3^{3}17^{1}$&$973=7^{1}139^{1}$&$1028=2^{2}257^{1}$&$1083=3^{1}19^{2}$\\
$866=2^{1}433^{1}$&$920=2^{3}5^{1}23^{1}$&$974=2^{1}487^{1}$&$1029=3^{1}7^{3}$&$1084=2^{2}271^{1}$\\
$867=3^{1}17^{2}$&$921=3^{1}307^{1}$&$975=3^{1}5^{2}13^{1}$&$1030=2^{1}5^{1}103^{1}$&$1085=5^{1}7^{1}31^{1}$\\
$868=2^{2}7^{1}31^{1}$&$922=2^{1}461^{1}$&$976=2^{4}61^{1}$&$1032=2^{3}3^{1}43^{1}$&$1086=2^{1}3^{1}181^{1}$\\
$869=11^{1}79^{1}$&$923=13^{1}71^{1}$&$978=2^{1}3^{1}163^{1}$&$1034=2^{1}11^{1}47^{1}$&$1088=2^{6}17^{1}$\\
$870=2^{1}3^{1}5^{1}29^{1}$&$924=2^{2}3^{1}7^{1}11^{1}$&$979=11^{1}89^{1}$&$1035=3^{2}5^{1}23^{1}$&$1089=3^{2}11^{2}$\\
$871=13^{1}67^{1}$&$925=5^{2}37^{1}$&$980=2^{2}5^{1}7^{2}$&$1036=2^{2}7^{1}37^{1}$&$1090=2^{1}5^{1}109^{1}$\\
$872=2^{3}109^{1}$&$926=2^{1}463^{1}$&$981=3^{2}109^{1}$&$1037=17^{1}61^{1}$&$1092=2^{2}3^{1}7^{1}13^{1}$\\
$873=3^{2}97^{1}$&$927=3^{2}103^{1}$&$982=2^{1}491^{1}$&$1038=2^{1}3^{1}173^{1}$&$1094=2^{1}547^{1}$\\
$874=2^{1}19^{1}23^{1}$&$928=2^{5}29^{1}$&$984=2^{3}3^{1}41^{1}$&$1040=2^{4}5^{1}13^{1}$&$1095=3^{1}5^{1}73^{1}$\\
$875=5^{3}7^{1}$&$930=2^{1}3^{1}5^{1}31^{1}$&$985=5^{1}197^{1}$&$1041=3^{1}347^{1}$&$1096=2^{3}137^{1}$\\
$876=2^{2}3^{1}73^{1}$&$931=7^{2}19^{1}$&$986=2^{1}17^{1}29^{1}$&$1042=2^{1}521^{1}$&$1098=2^{1}3^{2}61^{1}$\\
$878=2^{1}439^{1}$&$932=2^{2}233^{1}$&$987=3^{1}7^{1}47^{1}$&$1043=7^{1}149^{1}$&$1099=7^{1}157^{1}$\\
$879=3^{1}293^{1}$&$933=3^{1}311^{1}$&$988=2^{2}13^{1}19^{1}$&$1044=2^{2}3^{2}29^{1}$&$1100=2^{2}5^{2}11^{1}$\\
$880=2^{4}5^{1}11^{1}$&$934=2^{1}467^{1}$&$989=23^{1}43^{1}$&$1045=5^{1}11^{1}19^{1}$&$1101=3^{1}367^{1}$\\
$882=2^{1}3^{2}7^{2}$&$935=5^{1}11^{1}17^{1}$&$990=2^{1}3^{2}5^{1}11^{1}$&$1046=2^{1}523^{1}$&$1102=2^{1}19^{1}29^{1}$\\
$884=2^{2}13^{1}17^{1}$&$936=2^{3}3^{2}13^{1}$&$992=2^{5}31^{1}$&$1047=3^{1}349^{1}$&$1104=2^{4}3^{1}23^{1}$\\
$885=3^{1}5^{1}59^{1}$&$938=2^{1}7^{1}67^{1}$&$993=3^{1}331^{1}$&$1048=2^{3}131^{1}$&$1105=5^{1}13^{1}17^{1}$\\
$886=2^{1}443^{1}$&$939=3^{1}313^{1}$&$994=2^{1}7^{1}71^{1}$&$1050=2^{1}3^{1}5^{2}7^{1}$&$1106=2^{1}7^{1}79^{1}$\\
$888=2^{3}3^{1}37^{1}$&$940=2^{2}5^{1}47^{1}$&$995=5^{1}199^{1}$&$1052=2^{2}263^{1}$&$1107=3^{3}41^{1}$\\
$889=7^{1}127^{1}$&$942=2^{1}3^{1}157^{1}$&$996=2^{2}3^{1}83^{1}$&$1053=3^{4}13^{1}$&$1108=2^{2}277^{1}$\\
$890=2^{1}5^{1}89^{1}$&$943=23^{1}41^{1}$&$998=2^{1}499^{1}$&$1054=2^{1}17^{1}31^{1}$&$1110=2^{1}3^{1}5^{1}37^{1}$\\
$891=3^{4}11^{1}$&$944=2^{4}59^{1}$&$999=3^{3}37^{1}$&$1055=5^{1}211^{1}$&$1111=11^{1}101^{1}$\\
$892=2^{2}223^{1}$&$945=3^{3}5^{1}7^{1}$&$1000=2^{3}5^{3}$&$1056=2^{5}3^{1}11^{1}$&$1112=2^{3}139^{1}$\\
$893=19^{1}47^{1}$&$946=2^{1}11^{1}43^{1}$&$1001=7^{1}11^{1}13^{1}$&$1057=7^{1}151^{1}$&$1113=3^{1}7^{1}53^{1}$\\
$894=2^{1}3^{1}149^{1}$&$948=2^{2}3^{1}79^{1}$&$1002=2^{1}3^{1}167^{1}$&$1058=2^{1}23^{2}$&$1114=2^{1}557^{1}$\\
$895=5^{1}179^{1}$&$949=13^{1}73^{1}$&$1003=17^{1}59^{1}$&$1059=3^{1}353^{1}$&$1115=5^{1}223^{1}$\\
$896=2^{7}7^{1}$&$950=2^{1}5^{2}19^{1}$&$1004=2^{2}251^{1}$&$1060=2^{2}5^{1}53^{1}$&$1116=2^{2}3^{2}31^{1}$\\
$897=3^{1}13^{1}23^{1}$&$951=3^{1}317^{1}$&$1005=3^{1}5^{1}67^{1}$&$1062=2^{1}3^{2}59^{1}$&$1118=2^{1}13^{1}43^{1}$\\
$898=2^{1}449^{1}$&$952=2^{3}7^{1}17^{1}$&$1006=2^{1}503^{1}$&$1064=2^{3}7^{1}19^{1}$&$1119=3^{1}373^{1}$\\
$899=29^{1}31^{1}$&$954=2^{1}3^{2}53^{1}$&$1007=19^{1}53^{1}$&$1065=3^{1}5^{1}71^{1}$&$1120=2^{5}5^{1}7^{1}$\\
$900=2^{2}3^{2}5^{2}$&$955=5^{1}191^{1}$&$1008=2^{4}3^{2}7^{1}$&$1066=2^{1}13^{1}41^{1}$&$1121=19^{1}59^{1}$\\
$901=17^{1}53^{1}$&$956=2^{2}239^{1}$&$1010=2^{1}5^{1}101^{1}$&$1067=11^{1}97^{1}$&$1122=2^{1}3^{1}11^{1}17^{1}$\\
$902=2^{1}11^{1}41^{1}$&$957=3^{1}11^{1}29^{1}$&$1011=3^{1}337^{1}$&$1068=2^{2}3^{1}89^{1}$&$1124=2^{2}281^{1}$\\
$903=3^{1}7^{1}43^{1}$&$958=2^{1}479^{1}$&$1012=2^{2}11^{1}23^{1}$&$1070=2^{1}5^{1}107^{1}$&$1125=3^{2}5^{3}$\\
$904=2^{3}113^{1}$&$959=7^{1}137^{1}$&$1014=2^{1}3^{1}13^{2}$&$1071=3^{2}7^{1}17^{1}$&$1126=2^{1}563^{1}$\\
$905=5^{1}181^{1}$&$960=2^{6}3^{1}5^{1}$&$1015=5^{1}7^{1}29^{1}$&$1072=2^{4}67^{1}$&$1127=7^{2}23^{1}$\\
$906=2^{1}3^{1}151^{1}$&$961=31^{2}$&$1016=2^{3}127^{1}$&$1073=29^{1}37^{1}$&$1128=2^{3}3^{1}47^{1}$\\
$908=2^{2}227^{1}$&$962=2^{1}13^{1}37^{1}$&$1017=3^{2}113^{1}$&$1074=2^{1}3^{1}179^{1}$&$1130=2^{1}5^{1}113^{1}$\\
\pagebreak
$1131=3^{1}13^{1}29^{1}$&$1183=7^{1}13^{2}$&$1239=3^{1}7^{1}59^{1}$&$1293=3^{1}431^{1}$&$1347=3^{1}449^{1}$\\
$1132=2^{2}283^{1}$&$1184=2^{5}37^{1}$&$1240=2^{3}5^{1}31^{1}$&$1294=2^{1}647^{1}$&$1348=2^{2}337^{1}$\\
$1133=11^{1}103^{1}$&$1185=3^{1}5^{1}79^{1}$&$1241=17^{1}73^{1}$&$1295=5^{1}7^{1}37^{1}$&$1349=19^{1}71^{1}$\\
$1134=2^{1}3^{4}7^{1}$&$1186=2^{1}593^{1}$&$1242=2^{1}3^{3}23^{1}$&$1296=2^{4}3^{4}$&$1350=2^{1}3^{3}5^{2}$\\
$1135=5^{1}227^{1}$&$1188=2^{2}3^{3}11^{1}$&$1243=11^{1}113^{1}$&$1298=2^{1}11^{1}59^{1}$&$1351=7^{1}193^{1}$\\
$1136=2^{4}71^{1}$&$1189=29^{1}41^{1}$&$1244=2^{2}311^{1}$&$1299=3^{1}433^{1}$&$1352=2^{3}13^{2}$\\
$1137=3^{1}379^{1}$&$1190=2^{1}5^{1}7^{1}17^{1}$&$1245=3^{1}5^{1}83^{1}$&$1300=2^{2}5^{2}13^{1}$&$1353=3^{1}11^{1}41^{1}$\\
$1138=2^{1}569^{1}$&$1191=3^{1}397^{1}$&$1246=2^{1}7^{1}89^{1}$&$1302=2^{1}3^{1}7^{1}31^{1}$&$1354=2^{1}677^{1}$\\
$1139=17^{1}67^{1}$&$1192=2^{3}149^{1}$&$1247=29^{1}43^{1}$&$1304=2^{3}163^{1}$&$1355=5^{1}271^{1}$\\
$1140=2^{2}3^{1}5^{1}19^{1}$&$1194=2^{1}3^{1}199^{1}$&$1248=2^{5}3^{1}13^{1}$&$1305=3^{2}5^{1}29^{1}$&$1356=2^{2}3^{1}113^{1}$\\
$1141=7^{1}163^{1}$&$1195=5^{1}239^{1}$&$1250=2^{1}5^{4}$&$1306=2^{1}653^{1}$&$1357=23^{1}59^{1}$\\
$1142=2^{1}571^{1}$&$1196=2^{2}13^{1}23^{1}$&$1251=3^{2}139^{1}$&$1308=2^{2}3^{1}109^{1}$&$1358=2^{1}7^{1}97^{1}$\\
$1143=3^{2}127^{1}$&$1197=3^{2}7^{1}19^{1}$&$1252=2^{2}313^{1}$&$1309=7^{1}11^{1}17^{1}$&$1359=3^{2}151^{1}$\\
$1144=2^{3}11^{1}13^{1}$&$1198=2^{1}599^{1}$&$1253=7^{1}179^{1}$&$1310=2^{1}5^{1}131^{1}$&$1360=2^{4}5^{1}17^{1}$\\
$1145=5^{1}229^{1}$&$1199=11^{1}109^{1}$&$1254=2^{1}3^{1}11^{1}19^{1}$&$1311=3^{1}19^{1}23^{1}$&$1362=2^{1}3^{1}227^{1}$\\
$1146=2^{1}3^{1}191^{1}$&$1200=2^{4}3^{1}5^{2}$&$1255=5^{1}251^{1}$&$1312=2^{5}41^{1}$&$1363=29^{1}47^{1}$\\
$1147=31^{1}37^{1}$&$1202=2^{1}601^{1}$&$1256=2^{3}157^{1}$&$1313=13^{1}101^{1}$&$1364=2^{2}11^{1}31^{1}$\\
$1148=2^{2}7^{1}41^{1}$&$1203=3^{1}401^{1}$&$1257=3^{1}419^{1}$&$1314=2^{1}3^{2}73^{1}$&$1365=3^{1}5^{1}7^{1}13^{1}$\\
$1149=3^{1}383^{1}$&$1204=2^{2}7^{1}43^{1}$&$1258=2^{1}17^{1}37^{1}$&$1315=5^{1}263^{1}$&$1366=2^{1}683^{1}$\\
$1150=2^{1}5^{2}23^{1}$&$1205=5^{1}241^{1}$&$1260=2^{2}3^{2}5^{1}7^{1}$&$1316=2^{2}7^{1}47^{1}$&$1368=2^{3}3^{2}19^{1}$\\
$1152=2^{7}3^{2}$&$1206=2^{1}3^{2}67^{1}$&$1261=13^{1}97^{1}$&$1317=3^{1}439^{1}$&$1369=37^{2}$\\
$1154=2^{1}577^{1}$&$1207=17^{1}71^{1}$&$1262=2^{1}631^{1}$&$1318=2^{1}659^{1}$&$1370=2^{1}5^{1}137^{1}$\\
$1155=3^{1}5^{1}7^{1}11^{1}$&$1208=2^{3}151^{1}$&$1263=3^{1}421^{1}$&$1320=2^{3}3^{1}5^{1}11^{1}$&$1371=3^{1}457^{1}$\\
$1156=2^{2}17^{2}$&$1209=3^{1}13^{1}31^{1}$&$1264=2^{4}79^{1}$&$1322=2^{1}661^{1}$&$1372=2^{2}7^{3}$\\
$1157=13^{1}89^{1}$&$1210=2^{1}5^{1}11^{2}$&$1265=5^{1}11^{1}23^{1}$&$1323=3^{3}7^{2}$&$1374=2^{1}3^{1}229^{1}$\\
$1158=2^{1}3^{1}193^{1}$&$1211=7^{1}173^{1}$&$1266=2^{1}3^{1}211^{1}$&$1324=2^{2}331^{1}$&$1375=5^{3}11^{1}$\\
$1159=19^{1}61^{1}$&$1212=2^{2}3^{1}101^{1}$&$1267=7^{1}181^{1}$&$1325=5^{2}53^{1}$&$1376=2^{5}43^{1}$\\
$1160=2^{3}5^{1}29^{1}$&$1214=2^{1}607^{1}$&$1268=2^{2}317^{1}$&$1326=2^{1}3^{1}13^{1}17^{1}$&$1377=3^{4}17^{1}$\\
$1161=3^{3}43^{1}$&$1215=3^{5}5^{1}$&$1269=3^{3}47^{1}$&$1328=2^{4}83^{1}$&$1378=2^{1}13^{1}53^{1}$\\
$1162=2^{1}7^{1}83^{1}$&$1216=2^{6}19^{1}$&$1270=2^{1}5^{1}127^{1}$&$1329=3^{1}443^{1}$&$1379=7^{1}197^{1}$\\
$1164=2^{2}3^{1}97^{1}$&$1218=2^{1}3^{1}7^{1}29^{1}$&$1271=31^{1}41^{1}$&$1330=2^{1}5^{1}7^{1}19^{1}$&$1380=2^{2}3^{1}5^{1}23^{1}$\\
$1165=5^{1}233^{1}$&$1219=23^{1}53^{1}$&$1272=2^{3}3^{1}53^{1}$&$1331=11^{3}$&$1382=2^{1}691^{1}$\\
$1166=2^{1}11^{1}53^{1}$&$1220=2^{2}5^{1}61^{1}$&$1273=19^{1}67^{1}$&$1332=2^{2}3^{2}37^{1}$&$1383=3^{1}461^{1}$\\
$1167=3^{1}389^{1}$&$1221=3^{1}11^{1}37^{1}$&$1274=2^{1}7^{2}13^{1}$&$1333=31^{1}43^{1}$&$1384=2^{3}173^{1}$\\
$1168=2^{4}73^{1}$&$1222=2^{1}13^{1}47^{1}$&$1275=3^{1}5^{2}17^{1}$&$1334=2^{1}23^{1}29^{1}$&$1385=5^{1}277^{1}$\\
$1169=7^{1}167^{1}$&$1224=2^{3}3^{2}17^{1}$&$1276=2^{2}11^{1}29^{1}$&$1335=3^{1}5^{1}89^{1}$&$1386=2^{1}3^{2}7^{1}11^{1}$\\
$1170=2^{1}3^{2}5^{1}13^{1}$&$1225=5^{2}7^{2}$&$1278=2^{1}3^{2}71^{1}$&$1336=2^{3}167^{1}$&$1387=19^{1}73^{1}$\\
$1172=2^{2}293^{1}$&$1226=2^{1}613^{1}$&$1280=2^{8}5^{1}$&$1337=7^{1}191^{1}$&$1388=2^{2}347^{1}$\\
$1173=3^{1}17^{1}23^{1}$&$1227=3^{1}409^{1}$&$1281=3^{1}7^{1}61^{1}$&$1338=2^{1}3^{1}223^{1}$&$1389=3^{1}463^{1}$\\
$1174=2^{1}587^{1}$&$1228=2^{2}307^{1}$&$1282=2^{1}641^{1}$&$1339=13^{1}103^{1}$&$1390=2^{1}5^{1}139^{1}$\\
$1175=5^{2}47^{1}$&$1230=2^{1}3^{1}5^{1}41^{1}$&$1284=2^{2}3^{1}107^{1}$&$1340=2^{2}5^{1}67^{1}$&$1391=13^{1}107^{1}$\\
$1176=2^{3}3^{1}7^{2}$&$1232=2^{4}7^{1}11^{1}$&$1285=5^{1}257^{1}$&$1341=3^{2}149^{1}$&$1392=2^{4}3^{1}29^{1}$\\
$1177=11^{1}107^{1}$&$1233=3^{2}137^{1}$&$1286=2^{1}643^{1}$&$1342=2^{1}11^{1}61^{1}$&$1393=7^{1}199^{1}$\\
$1178=2^{1}19^{1}31^{1}$&$1234=2^{1}617^{1}$&$1287=3^{2}11^{1}13^{1}$&$1343=17^{1}79^{1}$&$1394=2^{1}17^{1}41^{1}$\\
$1179=3^{2}131^{1}$&$1235=5^{1}13^{1}19^{1}$&$1288=2^{3}7^{1}23^{1}$&$1344=2^{6}3^{1}7^{1}$&$1395=3^{2}5^{1}31^{1}$\\
$1180=2^{2}5^{1}59^{1}$&$1236=2^{2}3^{1}103^{1}$&$1290=2^{1}3^{1}5^{1}43^{1}$&$1345=5^{1}269^{1}$&$1396=2^{2}349^{1}$\\
$1182=2^{1}3^{1}197^{1}$&$1238=2^{1}619^{1}$&$1292=2^{2}17^{1}19^{1}$&$1346=2^{1}673^{1}$&$1397=11^{1}127^{1}$\\
\pagebreak
$1398=2^{1}3^{1}233^{1}$&$1455=3^{1}5^{1}97^{1}$&$1510=2^{1}5^{1}151^{1}$&$1564=2^{2}17^{1}23^{1}$&$1622=2^{1}811^{1}$\\
$1400=2^{3}5^{2}7^{1}$&$1456=2^{4}7^{1}13^{1}$&$1512=2^{3}3^{3}7^{1}$&$1565=5^{1}313^{1}$&$1623=3^{1}541^{1}$\\
$1401=3^{1}467^{1}$&$1457=31^{1}47^{1}$&$1513=17^{1}89^{1}$&$1566=2^{1}3^{3}29^{1}$&$1624=2^{3}7^{1}29^{1}$\\
$1402=2^{1}701^{1}$&$1458=2^{1}3^{6}$&$1514=2^{1}757^{1}$&$1568=2^{5}7^{2}$&$1625=5^{3}13^{1}$\\
$1403=23^{1}61^{1}$&$1460=2^{2}5^{1}73^{1}$&$1515=3^{1}5^{1}101^{1}$&$1569=3^{1}523^{1}$&$1626=2^{1}3^{1}271^{1}$\\
$1404=2^{2}3^{3}13^{1}$&$1461=3^{1}487^{1}$&$1516=2^{2}379^{1}$&$1570=2^{1}5^{1}157^{1}$&$1628=2^{2}11^{1}37^{1}$\\
$1405=5^{1}281^{1}$&$1462=2^{1}17^{1}43^{1}$&$1517=37^{1}41^{1}$&$1572=2^{2}3^{1}131^{1}$&$1629=3^{2}181^{1}$\\
$1406=2^{1}19^{1}37^{1}$&$1463=7^{1}11^{1}19^{1}$&$1518=2^{1}3^{1}11^{1}23^{1}$&$1573=11^{2}13^{1}$&$1630=2^{1}5^{1}163^{1}$\\
$1407=3^{1}7^{1}67^{1}$&$1464=2^{3}3^{1}61^{1}$&$1519=7^{2}31^{1}$&$1574=2^{1}787^{1}$&$1631=7^{1}233^{1}$\\
$1408=2^{7}11^{1}$&$1465=5^{1}293^{1}$&$1520=2^{4}5^{1}19^{1}$&$1575=3^{2}5^{2}7^{1}$&$1632=2^{5}3^{1}17^{1}$\\
$1410=2^{1}3^{1}5^{1}47^{1}$&$1466=2^{1}733^{1}$&$1521=3^{2}13^{2}$&$1576=2^{3}197^{1}$&$1633=23^{1}71^{1}$\\
$1411=17^{1}83^{1}$&$1467=3^{2}163^{1}$&$1522=2^{1}761^{1}$&$1577=19^{1}83^{1}$&$1634=2^{1}19^{1}43^{1}$\\
$1412=2^{2}353^{1}$&$1468=2^{2}367^{1}$&$1524=2^{2}3^{1}127^{1}$&$1578=2^{1}3^{1}263^{1}$&$1635=3^{1}5^{1}109^{1}$\\
$1413=3^{2}157^{1}$&$1469=13^{1}113^{1}$&$1525=5^{2}61^{1}$&$1580=2^{2}5^{1}79^{1}$&$1636=2^{2}409^{1}$\\
$1414=2^{1}7^{1}101^{1}$&$1470=2^{1}3^{1}5^{1}7^{2}$&$1526=2^{1}7^{1}109^{1}$&$1581=3^{1}17^{1}31^{1}$&$1638=2^{1}3^{2}7^{1}13^{1}$\\
$1415=5^{1}283^{1}$&$1472=2^{6}23^{1}$&$1527=3^{1}509^{1}$&$1582=2^{1}7^{1}113^{1}$&$1639=11^{1}149^{1}$\\
$1416=2^{3}3^{1}59^{1}$&$1473=3^{1}491^{1}$&$1528=2^{3}191^{1}$&$1584=2^{4}3^{2}11^{1}$&$1640=2^{3}5^{1}41^{1}$\\
$1417=13^{1}109^{1}$&$1474=2^{1}11^{1}67^{1}$&$1529=11^{1}139^{1}$&$1585=5^{1}317^{1}$&$1641=3^{1}547^{1}$\\
$1418=2^{1}709^{1}$&$1475=5^{2}59^{1}$&$1530=2^{1}3^{2}5^{1}17^{1}$&$1586=2^{1}13^{1}61^{1}$&$1642=2^{1}821^{1}$\\
$1419=3^{1}11^{1}43^{1}$&$1476=2^{2}3^{2}41^{1}$&$1532=2^{2}383^{1}$&$1587=3^{1}23^{2}$&$1643=31^{1}53^{1}$\\
$1420=2^{2}5^{1}71^{1}$&$1477=7^{1}211^{1}$&$1533=3^{1}7^{1}73^{1}$&$1588=2^{2}397^{1}$&$1644=2^{2}3^{1}137^{1}$\\
$1421=7^{2}29^{1}$&$1478=2^{1}739^{1}$&$1534=2^{1}13^{1}59^{1}$&$1589=7^{1}227^{1}$&$1645=5^{1}7^{1}47^{1}$\\
$1422=2^{1}3^{2}79^{1}$&$1479=3^{1}17^{1}29^{1}$&$1535=5^{1}307^{1}$&$1590=2^{1}3^{1}5^{1}53^{1}$&$1646=2^{1}823^{1}$\\
$1424=2^{4}89^{1}$&$1480=2^{3}5^{1}37^{1}$&$1536=2^{9}3^{1}$&$1591=37^{1}43^{1}$&$1647=3^{3}61^{1}$\\
$1425=3^{1}5^{2}19^{1}$&$1482=2^{1}3^{1}13^{1}19^{1}$&$1537=29^{1}53^{1}$&$1592=2^{3}199^{1}$&$1648=2^{4}103^{1}$\\
$1426=2^{1}23^{1}31^{1}$&$1484=2^{2}7^{1}53^{1}$&$1538=2^{1}769^{1}$&$1593=3^{3}59^{1}$&$1649=17^{1}97^{1}$\\
$1428=2^{2}3^{1}7^{1}17^{1}$&$1485=3^{3}5^{1}11^{1}$&$1539=3^{4}19^{1}$&$1594=2^{1}797^{1}$&$1650=2^{1}3^{1}5^{2}11^{1}$\\
$1430=2^{1}5^{1}11^{1}13^{1}$&$1486=2^{1}743^{1}$&$1540=2^{2}5^{1}7^{1}11^{1}$&$1595=5^{1}11^{1}29^{1}$&$1651=13^{1}127^{1}$\\
$1431=3^{3}53^{1}$&$1488=2^{4}3^{1}31^{1}$&$1541=23^{1}67^{1}$&$1596=2^{2}3^{1}7^{1}19^{1}$&$1652=2^{2}7^{1}59^{1}$\\
$1432=2^{3}179^{1}$&$1490=2^{1}5^{1}149^{1}$&$1542=2^{1}3^{1}257^{1}$&$1598=2^{1}17^{1}47^{1}$&$1653=3^{1}19^{1}29^{1}$\\
$1434=2^{1}3^{1}239^{1}$&$1491=3^{1}7^{1}71^{1}$&$1544=2^{3}193^{1}$&$1599=3^{1}13^{1}41^{1}$&$1654=2^{1}827^{1}$\\
$1435=5^{1}7^{1}41^{1}$&$1492=2^{2}373^{1}$&$1545=3^{1}5^{1}103^{1}$&$1600=2^{6}5^{2}$&$1655=5^{1}331^{1}$\\
$1436=2^{2}359^{1}$&$1494=2^{1}3^{2}83^{1}$&$1546=2^{1}773^{1}$&$1602=2^{1}3^{2}89^{1}$&$1656=2^{3}3^{2}23^{1}$\\
$1437=3^{1}479^{1}$&$1495=5^{1}13^{1}23^{1}$&$1547=7^{1}13^{1}17^{1}$&$1603=7^{1}229^{1}$&$1658=2^{1}829^{1}$\\
$1438=2^{1}719^{1}$&$1496=2^{3}11^{1}17^{1}$&$1548=2^{2}3^{2}43^{1}$&$1604=2^{2}401^{1}$&$1659=3^{1}7^{1}79^{1}$\\
$1440=2^{5}3^{2}5^{1}$&$1497=3^{1}499^{1}$&$1550=2^{1}5^{2}31^{1}$&$1605=3^{1}5^{1}107^{1}$&$1660=2^{2}5^{1}83^{1}$\\
$1441=11^{1}131^{1}$&$1498=2^{1}7^{1}107^{1}$&$1551=3^{1}11^{1}47^{1}$&$1606=2^{1}11^{1}73^{1}$&$1661=11^{1}151^{1}$\\
$1442=2^{1}7^{1}103^{1}$&$1500=2^{2}3^{1}5^{3}$&$1552=2^{4}97^{1}$&$1608=2^{3}3^{1}67^{1}$&$1662=2^{1}3^{1}277^{1}$\\
$1443=3^{1}13^{1}37^{1}$&$1501=19^{1}79^{1}$&$1554=2^{1}3^{1}7^{1}37^{1}$&$1610=2^{1}5^{1}7^{1}23^{1}$&$1664=2^{7}13^{1}$\\
$1444=2^{2}19^{2}$&$1502=2^{1}751^{1}$&$1555=5^{1}311^{1}$&$1611=3^{2}179^{1}$&$1665=3^{2}5^{1}37^{1}$\\
$1445=5^{1}17^{2}$&$1503=3^{2}167^{1}$&$1556=2^{2}389^{1}$&$1612=2^{2}13^{1}31^{1}$&$1666=2^{1}7^{2}17^{1}$\\
$1446=2^{1}3^{1}241^{1}$&$1504=2^{5}47^{1}$&$1557=3^{2}173^{1}$&$1614=2^{1}3^{1}269^{1}$&$1668=2^{2}3^{1}139^{1}$\\
$1448=2^{3}181^{1}$&$1505=5^{1}7^{1}43^{1}$&$1558=2^{1}19^{1}41^{1}$&$1615=5^{1}17^{1}19^{1}$&$1670=2^{1}5^{1}167^{1}$\\
$1449=3^{2}7^{1}23^{1}$&$1506=2^{1}3^{1}251^{1}$&$1560=2^{3}3^{1}5^{1}13^{1}$&$1616=2^{4}101^{1}$&$1671=3^{1}557^{1}$\\
$1450=2^{1}5^{2}29^{1}$&$1507=11^{1}137^{1}$&$1561=7^{1}223^{1}$&$1617=3^{1}7^{2}11^{1}$&$1672=2^{3}11^{1}19^{1}$\\
$1452=2^{2}3^{1}11^{2}$&$1508=2^{2}13^{1}29^{1}$&$1562=2^{1}11^{1}71^{1}$&$1618=2^{1}809^{1}$&$1673=7^{1}239^{1}$\\
$1454=2^{1}727^{1}$&$1509=3^{1}503^{1}$&$1563=3^{1}521^{1}$&$1620=2^{2}3^{4}5^{1}$&$1674=2^{1}3^{3}31^{1}$\\
\pagebreak
$1675=5^{2}67^{1}$&$1728=2^{6}3^{3}$&$1781=13^{1}137^{1}$&$1835=5^{1}367^{1}$&$1890=2^{1}3^{3}5^{1}7^{1}$\\
$1676=2^{2}419^{1}$&$1729=7^{1}13^{1}19^{1}$&$1782=2^{1}3^{4}11^{1}$&$1836=2^{2}3^{3}17^{1}$&$1891=31^{1}61^{1}$\\
$1677=3^{1}13^{1}43^{1}$&$1730=2^{1}5^{1}173^{1}$&$1784=2^{3}223^{1}$&$1837=11^{1}167^{1}$&$1892=2^{2}11^{1}43^{1}$\\
$1678=2^{1}839^{1}$&$1731=3^{1}577^{1}$&$1785=3^{1}5^{1}7^{1}17^{1}$&$1838=2^{1}919^{1}$&$1893=3^{1}631^{1}$\\
$1679=23^{1}73^{1}$&$1732=2^{2}433^{1}$&$1786=2^{1}19^{1}47^{1}$&$1839=3^{1}613^{1}$&$1894=2^{1}947^{1}$\\
$1680=2^{4}3^{1}5^{1}7^{1}$&$1734=2^{1}3^{1}17^{2}$&$1788=2^{2}3^{1}149^{1}$&$1840=2^{4}5^{1}23^{1}$&$1895=5^{1}379^{1}$\\
$1681=41^{2}$&$1735=5^{1}347^{1}$&$1790=2^{1}5^{1}179^{1}$&$1841=7^{1}263^{1}$&$1896=2^{3}3^{1}79^{1}$\\
$1682=2^{1}29^{2}$&$1736=2^{3}7^{1}31^{1}$&$1791=3^{2}199^{1}$&$1842=2^{1}3^{1}307^{1}$&$1897=7^{1}271^{1}$\\
$1683=3^{2}11^{1}17^{1}$&$1737=3^{2}193^{1}$&$1792=2^{8}7^{1}$&$1843=19^{1}97^{1}$&$1898=2^{1}13^{1}73^{1}$\\
$1684=2^{2}421^{1}$&$1738=2^{1}11^{1}79^{1}$&$1793=11^{1}163^{1}$&$1844=2^{2}461^{1}$&$1899=3^{2}211^{1}$\\
$1685=5^{1}337^{1}$&$1739=37^{1}47^{1}$&$1794=2^{1}3^{1}13^{1}23^{1}$&$1845=3^{2}5^{1}41^{1}$&$1900=2^{2}5^{2}19^{1}$\\
$1686=2^{1}3^{1}281^{1}$&$1740=2^{2}3^{1}5^{1}29^{1}$&$1795=5^{1}359^{1}$&$1846=2^{1}13^{1}71^{1}$&$1902=2^{1}3^{1}317^{1}$\\
$1687=7^{1}241^{1}$&$1742=2^{1}13^{1}67^{1}$&$1796=2^{2}449^{1}$&$1848=2^{3}3^{1}7^{1}11^{1}$&$1903=11^{1}173^{1}$\\
$1688=2^{3}211^{1}$&$1743=3^{1}7^{1}83^{1}$&$1797=3^{1}599^{1}$&$1849=43^{2}$&$1904=2^{4}7^{1}17^{1}$\\
$1689=3^{1}563^{1}$&$1744=2^{4}109^{1}$&$1798=2^{1}29^{1}31^{1}$&$1850=2^{1}5^{2}37^{1}$&$1905=3^{1}5^{1}127^{1}$\\
$1690=2^{1}5^{1}13^{2}$&$1745=5^{1}349^{1}$&$1799=7^{1}257^{1}$&$1851=3^{1}617^{1}$&$1906=2^{1}953^{1}$\\
$1691=19^{1}89^{1}$&$1746=2^{1}3^{2}97^{1}$&$1800=2^{3}3^{2}5^{2}$&$1852=2^{2}463^{1}$&$1908=2^{2}3^{2}53^{1}$\\
$1692=2^{2}3^{2}47^{1}$&$1748=2^{2}19^{1}23^{1}$&$1802=2^{1}17^{1}53^{1}$&$1853=17^{1}109^{1}$&$1909=23^{1}83^{1}$\\
$1694=2^{1}7^{1}11^{2}$&$1749=3^{1}11^{1}53^{1}$&$1803=3^{1}601^{1}$&$1854=2^{1}3^{2}103^{1}$&$1910=2^{1}5^{1}191^{1}$\\
$1695=3^{1}5^{1}113^{1}$&$1750=2^{1}5^{3}7^{1}$&$1804=2^{2}11^{1}41^{1}$&$1855=5^{1}7^{1}53^{1}$&$1911=3^{1}7^{2}13^{1}$\\
$1696=2^{5}53^{1}$&$1751=17^{1}103^{1}$&$1805=5^{1}19^{2}$&$1856=2^{6}29^{1}$&$1912=2^{3}239^{1}$\\
$1698=2^{1}3^{1}283^{1}$&$1752=2^{3}3^{1}73^{1}$&$1806=2^{1}3^{1}7^{1}43^{1}$&$1857=3^{1}619^{1}$&$1914=2^{1}3^{1}11^{1}29^{1}$\\
$1700=2^{2}5^{2}17^{1}$&$1754=2^{1}877^{1}$&$1807=13^{1}139^{1}$&$1858=2^{1}929^{1}$&$1915=5^{1}383^{1}$\\
$1701=3^{5}7^{1}$&$1755=3^{3}5^{1}13^{1}$&$1808=2^{4}113^{1}$&$1859=11^{1}13^{2}$&$1916=2^{2}479^{1}$\\
$1702=2^{1}23^{1}37^{1}$&$1756=2^{2}439^{1}$&$1809=3^{3}67^{1}$&$1860=2^{2}3^{1}5^{1}31^{1}$&$1917=3^{3}71^{1}$\\
$1703=13^{1}131^{1}$&$1757=7^{1}251^{1}$&$1810=2^{1}5^{1}181^{1}$&$1862=2^{1}7^{2}19^{1}$&$1918=2^{1}7^{1}137^{1}$\\
$1704=2^{3}3^{1}71^{1}$&$1758=2^{1}3^{1}293^{1}$&$1812=2^{2}3^{1}151^{1}$&$1863=3^{4}23^{1}$&$1919=19^{1}101^{1}$\\
$1705=5^{1}11^{1}31^{1}$&$1760=2^{5}5^{1}11^{1}$&$1813=7^{2}37^{1}$&$1864=2^{3}233^{1}$&$1920=2^{7}3^{1}5^{1}$\\
$1706=2^{1}853^{1}$&$1761=3^{1}587^{1}$&$1814=2^{1}907^{1}$&$1865=5^{1}373^{1}$&$1921=17^{1}113^{1}$\\
$1707=3^{1}569^{1}$&$1762=2^{1}881^{1}$&$1815=3^{1}5^{1}11^{2}$&$1866=2^{1}3^{1}311^{1}$&$1922=2^{1}31^{2}$\\
$1708=2^{2}7^{1}61^{1}$&$1763=41^{1}43^{1}$&$1816=2^{3}227^{1}$&$1868=2^{2}467^{1}$&$1923=3^{1}641^{1}$\\
$1710=2^{1}3^{2}5^{1}19^{1}$&$1764=2^{2}3^{2}7^{2}$&$1817=23^{1}79^{1}$&$1869=3^{1}7^{1}89^{1}$&$1924=2^{2}13^{1}37^{1}$\\
$1711=29^{1}59^{1}$&$1765=5^{1}353^{1}$&$1818=2^{1}3^{2}101^{1}$&$1870=2^{1}5^{1}11^{1}17^{1}$&$1925=5^{2}7^{1}11^{1}$\\
$1712=2^{4}107^{1}$&$1766=2^{1}883^{1}$&$1819=17^{1}107^{1}$&$1872=2^{4}3^{2}13^{1}$&$1926=2^{1}3^{2}107^{1}$\\
$1713=3^{1}571^{1}$&$1767=3^{1}19^{1}31^{1}$&$1820=2^{2}5^{1}7^{1}13^{1}$&$1874=2^{1}937^{1}$&$1927=41^{1}47^{1}$\\
$1714=2^{1}857^{1}$&$1768=2^{3}13^{1}17^{1}$&$1821=3^{1}607^{1}$&$1875=3^{1}5^{4}$&$1928=2^{3}241^{1}$\\
$1715=5^{1}7^{3}$&$1769=29^{1}61^{1}$&$1822=2^{1}911^{1}$&$1876=2^{2}7^{1}67^{1}$&$1929=3^{1}643^{1}$\\
$1716=2^{2}3^{1}11^{1}13^{1}$&$1770=2^{1}3^{1}5^{1}59^{1}$&$1824=2^{5}3^{1}19^{1}$&$1878=2^{1}3^{1}313^{1}$&$1930=2^{1}5^{1}193^{1}$\\
$1717=17^{1}101^{1}$&$1771=7^{1}11^{1}23^{1}$&$1825=5^{2}73^{1}$&$1880=2^{3}5^{1}47^{1}$&$1932=2^{2}3^{1}7^{1}23^{1}$\\
$1718=2^{1}859^{1}$&$1772=2^{2}443^{1}$&$1826=2^{1}11^{1}83^{1}$&$1881=3^{2}11^{1}19^{1}$&$1934=2^{1}967^{1}$\\
$1719=3^{2}191^{1}$&$1773=3^{2}197^{1}$&$1827=3^{2}7^{1}29^{1}$&$1882=2^{1}941^{1}$&$1935=3^{2}5^{1}43^{1}$\\
$1720=2^{3}5^{1}43^{1}$&$1774=2^{1}887^{1}$&$1828=2^{2}457^{1}$&$1883=7^{1}269^{1}$&$1936=2^{4}11^{2}$\\
$1722=2^{1}3^{1}7^{1}41^{1}$&$1775=5^{2}71^{1}$&$1829=31^{1}59^{1}$&$1884=2^{2}3^{1}157^{1}$&$1937=13^{1}149^{1}$\\
$1724=2^{2}431^{1}$&$1776=2^{4}3^{1}37^{1}$&$1830=2^{1}3^{1}5^{1}61^{1}$&$1885=5^{1}13^{1}29^{1}$&$1938=2^{1}3^{1}17^{1}19^{1}$\\
$1725=3^{1}5^{2}23^{1}$&$1778=2^{1}7^{1}127^{1}$&$1832=2^{3}229^{1}$&$1886=2^{1}23^{1}41^{1}$&$1939=7^{1}277^{1}$\\
$1726=2^{1}863^{1}$&$1779=3^{1}593^{1}$&$1833=3^{1}13^{1}47^{1}$&$1887=3^{1}17^{1}37^{1}$&$1940=2^{2}5^{1}97^{1}$\\
$1727=11^{1}157^{1}$&$1780=2^{2}5^{1}89^{1}$&$1834=2^{1}7^{1}131^{1}$&$1888=2^{5}59^{1}$&$1941=3^{1}647^{1}$\\
\pagebreak
$1942=2^{1}971^{1}$&$1995=3^{1}5^{1}7^{1}19^{1}$&$2050=2^{1}5^{2}41^{1}$&$2105=5^{1}421^{1}$&$2160=2^{4}3^{3}5^{1}$\\
$1943=29^{1}67^{1}$&$1996=2^{2}499^{1}$&$2051=7^{1}293^{1}$&$2106=2^{1}3^{4}13^{1}$&$2162=2^{1}23^{1}47^{1}$\\
$1944=2^{3}3^{5}$&$1998=2^{1}3^{3}37^{1}$&$2052=2^{2}3^{3}19^{1}$&$2107=7^{2}43^{1}$&$2163=3^{1}7^{1}103^{1}$\\
$1945=5^{1}389^{1}$&$2000=2^{4}5^{3}$&$2054=2^{1}13^{1}79^{1}$&$2108=2^{2}17^{1}31^{1}$&$2164=2^{2}541^{1}$\\
$1946=2^{1}7^{1}139^{1}$&$2001=3^{1}23^{1}29^{1}$&$2055=3^{1}5^{1}137^{1}$&$2109=3^{1}19^{1}37^{1}$&$2165=5^{1}433^{1}$\\
$1947=3^{1}11^{1}59^{1}$&$2002=2^{1}7^{1}11^{1}13^{1}$&$2056=2^{3}257^{1}$&$2110=2^{1}5^{1}211^{1}$&$2166=2^{1}3^{1}19^{2}$\\
$1948=2^{2}487^{1}$&$2004=2^{2}3^{1}167^{1}$&$2057=11^{2}17^{1}$&$2112=2^{6}3^{1}11^{1}$&$2167=11^{1}197^{1}$\\
$1950=2^{1}3^{1}5^{2}13^{1}$&$2005=5^{1}401^{1}$&$2058=2^{1}3^{1}7^{3}$&$2114=2^{1}7^{1}151^{1}$&$2168=2^{3}271^{1}$\\
$1952=2^{5}61^{1}$&$2006=2^{1}17^{1}59^{1}$&$2059=29^{1}71^{1}$&$2115=3^{2}5^{1}47^{1}$&$2169=3^{2}241^{1}$\\
$1953=3^{2}7^{1}31^{1}$&$2007=3^{2}223^{1}$&$2060=2^{2}5^{1}103^{1}$&$2116=2^{2}23^{2}$&$2170=2^{1}5^{1}7^{1}31^{1}$\\
$1954=2^{1}977^{1}$&$2008=2^{3}251^{1}$&$2061=3^{2}229^{1}$&$2117=29^{1}73^{1}$&$2171=13^{1}167^{1}$\\
$1955=5^{1}17^{1}23^{1}$&$2009=7^{2}41^{1}$&$2062=2^{1}1031^{1}$&$2118=2^{1}3^{1}353^{1}$&$2172=2^{2}3^{1}181^{1}$\\
$1956=2^{2}3^{1}163^{1}$&$2010=2^{1}3^{1}5^{1}67^{1}$&$2064=2^{4}3^{1}43^{1}$&$2119=13^{1}163^{1}$&$2173=41^{1}53^{1}$\\
$1957=19^{1}103^{1}$&$2012=2^{2}503^{1}$&$2065=5^{1}7^{1}59^{1}$&$2120=2^{3}5^{1}53^{1}$&$2174=2^{1}1087^{1}$\\
$1958=2^{1}11^{1}89^{1}$&$2013=3^{1}11^{1}61^{1}$&$2066=2^{1}1033^{1}$&$2121=3^{1}7^{1}101^{1}$&$2175=3^{1}5^{2}29^{1}$\\
$1959=3^{1}653^{1}$&$2014=2^{1}19^{1}53^{1}$&$2067=3^{1}13^{1}53^{1}$&$2122=2^{1}1061^{1}$&$2176=2^{7}17^{1}$\\
$1960=2^{3}5^{1}7^{2}$&$2015=5^{1}13^{1}31^{1}$&$2068=2^{2}11^{1}47^{1}$&$2123=11^{1}193^{1}$&$2177=7^{1}311^{1}$\\
$1961=37^{1}53^{1}$&$2016=2^{5}3^{2}7^{1}$&$2070=2^{1}3^{2}5^{1}23^{1}$&$2124=2^{2}3^{2}59^{1}$&$2178=2^{1}3^{2}11^{2}$\\
$1962=2^{1}3^{2}109^{1}$&$2018=2^{1}1009^{1}$&$2071=19^{1}109^{1}$&$2125=5^{3}17^{1}$&$2180=2^{2}5^{1}109^{1}$\\
$1963=13^{1}151^{1}$&$2019=3^{1}673^{1}$&$2072=2^{3}7^{1}37^{1}$&$2126=2^{1}1063^{1}$&$2181=3^{1}727^{1}$\\
$1964=2^{2}491^{1}$&$2020=2^{2}5^{1}101^{1}$&$2073=3^{1}691^{1}$&$2127=3^{1}709^{1}$&$2182=2^{1}1091^{1}$\\
$1965=3^{1}5^{1}131^{1}$&$2021=43^{1}47^{1}$&$2074=2^{1}17^{1}61^{1}$&$2128=2^{4}7^{1}19^{1}$&$2183=37^{1}59^{1}$\\
$1966=2^{1}983^{1}$&$2022=2^{1}3^{1}337^{1}$&$2075=5^{2}83^{1}$&$2130=2^{1}3^{1}5^{1}71^{1}$&$2184=2^{3}3^{1}7^{1}13^{1}$\\
$1967=7^{1}281^{1}$&$2023=7^{1}17^{2}$&$2076=2^{2}3^{1}173^{1}$&$2132=2^{2}13^{1}41^{1}$&$2185=5^{1}19^{1}23^{1}$\\
$1968=2^{4}3^{1}41^{1}$&$2024=2^{3}11^{1}23^{1}$&$2077=31^{1}67^{1}$&$2133=3^{3}79^{1}$&$2186=2^{1}1093^{1}$\\
$1969=11^{1}179^{1}$&$2025=3^{4}5^{2}$&$2078=2^{1}1039^{1}$&$2134=2^{1}11^{1}97^{1}$&$2187=3^{7}$\\
$1970=2^{1}5^{1}197^{1}$&$2026=2^{1}1013^{1}$&$2079=3^{3}7^{1}11^{1}$&$2135=5^{1}7^{1}61^{1}$&$2188=2^{2}547^{1}$\\
$1971=3^{3}73^{1}$&$2028=2^{2}3^{1}13^{2}$&$2080=2^{5}5^{1}13^{1}$&$2136=2^{3}3^{1}89^{1}$&$2189=11^{1}199^{1}$\\
$1972=2^{2}17^{1}29^{1}$&$2030=2^{1}5^{1}7^{1}29^{1}$&$2082=2^{1}3^{1}347^{1}$&$2138=2^{1}1069^{1}$&$2190=2^{1}3^{1}5^{1}73^{1}$\\
$1974=2^{1}3^{1}7^{1}47^{1}$&$2031=3^{1}677^{1}$&$2084=2^{2}521^{1}$&$2139=3^{1}23^{1}31^{1}$&$2191=7^{1}313^{1}$\\
$1975=5^{2}79^{1}$&$2032=2^{4}127^{1}$&$2085=3^{1}5^{1}139^{1}$&$2140=2^{2}5^{1}107^{1}$&$2192=2^{4}137^{1}$\\
$1976=2^{3}13^{1}19^{1}$&$2033=19^{1}107^{1}$&$2086=2^{1}7^{1}149^{1}$&$2142=2^{1}3^{2}7^{1}17^{1}$&$2193=3^{1}17^{1}43^{1}$\\
$1977=3^{1}659^{1}$&$2034=2^{1}3^{2}113^{1}$&$2088=2^{3}3^{2}29^{1}$&$2144=2^{5}67^{1}$&$2194=2^{1}1097^{1}$\\
$1978=2^{1}23^{1}43^{1}$&$2035=5^{1}11^{1}37^{1}$&$2090=2^{1}5^{1}11^{1}19^{1}$&$2145=3^{1}5^{1}11^{1}13^{1}$&$2195=5^{1}439^{1}$\\
$1980=2^{2}3^{2}5^{1}11^{1}$&$2036=2^{2}509^{1}$&$2091=3^{1}17^{1}41^{1}$&$2146=2^{1}29^{1}37^{1}$&$2196=2^{2}3^{2}61^{1}$\\
$1981=7^{1}283^{1}$&$2037=3^{1}7^{1}97^{1}$&$2092=2^{2}523^{1}$&$2147=19^{1}113^{1}$&$2197=13^{3}$\\
$1982=2^{1}991^{1}$&$2038=2^{1}1019^{1}$&$2093=7^{1}13^{1}23^{1}$&$2148=2^{2}3^{1}179^{1}$&$2198=2^{1}7^{1}157^{1}$\\
$1983=3^{1}661^{1}$&$2040=2^{3}3^{1}5^{1}17^{1}$&$2094=2^{1}3^{1}349^{1}$&$2149=7^{1}307^{1}$&$2199=3^{1}733^{1}$\\
$1984=2^{6}31^{1}$&$2041=13^{1}157^{1}$&$2095=5^{1}419^{1}$&$2150=2^{1}5^{2}43^{1}$&$2200=2^{3}5^{2}11^{1}$\\
$1985=5^{1}397^{1}$&$2042=2^{1}1021^{1}$&$2096=2^{4}131^{1}$&$2151=3^{2}239^{1}$&$2201=31^{1}71^{1}$\\
$1986=2^{1}3^{1}331^{1}$&$2043=3^{2}227^{1}$&$2097=3^{2}233^{1}$&$2152=2^{3}269^{1}$&$2202=2^{1}3^{1}367^{1}$\\
$1988=2^{2}7^{1}71^{1}$&$2044=2^{2}7^{1}73^{1}$&$2098=2^{1}1049^{1}$&$2154=2^{1}3^{1}359^{1}$&$2204=2^{2}19^{1}29^{1}$\\
$1989=3^{2}13^{1}17^{1}$&$2045=5^{1}409^{1}$&$2100=2^{2}3^{1}5^{2}7^{1}$&$2155=5^{1}431^{1}$&$2205=3^{2}5^{1}7^{2}$\\
$1990=2^{1}5^{1}199^{1}$&$2046=2^{1}3^{1}11^{1}31^{1}$&$2101=11^{1}191^{1}$&$2156=2^{2}7^{2}11^{1}$&$2206=2^{1}1103^{1}$\\
$1991=11^{1}181^{1}$&$2047=23^{1}89^{1}$&$2102=2^{1}1051^{1}$&$2157=3^{1}719^{1}$&$2208=2^{5}3^{1}23^{1}$\\
$1992=2^{3}3^{1}83^{1}$&$2048=2^{11}$&$2103=3^{1}701^{1}$&$2158=2^{1}13^{1}83^{1}$&$2209=47^{2}$\\
$1994=2^{1}997^{1}$&$2049=3^{1}683^{1}$&$2104=2^{3}263^{1}$&$2159=17^{1}127^{1}$&$2210=2^{1}5^{1}13^{1}17^{1}$\\
\pagebreak
$2211=3^{1}11^{1}67^{1}$&$2264=2^{3}283^{1}$&$2320=2^{4}5^{1}29^{1}$&$2374=2^{1}1187^{1}$&$2430=2^{1}3^{5}5^{1}$\\
$2212=2^{2}7^{1}79^{1}$&$2265=3^{1}5^{1}151^{1}$&$2321=11^{1}211^{1}$&$2375=5^{3}19^{1}$&$2431=11^{1}13^{1}17^{1}$\\
$2214=2^{1}3^{3}41^{1}$&$2266=2^{1}11^{1}103^{1}$&$2322=2^{1}3^{3}43^{1}$&$2376=2^{3}3^{3}11^{1}$&$2432=2^{7}19^{1}$\\
$2215=5^{1}443^{1}$&$2268=2^{2}3^{4}7^{1}$&$2323=23^{1}101^{1}$&$2378=2^{1}29^{1}41^{1}$&$2433=3^{1}811^{1}$\\
$2216=2^{3}277^{1}$&$2270=2^{1}5^{1}227^{1}$&$2324=2^{2}7^{1}83^{1}$&$2379=3^{1}13^{1}61^{1}$&$2434=2^{1}1217^{1}$\\
$2217=3^{1}739^{1}$&$2271=3^{1}757^{1}$&$2325=3^{1}5^{2}31^{1}$&$2380=2^{2}5^{1}7^{1}17^{1}$&$2435=5^{1}487^{1}$\\
$2218=2^{1}1109^{1}$&$2272=2^{5}71^{1}$&$2326=2^{1}1163^{1}$&$2382=2^{1}3^{1}397^{1}$&$2436=2^{2}3^{1}7^{1}29^{1}$\\
$2219=7^{1}317^{1}$&$2274=2^{1}3^{1}379^{1}$&$2327=13^{1}179^{1}$&$2384=2^{4}149^{1}$&$2438=2^{1}23^{1}53^{1}$\\
$2220=2^{2}3^{1}5^{1}37^{1}$&$2275=5^{2}7^{1}13^{1}$&$2328=2^{3}3^{1}97^{1}$&$2385=3^{2}5^{1}53^{1}$&$2439=3^{2}271^{1}$\\
$2222=2^{1}11^{1}101^{1}$&$2276=2^{2}569^{1}$&$2329=17^{1}137^{1}$&$2386=2^{1}1193^{1}$&$2440=2^{3}5^{1}61^{1}$\\
$2223=3^{2}13^{1}19^{1}$&$2277=3^{2}11^{1}23^{1}$&$2330=2^{1}5^{1}233^{1}$&$2387=7^{1}11^{1}31^{1}$&$2442=2^{1}3^{1}11^{1}37^{1}$\\
$2224=2^{4}139^{1}$&$2278=2^{1}17^{1}67^{1}$&$2331=3^{2}7^{1}37^{1}$&$2388=2^{2}3^{1}199^{1}$&$2443=7^{1}349^{1}$\\
$2225=5^{2}89^{1}$&$2279=43^{1}53^{1}$&$2332=2^{2}11^{1}53^{1}$&$2390=2^{1}5^{1}239^{1}$&$2444=2^{2}13^{1}47^{1}$\\
$2226=2^{1}3^{1}7^{1}53^{1}$&$2280=2^{3}3^{1}5^{1}19^{1}$&$2334=2^{1}3^{1}389^{1}$&$2391=3^{1}797^{1}$&$2445=3^{1}5^{1}163^{1}$\\
$2227=17^{1}131^{1}$&$2282=2^{1}7^{1}163^{1}$&$2335=5^{1}467^{1}$&$2392=2^{3}13^{1}23^{1}$&$2446=2^{1}1223^{1}$\\
$2228=2^{2}557^{1}$&$2283=3^{1}761^{1}$&$2336=2^{5}73^{1}$&$2394=2^{1}3^{2}7^{1}19^{1}$&$2448=2^{4}3^{2}17^{1}$\\
$2229=3^{1}743^{1}$&$2284=2^{2}571^{1}$&$2337=3^{1}19^{1}41^{1}$&$2395=5^{1}479^{1}$&$2449=31^{1}79^{1}$\\
$2230=2^{1}5^{1}223^{1}$&$2285=5^{1}457^{1}$&$2338=2^{1}7^{1}167^{1}$&$2396=2^{2}599^{1}$&$2450=2^{1}5^{2}7^{2}$\\
$2231=23^{1}97^{1}$&$2286=2^{1}3^{2}127^{1}$&$2340=2^{2}3^{2}5^{1}13^{1}$&$2397=3^{1}17^{1}47^{1}$&$2451=3^{1}19^{1}43^{1}$\\
$2232=2^{3}3^{2}31^{1}$&$2288=2^{4}11^{1}13^{1}$&$2342=2^{1}1171^{1}$&$2398=2^{1}11^{1}109^{1}$&$2452=2^{2}613^{1}$\\
$2233=7^{1}11^{1}29^{1}$&$2289=3^{1}7^{1}109^{1}$&$2343=3^{1}11^{1}71^{1}$&$2400=2^{5}3^{1}5^{2}$&$2453=11^{1}223^{1}$\\
$2234=2^{1}1117^{1}$&$2290=2^{1}5^{1}229^{1}$&$2344=2^{3}293^{1}$&$2401=7^{4}$&$2454=2^{1}3^{1}409^{1}$\\
$2235=3^{1}5^{1}149^{1}$&$2291=29^{1}79^{1}$&$2345=5^{1}7^{1}67^{1}$&$2402=2^{1}1201^{1}$&$2455=5^{1}491^{1}$\\
$2236=2^{2}13^{1}43^{1}$&$2292=2^{2}3^{1}191^{1}$&$2346=2^{1}3^{1}17^{1}23^{1}$&$2403=3^{3}89^{1}$&$2456=2^{3}307^{1}$\\
$2238=2^{1}3^{1}373^{1}$&$2294=2^{1}31^{1}37^{1}$&$2348=2^{2}587^{1}$&$2404=2^{2}601^{1}$&$2457=3^{3}7^{1}13^{1}$\\
$2240=2^{6}5^{1}7^{1}$&$2295=3^{3}5^{1}17^{1}$&$2349=3^{4}29^{1}$&$2405=5^{1}13^{1}37^{1}$&$2458=2^{1}1229^{1}$\\
$2241=3^{3}83^{1}$&$2296=2^{3}7^{1}41^{1}$&$2350=2^{1}5^{2}47^{1}$&$2406=2^{1}3^{1}401^{1}$&$2460=2^{2}3^{1}5^{1}41^{1}$\\
$2242=2^{1}19^{1}59^{1}$&$2298=2^{1}3^{1}383^{1}$&$2352=2^{4}3^{1}7^{2}$&$2407=29^{1}83^{1}$&$2461=23^{1}107^{1}$\\
$2244=2^{2}3^{1}11^{1}17^{1}$&$2299=11^{2}19^{1}$&$2353=13^{1}181^{1}$&$2408=2^{3}7^{1}43^{1}$&$2462=2^{1}1231^{1}$\\
$2245=5^{1}449^{1}$&$2300=2^{2}5^{2}23^{1}$&$2354=2^{1}11^{1}107^{1}$&$2409=3^{1}11^{1}73^{1}$&$2463=3^{1}821^{1}$\\
$2246=2^{1}1123^{1}$&$2301=3^{1}13^{1}59^{1}$&$2355=3^{1}5^{1}157^{1}$&$2410=2^{1}5^{1}241^{1}$&$2464=2^{5}7^{1}11^{1}$\\
$2247=3^{1}7^{1}107^{1}$&$2302=2^{1}1151^{1}$&$2356=2^{2}19^{1}31^{1}$&$2412=2^{2}3^{2}67^{1}$&$2465=5^{1}17^{1}29^{1}$\\
$2248=2^{3}281^{1}$&$2303=7^{2}47^{1}$&$2358=2^{1}3^{2}131^{1}$&$2413=19^{1}127^{1}$&$2466=2^{1}3^{2}137^{1}$\\
$2249=13^{1}173^{1}$&$2304=2^{8}3^{2}$&$2359=7^{1}337^{1}$&$2414=2^{1}17^{1}71^{1}$&$2468=2^{2}617^{1}$\\
$2250=2^{1}3^{2}5^{3}$&$2305=5^{1}461^{1}$&$2360=2^{3}5^{1}59^{1}$&$2415=3^{1}5^{1}7^{1}23^{1}$&$2469=3^{1}823^{1}$\\
$2252=2^{2}563^{1}$&$2306=2^{1}1153^{1}$&$2361=3^{1}787^{1}$&$2416=2^{4}151^{1}$&$2470=2^{1}5^{1}13^{1}19^{1}$\\
$2253=3^{1}751^{1}$&$2307=3^{1}769^{1}$&$2362=2^{1}1181^{1}$&$2418=2^{1}3^{1}13^{1}31^{1}$&$2471=7^{1}353^{1}$\\
$2254=2^{1}7^{2}23^{1}$&$2308=2^{2}577^{1}$&$2363=17^{1}139^{1}$&$2419=41^{1}59^{1}$&$2472=2^{3}3^{1}103^{1}$\\
$2255=5^{1}11^{1}41^{1}$&$2310=2^{1}3^{1}5^{1}7^{1}11^{1}$&$2364=2^{2}3^{1}197^{1}$&$2420=2^{2}5^{1}11^{2}$&$2474=2^{1}1237^{1}$\\
$2256=2^{4}3^{1}47^{1}$&$2312=2^{3}17^{2}$&$2365=5^{1}11^{1}43^{1}$&$2421=3^{2}269^{1}$&$2475=3^{2}5^{2}11^{1}$\\
$2257=37^{1}61^{1}$&$2313=3^{2}257^{1}$&$2366=2^{1}7^{1}13^{2}$&$2422=2^{1}7^{1}173^{1}$&$2476=2^{2}619^{1}$\\
$2258=2^{1}1129^{1}$&$2314=2^{1}13^{1}89^{1}$&$2367=3^{2}263^{1}$&$2424=2^{3}3^{1}101^{1}$&$2478=2^{1}3^{1}7^{1}59^{1}$\\
$2259=3^{2}251^{1}$&$2315=5^{1}463^{1}$&$2368=2^{6}37^{1}$&$2425=5^{2}97^{1}$&$2479=37^{1}67^{1}$\\
$2260=2^{2}5^{1}113^{1}$&$2316=2^{2}3^{1}193^{1}$&$2369=23^{1}103^{1}$&$2426=2^{1}1213^{1}$&$2480=2^{4}5^{1}31^{1}$\\
$2261=7^{1}17^{1}19^{1}$&$2317=7^{1}331^{1}$&$2370=2^{1}3^{1}5^{1}79^{1}$&$2427=3^{1}809^{1}$&$2481=3^{1}827^{1}$\\
$2262=2^{1}3^{1}13^{1}29^{1}$&$2318=2^{1}19^{1}61^{1}$&$2372=2^{2}593^{1}$&$2428=2^{2}607^{1}$&$2482=2^{1}17^{1}73^{1}$\\
$2263=31^{1}73^{1}$&$2319=3^{1}773^{1}$&$2373=3^{1}7^{1}113^{1}$&$2429=7^{1}347^{1}$&$2483=13^{1}191^{1}$\\
\pagebreak
$2484=2^{2}3^{3}23^{1}$&$2534=2^{1}7^{1}181^{1}$&$2587=13^{1}199^{1}$&$2640=2^{4}3^{1}5^{1}11^{1}$&$2697=3^{1}29^{1}31^{1}$\\
$2485=5^{1}7^{1}71^{1}$&$2535=3^{1}5^{1}13^{2}$&$2588=2^{2}647^{1}$&$2641=19^{1}139^{1}$&$2698=2^{1}19^{1}71^{1}$\\
$2486=2^{1}11^{1}113^{1}$&$2536=2^{3}317^{1}$&$2589=3^{1}863^{1}$&$2642=2^{1}1321^{1}$&$2700=2^{2}3^{3}5^{2}$\\
$2487=3^{1}829^{1}$&$2537=43^{1}59^{1}$&$2590=2^{1}5^{1}7^{1}37^{1}$&$2643=3^{1}881^{1}$&$2701=37^{1}73^{1}$\\
$2488=2^{3}311^{1}$&$2538=2^{1}3^{3}47^{1}$&$2592=2^{5}3^{4}$&$2644=2^{2}661^{1}$&$2702=2^{1}7^{1}193^{1}$\\
$2489=19^{1}131^{1}$&$2540=2^{2}5^{1}127^{1}$&$2594=2^{1}1297^{1}$&$2645=5^{1}23^{2}$&$2703=3^{1}17^{1}53^{1}$\\
$2490=2^{1}3^{1}5^{1}83^{1}$&$2541=3^{1}7^{1}11^{2}$&$2595=3^{1}5^{1}173^{1}$&$2646=2^{1}3^{3}7^{2}$&$2704=2^{4}13^{2}$\\
$2491=47^{1}53^{1}$&$2542=2^{1}31^{1}41^{1}$&$2596=2^{2}11^{1}59^{1}$&$2648=2^{3}331^{1}$&$2705=5^{1}541^{1}$\\
$2492=2^{2}7^{1}89^{1}$&$2544=2^{4}3^{1}53^{1}$&$2597=7^{2}53^{1}$&$2649=3^{1}883^{1}$&$2706=2^{1}3^{1}11^{1}41^{1}$\\
$2493=3^{2}277^{1}$&$2545=5^{1}509^{1}$&$2598=2^{1}3^{1}433^{1}$&$2650=2^{1}5^{2}53^{1}$&$2708=2^{2}677^{1}$\\
$2494=2^{1}29^{1}43^{1}$&$2546=2^{1}19^{1}67^{1}$&$2599=23^{1}113^{1}$&$2651=11^{1}241^{1}$&$2709=3^{2}7^{1}43^{1}$\\
$2495=5^{1}499^{1}$&$2547=3^{2}283^{1}$&$2600=2^{3}5^{2}13^{1}$&$2652=2^{2}3^{1}13^{1}17^{1}$&$2710=2^{1}5^{1}271^{1}$\\
$2496=2^{6}3^{1}13^{1}$&$2548=2^{2}7^{2}13^{1}$&$2601=3^{2}17^{2}$&$2653=7^{1}379^{1}$&$2712=2^{3}3^{1}113^{1}$\\
$2497=11^{1}227^{1}$&$2550=2^{1}3^{1}5^{2}17^{1}$&$2602=2^{1}1301^{1}$&$2654=2^{1}1327^{1}$&$2714=2^{1}23^{1}59^{1}$\\
$2498=2^{1}1249^{1}$&$2552=2^{3}11^{1}29^{1}$&$2603=19^{1}137^{1}$&$2655=3^{2}5^{1}59^{1}$&$2715=3^{1}5^{1}181^{1}$\\
$2499=3^{1}7^{2}17^{1}$&$2553=3^{1}23^{1}37^{1}$&$2604=2^{2}3^{1}7^{1}31^{1}$&$2656=2^{5}83^{1}$&$2716=2^{2}7^{1}97^{1}$\\
$2500=2^{2}5^{4}$&$2554=2^{1}1277^{1}$&$2605=5^{1}521^{1}$&$2658=2^{1}3^{1}443^{1}$&$2717=11^{1}13^{1}19^{1}$\\
$2501=41^{1}61^{1}$&$2555=5^{1}7^{1}73^{1}$&$2606=2^{1}1303^{1}$&$2660=2^{2}5^{1}7^{1}19^{1}$&$2718=2^{1}3^{2}151^{1}$\\
$2502=2^{1}3^{2}139^{1}$&$2556=2^{2}3^{2}71^{1}$&$2607=3^{1}11^{1}79^{1}$&$2661=3^{1}887^{1}$&$2720=2^{5}5^{1}17^{1}$\\
$2504=2^{3}313^{1}$&$2558=2^{1}1279^{1}$&$2608=2^{4}163^{1}$&$2662=2^{1}11^{3}$&$2721=3^{1}907^{1}$\\
$2505=3^{1}5^{1}167^{1}$&$2559=3^{1}853^{1}$&$2610=2^{1}3^{2}5^{1}29^{1}$&$2664=2^{3}3^{2}37^{1}$&$2722=2^{1}1361^{1}$\\
$2506=2^{1}7^{1}179^{1}$&$2560=2^{9}5^{1}$&$2611=7^{1}373^{1}$&$2665=5^{1}13^{1}41^{1}$&$2723=7^{1}389^{1}$\\
$2507=23^{1}109^{1}$&$2561=13^{1}197^{1}$&$2612=2^{2}653^{1}$&$2666=2^{1}31^{1}43^{1}$&$2724=2^{2}3^{1}227^{1}$\\
$2508=2^{2}3^{1}11^{1}19^{1}$&$2562=2^{1}3^{1}7^{1}61^{1}$&$2613=3^{1}13^{1}67^{1}$&$2667=3^{1}7^{1}127^{1}$&$2725=5^{2}109^{1}$\\
$2509=13^{1}193^{1}$&$2563=11^{1}233^{1}$&$2614=2^{1}1307^{1}$&$2668=2^{2}23^{1}29^{1}$&$2726=2^{1}29^{1}47^{1}$\\
$2510=2^{1}5^{1}251^{1}$&$2564=2^{2}641^{1}$&$2615=5^{1}523^{1}$&$2669=17^{1}157^{1}$&$2727=3^{3}101^{1}$\\
$2511=3^{4}31^{1}$&$2565=3^{3}5^{1}19^{1}$&$2616=2^{3}3^{1}109^{1}$&$2670=2^{1}3^{1}5^{1}89^{1}$&$2728=2^{3}11^{1}31^{1}$\\
$2512=2^{4}157^{1}$&$2566=2^{1}1283^{1}$&$2618=2^{1}7^{1}11^{1}17^{1}$&$2672=2^{4}167^{1}$&$2730=2^{1}3^{1}5^{1}7^{1}13^{1}$\\
$2513=7^{1}359^{1}$&$2567=17^{1}151^{1}$&$2619=3^{3}97^{1}$&$2673=3^{5}11^{1}$&$2732=2^{2}683^{1}$\\
$2514=2^{1}3^{1}419^{1}$&$2568=2^{3}3^{1}107^{1}$&$2620=2^{2}5^{1}131^{1}$&$2674=2^{1}7^{1}191^{1}$&$2733=3^{1}911^{1}$\\
$2515=5^{1}503^{1}$&$2569=7^{1}367^{1}$&$2622=2^{1}3^{1}19^{1}23^{1}$&$2675=5^{2}107^{1}$&$2734=2^{1}1367^{1}$\\
$2516=2^{2}17^{1}37^{1}$&$2570=2^{1}5^{1}257^{1}$&$2623=43^{1}61^{1}$&$2676=2^{2}3^{1}223^{1}$&$2735=5^{1}547^{1}$\\
$2517=3^{1}839^{1}$&$2571=3^{1}857^{1}$&$2624=2^{6}41^{1}$&$2678=2^{1}13^{1}103^{1}$&$2736=2^{4}3^{2}19^{1}$\\
$2518=2^{1}1259^{1}$&$2572=2^{2}643^{1}$&$2625=3^{1}5^{3}7^{1}$&$2679=3^{1}19^{1}47^{1}$&$2737=7^{1}17^{1}23^{1}$\\
$2519=11^{1}229^{1}$&$2573=31^{1}83^{1}$&$2626=2^{1}13^{1}101^{1}$&$2680=2^{3}5^{1}67^{1}$&$2738=2^{1}37^{2}$\\
$2520=2^{3}3^{2}5^{1}7^{1}$&$2574=2^{1}3^{2}11^{1}13^{1}$&$2627=37^{1}71^{1}$&$2681=7^{1}383^{1}$&$2739=3^{1}11^{1}83^{1}$\\
$2522=2^{1}13^{1}97^{1}$&$2575=5^{2}103^{1}$&$2628=2^{2}3^{2}73^{1}$&$2682=2^{1}3^{2}149^{1}$&$2740=2^{2}5^{1}137^{1}$\\
$2523=3^{1}29^{2}$&$2576=2^{4}7^{1}23^{1}$&$2629=11^{1}239^{1}$&$2684=2^{2}11^{1}61^{1}$&$2742=2^{1}3^{1}457^{1}$\\
$2524=2^{2}631^{1}$&$2577=3^{1}859^{1}$&$2630=2^{1}5^{1}263^{1}$&$2685=3^{1}5^{1}179^{1}$&$2743=13^{1}211^{1}$\\
$2525=5^{2}101^{1}$&$2578=2^{1}1289^{1}$&$2631=3^{1}877^{1}$&$2686=2^{1}17^{1}79^{1}$&$2744=2^{3}7^{3}$\\
$2526=2^{1}3^{1}421^{1}$&$2580=2^{2}3^{1}5^{1}43^{1}$&$2632=2^{3}7^{1}47^{1}$&$2688=2^{7}3^{1}7^{1}$&$2745=3^{2}5^{1}61^{1}$\\
$2527=7^{1}19^{2}$&$2581=29^{1}89^{1}$&$2634=2^{1}3^{1}439^{1}$&$2690=2^{1}5^{1}269^{1}$&$2746=2^{1}1373^{1}$\\
$2528=2^{5}79^{1}$&$2582=2^{1}1291^{1}$&$2635=5^{1}17^{1}31^{1}$&$2691=3^{2}13^{1}23^{1}$&$2747=41^{1}67^{1}$\\
$2529=3^{2}281^{1}$&$2583=3^{2}7^{1}41^{1}$&$2636=2^{2}659^{1}$&$2692=2^{2}673^{1}$&$2748=2^{2}3^{1}229^{1}$\\
$2530=2^{1}5^{1}11^{1}23^{1}$&$2584=2^{3}17^{1}19^{1}$&$2637=3^{2}293^{1}$&$2694=2^{1}3^{1}449^{1}$&$2750=2^{1}5^{3}11^{1}$\\
$2532=2^{2}3^{1}211^{1}$&$2585=5^{1}11^{1}47^{1}$&$2638=2^{1}1319^{1}$&$2695=5^{1}7^{2}11^{1}$&$2751=3^{1}7^{1}131^{1}$\\
$2533=17^{1}149^{1}$&$2586=2^{1}3^{1}431^{1}$&$2639=7^{1}13^{1}29^{1}$&$2696=2^{3}337^{1}$&$2752=2^{6}43^{1}$\\
\end{longtable}}\index{Numero!fattore}
\section{Tavola pitagorica}
\label{sec:TavolaPitagorica}

%\begin{table}[!ht]
%\centering
%\renewcommand\arraystretch{1.9}
%\begin{tabular}{*{11}{|m{.5cm}}|}
%\cline{2-11}
%\multicolumn{1}{c|} {}& 1 & 2 & 3 & 4 & 5 & 6 & 7 & 8 & 9 & 10 \\\hline
%1 & 1 & 2 & 3 & 4 & 5 & 6 & 7 & 8 & 9 & 10 \\\hline
%2 & 2 & 4 & 6 & 8 & 10 & 12 & 14 & 16 & 18 & 20 \\\hline
%3 & 3 & 6 & 9 & 12 & 15 & 18 & 21 & 24 & 27 & 30 \\\hline
%4 & 4 & 8 & 12 & 16 & 20 & 24 & 28 & 32 & 36 & 40 \\\hline
%5 & 5 & 10 & 15 & 20 & 25 & 30 & 35 & 40 & 45 & 50 \\\hline
%6 & 6 & 12 & 18 & 24 & 30 & 36 & 42 & 48 & 54 & 60 \\\hline
%7 & 7 & 14 & 21 & 28 & 35 & 42 & 49 & 56 & 63 & 70 \\\hline
%8 & 8 & 16 & 24 & 32 & 40 & 48 & 56 & 64 & 72 & 80 \\\hline
%9 & 9 & 18 & 27 & 36 & 45 & 54 & 63 & 72 & 81 & 90 \\\hline
%10 & 10 & 20 & 30 & 40 & 50 & 60 & 70 & 80 & 90 & 100 \\\hline
%\end{tabular} 
%\caption{Tavola pitagorica}
%\label{tab:Tavolapitagorica}
%\end{table}
\begin{table}[H]
\centering
%codice Enrico Gregorio Guit
\def\mybox#1{\fbox{\kern.5cm
  \vrule height .5cm depth .5cm width 0pt \makebox(0,0){#1}%
  \kern.5cm}}
\def\riga#1{%
  #1&\number\numexpr#1*1\relax
    &\number\numexpr#1*2\relax
    &\number\numexpr#1*3\relax
    &\number\numexpr#1*4\relax
    &\number\numexpr#1*5\relax
    &\number\numexpr#1*6\relax
    &\number\numexpr#1*7\relax
    &\number\numexpr#1*8\relax
    &\number\numexpr#1*9\relax
    &\number\numexpr#1*10\relax}
\leavevmode\vbox{\fboxsep=0pt \offinterlineskip
\halign{&\mybox{#}\kern-\fboxrule\cr
\omit&1&2&3&4&5&6&7&8&9&10\cr\noalign{\kern-\fboxrule}
\riga{1}\cr\noalign{\kern-\fboxrule}
\riga{2}\cr\noalign{\kern-\fboxrule}
\riga{3}\cr\noalign{\kern-\fboxrule}
\riga{4}\cr\noalign{\kern-\fboxrule}
\riga{5}\cr\noalign{\kern-\fboxrule}
\riga{6}\cr\noalign{\kern-\fboxrule}
\riga{7}\cr\noalign{\kern-\fboxrule}
\riga{8}\cr\noalign{\kern-\fboxrule}
\riga{9}\cr\noalign{\kern-\fboxrule}
\riga{10}\cr\noalign{\kern-\fboxrule}
}} 
\label{tab:tavolepitagorica}
\end{table}\index{Tavola!pitagorica}

\section{Fattoriali}
\citaoeis{A000142}
\begin{center}
	\begin{tabular}{Q}
\toprule
n&n!\\	
\midrule
1&1\\
2&2\\
3&6\\
4&24\\
5&120\\
6&720\\
7&5040\\
8&40320\\
9&362880\\
10&3628800\\
11&39916800\\
12&479001600\\
13&6227020800\\
14&87178291200\\
15&1307674368000\\
16&20922789888000\\
17&355687428096000\\
18&6402373705728000\\
19&121645100408832000\\
20&2432902008176640000\\
\bottomrule
\end{tabular}\captionof{table}{Numeri fattoriali} 
\end{center}\index{Numero!fattoriale}
\section{Triangolo di Tartaglia}
\begin{center}
	\includestandalone{grafici/tartaglia}
	\captionof{figure}{Triangolo di Tartaglia}
\end{center}\index{Triangolo!Tartaglia}

\section{Criteri di divisibilità}
\label{sec:CriteridiDivisibilita}
\begin{center}
	\begin{tabular}{Ccp{0.25\textwidth}p{0.25\textwidth}}
\toprule  N&  &\multicolumn{1}{c}{Regola}&\multicolumn{1}{c}{Esempio}    \\ 
\midrule 2 & Se & l'ultima cifra è pari, cioè è  \numlist{0;2;4;6;8}& \\ 
3 & Se & la somma delle cifre è divisibile per tre & \num{375} $3+7+5=15\div3=5$ infatti $375\div 3=125$ \\ 
 4 & Se & le ultime due cifre sono divisibili per quattro o sono due zeri $\mathbf{00}$& $4\mathbf{60}$ $60\div 4=15$ $469\div 4=115$ \\
 5 & Se & l'ultima cifra è  cinque o zero&\num{10}, \num{735} \\  
 6 & Se & è divisibile contemporaneamente per tre e per due& \num{54} \\  
 8 & Se & ultime tre cifre sono divisibili per 8 o sono tre zeri $\mathbf{000}$& $9\mathbf{872}$ le ultime tre cifre sono divisibili per otto $872\div 8= 109$ $9872\div 8=1234$ \\  
 9 & Se & la somma delle cifre è divisibile per 9& $405$ $4+0+5=9$ $405\div9=45$  \\
 10 & Se & l'ultima sua cifra è zero& \num{100},\num{140}\\
 11 & Se& la differenza della somma delle cifre di posto pari e le cifre di posto dispari è zero o si divide per undici&  $25652$ $(5+5)-(2+6+2)=0$ $25652\div 11=2332$. Esempio \num{4145889} $(4+4+8+9)-(1+5+8=11)$ $4145889\div 11=376899$  \\    
 12 & Se & è divisibile contemporaneamente per tre e per quattro&\num{144}  \\  
 25 & Se & il numero  formato dalle ultime due cifre è divisibile per venticinque&\\
\bottomrule
\end{tabular}
\end{center}





\chapter{Altro}
\section{Numeri di Fibonacci}
\citaoeis{A000045}
\begin{center}
	\begin{tabular}{*4{Q@{\hspace*{7mm}}Q}}%{*{10}{R}}
\toprule
n&F&n&F&n&F&n&F&n&F\\
%\multicolumn{1}{C}{n} &\multicolumn{1}{C}{F}&\multicolumn{1}{C}{n}&\multicolumn{1}{C}{F}&\multicolumn{1}{C}{n}&\multicolumn{1}{C}{F}&\multicolumn{1}{C}{n}&\multicolumn{1}{C}{F}&\multicolumn{1}{C}{n}&\multicolumn{1}{C}{F}\\
\midrule
1&1&11&89&21&10946&31&1346269&41&165580141\\
2&1&12&144&22&17711&32&2178309&42&267914296\\
3&2&13&233&23&28657&33&3524578&43&433494437\\
4&3&14&377&24&46368&34&5702887&44&701408733\\
5&5&15&610&25&75025&35&9227465&45&1134903170\\
6&8&16&987&26&121393&36&14930352&46&1836311903\\
7&13&17&1597&27&196418&37&24157817&47&2971215073\\
8&21&18&2584&28&317811&38&39088169&48&4807526976\\
9&34&19&4181&29&514229&39&63245986&49&7778742049\\
10&55&20&6765&30&832040&40&102334155&50&12586269025\\
\bottomrule
\end{tabular}\captionof{table}{Numeri di Fibonacci}\index{Numero!Fibonacci} 
\end{center}
\section{Numeri di Lucas}
\citaoeis{A000032}
\begin{center}
		\begin{tabular}{*4{Q@{\hspace*{7mm}}Q}}%{*{10}{R}}
		\toprule
		n&L&n&L&n&L&n&L&n&L\\
		\midrule
		1&2&11&123&21&15127&31&1860498&41&228826127\\
		2&1&12&199&22&24476&32&3010349&42&370248451\\
		3&3&13&322&23&39603&33&4870847&43&599074578\\
		4&4&14&521&24&64079&34&7881196&44&969323029\\
		5&7&15&843&25&103682&35&12752043&45&1568397607\\
		6&11&16&1364&26&167761&36&20633239&46&2537720636\\
		7&18&17&2207&27&271443&37&33385282&47&4106118243\\
		8&29&18&3571&28&439204&38&54018521&48&6643838879\\
		9&47&19&5778&29&710647&39&87403803&49&10749957122\\
		10&76&20&9349&30&1149851&40&141422324&50&17393796001\\
		\bottomrule
	\end{tabular}\captionof{table}{Numeri di Lucas}\index{Numero!Lucas} 
\end{center}
\section{Terne pitagoriche}
\citaoeis{A263728}
\begin{center}
\footnotesize
\begin{longtable}{*5{A@{\hspace*{5mm}}A}}
	\toprule\endhead
\bottomrule \endfoot
3&4&5&67&2244&2245&119&120&169&165&532&557&220&3021&3029&315&988&1037\\
5&12&13&68&285&293&119&7080&7081&165&1508&1517&225&272&353&315&1972&1997\\
7&24&25&68&1155&1157&120&209&241&165&13612&13613&228&325&397&316&6237&6245\\
8&15&17&69&260&269&120&391&409&167&13944&13945&228&1435&1453&319&360&481\\
9&40&41&69&2380&2381&120&3599&3601&168&425&457&228&3245&3253&320&999&1049\\
11&60&61&71&2520&2521&121&7320&7321&168&775&793&231&520&569&321&5720&5729\\
12&35&37&72&1295&1297&123&836&845&168&7055&7057&231&2960&2969&324&6557&6565\\
13&84&85&73&2664&2665&123&7564&7565&169&14280&14281&232&825&857&327&5936&5945\\
15&112&113&75&308&317&124&957&965&171&14620&14621&235&1092&1117&328&1665&1697\\
16&63&65&75&2812&2813&124&3843&3845&172&1845&1853&236&3477&3485&329&1080&1129\\
17&144&145&76&357&365&125&7812&7813&172&7395&7397&237&3116&3125&332&6885&6893\\
19&180&181&76&1443&1445&127&8064&8065&173&14964&14965&240&551&601&333&644&725\\
20&21&29&77&2964&2965&128&4095&4097&175&288&337&240&1591&1609&335&2232&2257\\
20&99&101&79&3120&3121&129&920&929&175&15312&15313&244&3717&3725&336&377&505\\
21&220&221&80&1599&1601&129&8320&8321&176&7743&7745&245&1188&1213&336&527&625\\
23&264&265&81&3280&3281&131&8580&8581&177&1736&1745&248&945&977&336&3127&3145\\
24&143&145&83&3444&3445&132&475&493&177&15664&15665&249&3440&3449&339&6380&6389\\
25&312&313&84&187&205&132&1085&1093&179&16020&16021&252&275&373&340&1131&1181\\
27&364&365&84&437&445&132&4355&4357&180&299&349&252&3965&3973&340&7221&7229\\
28&45&53&84&1763&1765&133&156&205&180&2021&2029&255&1288&1313&341&420&541\\
28&195&197&85&132&157&133&8844&8845&180&8099&8101&255&3608&3617&344&1833&1865\\
29&420&421&85&3612&3613&135&352&377&181&16380&16381&259&660&709&345&2368&2393\\
31&480&481&87&416&425&135&9112&9113&183&1856&1865&260&651&701&345&6608&6617\\
32&255&257&87&3784&3785&136&273&305&183&16744&16745&260&4221&4229&348&805&877\\
33&56&65&88&105&137&136&4623&4625&184&513&545&261&380&461&348&3355&3373\\
33&544&545&88&1935&1937&137&9384&9385&184&8463&8465&264&1073&1105&348&7565&7573\\
35&612&613&89&3960&3961&139&9660&9661&185&672&697&264&1927&1945&355&2508&2533\\
36&77&85&91&4140&4141&140&171&221&185&17112&17113&265&1392&1417&356&7917&7925\\
36&323&325&92&525&533&140&1221&1229&187&17484&17485&267&3956&3965&357&1276&1325\\
37&684&685&92&2115&2117&140&4899&4901&188&2205&2213&268&4485&4493&357&7076&7085\\
39&80&89&93&476&485&141&1100&1109&188&8835&8837&273&736&785&360&1271&1321\\
39&760&761&93&4324&4325&141&9940&9941&189&340&389&273&4136&4145&360&2009&2041\\
40&399&401&95&168&193&143&10224&10225&189&17860&17861&276&493&565&363&7316&7325\\
41&840&841&95&4512&4513&144&5183&5185&191&18240&18241&276&2107&2125&364&627&725\\
43&924&925&96&247&265&145&408&433&192&1015&1033&276&4757&4765&364&8277&8285\\
44&117&125&96&2303&2305&145&10512&10513&192&9215&9217&279&440&521&365&2652&2677\\
44&483&485&97&4704&4705&147&1196&1205&193&18624&18625&280&351&449&368&465&593\\
45&1012&1013&99&4900&4901&147&10804&10805&195&748&773&280&759&809&369&800&881\\
47&1104&1105&100&621&629&148&1365&1373&195&2108&2117&280&1209&1241&371&1380&1429\\
48&55&73&100&2499&2501&148&5475&5477&195&19012&19013&284&5037&5045&372&925&997\\
48&575&577&101&5100&5101&149&11100&11101&196&2397&2405&285&1612&1637&372&3835&3853\\
49&1200&1201&103&5304&5305&151&11400&11401&196&9603&9605&285&4508&4517&372&8645&8653\\
51&140&149&104&153&185&152&345&377&197&19404&19405&287&816&865&375&7808&7817\\
51&1300&1301&104&2703&2705&152&5775&5777&199&19800&19801&291&4700&4709&376&2193&2225\\
52&165&173&105&208&233&153&11704&11705&200&609&641&292&5325&5333&380&1419&1469\\
52&675&677&105&608&617&155&468&493&200&9999&10001&295&1728&1753&380&9021&9029\\
53&1404&1405&105&5512&5513&155&12012&12013&201&2240&2249&296&1353&1385&381&8060&8069\\
55&1512&1513&107&5724&5725&156&667&685&203&396&445&297&304&425&384&4087&4105\\
56&783&785&108&725&733&156&1517&1525&204&253&325&300&589&661&385&552&673\\
57&176&185&108&2915&2917&156&6083&6085&204&1147&1165&300&2491&2509&385&1488&1537\\
57&1624&1625&109&5940&5941&157&12324&12325&204&2597&2605&300&5621&5629&385&2952&2977\\
59&1740&1741&111&680&689&159&1400&1409&205&828&853&301&900&949&387&884&965\\
60&91&109&111&6160&6161&159&12640&12641&207&224&305&303&5096&5105&388&9405&9413\\
60&221&229&112&3135&3137&160&231&281&212&2805&2813&305&1848&1873&392&2385&2417\\
60&899&901&113&6384&6385&160&6399&6401&213&2516&2525&308&435&533&393&8576&8585\\
61&1860&1861&115&252&277&161&240&289&215&912&937&308&5925&5933&395&3108&3133\\
63&1984&1985&115&6612&6613&161&12960&12961&216&713&745&309&5300&5309&396&403&565\\
64&1023&1025&116&837&845&163&13284&13285&217&456&505&312&1505&1537&396&9797&9805\\
65&72&97&116&3363&3365&164&1677&1685&219&2660&2669&312&2695&2713&399&1600&1649\\
65&2112&2113&117&6844&6845&164&6723&6725&220&459&509&315&572&653&399&8840&8849\\
\newpage
400&561&689&495&4888&4913&585&6832&6857&696&697&985&801&3920&4001&917&8556&8605\\
405&3268&3293&496&897&1025&588&2365&2437&696&7553&7585&803&2604&2725&920&8439&8489\\
407&624&745&497&2496&2545&588&9595&9613&700&2451&2549&804&4453&4525&923&2436&2605\\
408&2585&2617&501&13940&13949&591&19400&19409&704&903&1145&805&6588&6637&924&1643&1885\\
408&4615&4633&504&703&865&592&1305&1433&705&992&1217&805&12948&12973&924&4307&4405\\
411&9380&9389&504&1247&1345&595&3588&3637&705&9928&9953&812&3315&3413&924&5893&5965\\
413&1716&1765&504&3953&3985&595&7068&7093&707&5076&5125&815&13272&13297&927&5264&5345\\
415&3432&3457&505&5088&5113&600&5609&5641&708&3445&3517&816&2537&2665&931&1020&1381\\
417&9656&9665&507&14276&14285&600&9991&10009&711&3080&3161&819&1900&2069&935&1368&1657\\
420&851&949&511&2640&2689&603&2204&2285&712&7905&7937&819&4100&4181&935&3552&3673\\
420&1189&1261&515&5292&5317&605&7308&7333&715&1428&1597&819&6820&6869&935&17472&17497\\
420&1739&1789&516&1813&1885&609&3760&3809&715&2052&2173&820&1581&1781&936&1127&1465\\
420&4891&4909&516&7387&7405&611&1020&1189&715&10212&10237&820&6699&6749&936&2623&2785\\
423&1064&1145&517&1044&1165&612&1075&1237&720&1519&1681&825&2752&2873&940&2109&2309\\
424&2793&2825&519&14960&14969&615&728&953&720&1961&2089&828&2035&2197&940&8811&8861\\
427&1836&1885&520&2679&2729&615&7552&7577&720&5159&5209&832&855&1193&944&3417&3545\\
429&460&629&520&4209&4241&616&663&905&721&5280&5329&833&1056&1345&945&9088&9137\\
429&700&821&525&2788&2837&616&1887&1985&728&2655&2753&835&13932&13957&945&17848&17873\\
429&10220&10229&525&15308&15317&616&5913&5945&728&8265&8297&836&1323&1565&948&6205&6277\\
432&665&793&528&1025&1153&620&861&1061&731&780&1069&840&1081&1369&949&2580&2749\\
435&3772&3797&528&7735&7753&620&3819&3869&732&3685&3757&840&3551&3649&952&4575&4673\\
435&10508&10517&531&1700&1781&623&3936&3985&735&1088&1313&840&7031&7081&955&18228&18253\\
440&1911&1961&532&1395&1493&624&1457&1585&735&10792&10817&845&14268&14293&957&3724&3845\\
440&3009&3041&533&756&925&627&1564&1685&737&2184&2305&847&7296&7345&959&9360&9409\\
441&1160&1241&535&5712&5737&632&6225&6257&740&1269&1469&848&2745&2873&960&9191&9241\\
444&1333&1405&536&4473&4505&635&8052&8077&740&5451&5501&852&5005&5077&963&5684&5765\\
444&5467&5485&537&16016&16025&636&2773&2845&741&1540&1709&855&4472&4553&965&18612&18637\\
445&3948&3973&539&1140&1261&637&1116&1285&744&817&1105&855&14608&14633&969&1120&1481\\
447&11096&11105&540&629&829&639&2480&2561&744&8633&8665&860&1749&1949&969&1480&1769\\
448&975&1073&540&2891&2941&640&4071&4121&745&11088&11113&860&7371&7421&973&9636&9685\\
451&780&901&543&16376&16385&644&2067&2165&747&3404&3485&861&7540&7589&975&2728&2897\\
453&11396&11405&545&5928&5953&645&812&1037&748&1035&1277&865&14952&14977&976&3657&3785\\
455&528&697&549&1820&1901&645&8308&8333&749&5700&5749&868&3795&3893&979&3900&4021\\
455&2088&2137&552&4745&4777&648&6545&6577&752&2145&2273&869&3060&3181&980&2301&2501\\
455&4128&4153&552&8455&8473&649&1680&1801&755&11388&11413&871&2160&2329&980&9579&9629\\
456&3233&3265&553&3096&3145&651&4300&4349&756&2867&2965&873&4664&4745&981&5900&5981\\
456&5767&5785&555&572&797&655&8568&8593&759&2320&2441&875&7788&7837&984&1537&1825\\
460&2091&2141&555&6148&6173&656&1617&1745&760&5751&5801&876&5293&5365&987&9916&9965\\
464&777&905&555&17108&17117&657&2624&2705&760&9009&9041&880&1479&1721&988&1275&1613\\
465&4312&4337&559&840&1009&660&779&1021&763&5916&5965&880&2961&3089&996&6853&6925\\
465&12008&12017&560&1161&1289&660&989&1189&765&868&1157&880&7719&7769&1001&2880&3049\\
468&595&757&560&1551&1649&660&2989&3061&765&3572&3653&884&987&1325&1001&4080&4201\\
469&2220&2269&560&3111&3161&660&4331&4381&765&11692&11717&885&1628&1853&1001&10200&10249\\
471&12320&12329&561&1240&1361&663&1216&1385&767&1656&1825&885&15652&15677&1003&1596&1885\\
472&3465&3497&561&17480&17489&664&6873&6905&776&9393&9425&888&1225&1513&1005&2132&2357\\
473&864&985&564&2173&2245&665&4488&4537&777&6136&6185&889&8040&8089&1007&1224&1585\\
476&1107&1205&564&8827&8845&665&8832&8857&780&1421&1621&891&3220&3341&1008&3055&3217\\
477&1364&1445&565&6372&6397&671&1800&1921&780&4189&4261&893&924&1285&1008&3905&4033\\
480&2279&2329&567&3256&3305&672&2255&2353&780&6059&6109&895&16008&16033&1008&5135&5233\\
480&6391&6409&568&5025&5057&679&4680&4729&781&2460&2581&896&4047&4145&1012&1995&2237\\
481&600&769&573&18236&18245&680&4599&4649&784&2337&2465&897&2296&2465&1015&10488&10537\\
483&2356&2405&576&943&1105&680&7209&7241&785&12312&12337&900&2419&2581&1017&6344&6425\\
483&12956&12965&579&18620&18629&684&1363&1525&791&6360&6409&901&1260&1549&1020&2501&2701\\
485&4692&4717&580&741&941&685&9372&9397&792&1175&1417&903&8296&8345&1020&7189&7261\\
488&3705&3737&580&3339&3389&688&1785&1913&792&1855&2017&905&16368&16393&1023&4264&4385\\
489&13280&13289&581&3420&3469&689&1320&1489&792&9785&9817&909&5060&5141&1027&3036&3205\\
492&1645&1717&583&1344&1465&693&1924&2045&793&1776&1945&912&3185&3313&1032&1705&1993\\
492&6715&6733&584&5313&5345&693&2924&3005&795&1292&1517&913&3384&3505&1035&6572&6653\\
495&952&1073&585&928&1097&693&4876&4925&795&12628&12653&915&1748&1973&1036&1173&1565\\
495&1472&1553&585&2072&2153&695&9648&9673&799&960&1249&915&16732&16757&1036&5427&5525\\
\newpage
1037&1716&2005&1164&9373&9445&1300&2331&2669&1440&6319&6481&1584&7663&7825&1743&3224&3665\\
1040&1431&1769&1168&5265&5393&1305&10472&10553&1441&8520&8641&1584&9737&9865&1748&1755&2477\\
1040&4161&4289&1169&13920&13969&1309&2820&3109&1443&6076&6245&1595&10452&10573&1749&12580&12701\\
1043&11076&11125&1173&2236&2525&1309&7020&7141&1449&1720&2249&1596&3053&3445&1751&5160&5449\\
1044&3283&3445&1176&2257&2545&1309&17460&17509&1449&12920&13001&1599&7480&7649&1752&5185&5473\\
1045&1332&1693&1177&5664&5785&1311&1360&1889&1455&4592&4817&1605&5612&5837&1755&9028&9197\\
1045&4452&4573&1179&8540&8621&1311&2200&2561&1456&2967&3305&1608&4345&4633&1760&2769&3281\\
1053&3196&3365&1180&3381&3581&1312&1425&1937&1456&8217&8345&1611&15980&16061&1760&6279&6521\\
1056&2183&2425&1183&14256&14305&1313&5016&5185&1460&5229&5429&1612&3675&4013&1764&2077&2725\\
1057&11376&11425&1185&3008&3233&1316&2013&2405&1463&2784&3145&1615&3432&3793&1764&9523&9685\\
1060&2709&2909&1188&2795&3037&1316&8787&8885&1463&8784&8905&1615&4368&4657&1767&4144&4505\\
1064&5727&5825&1196&1947&2285&1320&1711&2161&1464&3577&3865&1617&10744&10865&1768&2415&2993\\
1065&2408&2633&1197&1804&2165&1320&2881&3169&1467&13244&13325&1620&6461&6661&1768&4455&4793\\
1067&4644&4765&1197&8804&8885&1320&3479&3721&1469&6300&6469&1625&7728&7897&1771&2700&3229\\
1068&7885&7957&1197&14596&14645&1328&6825&6953&1476&6643&6805&1628&5355&5597&1771&12900&13021\\
1071&1840&2129&1199&5880&6001&1332&5395&5557&1479&3640&3929&1629&16340&16421&1775&2208&2833\\
1071&7040&7121&1200&5561&5689&1335&3848&4073&1484&2613&3005&1632&2015&2593&1780&7821&8021\\
1071&11680&11729&1204&1653&2045&1337&18216&18265&1485&9052&9173&1632&2345&2857&1781&9300&9469\\
1072&4425&4553&1204&7347&7445&1339&5220&5389&1488&8585&8713&1633&2256&2785&1785&3392&3833\\
1079&3360&3529&1207&2376&2665&1340&4389&4589&1491&2300&2741&1635&5828&6053&1785&5368&5657\\
1085&11988&12037&1209&4240&4409&1341&11060&11141&1495&1848&2377&1639&11040&11161&1785&6968&7193\\
1089&7280&7361&1211&14940&14989&1343&2976&3265&1495&6528&6697&1649&4560&4849&1793&13224&13345\\
1092&1325&1717&1220&3621&3821&1344&9167&9265&1496&1647&2225&1651&7980&8149&1800&9919&10081\\
1092&1595&1933&1221&6100&6221&1349&2340&2701&1496&4503&4745&1652&3285&3677&1804&6603&6845\\
1092&6035&6133&1224&4543&4705&1351&18600&18649&1501&2940&3301&1653&3604&3965&1807&9576&9745\\
1092&8245&8317&1232&3015&3257&1353&7504&7625&1503&13904&13985&1656&8383&8545&1809&1880&2609\\
1095&2552&2777&1232&5865&5993&1357&1476&2005&1504&1953&2465&1659&2900&3341&1815&7208&7433\\
1099&12300&12349&1232&7695&7793&1359&11360&11441&1507&9324&9445&1660&6789&6989&1817&2856&3385\\
1100&2379&2621&1233&9344&9425&1360&7161&7289&1508&3195&3533&1661&11340&11461&1819&5580&5869\\
1104&4697&4825&1235&1932&2293&1364&3723&3965&1513&3816&4105&1664&3927&4265&1820&4029&4421\\
1105&1968&2257&1235&4428&4597&1365&1892&2333&1515&4988&5213&1665&17072&17153&1820&4731&5069\\
1105&3528&3697&1239&1520&1961&1365&4028&4253&1520&8961&9089&1672&5655&5897&1820&8181&8381\\
1111&5040&5161&1239&15640&15689&1365&5428&5597&1521&14240&14321&1675&1932&2557&1824&1943&2665\\
1113&1184&1625&1241&2520&2809&1368&5695&5857&1525&1548&2173&1677&8236&8405&1824&2993&3505\\
1113&12616&12665&1243&6324&6445&1375&7752&7873&1529&9600&9721&1679&2400&2929&1825&2352&2977\\
1116&3763&3925&1245&3332&3557&1376&1593&2105&1533&2444&2885&1680&2911&3361&1833&9856&10025\\
1120&6351&6449&1248&1265&1777&1377&3136&3425&1539&3100&3461&1683&4756&5045&1836&2627&3205\\
1121&1560&1921&1248&2135&2473&1380&1891&2341&1540&2829&3221&1683&11644&11765&1837&13884&14005\\
1125&7772&7853&1251&9620&9701&1380&4661&4861&1540&4779&5021&1683&17444&17525&1843&4524&4885\\
1128&2065&2353&1253&15996&16045&1387&2484&2845&1540&5829&6029&1691&3780&4141&1848&5785&6073\\
1131&3700&3869&1260&1829&2221&1391&5640&5809&1541&1980&2509&1692&1885&2533&1848&6935&7177\\
1133&5244&5365&1260&3869&4069&1392&7505&7633&1545&5192&5417&1692&8755&8917&1853&5796&6085\\
1136&4977&5105&1260&4819&4981&1395&11972&12053&1547&3996&4285&1695&6272&6497&1859&14220&14341\\
1139&2100&2389&1260&8051&8149&1397&8004&8125&1547&6996&7165&1696&2553&3065&1860&3619&4069\\
1140&1219&1669&1261&4620&4789&1400&9951&10049&1548&7315&7477&1700&2211&2789&1860&8549&8749\\
1140&3149&3349&1264&6177&6305&1403&1596&2125&1551&9880&10001&1703&8496&8665&1863&3016&3545\\
1140&8989&9061&1265&6552&6673&1404&2747&3085&1552&9345&9473&1704&4897&5185&1869&3740&4181\\
1141&13260&13309&1267&16356&16405&1407&2024&2465&1557&14924&15005&1705&11952&12073&1872&5015&5353\\
1143&8024&8105&1272&2665&2953&1408&3975&4217&1560&2479&2929&1708&3525&3917&1876&4293&4685\\
1144&1767&2105&1273&2064&2425&1411&3300&3589&1560&3431&3769&1716&4187&4525&1881&4720&5081\\
1144&2583&2825&1275&2668&2957&1413&12284&12365&1560&4081&4369&1716&5963&6205&1881&14560&14681\\
1148&1485&1877&1276&3243&3485&1416&3337&3625&1564&1827&2405&1717&4956&5245&1885&10428&10597\\
1148&6675&6773&1281&1640&2081&1417&5856&6025&1568&2145&2657&1719&18200&18281&1887&6016&6305\\
1152&4015&4177&1281&16720&16769&1419&8260&8381&1573&7236&7405&1725&2068&2693&1888&3225&3737\\
1155&1292&1733&1287&4816&4985&1420&4941&5141&1575&1672&2297&1725&2548&3077&1892&7275&7517\\
1155&2852&3077&1287&6784&6905&1424&7857&7985&1575&15272&15353&1727&12264&12385&1896&6097&6385\\
1155&5452&5573&1287&10184&10265&1425&2632&2993&1577&3264&3625&1729&3960&4321&1900&2139&2861\\
1155&13588&13637&1288&8415&8513&1428&1475&2053&1580&6141&6341&1729&8760&8929&1903&14904&15025\\
1157&3876&4045&1295&17088&17137&1428&2405&2797&1581&4180&4469&1740&3139&3589&1904&2847&3425\\
1159&1680&2041&1296&6497&6625&1440&1769&2281&1584&5063&5305&1740&7469&7669&1905&7952&8177\\
\newpage
1908&2485&3133&2071&5760&6121&2261&6900&7261&2440&3321&4121&2660&4539&5261&2860&3741&4709\\
1909&3180&3709&2075&3132&3757&2261&8700&8989&2444&8667&9005&2660&8829&9221&2869&11220&11581\\
1911&10720&10889&2079&2600&3329&2265&11288&11513&2445&13172&13397&2667&7844&8285&2871&4480&5321\\
1917&2156&2885&2079&17800&17921&2268&6365&6757&2448&4895&5473&2668&2835&3893&2873&14136&14425\\
1919&4920&5281&2080&3969&4481&2272&4785&5297&2449&2640&3601&2669&12180&12469&2875&7548&8077\\
1920&3871&4321&2080&6231&6569&2275&3828&4453&2451&8140&8501&2675&5412&6037&2877&9164&9605\\
1921&6240&6529&2085&9548&9773&2275&15228&15397&2457&3776&4505&2679&9760&10121&2889&5360&6089\\
1924&5307&5645&2091&7420&7709&2277&4636&5165&2460&6499&6949&2680&4089&4889&2907&11524&11885\\
1925&2652&3277&2093&3876&4405&2280&2849&3649&2461&5460&5989&2684&3237&4205&2907&14476&14765\\
1925&15252&15373&2093&12876&13045&2280&3239&3961&2464&5673&6185&2685&15908&16133&2912&8025&8537\\
1932&4565&4957&2100&5429&5821&2280&5551&6001&2465&3192&4033&2688&3655&4537&2919&9440&9881\\
1937&11016&11185&2108&3555&4133&2280&8881&9169&2465&10368&10657&2691&6580&7109&2920&4929&5729\\
1940&9309&9509&2109&5980&6341&2288&7575&7913&2475&4588&5213&2697&3304&4265&2921&7800&8329\\
1947&15604&15725&2112&9095&9337&2289&5720&6161&2489&8400&8761&2697&3904&4745&2924&7107&7685\\
1952&3465&3977&2117&2244&3085&2291&2700&3541&2492&7725&8117&2703&12496&12785&2925&6532&7157\\
1955&3348&3877&2119&13200&13369&2295&3248&3977&2496&9047&9385&2715&16268&16493&2928&3145&4297\\
1955&6468&6757&2120&2409&3209&2295&8968&9257&2499&10660&10949&2716&9213&9605&2929&4680&5521\\
1957&5124&5485&2121&4880&5321&2299&7140&7501&2505&13832&14057&2717&10044&10405&2937&3416&4505\\
1960&2001&2801&2124&3157&3805&2301&15580&15749&2507&5676&6205&2720&6111&6689&2940&9379&9829\\
1963&11316&11485&2125&7668&7957&2323&4836&5365&2508&2765&3733&2720&6969&7481&2941&14820&15109\\
1965&8468&8693&2128&2775&3497&2324&6693&7085&2508&3995&4717&2725&5628&6253&2943&5576&6305\\
1969&15960&16081&2132&6555&6893&2325&2332&3293&2511&2800&3761&2727&4736&5465&2944&3567&4625\\
1971&2300&3029&2133&2756&3485&2325&4012&4637&2516&5187&5765&2736&4823&5545&2945&4032&4993\\
1972&3075&3653&2136&7777&8065&2327&15936&16105&2520&3569&4369&2737&6816&7345&2945&11832&12193\\
1975&2808&3433&2139&4060&4589&2328&9265&9553&2525&4788&5413&2737&12816&13105&2948&4005&4973\\
1976&2343&3065&2144&4233&4745&2329&9240&9529&2528&5985&6497&2739&2900&3989&2964&5723&6445\\
1976&5607&5945&2145&10112&10337&2336&5073&5585&2533&10956&11245&2751&8360&8801&2967&8056&8585\\
1980&2701&3349&2145&13528&13697&2337&7384&7745&2535&14168&14393&2755&4092&4933&2975&6768&7393\\
1980&7979&8221&2147&6204&6565&2340&3901&4549&2541&7100&7541&2755&10332&10693&2975&15168&15457\\
1980&9701&9901&2156&9483&9725&2340&7931&8269&2548&9435&9773&2759&3480&4441&2976&8393&8905\\
1988&4845&5237&2159&7920&8209&2349&2860&3701&2553&5896&6425&2760&3071&4129&2983&12144&12505\\
1989&6700&6989&2163&5084&5525&2353&16296&16465&2556&4717&5365&2760&4361&5161&2987&4884&5725\\
1989&11620&11789&2171&13860&14029&2355&12212&12437&2565&4148&4877&2760&8239&8689&2988&6565&7213\\
1991&16320&16441&2175&2392&3233&2356&3483&4205&2565&8932&9293&2771&13140&13429&2992&7455&8033\\
1992&6745&7033&2175&3472&4097&2360&3081&3881&2567&11256&11545&2772&3485&4453&3003&3596&4685\\
1995&4292&4733&2176&3807&4385&2363&9516&9805&2573&2964&3925&2772&5605&6253&3003&10004&10445\\
1995&5332&5693&2184&2263&3145&2369&5040&5569&2575&4992&5617&2772&9605&9997&3007&4224&5185\\
1995&8732&8957&2184&6887&7225&2373&6164&6605&2576&2607&3665&2775&5848&6473&3009&15520&15809\\
2001&3520&4049&2184&8137&8425&2375&7632&7993&2580&7171&7621&2781&4940&5669&3013&8316&8845\\
2013&16684&16805&2185&4248&4777&2379&16660&16829&2581&3540&4381&2783&7056&7585&3021&12460&12821\\
2015&11928&12097&2185&6432&6793&2380&4611&5189&2584&4263&4985&2784&7313&7825&3025&7008&7633\\
2016&3713&4225&2193&8176&8465&2380&7029&7421&2584&5487&6065&2788&6435&7013&3036&3827&4885\\
2024&8343&8585&2196&3397&4045&2387&2484&3445&2592&6305&6817&2793&10624&10985&3036&4277&5245\\
2025&2968&3593&2200&9879&10121&2392&8295&8633&2595&14852&15077&2805&3068&4157&3040&6039&6761\\
2033&5544&5905&2204&3003&3725&2397&9796&10085&2596&2997&3965&2805&13468&13757&3040&8769&9281\\
2035&17052&17173&2208&4505&5017&2400&5369&5881&2599&6120&6649&2812&5115&5837&3043&15876&16165\\
2037&4484&4925&2212&6045&6437&2403&3596&4325&2600&9831&10169&2813&4284&5125&3045&3172&4397\\
2040&2201&3001&2220&5251&5701&2405&17028&17197&2603&9204&9565&2820&8611&9061&3045&5092&5933\\
2040&3311&3889&2223&6664&7025&2407&3024&3865&2604&3403&4285&2821&3660&4621&3045&10292&10733\\
2040&4399&4849&2223&14536&14705&2412&4165&4813&2604&8453&8845&2825&6072&6697&3051&6020&6749\\
2040&7081&7369&2225&3648&4273&2413&7884&8245&2619&4340&5069&2829&7300&7829&3059&8580&9109\\
2041&12240&12409&2227&8436&8725&2415&5248&5777&2625&7592&8033&2831&10920&11281&3059&12780&13141\\
2044&5133&5525&2231&4440&4969&2415&6392&6833&2628&5005&5653&2832&2905&4057&3060&6901&7549\\
2047&3696&4225&2233&2544&3385&2415&12848&13073&2635&3132&4093&2839&13800&14089&3060&7811&8389\\
2052&2555&3277&2235&10988&11213&2425&4392&5017&2635&11868&12157&2840&4641&5441&3069&4420&5381\\
2055&9272&9497&2236&7227&7565&2431&10080&10369&2639&3720&4561&2844&5917&6565&3075&7252&7877\\
2057&7176&7465&2241&3080&3809&2431&17400&17569&2640&7519&7969&2848&7665&8177&3077&16236&16525\\
2059&2100&2941&2244&4067&4645&2432&3735&4457&2641&9480&9841&2852&3315&4373&3080&5529&6329\\
2067&12556&12725&2247&5504&5945&2436&2923&3805&2652&5795&6373&2856&4183&5065&3097&13104&13465\\
2068&8715&8957&2249&14880&15049&2436&7373&7765&2656&6633&7145&2856&6767&7345&3100&3219&4469\\
\newpage
3103&5304&6145&3332&9315&9893&3591&8480&9209&3880&9009&9809&4191&7520&8609&4551&5320&7001\\
3104&9153&9665&3333&4556&5645&3597&5396&6485&3885&4828&6197&4200&4841&6409&4551&6880&8249\\
3105&6248&6977&3335&6192&7033&3600&4559&5809&3885&5548&6773&4200&6431&7681&4553&11904&12745\\
3105&8848&9377&3335&10248&10777&3605&4692&5917&3887&14016&14545&4200&9559&10441&4557&10324&11285\\
3108&5035&5917&3344&7383&8105&3611&12060&12589&3900&4949&6301&4212&5885&7237&4560&8449&9601\\
3111&16600&16889&3360&5959&6841&3612&6955&7837&3900&5459&6709&4223&4464&6145&4563&13916&14645\\
3115&3348&4573&3363&15484&15845&3625&7392&8233&3915&8692&9533&4225&13968&14593&4585&7968&9193\\
3116&6363&7085&3367&3456&4825&3627&6364&7325&3915&10148&10877&4235&6708&7933&4587&9116&10205\\
3120&3649&4801&3375&7448&8177&3633&14744&15185&3916&7437&8405&4239&11960&12689&4600&7839&9089\\
3124&4557&5525&3379&5460&6421&3640&7881&8681&3925&12012&12637&4247&8904&9865&4600&9471&10529\\
3128&4095&5153&3381&10540&11069&3648&8855&9577&3927&6536&7625&4251&5180&6701&4601&4800&6649\\
3128&8175&8753&3393&6424&7265&3652&6405&7373&3933&14356&14885&4255&5928&7297&4611&12220&13061\\
3129&10880&11321&3395&4092&5317&3657&12376&12905&3937&7584&8545&4263&10384&11225&4619&10620&11581\\
3131&4620&5581&3399&4760&5849&3663&4216&5585&3939&4340&5861&4268&8925&9893&4620&5029&6829\\
3135&3968&5057&3400&3999&5249&3672&3895&5353&3948&8395&9277&4272&7345&8497&4625&7128&8497\\
3135&13432&13793&3400&9711&10289&3675&10492&11117&3955&5772&6997&4275&14308&14933&4628&7245&8597\\
3151&9120&9649&3401&15840&16201&3680&5871&6929&3956&6867&7925&4292&4635&6317&4633&5544&7225\\
3160&5841&6641&3404&4947&6005&3683&7644&8485&3959&5040&6409&4300&6771&8021&4640&5559&7241\\
3161&5520&6361&3408&4465&5617&3689&6600&7561&3960&9401&10201&4305&4672&6353&4641&6320&7841\\
3168&9545&10057&3420&7739&8461&3692&4365&5717&3973&8964&9805&4305&6952&8177&4644&6667&8125\\
3171&11180&11621&3420&8701&9349&3696&5353&6505&3975&12328&12953&4309&9180&10141&4648&6105&7673\\
3173&13764&14125&3423&13064&13505&3696&7303&8185&3976&4257&5825&4316&6213&7565&4656&8833&9985\\
3175&7752&8377&3425&9072&9697&3699&9020&9749&3979&14700&15229&4320&5671&7129&4669&12540&13381\\
3192&5335&6217&3427&10836&11365&3700&4851&6101&3984&6313&7465&4321&10680&11521&4671&14600&15329\\
3192&6695&7417&3429&7700&8429&3705&3752&5273&3996&4747&6205&4323&8036&9125&4681&10920&11881\\
3193&4824&5785&3439&16200&16561&3720&8249&9049&3999&7840&8801&4324&8307&9365&4687&5016&6865\\
3196&8547&9125&3441&3640&5009&3724&9243&9965&4004&5253&6605&4325&14652&15277&4699&7380&8749\\
3197&9396&9925&3441&5680&6641&3725&10788&11413&4004&7797&8765&4329&6160&7529&4700&8211&9461\\
3200&3471&4721&3444&6283&7165&3729&5840&6929&4017&4544&6065&4347&12596&13325&4712&4815&6737\\
3201&4160&5249&3451&6660&7501&3737&4416&5785&4023&10736&11465&4368&7705&8857&4715&5772&7453\\
3204&7597&8245&3465&4288&5513&3740&6741&7709&4025&12648&13273&4371&9460&10421&4719&6560&8081\\
3211&14100&14461&3473&11136&11665&3741&7900&8741&4025&15048&15577&4379&10980&11821&4727&12864&13705\\
3212&4845&5813&3475&9348&9973&3745&5112&6337&4031&9240&10081&4380&4429&6229&4740&5341&7141\\
3213&6716&7445&3476&5757&6725&3749&13020&13549&4033&5256&6625&4387&4884&6565&4743&11224&12185\\
3216&3913&5065&3480&7169&7969&3751&6840&7801&4048&7215&8273&4389&8300&9389&4752&7015&8473\\
3219&5740&6581&3484&3813&5165&3753&9296&10025&4059&4060&5741&4400&7119&8369&4756&5883&7565\\
3220&4371&5429&3492&9085&9733&3759&15800&16241&4061&8100&9061&4401&12920&13649&4760&6441&8009\\
3225&8008&8633&3496&5247&6305&3772&6195&7253&4071&15400&15929&4403&6396&7765&4773&5236&7085\\
3240&6161&6961&3496&8103&8825&3775&11088&11713&4075&12972&13597&4407&5624&7145&4773&7636&9005\\
3243&9676&10205&3503&5904&6865&3780&4171&5629&4077&11036&11765&4408&4935&6617&4785&9968&11057\\
3255&3712&4937&3504&4753&5905&3783&3944&5465&4080&6649&7801&4416&8687&9745&4785&13192&14033\\
3255&5032&5993&3507&13724&14165&3792&5665&6817&4088&4545&6113&4420&6549&7901&4795&8772&9997\\
3255&11792&12233&3509&6900&7741&3795&6068&7157&4089&9520&10361&4424&5457&7025&4797&6004&7685\\
3264&8927&9505&3515&3828&5197&3795&13348&13877&4092&8165&9133&4428&5995&7453&4800&8591&9841\\
3267&6956&7685&3519&11440&11969&3796&4653&6005&4095&6232&7457&4433&9744&10705&4836&5123&7045\\
3268&7035&7757&3525&9628&10253&3799&8160&9001&4100&6099&7349&4437&11284&12125&4836&7973&9325\\
3275&8268&8893&3531&5180&6269&3800&5151&6401&4104&5047&6505&4445&7452&8677&4843&13524&14365\\
3276&3293&4645&3535&4488&5713&3800&9639&10361&4108&5565&6917&4469&5100&6781&4847&7896&9265\\
3276&7957&8605&3537&8216&8945&3811&4620&5989&4123&8364&9325&4477&6636&8005&4859&5460&7309\\
3277&5964&6805&3549&14060&14501&3813&7084&8045&4125&7268&8357&4485&5852&7373&4865&9048&10273\\
3287&14784&15145&3560&7521&8321&3815&5328&6553&4140&7571&8629&4495&10032&10993&4867&11844&12805\\
3289&9960&10489&3564&6077&7045&3825&11392&12017&4141&4260&5941&4495&11592&12433&4872&6215&7897\\
3297&12104&12545&3565&6132&7093&3828&7085&8053&4147&9804&10645&4508&9075&10133&4872&6785&8353\\
3300&3731&4981&3565&11748&12277&3841&13680&14209&4173&4964&6485&4509&13580&14309&4875&7052&8573\\
3300&5141&6109&3567&7144&7985&3857&8424&9265&4175&13632&14257&4515&4588&6437&4879&6240&7921\\
3312&4655&5713&3572&8475&9197&3861&9860&10589&4176&4343&6025&4515&7708&8933&4900&8979&10229\\
3317&5244&6205&3575&9912&10537&3864&3977&5545&4180&8541&9509&4521&8840&9929&4901&13860&14701\\
3320&6489&7289&3588&4085&5437&3864&6527&7585&4181&5700&7069&4524&5243&6925&4905&4928&6953\\
3325&8532&9157&3588&5555&6613&3864&8023&8905&4185&8632&9593&4524&6893&8245&4917&10556&11645\\
3325&15132&15493&3589&4020&5389&3875&7332&8293&4185&11648&12377&4536&5777&7345&4921&8160&9529\\
\end{longtable}
\end{center}\index{Terne!pitagoriche}
\newpage
\section{Terne pitagoriche primitive}
%\begin{longtable}{@{}r@{\hspace*{1mm}}r@{\hspace*{1mm}}r
%		@{\hspace*{5mm}}
%		r@{\hspace*{1mm}}r@{\hspace*{1mm}}r
%		@{\hspace*{5mm}}
%		r@{\hspace*{1mm}}r@{\hspace*{1mm}}r
%		@{\hspace*{5mm}}
%		r@{\hspace*{1mm}}r@{\hspace*{1mm}}r
%		@{\hspace*{5mm}}
%		r@{\hspace*{1mm}}r@{\hspace*{1mm}}r
%		%			@{\hspace*{5mm}}
%		%			r@{\hspace*{1mm}}r@{\hspace*{1mm}}r
%	} 
%	\toprule
%	a&b&c&a&b&c&a&b&c&a&b&c&a&b&c\\
%	\midrule \endhead
%	\bottomrule \endfoot\
	
		\begin{longtable}{*4{A@{\hspace*{5mm}}A}} 
		\toprule
		a&b&c&a&b&c&a&b&c&a&b&c&a&b&c\\
		\midrule \endhead
		\bottomrule \endfoot
	3&4&5&67&2244&2245&121&7320&7321&175&288&337&255&1288&1313\\
	5&12&13&69&260&269&123&836&845&175&15312&15313&255&3608&3617\\
	7&24&25&69&2380&2381&123&7564&7565&177&1736&1745&259&660&709\\
	9&40&41&71&2520&2521&125&7812&7813&177&15664&15665&261&380&461\\
	11&60&61&73&2664&2665&127&8064&8065&179&16020&16021&265&1392&1417\\
	13&84&85&75&308&317&129&920&929&181&16380&16381&267&3956&3965\\
	15&8&17&75&2812&2813&129&8320&8321&183&1856&1865&273&136&305\\
	15&112&113&77&36&85&131&8580&8581&183&16744&16745&273&736&785\\
	17&144&145&77&2964&2965&133&156&205&185&672&697&273&4136&4145\\
	19&180&181&79&3120&3121&133&8844&8845&185&17112&17113&275&252&373\\
	21&20&29&81&3280&3281&135&352&377&187&84&205&279&440&521\\
	21&220&221&83&3444&3445&135&9112&9113&187&17484&17485&285&68&293\\
	23&264&265&85&132&157&137&9384&9385&189&340&389&285&1612&1637\\
	25&312&313&85&3612&3613&139&9660&9661&189&17860&17861&285&4508&4517\\
	27&364&365&87&416&425&141&1100&1109&191&18240&18241&287&816&865\\
	29&420&421&87&3784&3785&141&9940&9941&193&18624&18625&291&4700&4709\\
	31&480&481&89&3960&3961&143&24&145&195&28&197&295&1728&1753\\
	33&56&65&91&60&109&143&10224&10225&195&748&773&297&304&425\\
	33&544&545&91&4140&4141&145&408&433&195&2108&2117&299&180&349\\
	35&12&37&93&476&485&145&10512&10513&195&19012&19013&301&900&949\\
	35&612&613&93&4324&4325&147&1196&1205&197&19404&19405&303&5096&5105\\
	37&684&685&95&168&193&147&10804&10805&199&19800&19801&305&1848&1873\\
	39&80&89&95&4512&4513&149&11100&11101&201&2240&2249&309&5300&5309\\
	39&760&761&97&4704&4705&151&11400&11401&203&396&445&315&572&653\\
	41&840&841&99&20&101&153&104&185&205&828&853&315&988&1037\\
	43&924&925&99&4900&4901&153&11704&11705&207&224&305&315&1972&1997\\
	45&28&53&101&5100&5101&155&468&493&209&120&241&319&360&481\\
	45&1012&1013&103&5304&5305&155&12012&12013&213&2516&2525&321&5720&5729\\
	47&1104&1105&105&88&137&157&12324&12325&215&912&937&323&36&325\\
	49&1200&1201&105&208&233&159&1400&1409&217&456&505&325&228&397\\
	51&140&149&105&608&617&159&12640&12641&219&2660&2669&327&5936&5945\\
	51&1300&1301&105&5512&5513&161&240&289&221&60&229&329&1080&1129\\
	53&1404&1405&107&5724&5725&161&12960&12961&225&272&353&333&644&725\\
	55&48&73&109&5940&5941&163&13284&13285&231&160&281&335&2232&2257\\
	55&1512&1513&111&680&689&165&52&173&231&520&569&339&6380&6389\\
	57&176&185&111&6160&6161&165&532&557&231&2960&2969&341&420&541\\
	57&1624&1625&113&6384&6385&165&1508&1517&235&1092&1117&345&152&377\\
	59&1740&1741&115&252&277&165&13612&13613&237&3116&3125&345&2368&2393\\
	61&1860&1861&115&6612&6613&167&13944&13945&245&1188&1213&345&6608&6617\\
	63&16&65&117&44&125&169&14280&14281&247&96&265&351&280&449\\
	63&1984&1985&117&6844&6845&171&140&221&249&3440&3449&355&2508&2533\\
	65&72&97&119&120&169&171&14620&14621&253&204&325&357&76&365\\
	65&2112&2113&119&7080&7081&173&14964&14965&255&32&257&357&1276&1325\\
	\newpage
	357&7076&7085&459&220&509&555&6148&6173&663&616&905&767&1656&1825\\
	363&7316&7325&465&368&593&555&17108&17117&663&1216&1385&775&168&793\\
	365&2652&2677&465&4312&4337&559&840&1009&665&432&793&777&464&905\\
	369&800&881&465&12008&12017&561&400&689&665&4488&4537&777&6136&6185\\
	371&1380&1429&469&2220&2269&561&1240&1361&665&8832&8857&779&660&1021\\
	375&7808&7817&471&12320&12329&561&17480&17489&667&156&685&781&2460&2581\\
	377&336&505&473&864&985&565&6372&6397&671&1800&1921&783&56&785\\
	381&8060&8069&475&132&493&567&3256&3305&675&52&677&785&12312&12337\\
	385&552&673&477&1364&1445&573&18236&18245&679&4680&4729&791&6360&6409\\
	385&1488&1537&481&600&769&575&48&577&685&9372&9397&793&1776&1945\\
	385&2952&2977&483&44&485&579&18620&18629&689&1320&1489&795&1292&1517\\
	387&884&965&483&2356&2405&581&3420&3469&693&1924&2045&795&12628&12653\\
	391&120&409&483&12956&12965&583&1344&1465&693&2924&3005&799&960&1249\\
	393&8576&8585&485&4692&4717&585&928&1097&693&4876&4925&801&3920&4001\\
	395&3108&3133&489&13280&13289&585&2072&2153&695&9648&9673&803&2604&2725\\
	399&40&401&493&276&565&585&6832&6857&697&696&985&805&348&877\\
	399&1600&1649&495&952&1073&589&300&661&703&504&865&805&6588&6637\\
	399&8840&8849&495&1472&1553&591&19400&19409&705&992&1217&805&12948&12973\\
	403&396&565&495&4888&4913&595&468&757&705&9928&9953&815&13272&13297\\
	405&3268&3293&497&2496&2545&595&3588&3637&707&5076&5125&817&744&1105\\
	407&624&745&501&13940&13949&595&7068&7093&711&3080&3161&819&1900&2069\\
	411&9380&9389&505&5088&5113&603&2204&2285&713&216&745&819&4100&4181\\
	413&1716&1765&507&14276&14285&605&7308&7333&715&1428&1597&819&6820&6869\\
	415&3432&3457&511&2640&2689&609&200&641&715&2052&2173&825&232&857\\
	417&9656&9665&513&184&545&609&3760&3809&715&10212&10237&825&2752&2873\\
	423&1064&1145&515&5292&5317&611&1020&1189&721&5280&5329&833&1056&1345\\
	425&168&457&517&1044&1165&615&728&953&725&108&733&835&13932&13957\\
	427&1836&1885&519&14960&14969&615&7552&7577&731&780&1069&837&116&845\\
	429&460&629&525&92&533&621&100&629&735&1088&1313&845&14268&14293\\
	429&700&821&525&2788&2837&623&3936&3985&735&10792&10817&847&7296&7345\\
	429&10220&10229&525&15308&15317&627&364&725&737&2184&2305&851&420&949\\
	435&308&533&527&336&625&627&1564&1685&741&580&941&855&832&1193\\
	435&3772&3797&531&1700&1781&629&540&829&741&1540&1709&855&4472&4553\\
	435&10508&10517&533&756&925&635&8052&8077&745&11088&11113&855&14608&14633\\
	437&84&445&535&5712&5737&637&1116&1285&747&3404&3485&861&620&1061\\
	441&1160&1241&537&16016&16025&639&2480&2561&749&5700&5749&861&7540&7589\\
	445&3948&3973&539&1140&1261&645&812&1037&755&11388&11413&865&14952&14977\\
	447&11096&11105&543&16376&16385&645&8308&8333&759&280&809&869&3060&3181\\
	451&780&901&545&5928&5953&649&1680&1801&759&2320&2441&871&2160&2329\\
	453&11396&11405&549&1820&1901&651&260&701&763&5916&5965&873&4664&4745\\
	455&528&697&551&240&601&651&4300&4349&765&868&1157&875&7788&7837\\
	455&2088&2137&553&3096&3145&655&8568&8593&765&3572&3653&885&1628&1853\\
	455&4128&4153&555&572&797&657&2624&2705&765&11692&11717&885&15652&15677\\
	889&8040&8089&989&660&1189&1121&1560&1921&1239&15640&15689&1357&1476&2005\\
	891&3220&3341&999&320&1049&1125&7772&7853&1241&2520&2809&1359&11360&11441\\
	893&924&1285&1001&2880&3049&1127&936&1465&1243&6324&6445&1363&684&1525\\
	895&16008&16033&1001&4080&4201&1131&340&1181&1245&3332&3557&1365&148&1373\\
	897&496&1025&1001&10200&10249&1131&3700&3869&1247&504&1345&1365&1892&2333\\
	897&2296&2465&1003&1596&1885&1133&5244&5365&1251&9620&9701&1365&4028&4253\\
	899&60&901&1005&2132&2357&1139&2100&2389&1253&15996&16045&1365&5428&5597\\
	\newpage
	901&1260&1549&1007&1224&1585&1141&13260&13309&1261&4620&4789&1375&7752&7873\\
	903&704&1145&1015&192&1033&1143&8024&8105&1265&1248&1777&1377&3136&3425\\
	903&8296&8345&1015&10488&10537&1147&204&1165&1265&6552&6673&1387&2484&2845\\
	905&16368&16393&1017&6344&6425&1155&68&1157&1267&16356&16405&1391&5640&5809\\
	909&5060&5141&1023&64&1025&1155&1292&1733&1269&740&1469&1395&532&1493\\
	913&3384&3505&1023&4264&4385&1155&2852&3077&1271&360&1321&1395&11972&12053\\
	915&1748&1973&1025&528&1153&1155&5452&5573&1273&2064&2425&1397&8004&8125\\
	915&16732&16757&1027&3036&3205&1155&13588&13637&1275&988&1613&1403&1596&2125\\
	917&8556&8605&1035&748&1277&1157&3876&4045&1275&2668&2957&1407&2024&2465\\
	923&2436&2605&1035&6572&6653&1159&1680&2041&1281&1640&2081&1411&3300&3589\\
	925&372&997&1037&1716&2005&1161&560&1289&1281&16720&16769&1413&12284&12365\\
	927&5264&5345&1043&11076&11125&1169&13920&13969&1287&4816&4985&1417&5856&6025\\
	931&1020&1381&1045&1332&1693&1173&1036&1565&1287&6784&6905&1419&380&1469\\
	935&1368&1657&1045&4452&4573&1173&2236&2525&1287&10184&10265&1419&8260&8381\\
	935&3552&3673&1053&3196&3365&1175&792&1417&1295&72&1297&1421&780&1621\\
	935&17472&17497&1057&11376&11425&1177&5664&5785&1295&17088&17137&1425&1312&1937\\
	943&576&1105&1065&2408&2633&1179&8540&8621&1305&592&1433&1425&2632&2993\\
	945&248&977&1067&4644&4765&1183&14256&14305&1305&10472&10553&1431&1040&1769\\
	945&9088&9137&1071&1840&2129&1185&3008&3233&1309&2820&3109&1435&228&1453\\
	945&17848&17873&1071&7040&7121&1189&420&1261&1309&7020&7141&1441&8520&8641\\
	949&2580&2749&1071&11680&11729&1197&1804&2165&1309&17460&17509&1443&76&1445\\
	955&18228&18253&1073&264&1105&1197&8804&8885&1311&1360&1889&1443&6076&6245\\
	957&124&965&1075&612&1237&1197&14596&14645&1311&2200&2561&1449&1720&2249\\
	957&3724&3845&1079&3360&3529&1199&5880&6001&1313&5016&5185&1449&12920&13001\\
	959&9360&9409&1081&840&1369&1207&2376&2665&1323&836&1565&1455&4592&4817\\
	963&5684&5765&1085&132&1093&1209&280&1241&1325&1092&1717&1457&624&1585\\
	965&18612&18637&1085&11988&12037&1209&4240&4409&1333&444&1405&1463&2784&3145\\
	969&1120&1481&1089&7280&7361&1211&14940&14989&1335&3848&4073&1463&8784&8905\\
	969&1480&1769&1095&2552&2777&1219&1140&1669&1337&18216&18265&1467&13244&13325\\
	973&9636&9685&1099&12300&12349&1221&140&1229&1339&5220&5389&1469&6300&6469\\
	975&448&1073&1105&1968&2257&1221&6100&6221&1341&11060&11141&1475&1428&2053\\
	975&2728&2897&1105&3528&3697&1225&888&1513&1343&2976&3265&1479&880&1721\\
	979&3900&4021&1107&476&1205&1233&9344&9425&1349&2340&2701&1479&3640&3929\\
	981&5900&5981&1111&5040&5161&1235&1932&2293&1351&18600&18649&1485&1148&1877\\
	987&884&1325&1113&1184&1625&1235&4428&4597&1353&296&1385&1485&9052&9173\\
	987&9916&9965&1113&12616&12665&1239&1520&1961&1353&7504&7625&1491&2300&2741\\
	\newpage
	1495&1848&2377&1633&2256&2785&1767&4144&4505&1919&4920&5281&2057&7176&7465\\
	1495&6528&6697&1635&5828&6053&1769&1440&2281&1921&6240&6529&2059&2100&2941\\
	1501&2940&3301&1639&11040&11161&1771&2700&3229&1925&2652&3277&2065&1128&2353\\
	1503&13904&13985&1643&924&1885&1771&12900&13021&1925&15252&15373&2067&644&2165\\
	1505&312&1537&1645&492&1717&1775&2208&2833&1927&264&1945&2067&12556&12725\\
	1507&9324&9445&1647&1496&2225&1781&9300&9469&1935&88&1937&2071&5760&6121\\
	1513&3816&4105&1649&4560&4849&1785&688&1913&1937&11016&11185&2075&3132&3757\\
	1515&4988&5213&1651&7980&8149&1785&3392&3833&1943&1824&2665&2077&1764&2725\\
	1517&156&1525&1653&1204&2045&1785&5368&5657&1947&1196&2285&2079&2600&3329\\
	1519&720&1681&1653&3604&3965&1785&6968&7193&1947&15604&15725&2079&17800&17921\\
	1521&14240&14321&1659&2900&3341&1793&13224&13345&1953&1504&2465&2085&9548&9773\\
	1525&1548&2173&1661&11340&11461&1807&9576&9745&1955&3348&3877&2091&460&2141\\
	1529&9600&9721&1665&328&1697&1809&1880&2609&1955&6468&6757&2091&7420&7709\\
	1533&2444&2885&1665&17072&17153&1813&516&1885&1957&5124&5485&2093&3876&4405\\
	1537&984&1825&1675&1932&2557&1815&7208&7433&1961&720&2089&2093&12876&13045\\
	1539&3100&3461&1677&164&1685&1817&2856&3385&1963&11316&11485&2107&276&2125\\
	1541&1980&2509&1677&8236&8405&1819&5580&5869&1965&8468&8693&2109&940&2309\\
	1545&5192&5417&1679&2400&2929&1825&2352&2977&1969&15960&16081&2109&5980&6341\\
	1547&3996&4285&1683&4756&5045&1827&1564&2405&1971&2300&3029&2115&92&2117\\
	1547&6996&7165&1683&11644&11765&1829&1260&2221&1975&2808&3433&2117&2244&3085\\
	1551&560&1649&1683&17444&17525&1833&344&1865&1989&6700&6989&2119&13200&13369\\
	1551&9880&10001&1691&3780&4141&1833&9856&10025&1989&11620&11789&2121&4880&5321\\
	1557&14924&15005&1695&6272&6497&1837&13884&14005&1991&16320&16441&2125&7668&7957\\
	1573&7236&7405&1703&8496&8665&1843&4524&4885&1995&1012&2237&2133&2756&3485\\
	1575&1672&2297&1705&1032&1993&1845&172&1853&1995&4292&4733&2135&1248&2473\\
	1575&15272&15353&1705&11952&12073&1853&5796&6085&1995&5332&5693&2139&1900&2861\\
	1577&3264&3625&1711&1320&2161&1855&792&2017&1995&8732&8957&2139&4060&4589\\
	1581&820&1781&1717&4956&5245&1859&14220&14341&2001&1960&2801&2145&752&2273\\
	1581&4180&4469&1719&18200&18281&1863&3016&3545&2001&3520&4049&2145&1568&2657\\
	1591&240&1609&1725&2068&2693&1869&3740&4181&2009&360&2041&2145&10112&10337\\
	1593&1376&2105&1725&2548&3077&1881&4720&5081&2013&1316&2405&2145&13528&13697\\
	1595&1092&1933&1727&12264&12385&1881&14560&14681&2013&16684&16805&2147&6204&6565\\
	1595&10452&10573&1729&3960&4321&1885&1692&2533&2015&1632&2593&2159&7920&8209\\
	1599&80&1601&1729&8760&8929&1885&10428&10597&2015&11928&12097&2163&5084&5525\\
	1599&7480&7649&1739&420&1789&1887&616&1985&2021&180&2029&2171&13860&14029\\
	1605&5612&5837&1743&3224&3665&1887&6016&6305&2025&2968&3593&2173&564&2245\\
	1611&15980&16061&1749&860&1949&1891&1380&2341&2033&5544&5905&2175&2392&3233\\
	1615&3432&3793&1749&12580&12701&1903&14904&15025&2035&828&2197&2175&3472&4097\\
	1615&4368&4657&1751&5160&5449&1905&7952&8177&2035&17052&17173&2183&1056&2425\\
	1617&656&1745&1755&1748&2477&1909&3180&3709&2037&4484&4925&2185&4248&4777\\
	1617&10744&10865&1755&9028&9197&1911&440&1961&2041&12240&12409&2185&6432&6793\\
	1625&7728&7897&1763&84&1765&1911&10720&10889&2047&3696&4225&2193&376&2225\\
	1629&16340&16421&1767&1144&2105&1917&2156&2885&2055&9272&9497&2193&8176&8465\\
	
	
	
\end{longtable}
\section{Coprimi minori di cento}

\begin{longtable}
%	{@{}
%		r@{\hspace*{1mm}}r
%		@{\hspace*{5mm}}
%		r@{\hspace*{1mm}}r
%		@{\hspace*{5mm}}
%		r@{\hspace*{1mm}}r
%		@{\hspace*{5mm}}
%		r@{\hspace*{1mm}}r
%		@{\hspace*{5mm}}
%		r@{\hspace*{1mm}}r
%		@{\hspace*{5mm}}
%		r@{\hspace*{1mm}}r
%		@{\hspace*{5mm}}
%		r@{\hspace*{1mm}}r
%	} 
{*6{Q@{\hspace*{7mm}}Q}}
	\toprule
	m&n&m&n&m&n&m&n&m&n&m&n&m&n\\
	\midrule \endhead
	\bottomrule \endfoot \index{Numero!coprimo}
	2&1&11&4&16&3&19&12&23&3&25&17&28&9\\
	&&11&5&16&5&19&13&23&4&25&18&28&11\\
	3&1&11&6&16&7&19&14&23&5&25&19&28&13\\
	3&2&11&7&16&9&19&15&23&6&25&21&28&15\\
	&&11&8&16&11&19&16&23&7&25&22&28&17\\
	4&1&11&9&16&13&19&17&23&8&25&23&28&19\\
	4&3&11&10&16&15&19&18&23&9&25&24&28&23\\
	&&&&&&&&23&10&&&28&25\\
	5&1&12&1&17&1&20&1&23&11&26&1&28&27\\
	5&2&12&5&17&2&20&3&23&12&26&3&&\\
	5&3&12&7&17&3&20&7&23&13&26&5&29&1\\
	5&4&12&11&17&4&20&9&23&14&26&7&29&2\\
	&&&&17&5&20&11&23&15&26&9&29&3\\
	6&1&13&1&17&6&20&13&23&16&26&11&29&4\\
	6&5&13&2&17&7&20&17&23&17&26&15&29&5\\
	&&13&3&17&8&20&19&23&18&26&17&29&6\\
	7&1&13&4&17&9&&&23&19&26&19&29&7\\
	7&2&13&5&17&10&21&1&23&20&26&21&29&8\\
	7&3&13&6&17&11&21&2&23&21&26&23&29&9\\
	7&4&13&7&17&12&21&4&23&22&26&25&29&10\\
	7&5&13&8&17&13&21&5&&&&&29&11\\
	7&6&13&9&17&14&21&8&24&1&27&1&29&12\\
	&&13&10&17&15&21&10&24&5&27&2&29&13\\
	8&1&13&11&17&16&21&11&24&7&27&4&29&14\\
	8&3&13&12&&&21&13&24&11&27&5&29&15\\
	8&5&&&18&1&21&16&24&13&27&7&29&16\\
	8&7&14&1&18&5&21&17&24&17&27&8&29&17\\
	&&14&3&18&7&21&19&24&19&27&10&29&18\\
	9&1&14&5&18&11&21&20&24&23&27&11&29&19\\
	9&2&14&9&18&13&&&&&27&13&29&20\\
	9&4&14&11&18&17&22&1&25&1&27&14&29&21\\
	9&5&14&13&&&22&3&25&2&27&16&29&22\\
	9&7&&&19&1&22&5&25&3&27&17&29&23\\
	9&8&15&1&19&2&22&7&25&4&27&19&29&24\\
	&&15&2&19&3&22&9&25&6&27&20&29&25\\
	10&1&15&4&19&4&22&13&25&7&27&22&29&26\\
	10&3&15&7&19&5&22&15&25&8&27&23&29&27\\
	10&7&15&8&19&6&22&17&25&9&27&25&29&28\\
	10&9&15&11&19&7&22&19&25&11&27&26&&\\
	&&15&13&19&8&22&21&25&12&&&30&1\\
	11&1&15&14&19&9&&&25&13&28&1&30&7\\
	11&2&&&19&10&23&1&25&14&28&3&30&11\\
	11&3&16&1&19&11&23&2&25&16&28&5&30&13\\
	\newpage
	30&17&32&15&34&27&37&1&38&13&40&13&41&32\\
	30&19&32&17&34&29&37&2&38&15&40&17&41&33\\
	30&23&32&19&34&31&37&3&38&17&40&19&41&34\\
	30&29&32&21&34&33&37&4&38&21&40&21&41&35\\
	&&32&23&&&37&5&38&23&40&23&41&36\\
	31&1&32&25&35&1&37&6&38&25&40&27&41&37\\
	31&2&32&27&35&2&37&7&38&27&40&29&41&38\\
	31&3&32&29&35&3&37&8&38&29&40&31&41&39\\
	31&4&32&31&35&4&37&9&38&31&40&33&41&40\\
	31&5&&&35&6&37&10&38&33&40&37&&\\
	31&6&33&1&35&8&37&11&38&35&40&39&42&1\\
	31&7&33&2&35&9&37&12&38&37&&&42&5\\
	31&8&33&4&35&11&37&13&&&41&1&42&11\\
	31&9&33&5&35&12&37&14&39&1&41&2&42&13\\
	31&10&33&7&35&13&37&15&39&2&41&3&42&17\\
	31&11&33&8&35&16&37&16&39&4&41&4&42&19\\
	31&12&33&10&35&17&37&17&39&5&41&5&42&23\\
	31&13&33&13&35&18&37&18&39&7&41&6&42&25\\
	31&14&33&14&35&19&37&19&39&8&41&7&42&29\\
	31&15&33&16&35&22&37&20&39&10&41&8&42&31\\
	31&16&33&17&35&23&37&21&39&11&41&9&42&37\\
	31&17&33&19&35&24&37&22&39&14&41&10&42&41\\
	31&18&33&20&35&26&37&23&39&16&41&11&&\\
	31&19&33&23&35&27&37&24&39&17&41&12&43&1\\
	31&20&33&25&35&29&37&25&39&19&41&13&43&2\\
	31&21&33&26&35&31&37&26&39&20&41&14&43&3\\
	31&22&33&28&35&32&37&27&39&22&41&15&43&4\\
	31&23&33&29&35&33&37&28&39&23&41&16&43&5\\
	31&24&33&31&35&34&37&29&39&25&41&17&43&6\\
	31&25&33&32&&&37&30&39&28&41&18&43&7\\
	31&26&&&36&1&37&31&39&29&41&19&43&8\\
	31&27&34&1&36&5&37&32&39&31&41&20&43&9\\
	31&28&34&3&36&7&37&33&39&32&41&21&43&10\\
	31&29&34&5&36&11&37&34&39&34&41&22&43&11\\
	31&30&34&7&36&13&37&35&39&35&41&23&43&12\\
	&&34&9&36&17&37&36&39&37&41&24&43&13\\
	32&1&34&11&36&19&&&39&38&41&25&43&14\\
	32&3&34&13&36&23&38&1&&&41&26&43&15\\
	32&5&34&15&36&25&38&3&40&1&41&27&43&16\\
	32&7&34&19&36&29&38&5&40&3&41&28&43&17\\
	32&9&34&21&36&31&38&7&40&7&41&29&43&18\\
	32&11&34&23&36&35&38&9&40&9&41&30&43&19\\
	32&13&34&25&&&38&11&40&11&41&31&43&20\\
	\newpage
	43&21&&&46&37&47&38&49&19&50&41&52&11\\
	43&22&45&1&46&39&47&39&49&20&50&43&52&15\\
	43&23&45&2&46&41&47&40&49&22&50&47&52&17\\
	43&24&45&4&46&43&47&41&49&23&50&49&52&19\\
	43&25&45&7&46&45&47&42&49&24&&&52&21\\
	43&26&45&8&&&47&43&49&25&51&1&52&23\\
	43&27&45&11&47&1&47&44&49&26&51&2&52&25\\
	43&28&45&13&47&2&47&45&49&27&51&4&52&27\\
	43&29&45&14&47&3&47&46&49&29&51&5&52&29\\
	43&30&45&16&47&4&&&49&30&51&7&52&31\\
	43&31&45&17&47&5&48&1&49&31&51&8&52&33\\
	43&32&45&19&47&6&48&5&49&32&51&10&52&35\\
	43&33&45&22&47&7&48&7&49&33&51&11&52&37\\
	43&34&45&23&47&8&48&11&49&34&51&13&52&41\\
	43&35&45&26&47&9&48&13&49&36&51&14&52&43\\
	43&36&45&28&47&10&48&17&49&37&51&16&52&45\\
	43&37&45&29&47&11&48&19&49&38&51&19&52&47\\
	43&38&45&31&47&12&48&23&49&39&51&20&52&49\\
	43&39&45&32&47&13&48&25&49&40&51&22&52&51\\
	43&40&45&34&47&14&48&29&49&41&51&23&&\\
	43&41&45&37&47&15&48&31&49&43&51&25&53&1\\
	43&42&45&38&47&16&48&35&49&44&51&26&53&2\\
	&&45&41&47&17&48&37&49&45&51&28&53&3\\
	44&1&45&43&47&18&48&41&49&46&51&29&53&4\\
	44&3&45&44&47&19&48&43&49&47&51&31&53&5\\
	44&5&&&47&20&48&47&49&48&51&32&53&6\\
	44&7&46&1&47&21&&&&&51&35&53&7\\
	44&9&46&3&47&22&49&1&50&1&51&37&53&8\\
	44&13&46&5&47&23&49&2&50&3&51&38&53&9\\
	44&15&46&7&47&24&49&3&50&7&51&40&53&10\\
	44&17&46&9&47&25&49&4&50&9&51&41&53&11\\
	44&19&46&11&47&26&49&5&50&11&51&43&53&12\\
	44&21&46&13&47&27&49&6&50&13&51&44&53&13\\
	44&23&46&15&47&28&49&8&50&17&51&46&53&14\\
	44&25&46&17&47&29&49&9&50&19&51&47&53&15\\
	44&27&46&19&47&30&49&10&50&21&51&49&53&16\\
	44&29&46&21&47&31&49&11&50&23&51&50&53&17\\
	44&31&46&25&47&32&49&12&50&27&&&53&18\\
	44&35&46&27&47&33&49&13&50&29&52&1&53&19\\
	44&37&46&29&47&34&49&15&50&31&52&3&53&20\\
	44&39&46&31&47&35&49&16&50&33&52&5&53&21\\
	44&41&46&33&47&36&49&17&50&37&52&7&53&22\\
	44&43&46&35&47&37&49&18&50&39&52&9&53&23\\
	\newpage
	53&24&54&41&55&52&57&23&58&43&59&35&61&2\\
	53&25&54&43&55&53&57&25&58&45&59&36&61&3\\
	53&26&54&47&55&54&57&26&58&47&59&37&61&4\\
	53&27&54&49&&&57&28&58&49&59&38&61&5\\
	53&28&54&53&56&1&57&29&58&51&59&39&61&6\\
	53&29&&&56&3&57&31&58&53&59&40&61&7\\
	53&30&55&1&56&5&57&32&58&55&59&41&61&8\\
	53&31&55&2&56&9&57&34&58&57&59&42&61&9\\
	53&32&55&3&56&11&57&35&&&59&43&61&10\\
	53&33&55&4&56&13&57&37&59&1&59&44&61&11\\
	53&34&55&6&56&15&57&40&59&2&59&45&61&12\\
	53&35&55&7&56&17&57&41&59&3&59&46&61&13\\
	53&36&55&8&56&19&57&43&59&4&59&47&61&14\\
	53&37&55&9&56&23&57&44&59&5&59&48&61&15\\
	53&38&55&12&56&25&57&46&59&6&59&49&61&16\\
	53&39&55&13&56&27&57&47&59&7&59&50&61&17\\
	53&40&55&14&56&29&57&49&59&8&59&51&61&18\\
	53&41&55&16&56&31&57&50&59&9&59&52&61&19\\
	53&42&55&17&56&33&57&52&59&10&59&53&61&20\\
	53&43&55&18&56&37&57&53&59&11&59&54&61&21\\
	53&44&55&19&56&39&57&55&59&12&59&55&61&22\\
	53&45&55&21&56&41&57&56&59&13&59&56&61&23\\
	53&46&55&23&56&43&&&59&14&59&57&61&24\\
	53&47&55&24&56&45&58&1&59&15&59&58&61&25\\
	53&48&55&26&56&47&58&3&59&16&&&61&26\\
	53&49&55&27&56&51&58&5&59&17&60&1&61&27\\
	53&50&55&28&56&53&58&7&59&18&60&7&61&28\\
	53&51&55&29&56&55&58&9&59&19&60&11&61&29\\
	53&52&55&31&&&58&11&59&20&60&13&61&30\\
	&&55&32&57&1&58&13&59&21&60&17&61&31\\
	54&1&55&34&57&2&58&15&59&22&60&19&61&32\\
	54&5&55&36&57&4&58&17&59&23&60&23&61&33\\
	54&7&55&37&57&5&58&19&59&24&60&29&61&34\\
	54&11&55&38&57&7&58&21&59&25&60&31&61&35\\
	54&13&55&39&57&8&58&23&59&26&60&37&61&36\\
	54&17&55&41&57&10&58&25&59&27&60&41&61&37\\
	54&19&55&42&57&11&58&27&59&28&60&43&61&38\\
	54&23&55&43&57&13&58&31&59&29&60&47&61&39\\
	54&25&55&46&57&14&58&33&59&30&60&49&61&40\\
	54&29&55&47&57&16&58&35&59&31&60&53&61&41\\
	54&31&55&48&57&17&58&37&59&32&60&59&61&42\\
	54&35&55&49&57&20&58&39&59&33&&&61&43\\
	54&37&55&51&57&22&58&41&59&34&61&1&61&44\\
	\newpage
	61&45&62&55&64&3&65&16&66&19&67&28&68&7\\
	61&46&62&57&64&5&65&17&66&23&67&29&68&9\\
	61&47&62&59&64&7&65&18&66&25&67&30&68&11\\
	61&48&62&61&64&9&65&19&66&29&67&31&68&13\\
	61&49&&&64&11&65&21&66&31&67&32&68&15\\
	61&50&63&1&64&13&65&22&66&35&67&33&68&19\\
	61&51&63&2&64&15&65&23&66&37&67&34&68&21\\
	61&52&63&4&64&17&65&24&66&41&67&35&68&23\\
	61&53&63&5&64&19&65&27&66&43&67&36&68&25\\
	61&54&63&8&64&21&65&28&66&47&67&37&68&27\\
	61&55&63&10&64&23&65&29&66&49&67&38&68&29\\
	61&56&63&11&64&25&65&31&66&53&67&39&68&31\\
	61&57&63&13&64&27&65&32&66&59&67&40&68&33\\
	61&58&63&16&64&29&65&33&66&61&67&41&68&35\\
	61&59&63&17&64&31&65&34&66&65&67&42&68&37\\
	61&60&63&19&64&33&65&36&&&67&43&68&39\\
	&&63&20&64&35&65&37&67&1&67&44&68&41\\
	62&1&63&22&64&37&65&38&67&2&67&45&68&43\\
	62&3&63&23&64&39&65&41&67&3&67&46&68&45\\
	62&5&63&25&64&41&65&42&67&4&67&47&68&47\\
	62&7&63&26&64&43&65&43&67&5&67&48&68&49\\
	62&9&63&29&64&45&65&44&67&6&67&49&68&53\\
	62&11&63&31&64&47&65&46&67&7&67&50&68&55\\
	62&13&63&32&64&49&65&47&67&8&67&51&68&57\\
	62&15&63&34&64&51&65&48&67&9&67&52&68&59\\
	62&17&63&37&64&53&65&49&67&10&67&53&68&61\\
	62&19&63&38&64&55&65&51&67&11&67&54&68&63\\
	62&21&63&40&64&57&65&53&67&12&67&55&68&65\\
	62&23&63&41&64&59&65&54&67&13&67&56&68&67\\
	62&25&63&43&64&61&65&56&67&14&67&57&&\\
	62&27&63&44&64&63&65&57&67&15&67&58&69&1\\
	62&29&63&46&&&65&58&67&16&67&59&69&2\\
	62&33&63&47&65&1&65&59&67&17&67&60&69&4\\
	62&35&63&50&65&2&65&61&67&18&67&61&69&5\\
	62&37&63&52&65&3&65&62&67&19&67&62&69&7\\
	62&39&63&53&65&4&65&63&67&20&67&63&69&8\\
	62&41&63&55&65&6&65&64&67&21&67&64&69&10\\
	62&43&63&58&65&7&&&67&22&67&65&69&11\\
	62&45&63&59&65&8&66&1&67&23&67&66&69&13\\
	62&47&63&61&65&9&66&5&67&24&&&69&14\\
	62&49&63&62&65&11&66&7&67&25&68&1&69&16\\
	62&51&&&65&12&66&13&67&26&68&3&69&17\\
	62&53&64&1&65&14&66&17&67&27&68&5&69&19\\
	\newpage
	69&20&70&33&71&30&72&5&73&20&73&63&74&67\\
	69&22&70&37&71&31&72&7&73&21&73&64&74&69\\
	69&25&70&39&71&32&72&11&73&22&73&65&74&71\\
	69&26&70&41&71&33&72&13&73&23&73&66&74&73\\
	69&28&70&43&71&34&72&17&73&24&73&67&&\\
	69&29&70&47&71&35&72&19&73&25&73&68&75&1\\
	69&31&70&51&71&36&72&23&73&26&73&69&75&2\\
	69&32&70&53&71&37&72&25&73&27&73&70&75&4\\
	69&34&70&57&71&38&72&29&73&28&73&71&75&7\\
	69&35&70&59&71&39&72&31&73&29&73&72&75&8\\
	69&37&70&61&71&40&72&35&73&30&&&75&11\\
	69&38&70&67&71&41&72&37&73&31&74&1&75&13\\
	69&40&70&69&71&42&72&41&73&32&74&3&75&14\\
	69&41&&&71&43&72&43&73&33&74&5&75&16\\
	69&43&71&1&71&44&72&47&73&34&74&7&75&17\\
	69&44&71&2&71&45&72&49&73&35&74&9&75&19\\
	69&47&71&3&71&46&72&53&73&36&74&11&75&22\\
	69&49&71&4&71&47&72&55&73&37&74&13&75&23\\
	69&50&71&5&71&48&72&59&73&38&74&15&75&26\\
	69&52&71&6&71&49&72&61&73&39&74&17&75&28\\
	69&53&71&7&71&50&72&65&73&40&74&19&75&29\\
	69&55&71&8&71&51&72&67&73&41&74&21&75&31\\
	69&56&71&9&71&52&72&71&73&42&74&23&75&32\\
	69&58&71&10&71&53&&&73&43&74&25&75&34\\
	69&59&71&11&71&54&73&1&73&44&74&27&75&37\\
	69&61&71&12&71&55&73&2&73&45&74&29&75&38\\
	69&62&71&13&71&56&73&3&73&46&74&31&75&41\\
	69&64&71&14&71&57&73&4&73&47&74&33&75&43\\
	69&65&71&15&71&58&73&5&73&48&74&35&75&44\\
	69&67&71&16&71&59&73&6&73&49&74&39&75&46\\
	69&68&71&17&71&60&73&7&73&50&74&41&75&47\\
	&&71&18&71&61&73&8&73&51&74&43&75&49\\
	70&1&71&19&71&62&73&9&73&52&74&45&75&52\\
	70&3&71&20&71&63&73&10&73&53&74&47&75&53\\
	70&9&71&21&71&64&73&11&73&54&74&49&75&56\\
	70&11&71&22&71&65&73&12&73&55&74&51&75&58\\
	70&13&71&23&71&66&73&13&73&56&74&53&75&59\\
	70&17&71&24&71&67&73&14&73&57&74&55&75&61\\
	70&19&71&25&71&68&73&15&73&58&74&57&75&62\\
	70&23&71&26&71&69&73&16&73&59&74&59&75&64\\
	70&27&71&27&71&70&73&17&73&60&74&61&75&67\\
	70&29&71&28&&&73&18&73&61&74&63&75&68\\
	70&31&71&29&72&1&73&19&73&62&74&65&75&71\\
	\newpage
	75&73&77&4&77&60&79&4&79&47&80&27&81&31\\
	75&74&77&5&77&61&79&5&79&48&80&29&81&32\\
	&&77&6&77&62&79&6&79&49&80&31&81&34\\
	76&1&77&8&77&64&79&7&79&50&80&33&81&35\\
	76&3&77&9&77&65&79&8&79&51&80&37&81&37\\
	76&5&77&10&77&67&79&9&79&52&80&39&81&38\\
	76&7&77&12&77&68&79&10&79&53&80&41&81&40\\
	76&9&77&13&77&69&79&11&79&54&80&43&81&41\\
	76&11&77&15&77&71&79&12&79&55&80&47&81&43\\
	76&13&77&16&77&72&79&13&79&56&80&49&81&44\\
	76&15&77&17&77&73&79&14&79&57&80&51&81&46\\
	76&17&77&18&77&74&79&15&79&58&80&53&81&47\\
	76&21&77&19&77&75&79&16&79&59&80&57&81&49\\
	76&23&77&20&77&76&79&17&79&60&80&59&81&50\\
	76&25&77&23&&&79&18&79&61&80&61&81&52\\
	76&27&77&24&78&1&79&19&79&62&80&63&81&53\\
	76&29&77&25&78&5&79&20&79&63&80&67&81&55\\
	76&31&77&26&78&7&79&21&79&64&80&69&81&56\\
	76&33&77&27&78&11&79&22&79&65&80&71&81&58\\
	76&35&77&29&78&17&79&23&79&66&80&73&81&59\\
	76&37&77&30&78&19&79&24&79&67&80&77&81&61\\
	76&39&77&31&78&23&79&25&79&68&80&79&81&62\\
	76&41&77&32&78&25&79&26&79&69&&&81&64\\
	76&43&77&34&78&29&79&27&79&70&81&1&81&65\\
	76&45&77&36&78&31&79&28&79&71&81&2&81&67\\
	76&47&77&37&78&35&79&29&79&72&81&4&81&68\\
	76&49&77&38&78&37&79&30&79&73&81&5&81&70\\
	76&51&77&39&78&41&79&31&79&74&81&7&81&71\\
	76&53&77&40&78&43&79&32&79&75&81&8&81&73\\
	76&55&77&41&78&47&79&33&79&76&81&10&81&74\\
	76&59&77&43&78&49&79&34&79&77&81&11&81&76\\
	76&61&77&45&78&53&79&35&79&78&81&13&81&77\\
	76&63&77&46&78&55&79&36&&&81&14&81&79\\
	76&65&77&47&78&59&79&37&80&1&81&16&81&80\\
	76&67&77&48&78&61&79&38&80&3&81&17&&\\
	76&69&77&50&78&67&79&39&80&7&81&19&82&1\\
	76&71&77&51&78&71&79&40&80&9&81&20&82&3\\
	76&73&77&52&78&73&79&41&80&11&81&22&82&5\\
	76&75&77&53&78&77&79&42&80&13&81&23&82&7\\
	&&77&54&&&79&43&80&17&81&25&82&9\\
	77&1&77&57&79&1&79&44&80&19&81&26&82&11\\
	77&2&77&58&79&2&79&45&80&21&81&28&82&13\\
	77&3&77&59&79&3&79&46&80&23&81&29&82&15\\
	\newpage
	82&17&83&11&83&54&84&47&85&42&86&19&87&14\\
	82&19&83&12&83&55&84&53&85&43&86&21&87&16\\
	82&21&83&13&83&56&84&55&85&44&86&23&87&17\\
	82&23&83&14&83&57&84&59&85&46&86&25&87&19\\
	82&25&83&15&83&58&84&61&85&47&86&27&87&20\\
	82&27&83&16&83&59&84&65&85&48&86&29&87&22\\
	82&29&83&17&83&60&84&67&85&49&86&31&87&23\\
	82&31&83&18&83&61&84&71&85&52&86&33&87&25\\
	82&33&83&19&83&62&84&73&85&53&86&35&87&26\\
	82&35&83&20&83&63&84&79&85&54&86&37&87&28\\
	82&37&83&21&83&64&84&83&85&56&86&39&87&31\\
	82&39&83&22&83&65&&&85&57&86&41&87&32\\
	82&43&83&23&83&66&85&1&85&58&86&45&87&34\\
	82&45&83&24&83&67&85&2&85&59&86&47&87&35\\
	82&47&83&25&83&68&85&3&85&61&86&49&87&37\\
	82&49&83&26&83&69&85&4&85&62&86&51&87&38\\
	82&51&83&27&83&70&85&6&85&63&86&53&87&40\\
	82&53&83&28&83&71&85&7&85&64&86&55&87&41\\
	82&55&83&29&83&72&85&8&85&66&86&57&87&43\\
	82&57&83&30&83&73&85&9&85&67&86&59&87&44\\
	82&59&83&31&83&74&85&11&85&69&86&61&87&46\\
	82&61&83&32&83&75&85&12&85&71&86&63&87&47\\
	82&63&83&33&83&76&85&13&85&72&86&65&87&49\\
	82&65&83&34&83&77&85&14&85&73&86&67&87&50\\
	82&67&83&35&83&78&85&16&85&74&86&69&87&52\\
	82&69&83&36&83&79&85&18&85&76&86&71&87&53\\
	82&71&83&37&83&80&85&19&85&77&86&73&87&55\\
	82&73&83&38&83&81&85&21&85&78&86&75&87&56\\
	82&75&83&39&83&82&85&22&85&79&86&77&87&59\\
	82&77&83&40&&&85&23&85&81&86&79&87&61\\
	82&79&83&41&84&1&85&24&85&82&86&81&87&62\\
	82&81&83&42&84&5&85&26&85&83&86&83&87&64\\
	&&83&43&84&11&85&27&85&84&86&85&87&65\\
	83&1&83&44&84&13&85&28&&&&&87&67\\
	83&2&83&45&84&17&85&29&86&1&87&1&87&68\\
	83&3&83&46&84&19&85&31&86&3&87&2&87&70\\
	83&4&83&47&84&23&85&32&86&5&87&4&87&71\\
	83&5&83&48&84&25&85&33&86&7&87&5&87&73\\
	83&6&83&49&84&29&85&36&86&9&87&7&87&74\\
	83&7&83&50&84&31&85&37&86&11&87&8&87&76\\
	83&8&83&51&84&37&85&38&86&13&87&10&87&77\\
	83&9&83&52&84&41&85&39&86&15&87&11&87&79\\
	83&10&83&53&84&43&85&41&86&17&87&13&87&80\\
	\newpage
	87&82&88&85&89&41&89&84&91&16&91&71&92&53\\
	87&83&88&87&89&42&89&85&91&17&91&72&92&55\\
	87&85&&&89&43&89&86&91&18&91&73&92&57\\
	87&86&89&1&89&44&89&87&91&19&91&74&92&59\\
	&&89&2&89&45&89&88&91&20&91&75&92&61\\
	88&1&89&3&89&46&&&91&22&91&76&92&63\\
	88&3&89&4&89&47&90&1&91&23&91&79&92&65\\
	88&5&89&5&89&48&90&7&91&24&91&80&92&67\\
	88&7&89&6&89&49&90&11&91&25&91&81&92&71\\
	88&9&89&7&89&50&90&13&91&27&91&82&92&73\\
	88&13&89&8&89&51&90&17&91&29&91&83&92&75\\
	88&15&89&9&89&52&90&19&91&30&91&85&92&77\\
	88&17&89&10&89&53&90&23&91&31&91&86&92&79\\
	88&19&89&11&89&54&90&29&91&32&91&87&92&81\\
	88&21&89&12&89&55&90&31&91&33&91&88&92&83\\
	88&23&89&13&89&56&90&37&91&34&91&89&92&85\\
	88&25&89&14&89&57&90&41&91&36&91&90&92&87\\
	88&27&89&15&89&58&90&43&91&37&&&92&89\\
	88&29&89&16&89&59&90&47&91&38&92&1&92&91\\
	88&31&89&17&89&60&90&49&91&40&92&3&&\\
	88&35&89&18&89&61&90&53&91&41&92&5&93&1\\
	88&37&89&19&89&62&90&59&91&43&92&7&93&2\\
	88&39&89&20&89&63&90&61&91&44&92&9&93&4\\
	88&41&89&21&89&64&90&67&91&45&92&11&93&5\\
	88&43&89&22&89&65&90&71&91&46&92&13&93&7\\
	88&45&89&23&89&66&90&73&91&47&92&15&93&8\\
	88&47&89&24&89&67&90&77&91&48&92&17&93&10\\
	88&49&89&25&89&68&90&79&91&50&92&19&93&11\\
	88&51&89&26&89&69&90&83&91&51&92&21&93&13\\
	88&53&89&27&89&70&90&89&91&53&92&25&93&14\\
	88&57&89&28&89&71&&&91&54&92&27&93&16\\
	88&59&89&29&89&72&91&1&91&55&92&29&93&17\\
	88&61&89&30&89&73&91&2&91&57&92&31&93&19\\
	88&63&89&31&89&74&91&3&91&58&92&33&93&20\\
	88&65&89&32&89&75&91&4&91&59&92&35&93&22\\
	88&67&89&33&89&76&91&5&91&60&92&37&93&23\\
	88&69&89&34&89&77&91&6&91&61&92&39&93&25\\
	88&71&89&35&89&78&91&8&91&62&92&41&93&26\\
	88&73&89&36&89&79&91&9&91&64&92&43&93&28\\
	88&75&89&37&89&80&91&10&91&66&92&45&93&29\\
	88&79&89&38&89&81&91&11&91&67&92&47&93&32\\
	88&81&89&39&89&82&91&12&91&68&92&49&93&34\\
	88&83&89&40&89&83&91&15&91&69&92&51&93&35\\
	\newpage
	93&37&94&11&95&2&95&59&96&43&97&25&97&68\\
	93&38&94&13&95&3&95&61&96&47&97&26&97&69\\
	93&40&94&15&95&4&95&62&96&49&97&27&97&70\\
	93&41&94&17&95&6&95&63&96&53&97&28&97&71\\
	93&43&94&19&95&7&95&64&96&55&97&29&97&72\\
	93&44&94&21&95&8&95&66&96&59&97&30&97&73\\
	93&46&94&23&95&9&95&67&96&61&97&31&97&74\\
	93&47&94&25&95&11&95&68&96&65&97&32&97&75\\
	93&49&94&27&95&12&95&69&96&67&97&33&97&76\\
	93&50&94&29&95&13&95&71&96&71&97&34&97&77\\
	93&52&94&31&95&14&95&72&96&73&97&35&97&78\\
	93&53&94&33&95&16&95&73&96&77&97&36&97&79\\
	93&55&94&35&95&17&95&74&96&79&97&37&97&80\\
	93&56&94&37&95&18&95&77&96&83&97&38&97&81\\
	93&58&94&39&95&21&95&78&96&85&97&39&97&82\\
	93&59&94&41&95&22&95&79&96&89&97&40&97&83\\
	93&61&94&43&95&23&95&81&96&91&97&41&97&84\\
	93&64&94&45&95&24&95&82&96&95&97&42&97&85\\
	93&65&94&49&95&26&95&83&&&97&43&97&86\\
	93&67&94&51&95&27&95&84&97&1&97&44&97&87\\
	93&68&94&53&95&28&95&86&97&2&97&45&97&88\\
	93&70&94&55&95&29&95&87&97&3&97&46&97&89\\
	93&71&94&57&95&31&95&88&97&4&97&47&97&90\\
	93&73&94&59&95&32&95&89&97&5&97&48&97&91\\
	93&74&94&61&95&33&95&91&97&6&97&49&97&92\\
	93&76&94&63&95&34&95&92&97&7&97&50&97&93\\
	93&77&94&65&95&36&95&93&97&8&97&51&97&94\\
	93&79&94&67&95&37&95&94&97&9&97&52&97&95\\
	93&80&94&69&95&39&&&97&10&97&53&97&96\\
	93&82&94&71&95&41&96&1&97&11&97&54&&\\
	93&83&94&73&95&42&96&5&97&12&97&55&98&1\\
	93&85&94&75&95&43&96&7&97&13&97&56&98&3\\
	93&86&94&77&95&44&96&11&97&14&97&57&98&5\\
	93&88&94&79&95&46&96&13&97&15&97&58&98&9\\
	93&89&94&81&95&47&96&17&97&16&97&59&98&11\\
	93&91&94&83&95&48&96&19&97&17&97&60&98&13\\
	93&92&94&85&95&49&96&23&97&18&97&61&98&15\\
	&&94&87&95&51&96&25&97&19&97&62&98&17\\
	94&1&94&89&95&52&96&29&97&20&97&63&98&19\\
	94&3&94&91&95&53&96&31&97&21&97&64&98&23\\
	94&5&94&93&95&54&96&35&97&22&97&65&98&25\\
	94&7&&&95&56&96&37&97&23&97&66&98&27\\
	94&9&95&1&95&58&96&41&97&24&97&67&98&29\\
	\newpage
	98&31&99&23&99&94&100&97&&&&&&\\
	98&33&99&25&99&95&100&99&&&&&&\\
	98&37&99&26&99&97&&&&&&&&\\
	98&39&99&28&99&98&&&&&&&&\\
	98&41&99&29&&&&&&&&&&\\
	98&43&99&31&100&1&&&&&&&&\\
	98&45&99&32&100&3&&&&&&&&\\
	98&47&99&34&100&7&&&&&&&&\\
	98&51&99&35&100&9&&&&&&&&\\
	98&53&99&37&100&11&&&&&&&&\\
	98&55&99&38&100&13&&&&&&&&\\
	98&57&99&40&100&17&&&&&&&&\\
	98&59&99&41&100&19&&&&&&&&\\
	98&61&99&43&100&21&&&&&&&&\\
	98&65&99&46&100&23&&&&&&&&\\
	98&67&99&47&100&27&&&&&&&&\\
	98&69&99&49&100&29&&&&&&&&\\
	98&71&99&50&100&31&&&&&&&&\\
	98&73&99&52&100&33&&&&&&&&\\
	98&75&99&53&100&37&&&&&&&&\\
	98&79&99&56&100&39&&&&&&&&\\
	98&81&99&58&100&41&&&&&&&&\\
	98&83&99&59&100&43&&&&&&&&\\
	98&85&99&61&100&47&&&&&&&&\\
	98&87&99&62&100&49&&&&&&&&\\
	98&89&99&64&100&51&&&&&&&&\\
	98&93&99&65&100&53&&&&&&&&\\
	98&95&99&67&100&57&&&&&&&&\\
	98&97&99&68&100&59&&&&&&&&\\
	&&99&70&100&61&&&&&&&&\\
	99&1&99&71&100&63&&&&&&&&\\
	99&2&99&73&100&67&&&&&&&&\\
	99&4&99&74&100&69&&&&&&&&\\
	99&5&99&76&100&71&&&&&&&&\\
	99&7&99&79&100&73&&&&&&&&\\
	99&8&99&80&100&77&&&&&&&&\\
	99&10&99&82&100&79&&&&&&&&\\
	99&13&99&83&100&81&&&&&&&&\\
	99&14&99&85&100&83&&&&&&&&\\
	99&16&99&86&100&87&&&&&&&&\\
	99&17&99&89&100&89&&&&&&&&\\
	99&19&99&91&100&91&&&&&&&&\\
	99&20&99&92&100&93&&&&&&&&\\
	\end{longtable}
\newpage
	\section{Numero di coprimi con N}
	\citaoeis{A000010}
\begin{center}
	\begin{tabular}{*4{Q@{\hspace*{8mm}}Q}}
		\toprule
		N&Num& N&Num& N&Num& N&Num& N&Num\\
		\midrule
		1&0&21&12&41&40&61&60&81&54\\
		2&1&22&10&42&12&62&30&82&40\\
		3&2&23&22&43&42&63&36&83&82\\
		4&2&24&8&44&20&64&32&84&24\\
		5&4&25&20&45&24&65&48&85&64\\
		6&2&26&12&46&22&66&20&86&42\\
		7&6&27&18&47&46&67&66&87&56\\
		8&4&28&12&48&16&68&32&88&40\\
		9&6&29&28&49&42&69&44&89&88\\
		10&4&30&8&50&20&70&24&90&24\\
		11&10&31&30&51&32&71&70&91&72\\
		12&4&32&16&52&24&72&24&92&44\\
		13&12&33&20&53&52&73&72&93&60\\
		14&6&34&16&54&18&74&36&94&46\\
		15&8&35&24&55&40&75&40&95&72\\
		16&8&36&12&56&24&76&36&96&32\\
		17&16&37&36&57&36&77&60&97&96\\
		18&6&38&18&58&28&78&24&98&42\\
		19&18&39&24&59&58&79&78&99&60\\
		20&8&40&16&60&16&80&32&100&40\\
		\bottomrule
	\end{tabular}\captionof{table}{Numero di coprimi di N}
\end{center}
\chapter{Numeri figurati}
\section{Numeri triangolari}
\begin{center}
	\includestandalone{grafici/triangolari}
	\captionof{figure}{Numeri Triangolari}
\end{center}\index{Numero!triangolare}
\begin{center}
	\begin{tabular}{*{8}{R} }
\toprule
n&T&n&T&n&T&n&T\\
\midrule
1&1&11&66&21&231&31&496\\
2&3&12&78&22&253&32&528\\
3&6&13&91&23&276&33&561\\
4&10&14&105&24&300&34&595\\
5&15&15&120&25&325&35&630\\
6&21&16&136&26&351&36&666\\
7&28&17&153&27&378&37&703\\
8&36&18&171&28&406&38&741\\
9&45&19&190&29&435&39&780\\
10&55&20&210&30&465&40&820\\
\bottomrule
\end{tabular}\captionof{table}{Numeri triangolari} \index{Numero!triangolare}
\end{center}
\section{Numeri quadrati}
\begin{center}
\includestandalone{grafici/quadrati}
\captionof{figure}{Numeri Quadrati}
\end{center}\index{Numero!quadrato}
\begin{center}
	\begin{tabular}{*{8}{R} }
		\toprule
		n&Q&n&Q&n&Q&n&Q\\
		\midrule
		1&1&11&121&21&441&31&961\\
		2&4&12&144&22&484&32&1024\\
		3&9&13&169&23&529&33&1089\\
		4&16&14&196&24&576&34&1156\\
		5&25&15&225&25&625&35&1225\\
		6&36&16&256&26&676&36&1296\\
		7&49&17&289&27&729&37&1369\\
		8&64&18&324&28&784&38&1444\\
		9&81&19&361&29&841&39&1521\\
		10&100&20&400&30&900&40&1600\\
		\bottomrule
	\end{tabular}\captionof{table}{Numeri quadrati} \index{Numero!quadrato}
\end{center}
\section{Triangolari quadrati}
\begin{center}
	\begin{tabular}{RR}
	\toprule
n&TQ\\
\midrule
1&1\\
2&36\\
3&1225\\
4&41616\\
5&1413721\\
6&48024900\\
\bottomrule
\end{tabular}\captionof{table}{Numeri quadrati triangolari} 
\end{center}
\chapter{Quadrati magici}
\section{Quadrati 3x3}
\begin{center}
	\includestandalone{grafici/magicotre1}
	\captionof{figure}{Lo Shu}
\end{center}\index{Quadrato magico 3x3}\index{Lo Shu}
\begin{center}
	\includestandalone{grafici/magicotreT}
	\captionof{figure}{Quadrati magici derivati}
\end{center}\index{Quadrato magico 3x3}


\chapter{Congettura di Collatz}
\citaoeis{A006577}
\begin{longtable}{llllllllllll}\toprule
\caption{Numero e lunghezza ciclo}\\
\midrule
\endfirsthead
\multicolumn{12}{c} {\tablename\ \thetable\ -- \textit{Continua dalla pagina precedente}} \\
\toprule
\endhead
\bottomrule
\multicolumn{12}{r} {\textit{Continua nella pagina successiva}} \\
\endfoot
\endlastfoot
3 &1& 7 &2 &5& 8&16 &3& 19 &6 &14& 9\\
9 &17& 17 &4 &12& 20&20 &7& 7 &15 &15& 10\\
23 &10& 111 &18 &18& 18&106 &5& 26 &13 &13& 21\\
21 &21& 34 &8 &109& 8&29 &16& 16 &16 &104& 11\\
24 &24& 24 &11 &11& 112&112 &19& 32 &19 &32& 19\\
19 &107& 107 &6 &27& 27&27 &14& 14 &14 &102& 22\\
115 &22& 14 &22 &22& 35&35 &9& 22 &110 &110& 9\\
9 &30& 30 &17 &30& 17&92 &17& 17 &105 &105& 12\\
118 &25& 25 &25 &25& 25&87 &12& 38 &12 &100& 113\\
113 &113& 69 &20 &12& 33&33 &20& 20 &33 &33& 20\\
95 &20& 46 &108 &108& 108&46 &7& 121 &28 &28& 28\\
28 &28& 41 &15 &90& 15&41 &15& 15 &103 &103& 23\\
116 &116& 116 &23 &23& 15&15 &23& 36 &23 &85& 36\\
36 &36& 54 &10 &98& 23&23 &111& 111 &111 &67& 10\\
49 &10& 124 &31 &31& 31&80 &18& 31 &31 &31& 18\\
18 &93& 93 &18 &44& 18&44 &106& 106 &106 &44& 13\\
119 &119& 119 &26 &26& 26&119 &26& 18 &26 &39& 26\\
26 &88& 88 &13 &39& 39&39 &13& 13 &101 &101& 114\\
26 &114& 52 &114 &114& 70&70 &21& 52 &13 &13& 34\\
34 &34& 127 &21 &83& 21&127 &34& 34 &34 &52& 21\\
21 &96& 96 &21 &21& 47&47 &109& 47 &109 &65& 109\\
109 &47& 47 &8 &122& 122&122 &29& 29 &29 &78& 29\\
122 &29& 21 &29 &29& 42&42 &16& 29 &91 &91& 16\\
16 &42& 42 &16 &42& 16&60 &104& 104 &104 &42& 24\\
29 &117& 117 &117 &117& 117&55 &24& 73 &24 &117& 16\\
16 &16& 42 &24 &37& 37&37 &24& 24 &86 &86& 37\\
130 &37& 37 &37 &37& 55&55 &11& 24 &99 &99& 24\\
24 &24& 143 &112 &50& 112&24 &112& 112 &68 &68& 11\\
112 &50& 50 &11 &11& 125&125 &32& 125 &32 &125& 32\\
32 &81& 81 &19 &125& 32&32 &32& 32 &32 &50& 19\\
45 &19& 45 &94 &94& 94&45 &19& 19 &45 &45& 19\\
19 &45& 45 &107 &63& 107&58 &107& 107 &45 &45& 14\\
32 &120& 120 &120 &120& 120&120 &27& 58 &27 &76& 27\\
27 &120& 120 &27 &19& 19&19 &27& 27 &40 &40& 27\\
40 &27& 133 &89 &89& 89&133 &14& 133 &40 &40& 40\\
40 &40& 32 &14 &58& 14&53 &102& 102 &102 &40& 115\\
27 &27& 27 &115 &115& 53&53 &115& 27 &115 &53& 71\\
71 &71& 97 &22 &115& 53&53 &14& 14 &14 &40& 35\\
128 &35& 128 &35 &35& 128&128 &22& 35 &84 &84& 22\\
22 &128& 128 &35 &35& 35&27 &35& 35 &53 &53& 22\\
48 &22& 22 &97 &97& 97&141 &22& 48 &22 &141& 48\\
48 &48& 97 &110 &22& 48&48 &110& 110 &66 &66& 110\\
61 &110& 35 &48 &48& 48&61 &9& 35 &123 &123& 123\\
123 &123& 61 &30 &123& 30&123 &30& 30 &79 &79& 30\\
30 &123& 123 &30 &30& 22&22 &30& 22 &30 &48& 43\\
43 &43& 136 &17 &43& 30&30 &92& 92 &92 &43& 17\\
136 &17& 30 &43 &43& 43&87 &17& 43 &43 &43& 17\\
17 &61& 61 &105 &56& 105&30 &105& 105 &43 &43& 25\\
30 &30& 30 &118 &118& 118&30 &118& 56 &118 &118& 118\\
118 &56& 56 &25 &74& 74&74 &25& 25 &118 &118& 17\\
56 &17& 69 &17 &17& 43&43 &25& 131 &38 &38& 38\\
38 &38& 69 &25 &131& 25&131 &87& 87 &87 &131& 38\\
25 &131& 131 &38 &38& 38&38 &38& 30 &38 &30& 56\\
56 &56& 131 &12 &51& 25&25 &100& 100 &100 &38& 25\\
144 &25& 100 &25 &25& 144&144 &113& 51 &51 &51& 113\\
113 &25& 25 &113 &51& 113&144 &69& 69 &69 &95& 12\\
64 &113& 113 &51 &51& 51&64 &12& 64 &12 &38& 126\\
126 &126& 38 &33 &126& 126&126 &33& 33 &126 &126& 33\\
126 &33& 64 &82 &82& 82&170 &20& 33 &126 &126& 33\\
33 &33& 64 &33 &25& 33&25 &33& 33 &51 &51& 20\\
46 &46& 46 &20 &20& 46&46 &95& 33 &95 &139& 95\\
95 &46& 46 &20 &139& 20&20 &46& 46 &46 &95& 20\\
90 &20& 46 &46 &46& 46&139 &108& 20 &64 &64& 108\\
108 &59& 59 &108 &33& 108&152 &46& 46 &46 &59& 15\\
33 &33& 33 &121 &121& 121&152 &121& 33 &121 &59& 121\\
121 &121& 121 &28 &121& 59&59 &28& 28 &77 &77& 28\\
77 &28& 103 &121 &121& 121&72 &28& 59 &20 &20& 20\\
20 &20& 72 &28 &46& 28&134 &41& 41 &41 &134& 28\\
41 &41& 41 &28 &28& 134&134 &90& 134 &90 &41& 90\\
90 &134& 134 &15 &28& 134&134 &41& 41 &41 &85& 41\\
41 &41& 41 &41 &41& 33&33 &15& 59 &59 &59& 15\\
15 &54& 54 &103 &28& 103&147 &103& 103 &41 &41& 116\\
147 &28& 28 &28 &28& 28&178 &116& 147 &116 &28& 54\\
54 &54& 147 &116 &116& 28&28 &116& 116 &54 &54& 72\\
147 &72& 46 &72 &72& 98&98 &23& 67 &116 &116& 54\\
54 &54& 116 &15 &67& 15&54 &15& 15 &41 &41& 36\\
129 &129& 129 &36 &36& 129&129 &36& 129 &36 &67& 129\\
129 &129& 116 &23 &129& 36&36 &85& 85 &85 &129& 23\\
173 &23& 85 &129 &129& 129&36 &36& 36 &36 &36& 36\\
36 &28& 28 &36 &28& 36&28 &54& 54 &54 &129& 23\\
49 &49& 49 &23 &23& 23&142 &98& 49 &98 &36& 98\\
98 &142& 142 &23 &98& 49&49 &23& 23 &142 &142& 49\\
23 &49& 36 &49 &49& 98&98 &111& 93 &23 &23& 49\\
49 &49& 49 &111 &142& 111&41 &67& 67 &67 &93& 111\\
111 &62& 62 &111 &111& 36&36 &49& 155 &49 &62& 49\\
49 &62& 62 &10 &36& 36&36 &124& 124 &124 &36& 124\\
155 &124& 124 &124 &124& 62&62 &31& 124 &124 &124& 31\\
31 &124& 124 &31 &62& 31&93 &80& 80 &80 &168& 31\\
80 &31& 31 &124 &124& 124&75 &31& 75 &31 &62& 23\\
23 &23& 168 &31 &23& 23&23 &31& 31 &49 &49& 44\\
137 &44& 137 &44 &44& 137&137 &18& 44 &44 &44& 31\\
31 &31& 75 &93 &137& 93&31 &93& 93 &44 &44& 18\\
93 &137& 137 &18 &18& 31&31 &44& 137 &44 &93& 44\\
44 &88& 88 &18 &44& 44&44 &44& 44 &44 &137& 18\\
36 &18& 36 &62 &62& 62&62 &106& 18 &57 &57& 106\\
106 &31& 31 &106 &150& 106&57 &44& 44 &44 &57& 26\\
150 &31& 31 &31 &31& 31&57 &119& 181 &119 &150& 119\\
119 &31& 31 &119 &57& 57&57 &119& 119 &119 &119& 119\\
31 &119& 57 &57 &57& 57&88 &26& 150 &75 &75& 75\\
75 &75& 49 &26 &101& 26&119 &119& 119 &119 &70& 18\\
57 &57& 57 &18 &18& 70&70 &18& 57 &18 &70& 44\\
44 &44& 163 &26 &132& 132&132 &39& 39 &39 &132& 39\\
132 &39& 132 &39 &39& 70&70 &26& 132 &132 &132& 26\\
26 &132& 132 &88 &39& 88&70 &88& 88 &132 &132& 39\\
176 &26& 26 &132 &132& 132&88 &39& 39 &39 &83& 39\\
39 &39& 176 &39 &39& 31&31 &39& 39 &31 &31& 57\\
31 &57& 83 &57 &57& 132&132 &13& 52 &52 &52& 26\\
26 &26& 145 &101 &145& 101&52 &101& 101 &39 &39& 26\\
101 &145& 145 &26 &26& 101&101 &26& 52 &26 &176& 145\\
145 &145& 101 &114 &26& 52&52 &52& 52 &52 &145& 114\\
101 &114& 52 &26 &26& 26&52 &114& 52 &52 &52& 114\\
114 &145& 145 &70 &44& 70&26 &70& 70 &96 &96& 13\\
114 &65& 65 &114 &114& 114&158 &52& 39 &52 &114& 52\\
52 &65& 65 &13 &52& 65&65 &13& 13 &39 &39& 127\\
39 &127& 114 &127 &127& 39&39 &34& 158 &127 &127& 127\\
127 &127& 96 &34 &65& 34&65 &127& 127 &127 &114& 34\\
34 &127& 127 &34 &34& 65&65 &83& 96 &83 &127& 83\\
83 &171& 171 &21 &83& 34&34 &127& 127 &127 &34& 34\\
78 &34& 127 &34 &34& 65&65 &34& 26 &26 &26& 34\\
34 &26& 26 &34 &26& 34&78 &52& 52 &52 &127& 21\\
140 &47& 47 &47 &47& 47&52 &21& 140 &21 &140& 47\\
47 &47& 140 &96 &34& 34&34 &96& 96 &140 &140& 96\\
34 &96& 140 &47 &47& 47&171 &21& 96 &140 &140& 21\\
21 &21& 96 &47 &34& 47&140 &47& 47 &96 &96& 21\\
47 &91& 91 &21 &21& 47&47 &47& 47 &47 &47& 47\\
47 &140& 140 &109 &39& 21&21 &65& 65 &65 &91& 109\\
65 &109& 140 &60 &60& 60&153 &109& 109 &34 &34& 109\\
109 &153& 153 &47 &60& 47&60 &47& 47 &60 &60& 16\\
153 &34& 34 &34 &34& 34&109 &122& 60 &122 &34& 122\\
122 &153& 153 &122 &122& 34&34 &122& 122 &60 &60& 122\\
60 &122& 153 &122 &122& 122&60 &29& 34 &122 &122& 60\\
60 &60& 60 &29 &91& 29&122 &78& 78 &78 &166& 29\\
78 &78& 78 &29 &29& 104&104 &122& 122 &122 &73& 122\\
122 &73& 73 &29 &60& 60&60 &21& 21 &21 &166& 21\\
73 &21& 21 &21 &21& 73&73 &29& 47 &47 &47& 29\\
29 &135& 135 &42 &135& 42&73 &42& 42 &135 &135& 29\\
135 &42& 42 &42 &42& 42&104 &29& 73 &29 &73& 135\\
135 &135& 135 &91 &29& 135&135 &91& 91 &42 &42& 91\\
73 &91& 42 &135 &135& 135&73 &16& 179 &29 &29& 135\\
135 &135& 42 &42 &91& 42&135 &42& 42 &86 &86& 42\\
42 &42& 42 &42 &42& 42&42 &42& 34 &42 &135& 34\\
34 &34& 179 &16 &34& 60&60 &60& 60 &60 &60& 16\\
135 &16& 148 &55 &55& 55&148 &104& 29 &29 &29& 104\\
104 &148& 148 &104 &55& 104&55 &42& 42 &42 &55& 117\\
104 &148& 148 &29 &29& 29&104 &29& 104 &29 &55& 29\\
29 &179& 179 &117 &148& 148&148 &117& 117 &29 &29& 55\\
55 &55& 42 &55 &55& 148&148 &117& 104 &117 &117& 29\\
29 &29& 148 &117 &55& 117&55 &55& 55 &55 &86& 73\\
117 &148& 148 &73 &73& 47&47 &73& 29 &73 &47& 99\\
99 &99& 99 &24 &117& 68&68 &117& 117 &117 &68& 55\\
161 &55& 42 &55 &55& 117&117 &16& 55 &68 &68& 16\\
16 &55& 55 &16 &68& 16&161 &42& 42 &42 &161& 37\\
42 &130& 130 &130 &130& 130&68 &37& 42 &37 &130& 130\\
130 &130& 161 &37 &130& 130&130 &37& 37 &68 &68& 130\\
68 &130& 130 &130 &130& 117&117 &24& 37 &130 &130& 37\\
37 &37& 68 &86 &68& 86&37 &86& 86 &130 &130& 24\\
86 &174& 174 &24 &24& 86&86 &130& 37 &130 &86& 130\\
130 &37& 37 &37 &81& 37&37 &37& 37 &37 &174& 37\\
68 &37& 68 &29 &29& 29&130 &37& 37 &29 &29& 37\\
37 &29& 29 &55 &81& 55&174 &55& 55 &130 &130& 24\\
143 &50& 50 &50 &50& 50&50 &24& 55 &24 &143& 24\\
24 &143& 143 &99 &50& 50&50 &99& 99 &37 &37& 99\\
37 &99& 81 &143 &143& 143&174 &24& 37 &99 &99& 50\\
50 &50& 81 &24 &174& 24&81 &143& 143 &143 &99& 50\\
24 &24& 24 &50 &50& 37&37 &50& 143 &50 &143& 99\\
99 &99& 50 &112 &50& 94&94 &24& 24 &24 &50& 50\\
50 &50& 50 &50 &50& 50&50 &112& 50 &143 &143& 112\\
112 &42& 42 &68 &24& 68&42 &68& 68 &94 &94& 112\\
68 &112& 112 &63 &63& 63&156 &112& 156 &112 &156& 37\\
37 &37& 63 &50 &112& 156&156 &50& 50 &63 &63& 50\\
63 &50& 156 &63 &63& 63&156 &11& 156 &37 &37& 37\\
37 &37& 37 &125 &112& 125&37 &125& 125 &37 &37& 125\\
125 &156& 156 &125 &125& 125&125 &125& 37 &125 &94& 63\\
63 &63& 50 &32 &63& 125&125 &125& 125 &125 &112& 32\\
63 &32& 37 &125 &125& 125&107 &32& 63 &63 &63& 32\\
32 &94& 94 &81 &125& 81&156 &81& 81 &169 &169& 32\\
81 &81& 81 &32 &32& 32&81 &125& 107 &125 &32& 125\\
125 &76& 76 &32 &125& 76&76 &32& 32 &63 &63& 24\\
63 &24& 76 &24 &24& 169&169 &32& 76 &24 &24& 24\\
24 &24& 107 &32 &76& 32&169 &50& 50 &50 &125& 45\\
32 &138& 138 &45 &45& 138&138 &45& 76 &45 &50& 138\\
138 &138& 76 &19 &138& 45&45 &45& 45 &45 &138& 32\\
107 &32& 76 &32 &32& 76&76 &94& 138 &138 &138& 94\\
94 &32& 32 &94 &138& 94&50 &45& 45 &45 &169& 19\\
76 &94& 94 &138 &138& 138&120 &19& 76 &19 &94& 32\\
32 &32& 182 &45 &138& 138&138 &45& 45 &94 &94& 45\\
138 &45& 45 &89 &89& 89&138 &19& 45 &45 &45& 45\\
45 &45& 76 &45 &45& 45&37 &45& 45 &138 &138& 19\\
37 &37& 37 &19 &19& 37&37 &63& 63 &63 &89& 63\\
63 &63& 63 &107 &138& 19&19 &58& 58 &58 &151& 107\\
151 &107& 58 &32 &32& 32&120 &107& 107 &151 &151& 107\\
107 &58& 58 &45 &58& 45&151 &45& 45 &58 &58& 27\\
107 &151& 151 &32 &32& 32&151 &32& 107 &32 &107& 32\\
32 &58& 58 &120 &32& 182&182 &120& 120 &151 &151& 120\\
151 &120& 107 &32 &32& 32&89 &120& 58 &58 &58& 58\\
58 &58& 151 &120 &151& 120&107 &120& 120 &120 &58& 120\\
32 &32& 32 &120 &120& 58&58 &58& 58 &58 &58& 58\\
58 &89& 89 &27 &120& 151&151 &76& 76 &76 &164& 76\\
50 &76& 76 &76 &76& 50&50 &27& 102 &102 &102& 27\\
27 &120& 120 &120 &71& 120&32 &120& 120 &71 &71& 19\\
\bottomrule\end{longtable}
\section{Frequenza cicli}
\begin{longtable}{llllllllllll}\toprule
\caption{Frequenza cicli}\\
\midrule
\textbf{c} & \textbf{f} & \textbf{c} & \textbf{f}& \textbf{c} & \textbf{f} & \textbf{c} & \textbf{f}& \textbf{c} & \textbf{f} & \textbf{c} & \textbf{f}\\
\midrule
\endfirsthead
\multicolumn{12}{c} {\tablename\ \thetable\ -- \textit{Continua dalla pagina precedente}} \\
\textbf{c} & \textbf{f} & \textbf{c} & \textbf{f}& \textbf{c} & \textbf{f} & \textbf{c} & \textbf{f}& \textbf{c} & \textbf{f} & \textbf{c} & \textbf{f}\\
\toprule
\endhead
\bottomrule
\multicolumn{12}{r} {\textit{Continua nella pagina successiva}} \\
\endfoot
\endlastfoot
1 & 1&2 &1&3& 2&4 &1&5 &2&6& 2\\
7 & 4&8 &4&9& 6&10 &6&11 &8&12& 9\\
13 & 13&14 &12&15& 17&16 &22&17 &16&18& 22\\
19 & 26&20 &22&21& 32&22 &18&23 &28&24& 43\\
25 & 23&26 &38&27& 22&28 &30&29 &49&30& 23\\
31 & 40&32 &58&33& 25&34 &47&35 &14&36& 30\\
37 & 60&38 &16&39& 32&40 &10&41 &23&42& 47\\
43 & 15&44 &29&45& 43&46 &21&47 &39&48& 11\\
49 & 22&50 &47&51& 10&52 &27&53 &8&54& 15\\
55 & 32&56 &8&57& 20&58 &29&59 &10&60& 26\\
61 & 5&62 &13&63& 29&64 &7&65 &17&66& 2\\
67 & 7&68 &21&69& 6&70 &13&71 &6&72& 6\\
73 & 17&74 &3&75& 8&76 &24&77 &3&78& 9\\
79 & 2&80 &5&81& 15&82 &3&83 &8&84& 2\\
85 & 6&86 &13&87& 5&88 &10&89 &10&90& 6\\
91 & 12&92 &4&93& 11&94 &20&95 &7&96& 15\\
97 & 5&98 &10&99& 17&100 &5&101 &12&102& 7\\
103 & 7&104 &16&105& 6&106 &9&107 &22&108& 8\\
109 & 13&110 &7&111& 11&112 &19&113 &10&114& 17\\
115 & 7&116 &14&117& 24&118 &11&119 &19&120& 31\\
121 & 14&122 &25&123& 9&124 &17&125 &32&126& 11\\
127 & 22&128 &6&129& 15&130 &29&131 &7&132& 18\\
133 & 3&134 &9&135& 21&136 &2&137 &9&138& 24\\
139 & 3&140 &13&141& 2&142 &6&143 &15&144& 4\\
145 & 10&146 &0&147& 5&148 &14&149 &0&150& 4\\
151 & 15&152 &2&153& 7&154 &0&155 &2&156& 11\\
157 & 0&158 &2&159& 0&160 &0&161 &4&162& 0\\
163 & 1&164 &1&165& 0&166 &2&167 &0&168& 2\\
169 & 6&170 &1&171& 3&172 &0&173 &1&174& 6\\
175 & 0&176 &3&177& 0&178 &1&179 &4&180& 0\\
181 & 1&182 &3&183& 0&184 &0&185 &0&186& 0\\
\bottomrule\end{longtable}
\section{Valori massini}
\begin{longtable}{llllllllllll}\toprule
\caption{Valori massimi}\\
\midrule
\textbf{N} & \textbf{R} & \textbf{N} & \textbf{R}& \textbf{N} & \textbf{R} & \textbf{N} & \textbf{R}& \textbf{N} & \textbf{R} & \textbf{N} & \textbf{R}\\
\midrule
\endfirsthead
\multicolumn{12}{c} {\tablename\ \thetable\ -- \textit{Continua dalla pagina precedente}} \\
\textbf{N} & \textbf{R} & \textbf{N} & \textbf{R}& \textbf{N} & \textbf{R} & \textbf{N} & \textbf{R}& \textbf{N} & \textbf{R} & \textbf{N} & \textbf{R}\\
\toprule
\endhead
\bottomrule
\multicolumn{12}{r} {\textit{Continua nella pagina successiva}} \\
\endfoot
\endlastfoot
1 & 4&2 &1&3& 16&4 &2&5 &16&6& 16\\
7 & 52&8 &4&9& 52&10 &16&11 &52&12& 16\\
13 & 40&14 &52&15& 160&16 &8&17 &52&18& 52\\
19 & 88&20 &16&21& 64&22 &52&23 &160&24& 16\\
25 & 88&26 &40&27& 9232&28 &52&29 &88&30& 160\\
31 & 9232&32 &16&33& 100&34 &52&35 &160&36& 52\\
37 & 112&38 &88&39& 304&40 &20&41 &9232&42& 64\\
43 & 196&44 &52&45& 136&46 &160&47 &9232&48& 24\\
49 & 148&50 &88&51& 232&52 &40&53 &160&54& 9232\\
55 & 9232&56 &52&57& 196&58 &88&59 &304&60& 160\\
61 & 184&62 &9232&63& 9232&64 &32&65 &196&66& 100\\
67 & 304&68 &52&69& 208&70 &160&71 &9232&72& 52\\
73 & 9232&74 &112&75& 340&76 &88&77 &232&78& 304\\
79 & 808&80 &40&81& 244&82 &9232&83 &9232&84& 64\\
85 & 256&86 &196&87& 592&88 &52&89 &304&90& 136\\
91 & 9232&92 &160&93& 280&94 &9232&95 &9232&96& 48\\
97 & 9232&98 &148&99& 448&100 &88&101 &304&102& 232\\
103 & 9232&104 &52&105& 808&106 &160&107 &9232&108& 9232\\
109 & 9232&110 &9232&111& 9232&112 &56&113 &340&114& 196\\
115 & 520&116 &88&117& 352&118 &304&119 &808&120& 160\\
121 & 9232&122 &184&123& 628&124 &9232&125 &9232&126& 9232\\
127 & 4372&128 &64&129& 9232&130 &196&131 &592&132& 100\\
133 & 400&134 &304&135& 916&136 &68&137 &9232&138& 208\\
139 & 628&140 &160&141& 424&142 &9232&143 &9232&144& 72\\
145 & 9232&146 &9232&147& 9232&148 &112&149 &448&150& 340\\
151 & 1024&152 &88&153& 520&154 &232&155 &9232&156& 304\\
157 & 472&158 &808&159& 9232&160 &80&161 &9232&162& 244\\
163 & 736&164 &9232&165& 9232&166 &9232&167 &9232&168& 84\\
169 & 4372&170 &256&171& 9232&172 &196&173 &520&174& 592\\
175 & 9232&176 &88&177& 532&178 &304&179 &808&180& 136\\
181 & 544&182 &9232&183& 9232&184 &160&185 &628&186& 280\\
187 & 952&188 &9232&189& 9232&190 &9232&191 &4372&192& 96\\
193 & 9232&194 &9232&195& 9232&196 &148&197 &592&198& 448\\
199 & 9232&200 &100&201& 1024&202 &304&203 &916&204& 232\\
205 & 616&206 &9232&207& 9232&208 &104&209 &628&210& 808\\
211 & 952&212 &160&213& 640&214 &9232&215 &9232&216& 9232\\
217 & 736&218 &9232&219& 1672&220 &9232&221 &9232&222& 9232\\
223 & 9232&224 &112&225& 4372&226 &340&227 &1024&228& 196\\
229 & 688&230 &520&231& 9232&232 &116&233 &9232&234& 352\\
235 & 9232&236 &304&237& 712&238 &808&239 &9232&240& 160\\
241 & 724&242 &9232&243& 9232&244 &184&245 &736&246& 628\\
247 & 1672&248 &9232&249& 952&250 &9232&251 &9232&252& 9232\\
253 & 9232&254 &4372&255& 13120&256 &128&257 &9232&258& 9232\\
259 & 9232&260 &196&261& 784&262 &592&263 &9232&264& 132\\
265 & 9232&266 &400&267& 1204&268 &304&269 &808&270& 916\\
271 & 2752&272 &136&273& 820&274 &9232&275 &9232&276& 208\\
277 & 832&278 &628&279& 1888&280 &160&281 &952&282& 424\\
283 & 9232&284 &9232&285& 9232&286 &9232&287 &4372&288& 144\\
289 & 868&290 &9232&291& 9232&292 &9232&293 &9232&294& 9232\\
295 & 2248&296 &148&297& 9232&298 &448&299 &9232&300& 340\\
301 & 904&302 &1024&303& 3076&304 &152&305 &916&306& 520\\
307 & 1384&308 &232&309& 928&310 &9232&311 &9232&312& 304\\
313 & 9232&314 &472&315& 1600&316 &808&317 &952&318& 9232\\
319 & 9232&320 &160&321& 964&322 &9232&323 &9232&324& 244\\
325 & 976&326 &736&327& 9232&328 &9232&329 &1672&330& 9232\\
331 & 1492&332 &9232&333& 9232&334 &9232&335 &9232&336& 168\\
337 & 9232&338 &4372&339& 4372&340 &256&341 &1024&342& 9232\\
343 & 9232&344 &196&345& 9232&346 &520&347 &9232&348& 592\\
349 & 1048&350 &9232&351& 9232&352 &176&353 &9232&354& 532\\
355 & 1600&356 &304&357& 1072&358 &808&359 &9232&360& 180\\
361 & 2752&362 &544&363& 1636&364 &9232&365 &9232&366& 9232\\
367 & 4192&368 &184&369& 1108&370 &628&371 &1672&372& 280\\
373 & 1120&374 &952&375& 2536&376 &9232&377 &9232&378& 9232\\
379 & 2752&380 &9232&381& 9232&382 &4372&383 &13120&384& 192\\
385 & 1156&386 &9232&387& 9232&388 &9232&389 &9232&390& 9232\\
391 & 9232&392 &196&393& 2248&394 &592&395 &9232&396& 448\\
397 & 1192&398 &9232&399& 9232&400 &200&401 &1204&402& 1024\\
403 & 1816&404 &304&405& 1216&406 &916&407 &2752&408& 232\\
409 & 1384&410 &616&411& 9232&412 &9232&413 &9232&414& 9232\\
415 & 9232&416 &208&417& 9232&418 &628&419 &1888&420& 808\\
421 & 1264&422 &952&423& 3220&424 &212&425 &9232&426& 640\\
427 & 2752&428 &9232&429& 9232&430 &9232&431 &4372&432& 9232\\
433 & 1300&434 &736&435& 1960&436 &9232&437 &9232&438& 1672\\
439 & 2968&440 &9232&441& 1492&442 &9232&443 &2248&444& 9232\\
445 & 9232&446 &9232&447& 39364&448 &224&449 &9232&450& 4372\\
451 & 4372&452 &340&453& 1360&454 &1024&455 &3076&456& 228\\
457 & 9232&458 &688&459& 9232&460 &520&461 &1384&462& 9232\\
463 & 9232&464 &232&465& 1396&466 &9232&467 &9232&468& 352\\
469 & 1408&470 &9232&471& 9232&472 &304&473 &1600&474& 712\\
475 & 3616&476 &808&477& 1432&478 &9232&479 &9232&480& 240\\
481 & 2752&482 &724&483& 2176&484 &9232&485 &9232&486& 9232\\
487 & 9232&488 &244&489& 4192&490 &736&491 &9232&492& 628\\
493 & 1480&494 &1672&495& 14308&496 &9232&497 &1492&498& 952\\
499 & 2248&500 &9232&501& 9232&502 &9232&503 &9232&504& 9232\\
505 & 2752&506 &9232&507& 2896&508 &4372&509 &4372&510& 13120\\
511 & 39364&512 &256&513& 1540&514 &9232&515 &9232&516& 9232\\
517 & 9232&518 &9232&519& 3508&520 &260&521 &9232&522& 784\\
523 & 9232&524 &592&525& 1576&526 &9232&527 &9232&528& 264\\
529 & 1588&530 &9232&531& 9232&532 &400&533 &1600&534& 1204\\
535 & 3616&536 &304&537& 1816&538 &808&539 &9232&540& 916\\
541 & 1624&542 &2752&543& 9232&544 &272&545 &1636&546& 820\\
547 & 2464&548 &9232&549& 9232&550 &9232&551 &4192&552& 276\\
553 & 9232&554 &832&555& 2500&556 &628&557 &1672&558& 1888\\
559 & 8080&560 &280&561& 1684&562 &952&563 &2536&564& 424\\
565 & 1696&566 &9232&567& 9232&568 &9232&569 &2752&570& 9232\\
571 & 2896&572 &9232&573& 9232&574 &4372&575 &13120&576& 288\\
577 & 1732&578 &868&579& 2608&580 &9232&581 &9232&582& 9232\\
583 & 3940&584 &9232&585& 2968&586 &9232&587 &9232&588& 9232\\
589 & 9232&590 &2248&591& 5992&592 &296&593 &9232&594& 9232\\
595 & 9232&596 &448&597& 1792&598 &9232&599 &9232&600& 340\\
601 & 4372&602 &904&603& 5812&604 &1024&605 &1816&606& 3076\\
607 & 9232&608 &304&609& 9232&610 &916&611 &2752&612& 520\\
613 & 1840&614 &1384&615& 10528&616 &308&617 &9232&618& 928\\
619 & 9232&620 &9232&621& 9232&622 &9232&623 &9232&624& 312\\
625 & 1876&626 &9232&627& 9232&628 &472&629 &1888&630& 1600\\
631 & 4264&632 &808&633& 3616&634 &952&635 &3220&636& 9232\\
637 & 9232&638 &9232&639& 41524&640 &320&641 &2752&642& 964\\
643 & 2896&644 &9232&645& 9232&646 &9232&647 &4372&648& 324\\
649 & 9232&650 &976&651& 9232&652 &736&653 &1960&654& 9232\\
655 & 9232&656 &9232&657& 1972&658 &1672&659 &2968&660& 9232\\
661 & 9232&662 &1492&663& 4480&664 &9232&665 &2248&666& 9232\\
667 & 21688&668 &9232&669& 9232&670 &9232&671 &39364&672& 336\\
673 & 2752&674 &9232&675& 9232&676 &4372&677 &4372&678& 4372\\
679 & 5812&680 &340&681& 39364&682 &1024&683 &3076&684& 9232\\
685 & 9232&686 &9232&687& 6964&688 &344&689 &9232&690& 9232\\
691 & 9232&692 &520&693& 2080&694 &9232&695 &9232&696& 592\\
697 & 9232&698 &1048&699& 3544&700 &9232&701 &9232&702& 9232\\
703 & 250504&704 &352&705& 2116&706 &9232&707 &9232&708& 532\\
709 & 2128&710 &1600&711& 4804&712 &356&713 &3616&714& 1072\\
715 & 3220&716 &808&717& 2152&718 &9232&719 &9232&720& 360\\
721 & 2164&722 &2752&723& 3256&724 &544&725 &2176&726& 1636\\
727 & 4912&728 &9232&729& 2464&730 &9232&731 &9232&732& 9232\\
733 & 9232&734 &4192&735& 11176&736 &368&737 &9232&738& 1108\\
739 & 3328&740 &628&741& 2224&742 &1672&743 &14308&744& 372\\
745 & 8080&746 &1120&747& 4264&748 &952&749 &2248&750& 2536\\
751 & 21688&752 &9232&753& 2260&754 &9232&755 &9232&756& 9232\\
757 & 9232&758 &2752&759& 5128&760 &9232&761 &2896&762& 9232\\
763 & 9232&764 &4372&765& 4372&766 &13120&767 &39364&768& 384\\
769 & 2308&770 &1156&771& 3472&772 &9232&773 &9232&774& 9232\\
775 & 9232&776 &9232&777& 3940&778 &9232&779 &3508&780& 9232\\
781 & 9232&782 &9232&783& 9232&784 &392&785 &9232&786& 2248\\
787 & 3544&788 &592&789& 2368&790 &9232&791 &9232&792& 448\\
793 & 9232&794 &1192&795& 39364&796 &9232&797 &9232&798& 9232\\
799 & 12148&800 &400&801& 4372&802 &1204&803 &3616&804& 1024\\
805 & 2416&806 &1816&807& 39364&808 &404&809 &9232&810& 1216\\
811 & 9232&812 &916&813& 2440&814 &2752&815 &9232&816& 408\\
817 & 2452&818 &1384&819& 3688&820 &616&821 &2464&822& 9232\\
823 & 9232&824 &9232&825& 9232&826 &9232&827 &4192&828& 9232\\
829 & 9232&830 &9232&831& 18952&832 &416&833 &2500&834& 9232\\
835 & 9232&836 &628&837& 2512&838 &1888&839 &8080&840& 808\\
841 & 4264&842 &1264&843& 3796&844 &952&845 &2536&846& 3220\\
847 & 8584&848 &424&849& 9232&850 &9232&851 &9232&852& 640\\
853 & 2560&854 &2752&855& 5776&856 &9232&857 &2896&858& 9232\\
859 & 9232&860 &9232&861& 9232&862 &4372&863 &13120&864& 9232\\
865 & 9232&866 &1300&867& 3904&868 &736&869 &2608&870& 1960\\
871 & 190996&872 &9232&873& 9232&874 &9232&875 &3940&876& 1672\\
877 & 2632&878 &2968&879& 10024&880 &9232&881 &9232&882& 1492\\
883 & 3976&884 &9232&885& 9232&886 &2248&887 &5992&888& 9232\\
889 & 21688&890 &9232&891& 8584&892 &9232&893 &9232&894& 39364\\
895 & 39364&896 &448&897& 2752&898 &9232&899 &9232&900& 4372\\
901 & 4372&902 &4372&903& 9232&904 &452&905 &5812&906& 1360\\
907 & 13120&908 &1024&909& 2728&910 &3076&911 &9232&912& 456\\
913 & 9232&914 &9232&915& 9232&916 &688&917 &2752&918& 9232\\
919 & 9232&920 &520&921& 9232&922 &1384&923 &10528&924& 9232\\
925 & 9232&926 &9232&927& 15856&928 &464&929 &9232&930& 1396\\
931 & 4192&932 &9232&933& 9232&934 &9232&935 &9232&936& 468\\
937 & 250504&938 &1408&939& 9232&940 &9232&941 &9232&942& 9232\\
943 & 9556&944 &472&945& 2836&946 &1600&947 &4264&948& 712\\
949 & 2848&950 &3616&951& 6424&952 &808&953 &3220&954& 1432\\
955 & 4840&956 &9232&957& 9232&958 &9232&959 &41524&960& 480\\
961 & 2884&962 &2752&963& 4336&964 &724&965 &2896&966& 2176\\
967 & 9232&968 &9232&969& 4912&970 &9232&971 &4372&972& 9232\\
973 & 9232&974 &9232&975& 9880&976 &488&977 &9232&978& 4192\\
979 & 4408&980 &736&981& 2944&982 &9232&983 &9232&984& 628\\
985 & 3328&986 &1480&987& 7504&988 &1672&989 &2968&990& 14308\\
991 & 15064&992 &9232&993& 8080&994 &1492&995 &4480&996& 952\\
997 & 2992&998 &2248&999& 11392&1000 &9232&1001 &21688&1002& 9232\\
1003 & 8584&1004 &9232&1005& 9232&1006 &9232&1007 &39364&1008& 9232\\
1009 & 9232&1010 &2752&1011& 4552&1012 &9232&1013 &9232&1014& 2896\\
1015 & 6856&1016 &4372&1017& 9232&1018 &4372&1019 &5812&1020& 13120\\
1021 & 13120&1022 &39364&1023& 118096&1024 &512&1025 &3076&1026& 1540\\
1027 & 4624&1028 &9232&1029& 9232&1030 &9232&1031 &6964&1032& 9232\\
1033 & 9232&1034 &9232&1035& 9232&1036 &9232&1037 &9232&1038& 3508\\
1039 & 10528&1040 &520&1041& 9232&1042 &9232&1043 &9232&1044& 784\\
1045 & 3136&1046 &9232&1047& 9232&1048 &592&1049 &3544&1050& 1576\\
1051 & 45520&1052 &9232&1053& 9232&1054 &9232&1055 &250504&1056& 528\\
1057 & 9232&1058 &1588&1059& 4768&1060 &9232&1061 &9232&1062& 9232\\
1063 & 8080&1064 &532&1065& 12148&1066 &1600&1067 &4804&1068& 1204\\
1069 & 3208&1070 &3616&1071& 10852&1072 &536&1073 &3220&1074& 1816\\
1075 & 4840&1076 &808&1077& 3232&1078 &9232&1079 &9232&1080& 916\\
1081 & 9232&1082 &1624&1083& 9232&1084 &2752&1085 &3256&1086& 9232\\
1087 & 24784&1088 &544&1089& 3268&1090 &1636&1091 &4912&1092& 820\\
1093 & 3280&1094 &2464&1095& 7396&1096 &9232&1097 &9232&1098& 9232\\
1099 & 4948&1100 &9232&1101& 9232&1102 &4192&1103 &11176&1104& 552\\
1105 & 9232&1106 &9232&1107& 9232&1108 &832&1109 &3328&1110& 2500\\
1111 & 7504&1112 &628&1113& 9232&1114 &1672&1115 &14308&1116& 1888\\
1117 & 3352&1118 &8080&1119& 17008&1120 &560&1121 &4264&1122& 1684\\
1123 & 5056&1124 &952&1125& 3376&1126 &2536&1127 &21688&1128& 564\\
1129 & 8584&1130 &1696&1131& 5092&1132 &9232&1133 &9232&1134& 9232\\
1135 & 14560&1136 &9232&1137& 3412&1138 &2752&1139 &5128&1140& 9232\\
1141 & 9232&1142 &2896&1143& 7720&1144 &9232&1145 &9232&1146& 9232\\
1147 & 5812&1148 &4372&1149& 4372&1150 &13120&1151 &39364&1152& 576\\
1153 & 9232&1154 &1732&1155& 5200&1156 &868&1157 &3472&1158& 2608\\
1159 & 7828&1160 &9232&1161& 190996&1162 &9232&1163 &9232&1164& 9232\\
1165 & 9232&1166 &3940&1167& 11824&1168 &9232&1169 &3508&1170& 2968\\
1171 & 5272&1172 &9232&1173& 9232&1174 &9232&1175 &9232&1176& 9232\\
1177 & 3976&1178 &9232&1179& 13444&1180 &2248&1181 &3544&1182& 5992\\
1183 & 45520&1184 &592&1185& 21688&1186 &9232&1187 &9232&1188& 9232\\
1189 & 9232&1190 &9232&1191& 15280&1192 &596&1193 &39364&1194& 1792\\
1195 & 9232&1196 &9232&1197& 9232&1198 &9232&1199 &12148&1200& 600\\
1201 & 4372&1202 &4372&1203& 5416&1204 &904&1205 &3616&1206& 5812\\
1207 & 8152&1208 &1024&1209& 13120&1210 &1816&1211 &39364&1212& 3076\\
1213 & 3640&1214 &9232&1215& 27700&1216 &608&1217 &9232&1218& 9232\\
1219 & 9232&1220 &916&1221& 3664&1222 &2752&1223 &9232&1224& 612\\
1225 & 9232&1226 &1840&1227& 9232&1228 &1384&1229 &3688&1230& 10528\\
1231 & 12472&1232 &616&1233& 9232&1234 &9232&1235 &9232&1236& 928\\
1237 & 3712&1238 &9232&1239& 9232&1240 &9232&1241 &4192&1242& 9232\\
1243 & 9448&1244 &9232&1245& 9232&1246 &9232&1247 &18952&1248& 624\\
1249 & 250504&1250 &1876&1251& 5632&1252 &9232&1253 &9232&1254& 9232\\
1255 & 14308&1256 &628&1257& 9556&1258 &1888&1259 &8080&1260& 1600\\
1261 & 3784&1262 &4264&1263& 138400&1264 &808&1265 &3796&1266& 3616\\
1267 & 5704&1268 &952&1269& 3808&1270 &3220&1271 &8584&1272& 9232\\
1273 & 4840&1274 &9232&1275& 39364&1276 &9232&1277 &9232&1278& 41524\\
1279 & 65608&1280 &640&1281& 3844&1282 &2752&1283 &5776&1284& 964\\
1285 & 3856&1286 &2896&1287& 9232&1288 &9232&1289 &9232&1290& 9232\\
1291 & 5812&1292 &9232&1293& 9232&1294 &4372&1295 &13120&1296& 648\\
1297 & 9232&1298 &9232&1299& 9232&1300 &976&1301 &3904&1302& 9232\\
1303 & 9232&1304 &736&1305& 4408&1306 &1960&1307 &190996&1308& 9232\\
1309 & 9232&1310 &9232&1311& 19924&1312 &9232&1313 &3940&1314& 1972\\
1315 & 5920&1316 &1672&1317& 3952&1318 &2968&1319 &10024&1320& 9232\\
1321 & 15064&1322 &9232&1323& 5956&1324 &1492&1325 &3976&1326& 4480\\
1327 & 13444&1328 &9232&1329& 3988&1330 &2248&1331 &5992&1332& 9232\\
1333 & 9232&1334 &21688&1335& 21688&1336 &9232&1337 &8584&1338& 9232\\
1339 & 6784&1340 &9232&1341& 9232&1342 &39364&1343 &39364&1344& 672\\
1345 & 9232&1346 &2752&1347& 6064&1348 &9232&1349 &9232&1350& 9232\\
1351 & 17332&1352 &4372&1353& 6856&1354 &4372&1355 &9232&1356& 4372\\
1357 & 4372&1358 &5812&1359& 13768&1360 &680&1361 &13120&1362& 39364\\
1363 & 39364&1364 &1024&1365& 4096&1366 &3076&1367 &9232&1368& 9232\\
1369 & 4624&1370 &9232&1371& 10420&1372 &9232&1373 &9232&1374& 6964\\
1375 & 20896&1376 &688&1377& 9232&1378 &9232&1379 &9232&1380& 9232\\
1381 & 9232&1382 &9232&1383& 95956&1384 &692&1385 &10528&1386& 2080\\
1387 & 6244&1388 &9232&1389& 9232&1390 &9232&1391 &15856&1392& 696\\
1393 & 4180&1394 &9232&1395& 9232&1396 &1048&1397 &4192&1398& 3544\\
1399 & 9448&1400 &9232&1401& 45520&1402 &9232&1403 &9232&1404& 9232\\
1405 & 9232&1406 &250504&1407& 250504&1408 &704&1409 &9232&1410& 2116\\
1411 & 6352&1412 &9232&1413& 9232&1414 &9232&1415 &9556&1416& 708\\
1417 & 8080&1418 &2128&1419& 9232&1420 &1600&1421 &4264&1422& 4804\\
1423 & 14416&1424 &712&1425& 4276&1426 &3616&1427 &6424&1428& 1072\\
1429 & 4288&1430 &3220&1431& 9664&1432 &808&1433 &4840&1434& 2152\\
1435 & 39364&1436 &9232&1437& 9232&1438 &9232&1439 &41524&1440& 720\\
1441 & 9232&1442 &2164&1443& 6496&1444 &2752&1445 &4336&1446& 3256\\
1447 & 10996&1448 &724&1449& 24784&1450 &2176&1451 &9232&1452& 1636\\
1453 & 4360&1454 &4912&1455& 14740&1456 &9232&1457 &4372&1458& 2464\\
1459 & 6568&1460 &9232&1461& 9232&1462 &9232&1463 &9880&1464& 9232\\
1465 & 4948&1466 &9232&1467& 9232&1468 &4192&1469 &4408&1470& 11176\\
1471 & 190996&1472 &736&1473& 9232&1474 &9232&1475 &9232&1476& 1108\\
1477 & 4432&1478 &3328&1479& 9988&1480 &740&1481 &7504&1482& 2224\\
1483 & 9232&1484 &1672&1485& 4456&1486 &14308&1487 &15064&1488& 744\\
1489 & 4468&1490 &8080&1491& 8080&1492 &1120&1493 &4480&1494& 4264\\
1495 & 10096&1496 &952&1497& 5056&1498 &2248&1499 &11392&1500& 2536\\
1501 & 4504&1502 &21688&1503& 22840&1504 &9232&1505 &8584&1506& 2260\\
1507 & 6784&1508 &9232&1509& 9232&1510 &9232&1511 &39364&1512& 9232\\
1513 & 14560&1514 &9232&1515& 65608&1516 &2752&1517 &4552&1518& 5128\\
1519 & 25972&1520 &9232&1521& 9232&1522 &2896&1523 &6856&1524& 9232\\
1525 & 9232&1526 &9232&1527& 10312&1528 &4372&1529 &5812&1530& 4372\\
1531 & 8728&1532 &13120&1533& 13120&1534 &39364&1535 &118096&1536& 768\\
1537 & 9232&1538 &2308&1539& 6928&1540 &1156&1541 &4624&1542& 3472\\
1543 & 10420&1544 &9232&1545& 7828&1546 &9232&1547 &6964&1548& 9232\\
1549 & 9232&1550 &9232&1551& 15712&1552 &9232&1553 &9232&1554& 3940\\
1555 & 7000&1556 &9232&1557& 9232&1558 &3508&1559 &10528&1560& 9232\\
1561 & 5272&1562 &9232&1563& 26728&1564 &9232&1565 &9232&1566& 9232\\
1567 & 23812&1568 &784&1569& 4708&1570 &9232&1571 &9232&1572& 2248\\
1573 & 4720&1574 &3544&1575& 11968&1576 &788&1577 &45520&1578& 2368\\
1579 & 9232&1580 &9232&1581& 9232&1582 &9232&1583 &250504&1584& 792\\
1585 & 9232&1586 &9232&1587& 9232&1588 &1192&1589 &4768&1590& 39364\\
1591 & 39364&1592 &9232&1593& 9232&1594 &9232&1595 &8080&1596& 9232\\
1597 & 9232&1598 &12148&1599& 36448&1600 &800&1601 &4804&1602& 4372\\
1603 & 7216&1604 &1204&1605& 4816&1606 &3616&1607 &10852&1608& 1024\\
1609 & 8152&1610 &2416&1611& 7252&1612 &1816&1613 &4840&1614& 39364\\
1615 & 39364&1616 &808&1617& 4852&1618 &9232&1619 &9232&1620& 1216\\
1621 & 4864&1622 &9232&1623& 10960&1624 &916&1625 &9232&1626& 2440\\
1627 & 13912&1628 &2752&1629& 4888&1630 &9232&1631 &24784&1632& 816\\
1633 & 9232&1634 &2452&1635& 7360&1636 &1384&1637 &4912&1638& 3688\\
1639 & 95956&1640 &820&1641& 12472&1642 &2464&1643 &7396&1644& 9232\\
1645 & 9232&1646 &9232&1647& 18772&1648 &9232&1649 &4948&1650& 9232\\
1651 & 9232&1652 &9232&1653& 9232&1654 &4192&1655 &11176&1656& 9232\\
1657 & 9448&1658 &9232&1659& 8404&1660 &9232&1661 &9232&1662& 18952\\
1663 & 95956&1664 &832&1665& 250504&1666 &2500&1667 &7504&1668& 9232\\
1669 & 9232&1670 &9232&1671& 11284&1672 &836&1673 &14308&1674& 2512\\
1675 & 9232&1676 &1888&1677& 5032&1678 &8080&1679 &17008&1680& 840\\
1681 & 5044&1682 &4264&1683& 7576&1684 &1264&1685 &5056&1686& 3796\\
1687 & 11392&1688 &952&1689& 5704&1690 &2536&1691 &21688&1692& 3220\\
1693 & 5080&1694 &8584&1695& 48904&1696 &848&1697 &5092&1698& 9232\\
1699 & 9232&1700 &9232&1701& 9232&1702 &9232&1703 &14560&1704& 852\\
1705 & 65608&1706 &2560&1707& 9232&1708 &2752&1709 &5128&1710& 5776\\
1711 & 17332&1712 &9232&1713& 5140&1714 &2896&1715 &7720&1716& 9232\\
1717 & 9232&1718 &9232&1719& 11608&1720 &9232&1721 &5812&1722& 9232\\
1723 & 8728&1724 &4372&1725& 5176&1726 &13120&1727 &39364&1728& 9232\\
1729 & 9232&1730 &9232&1731& 9232&1732 &1300&1733 &5200&1734& 3904\\
1735 & 11716&1736 &868&1737& 9232&1738 &2608&1739 &7828&1740& 1960\\
1741 & 5224&1742 &190996&1743& 190996&1744 &9232&1745 &9232&1746& 9232\\
1747 & 9232&1748 &9232&1749& 9232&1750 &3940&1751 &11824&1752& 1672\\
1753 & 5920&1754 &2632&1755& 13336&1756 &2968&1757 &5272&1758& 10024\\
1759 & 26728&1760 &9232&1761& 15064&1762 &9232&1763 &9232&1764& 1492\\
1765 & 5296&1766 &3976&1767& 22660&1768 &9232&1769 &13444&1770& 9232\\
1771 & 10096&1772 &2248&1773& 5320&1774 &5992&1775 &45520&1776& 9232\\
1777 & 9232&1778 &21688&1779& 21688&1780 &9232&1781 &9232&1782& 8584\\
1783 & 12040&1784 &9232&1785& 6784&1786 &9232&1787 &15280&1788& 39364\\
1789 & 39364&1790 &39364&1791& 103336&1792 &896&1793 &9232&1794& 2752\\
1795 & 8080&1796 &9232&1797& 9232&1798 &9232&1799 &12148&1800& 4372\\
1801 & 17332&1802 &4372&1803& 8116&1804 &4372&1805 &5416&1806& 9232\\
1807 & 18304&1808 &904&1809& 5428&1810 &5812&1811 &8152&1812& 1360\\
1813 & 5440&1814 &13120&1815& 13120&1816 &1024&1817 &39364&1818& 2728\\
1819 & 1276936&1820 &3076&1821& 5464&1822 &9232&1823 &27700&1824& 912\\
1825 & 5476&1826 &9232&1827& 9232&1828 &9232&1829 &9232&1830& 9232\\
1831 & 13912&1832 &916&1833& 20896&1834 &2752&1835 &9232&1836& 9232\\
1837 & 9232&1838 &9232&1839& 18628&1840 &920&1841 &9232&1842& 9232\\
1843 & 9232&1844 &1384&1845& 5536&1846 &10528&1847 &12472&1848& 9232\\
1849 & 6244&1850 &9232&1851& 9376&1852 &9232&1853 &9232&1854& 15856\\
1855 & 42280&1856 &928&1857& 5572&1858 &9232&1859 &9232&1860& 1396\\
1861 & 5584&1862 &4192&1863& 15928&1864 &9232&1865 &9448&1866& 9232\\
1867 & 8404&1868 &9232&1869& 9232&1870 &9232&1871 &18952&1872& 936\\
1873 & 9232&1874 &250504&1875& 250504&1876 &1408&1877 &5632&1878& 9232\\
1879 & 12688&1880 &9232&1881& 6352&1882 &9232&1883 &14308&1884& 9232\\
1885 & 9232&1886 &9556&1887& 28672&1888 &944&1889 &8080&1890& 2836\\
1891 & 8512&1892 &1600&1893& 5680&1894 &4264&1895 &138400&1896& 948\\
1897 & 14416&1898 &2848&1899& 8548&1900 &3616&1901 &5704&1902& 6424\\
1903 & 21688&1904 &952&1905& 5716&1906 &3220&1907 &8584&1908& 1432\\
1909 & 5728&1910 &4840&1911& 12904&1912 &9232&1913 &39364&1914& 9232\\
1915 & 1276936&1916 &9232&1917& 9232&1918 &41524&1919 &65608&1920& 960\\
1921 & 9232&1922 &2884&1923& 8656&1924 &2752&1925 &5776&1926& 4336\\
1927 & 13012&1928 &964&1929& 10996&1930 &2896&1931 &9232&1932& 2176\\
1933 & 5800&1934 &9232&1935& 19600&1936 &9232&1937 &5812&1938& 4912\\
1939 & 8728&1940 &9232&1941& 9232&1942 &4372&1943 &13120&1944& 9232\\
1945 & 6568&1946 &9232&1947& 22192&1948 &9232&1949 &9232&1950& 9880\\
1951 & 33352&1952 &976&1953& 5860&1954 &9232&1955 &9232&1956& 4192\\
1957 & 5872&1958 &4408&1959& 80512&1960 &980&1961 &190996&1962& 2944\\
1963 & 9448&1964 &9232&1965& 9232&1966 &9232&1967 &19924&1968& 984\\
1969 & 5908&1970 &3328&1971& 8872&1972 &1480&1973 &5920&1974& 7504\\
1975 & 13336&1976 &1672&1977& 9232&1978 &2968&1979 &10024&1980& 14308\\
1981 & 14308&1982 &15064&1983& 50848&1984 &9232&1985 &5956&1986& 8080\\
1987 & 8944&1988 &1492&1989& 5968&1990 &4480&1991 &13444&1992& 996\\
1993 & 10096&1994 &2992&1995& 8980&1996 &2248&1997 &5992&1998& 11392\\
1999 & 20248&2000 &9232&2001& 6004&2002 &21688&2003 &21688&2004& 9232\\
2005 & 9232&2006 &8584&2007& 13552&2008 &9232&2009 &6784&2010& 9232\\
2011 & 15280&2012 &9232&2013& 9232&2014 &39364&2015 &39364&2016& 9232\\
2017 & 14560&2018 &9232&2019& 9232&2020 &2752&2021 &6064&2022& 4552\\
2023 & 23056&2024 &9232&2025& 25972&2026 &9232&2027 &17332&2028& 2896\\
2029 & 6088&2030 &6856&2031& 34720&2032 &4372&2033 &9232&2034& 9232\\
2035 & 9232&2036 &4372&2037& 6112&2038 &5812&2039 &13768&2040& 13120\\
2041 & 8728&2042 &13120&2043& 44224&2044 &39364&2045 &39364&2046& 118096\\
2047 & 1276936&2048 &1024&2049& 9232&2050 &3076&2051 &9232&2052& 1540\\
2053 & 6160&2054 &4624&2055& 13876&2056 &9232&2057 &10420&2058& 9232\\
2059 & 9268&2060 &9232&2061& 9232&2062 &6964&2063 &20896&2064& 9232\\
2065 & 9232&2066 &9232&2067& 9304&2068 &9232&2069 &9232&2070& 9232\\
2071 & 13984&2072 &9232&2073& 7000&2074 &9232&2075 &95956&2076& 3508\\
2077 & 6232&2078 &10528&2079& 31588&2080 &1040&2081 &6244&2082& 9232\\
2083 & 9376&2084 &9232&2085& 9232&2086 &9232&2087 &15856&2088& 1044\\
2089 & 23812&2090 &3136&2091& 9412&2092 &9232&2093 &9232&2094& 9232\\
2095 & 21220&2096 &1048&2097& 6292&2098 &3544&2099 &9448&2100& 1576\\
2101 & 6304&2102 &45520&2103& 45520&2104 &9232&2105 &9232&2106& 9232\\
2107 & 21688&2108 &9232&2109& 9232&2110 &250504&2111 &250504&2112& 1056\\
2113 & 9232&2114 &9232&2115& 9520&2116 &1588&2117 &6352&2118& 4768\\
2119 & 14308&2120 &9232&2121& 39364&2122 &9232&2123 &9556&2124& 9232\\
2125 & 9232&2126 &8080&2127& 21544&2128 &1064&2129 &9232&2130& 12148\\
2131 & 12148&2132 &1600&2133& 6400&2134 &4804&2135 &14416&2136& 1204\\
2137 & 7216&2138 &3208&2139& 61720&2140 &3616&2141 &6424&2142& 10852\\
2143 & 32560&2144 &1072&2145& 8152&2146 &3220&2147 &9664&2148& 1816\\
2149 & 6448&2150 &4840&2151& 176740&2152 &1076&2153 &39364&2154& 3232\\
2155 & 1276936&2156 &9232&2157& 9232&2158 &9232&2159 &41524&2160& 1080\\
2161 & 6484&2162 &9232&2163& 9736&2164 &1624&2165 &6496&2166& 9232\\
2167 & 14632&2168 &2752&2169& 13912&2170 &3256&2171 &10996&2172& 9232\\
2173 & 9232&2174 &24784&2175& 74356&2176 &1088&2177 &9232&2178& 3268\\
2179 & 9808&2180 &1636&2181& 6544&2182 &4912&2183 &14740&2184& 1092\\
2185 & 95956&2186 &3280&2187& 10528&2188 &2464&2189 &6568&2190& 7396\\
2191 & 22192&2192 &9232&2193& 9232&2194 &9232&2195 &9880&2196& 9232\\
2197 & 9232&2198 &4948&2199& 14848&2200 &9232&2201 &9232&2202& 9232\\
2203 & 25108&2204 &4192&2205& 6616&2206 &11176&2207 &190996&2208& 1104\\
2209 & 9448&2210 &9232&2211& 9952&2212 &9232&2213 &9232&2214& 9232\\
2215 & 345544&2216 &1108&2217& 95956&2218 &3328&2219 &9988&2220& 2500\\
2221 & 6664&2222 &7504&2223& 250504&2224 &1112&2225 &9232&2226& 9232\\
2227 & 10024&2228 &1672&2229& 6688&2230 &14308&2231 &15064&2232& 1888\\
2233 & 9232&2234 &3352&2235& 11320&2236 &8080&2237 &8080&2238& 17008\\
2239 & 51028&2240 &1120&2241& 6724&2242 &4264&2243 &10096&2244& 1684\\
2245 & 6736&2246 &5056&2247& 15172&2248 &1124&2249 &11392&2250& 3376\\
2251 & 10132&2252 &2536&2253& 6760&2254 &21688&2255 &22840&2256& 1128\\
2257 & 6772&2258 &8584&2259& 10168&2260 &1696&2261 &6784&2262& 5092\\
2263 & 15280&2264 &9232&2265& 9232&2266 &9232&2267 &39364&2268& 9232\\
2269 & 9232&2270 &14560&2271& 34504&2272 &9232&2273 &65608&2274& 3412\\
2275 & 10240&2276 &2752&2277& 6832&2278 &5128&2279 &25972&2280& 9232\\
2281 & 17332&2282 &9232&2283& 13012&2284 &2896&2285 &6856&2286& 7720\\
2287 & 250504&2288 &9232&2289& 9232&2290 &9232&2291 &10312&2292& 9232\\
2293 & 9232&2294 &5812&2295& 15496&2296 &4372&2297 &8728&2298& 4372\\
2299 & 44224&2300 &13120&2301& 13120&2302 &39364&2303 &118096&2304& 1152\\
2305 & 9232&2306 &9232&2307& 10384&2308 &1732&2309 &6928&2310& 5200\\
2311 & 15604&2312 &1156&2313& 11716&2314 &3472&2315 &10420&2316& 2608\\
2317 & 6952&2318 &7828&2319& 23488&2320 &9232&2321 &6964&2322& 190996\\
2323 & 190996&2324 &9232&2325& 9232&2326 &9232&2327 &15712&2328& 9232\\
2329 & 9232&2330 &9232&2331& 59776&2332 &3940&2333 &7000&2334& 11824\\
2335 & 95956&2336 &9232&2337& 7012&2338 &3508&2339 &10528&2340& 2968\\
2341 & 7024&2342 &5272&2343& 17800&2344 &9232&2345 &26728&2346& 9232\\
2347 & 14308&2348 &9232&2349& 9232&2350 &9232&2351 &23812&2352& 9232\\
2353 & 7060&2354 &3976&2355& 10600&2356 &9232&2357 &9232&2358& 13444\\
2359 & 15928&2360 &2248&2361& 10096&2362 &3544&2363 &11968&2364& 5992\\
2365 & 7096&2366 &45520&2367& 53944&2368 &1184&2369 &9232&2370& 21688\\
2371 & 21688&2372 &9232&2373& 9232&2374 &9232&2375 &250504&2376& 9232\\
2377 & 12040&2378 &9232&2379& 10708&2380 &9232&2381 &9232&2382& 15280\\
2383 & 24136&2384 &1192&2385& 39364&2386 &39364&2387 &39364&2388& 1792\\
2389 & 7168&2390 &9232&2391& 16144&2392 &9232&2393 &8080&2394& 9232\\
2395 & 18196&2396 &9232&2397& 9232&2398 &12148&2399 &36448&2400& 1200\\
\bottomrule\end{longtable}
\section{Cicli massini}
\begin{longtable}{llllllllllll}\toprule
\caption{Cicli massimi}\\
\midrule
\endfirsthead
\multicolumn{12}{c} {\tablename\ \thetable\ -- \textit{Continua dalla pagina precedente}} \\
\toprule
\endhead
\bottomrule
\multicolumn{12}{r} {\textit{Continua nella pagina successiva}} \\
\endfoot
\endlastfoot
1 &2& 4 &8 &16& 20&24 &32& 40 &48 &52& 56\\
64 &68& 72 &80 &84& 88&96 &100& 104 &112 &116& 128\\
132 &136& 144 &148 &152& 160&168 &176& 180 &184 &192& 196\\
200 &208& 212 &224 &228& 232&240 &244& 256 &260 &264& 272\\
276 &280& 288 &296 &304& 308&312 &320& 324 &336 &340& 344\\
352 &356& 360 &368 &372& 384&392 &400& 404 &408 &416& 424\\
448 &452& 456 &464 &468& 472&480 &488& 512 &520 &528& 532\\
536 &544& 552 &560 &564& 576&592 &596& 600 &608 &612& 616\\
624 &628& 640 &648 &672& 680&688 &692& 696 &704 &708& 712\\
720 &724& 736 &740 &744& 768&784 &788& 792 &800 &808& 816\\
820 &832& 836 &840 &848& 852&868 &896& 904 &912 &916& 920\\
928 &936& 944 &948 &952& 960&964 &976& 980 &984 &996& 1024\\
1040 &1044& 1048 &1056 &1064& 1072&1076 &1080& 1088 &1092 &1104& 1108\\
1112 &1120& 1124 &1128 &1152& 1156&1184 &1192& 1200 &1204 &1216& 1264\\
1300 &1360& 1384 &1396 &1408& 1432&1480 &1492& 1540 &1576 &1588& 1600\\
1624 &1636& 1672 &1684 &1696& 1732&1792 &1816& 1840 &1876 &1888& 1960\\
1972 &2080& 2116 &2128 &2152& 2164&2176 &2224& 2248 &2260 &2308& 2368\\
2416 &2440& 2452 &2464 &2500& 2512&2536 &2560& 2608 &2632 &2728& 2752\\
2836 &2848& 2884 &2896 &2944& 2968&2992 &3076& 3136 &3208 &3220& 3232\\
3256 &3268& 3280 &3328 &3352& 3376&3412 &3472& 3508 &3544 &3616& 3640\\
3664 &3688& 3712 &3784 &3796& 3808&3844 &3856& 3904 &3940 &3952& 3976\\
3988 &4096& 4180 &4192 &4264& 4276&4288 &4336& 4360 &4372 &4408& 4432\\
4456 &4468& 4480 &4504 &4552& 4624&4708 &4720& 4768 &4804 &4816& 4840\\
4852 &4864& 4888 &4912 &4948& 5032&5044 &5056& 5080 &5092 &5128& 5140\\
5176 &5200& 5224 &5272 &5296& 5320&5416 &5428& 5440 &5464 &5476& 5536\\
5572 &5584& 5632 &5680 &5704& 5716&5728 &5776& 5800 &5812 &5860& 5872\\
5908 &5920& 5956 &5968 &5992& 6004&6064 &6088& 6112 &6160 &6232& 6244\\
6292 &6304& 6352 &6400 &6424& 6448&6484 &6496& 6544 &6568 &6616& 6664\\
6688 &6724& 6736 &6760 &6772& 6784&6832 &6856& 6928 &6952 &6964& 7000\\
7012 &7024& 7060 &7096 &7168& 7216&7252 &7360& 7396 &7504 &7576& 7720\\
7828 &8080& 8116 &8152 &8404& 8512&8548 &8584& 8656 &8728 &8872& 8944\\
8980 &9232& 9268 &9304 &9376& 9412&9448 &9520& 9556 &9664 &9736& 9808\\
9880 &9952& 9988 &10024 &10096& 10132&10168 &10240& 10312 &10384 &10420& 10528\\
10600 &10708& 10852 &10960 &10996& 11176&11284 &11320& 11392 &11608 &11716& 11824\\
11968 &12040& 12148 &12472 &12688& 12904&13012 &13120& 13336 &13444 &13552& 13768\\
13876 &13912& 13984 &14308 &14416& 14560&14632 &14740& 14848 &15064 &15172& 15280\\
15496 &15604& 15712 &15856 &15928& 16144&17008 &17332& 17800 &18196 &18304& 18628\\
18772 &18952& 19600 &19924 &20248& 20896&21220 &21544& 21688 &22192 &22660& 22840\\
23056 &23488& 23812 &24136 &24784& 25108&25972 &26728& 27700 &28672 &31588& 32560\\
33352 &34504& 34720 &36448 &39364& 41524&42280 &44224& 45520 &48904 &50848& 51028\\
53944 &59776& 61720 &65608 &74356& 80512&95956 &103336& 118096 &138400 &176740& 190996\\
250504 &345544& 1276936 & && & &&  & && \\
\bottomrule\end{longtable}

%\begin{longtable}{llllll}\toprule
\caption{Frequenza cicli}\\
\toprule
\textbf{n l} & \textbf{n l} & \textbf{n l} & \textbf{n l}& \textbf{n l} & \textbf{n l}\\
\midrule
\endfirsthead
\multicolumn{6}{c} {\tablename\ \thetable\ -- \textit{Continua dalla pagina precedente}} \\
\textbf{n l} & \textbf{n l} & \textbf{n l} & \textbf{n l}& \textbf{n l} & \textbf{n l}\\
\midrule
\endhead
\midrule
\multicolumn{6}{r} {\textit{Continua nella pagina successiva}} \\
\endfoot
\bottomrule
\endlastfoot
1  1&2 1&3 2&4 1&5 2&6 2\\
7  4&8 4&9 6&10 6&11 8&12 9\\
13  13&14 12&15 17&16 22&17 16&18 22\\
19  26&20 22&21 32&22 18&23 28&24 43\\
25  23&26 38&27 22&28 30&29 49&30 23\\
31  40&32 58&33 25&34 47&35 14&36 30\\
37  60&38 16&39 32&40 10&41 23&42 47\\
43  15&44 29&45 43&46 21&47 39&48 11\\
49  22&50 47&51 10&52 27&53 8&54 15\\
55  32&56 8&57 20&58 29&59 10&60 26\\
61  5&62 13&63 29&64 7&65 17&66 2\\
67  7&68 21&69 6&70 13&71 6&72 6\\
73  17&74 3&75 8&76 24&77 3&78 9\\
79  2&80 5&81 15&82 3&83 8&84 2\\
85  6&86 13&87 5&88 10&89 10&90 6\\
91  12&92 4&93 11&94 20&95 7&96 15\\
97  5&98 10&99 17&100 5&101 12&102 7\\
103  7&104 16&105 6&106 9&107 22&108 8\\
109  13&110 7&111 11&112 19&113 10&114 17\\
115  7&116 14&117 24&118 11&119 19&120 31\\
121  14&122 25&123 9&124 17&125 32&126 11\\
127  22&128 6&129 15&130 29&131 7&132 18\\
133  3&134 9&135 21&136 2&137 9&138 24\\
139  3&140 13&141 2&142 6&143 15&144 4\\
145  10&146 0&147 5&148 14&149 0&150 4\\
151  15&152 2&153 7&154 0&155 2&156 11\\
157  0&158 2&159 0&160 0&161 4&162 0\\
163  1&164 1&165 0&166 2&167 0&168 2\\
169  6&170 1&171 3&172 0&173 1&174 6\\
175  0&176 3&177 0&178 1&179 4&180 0\\
181  1&182 3&183 0&184 0&185 0&186 0\\
\bottomrule\end{longtable}
	 
\cleardoublepage	
\begin{appendices}
\chapter{Formule}
\section{Fattoriale}
\citaoeis{A000142}
\begin{equation}
n!=1\cdot 2\cdot 3\dots (n-1)\cdot n
\end{equation}\index{Numero!Fattoriale}
\section{Numeri di Fibonacci}
\citaoeis{A000045}
\begin{equation}
F_n=F_{n-1}+F_{n-2}\qquad\ F_{1}=1\; F_{2}=1 
\end{equation}\index{Numero!Fibonacci}
\section{Numeri di Lucas}
\citaoeis{A000032}
\begin{equation}
L_n=L_{n-1}+L_{n-2}\qquad\ L_{1}=2\; L_{2}=1 
\end{equation}\index{Numero!Lucas}
\section{Terne}
\subsection{Terna pitagorica}
\citaoeis{A263728}
\begin{defn}
$a$, $b$ e $c$ sono una terna pitagorica se \[a^2+b^2=c^2\]
\end{defn}
\begin{defn}
Una terna pitagorica  $a$, $b$ e $c$ è una terna primitiva se
\[\mcd(a,b,c)=1\]
\end{defn}
\begin{thm}
	Se $m$ e $n$ sono coprimi e sono uno pari e l'altro dispari allora
	\begin{align*}
	a=&m^2-n^2\\
	b=&2mn\\
	c=&m^2+n^2
	\end{align*}
	è una terna primitiva e viceversa. 
\end{thm}
\begin{proof}
	Dimostriamo che è una terna pitagorica cioè che \[a^2+b^2=c^2\]
	ora 
		\begin{align*}
	a=&m^4-2m^2n^2+n^4\\
	b=&4m^2n^2\\
	c=&m^4+2m^2n^2+n^4\\
	m^4-2m^2n^2+n^4+4m^2n^2=&m^4+2m^2n^2+n^4\\
	\end{align*}
	da cui la dimostrazione della prima parte. 
	
	Dimostriamo che sono primi
\end{proof}
\begin{thm}
	Se $m$ è un numero dispari allora
		\begin{align*}
	a=&m\\
	b=&\dfrac{m^2-1}{2} \\
	c=&\dfrac{m^2+1}{2}
	\end{align*}
	è una terna pitagorica.\par
		Se $m$ è un numero pari non prodotto fra un pari e un dispari, allora
	\begin{align*}
	a=&m\\
	b=&\dfrac{m^2-4}{4} 
	c=&\dfrac{m^2+4}{4}
	\end{align*}
		è una terna pitagorica\index{Terna!pitagorica}.
\end{thm}
\subsection{Terne di Eisenstein}
\citaoeis{A121992}
In un triangolo con un angolo  di $\ang{60}$
abbiamo una terna di Eisenstein $(a,b,c)$ con
\begin{equation*}
	c^2=a^2-ab+b^2
\end{equation*}

In un triangolo con un angolo  di $\ang{120}$
abbiamo una terna di Eisenstein $(a,b,c)$ con
\begin{equation*}
	c^2=a^2+ab+b^2
\end{equation*}\index{Terna!Eisenstein}
\cleardoublepage
\nocite{*}
\addcontentsline{toc}{chapter}{\bibname}
\printbibliography
\cleardoublepage
\addcontentsline{toc}{chapter}{\indexname}
\printindex
\chapter{Mezzi usati}
\CDMezziUsati
\end{appendices}
\end{document}
